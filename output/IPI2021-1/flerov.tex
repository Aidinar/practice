\def\stat{flerov}

\def\tit{ИНФОРМАЦИОННАЯ МОДЕЛЬ ВЕСОВОГО ОБЛИКА ЛЕТАТЕЛЬНЫХ АППАРАТОВ}

\def\titkol{Информационная модель весового облика летательных аппаратов}

\def\aut{Л.\,Л.~Вышинский$^1$, Ю.\,А.~Флёров$^2$}

\def\autkol{Л.\,Л.~Вышинский, Ю.\,А.~Флёров}

\titel{\tit}{\aut}{\autkol}{\titkol}

\index{Вышинский Л.\,Л.}
\index{Флёров Ю.\,А.}
\index{Vyshinsky L.\,L.}
\index{Flerov Yu.\,A.}

%{\renewcommand{\thefootnote}{\fnsymbol{footnote}} \footnotetext[1]
%{Работа выполнена при частичной финансовой поддержке РФФИ
%(проекты 18-07-00692, 19-07-00739 и~20-07-00804).}}

\renewcommand{\thefootnote}{\arabic{footnote}}
\footnotetext[1]{Вычислительный центр им.\ А.\,А.~Дородницына Федерального исследовательского центра <<Информатика 
и~управ\-ле\-ние>> Российской академии наук, \mbox{wyshinsky@mail.ru}}
\footnotetext[2]{Вычислительный центр им.\ А.\,А.~Дородницына Федерального исследовательского центра 
<<Информатика и~управ\-ле\-ние>> Российской академии наук, \mbox{fler@ccas.ru}}

%\vspace*{-12pt}

 

  \Abst{Статья посвящена описанию информационной модели весового облика 
летательных аппаратов (ЛА). Под весовым обликом ЛА здесь 
понимается совокупность взаимосвязанных между собой информационных 
объектов, содержащих описание конструкции, параметров и~характеристик 
ЛА, достаточное для проведения весовых расчетов, весового 
анализа и~весового контроля на всех этапах жизненного цикла изделия. 
Описанная информационная модель может служить основой логической схемы 
базы данных (БД) при разработке автоматизированных систем весового 
проектирования (АСВП). В~статье модель весового облика ЛА
описана в~терминах сетевых структур данных.}
  
  \KW{автоматизация проектирования; летательный аппарат; весовое 
проектирование; весовая модель; дерево конструкции; генератор проектов}

\DOI{10.14357/19922264210107}


\vskip 10pt plus 9pt minus 6pt

\thispagestyle{headings}

\begin{multicols}{2}

\label{st\stat}

\section{Информационная структура весовой модели летательного 
аппарата}

  В работе~[1] была представлена АСВП ЛА, которая разработана как 
многопользовательская информационная система кли\-ент-сер\-вер\-ной 
архитектуры с~централизованной БД сетевого типа. \mbox{Настоящая} 
статья посвящена описанию струк\-тур\-но-па\-ра\-мет\-ри\-че\-ской весовой 
модели ЛА, которая была положена в~основу логической 
схемы БД этой сис\-те\-мы. Выбор сетевой модели данных был 
обуслов\-лен спецификой предметной об\-ласти, основными объектами которой 
являются конструкции, состоящие из различных агрегатов, узлов и~деталей, 
объединенные многочисленными связями и~отношениями. Сетевая модель 
данных в~наибольшей степени приспособлена для описания слож\-но 
организованных сис\-тем. 
  
  В основе сетевой модели данных лежат два основных понятия~--- 
структурированная запись данных (Record) и~набор записей (Set), отра\-жа\-ющий 
бинарные отношения типа один ко многим. Записи данных представляют собой 
поименованные кортежи фиксированной длины:
  $$
  r=\left( x_1, x_2,\ldots , x_n\right)\,,
  $$
  где $x_i\in X_i$, $i\hm=1,\ldots , n$; $r\hm\in R$.
  Множества~$X_i$ определяют типы данных атрибутов $x_i$ кортежа, 
а~множество~$R$ определяет тип записи~$r$. 

Типы данных атрибутов кортежей 
могут быть\linebreak примитивными (целыми и~действительными чис\-ла\-ми, строками, 
булевскими значениями, пе\-ре\-чис\-ли\-мы\-ми типами) или агрегатными типами, 
являющимися векторами, структурами (в~смыс\-ле\linebreak \mbox{языка}~С) или более 
сложными конструкциями. В~описании произвольной предметной об\-ласти 
экземпляры записей представляют конкретные объекты этой прикладной 
об\-ласти, а~наборы задают связи между объектами~--- бинарные отношения 
один ко многим. 

  \begin{figure*} %fig1
  \vspace*{1pt}
  \begin{center}
    \mbox{%
 \epsfxsize=128.091mm 
 \epsfbox{fle-1.eps}
 }
\end{center}
\vspace*{-3pt}
  \Caption{Сетевая структура весовой модели ЛА: \textbf{А}~--- пустой 
снаряженный ЛА; \textbf{Б}~--- полезная нагрузка; \textbf{В}~--- геометрические модели}
\vspace*{2pt}
  \end{figure*}

Тип набора определяется как упорядоченная поименованная 
пара 
\begin{equation}
S=\langle R_i\to R_j\rangle.
\label{e1-v}
\end{equation}
 Экземпляр~$s$ набора типа~$S$ 
определяется как пара, состоящая из экземпляра записи типа~$R_i$, который 
принято называть владельцем набора, и~подмножества экземпляров записей 
типа~$R_j$, которые называют членами набора. Множество подчиненных 
членов набора может быть упорядочено по ключу, который задается при 
описании набора. Такие наборы называют ключевыми наборами. Возможно 
создание так называемых сингулярных наборов типа $S\hm= \langle \varnothing 
\hm\to R\rangle$, где формально владелец набора не задается, а~членами такого 
набора являются записи из упорядоченного подмножества записей типа~$R$. 
Возможно создание набора с~одинаковыми типами владельца и~членов набора, 
так на\-зы\-ва\-емые рекурсивные наборы $S\hm= \langle R\hm\to R\rangle$, которые 
устанавливают отношения между записями одного типа, например отношения 
наследования или отношения вхож\-де\-ния. Такие наборы используются для 
описания строго иерархических \mbox{структур.} Заметим, что в~таких наборах 
множества членов разных экземпляров набора не должны пе\-ре\-се\-каться. 
{\looseness=1

}
  
  Описание модели предметной области в~парадигме сетевой модели данных 
задается множеством типов записей и~множеством типов наборов. Графически 
такое описание может быть представлено в~виде ориентированного графа, 
у~которого узлам соответствуют типы записей, а~дуге (ребру), идущей от 
узла~$R_i$ к~узлу~$R_j$, соответствует тип набора~(1). 
Сингулярным наборам соответствуют дуги, входящие в~узлы 
<<извне>>, а~рекурсивным наборам~--- дуги, выходящие из некоторого узла 
графа и~возвращающиеся в~тот же узел.
  
  На рис.~1 приведено графическое изображение сетевой структуры весовой 
модели, положенной в~основу АСВП ЛА. Узлам этой структуры соответствуют 
типы записей весовой модели, а стрелкам~--- связи между записями~--- наборы.
  

  
  Основная функция весовой модели состоит в~информационном обеспечении 
решения задач весового проектирования на всех этапах жизненного цикла ЛА. 
Вся текущая информация о~структуре объекта, о~мас\-со\-во-инер\-ци\-он\-ных 
характеристиках (МИХ), о~геометрических и~других параметрах, необходимых для 
весовых расчетов, для весового анализа и~для весового контроля в~процессе 
создания и~эксплуатации ЛА, может быть размещена 
в~структурах весовой модели. Вход в~эту структуру осуществляется через 
сингулярный набор $\langle \varnothing\hm\to \mbox{\textbf{Aircraft}}\rangle$, 
задающий множество представленных в~БД 
\mbox{АСВП}~ЛА. 

Обязательными атрибутами всех записей 
в~весовой модели являются их идентификационные данные~--- это номера, 
идентификаторы, шиф\-ры, наименования в~зависимости от роли записи 
в~весовой модели. Для записей типа \textbf{Aircraft} 
идентификационными атрибутами служат паспортные данные ЛА 
(обозначение, производитель, тип ЛА, его назначение, год выпуска). 
  
  Центральной задачей весового проектирования является вычисление  
МИХ ЛА и~его компонент~--- 
массы, положения центра масс, статических моментов, собственных и~полных 
моментов инерции. 

Другая важная задача весовой модели состоит 
в~обеспечении весового контроля в~процессе проектирования, изготовления 
и~сборки изделий в~процессе опытного и~серийного производства. Для этого 
весовая модель должна содержать полную информацию о~конструкции ЛА 
и~о~вариантах целевого использования будущего изделия. Эти два аспекта 
в~весовой модели на рис.~1 выделены в~блоках~\textbf{А} и~\textbf{Б}. 
Блок~\textbf{А} пред\-став\-ля\-ет собой струк\-тур\-но-па\-ра\-мет\-ри\-че\-скую 
модель пустого снаряженного изделия, а~блок~\textbf{Б} содержит 
информацию о~возможных вариантах загрузки ЛА, о~па\-ра\-мет\-рах 
и~характеристиках топ\-лив\-ных баков, а~так\-же о~программах заправки 
и~выработки топлива. Структуры, объединенные в~блоке~\textbf{В}, 
предназначены для описания геометрических моделей, использующихся при 
расчетах МИХ и~для визуализации результатов расчетов.
  
\section{Структурно-параметрическая модель пустого снаряженного 
летательного аппарата}

  Структурно-параметрическая модель пустого снаряженного ЛА содержит 
описание всех элементов конструкции изделия и~его снаряжения, а также 
полный состав конструктивных па\-ра\-мет\-ров ЛА и~его агрегатов, влияющих на 
их МИХ. Параметры в~весовой модели ЛА 
задаются записями типа \textbf{Param}. Атрибуты этих записей 
включают идентификатор параметра, наименование, тип (real, int, string, 
boolean), способ задания (интерактивный ввод, вычисление по формулам, сами 
формулы, зависящие от других параметров), область допустимых значений. 
Набор типа  определяет полное множество параметров ЛА. Это множество 
содержит весовые, геометрические, конструктивные и~другие данные, от 
которых зависят масса, конструкция и~другие МИХ ЛА. Множество весовых 
параметров ЛА представляет весовой облик изделия.
  
  Структура пустого снаряженного ЛА состоит из двух ветвей: 
\textbf{Empty}~--- структуры пустого ЛА и~\textbf{Equip}~--- 
состава снаряжения. Ветвь пустого ЛА отражает конструкцию планера 
и~размещаемого внут\-ри планера стационарного комплекса бортового 
оборудования. Обычно в~весовой модели пред\-став\-ле\-но несколько 
альтернативных вариантов\linebreak конструкции, включая возможные варианты состава 
комплектующих систем. Варианты пустого изделия задаются набором $\langle \mbox{\textbf{Aircrft}}\hm\to
\mbox{\textbf{Empty}}\rangle$. 
Идентифицирующими атрибутами записей типа \textbf{Empty}\linebreak служат 
обозначение или номер варианта. Основ\-ными расчетными па\-ра\-мет\-ра\-ми 
являются МИХ\linebreak пустого изделия. Наборы $\langle\mbox{\textbf{Empty}}\hm \to\mbox{\textbf{Constr}}\rangle$ 
и~$\langle\mbox{\textbf{Constr}} \hm\to \mbox{\textit{Constr}}\rangle$
образуют иерархическую 
\mbox{структуру} пус\-то\-го ЛА, которую называют деревом конструкции изделия. 
Записи типа \mbox{\textbf{Constr}} являются узлами дерева конструкции 
и~в~совокупности представляют детальное описание агрегатов~ЛА.
{\looseness=1

}
  
  Описание каждого элемента конструкции может быть дополнено 
множеством конструктивных параметров, которое задается набором типа 
$\langle\mbox{\textbf{Constr}}\hm\to \mbox{\textbf{Param}}\rangle$. Визуализация общих 
видов ЛА и~элементов конструкций обеспечивается наличием в~весовой модели 
записей типа \textbf{Plan} и~наборами типа 
$\langle \mbox{\textbf{Empty}}\hm\to \mbox{\textbf{Plan}}\rangle$ 
и~$\langle \mbox{\textbf{Constr}}\hm\to \mbox{\textbf{Plan}}\rangle$. 
Записи типа \textbf{Plan} содержат идентификатор чертежа и~тип 
общего вида (план, вид сбоку или спереди). 
  
  Локализация элементов конструкции в~системе координат (СК) в~весовой 
модели задается набором типа 
$\langle \mbox{\textbf{Sys}}\hm\to\mbox{\textbf{Constr}}\rangle$, 
владельцем которого выступает запись типа \textbf{Sys}. Множество 
ис\-поль\-зу\-емых в~процессе проектирования ЛА систем координат в~весовой 
модели описывается набором 
$\langle \mbox{\textbf{Aircraft}}\hm \to \mbox{\textbf{Sys}}\rangle$. Это 
множество должно образовывать строго иерархическую структуру. Иерархия 
СК задается рекурсивным набором типа 
$\langle\mbox{\textbf{Sys}}\hm \to \mbox{\textbf{Sys}}\rangle$, 
а~атрибутами записей типа \textbf{Sys} служат идентификатор 
СК, название или комментарий к~СК, а~также 
положение локальной СК относительно вышестоящей СК, которое задается 
кортежем $(\boldsymbol{x}, \boldsymbol{y}, \boldsymbol{z}, \bm{\alpha}, 
\bm{\beta}, \bm{\gamma})$. Пересчет центра масс и~расчет полных моментов 
инерции осуществляется от локальной СК конструкции и~далее по иерархии 
в~главную СК. На рис.~2 приведено описание структуры дерева конструкции ЛА. 
Там же дан текст процедуры расчета МИХ, использующей эту структуру.
  
\begin{figure*} %fig2
\begin{center}
\tabcolsep=1.75pt
{\small\begin{tabular}{llll}
\multicolumn{3}{l}{// ОПИСАНИЕ ДЕРЕВА КОНСТРУКЦИИ:}&                      // ПРОЦЕДУРА РАСЧЕТА МИХ КОНСТРУКЦИИ:\\[6pt]                             
 \multicolumn{3}{l}{type t\_Constr: struct // \textit{тип конструкции}}&  proc eval\_constr (Tree\_Constr.Constr RC0, t\_Constr С0) \\             
\multicolumn{3}{l}{{\raisebox{-6pt}[0pt][-6pt]{\hspace*{10mm}// \textit{атрибуты записи}:}}}&           // RC0~--- \textit{ссылка на запись вершины дерева} \\                   
\multicolumn{3}{l}{\ } &// С0~--- \textit{запись вершины дерева}\\                               
(text &cod, &// \textit{обозначение}&                                     \{ var Tree\_Constr.Constr RCN, t\_Constr CN,\\                          
text &name, &// \textit{наименование}&                                    // RCN \textit{ссылка на подчиненные С0 записи}\\                        
text &type, &// \textit{тип конструкции }&                                // CN \textit{подчиненные С0 записи}\\                                   
real &m\_thr, &// \textit{теоретическая масса }&                          real РМх, РМу, PMz; // \textit{статические моменты С0}\\                 
real &m\_lime,& // \textit{лимитная масса }&                              // \textit{построение списка подчиненных вершин}:\\                      
real &m\_plot, &// \textit{чертежная масса }&                             Tree\_Constr.Constr\_Constr\_mem (RC0, 0, RCN, CN);\\                    
real &m\_fact, &// \textit{фактическая масса}&                            // \textit{если список пуст} (RCN=null), \textit{то выход из процедуры}\\
real &m\_fast, &// \textit{масса крепежа}&                                {while} (isnotnull(RCN)) // \textit{цикл по подчиненным вершинам}\\      
t\_gab &gab, &// \textit{габариты конструкции }&                          \{  // \textit{рекурсивный спуск по дереву}: \\                          
t\_pos &pops, &// \textit{положение в локальной СК }&                     call eval\_constr\_constr(RCN, CN);\\                                    
text &mat, &// \textit{материал конструкции}&                             if (CN.mich.M>0) // \textit{при} M<=0 \textit{запись пропускается}\\     
real &dens, &// \textit{плотность материала}&                             \{ PMx:=C0.mich.M*C0.mich.X; PMy:=C0.mich.M*C0.mich.Y;\\                 
t\_mich &mich &// МИХ~--- $M$, $X$, $Y$, $Z$, $Jxx$, $Jyy$,\,$\ldots$&    PMz:= C0.mich.М* C0.mich.Z; C0.mich.M + = CN.mich.M;\\                   
\multicolumn{3}{l}{);}&                                                   C0.mich.X := (PMx + CN.mich.M * CN.mich.X)/C0.mich.M;\\                  
\multicolumn{3}{l}{{\raisebox{-6pt}[0pt][-6pt]{\hspace*{10mm}// \textit{дерево конструкции}:}}} &C0.mich.Y := (PMy + CN.mich.M * CN.mich.Y)/C0.mich.M;\\
\multicolumn{2}{l}{{\raisebox{-6pt}[0pt][-6pt]{type Tree\_Constr:}}} & {\raisebox{-6pt}[0pt][-6pt]{document}}
                      &C0.mich.Z := (PMz + CN.mich.M * CN.mich.Z)/C0.mich.M;\\
\multicolumn{2}{l}{{\raisebox{-6pt}[0pt][-6pt]{(record Constr:}}}   & {\raisebox{-6pt}[0pt][-6pt]{t\_Constr; // \textit{узлы дерева}}}
         &C0.mich.Jxx += CN.mich.Jxx; C0.mich.Jyy += CN.mich.Jyy;\\
\multicolumn{3}{l}{\ } & C0.mich.Jzz += CN.mich.Jzz; C0.mich.Jxy += CN.mich.Jxy;\\               
\multicolumn{2}{l}{\ set Constr\_Constr} & // \textit{набор, задающий}&   C0.mich.Jxz += CNr.mich.Jxz; C0.mich.Jyz += CN.mich.Jyz;\\              
\multicolumn{2}{l}{\ \ owner Constr} &// \textit{отношение }&             \} Tree Constr.Constr\_Constr\_next (RCN, CN);\\                        
\multicolumn{2}{l}{\ \ member Constr;} & // \textit{подчиненности}&       \}// \textit{выход из цикла}\\                                           
\multicolumn{3}{l}{);} &                                                  \}// \textit{конец процедуры}                                            
\end{tabular}}                                                            
\end{center}

\vspace*{-4pt}                                                              
 
\Caption{Фрагмент описания структуры весовой модели ЛА}
\end{figure*}

  Дерево конструкции пустого ЛА является центральным объектом весовой 
модели как по значимости его в~процессе проектирования ЛА, так и~по объему 
создаваемой и~хранящейся в~БД информации. Число записей в~дереве 
конструкции исчисляется десятками, а то и~сотнями тысяч. Однако, благодаря 
использованию сетевых структур в~весовой модели ЛА, расчет МИХ 
представляет собой хотя и~длительную, но достаточно просто организованную 
рекурсивную процедуру. Приведенная процедура позволяет рассчитывать МИХ 
любой конструкции, вплоть до всего дерева пустого ЛА, состоящего из сотен 
тысяч деталей.
  
  Расчет МИХ снаряжения ЛА ведется аналогично расчету пус\-то\-го изделия. 
Снаряжение~--- это объекты, которые размещаются на борту ЛА в~процессе его 
подготовки перед полетом и~предназначены для обеспечения выполнения 
полетного задания. Снаряжение ЛА обычно подразделяют на базовое 
и~специальное. К~базовому снаряжению относят экипаж, средства и~системы 
жизнеобеспечения экипажа и~пассажиров, системы технологического 
обеспечения полета, а~также заправляемые жидкости и~газы, включая 
невырабатываемый остаток топлива. К~специальному снаряжению относят 
технологические устройства и~системы, предназначенные для выполнения 
целевых задач полетного задания, в~том числе средства закрепления 
перевозимых грузов в~отсеках ЛА и~на внешних узлах подвески. 
  
  Варианты снаряжения в~весовой модели задаются записями типа 
\textbf{Equip}, а~все множество вариантов задается набором 
$\langle \mbox{\textbf{Aircraft}} \hm \to \mbox{\textbf{Equip}}\rangle$. 
Каждый вариант снаряжения состоит из множества элементов\linebreak снаряжения 
(наборы типа 
$\langle \mbox{\textbf{Equip}} \hm \to \mbox{\textbf{Item}}\rangle$). 
Элементы снаряжения могут быть организованы в~иерархию средствами 
рекурсивных наборов типа 
$\langle \mbox{\textbf{Item}}\hm \to \mbox{\textbf{Item}}\rangle$. Как 
правило, существуют несколько\linebreak типовых вариантов комплектации экипажа 
и~базового снаряжения. Весовая модель должна содержать перечень 
используемых вариантов снаряжения и~их характеристики. Размещение 
снаряжения\linebreak на борту ЛА задается аналогично конструкциям пус\-то\-го ЛА 
с~помощью наборов типа 
$\langle \mbox{\textbf{Sys}}\hm \to \mbox{\textbf{Item}}\rangle$. 
В~дополнение к~такому способу размещения отдельные элементы снаряжения 
могут быть привязаны к~определенным точкам конструкции ЛА, мес\-там 
размещения, которые задаются записями типа \textbf{Place}. Привязка 
элемента снаряжения к~мес\-ту размещения обеспечивается наборами типа 
$\langle \mbox{\textbf{Place}}\hm \to \mbox{\textbf{Item}}\rangle$. Мес\-та 
размещения снаряжения локализованы в~одной из СК изделия 
$\langle \mbox{\textbf{Sys}}\hm \to \mbox{\textbf{Place}}\rangle$.
  
  Перечень элементов снаряжения может быть каталогизирован, что позволит 
упростить компоновку взлетной конфигурации ЛА и~выполнение расчетов 
МИХ пустого снаряженного ЛА. Связь элементов снаряжения 
с~соответствующим каталогом в~весовой модели реализуется с~помощью 
набора типа 
$\langle \mbox{\textbf{Unit}}\hm \to \mbox{\textbf{Place}}\rangle$, 
владельцем которого выступают записи типа \textbf{Unit}, 
представляющие каталогизированные элементы снаряжения.

%\vspace*{-7pt}
  
\section{Модель размещения полезной нагрузки летательного аппарата}

%\vspace*{-2pt}

  Целевую нагрузку ЛА вместе с~заправляемым топливом обычно называют 
полезной нагрузкой. На рис.~1 структуры модели размещения на борту ЛА 
полезной нагрузки сосредоточены в~блоке~\textbf{Б}. Модели полезной 
нагрузки задаются наборами типа 
$\langle \mbox{\textbf{Aircraft}}\hm \to \mbox{\textbf{Loading}}\rangle$~--- 
множеством вариантов целевой загрузки ЛА, наборами типа 
$\langle \mbox{\textbf{Aircraft}}\hm \to \mbox{\textbf{Fuel}}\rangle$~--- 
множеством вариантов стартовой заправки ЛА топливом 
и~$\langle \mbox{\textbf{Aircraft}}\hm \to \mbox{\textbf{Progr}}\rangle$~---  
множество вариантов программ выработки топлива.
  
  Каждый из вариантов целевой нагрузки, задаваемых записями типа 
\textbf{Loading}, состоит из множества различных элементов нагрузки 
(наборы типа 
$\langle \mbox{\textbf{Loading}}\hm \to \mbox{\textbf{Loading}}\rangle$). 
Элементы нагрузки (записи типа \textbf{Load})~--- это внешние 
объекты по отношению к~ЛА. Разумеется, специфика размещаемой нагрузки 
связана с~типом ЛА. Для пассажирских ЛА целевой нагрузкой считаются 
пассажиры и~их багаж, для транспортных самолетов~--- разнообразные 
перевозимые коммерческие грузы, для военных~---  вооружение. 

%\pagebreak

Разнообразие элементов нагрузки требует их классификации и~каталогизации. 
Для этой цели в~рамках весовой модели предназначена специальная структура 
каталогов нагрузки. Каталоги элементов нагрузки в~весовой модели задаются 
сингулярными наборами типа 
$\langle$$\varnothing$\;$\to$\;{\textbf{Catalog}}$\rangle$. Как правило, 
каталоги состоят из нескольких разделов и~подразделов. Их структура задается 
рекурсивными наборами 
$\langle\mbox{\textbf{Catalog}}\hm\to \mbox{\textbf{Catalog}}\rangle$. 
Множество элементов нагрузки каждого из разделов классификатора 
представляется наборами 
$\langle \mbox{\textbf{Catalog}}\hm \to \mbox{\textbf{Unit}}\rangle$. 
В~записях типа \textbf{Unit} дается исчерпывающая информация об 
элементе нагрузки, включая его обозначение и~полное название, а~главное, все 
необходимые для размещения и~расчетов данные~--- тип нагрузки, МИХ, 
габариты, форма и~пр. 

В~дополнение к~описанию технических 
характеристик в~каталоге может быть задана дополнительная информация об 
узлах крепления (наборы типа 
$\langle \mbox{\textbf{Unit}}\hm\to \mbox{\textbf{Node}}\rangle$), 
конструктивно выполняемых на элементах нагрузки для пристыковки к~ним 
других объектов. Такие узлы характерны для временно устанавливаемых на 
военных ЛА переходных балок, предназначенных для подвески вооружения. 
{\looseness=1

}

Записи типа \textbf{Load} связываются с~каталогом\linebreak элементов нагрузки с~помощью наборов типа 
$\langle \mbox{\textbf{Unit}}\hm \to \mbox{\textbf{Load}}\rangle$. Элементы 
нагрузки могут быть многоярусными (переходная бал\-ка\,--\,узел  
креп\-ле\-ния\,--\,под\-вес\-ка). Это отражается в~структуре весовой модели 
рекурсивным набором типа 
$\langle \mbox{\textbf{Load}}\hm\to \mbox{\textbf{Load}}\rangle$. 
Структуры размещения полезной нагрузки непосредственно связаны со 
структурами пустого снаряженного ЛА. Положение нагрузки, размещаемой на 
борту или на подвесках, связывается с~СК пустого ЛА 
(наборы типа 
$\langle \mbox{\textbf{Sys}}\hm \to \mbox{\textbf{Load}}\rangle$) или 
непосредственно с~местом размещения, например с~местом в~пассажирском 
салоне, местом на стеллажах багажных отсеков, с~точкой подвески под крылом 
или под фюзеляжем (наборы типа 
$\langle \mbox{\textbf{Place}}\hm \to \mbox{\textbf{Load}}\rangle$). 
В~состав варианта загрузки ЛА могут входить кроме элементов целевой 
нагрузки временно размещаемые подвесные или вкладные топливные баки 
(наборы типа 
$\langle \mbox{\textbf{Loading}}\hm \to\mbox{\textbf{Tank}}\rangle$). 
Характеристики пустых подвесных баков, которые задаются в~записях типа 
\textbf{Tank}, берутся из каталога (наборы типа 
$\langle \mbox{\textbf{Unit}}\hm \to \mbox{\textbf{Tank}}\rangle$), 
а~размещение задается наборами типа 
$\langle \mbox{\textbf{Sys}}\hm \to\mbox{\textbf{Tank}}\rangle$ 
и~$\langle \mbox{\textbf{Place}}\hm \to \mbox{\textbf{Tank}}\rangle$.


\begin{figure*}[b] %fig3
   \vspace*{6pt}
  \begin{center}
    \mbox{%
 \epsfxsize=147.507mm 
 \epsfbox{fle-3.eps}
 }
\end{center}
\vspace*{-6pt}
  \Caption{Геометрические объекты, представимые в~весовой модели ЛА}
  \end{figure*}
  
  Важнейшей составляющей полезной нагрузки является топливо. В весовой 
модели ЛА варианты стартовой заправки топливом задаются записями типа 
\textbf{Fuel}. Общий стартовый объем топлива зависит от 
конфигурации топливной системы, от наличия временно размещаемых 
подвесных или вкладных топливных баков. Конфигурация топливных емкостей 
задается наборами типа 
$\langle \mbox{\textbf{Fuel}}\hm \to \mbox{\textbf{Tank}}\rangle$. Это 
множество включает все множество топливных емкостей, включая и~временно 
устанавливаемые баки, и~стационарные, встроенные в~конструкцию ЛА. 
Располагаемый объем топливных емкостей, МИХ заправленного топлива 
определяются геометрией баков. От геометрии зависит и~характер изменения 
МИХ топлива по мере его выработки 
в~полете ЛА. Основными характеристиками баков, которые позволяют 
анализировать изменения общей массы ЛА и~положение его центра масс 
в~процессе выработки топлива, служат тарировочные характеристики баков~--- 
зависимости положения центра тяжести и~моментов инерции топлива в~баке от 
массы невыработанного остатка, а~также от углов тангажа и~крена ЛА. 
В~весовой модели могут быть представлены несколько вариантов 
тарировочных характеристик (наборы типа 
$\langle \mbox{\textbf{Tank}}\hm \to \mbox{\textbf{Calibr}}\rangle$). 
Построение тарировочных характеристик основано на разбиении всего объема 
топлива в~баке на тонкие\linebreak слои параллельными плоскостями. Каждый слой 
топлива (записи типа \textbf{Layer}) характеризуется своей массой 
и~другими МИХ. Варианты тарировочных характеристик представляются 
упорядоченными множествами слоев (наборы 
типа $\langle \mbox{\textbf{Calibr}}\hm \to \mbox{\textbf{Layer}}\rangle$).

  
  Процесс выработки топлива из топливных баков управляется специальными 
программами топливной системы, которые обеспечивают определенный 
порядок перекачки топлива в~расходные баки силовой установки. Эти 
программы должны обеспечивать равномерность и~бесперебойность подачи 
топлива при любых эволюциях ЛА, а также сохранение центровки ЛА 
в~заданных пределах.\linebreak
 Под управлением программы выработки происходит 
изменение %\linebreak 
состояния различных устройств\linebreak топливной системы~--- клапанов, 
вентилей, перекачивающих насосов. В весовой модели ЛА возможно задание 
нескольких программ (записями типа \textbf{Progr}) для {раз\-лич\-ных} конфигураций 
и~со\-ста\-ва топ\-лив\-ных баков. Каждая программа характеризуется 
по\-сле\-до\-ва\-тель\-ностью пе\-ре\-клю\-чений управ\-ля\-ющих устройств~--- 
$\langle \mbox{\textbf{Progr}}\hm \to \mbox{\textbf{Switch}}\rangle$. Между 
переключениями происходит выработка топлива из определенных \mbox{баков}. Эти 
периоды можно связать с~выработкой очередных слоев топлива из разных 
\mbox{баков}. Множество вырабатываемых слоев между переключениями определяется 
наборами 
$\langle \mbox{\textbf{Switch}}\hm \to \mbox{\textbf{Layer}}\rangle$. Таким 
образом, каждой программе выработки топлива соответствует упорядоченная 
после\-до\-ва\-тель\-ность вырабатываемых объемов топлива из разных баков. Это 
позволяет строить таблицы и~графики изменения положения центра масс 
и~моментов инерции текущего остатка топлива в~полете.
{ %\looseness=1

} 
  
  Массово-инерционные характеристики топлива на старте и~их изменение в~процессе выработки всецело 
зависят от геометрических форм топливных баков. Поэтому одной из 
предпосылок эффективности проведения расчетов является наличие в~рамках 
весовой модели ЛА адекватных геометрических моделей трехмерных (3D)
конструкций. 


\vspace*{-3pt}
  
\section{Геометрические расчетные модели}

%\vspace*{-3pt}

  В проектировании ЛА геометрическое моделирование занимает центральное 
место. И~хотя основная нагрузка в~части решения задач весового 
проектирования~--- это численные расчеты, без геометриче\-ско\-го, 
пространственного представления рассчитываемых изделий расчеты 
невозможны. Выше, на схеме информационной весовой модели (см.\ рис.~1), 
структуры геометрических моделей отоб\-ра\-же\-ны в~блоке~\textbf{В}. На рис.~3 
приведены три вида геометрических объектов, представимых в~структурах 
весовой модели~ЛА. 
  
  
  
  Двумерные модели уже упоминались при описании пустого ЛА. Это чертежи 
общих видов ЛА (рис.~3,\,\textit{а}) и~чертежи элементов конструкции, которые в~структуре
 весовой модели представлены записями типа \textbf{Plan}. 
Любой чертеж состоит из множества элементов чертежа, которое представлено 
наборами типа 
$\langle \mbox{\textbf{Plan}}\hm \to \mbox{\textbf{Draw}}\rangle$. Под 
элементами чертежа (записи типа \textbf{Draw}) понимаются такие 
объекты, как ломаные линии, произвольные многоугольники и~параметрически 
задаваемые плоские фигуры. Атрибутами записей типа \textbf{Draw} 
кроме позиционного номера служат общепринятые при отображении плоских 
рисунков параметры~--- цвет, толщина и~тип линий, цвет и~характер заливки 
плоских фигур, их положение на поле чертежа. Ломаные линии и~плоские 
фигуры задаются наборами типа 
$\langle \mbox{\textbf{Draw}}\hm \to \mbox{\textbf{Pnt2}}\rangle$, которые 
представляют множества нумерованных двумерных точек~--- записей типа 
$\mathbf{Pnt2}\hm=(\boldsymbol{n}, \boldsymbol{x},\boldsymbol{y})$.
  
  Структуры 3D представления объектов в~весовой модели ЛА 
формируются записями типа \textbf{Frame}. 
\textbf{Frame}~--- это головная запись, содержащая 
идентификационное обозначение геометрической модели и~комментарий. 
Трехмерное геометрическое представление в~весовой модели ЛА могут иметь 
три типа объектов~--- конструкции пус\-то\-го ЛА (наборы типа 
$\langle \mbox{\textbf{Constr}}\hm \to \mbox{\textbf{Frame}}\rangle$), 
топливные баки (наборы типа 
$\langle \mbox{\textbf{Tank}}\hm \to \mbox{\textbf{Frame}}\rangle$) 
и~элементы целевой нагрузки и~снаряжения (наборы типа 
$\langle \mbox{\textbf{Unit}}\hm \to \mbox{\textbf{Frame}}\rangle$). 
Трехмерное геометрическое представление может быть реализовано в~двух 
формах~--- с~по\-мощью 3D каркасной модели поверхности объекта 
(рис.~3,\,\textit{б}) и~в~виде 3D-мо\-де\-ли твердого тела (рис.~3,\,\textit{в}). 
Каркасная модель задается набором отсеков (набор типа 
$\langle \mbox{\textbf{Frame}}\hm\to \mbox{\textbf{Cell}}\rangle$), 
которые в~свою очередь состоят из упорядоченного множества плоских 
пространственных сечений (наборы типа 
$\langle \mbox{\textbf{Cell}}\hm \to \mbox{\textbf{Cross}}\rangle$ 
и~$\langle \mbox{\textbf{Cross}}\hm \to \mbox{\textbf{Pnt3}}\rangle$), где 
$\mathbf{Pnt3}\hm=(\boldsymbol{n},\boldsymbol{x},\boldsymbol{y},\boldsymbol{z})$~--- 
записи, пред\-став\-ля\-ющие трехмерные нумерованные точки). 

Гео\-мет\-ри\-че\-ское описание отсеков может использоваться для представления 
конструктивных отсеков пустого ЛА и~размещения внутри этих отсеков других 
конструкций (наборы 
$\langle \mbox{\textbf{Cell}}\hm \to \mbox{\textbf{Constr}}\rangle$).  
Трехмерная мо\-дель твердого тела может состоять из 3D-от\-се\-ков (записи типа 
\textbf{Solid}, которые задаются наборами плоских сечений (наборы 
$\langle \mbox{\textbf{Frame}}\hm \to \mbox{\textbf{Solid}}\rangle$) 
и~$\langle \mbox{\textbf{Solid}}\hm \to \mbox{\textbf{Facet}}\rangle$). Плоские 
пространственные сечения  
в~3D-мо\-де\-лях (записи типа\linebreak \textbf{Facet}) представляют грани  
3D-от\-се\-ков (наборы типа 
$\langle \mbox{\textbf{Facet}}\hm\to \mbox{\textbf{Pnt3}}\rangle$). 
В~расчетах МИХ топлива и~элементов 
конструкции пустого ЛА используют только  
3D-мо\-де\-ли. На стадиях рабочего проектирования ЛА в~современных 
условиях 3D-мо\-де\-ли конструкций ЛА строятся в~специализированных 
системах геометрического моделирования и~при необходимости могут 
передаваться в~весовую модель ЛА в~стандартных форматах представления 
твердых тел. Каркасные модели могут применяться при первичном задании 
форм трехмерных объектов на начальных стадиях проектирования, при этом 
каркасные модели легко преобразуются  
в~3D-пред\-став\-ле\-ние гео\-мет\-рии. Визуализация геометрических моделей 
возможна при всех формах задания геометрии.

\section{Заключение}

  Представленная в~статье информационная весовая модель ЛА, как уже 
говорилось, была
 разработана в~процессе создания АСВП самолетов. Модель была апробирована при 
реализации ряда программных модулей этой системы. В~работе~[2] был 
предложен подход к~созданию цифрового весового паспорта самолета, который 
также основан на представленной в~статье информационной модели весового 
облика~ЛА.
  
{\small\frenchspacing
{%\baselineskip=10.8pt
%\addcontentsline{toc}{section}{References}
\begin{thebibliography}{9}
\bibitem{1-fl}
\Au{Вышинский Л.\,Л., Флёров~Ю.\,А., Широков~Н.\,И.} Автоматизированная система 
весового проектирования самолетов~// Информатика и~её применения, 2018. Т.~12. Вып.~1. 
С.~18--30. doi: 10.14357/19922264180103.
\bibitem{2-fl}
\Au{Вышинский Л.\,Л., Курьянский~М.\,К., Флеров~Ю.\,А.} Циф\-ро\-вая модель весового 
паспорта летательного аппарата~// Информатика и~её применения, 2019. Т.~13. Вып.~4. 
С.~3--10. doi: 10.14357/19922264190401.
 \end{thebibliography}

}
}

\end{multicols}

\vspace*{-6pt}

\hfill{\small\textit{Поступила в~редакцию 20.01.2020}}

\vspace*{6pt}

%\pagebreak

%\newpage

%\vspace*{-28pt}

\hrule

\vspace*{2pt}

\hrule

\vspace*{-2pt}

\def\tit{INFORMATION MODEL OF~AIRCRAFT WEIGHT PROFILE}

\def\titkol{Information model of~aircraft weight profile}

\def\aut{L.\,L.~Vyshinsky and Yu.\,A.~Flerov}

\def\autkol{L.\,L.~Vyshinsky and Yu.\,A.~Flerov}

\titel{\tit}{\aut}{\autkol}{\titkol}

\vspace*{-16pt}


\noindent
A.\,A.~Dorodnicyn Computing Center, Federal Research Center ``Computer Science and Control'' 
of the Russian Academy of Sciences, 40~Vavilov Str., Moscow 119333, Russian Federation

\def\leftfootline{\small{\textbf{\thepage}
\hfill INFORMATIKA I EE PRIMENENIYA~--- INFORMATICS AND
APPLICATIONS\ \ \ 2021\ \ \ volume~15\ \ \ issue\ 1}
}%
\def\rightfootline{\small{INFORMATIKA I EE PRIMENENIYA~---
INFORMATICS AND APPLICATIONS\ \ \ 2021\ \ \ volume~15\ \ \ issue\ 1
\hfill \textbf{\thepage}}}

\vspace*{3pt}     
      
   
  

\Abste{The article is devoted to the description of the information model of the weight profile of an 
aircraft. The weight profile of an aircraft is understood as a set of interconnected information 
objects containing a description of the structure, parameters, and characteristics of the aircraft 
sufficient for weight calculations, weight analysis, and weight control at all stages of the product life 
cycle. The described information model can serve as the basis for a~scheme of a~database in the 
development of automated weight design systems. In the article, the weight model of an aircraft is 
described in terms of network data structures.}

\KWE{design automation; aircraft; weight design; weighting model; design tree; project generator}

\DOI{10.14357/19922264210107}

%\vspace*{-15pt}

%\Ack
%\noindent

\vspace*{12pt}

  \begin{multicols}{2}

\renewcommand{\bibname}{\protect\rmfamily References}
%\renewcommand{\bibname}{\large\protect\rm References}

{\small\frenchspacing
 {%\baselineskip=10.8pt
 \addcontentsline{toc}{section}{References}
 \begin{thebibliography}{9}
  \bibitem{1-fl-1}
  \Aue{Vyshinsky, L.\,L., Yu.\,A.~Flerov, and N.\,I.~Shirokov.} 2018. Avtomatizirovannaya sistema 
vesovogo proektirovaniya samoletov [Computer-aided system of aircraft weight \mbox{design}].
\textit{Informatika i~ee Primeneniya~--- Inform Appl.} 12(1):18--30. doi: 10.14357/ 19922264180103.
{\looseness=1

}

\vspace*{-2pt}

  \bibitem{2-fl-1}
  \Aue{Vyshinsky, L.\,L., M.\,K.~Kuryansky, and Yu.\,A.~Flerov.} 2018.Tsifrovaya model' 
vesovogo pasporta letatel'nogo apparata [Digital model of the aircraft's weight passport]. 
\textit{Informatika i~ee Primeneniya~--- Inform Appl.} 13(4):3--10. doi: 10.14357/19922264190401.
\end{thebibliography}

 }
 }

\end{multicols}

\vspace*{-3pt}

  \hfill{\small\textit{Received January~20, 2020}}


%\pagebreak

\vspace*{-12pt}

  \Contr
  
  \noindent
  \textbf{Vyshinsky Leonid L.} (b.\ 1941)~--- Candidate of Science (PhD) in physics and 
mathematics, leading scientist, A.\,A.~Dorodnicyn Computing Center, Federal Research Center 
``Computer Science and Control'' of the Russian Academy of Sciences, 40~Vavilov Str., Moscow 
119333, Russian Federation; \mbox{wysh@ccas.ru} 
  
  \vspace*{3pt}
  
  \noindent
  \textbf{Flerov Yuri A.} (b.\ 1942)~--- 
   Doctor of Science in physics and mathematics, professor, 
   Corresponding Member of the Russian Academy of Sciences,
principal scientist, 
A.\,A.~Dorodnicyn Computing Center, Federal Research Center ``Computer Science and Control'' of 
the Russian Academy of Sciences, 40~Vavilov Str., Moscow 119333, Russian Federation; 
\mbox{fler@ccas.ru}
      
\label{end\stat}

\renewcommand{\bibname}{\protect\rm Литература}