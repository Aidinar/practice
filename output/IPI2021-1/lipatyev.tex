%\DeclareMathOperator{\mathrm{tr}\,}{tr}
\def\stat{lipatyev}

\def\tit{НЕАСИМПТОТИЧЕСКИЙ АНАЛИЗ СТАТИСТИКИ БАРТЛЕТТА--НАНДА--ПИЛАЯ 
ДЛЯ~ДАННЫХ\\ БОЛЬШОЙ РАЗМЕРНОСТИ}

\def\titkol{Неасимптотический анализ статистики Бартлетта--Нанда--Пилая 
для~данных большой размерности}

\def\aut{А.\,А.~Липатьев$^1$}

\def\autkol{А.\,А.~Липатьев}

\titel{\tit}{\aut}{\autkol}{\titkol}

\index{Липатьев А.\,А.}
\index{Lipatiev A.\,A.}

%{\renewcommand{\thefootnote}{\fnsymbol{footnote}} \footnotetext[1]
%{Работа выполнена при частичной финансовой поддержке РФФИ
%(проекты 18-07-00692, 19-07-00739 и~20-07-00804).}}

\renewcommand{\thefootnote}{\arabic{footnote}}
\footnotetext[1]{Московский государственный университет имени М.\,В.~Ломоносова, 
факультет вычислительной математики и~кибернетики,
кафедра математической статистики, \mbox{allipatev@cs.msu.ru}}


\vspace*{-10pt}


\Abst{Представлены вычислимые оценки скорости сходимости нормированной статистики Барт\-лет\-та--Нан\-да--Пи\-лая 
к~стандартному нормальному распределению при условии, что размерность данных возрастает 
пропорционально объему выборки. Приведенный результат позволяет корректно вычислять 
p-зна\-че\-ния в~прикладных задачах многомерного анализа данных. Задачи в~постановке, когда 
число анализируемых признаков сравнимо с~объемом выборки, все чаще возникают в~об\-ласти 
обработки сигналов. Доказательство базируется существенным образом на нормальности распределения 
элементов рассматриваемых матриц с~распределением Уишарта. Для случайных величин, представляющих 
собой матричные следы произведения и~квадратов мат\-риц с~нормированным распределением Уишарта, 
находятся удобные оценки сверху для $1-F$, где $F$~--- функция распределения соответствующего 
следа мат\-ри\-цы. Применяя свойства обратных мат\-риц и~неотрицательно определенных мат\-риц, статистика 
Барт\-лет\-та--Нан\-да--Пи\-лая ограничивается сверху комбинацией из упомянутых выше следов матриц.}

\KW{точность приближений; многомерный дисперсионный анализ; вычислимые оценки; 
статистика Барт\-лет\-та--Нан\-да--Пи\-лая; данные большой размерности}

\DOI{10.14357/19922264210110}


\vspace*{-2pt}

\vskip 10pt plus 9pt minus 6pt

\thispagestyle{headings}

\begin{multicols}{2}

\label{st\stat}




\section{Введение}
\label{sec:intro}

В большом числе прикладных задач исследователи анализируют многомерные данные, 
в~которых количество~$p$ признаков сравнимо с~числом~$n$\linebreak наблюдений. Для анализа 
данных фиксированной размерности существует множество статистических процедур, 
уже ставших классическими.\linebreak
Однако час\-то нет возможности использовать традиционную статистическую процедуру, 
лишь устремив в~ней чис\-ло признаков к~бесконечности,
так как при этом изменяется предельное распределение статистики критерия  (см.~[1,  
разд.~6.3.4]).

Цель данной работы~--- нахождение вычислимых оценок точности аппроксимации 
статистики Барт\-лет\-та--Нан\-да--Пи\-лая (Bartlett--Nanda--Pillai test) нормальным 
распределением в~модели многомерного дисперсионного анализа (MANOVA~--- multivariate analysis of variance) 
для данных 
большой размерности,
когда отношение числа \mbox{признаков} к~чис\-лу  наблюдений~$p/n$ стремится к~некоторой 
константе из интервала $(0, 1)$.

Результаты, касающиеся распределений статистик, возникающих в~модели MANOVA при 
условии, что нулевая гипотеза верна, оказываются полезны в~области обработки 
сигналов. \mbox{Например}, в~\cite{lit:Johnstone_RoyStat} показано, каким образом 
результаты из MANOVA и~обработки сигналов могут сводиться к~спектру определенной 
матрицы~$E^{-1}H$. В~\mbox{статье}~\cite{lit:Akbari_AppliedLH} приводится пример 
применения статистики Лоу\-ли--Хо\-тел\-лин\-га, родственной статистике 
Барт\-лет\-та--Нан\-да--Пи\-лая, в~контексте обработки данных радаров с~синтезированной апертурой.

В разд.~\ref{sec:results} сформулирован основной результат работы~--- 
теорема~1.
Теорема~2 является   вспомогательной, но при этом 
представляет самостоятельный интерес.
В~разд.~\ref{sec:proofs} даны доказательства основных теорем, которые 
опираются на леммы из разд.~\ref{sec:lemmas}.

%%%%%%%%%%%%%%%%%
\section{Постановка задачи и~основной результат}
\label{sec:results}


В рамках многомерного дисперсионного анализа исследуется следующая многомерная 
линейная модель:
$
X\hm=Q\mathbb{B}\hm+\mathcal{E},
$
где $X$~--- случайная матрица наблюдений размера $N \times p$; $Q$~--- неслучайная 
матрица плана эксперимента размера $N \times k$;
$\mathbb{B}$~---  неслучайная матрица $k \times p$ регрессионных коэффициентов;
$\mathcal{E}$~--- матрица ошибок  $N \times p$ с~распределением $N_{N\times 
p}\left(O,I_{N}\otimes\Sigma\right)$.

Рассмотрим следующую линейную гипотезу:
$
H_{0} : C\mathbb{B}\hm=O,
$
где $C$~--- известная матрица размера $q \times k$ ранга~$q$.  Статистики 
критериев, инвариантные относительно некоторой группы аффинных преобразований, 
оказываются функциями от ненулевых собственных значений матрицы $S_{h}S_{e}^{-1}$, где
\begin{equation}
\left.
\begin{array}{rl}
S_{h}&=\hat{\mathbb{B}}^{\mathrm{T}}C^{\mathrm{T}}\left(C\left(Q^{\mathrm{T}}Q\right) ^{-1}C^{\mathrm{T}}\right)^{-
1}C\hat{\mathbb{B}} \,;\\[6pt]
 S_{e}&=\left(X-Q\hat{\mathbb{B}}\right)^{\mathrm{T}}\left(X-
Q\hat{\mathbb{B}}\right)
\end{array}
\right\}
\label{S_h_and_S_e}
\end{equation}
при $\hat{\mathbb{B}}=\left(Q^{\mathrm{T}}Q\right)^{-1}Q^{\mathrm{T}}X$ (см.~[4, гл.~8]). 
Одной из наиболее известных инвариантных статистик является 
статистика Барт\-лет\-та--Нан\-да--Пи\-лая:
$
V_{\mathrm{BNP}}\hm=\left(n\hm+q\right)\mathrm{tr}\, S_{h}\left(S_{h}\hm+S_{e}\right)^{-1}.
$
В~дальнейшем предполагаем, что гипотеза~$H_0$ верна.

В~\cite{lit:MuirLargeSamp} рассмотрен случай большого объема выборки, т.\,е.\ 
выполнено \textit{условие}~\textbf{А1}:
$$
{\bf A1}: p\mbox{ и~}q\mbox{ фиксированы},\ n\rightarrow\infty ,
$$
и получены неасимптотические  оценки точности аппроксимации функции 
распределения статистики Барт\-лет\-та--Нан\-да--Пи\-лая:
\begin{multline*}
    \mathbf{P}\{V_{\mathrm{BNP}}<x\}=G_{a}\left(x\right)+\fr{3a}{4n}\{G_{a}\left(x\right)-{}\\
{}-2G_{a+2}\left(x\right)+G_{a+4}\left(x\right)\}+O\left(n^{-2}\right),
\end{multline*}
где $a=pq$; $G_{a}$~--- функция $\chi^{2}$-рас\-пре\-де\-ле\-ния
с~$a$~степенями свободы. В~\cite{lit:LU01} для остаточного члена найде\-на 
оценка сверху.

В~\cite{lit:WFU} рассмотрен случай большой размерности данных, т.\,е.\ выполнено 
\textit{условие} \textbf{А2}:
\begin{multline*}
{\bf A2}: q\mbox{ фиксировано},\enskip p\rightarrow\infty,\enskip n\rightarrow\infty,\\ 
\fr{p}{n}\rightarrow c\in\left(0;1\right),
\end{multline*}
и получено следующее приближение:
\begin{multline*}
\mathbf{P}\left(\fr{1}{\sigma}T_{\mathrm{BNP}}<z\right)={}\\
{}=\Phi(z)-
\phi(z)\left[\fr{1}{\sqrt{p}}\left\{\fr{1}{\sigma}b_{1}+\fr{1}{\sigma^{3}}
b_{3}\,H_{2}\left(z\right)\right\} + \right.\\
\left.{}+\fr{1}{p}\left\{\fr{1}{\sigma^{2}}b_{2}\,H_{1}\left(z\right) + 
\fr{1}{\sigma^{4}}b_{4}\,H_{3}\left(z\right) + 
\fr{1}{\sigma^{6}}b_{6}\,H_{5}\left(z\right)\right\}\right]
+{}\\
{}+O\left(\fr{1}{p\sqrt{p}}\right),
\end{multline*}
где   
$T_{\mathrm{BNP}}=\sqrt{p}\left(1+m^{-1}p\right)\left\{p^{-1}V_{\mathrm{BNP}}-q\right\}$;
$\Phi (z)$ и~$\phi (z)$~--- соответственно функция распределения 
и~плотность распределения стандартного нормального закона;
$m\hm=n\hm-p\hm+q$; $r \hm= p/m$;
$\sigma\hm=\sqrt{2q(1+r)}$; $b_{i}\hm=b_{i}(r,q)$ суть некоторые функции от~$r$ 
и~$q$;
$H_{i}(z)$~--- полиномы Эрмита.
При этом результат имел именно асимптотический вид, верхние оценки остаточного 
члена не находились.

Основной результат данной работы~--- две тео\-ре\-мы,
дающие оценку точности аппроксимации распределения статистики Барт\-лет\-та--Нан\-да--Пи\-лая
нормальным распределением для данных большой размерности, т.\,е.\ при выполнении 
условия~\textbf{A2}:


\smallskip

\noindent
\textbf{Теорема~1.}
\textit{При всех $m\hm>M\hm=M\left(r,q\right)$ справедливо неравенство}
    $$
     \sup\limits_{z}{\left|\mathbf{P}\left(\fr{T_{\mathrm{BNP}}}{\sqrt{2q\left(1+r\right)}}<z\right)-
\Phi\left(z\!\right)\right|}\leqslant
\fr{K_{2}\left(r, q\right)\ln m}{\sqrt{m}}\,,
    $$
   \textit{где $K_{2}\left(r, q\right)$~--- вычислимая функция от~$r$ и~$q$}.

\smallskip

Отметим, что результат теоремы~1 на логарифмический 
множитель уступает результату из~\cite{lit:WFU}, но превосходит последний в~том, 
что для ошибки погрешности дается вычислимая оценка сверху. При этом само 
доказательство является новым.

\smallskip

\noindent
\textbf{Теорема~2.}
\textit{Пусть матрицы $U$ и~$V$ суть нормированные варианты матриц~$B$ и~$W$}:
    \begin{equation}
    \label{U and V}
    U=  \fr{B-pI_{q}}{\sqrt{p}}\,;\quad V= \fr{W-mI_{q}}{\sqrt{m}}\,,
    \end{equation}
     \textit{где $B$ и~$W$ независимы и~имеют распределения Уишарта $W_{q}\left(p, 
I_{q}\right)$ и~$W_{q}\left(m, I_{q}\right)$ с~$m\hm=n\hm-p\hm +q$ соответственно.
    Если $\mathrm{tr}{\left(\sqrt{r}\,U\hm+V\right)^{2}}\hm<\left(r\hm+1\right)m$, то выполнено 
следующее неравенство}:

\noindent
    \begin{multline}
    \label{Theor2}
     \left|\sqrt{m}\left(r+1\right)\left(\left(r+1\right)\mathrm{tr}\,{B\left(B+W\right)^{-1}}-
     rq\right)-{}\right.\\
\left.     {}-\left(\sqrt{r}\,\mathrm{tr}\,{U}-r\mathrm{tr}\,{V}\right)
\vphantom{\sqrt{}W^{-1}}
\right|\leqslant{}\\
{}\leqslant
\fr{ \left(r+1\right)\sqrt{r}\left(\left| \mathrm{tr}\,{UV}\right|+
\sqrt{r}\,\mathrm{tr}\,U^{2}\right)}
{ \left(r+1\right)\sqrt{m}-
{\mathrm{tr}{\left(\sqrt{r}\,U+V\right)^{2}}}/{\sqrt{m}}}+{}\\
{}+
     \left(   rq\left(r+1\right)+\fr{\left(\sqrt{r}\,\mathrm{tr}\,{U}-r\mathrm{tr}\,{V}\right)}
     {\sqrt{m}}\right)\times{}\\
     {}\times 
     \fr{\mathrm{tr}{\left(\sqrt{r}\,U+V\right)^{2}}}
     { \left(r+1\right)\sqrt{m}-
{\mathrm{tr}{\left(\sqrt{r}\,U+V\right)^{2}}}/{\sqrt{m}}}.
    \end{multline}



Заметим, что   вероятность события, противоположного событию 
$\mathrm{tr}{\left(\sqrt{r}\,U\hm+V\right)^{2}}\hm<\left(r\hm+1\right)m$, фигурирующему 
в~теореме~2, имеет порядок $O\left( {1}/{\sqrt{m}}\right)$, 
как это станет ясно из результатов разд.~\ref{sec:lemmas}.

\section{Вспомогательные утверждения}
\label{sec:lemmas}


В этой части приведены вспомогательные утверж\-де\-ния, используемые 
в~доказательствах тео\-рем~1 и~2.

Введем дополнительные случайные величины:
\begin{equation}
\left.
\begin{array}{rlrl}
Z_1 &= \mathrm{tr}\,{UV};&\hspace{1cm} Z_3 &= \mathrm{tr}\,{U}-\sqrt{r}\,\mathrm{tr}\,{V};\\[6pt]
Z_2 &= \mathrm{tr}\,{V^{2}}; &\hspace{1cm} Z_4 &= \mathrm{tr}\,{U^{2}},
\end{array}
\right\}
\label{Z_i}
\end{equation}
где случайные матрицы~$U$ и~$V$ определены в~\eqref{U and V}.

\pagebreak


Положим
\begin{equation*}
%\label{def_B}
    B = B(q, r, m) = 4\,\left(q^2 + \sqrt{r}\right)\left(\sqrt{\ln m} + \sqrt{\ln p}\right)^2.
\end{equation*}

Определим также для $i \hm= 1, 2, 3, 4$ и~натуральных~$m$ случайные события $A_{i, m}$ 
как 
$$
A_{i, m} = \left\{\omega : |Z_i(\omega)|\leqslant B \right\}.
$$

Положим
\begin{multline*}
%\label{Z}
Z =
   %\left( 
   \vphantom{\fr{\left|\mathrm{tr}\,{U}-\sqrt{r}\,\mathrm{tr}\,{V}\right|}{\sqrt{m}}}
  \fr{\left(r+1\right)\sqrt{r}\left(\left|\mathrm{tr}\,{UV}\right|+\sqrt{r}\,\mathrm{tr}\,{U^{2}}\right)}
  {\left(r+1\right)\sqrt{m}-{S_{Z}}/{\sqrt{m}}}+{}\\
  {}+
\fr{
          rq\left(r+1\right)+\sqrt{r/m}\,{\left|\mathrm{tr}\,{U}-\sqrt{r}\,\mathrm{tr}\,{V}\right|}
}{\left(r+1\right)\sqrt{m}-{S_{Z}}/{\sqrt{m}}}\,
S_{Z}\,,
\end{multline*}
где $S_{Z}=\left(r\mathrm{tr}\,{U^{2}}+2\sqrt{r}\left|\mathrm{tr}\,{UV}\right|+\mathrm{tr}\,{V^{2}}\right).$

Ясно, что существует натуральное $M_1 \hm= M_1(r, q, c)$ такое, что при всех $m\hm\geq 
M_1$ и~$\omega \hm\in \mathop{\cap}\nolimits_{i=1}^4 A_{i, m}$ выполняется
\begin{multline}
\label{ineq_Z}
Z(\omega) \leqslant
\left(
(r+1)\sqrt{r}\left(1+\sqrt{r}\right)B+
    \left(
    \vphantom{\fr{B}{\sqrt{m}}}
    rq(r+1)+{}\right.\right.\\
\left.\left.    {}+\sqrt{r}\,\fr{B}{\sqrt{m}}\right)B\left(1+\sqrt{r}\right)^{2}\right)\Bigg/
\left( 
\vphantom{\fr{B\left(1+\sqrt{r}\right)^{2}}{\sqrt{m}}}
(r+1)\sqrt{m}-{}\right.\\
\left.{}-\fr{B\left(1+\sqrt{r}\right)^{2}}{\sqrt{m}}\right)
\leqslant{}\\
{}\leqslant 
16\,\fr{\left(1+\sqrt{r}\right)^{2}\,\left(2\sqrt{r}+r\,(rq+q+1)\right)
\left(q^2+\sqrt{r}\right)}{r+1}\times{}\\
{}\times \fr{\left(\ln m + \ln\sqrt{r}\right)}{\sqrt{m}}\,.
\end{multline}

Оценим вероятности $\mathbf{P}(A_{i, m}^c)$ для $i \hm= 1, 2, 3, 4$. Согласно 
леммам~1 и~2 из~\cite{lit:LU02} справедливо сле\-ду\-ющее неравенство:
\begin{multline}
    \label{L-1, f-la}
\mathbf{P}\left(|Z_1| > B\right) + \mathbf{P}\left(Z_2 > B\right) +
 \mathbf{P}\left(|Z_3| > B\right) + {}\\
 {}+
\mathbf{P}\left(Z_4 > B\right) \leqslant 25{,}8\,q^2\,\fr{1+1/\sqrt{r}}{\sqrt{m}}\,.
\end{multline}

\noindent
\textbf{Лемма~1.}
\textit{Пусть случайные величины $T$, $Y$ и~$Z$ определены на одном вероятностном 
пространстве $\left(\Omega,\mathbf{A},\mathbf{P}\right)$,
при этом распределение~$Y$ является абсолютно непрерывным с~ограниченной 
плотностью $f_{Y}\left(z\right)$.
Предположим, что для некоторого события $A\hm\in \mathbf{A}$ при всех $\omega\hm\in A$ 
выполнено следующее соотношение}:
$$
\left|T(\omega)-Y(\omega)\right|\leqslant Z(\omega) \leqslant a
$$
\textit{с некоторой положительной постоянной~$a$.
Тогда
справедливо неравенство}:
\begin{multline}
\label{uniform}
\sup\limits_{x}\left|\mathbf{P}(T<x)-\mathbf{P}(Y<x)\right| \leqslant{}\\
{}\leqslant \mathbf{P}(A^c) + 
a\sup\limits_{x}{f_{Y}\left(x\right)}.
\end{multline}  

\noindent
Д\,о\,к\,а\,з\,а\,т\,е\,л\,ь\,с\,т\,в\,о\ \ леммы~1. См.\ лемму~3 в~\cite{lit:LU02}.~\hfill $\Box$

В следующих двух леммах приводятся два известных результата о скорости 
сходимости в~центральной предельной теореме для независимых одинаково 
распределенных случайных величин.
Первый из результатов относится к~случайным величинам без ограничений на тип 
распределения.
Второй результат относится к~случайной величине с~распределением $\chi^{2}$, 
рассматриваемой как сумма независимых одинаково распределенных 
случайных величин с~известным распределением. Согласно~\cite{lit:Shevtsova1}, справедлива 
следующая лемма.

\noindent
\textbf{Лемма~2.}
\textit{Пусть случайные величины $\xi_{1},\,\xi_{2},\ldots$ независимы и~одинаково 
распределены, выполнено $\mathbf{D}\xi_{1}\hm=\sigma^{2}\hm>0$ и~cуществует 
$\mathbf{E}\left|\xi_{1}\right|^{3}\hm<\infty$. Тогда  для нормированной суммы}
$
T_{n}\hm=
{(S_{n}-\mathbf{E}{S_{N}})}/{\sqrt{\mathbf{D}\,{S_{N}}}}
$
\textit{выполнено неравенство}:
$$
\sup\limits_{x}{\left|F_{T_{n}}\left(x\right)-\Phi\left(x\right)\right|}\leqslant 
0{,}4748\fr{\mathbf{E}\left|\xi_{1}-
\mathbf{E}\xi_{1}\right|^{3}}{\sigma^{3}\sqrt{n}}\,.
$$


Случайная величина с~функцией распределения $G_{p}(x)$, имеющая   $\chi^{2}$-рас\-пре\-де\-ле\-ние 
с~$p$~степенями свободы, может быть представлена в~виде суммы~$p$~независимых 
одинаково распределенных случайных величин с~$\chi^{2}$-рас\-пре\-де\-ле\-ни\-ем с~одной степенью свободы.
Этот факт позволяет дать более точные оценки точ\-ности аппроксимации нормальным 
распределением, чем те, которые можно получить в~общем случае с~помощью
неравенства Бер\-ри--Ес\-се\-ена, а~именно: имеет место следующий результат (см.\ лемму~2 
в~\cite{lit:Kavaguchi} при $\lambda\hm=0{,}5$).

\smallskip

\noindent
\textbf{Лемма~3.}
\textit{Для всех $\lambda\hm\in\left(0; \sqrt{3}-1\right)$ и~целых $p\hm>1$ выполнено}
    $$
    \sup\limits_{x}{\left|G_{p}\left(p+x\sqrt{2p}\right)-
\Phi\left(x\right)\right|}\leqslant \fr{6{,}22}{\sqrt{p}}\,.
    $$

\smallskip

\noindent
\textbf{Лемма~4.}
\textit{Для любых случайных величин $X$ и~$Y$ и~любого действительного числа $a > 
0$ справедливы неравенства}:
    \begin{equation*}
    %\label{split_formula_1}
    \mathbf{P}(|X+Y|\geq 2a) \leqslant \mathbf{P}(|X|\geq a)  + 
\mathbf{P}(|Y|\geq a)\,;
    \end{equation*}
          \begin{equation*}
  %  \label{split_formula_2}
    \mathbf{P}(|X\cdot Y|\geq a^2) \leqslant \mathbf{P}(|X|\geq a)  + 
\mathbf{P}(|Y|\geq a).
    \end{equation*}

\noindent
Д\,о\,к\,а\,з\,а\,т\,е\,л\,ь\,с\,т\,в\,о\ \ леммы~4 очевидным образом вытекает из рассуждений 
от противного.~\hfill $\Box$

\smallskip

\noindent
\textbf{Лемма~5.}
\textit{Если случайные величины~$X_1, \dots , X_k$ независимы и~таковы, что   $|\mathbf{P}(X_j 
\hm\leqslant x) \hm- \Phi (x)| \hm\leqslant D_j$ при всех~$x$ и~$j \hm= 1, \dots , k$
 с~некоторыми постоянными   $D_1, \dots , D_k$, то}
  \begin{equation*}
  \left| \mathbf{P}\left(\sum\limits_{j=1}^{k}c_j\,X_j \leqslant x\right) - \Phi (x)\right| \leqslant 
\sum\limits_{j=1}^{k} D_j,
  \end{equation*}
\textit{где $c_1,\dots , c_k $ суть произвольные постоянные, для которых}   $c_1^2 + 
\dots + c_k^2 \hm= 1$.

\noindent
Д\,о\,к\,а\,з\,а\,т\,е\,л\,ь\,с\,т\,в\,о\ \ леммы~5 см., например, в~теореме~3.1 
в~~\cite{lit:Letters_2006}. \hfill $\Box$



\section{Доказательства теорем~1 и~2}
\label{sec:proofs}


Начнем с~доказательства теоремы~2, поскольку неравенство~\eqref{Theor2} 
является ключевым в~доказательстве 
теоремы~1.

\noindent
Д\,о\,к\,а\,з\,а\,т\,е\,л\,ь\,с\,т\,в\,о\ \ теоремы~2.
Воспользовавшись матричным равенством
\begin{equation*}
%\label{Matr Geom}
\left(I+A\right)^{-1}-\left(I-A\right)=A^{2}\left(I+A\right)^{-1},
\end{equation*}
из определения \eqref{U and V} получаем
\begin{multline*}
\left(B+W\right)^{-1}=\left(\sqrt{p}\,U+pI_{q}+\sqrt{m}\,V+mI_{q}\right)^{-1}={}\\
{}=\fr{1}{p+m}\left(
\vphantom{\fr{\left(\sqrt{p}\,U+\sqrt{m}\,V\right)^{2}}
{p+m}}
I_{q}-
\fr{1}{p+m}\left(\sqrt{p}\,U+\sqrt{m}\,V\right)+{}\right.\\
\left.{}+\fr{\left(\sqrt{p}\,U+\sqrt{m}\,V\right)^{2}}
{p+m}\left(B+W\right)^{-1}\right),
\end{multline*}
далее
\begin{multline*}
%\label{T2_1}
\sqrt{m}\left(r+1\right)\left(\left(r+1\right)B\left(B+W\right)^{-1}-
rI_{q}\right)-{}\\
{}-\left(\sqrt{r}\,U-rV\right)=\fr{\sqrt{r}}{\sqrt{m}}U\left(\sqrt{r}\,U+V\right)+{}\\
{}+
        \fr{1}{\sqrt{m}}\,B\left(\sqrt{r}\,U+V\right)^{2}\left(B+W\right)^{-1}.
\end{multline*}
Отсюда для следов этих матриц имеем следующее неравенство:
\begin{multline}
\label{trBW}
\left|\sqrt{m}\left(r+1\right)\left(\left(r+1\right)\mathrm{tr}\,{B\left(B+W\right)^{-1}}-
rq\right)-{}\right.\\
\left.{}-\left(\sqrt{r}\,\mathrm{tr}\,{U}-r\mathrm{tr}\,{V}\right)
\vphantom{\left(W\right)^{-1}}
\right|
\leqslant{}\\
{}\leqslant \fr{1}{\sqrt{m}}\sqrt{r}\left(\left|\mathrm{tr}\,{UV}\right|+\sqrt{r}\,\mathrm{tr}\,{U^{2}}\right)+{}\\
{}+
\fr{1}{\sqrt{m}}\left|\mathrm{tr}\,{\left[B\left(\sqrt{r}\,U+V\right)^{2}\left(B+W\right)^{-1}\right]}
\right|\leqslant{}\\
{}\leqslant
\fr{1}{\sqrt{m}}\sqrt{r}\left(\left|\mathrm{tr}\,{UV}\right|+\sqrt{r}\,\mathrm{tr}\,{U^{2}}\right)+{}\\
{}+
\fr{1}{\sqrt{m}}\,\mathrm{tr}\,{\left(\sqrt{r}\,U+V\right)^{2}}\mathrm{tr}\,{B\left(B+W\right)^{-1}}.
\end{multline}
Для получения предпоследнего неравенства использованы симметричность 
и~неотрицательная определенность обеих случайных матриц 
$\left(\sqrt{r}\,U+V\right)^{2}$ и~$B\left(B+W\right)^{-1}$, поскольку для 
сим\-мет\-рич\-ных неотрицательно определенных матриц~$X$ и~$Y$ выполнено   
соотношение (см.~\cite{lit:TraceIneq})
$\mathrm{tr}\,{XY}\hm\leqslant\mathrm{tr}\,{X}\mathrm{tr}\,{Y}.$

Видно, что случайная величина $\mathrm{tr}\,{B\left(B+W\right)^{-1}}$ фигурирует в~крайней 
левой и~крайней правой час\-тях неравенства~\eqref{trBW}.
Преобразуя полученное неравенство, получаем, что при 
$\mathrm{tr}\,{\left(\sqrt{r}\,U+\hm V\right)^{2}}\hm<\left(r\hm+1\right)m$ выполнено~\eqref{Theor2}. 
Тем самым доказательство теоремы~2 завершено.
\hfill $\Box$

\smallskip

Переходим к~доказательству теоремы~1.

\smallskip

\noindent
Д\,о\,к\,а\,з\,а\,т\,е\,л\,ь\,с\,т\,в\,о\ \ теоремы~1.
Используя лемму~1 из~\cite{lit:WFU}, перейдем к~представлению статистики 
Барт\-лет\-та--Нан\-да--Пи\-лая
\begin{multline*}
T_{\mathrm{BNP}}={}\\
{}=\sqrt{p}\left(\!1+\fr{p}{m}\!\right)\!\left\{\!\left(1+\fr{m}{p}\right)\mathrm{tr}\!\left[S_{h}
\left(S_{h}+S_{e}\right)^{-1}\right]-q\right\}\hspace*{-4.9pt}
\end{multline*}
в терминах матриц~$B$ и~$W$ размера $q \times q$ вместо матриц~$S_{h}$ и~$S_{e}$ 
размера $p \times p$, где $S_{h}$ и~$S_{e}$ определены в~\eqref{S_h_and_S_e}, 
а~матрицы~$B$ и~$W$ независимы и~имеют распределения Уишарта $W_{q}\left(p, 
I_{q}\right)$ и~$W_{q}\left(m, I_{q}\right)$ с~$m\hm=n\hm-p\hm+q$ соответственно.
При этом будем пользоваться сле\-ду\-ющим соотношением (см.~\cite{lit:WFU}):
$$
\mathrm{tr}\,{S_{h}\left(S_{h}+S_{e}\right)^{-1}}=
\mathrm{tr}\,{B\left(B+W\right)^{-1}}.
$$

Согласно~\eqref{Theor2} для $Z_1$, $Z_2$, $Z_3$ и~$Z_4$ (см.\ определение 
в~\eqref{Z_i}) при $rZ_{4}\hm+2\sqrt{r}\left|Z_{1}\right|\hm+Z_{2} \hm< m$ 
имеем:
\begin{multline*}
%\label{proof_T-1-1}
|\sqrt{r}\,T_{\mathrm{BNP}} - \left(\sqrt{r}\,\mathrm{tr}\,{U}-r\mathrm{tr}\,{V}\right)|={}\\
{}= \left|\sqrt{m}\left(r+1\right)\left(\left(r+1\right)\mathrm{tr}\,{B\left(B+W\right)^{-1}}
-rq\right)-{}\right.\\
\left.{}-\left(\sqrt{r}\,\mathrm{tr}\,{U}-r\mathrm{tr}\,{V}\right)
\vphantom{\left(W\right)^{-1}}
\right|\leqslant{}\\
{}\leqslant
  \left( 
   \vphantom{\fr{\sqrt{r}}{\sqrt{v}}}
   \left(r+1\right)\sqrt{r}\left(\left|Z_{1}\right|+\sqrt{r}\,Z_{4}\right)+{}\right.
 \\
 {}+     \left(
\vphantom{\fr{\sqrt{r}\left|Z_{3}\right|}{\sqrt{m}}}
    rq\left(r+1\right)+\fr{\sqrt{r}\left|Z_{3}\right|}{\sqrt{m}}\right)\left(rZ_{4}+
2\sqrt{r}\left|Z_{1}\right|+{}\right.\\
\left.\left.{}+Z_{2}\right)
\vphantom{\fr{\sqrt{r}}{\sqrt{v}}}
\right)\Bigg/
    \left( \left(r+1\right)\sqrt{m}-
\fr{rZ_{4}+2\sqrt{r}\left|Z_{1}\right|+Z_{2}}{\sqrt{m}}\right).\hspace*{-6.32483pt}
\end{multline*}
Следовательно, в~силу~\eqref{ineq_Z} и~\eqref{uniform} при всех $m\hm\geq M_1$ 
получаем
\begin{multline}
\label{interm}
\mathop{\sup}\limits_{z}\left|\mathbf{P}\left (\fr{T_{\mathrm{BNP}}}
{\sqrt{2q\left(1+r\right)}}<z\right)-{}\right.\\
\left.{}-
\mathbf{P}\left(\fr{\mathrm{tr}\,{U}-\sqrt{r}\,\mathrm{tr}\,{V}}{\sqrt{2q\left(1+r\right)}} < z 
\right)\right| \leqslant{}\\
{}\leqslant \sum\limits_{i=1}^{4} \mathbf{P}\left(|Z_i|> B\right) +  K_{4}(r, q)\,
\fr{\ln m}{\sqrt{m}}\, \sup\limits_{x}{f(x)},
\end{multline}
где $f(x)$ есть плотность случайной величины ${(\mathrm{tr}\,{U}\hm-
\sqrt{r}\,\mathrm{tr}\,{V})}/{\sqrt{2q\left(1\hm+r\right)}}$;
$K_{4}\left(r, q\right)$~--- 
некоторая вычислимая функция от~$r$ и~$q$.

Отметим, что, поскольку матрицы~$B$ и~$W$ независимы, матрицы $U$ и~$V$ также 
независимы между собой. Известно (см., например, гл.~2 в~\cite{lit:FUS01}), что 
$\mathrm{tr}\, B$ и~$\mathrm{tr}\, W$ имеют $\chi^{2}$-рас\-пре\-де\-ле\-ния с~$pq$ и~$mq$ степенями свободы 
соответственно. Известно также, что плотность $\chi^{2}$-распределения с~$k\hm\geq 3$ степенями свободы ограничена сверху величиной
$1/(2\sqrt{\pi\left(k-2\right)}).$
Поэтому для плотности $f(x)$ справедлива равномерная оценка
\begin{multline}
\label{density}
f(x) \leqslant{}\\
\hspace*{-1.5mm}{}\leqslant \min{\left( \fr{\sqrt{p}}{ \sqrt{ \left(pq-
2\right)}},\,\fr{\sqrt{m}}{\sqrt{r\left(mq-2\right)}}\right)} 
\fr{\sqrt{q(1+r)}}{\sqrt{2\pi}}\,.\!\!\!
\end{multline}
Объединяя утверждения лемм~3 и~5 и~соотношения~\eqref{U and V}, \eqref{L-1, f-la}, \eqref{interm} 
и~\eqref{density}, получаем утверждение теоремы~1.\hfill $\Box$%\\[2ex]


%\vspace*{-8pt}

{\small\frenchspacing
{%\baselineskip=10.8pt
%\addcontentsline{toc}{section}{References}
\begin{thebibliography}{99}

%\vspace*{-2pt}
\bibitem{lit:FUS01}
\Au{Fujikoshi Y., Ulyanov~V.\,V., Shimizu~R.}  Multivariate statistics: High-dimensional 
and large-sample approximations.~--- Hoboken, NJ, USA: John Wiley \& Sons, 2010. 512~p.

\bibitem{lit:Johnstone_RoyStat}
\Au{Johnstone I.\,M., Nadler~B.} Roy's largest root test under rank-one 
alternatives~// Biometrika, 2017. Vol.~104. No.\,1. P.~181--193.

\bibitem{lit:Akbari_AppliedLH}
\Au{Akbari V., Anfinsen S.\,N., Doulgeris~A.\,P., Eltoft~T., Moser~G., Serpico~S.\,B.} 
Polarimetric SAR change detection with the complex Hotelling--Lawley 
trace statistic~// IEEE T.~Geosci. Remote,  2016. Vol.~54. Iss.~7. 
P.~3953--3966.

\bibitem{lit:And}
\Au{Anderson T.\,W.}  An introduction to multivariate analysis.~--- 3rd ed.~--- Hoboken, NJ,
USA: John Wiley \& Sons,  2003. 742~p.

\bibitem{lit:MuirLargeSamp}
\Au{Muirhead R.\,J.}  Asymptotic distributions of some multivariate tests~// 
Ann. Math. Stat., 1970. Vol.~41. No.\,3. P.~1002--1010.

\bibitem{lit:LU01}
\Au{Липатьев А.\,А., Улья\-нов~В.\,В.} Вы\-чис\-ли\-мые оценки точ\-ности при\-бли\-же\-ний 
для статистики Барт\-лет\-та--Нан\-да--Пил\-лай~//
Математические труды, 2016. Т.~19. №\,2. С.~109--118.

\bibitem{lit:WFU}
\Au{Wakaki~H., Fujikoshi~Y., Ulyanov~V.\,V.}  Asymptotic expansions of the 
distributions of
MANOVA test statistics when the dimension is large~// Hiroshima Math.~J., 2014. 
Vol.~44. No.\,3. P.~247--259.

\bibitem{lit:LU02}
\Au{Липатьев А.\,А., Ульянов~В.\,В.} Неасимптотический анализ статистики 
Лоу\-ли--Хо\-тел\-лин\-га для данных большой раз\-мер\-ности~//
Записки научных семинаров \mbox{ПОМИ}, 2019. Т.~486. С.~178--189.

\bibitem{lit:Shevtsova1}
\Au{Shevtsova I.\,G.}  On the absolute constants in the Berry--Esseen type 
inequalities for identically distributed summands~//
arXiv.org, 2011. arXiv:1111.6554 [math.PR]. 7~p.

\bibitem{lit:Kavaguchi}
\Au{Кавагучи Ю., Ульянов~В.\,В., Фуджикоши~Я.}  Приближения для статистик,
описывающих геометрические свойства данных большой размерности, с~оценками 
ошибок~//
Информатика и~её применения, 2010. Т.~4. Вып.~1. С.~22--27.

\bibitem{lit:Letters_2006}
\Au{Ulyanov V.\,V., Wakaki~H., Fujikoshi~Y.}  Berry--Esseen bound for high 
dimensional asymptotic approximation of Wilks' Lambda distribution~//
Stat. Probabil. Lett., 2006. Vol.~76. No.\,12. P.~1191--1200.

\bibitem{lit:TraceIneq}
\Au{Coope I.\,D.}  On matrix trace inequalities and related topics for 
products of Hermitian matrices~//
J.~Math. Anal. Appl., 1949. Vol.~188. No.\,3. P.~999--1001.
\end{thebibliography}

}
}

\end{multicols}

\vspace*{-3pt}

\hfill{\small\textit{Поступила в~редакцию 07.01.2020}}

\vspace*{8pt}

%\pagebreak

%\newpage

%\vspace*{-28pt}

\hrule

\vspace*{2pt}

\hrule

%\vspace*{-2pt}

\def\tit{NONASYMPTOTIC ANALYSIS OF~BARTLETT--NANDA--PILLAI STATISTIC FOR~HIGH-DIMENSIONAL DATA}

\def\titkol{Nonasymptotic analysis of~Bartlett--Nanda--Pillai statistic for~high-dimensional data}

\def\aut{A.\,A.~Lipatiev}

\def\autkol{A.\,A.~Lipatiev}

\titel{\tit}{\aut}{\autkol}{\titkol}

\vspace*{-11pt}


\noindent
Department of Mathematical Statistics, Faculty of Computational 
Mathematics and Cybernetics, M.\,V.~Lomonosov Moscow State University, 
1-52~Leninskiye Gory, GSP-1, Moscow 119991, Russian Federation


\def\leftfootline{\small{\textbf{\thepage}
\hfill INFORMATIKA I EE PRIMENENIYA~--- INFORMATICS AND
APPLICATIONS\ \ \ 2021\ \ \ volume~15\ \ \ issue\ 1}
}%
\def\rightfootline{\small{INFORMATIKA I EE PRIMENENIYA~---
INFORMATICS AND APPLICATIONS\ \ \ 2021\ \ \ volume~15\ \ \ issue\ 1
\hfill \textbf{\thepage}}}

\vspace*{3pt}




\Abste{The author gets the computable error bounds for normal 
approximation of Bartlett--Nanda--Pillai statistic when dimensionality grows proportionally
 to the sample size. This result enables one to get more precise calculations of the p-values 
 in applications of multivariate analysis. In practice, more and more often, analysts encounter 
 situations when the number of factors is large and comparable with the sample size. 
 The examples include signal processing. The proof is essentially based on the normality 
 of the distribution of the elements of the matrices under consideration with the Wishart 
 distribution. For random variables that are the matrix traces of the product and squares 
 of matrices with the normalized Wishart distribution, convenient upper bounds for $1-F$ are 
 found\linebreak\vspace*{-12pt}}
 
 \Abstend{where $F$ is the distribution function of the corresponding matrix trace. Applying 
 the properties of inverse matrices and positive semidefinite matrices, the Bartlett--Nanda--Pillai 
statistic is bounded from above by a combination of the above-mentioned matrix traces.}

\KWE{computable estimates; accuracy of approximation; MANOVA; computable error bounds; 
Bartlett--Nanda--Pillai statistic; high-dimensional data}

\DOI{10.14357/19922264210110}

%\vspace*{-15pt}

%\Ack
%\noindent


\vspace*{12pt}

  \begin{multicols}{2}

\renewcommand{\bibname}{\protect\rmfamily References}
%\renewcommand{\bibname}{\large\protect\rm References}

{\small\frenchspacing
 {%\baselineskip=10.8pt
 \addcontentsline{toc}{section}{References}
 \begin{thebibliography}{99}
\bibitem{1-lip-1}
\Aue{Fujikoshi, Y., V.\,V.~Ulyanov, and R.~Shimizu.} 2010. 
\textit{Multivariate statistics: High-dimensional and large-sample approximations}.
 Hoboken, NJ: John Wiley \& Sons. 512~p.
\bibitem{2-lip-1}
\Aue{Johnstone, I.\,M., and B.~Nadler.} 2017. Roy's largest root test under rank-one 
alternatives. \textit{Biometrika} 104(1):181--193.
\bibitem{3-lip-1}
\Aue{Akbari, V., S.\,N.~Anfinsen, A.\,P.~Doulgeris, T.~Eltoft, G.~Moser, and S.\,B.~Serpico.}
 2016. Polarimetric SAR change detection with the complex Hotelling--Lawley trace statistic. 
 \textit{IEEE~T. Geosci. Remote} 54(7):3953--3966.
\bibitem{4-lip-1}
\Aue{Anderson, T.\,W.} 2003. \textit{An introduction to multivariate analysis}. 
3rd ed. Hoboken, NJ: John Wiley \& Sons. 742~p.
\bibitem{5-lip-1}
\Aue{Muirhead, R.\,J.} 1970. Asymptotic distributions of some multivariate tests. 
\textit{Ann. Math. Stat.} 41(3):1002--1010.
\bibitem{6-lip-1}
\Aue{Lipatiev, A.\,A., and V.\,V.~Ulyanov.} 2017. On computable estimates for accuracy of approximation
 for the Bartlett--Nanda--Pillai statistic. \textit{Siberian Adv. Math.} 27(3):153--159.
\bibitem{7-lip-1}
\Aue{Wakaki, H., Y.~Fujikoshi, and V.\,V.~Ulyanov.} 2014. Asymptotic expansions of the distributions 
of MANOVA test statistics when the dimension is large. \textit{Hiroshima Math.~J.} 44(3):247--259.
\bibitem{8-lip-1}
\Aue{Lipatiev, A.\,A., and V.\,V.~Ulyanov.}
 2019. Neasimptoticheskiy analiz statistiki Louli--Khotellinga dlya dannykh bol'shoy razmernosti 
 [Nonasymptotic analysis of Lawley--Hotelling statistic for high dimensional data]. 
 \textit{Zapiski nauchnykh seminarov POMI} [POMI Notes of Scientific Seminars] 486:178--189.
\bibitem{9-lip-1}
\Aue{Shevtsova, I.\,G.} 2011. On the absolute constants in the Berry--Esseen type inequalities 
for identically distributed summands. arXiv:1111.6554 [math.PR]. Available at: 
{\sf https://arxiv.org/pdf/1111.6554} (accessed December~16, 2020).
\bibitem{10-lip-1}
\Aue{Kawaguchi, Yu., V.\,V.~Ulyanov, and Ya.~Fujikoshi.}
 2010. Priblizheniya dlya statistik, opisyvayushchikh geo\-met\-ri\-che\-skie svoystva dannykh 
 bol'shoy razmernosti, s~otsenkami oshibok [Asymptotic distributions of basic statistics in 
 geometric representation for high-dimensional data and their error bounds]. 
 \textit{Informatika i~ee Primeneniya~--- Inform. Appl.} 4(1):22--27.
\bibitem{11-lip-1}
\Aue{Ulyanov, V.\,V., H.~Wakaki, and Y.~Fujikoshi.} 2006. 
Berry--Esseen bound for high dimensional asymptotic approximation of Wilks' Lambda distribution. 
\textit{Stat. Probabil. Lett.} 76(12):1191--1200.
\bibitem{12-lip-1}
\Aue{Coope, I.\,D.} 1949. On matrix trace inequalities and related topics for products of Hermitian matrices. 
\textit{J.~Math. Anal. Appl.} 188(3):999--1001.
 \end{thebibliography}

 }
 }

\end{multicols}

\vspace*{-3pt}

  \hfill{\small\textit{Received January~7, 2020}}


%\pagebreak

%\vspace*{-8pt}

\Contrl

\noindent
\textbf{Lipatiev Alexander A.} (b.\ 1988)~--- 
PhD student,  Faculty of Computational Mathematics and Cybernetics,
 M.\,V.~Lomonosov Moscow State University, 1-52~Leninskiye Gory, GSP-1, Moscow 119991, 
 Russian Federation; \mbox{allipatev@cs.msu.ru}


\label{end\stat}

\renewcommand{\bibname}{\protect\rm Литература}