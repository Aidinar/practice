\def\stat{kudr-ar}

\def\tit{ВЕРОЯТНОСТНЫЕ ХАРАКТЕРИСТИКИ ИНДЕКСА БАЛАНСА ФАКТОРОВ, ИМЕЮЩИХ ОБОБЩЕННЫЕ 
ГАММА-РАСПРЕДЕЛЕНИЯ$^*$}

\def\titkol{Вероятностные характеристики индекса баланса факторов, имеющих обобщенные 
гамма-распределения}

\def\aut{Е.\,Н.~Арутюнов$^1$,  А.\,А.~Кудрявцев$^2$, Ю.\,Н.~Недоливко$^3$}

\def\autkol{Е.\,Н.~Арутюнов,  А.\,А.~Кудрявцев, Ю.\,Н.~Недоливко}

\titel{\tit}{\aut}{\autkol}{\titkol}

\index{Арутюнов Е.\,Н.}
\index{Кудрявцев А.\,А.}
\index{Недоливко Ю.\,Н.}
\index{Arutyunov E.\,N.}
\index{Kudryavtsev A.\,A.}
\index{Nedolivko Iu.\,N.}

{\renewcommand{\thefootnote}{\fnsymbol{footnote}} \footnotetext[1]
{Работа выполнена при частичной финансовой поддержке РФФИ (проект 20-07-00655).}}

\renewcommand{\thefootnote}{\arabic{footnote}}
\footnotetext[1]{Институт проблем информатики Федерального исследовательского 
центра <<Информатика и управление>> Российской академии наук, 
\mbox{enapoleon@mail.ru}}
\footnotetext[2]{Московский государственный университет имени М.\,В.~Ломоносова, 
факультет вычислительной математики и~кибернетики, \mbox{nubigena@mail.ru}}
\footnotetext[3]{Московский государственный университет имени М.\,В.~Ломоносова,
 факультет вычислительной математики и кибернетики, \mbox{mouse98@mail.ru}}


\vspace*{-16pt}


\Abst{Приводятся основные вероятностные характеристики индекса баланса в байесовской 
постановке в предположении, что негативный и позитивный факторы имеют априорные 
обобщенные гам\-ма-рас\-пре\-де\-ле\-ния. Формулировка задачи сводится к изучению характеристик 
масштабной смеси обобщенных гам\-ма-за\-ко\-нов. Особое внимание уделяется случаю, в котором распределения 
факторов имеют параметры формы противоположных знаков. Приводятся моментные характеристики и различные 
представления для плот\-ности в терминах гам\-ма-экс\-по\-нен\-ци\-аль\-ной функции, функций 
Фокса и~Макдональда, а~также обобщенной гипергеометрической функции. Метод анализа 
основан на применении преобразования Меллина и~его обращении. Приводятся новые свойства 
гам\-ма-экс\-по\-нен\-ци\-аль\-ной функции. Полученные результаты могут найти широкое применение 
в~естественно-научных 
моделях, использующих для описания процессов и явлений распределения с положительным неограниченным 
носителем.}


\KW{байесовский подход; обобщенное гамма-распределение; гамма-экспоненциальная функция; модели баланса; смешанные 
распределения; преобразование Меллина; функция Фокса; гипергеометрическая функция}


\DOI{10.14357/19922264210109}

\vspace*{-6pt}


\vskip 10pt plus 9pt minus 6pt

\thispagestyle{headings}

\begin{multicols}{2}

\label{st\stat}

\section{Введение}


В работе~\cite{Ku2018} подробно описывалась байесовская модель баланса, в которой один из 
основных показателей, так называемый индекс баланса~$\rho$, представляет собой отношение двух 
случайных величин с заданными априорными распределениями: негативно влияющего на функционирование 
системы n-фак\-то\-ра~$\lambda$ к позитивному p-фак\-то\-ру~$\mu$.

В ряде предыдущих работ авторов рассматривались различные априорные распределения из гам\-ма-клас\-са. 
Основным требованием к изучаемым распределениям было ограничение на знаки параметров формы.
 Общий вид плотности индекса баланса факторов, имеющих распределения с параметрами формы одного 
 знака, был сформулирован в~\cite{Ku2019_1} в терминах гам\-ма-экс\-по\-нен\-ци\-аль\-ной 
 функции~\cite{KuTi2017}:
 
 \vspace*{-6pt}
 
 \noindent
\begin{multline*}
%\begin{equation}\label{GEF}
{\sf Ge}_{\alpha,\, \beta} (x) = \sum\limits_{k=0}^{\infty}
\fr{x^k}{k!}\, \Gamma(\alpha k + \beta), \\
 x\in\mathbb{R}\,, \enskip  0\le\alpha<1\,, \enskip  \beta> 0\,.
\end{multline*}

\vspace*{-4pt}

\noindent
В~предположении, что независимые n-фак\-тор~$\lambda$, имеющий обобщенное гам\-ма-рас\-пре\-де\-ле\-ние, 
или распределение Криц\-ко\-го--Мен\-ке\-ля~\cite{KrMe1946,KrMe1948}, $\mathrm{GG}\,(v,q,\theta)$ с~плот\-ностью

\vspace*{-4pt}

\noindent
\begin{multline*}
f_\lambda(x)=\fr{|v| x^{vq -1}e^{-(x/\theta)^v}}{\theta^{vq}\,\Gamma(q)}, \\
v\neq0\,, \enskip q>0\,,  \enskip \theta >0\,, \enskip x>0\,,
\end{multline*}

\vspace*{-4pt}

\noindent
и p-фактор~$\mu$, имеющий распределение $\mathrm{GG}\,(u,p,\alpha)$, удовлетворяют ограничению $uv\hm>0$ 
на параметры формы, справедливо следующее утверждение~\cite{Ku2019_1}.

\smallskip

\noindent
\textbf{Теорема~1.}\ 
\textit{Пусть независимые случайные величины~$\lambda$ и $\mu$ имеют соответственно 
распределения $\mathrm{GG}\,(v,q,\theta)$ 
и~$\mathrm{GG}\,(u,p,\alpha)$, причем $uv\hm>0$. Тогда их отношение $\rho\hm=\lambda/\mu$ 
при $x\hm>0$ имеет плот\-ность}

\vspace*{-6pt}

\noindent
\begin{multline*}
f_{\rho}(x) = {}\\[-3pt]
{}= \begin{cases}
   \displaystyle \fr{\abs{v}\alpha^{vq}x^{vq-1}}{\theta^{vq}\,
   \Gamma(p)\Gamma(q)}{\sf Ge}_{v/u,\, vq/u+p}\left(-\left(\fr{\alpha x}{\theta}\right)^{v}\right), &\\
 &   \hspace*{-15mm} |u|>|v|;\\
   \displaystyle \fr{\abs{u}\theta^{up}x^{-up-1}}{\alpha^{up}\Gamma(p)\Gamma(q)}\,
   {\sf Ge}_{u/v,\, up/v+q}\left(-\left(\fr{\alpha x}{\theta}\right)^{-u}\!\right), &\hspace*{-2.96pt}\\
  &\hspace*{-15mm}|v|>|u|;\\
   \displaystyle \fr{\abs{v}(\alpha/\theta)^{vq}x^{vq-1}}{B(p,q)(1+(\alpha x/\theta)^{v})^{p+q}}, 
   &\hspace*{-15mm}u = v.
 \end{cases}
 \hspace*{-2.96pt}
\end{multline*}


Утверждение теоремы~1 может быть легко переформулировано 
для масштабной смеси обобщенных гам\-ма-рас\-пре\-де\-ле\-ний 
с~параметрами формы противоположных знаков, поскольку если случайная величина~$\xi$ 
имеет распределение $\mathrm{GG}\,(v,q,\theta)$, то случайная величина $1/\xi$ имеет распределение 
$\mathrm{GG}\,(-v,q,1/\theta)$.

Заметим, что при $uv<0$ нахождение плот\-ности $\rho$ напрямую сопряжено 
с~принципиальными трудностями вычисления интеграла
$\int\nolimits_0^{\infty}y^{r-1}e^{-(y/\alpha)^u-(y/\theta)^v} \, dy$.

Далее описывается метод получения вероятностных характеристик индекса баланса~$\rho$, 
основанный на обращении преобразования Меллина.

\section{Основные результаты}


Пусть $\xi$~--- неотрицательная случайная величина с функцией распределения~$F_\xi(x)$. 
Преобразование Меллина случайной величины~$\xi$ определяется~\cite{Zolotarev1957,Galambos2004} 
сле\-ду\-ющим образом:
\begin{equation}
\label{Mellin_transform}
\mathcal{M}_\xi(s) = \int\limits_{0}^{\infty}x^s \, dF_\xi(x),\enskip s\in\mathbb{C}\,;
\end{equation}
$\mathcal{M}_\xi(s)$ предполагается конечным; величина $0^s$ предполагается равной нулю для всех~$s$.

Поскольку далее рассматривается преобразование Меллина смесей обобщенных гам\-ма-рас\-пре\-де\-ле\-ний, 
име\-ющих неотрицательный носитель, ограничимся определением~(\ref{Mellin_transform}), 
допускающим \mbox{обобщение} на случай произвольных случайных величин.

Приведем ряд известных свойств преобразования Меллина неотрицательных случайных 
величин~\cite{Zolotarev1957,Galambos2004}.

\smallskip

\noindent
\textbf{Лемма~1.}
\textit{1. Интеграл}~(\ref{Mellin_transform}) \textit{существует для всех значений параметра~$s$ 
из некоторой полосы $\mathcal{D}_\xi\hm=\{s\,:\, \sigma_{\xi,1}\hm\le 
{\sf Re} (s)\hm\le\sigma_{\xi,2}\}$, содержащей мнимую ось и,~возможно, вырождающейся в эту ось}.

\textit{2. Преобразование Меллина взаимно однозначно определяет распределение случайной величины}.

\textit{3. Пусть $\xi$, $\eta$~--- независимые случайные величины. Тогда}
$$
\mathcal{M}_{\xi\eta}(s) = \mathcal{M}_\xi(s)\mathcal{M}_\eta(s), \enskip 
s\in\mathcal{D}_\xi\cap\mathcal{D}_\eta.
$$

\textit{4.
Для точек непрерывности плотности распределения случайной величины~$\xi$ справедлива следующая формула}:
$$f_{\xi}(x)
%= \frac{1}{2\pi}\int\limits_{-\infty}^{+\infty}x^{-it-1}\mathcal{M}_{\xi}(it) \, dt=$$ $$
=\fr{1}{2\pi i}\lim\limits_{T\to\infty}\int\limits_{c-iT}^{c+iT}x^{-s-1}
\mathcal{M}_{\xi}(s) \, ds,\enskip x>0\,,
$$
\textit{где линия $(c-i\infty, c+i\infty)$ лежит в области аналитичности $\mathcal{M}_{\xi}(s)$}.

\textit{5.
Для точек непрерывности $a, b, x > 0$ функции распределения случайной величины~$\xi$ справедливы формулы}:

\vspace*{-2pt}

\noindent
\begin{align*}
F_\xi(b) - F_\xi(a) &={}\\
&\hspace*{-9mm}{}= \lim_{U\to\infty}\fr{1}{2\pi}\int\limits_{-U}^{U} 
\fr{e^{-it\ln a}-e^{-it\ln b}}{it}\,\mathcal{M}_{\xi}(t)\,dt\,;
\\
F_\xi(x) &= \fr{1}{2}-\fr{1}{\pi}\int\limits_0^{\infty}{\sf Im}
\left\{\mathcal{M}_{\xi}(it)x^{-it}\right\}\fr{dt}{t}\,.
\end{align*}

\vspace*{-2pt}

Непосредственно из леммы~1 вытекают свойства преобразования Меллина 
обобщенного гам\-ма-рас\-пре\-де\-ле\-ния и~его мультипликативных смесей.

\smallskip

\noindent
\textbf{Лемма~2.} %\label{GG_properties}
\textit{1. Если случайная величина~$\lambda$ имеет обобщенное гам\-ма-рас\-пре\-де\-ле\-ние 
$\mathrm{GG}\,(v,q,\theta)$, то}
$$
\mathcal{M}_\lambda(s) = \fr{\theta^s}{\Gamma(q)}\,\Gamma\left(q+\fr{s}{v}\right), \enskip
q+\fr{{\sf Re} (s)}{v}>0\,.
$$

\textit{2. Если независимые случайные величины~$\lambda$ и $\mu$ имеют распределения 
$\mathrm{GG}\,(v, q, \theta)$ 
и~$\mathrm{GG}\,(u, p, \alpha)$ соответственно,  то}
\begin{multline*}
\mathcal{M}_{\rho}(s) = \fr{\left(\theta/ \alpha\right)^s}{\Gamma(q)\Gamma(p)}\,
\Gamma\left(q+\fr{s}{v}\right)\Gamma\left(p-\fr{s}{u}\right), \\
 q+ \fr{{\sf Re} (s)}{v}>0\,, \quad p-\fr{{\sf Re} (s)}{u}>0\,,
 \end{multline*}
\textit{причем $\mathcal{M}_{\rho}(s)$~--- аналитическая функция в~об\-ласти~$\mathcal{D}_\rho$}.

\smallskip

Непосредственно из леммы~2 следуют утверждения о моментных характеристиках и функции 
распределения отношения двух случайных величин, имеющих обобщенное гам\-ма-рас\-пре\-де\-ле\-ние.

\smallskip

\noindent
\textbf{Теорема~2.}
\textit{Пусть независимые случайные величины~$\lambda$ и $\mu$ имеют соответственно 
распределения $\mathrm{GG}\,(v,q,\theta)$ и $\mathrm{GG}\,(u,p,\alpha)$.
Тогда для случайной величины $\rho=\lambda/\mu$}
\begin{align*}
{\sf E} \rho^k &= \fr{\left(\theta/ \alpha\right)^{k}}{\Gamma(q)\Gamma(p)}\,
\Gamma\left(q+\fr{k}{v}\right)\Gamma\left(p-\fr{k}{u}\right), 
\\
&\hspace*{30mm}q+\fr{k}{v}>0\,, \quad p-\fr{k}{u}>0\,;
\\
{\sf E} e^{it\ln\rho}&= \fr{\left(\theta/ \alpha\right)^{it}}{\Gamma(q)\Gamma(p)}\,
\Gamma\left(q+\fr{it}{v}\right)\Gamma\left(p-\fr{it}{u}\right), \enskip  t\in\mathbb{R}\,.
\end{align*}


\smallskip

\noindent
\textbf{Теорема~3.}
\textit{Пусть независимые случайные величины~$\lambda$ и $\mu$ имеют соответственно распределения 
$\mathrm{GG}\,(v,q,\theta)$ и $\mathrm{GG}\,(u,p,\alpha)$. Тогда их отношение $\rho\hm=\lambda/\mu$ при $x\hm>0$ 
имеет функцию распределения}

\noindent
\begin{multline*}
F_{\rho}(x) = \fr{1}{2}-{}\\
{}-\fr{1}{\pi}\!\int\limits_0^{\infty}\!{\sf Im}\left\{
\fr{\left(\theta/ \alpha\right)^{it}}{\Gamma(q)\,\Gamma(p)}\Gamma\!\left(q+\fr{it}{v}\right)\!
\Gamma\!\left(p-\fr{it}{u}\right)\!x^{-it}\!\right\}\!\fr{dt}{t}\,.\hspace*{-3pt}
\end{multline*}


\smallskip

Найдем представление плотности случайной величины~$\rho$. Для этого рассмотрим $H$-функ\-цию 
Фокса~\cite{Prudnikov3}
\begin{multline*}
H_{kl}^{mn}
\left[z \left\vert \
\begin{array}{c}
(a_1,A_1),\ldots,(a_k,A_k)\\
(b_1,B_1),\ldots,(b_l,B_l)
\end{array}\right.
\right]=\fr{1}{2\pi i}\times{}\\
{}\times \int\limits_L  \! \fr{\prod\nolimits_{j=1}^m
\Gamma(b_j+B_js)\prod\nolimits_{j=1}^n\Gamma(a_j-A_js)}
{\prod\nolimits_{j=n+1}^k\!\!\!\Gamma(a_j+A_js)\prod\nolimits_{j=m+1}^l
\!\!\!\Gamma(b_j-B_js)}z^{-s}\,ds,\hspace*{-5.89261pt}
\end{multline*}
где $0\le m\le l$, $0\hm\le n\hm\le k$, $A_{j_1},B_{j_2}\hm>0$, $j_1\hm=1,\ldots,k$, 
$j_2\hm=1,\ldots,l$. Везде далее в качестве контура~$L$ рассматривается область интегрирования 
$\left(c-i\infty, c+i\infty\right)\hm\subset\mathcal{D}_\rho$.

\smallskip

\noindent
\textbf{Теорема~4.}
\textit{Пусть независимые случайные величины~$\lambda$ и $\mu$ имеют 
соответственно распределения $\mathrm{GG}\,(v,q,\theta)$ и $\mathrm{GG}\,(u,p,\alpha)$. Тогда их 
отношение $\rho\hm=\lambda/\mu$ при $x\hm>0$ имеет плотность}

\noindent
\begin{multline*}
f_{\rho}(x) = \fr{x^{-1}}{\Gamma(q)\Gamma(p)}\times{}\\
{}\times
\begin{cases}
\displaystyle H_{11}^{11}
\left[\fr{\alpha x}{\theta} \left\vert
\begin{array}{c}
\left(p,\fr{1}{u}\right)\\[7pt]
\left(q,\fr{1}{v}\right)
\end{array}\right. \right], &u>0, v>0;\\[16pt]
\displaystyle H_{02}^{20}
\left[\fr{\alpha x}{\theta} \left\vert 
\begin{array}{c}
\mbox{---}\\
\left(q,\fr{1}{v}\right),\left(p,-\fr{1}{u}\right)
\end{array}\right.
\right]   , &u<0, v>0;\\[16pt]
\displaystyle H_{20}^{02}
\left[\fr{\alpha x}{\theta} \left\vert 
\begin{array}{c}
\left(q,-\fr{1}{v}\right),\left(p,\fr{1}{u}\right)\\
\mbox{---}
\end{array}\right.
\right]   , &u>0, v<0;\\[16pt]
\displaystyle H_{11}^{11}
\left[\fr{\alpha x}{\theta} \left\vert 
\begin{array}{c}
\left(q,-\fr{1}{v}\right)\\[7pt]
\left(p,-\fr{1}{u}\right)
\end{array}\right.
\right]   , &u<0, v<0.
\end{cases}\hspace*{-0.1052pt}
\end{multline*}



\noindent
Д\,о\,к\,а\,з\,а\,т\,е\,л\,ь\,с\,т\,в\,о\,.\ \ Согласно п.~4 леммы~1 и п.~2 леммы~2

\vspace*{-3pt}

\noindent
\begin{multline*}
f_{\rho}(x) = \fr{1}{2\pi i}\lim\limits_{T\to\infty}\int\limits_{c-iT}^{c+iT}x^{-s-1}
\mathcal{M}_{\rho}(s) \, ds={}\\
{}= \fr{x^{-1}}{2\pi i\Gamma(q)\Gamma(p)}\times{}\\
{}\times \lim_{T\to\infty}\int\limits_{c-iT}^{c+iT}
\Gamma\left(q+\fr{s}{v}\right)\Gamma\left(p-\fr{s}{u}\right)
\left(\fr{\alpha x}{\theta}\right)^{-s}\, ds,
\end{multline*}
откуда следует утверждение теоремы.

\columnbreak

\smallskip

Теоремы~1 и~4 дают возможность представить гам\-ма-экспоненциальную функцию через $H$-функ\-цию Фокса.

\smallskip

\noindent
\textbf{Следствие 1.}
Для всех $0\hm< \alpha\hm<1$ и $\beta\hm>0$ имеет место равенство
$$ 
{\sf Ge}_{\alpha,\, \beta}(-x) = H_{11}^{11}
\left[x\, \Big | \
\begin{array}{c}
(\beta,\alpha)\\
(0,1)
\end{array}\right],\enskip x>0\,. 
$$


Вид плотности случайной величины~$\rho$ допускает альтернативное представление в~случае,
 когда $u\hm=-v$. Рассмотрим  функцию Макдональда (модифицированную функцию Бесселя третьего 
 рода)~\cite{GR1971}, которую можно представить в виде:
$$
K_{\nu}(z) = \int\limits_0^{\infty}e^{-z \mathrm{ch}\,t}\mathrm{ch}\left(\nu t\right)\, dt\,, \enskip
 \left| \arg(z)\right|<\fr{\pi}{2}\,,
 $$
и функцию Мейера~\cite{Prudnikov3}
\begin{multline*}
G_{kl}^{mn}
\left[z\, \Big | \
\begin{array}{c}
a_1,\ldots,a_k\\
b_1,\ldots,b_l
\end{array}\right]=\fr{1}{2\pi i}\times{}\\
{}\times \int\limits_L \fr{\prod\nolimits_{j=1}^m\Gamma(b_j+s)\prod\nolimits_{j=1}^n\Gamma(a_j-s)}
{\prod\nolimits_{j=n+1}^k\Gamma(a_j+s)\prod\nolimits_{j=m+1}^l\Gamma(b_j-s)}\,z^{-s}\,ds,
\end{multline*}
где $0\le m\hm\le l$, $0\hm\le n\hm\le k$.

\medskip

\noindent
\textbf{Следствие 2.} 
Пусть независимые случайные величины~$\lambda$ и $\mu$ имеют соответственно распределения $\mathrm{GG}\,(v,q,\theta)$ 
и~$\mathrm{GG}\,(-v,p,\alpha)$. Тогда их отношение $\rho\hm=\lambda/\mu$ при $x\hm>0$ имеет плот\-ность:
$$
f_{\rho}(x) =
\fr{2|v| x^{({vp+vq})/{2}-1}}{\left(\theta/\alpha\right)^{({vp+vq})/{2}}\Gamma(p)\Gamma(q)}
   K_{p-q}\!\left(\!2\sqrt{\left(\fr{\alpha x}{\theta}\right)^v}\right)\!.
$$


\noindent
Д\,о\,к\,а\,з\,а\,т\,е\,л\,ь\,с\,т\,в\,о\,.\ \
 Утверждение следствия~2 вытекает из соотношений~\cite{Prudnikov3}:
$$
H_{k l}^{m n}
    \left[z  \left\vert 
    \begin{array}{c}
    (a_1,1),\ldots,(a_k,1)\\
    (b_1,1),\ldots,(b_l,1)
    \end{array}\right.\right] =
    \displaystyle G_{k l}^{m n}
    \left[z \left\vert 
    \begin{array}{c}
    a_1,\ldots,a_k\\
    b_1,\ldots,b_l
    \end{array}\right.\right]\!;
    $$
    
    \vspace*{-12pt}

\noindent
\begin{multline*}
 \!\!\! \hspace*{-5pt}G_{k l}^{m n}\!
    \left[z \left\vert
    \begin{array}{c}
    a_1-\alpha,\dots,a_n-\alpha,a_{n+1}+\alpha,\dots,a_k+\alpha\\
    b_1+\alpha,\dots,b_m+\alpha, b_{m+1}-\alpha,\dots,b_l-\alpha
    \end{array}\right.\right] ={}\hspace*{-3pt}
\\
{}=z^\alpha G_{k l}^{m n}
    \left[z \left\vert 
    \begin{array}{c}
    a_1,\ldots,a_k\\
    b_1,\ldots,b_l
    \end{array}\right.
    \right];
    \end{multline*}
    
    \vspace*{-12pt}

\noindent
\begin{align*}
K_\nu(2\sqrt{x}) &=
    \fr{1}{2}\, G_{0 2}^{2 0}
    \left[x \,\left\vert
    \begin{array}{c}
\mbox{---}\\
    \fr{\nu}{2}, -\fr{\nu}{2}
    \end{array}\right.
    \right];
    \\
    K_\nu\left(\fr{2}{\sqrt{x}}\right)& =
    \fr{1}{2}\,G_{2 0}^{0 2}
    \left[x \,\left\vert 
    \begin{array}{c}
    \fr{\nu}{2}, -\fr{\nu}{2}\\
    \mbox{---}
    \end{array}\right.\right]
    \end{align*}
и свойства~\cite{GR1971} $K_\nu(x) \hm= K_{-\nu}(x)$.

Также утверждение следствия~2 может быть получено напрямую, 
как показано в~\cite{Carolynne2013} для $v\hm>0$.

\smallskip

Для некоторых значений параметров формы априорных распределений факторов~$\lambda$ и $\mu$ 
может быть получен альтернативный вид плотности индекса баланса~$\rho$ при помощи гипергеометрической функции.

Через

\noindent
$$
(\alpha)_i=\alpha (\alpha+1)\cdots(\alpha+i-1),\  (\alpha)_0=1,\ 
\alpha\in\mathbb{R}\,,
$$
будем обозначать символ Похгаммера.
Рассмотрим обобщенную гипергеометрическую функцию~\cite{Prudnikov1}
$$
{}_mF_n\left[
\begin{array}{c}
a_1,\ldots, a_m;\\
b_1,\ldots, b_n
\end{array}\right]\left(
z
\right)
=\sum\limits_{k=0}^\infty \fr{(a_1)_k(a_2)_k\cdots (a_m)_k}{(b_1)_k(b_2)_k\cdots (b_n)_k}\,
\fr{z^k}{k!}.$$
Обозначим

\noindent
$$
\Delta(k,a)=\fr{a}{k},\fr{a+1}{k},\ldots,\fr{a+k-1}{k};
$$

\vspace*{-12pt}

\noindent
\begin{multline}
S_{\rho}\left(A,B,C;x\right)=\sum\limits_{n=0}^{A-1}\fr{x^n}{n!}\,
\Gamma\left(C - \fr{B}{A}n \right)\times{} \\
{}\times{}_1F_{A+B}\left[\begin{array}{c}
1; \Delta(A,1+n),\\
\Delta\left(B, 1 - C + \fr{B}{A}\,n \right)
\end{array}\right]\left(
\fr{(-1)^B}{A^A B^B}\, x^A
\right), \\ 
x\in\mathbb{R}\,,
\label{S_GGF}
\end{multline}
где $A$ и~$B$~--- взаимно простые натуральные числа, а~$C\hm>0$.

\smallskip

\noindent
\textbf{Теорема~5.}
\textit{Пусть независимые случайные величины~$\lambda$ и $\mu$ имеют соответственно распределения 
$\mathrm{GG}\,(v,q,\theta)$ и $\mathrm{GG}\,(u,p,\alpha)$, причем  $uv<0$ и для некоторых $\gamma>0$ и взаимно простых 
натуральных чисел $ad$ и $bc$ выполнены равенства $|v| \hm= \gamma a/b$ и $|u| \hm= \gamma c/d$. 
Тогда их отношение $\rho=\lambda/\mu$ при $x>0$ имеет плотность}:
\begin{multline*} 
f_\rho(x) =  \fr{x^{-1}}{\Gamma(p)\Gamma(q)} \times{}\\
{}\times
\Bigg[ |v|\left(\fr{\alpha x}{\theta}\right)^{vq}S_{\rho}\left(
bc,ad,p-\fr{ad}{bc}q;-\left( \fr{\alpha x}{\theta} \right)^v\right)  +{}\\
 {}+  |u|\left(\fr{\alpha x}{\theta}\right)^{-up}S_{\rho}\left(
ad,bc,q-\fr{bc}{ad}p;-\left( \fr{\alpha x}{\theta} \right)^{-u}\right)\Bigg], \\
pbc \neq qad.
\end{multline*}


\noindent
Д\,о\,к\,а\,з\,а\,т\,е\,л\,ь\,с\,т\,в\,о\,.\ \
 Без ограничения общности будем считать числа~$a$, $b$, $c$ и~$d$ положительными.

Заметим~\cite{Prudnikov1}, что для взаимно простых натуральных чисел~$m$ и $n$ и положительных 
чисел~$\delta$, $t$~и~$s$

\noindent
\begin{multline}
\int\limits_0^{\infty} x^{\delta-1}\exp\left\{-tx^{-m/n}-sx\right\}\,dx = {}\\
{}= s^{-\delta} \sum\limits_{l=0}^{n-1}\fr{(-1)^l}{l!}\left(ts^{m/n}\right)^l
\Gamma\left( \delta - \fr{m}{n}\,l \right)\times{}\\
{}\times
{}_1F_{m+n}\left[\begin{array}{c}
1; \Delta(n,1+l),\\
\Delta\left(m, 1-\delta+\fr{ml}{n}\right)
\end{array}
\right]\left(z \right) +{}\\
{}+ t^{\delta n/m}\fr{n}{m}
\sum\limits_{k=0}^{m-1} \fr{(-1)^k}{k!}\left(t^{n/m}s \right)^k \Gamma\left(-\fr{\delta+k}{m/n}\right)\times{}\\
{}\times
{}_1F_{m+n}\left[
\begin{array}{c}
1; \Delta(m,1+k),\\
\Delta\left(n, 1+\fr{\delta+k}{n/m}\right)
\end{array}\right]
\left( z \right),
\label{GGG}
\end{multline}
где
$$
z = (-1)^{m+n}\left( \fr{s}{m} \right)^m \left( \fr{t}{n} \right)^n.
$$

Для плотности~$\rho$ имеем:
\begin{multline*}
f_\rho(x) = \fr{|uv| x^{vq-1} }{ \theta^{vq}\alpha^{up} \Gamma(p)\Gamma(q)} \times{}\\
{}\times
\int\limits_0^\infty y^{-up-vq-1} \exp\left\{-\left(\fr{x}{\theta} \right)^{v}y^{-v}-
(\alpha y)^{-u}\right\} dy.
\end{multline*}
В случае $pbc \hm> qad$ получаем:
\begin{multline*}
f_\rho(x) = \fr{|v| x^{vq-1} }{\theta^{vq}\alpha^{up}  \Gamma(p)\Gamma(q)}\times{}\\
\!{}\times
\int\limits_0^\infty z^{p+qv/u-1} \exp\left\{-\left( \fr{x}{\theta} \right)^{v}z^{-ad/(bc)}-
\alpha^{-u}z\right\}dz.
\end{multline*}
Используя формулу~(\ref{GGG}) и обозначение~(\ref{S_GGF}), получим утверждение теоремы.

Аналогично рассматривается случай $ qad \hm> pbc$.
Теорема доказана.

\smallskip

Введем обозначение
\begin{multline}
T_\rho(A, B, C; x) = \sum\limits_{k=0}^{A-1}\fr{x^k}{k!}\,
\Gamma\left(C + k\fr{B}{A} \right)\times{}\\
{}\times {}_{B+1}F_{A}\left[
    \begin{array}{c}
    1, \Delta\left(B, C + \fr{kB}{A}\right); \\
    \Delta\left(A, 1+k\right)
    \end{array}\right]\left(
    \fr{B^B}{A^A}x^A
    \right), \\
      x\in\mathbb{R}\,,
    \label{T_GGF}
    \end{multline}
где $A$ и~$B$~--- взаимно простые натуральные числа, а~$C\hm>0$.

\smallskip

\noindent
\textbf{Теорема~6.}
\textit{Пусть независимые случайные величины~$\lambda$ и $\mu$ 
имеют соответственно распределения $\mathrm{GG}\,(v,q,\theta)$ и $\mathrm{GG}\,(u,p,\alpha)$, 
причем  $uv\hm>0$ и для некоторых $\gamma\hm>0$ и взаимно простых натуральных чисел~$ad$ и $bc$ 
выполнены равенства $|v| \hm= \gamma a/b$ и $|u| \hm = \gamma c/d$. 
Тогда их отношение $\rho\hm=\lambda/\mu$ при $x\hm>0$ имеет плотность}:
\begin{multline*}
 f_\rho(x) ={}\\[6pt]
 {}=
 \begin{cases}
   \displaystyle \fr{|v|\alpha^{vq} x^{vq-1}}{\theta^{vq}\Gamma(p)\Gamma(q)}T_\rho
   \left(bc,ad,\fr{vq}{u}+p ;-\left(\!\fr{\alpha x} {\theta}\!\right)^{v}\right)\!, &\\[6pt]
   &\hspace*{-16mm}|v|<|u|;\\[6pt]
   \displaystyle\fr{|u| \alpha^{up}x^{-up-1}}{\theta^{up}\Gamma(p)\Gamma(q)}T_\rho
   \!\left(\!ad,bc,\fr{up}{v}+q ;-\!\left(\!\fr{\alpha x} {\theta}\!\right)^{\!-u}\!\right)\!, &\\[6pt]
   &\hspace*{-16mm}|v|>|u|.
 \end{cases}\hspace*{-13.30913pt}
 \end{multline*}


\noindent
Д\,о\,к\,а\,з\,а\,т\,е\,л\,ь\,с\,т\,в\,о\,.\ \
Заметим, что для
\begin{equation*}
U=\int\limits_0^\infty x^{\delta-1}e^{-tx^r-sx}\,dx\,,
%\label{int}
\end{equation*}
где $t,s>0$, $r=\hm m/n$, а $m$ и $n$~--- 
взаимно простые натуральные числа, справедливы соотношения~\cite{Prudnikov1}:
\begin{multline} 
U = s^{-\delta}\sum\limits_{k=0}^{n-1}\fr{\left(-ts^{-r}\right)^k}{k!}\Gamma(\delta + rk)\times{}\\
{}\times
    {}_{m+1}F_{n}\left[\begin{array}{c}
    1, \Delta(m,\delta + rk);\\
    \Delta\left(n, 1+k\right)
    \end{array}\right] %\left(
    (-1)^n z\,, %    \right), 
    \\
     0<r<1\,; 
    \label{GGF_pos_les}
    \end{multline}
    
    \vspace*{-12pt}
    
    \noindent
\begin{multline}
U = \fr{t^{-\delta/r}}{r}\sum\limits_{k=0}^{m-1}\fr{\left(-t^{-1/r}s\right)^k}{k!}
\,\Gamma\left( \fr{\delta+k}{r} \right)\times{}\\
{}\times
    {}_{n+1}F_{m}\left[
    \begin{array}{c}
    1, \Delta\left(n,\fr{\delta + k}{r}\right); \\
    \Delta \left( m, 1+k\right)
    \end{array}\right]%\left(
    (-1)^m z^{-1}\,,\\
%    \right), \\
     r>1\,, 
    \label{GGF_pos_gre}
    \end{multline}
где
$$ 
z = \left( \fr{m}{s} \right)^m \left( \fr{t}{n} \right)^n. 
$$

Аналогично доказательству теоремы~5, используя для плотности~$f_\rho(x)$ 
преобразование~(\ref{GGF_pos_les}) в случае $|u|\hm<|v|$, формулу~(\ref{GGF_pos_gre}) 
в~случае $|v|\hm<|u|$ и обозначение~(\ref{T_GGF}), получаем утверждение теоремы.

\smallskip

Теоремы~1 и~6 дают возможность выявить еще одно свойство гам\-ма-экс\-по\-нен\-ци\-аль\-ной функции.

\smallskip

\noindent
\textbf{Следствие 3.}
Для взаимно простых натуральных $0\hm<m\hm<n$, действительных $c\hm>0$ и $x\hm>0$ выполнено соотношение:
$$
{\sf Ge}_{m/n,\, c}(-x) = T_\rho\left(n, m, c; -x\right).
$$



\section{Заключение}


Данная работа завершает этап описания основных вероятностных характеристик индекса баланса факторов, 
име\-ющих обобщенные гам\-ма-рас\-пре\-де\-ле\-ния. 
Результаты формулируются в терминах специальных гам\-ма-функ\-ции, гам\-ма-экс\-по\-нен\-ци\-аль\-ной функции, 
функций Фокса и~Макдональда, а~также обобщенной гипергеометрической функции. 
Несмотря на кажущуюся громоздкость результатов, полученные соотношения на практике 
дают существенное увеличение скорости вычислений характеристик при конкретных 
значениях параметров по сравнению с численными методами.

Полученные результаты могут быть легко переформулированы для масштабных 
смесей обобщенных гам\-ма-рас\-пре\-де\-ле\-ний и могут найти широкое применение в~естественно-на\-уч\-ных моделях, 
использующих для описания процессов и явлений распределения с положительным неограниченным носителем.

{\small\frenchspacing
{%\baselineskip=10.8pt
%\addcontentsline{toc}{section}{References}
\begin{thebibliography}{99}

\bibitem{Ku2018}
\Au{Кудрявцев~А.\,А.}
Байесовские модели баланса~// Информатика и её применения, 2018. Т.~12. Вып.~3. С.~18--27.

\bibitem{Ku2019_1}
\Au{Кудрявцев~А.\,А.}
Априорное обобщенное гам\-ма-рас\-пре\-де\-ле\-ние в байесовских моделях баланса~// 
Информатика и её применения, 2019. Т.~13. Вып.~3. С.~27--33.

\bibitem{KuTi2017}
\Au{Кудрявцев~А.\,А., Титова~А.\,И.}
Гам\-ма-экс\-по\-нен\-ци\-аль\-ная функция в байесовских моделях массового обслуживания~// 
Информатика и её применения, 2017. Т.~11. Вып.~4. С.~104--108.

\bibitem{KrMe1946}
\Au{Крицкий~С.\,Н., Менкель~М.\,Ф.}
О~приемах исследования случайных колебаний речного стока~// 
Труды НИУ ГУГМС. Сер.~IV, 1946. Вып.~29. С.~3--32.
%. Труды Гос. гидрологического ин-та, вып. 29, 1946.

\bibitem{KrMe1948}
\Au{Крицкий~С.\,Н., Менкель~М.\,Ф.}
Выбор кривых распределения вероятностей для расчетов речного стока~// 
Известия АН СССР. Отд. техн. наук, 1948. №\,6. С.~15--21.

\bibitem{Zolotarev1957}
\Au{Золотарев~В.\,М.}
Преобразования Мел\-ли\-на--Стильть\-еса в теории вероятностей~// 
Теория вероятностей и~её применения, 1957. Т.~2. Вып.~4. C.~444--469.

\bibitem{Galambos2004}
\Au{Galambos~J., Simonelli~I.}
Products of random variables: Applications to problems of physics and to arithmetical functions.~--- 
New York, NY, USA: Marcel Dekker, Inc., 2004. 315~p.

\bibitem{Prudnikov3}
\Au{Прудников~А.\,П., Брычков~Ю.\,А., Маричев~О.\,И.}
Интегралы и ряды: в 3~т. Т.~3: Специальные функции. Дополнительные главы.~--- 
2-е изд., испр.~--- М.: Физматлит, 2003. 688~c.

\bibitem{GR1971}
\Au{Градштейн~И.\,С., Рыжик~И.\,М.}
Таблицы интегралов, сумм, рядов и произведений.~--- М.: Наука, 1971. 1108~с.

\bibitem{Carolynne2013}
\Au{Ayienda~K.\,C.}
Gamma and related distributions.~--- Nairobi, Kenya: 
School of Mathematics, University of Nairobi, 2013. PhD Thesis. 148~p. 
%A~thesis submitted for the degree of Master of Sciences in Statistics.

\bibitem{Prudnikov1}
\Au{Прудников~А.\,П., Брычков~Ю.\,А., Маричев~О.\,И.}
Интегралы и ряды: в 3~т. Т.~1: Элементарные функции.~--- 2-е изд., испр.~--- М.: Физматлит, 2002. 632~c.
\end{thebibliography}

}
}

\end{multicols}

\vspace*{-3pt}

\hfill{\small\textit{Поступила в~редакцию 18.06.2020}}

\vspace*{8pt}

%\pagebreak

%\newpage

%\vspace*{-28pt}

\hrule

\vspace*{2pt}

\hrule

%\vspace*{-2pt}

\def\tit{PROBABILISTIC CHARACTERISTICS OF~BALANCE INDEX OF~FACTORS WITH~GENERALIZED GAMMA DISTRIBUTION}

\def\titkol{Probabilistic characteristics of~balance index of~factors with~generalized gamma distribution}

\def\aut{E.\,N.~Arutyunov$^1$, A.\,A.~Kudryavtsev$^2$, and~Iu.\,N.~Nedolivko$^2$}

\def\autkol{E.\,N.~Arutyunov, A.\,A.~Kudryavtsev, and~Iu.\,N.~Nedolivko}

\titel{\tit}{\aut}{\autkol}{\titkol}

\vspace*{-11pt}


\noindent
$^1$Institute of Informatics Problems, Federal Research Center ``Computer Science and Control''
of the Russian\linebreak
$\hphantom{^1}$Academy of Sciences, 44-2~Vavilov Str., Moscow 119333, Russian Federation

\noindent
$^2$Faculty of Computational Mathematics and Cybernetics,
 M.\,V.~Lomonosov Moscow State University, 1-52~Lenin-\linebreak
 $\hphantom{^1}$skie Gory, GSP-1, Moscow 119991, Russian Federation
 
 

\def\leftfootline{\small{\textbf{\thepage}
\hfill INFORMATIKA I EE PRIMENENIYA~--- INFORMATICS AND
APPLICATIONS\ \ \ 2021\ \ \ volume~15\ \ \ issue\ 1}
}%
\def\rightfootline{\small{INFORMATIKA I EE PRIMENENIYA~---
INFORMATICS AND APPLICATIONS\ \ \ 2021\ \ \ volume~15\ \ \ issue\ 1
\hfill \textbf{\thepage}}}

\vspace*{3pt}




\Abste{The main probabilistic characteristics of balance index in Bayesian formulation, 
assuming that negative and positive factors have \textit{a~priori} generalized gamma distribution, are given. 
The formulation of this problem is equivalent to a~study of generalized gamma laws scale 
mixture characteristics. Special attention is paid to the case in which the factors
 distributions have shape parameters of different signs. Moment characteristics and 
 different presentation of density in terms of gamma-exponential function, H-function, 
 Macdonald function, and generalized hypergeometric function are given. The analysis 
 method is based on Mellin transform and its inverse transform. New properties 
 of gamma-exponential function are given. The obtained results can be widely applied 
 within the natural science models that use distributions with positive unlimited 
 support to describe processes and phenomena.}

\KWE{Bayesian approach; generalized gamma distribution; gamma-exponential function; 
balance models; Mellin transform; H-function; hypergeometric function}



\DOI{10.14357/19922264210109}

\vspace*{-12pt}

\Ack

\vspace*{-2pt}

\noindent
The work was partly supported by the Russian Foundation for Basic Research (project 20-07-00655).

\vspace*{8pt}

  \begin{multicols}{2}

\renewcommand{\bibname}{\protect\rmfamily References}
%\renewcommand{\bibname}{\large\protect\rm References}

{\small\frenchspacing
 {%\baselineskip=10.8pt
 \addcontentsline{toc}{section}{References}
 \begin{thebibliography}{99}
\bibitem{1-ku-1}
\Aue{Kudryavtsev, A.\,A.} 
2018. Bayesovskie modeli balansa [Bayesian balance models]. 
\textit{Informatika i~ee Primeneniya~--- Inform. Appl.} 12(3):18--27.
\bibitem{2-ku-1}
\Aue{Kudryavtsev, A.\,A.} 2019. Apriornoe obobshchennoe gamma-raspredelenie 
v~bayesovskikh modelyakh balansa [\textit{A~priori} generalized gamma distribution in 
Bayesian balance models]. \textit{Informatika i~ee Primeneniya~--- Inform. Appl.} 13(3):27--33.
\bibitem{3-ku-1}
\Aue{Kudryavtsev, A.\,A., and A.\,I. Titova.}
 2017. Gamma-eksponentsial'naya funktsiya v~bayesovskikh modelyakh massovogo obsluzhivaniya 
 [Gamma-exponential function in Bayesian queuing models]. 
 \textit{Informatika i~ee Primeneniya~--- Inform. Appl.} 11(4):104--108.
\bibitem{4-ku-1}
\Aue{Kritsky, S.\,N., and M.\,F. Menkel.} 1946. 
O~priemakh issledovaniya sluchaynykh kolebaniy rechnogo stoka 
[Methods of investigation of random fluctuations of river flow].
\textit{Trudy NIU GUGMS. Ser.~IV} [Proceedings of GUGMS Research Institutions. Ser.~IV] 29:3--32.
\bibitem{5-ku-1}
\Aue{Kritsky, S.\,N., and M.\,F.~Menkel.} 1948. 
Vybor krivykh raspredeleniya veroyatnostey dlya raschetov rechnogo stoka 
[Selection of probability distribution curves for river flow calculations]. 
\textit{Izvestiya AN SSSR. Otd. tekhn. nauk}
 [Herald of the Russian Academy of Sciences. Technical Sciences] 6:15--21.
\bibitem{6-ku-1}
\Aue{Zolotarev, V.\,M.} 1957. 
Mellin--Stieltjes transformations in probability theory. \textit{Theor. Probab. Appl.} 2(4):433--460.
\bibitem{7-ku-1}
\Aue{Galambos, J., and I.~Simonelli.} 2004. 
\textit{Products of random variables: Applications to problems of physics and to arithmetical functions}. 
New York, NY: Marcel Dekker, Inc. 315~p.
\bibitem{8-ku-1}
\Aue{Prudnikov, A.\,P., Yu.\,A.~Brychkov, and O.\,I.~Marichev.}
 2003. \textit{Integraly i~ryady}: v~3~t. T.~3. 
 \textit{Spetsial'nye funktsii. Dopolnitel'nye glavy} [Integrals and series. In 3~vols. Vol.~3. 
 Special functions. Additional chapters]. 2nd ed. Moscow: Fizmatlit. 688~ p.
\bibitem{9-ku-1}
\Aue{Gradshteyn, I.\,S., and I.\,M.~Ryzhik.}
 1971. \textit{Tablitsy integralov, summ, ryadov i~proizvedeniy} 
 [Tables of integrals, sums, series, and products]. Moscow: Nauka. 1108~p.
\bibitem{10-ku-1}
\Aue{Ayienda, K.\,C.} 2013. Gamma and related distributions. 
%A~thesis submitted for the degree of Master of Sciences in statistics. 
Nairobi, Kenya: School of Mathematics, University of Nairobi. PhD Thesis. 148~p.
\bibitem{11-ku-1}
\Aue{Prudnikov, A.\,P., Yu.\,A.~Brychkov, and O.\,I.~Marichev.}
 2002. \textit{Integraly i~ryady}: v~3~t. T.~1. \textit{Elementarnye funktsii} 
 [Integrals and series: In 3~vols. Vol.~3. Elementary functions]. 2nd ed. Moscow: Fizmatlit. 632~p.
 \end{thebibliography}

 }
 }

\end{multicols}

\vspace*{-3pt}

  \hfill{\small\textit{Received June~18, 2020}}


%\pagebreak

%\vspace*{-8pt}

\Contr

\noindent
\textbf{Arutyunov Evgeny N.} (b.\ 1952)~--- Candidate of Science (PhD) 
in physics and mathematics, senior scientist, Institute of Informatics Problems, Federal Research Center 
``Computer Science and Control'' of the Russian Academy of Sciences, 44-2~Vavilov Str., 
Moscow 119333, Russian Federation; \mbox{enapoleon@mail.ru}

\vspace*{3pt}

\noindent
\textbf{Kudryavtsev Alexey A.} (b.\ 1978)~--- 
Candidate of Science (PhD) in physics and mathematics, associate professor, Department of Mathematical 
Statistics, Faculty of Computational Mathematics and Cybernetics, 
M.\,V.~Lomonosov Moscow State University, 1-52~Leninskie Gory, GSP-1, Moscow 119991, Russian Federation; 
\mbox{nubigena@mail.ru}

\vspace*{3pt}

\noindent
\textbf{Nedolivko Iuliia N.} (b.\ 1998)~--- 
student, Faculty of Computational Mathematics and Cybernetics, 
M.\,V.~Lomonosov Moscow State University, 1-52~Leninskie Gory, GSP-1, Moscow 119991, Russian Federation; 
\mbox{mouse98@mail.ru}
\label{end\stat}

\renewcommand{\bibname}{\protect\rm Литература}