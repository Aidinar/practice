\def\stat{agalarov}

\def\tit{ОПТИМАЛЬНОЕ ПОРОГОВОЕ УПРАВЛЕНИЕ ДОСТУПОМ В~СИСТЕМЕ $M/M/s$
С~НЕОДНОРОДНЫМИ ПРИБОРАМИ И~ОБЩИМ НАКОПИТЕЛЕМ$^*$}

\def\titkol{Оптимальное пороговое управление доступом в~системе $M/M/s$
с~неоднородными приборами и~общим накопителем}

\def\aut{Я.\,М.~Агаларов$^1$}

\def\autkol{Я.\,М.~Агаларов}

\titel{\tit}{\aut}{\autkol}{\titkol}

\index{Агаларов Я.\,М.}
\index{Agalarov Ya.\,M.}

{\renewcommand{\thefootnote}{\fnsymbol{footnote}} \footnotetext[1]
{Работа выполнена при частичной финансовой поддержке РФФИ
(проекты 18-07-00692, 19-07-00739 и~20-07-00804).}}

\renewcommand{\thefootnote}{\arabic{footnote}}
\footnotetext[1]{Институт проблем информатики Федерального
исследовательского центра <<Информатика и~управ\-ле\-ние>> Российской
академии наук, \mbox{agglar@yandex.ru}}


\vspace*{-10pt}

\Abst{Рассматривается система $M/M/s$ с~неоднородными приборами
и~общим накопителем с~возможностью управления длиной очереди с~целью
максимизации среднего предельного дохода. Функция дохода включает плату
за успешно обслуженную заявку, штраф за каждую отклоненную заявку,
штрафы за единицу времени простоя каждого прибора, штраф за единицу
времени ожидания заявки (или за превышение допустимого времени ожидания
заявки), затраты, связанные с~техническим обслуживанием мест в~накопителе.
Ставится задача максимизации предельного дохода на множестве простых
пороговых стратегий управления длиной очереди. Доказано свойство
выпуклости функции дохода и~получены условия существования конечного
оптимального порога длины очереди.}

\KW{система массового обслуживания; оптимизация; пороговая стратегия;
длина очереди}

\DOI{10.14357/19922264210108}


\vskip 10pt plus 9pt minus 6pt

\thispagestyle{headings}

\begin{multicols}{2}

\label{st\stat}

\section{Введение}

 Задача ограничения доступа к~ресурсам (управ\-ле\-ния потоками)
и~распределения допущенной нагрузки по ресурсам с~целью оптимальной их
загрузки~--- одна из наиболее важных подзадач общей задачи управления
потоковыми системами, в~частности информационными системами~[1]. Для
разработки и~исследования процедур управления потоками и~распределением
нагрузки в~таких системах успешно используется метод математического
моделирования, при этом в~качестве математических моделей потоковых
систем в~основном используются системы массового обслуживания (СМО)
различного типа.

Для СМО с~неоднородными приборами под\-задача оптимальной загрузки
ресурсов рассматривалась во многих работах, и~в~некоторых из них \mbox{получены}
точные аналитические результаты, позволяющие найти оптимальный план
диспетчеризации, минимизирующий среднее чис\-ло заявок в~системе (см.,
например, работы~[2--6]). Опишем пример решения указанной подзадачи,
приведенный в~работе~\cite{2-ag}. Рассматривается СМО типа $M/M/K/N$
($0\hm\leq N\hm\leq \infty$) с~$K$~неоднородными приборами, $N$~местами
для ожидания, пуассоновским входящим потоком интенсивности~$\lambda$
и~экспоненциально распределенными временами обслуживания
с~параметрами~$\mu_i$, $i\hm=1,\ldots , K$. Ставится задача оптимального
выбора свободного прибора, на который в~момент поступления будет
направлена вновь поступившая заявка, если очередь пуста, или в~момент
освобождения прибора будет направлена первая заявка в~очереди, если она не
пуста. Если в~момент поступления очередь не пуста, то новая заявка
становится в~очередь, если есть свободные места ожидания, а~иначе теряется.
Пересылка заявок с~одного прибора на другой не допускается. Для данной
СМО получен следующий оптимальный план диспетчеризации,
минимизирующий среднее чис\-ло заявок в~системе в~стационарном режиме ее
работы: если необходимо направить заявку на прибор, то нужно выбрать
свободный прибор с~наибольшей производительностью. 

Аналогичное решение
получено в~работе~\cite{6-ag} для СМО с~двумя гетерогенными приборами,
которые различаются по скорости обслуживания и~характеристикам
надежности (например, более медленный прибор абсолютно надежен, в~то
время как более быстрый прибор подвержен случайным сбоям). С~по\-мощью
марковской модели процесса принятия решений доказано, что оптимально
всегда направлять клиентов на более быстрый прибор, когда он доступен,
независимо от частоты его отказов и~скорости ремонта, если система
стабильна. Для более медленного прибора существует оптимальная пороговая
политика, которая зависит от длины очереди и~состояния более быстрого
прибора. 

Ниже в~рассматриваемой СМО используется такой же план
распределения заявок по приборам, как и~в~представленных выше работах, за
исключением того, что при отсутствии очереди разрешается пересылка заявки
с~прибора меньшей производительности на освободившийся прибор большей
производительности.

Одним из наиболее часто применяемых на практике способов ограничения
заявкам доступа в~сис\-те\-му стала так называемая пороговая стратегия, суть
которой заключается в~том, что решение о~допуске (план, политика допуска)
заявки в~систему меняется в~момент достижения наблюдаемой величиной
определенного порогового значения. 

Из множества работ, посвященных
пороговым стратегиям, отметим работы~[7--11], в~которых\linebreak исследованы
задачи, близкие по постановке к~рас\-смат\-ри\-ва\-емой в~данной статье. 
В~\cite{7-ag} рассматривается $M/M/1$ с~управляемой нагрузкой, в~которой можно
отклонять заявки по мере прибытия и~в~каж\-дый момент принятия решения
корректировать скорость обслуживания. Цель состоит в~минимизации
предельных (долгосрочных) средних затрат, которые включают затраты на
эксплуатацию\linebreak мест хранения, затраты за единицу времени обслуживания
заявки при заданной скорости обслуживания, штраф за отклонение задания.
Разработан\linebreak алгоритм вы\-чис\-ле\-ния оптимальной политики, который позволяет на
каждом шаге вычислять точ\-ность приближения политики. В~работе~\cite{8-ag}
в~качестве оптимальной политики ограничения \mbox{длины} очереди,
минимизирующей затраты СМО типа GI/GI/1 с~нетерпеливыми заявками
в~условиях интенсивного трафика, предлагается использовать решение
диффузионной аппроксимирующей задачи. Заявка после поступления
в~очередь через экспоненциально распределенный интервал времени покидает
систему, если она за это время не обслужилась. Система при поступлении
заявки в~очередь получает платеж, при отклонении заявки и~уходе
необслуженной заявки из очереди платит соответствующие штрафы. В~\cite{9-
ag} сформулирована оптимизационная задача допуска для СМО типа
$M/M/K/N$ с~однородными приборами, когда есть затраты, связанные
с~отклонением заявок, простоем приборов и~отказом допущенных в~систему
заявок от обслуживания. Предполагается, что процесс прибытия является
пуассоновским, время обслуживания экспоненциально и~заявка отказывается
от обслуживания после экспоненциального времени пребывания в~очереди.
Заданы стоимость одной отклоненной заявки, стоимость отказа одной заявки,
стоимость единицы времени простоя одного прибора. Построен
соответствующий марковский процесс принятия решений и~показано, что
оптимальная политика имеет пороговую форму. Решена аппроксимирующая
задача управ\-ле\-ния диффузией. Задача минимизации предельных средних затрат
при интенсивном трафике для по\-сле\-до\-ва\-тель\-ности СМО типа $M/M/1$
с~возможностью управ\-ле\-ния длиной очереди и~ско\-ростью об\-слу\-жи\-ва\-ния
рассмотрена в~работе~\cite{10-ag}. Функция затрат включает штраф за каж\-дую
отвергнутую заявку, затраты, связанные с~корректировкой ско\-рости
обслуживания, и~штраф за каж\-дую отказавшуюся заявку. Для построения
оптимального управления СМО в~условиях интенсивного трафика (выбора
оптимального размера буфера и~оптимальной скорости обслуживания)
предлагается использовать оптимальную стратегию для аппроксимирующей
задачи управления диффузией. 

В~работах~\cite{11-ag, 12-ag} для СМО типа
$M/M/K$ с~однородными приборами доказана унимодальность стоимостной
целевой функции от порогового значения числа заявок в~системе:
в~работе~\cite{11-ag} в~случае, когда целевая функция учитывает плату за
одну обслуженную заявку, штраф за отклонение заявки и~стоимость единицы
среднего времени ожидания заявки, а~в~работе~\cite{12-ag} в~случае, когда
целевая функция учитывает плату за одну своевременно обслуженную заявку,
штраф за отклонение заявки и~штраф за несвоевременно обслуженную заявку.

Ниже в~работе обобщаются результаты работ~\cite{11-ag, 12-ag} для СМО
типа $M/M/K$ на случай неоднородных приборов и~наличия штрафа за
единицу среднего времени простоя каждого прибора.

\section{Формулирование задачи}

Рассматривается СМО с~$s$~неоднородными приборами обслуживания, на
которую поступает пуассоновский поток заявок с~интенсивностью~$\lambda$.
Предполагается, что приборы имеют условные номера $1,\ldots , N$
и~закон распределения времени обслуживания на $i$-м приборе является
показательным с~параметром~$\mu_i$, $\mu_1\geq \cdots \geq \mu_s$. Заявки
допускаются в~систему согласно простой пороговой стратегии следующим
образом: поступившая заявка допускается в~систему (занимает любое
свободное место в~накопителе), если в~момент ее поступления число заявок
в~сис\-те\-ме меньше порогового значения~$k$, $k\hm\geq s$, в~противном случае
она отклоняется и~повторно не возвращается. При распределении заявок по
приборам предпочтение отдается наиболее производительным: если в~момент
поступления в~сис\-те\-ме есть свободный прибор, то новая заявка занимает
свободный прибор с~наименьшим номером и~начинает сразу обслуживаться,
иначе становится в~конец очереди на обслуживание. Заявка после окончания
обслуживания сразу покидает систему, освободив прибор и~место
в~накопителе. Если в~очереди нет заявок и~производительность
освободившегося прибора строго больше, чем интенсивность обслуживания
одного из занятых приборов, то заявка с~прибора наименьшей среди занятых
приборов производительности снимается и~пересылается на освободившийся
прибор. Параллельное обслуживание заявки на нескольких приборах не
допускается.

Отметим, что процесс обслуживания в~рас\-смат\-ри\-ва\-емой СМО описывается
марковским процессом гибели и~размножения с~непрерывным временем
с~множеством состояний $\{i:\ i\hm=0,\ldots , k\}$. Обозначим через~$\lambda$
интенсивность входного потока, $\tilde{\mu}_i\hm= \mu_1+\cdots + \mu_i$,
$\beta_{ij}$~--- интенсивность перехода процесса из состояния~$i$
в~состояние~$j$, $i\hm=0,\ldots , k$, $j\hm=0,\ldots , k$. Для $\beta_{ii+1}$
и~$\beta_{ii-1}$ в~данном случае имеют место равенства:
\begin{align*}
\beta_{ii+1} &=\lambda\,, \enskip i=0,\ldots , k-1\,;\\
\beta_{ii-1} &=\begin{cases}
\tilde{\mu}_s\,, & i=s+1,\ldots , k\,;\\
\tilde{\mu}_i\,, & i=1,\ldots , s\,.
\end{cases}
\end{align*}

Согласно формулам (см.~\cite{14-ag}) для стационарных вероятностей
состояний процесса гибели и~размножения~$\pi_j^k$ в~случае
рассматриваемой СМО справедливы формулы:
\begin{equation}
\pi_j^k=\begin{cases}
\displaystyle \pi_0^k\fr{\lambda^j}{\prod\nolimits_{i=1}^j \tilde{\mu}_i}\,, &1\leq
j\leq s\,;\\[12pt]
\pi_0^k \displaystyle \fr{\lambda^s}{\prod\nolimits^s_{i=1}\tilde{\mu_i}}\left( \fr{\lambda}{\tilde{\mu}_s}\right)^{j-s}\,, &
s<j\leq k\,,
\end{cases}
\label{e1-ag}
\end{equation}
где
$$
\pi_0^k=\displaystyle \left[
1+\sum\limits^s_{j=1}\fr{\lambda^j}{\prod\nolimits^j_{i=1}\tilde{\mu}_i}+
\fr{\lambda^s}{\prod\nolimits^s_{i=1}\tilde{\mu}_i} \sum\limits_{j=1}^{k-s}\left(
\fr{\lambda}{\tilde{\mu}_s}\right)^j\right]^{-1}.
$$

В качестве показателя эффективности системы используется $D(k)$~---
<<доход>> системы в~единицу времени в~стационарном режиме работы при
стратегии~$k$:
\begin{multline}
D(k)={C}_0\Lambda_{\mathrm{вых}}(k)-
C_1\Lambda_{\mathrm{отк}}(k) - C_2 W_{\mathrm{ож}}(k)-{}\\
{}- \sum\limits^s_{i=1} C_{3i}Q_{\mathrm{пр},i}(k) -C_4k\,,
\label{e2-ag}
\end{multline}
где $\Lambda_{\mathrm{вых}}(k)$~--- интенсивность обслуженных заявок;
$\Lambda_{\mathrm{отк}}(k)$~--- интенсивность отклоняемых заявок;
$W_{\mathrm{ож}}(k)$~--- суммарное среднее время ожидания в~очереди
обслуженных в~единицу времени заявок; $Q_{\mathrm{пр},i}(k)$~---
вероятность простоя $i$-го прибора; $C_0\hm\geq 0$~--- плата,
получаемая системой, если поступившая заявка будет обслужена системой;
$C_1\hm\geq 0$~--- штраф за отклонение поступившей заявки;
$C_2\hm\geq 0$~---  штраф за единицу времени ожидания заявки
в~очереди; $C_{3i}$~--- штраф за единицу времени простоя $i$-го
прибора; $C_4\hm\geq 0$~--- стоимость технического обслуживания
одного места в~накопителе в~единицу времени.

Задача оптимизации пороговой стратегии сформулирована в~виде
математической задачи
\begin{equation}
k^*=\mathrm{arg}\,\max\limits_{k\geq s} D(k)\,.
\label{e3-ag}
\end{equation}

\section{Решение задачи}

Рассмотрим вложенную цепь Mаркова, где переходы цепи определяются
моментами поступления заявок и~состояние системы есть число заявок,
находящихся в~системе в~момент поступления~\cite{14-ag}. Отметим, что
стационарные вероятности состояний вложенной цепи также вычисляются по
формулам~(1).

Обозначим через $q_i^k$ среднее значение дохода системы, получаемого
в~состоянии~$i$ цепи при стратегии~$k$ без учета простоя приборов. Тогда
функцию~$D(k)$ можно переписать в~виде:
\begin{equation}
D(k)=\lambda d(k) +d_{\mathrm{пр}}(k)\,,
\label{e4-ag}
\end{equation}
где
\begin{equation*}
d(k)= \sum\limits^k_{i=0}\pi_i^k q_i^k\,;\enskip
d_{\mathrm{пр}}(k)=\sum\limits^s_{i=1} C_{3i} Q_{\mathrm{пр},i}(k)\,.
\end{equation*}
Здесь
\begin{equation}
q_i^k = \begin{cases}
C_0 - C_2 W_i(k)- C_4 k\overline{v} \,, & s\leq i <k\,;\\
q^k_{k-1} -C_1 -C_0\,, & i=k\,;\\
C_0 -C_4 k \overline{v}\,, & 0\leq i\leq s-1\,,
\end{cases}
\label{e4a-ag}
\end{equation}
где
$W_i(k)$~--- суммарное среднее время ожидания заявок в~очереди
в~состоянии~$i$; $\overline{v}=1/\lambda$.

Для величины $W_i(k)$ имеют место равенства (см.~\cite{11-ag}):
$$
W_i(k) =
\begin{cases}
\fr{1}{\tilde{\mu}_s} \left[ \fr{1}{2} \sum\limits_{m=1}^{i+2-s} (m-1) mr_m - {}\right.&\\
\hspace*{2mm}\displaystyle{}- (i+1-s) 
\sum\limits_{m=1}^{i+2-s} mr_m -{}&\\
\hspace*{2mm}\left.{}- \fr{1}{2} (i+1-s)(i+2-s)
\sum\limits_{m=i+3-s}^{\infty}\!\!\! \!\! r_m \right], &\\
& \hspace*{-34mm}i = s-1, \ldots, k-1\,;\\
0, & \hspace*{-34mm}i=0, \ldots, s-1\,.
\end{cases}
$$
%
Отсюда и~из формул~(\ref{e4a-ag}) следуют равенства:
\begin{equation}
q^k_{i+1} =
\begin{cases}
 q^k_i - \fr{C_2}{\tilde{\mu}_s} \sum\limits_{m=1}^{i+2-s}
mr_m - {}&\\
\hspace*{5mm}{}-\fr{C_2 (i+2-s)}{ \tilde{\mu}_s} \sum\limits_{m=i+3-s}^{\infty}\!\!\!\!
r_m\,, &\\
&\hspace*{-20mm} s-1 \leq i \leq k-2\,;
\\
q^k_i = C_0 - C_4 k \bar{v}\,, &\hspace*{-20mm} i = 0,
\ldots, s-2\,;
\end{cases}\!
\label{e5-ag}
\end{equation}
$$
q^k_k = q^k_{k-1} - C_0 -C_1\,.
$$

Для стационарных вероятностей состояний справедлива формула:
\begin{equation}
\pi_{i+1}^{k+1} =A_{k+1}\pi_i^k\,,\enskip i=s-1, \ldots , k\,.
\label{e6-ag}
\end{equation}
Здесь
$$
A_{k+1}=\fr{1-Q_{s-1}^{k+1}}{1-Q^k_{s-2}}\,,
$$
где
$$
Q_{s-1}^{k+1}= \sum\limits_{i=0}^{s-1} \pi_i^{k+1}\,;\enskip  
Q^k_{s-2} = \begin{cases}
0\,, &s=1\,;\\
\displaystyle \sum\limits_{i=0}^{s-2} \pi_i^k\,,& s\geq 2\,.
\end{cases}
$$

Рассмотрим разность $d(k)\hm- d(k\hm+1)$. Использовав~(\ref{e5-ag})
и~(\ref{e6-ag}), находим:
\begin{multline*}
d(k) - d(k+1) =
\sum\limits^k_{i=0} \pi^k_i q^k_i -\sum\limits^{k+1}_{i=0} \pi^{k+1}_i q^{k+1}_i
={}\\
{}= \left(C_0 - C_4 k\right) \bar{v} \sum\limits^{s-2}_{i=0}
\pi^k_i  +
\sum\limits^k_{i=s-1} \pi^k_i q^k_i -{}\\
{}- \left[ C_0 -C_4
(k+1)\bar{v}\right]
\sum\limits^{s-1}_{i=0} \pi^{k+1}_i -
\sum\limits^{k+1}_{i=s} \pi^{k+1}_i q^{k+1}_i ={}\\
{}= \left(C_0 - C_4 k\right) \bar{v} \left( Q^k_{s-2} -
Q^{k+1}_{s-1}\right) +
C_4 Q^{k+1}_{s-1} \bar{v} +{}\\
{}+ \sum\limits^k_{i=s-1} \pi^k_i q^k_i -
A_{k+1} \sum\limits^k_{i=s-1} \pi^k_i q^{k+1}_{i+1}\,.
\end{multline*}

Введем для краткости изложения следующее обозначение:
$$
\Delta_{i+1}^{k+1} =q_{i+1}^{k+1}-q_i^{k+1}\,,\enskip s-1\leq i\leq k\,.
$$
Заметим, что
\begin{align*}
q^{k+1}_i &= q^k_i - C_4 \bar{v}\,, \enskip s-1\leq i \leq k-1\,;\\
q^{k+1}_{k+1} &= q^{k+1}_k - C_1 -C_0 ={}\\
&\hspace*{13mm}{}= q^{k+1}_{k-1} + \Delta^{k+1}_k - C_1 -C_0 ={}\\
&\hspace*{15mm}{}= q^k_{k-1} + \Delta^{k+1}_k - C_1 - C_0 -
C_4\bar{v} = {}\\
&\hspace*{40mm}{}=q^k_k+ \Delta^{k+1}_k - C_4 \bar{v}\,.
\end{align*}

Находим
\begin{multline}
d(k) - d(k+1) = (C_0 - C_4 k) \bar{v} \left(Q^k_{s-2} -
Q^{k+1}_{s-1}\right) +{}\\
{}+ C_4 Q^{k+1}_{s-1} \bar{v} +
 \sum\limits^k_{i=s-1} \pi^k_i q^k_i -
A_{k+1} \sum\limits^k_{i=s-1} \pi^k_i q^{k+1}_i -{}\\
{}-A_{k+1} \left(\sum\limits^{k-1}_{i=s-1} \pi^k_i \Delta^{k+1}_{i+1}+
\pi^k_k \Delta^{k+1}_{k+1}\right) ={}\\
{}= - \left(C_0 - C_4 k\right) \bar{v} \, \fr{Q^{k+1}_{s-1} -
Q^k_{s-2}}{1-Q^k_{s-2}} +{}\\
{}+ \fr{Q^{k+1}_{s-1} - Q^k_{s-2}}{1 - Q^k_{s-2}}
\left( C_0 -C_4k) Q^k_{s-2} \bar{v} +\sum\limits^k_{i=s-1}
\pi^k_i q^k_i\right) +{}\\
{}+ C_4 \bar{v} -A_{k+1} \left(\sum\limits^{k-1}_{i=s-1} \pi^k_i
\Delta^{k+1}_{i+1} + \pi^k_k \Delta^{k+1}_k\right) ={}\\
{}= ( 1 - A_{k+1}) \left[ d(k) - (C_0 - C_4 k) \bar{v} +
 C_4\, \fr{\bar{v}}{1 - A_{k+1 }} -{}\right.\\
\left.{}-\fr{A_{k+1}}{1 - A_{k+1}} \left(\sum\limits^{k-1}_{i=s-1} \pi^k_i
\Delta^{k+1}_{i+1} +
\pi^k_k \Delta^{k+1}_k\right)\right].
\label{e7-ag}
\end{multline}

Из~(\ref{e5-ag}) имеем:
\begin{multline}
\Delta_{i+1}^{k+1} ={}\\
{}=-\fr{C_2}{\tilde{\mu}_s} \left[
\sum\limits_{m=1}^{i+2-s} mr_m -(i+2-s)\sum\limits^\infty_{m=i+3-s}
r_m\right]\,,\\ 
\enskip i=s-1, \ldots , k-1\,;
\label{e8-ag}
\end{multline}

\vspace*{-12pt}

\noindent
\begin{multline*}
d(k) - d(k+1) = (1 - A_{k+1}) \!\left[ \vphantom{\sum\limits^k_{i=0}}
d(k) - ( C_0 -C_4 k) \bar{v} + {}\right.\\
{}+ C_4 \fr{\bar{v}}{1 - A_{k+1}} + 
 \fr{1}{1 - A_{k+1}}\,\fr{C_2}{\tilde{\mu}_s }
\left[  \sum\limits^k_{i=s} \pi^{k+1}_i \times{}\right.\\
{}\left[\sum\limits_{m=1}^{i+1-s} mr_m - (i+1-s) \sum\limits_{m=i+2-s}^{\infty}
r_m\right] +{}\\
\left.\left.{}+ \pi^{k+1}_{k+1} \left[\sum\limits_{m=1}^{k+1-s}  mr_m
- (k +1 -s) \sum\limits_{m=k+2-s}^{\infty}\!\! r_m\right]\right]\right].\hspace*{-2.35pt}
\end{multline*}

Обозначив
\begin{multline}
F(k) =
\fr{1}{1 - A_{k+1}}
\left[  \sum\limits^k_{i=s} \pi^{k+1}_i \times{}\right.\\
{}\times \left[\sum\limits_{m=1}^{i+1-s} mr_m - (i+1-s) \sum\limits_{m=i+2-s}^{\infty}
r_m\right] +{}\\
\left.{}+ \pi^{k+1}_{k+1}\! \left[\sum\limits_{m=1}^{k+1-s} \! mr_m
- (k +1 -s) \!\!\sum\limits_{m=k+2-s}^{\infty}\!\!\!\!\! r_m\right]\right];\!
\label{e9-ag}
\end{multline}
$$
G(k) = (C_0 -C_4 k)\bar{v} - C_4 \, \fr{\bar{v}}{1 -
A_{k+1}} - \fr{C_2}{\tilde{\mu}_s}\, F(k)\,,
$$
из (\ref{e7-ag}) и~(\ref{e8-ag}) получим
\begin{equation}
d(k)-d(k+1)=\left(1-A_{k+1}\right)[d(k)-G(k)]\,.
\label{e10-ag}
\end{equation}
Для $(1\hm- A_{k+1})F(k)$ из~(\ref{e9-ag}) после простых преобразований
получим:
\begin{multline*}
\fr{1}{\tilde{\mu}_s}(1 - A_{k+1}) F(k) =
\fr{1}{\tilde{\mu}_s} \sum\limits^k_{i=s} \pi^{k+1}_i \sum\limits_{m=1}^{i+1-s} mr_m +{}\\
{}+ \fr{1}{\tilde{\mu}_s} \pi^{k+1}_{k+1} \sum\limits_{m=1}^{k+1-s} mr_m +{}\\
{}+ \fr{1}{\tilde{\mu}_s} \sum\limits^k_{i=s} \pi^{k+1}_i (i +1 -s)
\sum\limits_{m=i+2-s}^{\infty} r_m +{}\\
{}+ \fr{1}{\tilde{\mu}_s} ( k +1 -s) \pi^{k+1}_{k+1}
\sum\limits_{m=k+2-s}^{\infty} r_m\,.
\end{multline*}

Легко показать, что правая часть последнего равенства выражает среднее время
ожидания поступившей заявки за период пребывания вложенной цепи
в~произвольно взятом состоянии. Тогда отсюда очевидно следует, что
в~рассматриваемой СМО с~увеличением $k\hm\geq s$ величина
$(1\hm- A_{k+1})F(k)$ возрастает и,~следовательно, $G(k)$ убывает, так как
$0\hm< A_{k+1}\hm< 1$ возрастает по~$k$ (см.~\cite{11-ag}).
Поскольку~$G(k)$~--- невозрастающая функция, то из~(\ref{e10-ag})
(см.~\cite{13-ag}) следует, что $d(k)$~--- унимодальная функция, причем при
$C_2\hm=0$ она всюду возрастает и~ее точка максимума~$k^0$
равна~$\infty$ ($k^0\hm=\infty$), при $d(s)\hm\geq G(s)$ всюду убывает
и~$k^0\hm=s$, в~остальных случаях существует глобальный оптимум
$s\hm<k^0\hm<\infty$. Более того, из формулы~(\ref{e10-ag}), свойства
монотонности функций~$G(k)$ и~$A_{k+1}$, а~также неравенства
$0\hm<A_{k+1}\hm<1$ следует, что $d(k)$~--- выпуклая вверх функция.

Находим
\begin{multline}
d_{\mathrm{пр}} (k) -
d_{\mathrm{пр}} (k+1) ={}\\
{}=
 \sum\limits^s_{i=1} C_{3i}\sum\limits^{i-1}_{j=0} \pi^k_j -
\sum\limits^s_{i=1} C_{3i}\sum\limits^{i-1}_{j=0} \pi^{k+1}_j ={}\\
{}= \sum\limits^s_{i=1} C_{3i}\sum\limits^{i-1}_{j=0}
\fr{\lambda^j}{\prod\nolimits^j_{i=1}\tilde{\mu}_s } \left( \pi^k_0 - \pi^{k+1}_0\right) =
 \sum\limits^s_{i=1} C_{3i}\times{}\\
 {}\times \sum\limits^{i-1}_{j=0}
\pi^k_0\,\fr{\lambda^j}{\prod\nolimits^j_{i=1}\tilde{\mu}_s }\,
 \pi^{k+1}_0\,\fr{\lambda^s}{\prod\nolimits^s_{i=1}\tilde{\mu}_i } \left(
\fr{\lambda}{\tilde{\mu}_s }\right)^{k+1-s} ={}\\
{}= \pi^{k+1}_{k+1}
 \sum\limits^s_{i=1} C_{3i}\sum\limits^{i-1}_{j=0} \pi^k_j =
\pi^{k+1}_{k+1} \sum\limits^s_{i=1} C_{3i} Q_{\mathrm{пр},i} (k)\,.
\label{e11-ag}
\end{multline}

Как следует из~(\ref{e11-ag}), функция $d_{\mathrm{пр}}(k)$~--- монотонно убывающая и~выпуклая
вниз функция, т.\,е.~$D(k)$~--- сумма двух выпуклых вверх функций
(см.~(\ref{e4-ag})).

Таким образом, доказано следующее утверждение.

\smallskip

\noindent
\textbf{Утверждение~1.}  \textit{Функция $D(k)$~--- выпуклая вверх функция,
и~решение задачи}~(\ref{e3-ag}) \textit{удовлетворяет усло\-виям}:
$$
k^*=\begin{cases}
\infty\,, &\hspace*{-1mm}\mbox{\textit{если} } C_2=0\,;\\
s\,, &\hspace*{-1mm}\mbox{\textit{если} } d_{\mathrm{пр}}(s+1)-d_{\mathrm{пр}}(s)<{}\\
&\hspace*{20mm}{}<\lambda
[d(s)-d(s+1)]\,,
\end{cases}
$$
\textit{иначе} $0<k^*<\infty$.

\smallskip

Отметим, что аналогичный результат имеет мес\-то и~тогда, когда
$C_2$~--- штраф за превышение временем пребывания заявки
в~очереди допустимого предельного значения (дедлайна)
$t_{\mathrm{д}}\hm>0$, т.\,е.\ при замене в~формуле~(\ref{e2-ag})
функции~$W_{\mathrm{ож}}(k)$ на $\Lambda_{\mathrm{д}}(k)$~---
интенсивность обслуженных заявок, время ожидания которых в~очереди
превышает~$t_{\mathrm{д}}$.

 Обозначим показатель эффективности через $\tilde{D}(k)$:
\begin{multline*}
\tilde{D}(k)=C_0\Lambda_{\mathrm{вых}}(k)-
C_1\Lambda_{\mathrm{отк}}(k) -
C_2\Lambda_{\mathrm{д}}(k) -{}\\
{}-\sum\limits^s_{i=1}
C_{3i}Q_{\mathrm{пр},i}(k)-C_4 k\,.
\end{multline*}

Рассмотрим вложенную цепь Маркова. Для краткости изложения положим
$C_4\hm=0$. Тогда имеет место равенство:
\begin{equation}
\tilde{D}(k) =\lambda\tilde{d}(k)+\sum\limits^s_{i=1}
C_{3i}Q_{\mathrm{пр},i}(k)\,.
\label{e12-ag}
\end{equation}
Здесь
\begin{equation*}
\tilde{d}(k) = \sum\limits^k_{i=0} \pi_i^k \tilde{q}_i^k\,,
\end{equation*}
где
$$
\tilde{q}_i^k = \!\begin{cases}
C_0\,,&\hspace*{-2.6mm}\mbox{если } 0\leq i<s\,;\\
\left( C_0+C_2\right)
\Gamma_{\tilde{\mu}_s,i-s+1}(t_{\mathrm{д}})-C_2, &\hspace*{-2.6mm}\mbox{если } s\leq
i<k\,;\\
-C_1\,, &\hspace*{-2.6mm}\mbox{если } i=k\,;
\end{cases}
$$
$\Gamma_{a,b}(\cdot)$~--- функ\-ция гам\-ма-рас\-пре\-де\-ле\-ния
с~па\-ра\-мет\-ра\-ми~$a$~и~$b$.
%
Отсюда при $0\hm\leq i\hm\leq k\hm-2$ получим:
$$
\tilde{q}^k_{i+1} =\begin{cases}
\tilde{q}_i^k\,, &\hspace*{-20mm}0\leq i\leq s-2\,;\\
\tilde{q}_i^k-\left( C_0+C_2\right)
\fr{(\tilde{\mu}_st_{\mathrm{д}})^{i+1}}{(i+1)!}\,
e^{-\tilde{\mu}_st_{\mathrm{д}}}\,, &\\
&\hspace*{-20mm} s-1\leq i\leq k-2\,.
\end{cases}
$$

Проведя такие же преобразования, как и~выше для разности $d(k)\hm-
d(k\hm+1)$ (см.\ также~\cite{12-ag}), получим

\pagebreak

%\begin{figure*} %fig1
\vspace*{-6pt}
\begin{center}
\mbox{%
\epsfxsize=79mm
\epsfbox{aga-1.eps}
}
\end{center}

\vspace*{-2pt}

\noindent
{\small Зависимость дохода системы от порогового значения длины очереди}

\vspace*{24pt}

%\end{figure*}



\noindent
$$
\tilde{d}(k)-\tilde{d}(k+1)=(1-A_{k+1})\left[ \tilde{d}(k)-\tilde{G}(k)\right]\,,
$$
где

\vspace*{-6pt}

\noindent
\begin{multline*}
\tilde{G}(k)= -\left(C_0+C_2\right) \fr{\pi_0^k}{1-A_{k+1}}\,
e^{ -\tilde{\mu}_st_{\mathrm{д}}} \fr{\lambda^s} {\prod\nolimits^s_{i=1}
\tilde{\mu}_i} \times{}\\
{}\times \sum\limits^k_{j=s} \left( \fr{\lambda} {\tilde{\mu}_s}\right)^{j-s}
\fr{(\tilde{\mu}_st_{\mathrm{д}})^{j-s}}{(j-s)!}\,.
\end{multline*}

Рассуждая так же, как и~в~\cite{12-ag}, можно доказать, что $\pi_0^k/(1-
A_{k+1})$ не зависит от~$k$ и~$\tilde{G}(k)$ убывает по $k\hm\geq s$. Таким
образом, для~$\tilde{d}(k)$ выполняются те же условия, что и~для $d(k)$, из
которых следовала выпуклость $d(k)$ по $k\hm\geq s$, т.\,е. $\tilde{d}(k)$~---
тоже выпуклая функция. Тогда из~(\ref{e12-ag}) получаем, что
$\tilde{D}(k)$~--- выпуклая функция по $k\hm\geq s$ и~для нее также
справедливо утверждение~1.

Утверждение~1 справедливо также в~случае однородных приборов
с~интенсивностью обслуживания $\mu_i\hm= \mu$, $i\hm=1,\ldots , s$, если
$\Lambda_{\mathrm{д}}(k)$~--- интенсивность обслуженных заявок, время
пребывания которых в~системе превысило дедлайн. В~этом случае имеем:
$$
\tilde{q}^k_i =
\begin{cases}
(C_0 + C_2)\Gamma_{\mu, 1}(t_{\mathrm{д}}) -
C_2, & \hspace*{-15mm}\mbox{если } 0\leq i < s\,;\\
(C_0 + C_2)\Gamma_{\tilde{\mu}, i+1}(t_{\mathrm{д}}) *
\Gamma_{\mu, 1}(t_{\mathrm{д}}) - C_2, &\\
&\hspace*{-15mm}\mbox{если } s\leq i < k\,;\\
-C_1, & \hspace*{-15mm}\mbox{если }  i = k\,;
\end{cases}
$$
$$
\tilde{q}^k_{i+1}= \begin{cases}
\tilde{q}^k_i\,, \qquad\qquad \hspace*{15mm} 0 \leq i \leq s-2\,;\\
\tilde{q}^k_i - ( C_0+ C_2) 
\left[ \Gamma_{\tilde{\mu}_s,i-s+1}(t_{\mathrm{д}})-{}\right.\\
&\hspace*{-146pt}\left.{}-\Gamma_{\tilde{\mu}_s,i-s+2}(t_{\mathrm{д}})\right]*
\Gamma_{\mu,1}(t_{\mathrm{д}})\,,\\
\qquad \qquad \hspace*{15mm}s -1 \leq i \leq k-2\,,
\end{cases}
$$
где $*$~--- знак свертки функций.



На рисунке проиллюстрированы примеры зависимости предельного дохода
системы в~единицу времени от порогового значения длины очереди при
параметрах $C_0 = 20$, $C_1 = 10$, $C_2 =
0{,}1$, $C_{31} = 1{,}5$,
$C_{32} = 1$, $C_{33} = 0{,}5$, $C_4 = 0{,}05$,
$\lambda = 2$, $\mu_1 = 1{,}5$, $\mu_2 = 1{,}0$ и~$\mu_3 = 0{,}5$~(\textit{1})
и~$C_0 = 20$, $C_1 = 10$, $C_2 = 0{,}1$,
$C_{31} = 1{,}5$, $C_{32} = 1$, $C_{33} = 0{,}5$, $C_4 = 0{,}05$,
$\lambda = 6$, $\mu_1 = 3$, $\mu_2 = 2$ и~$\mu_3 = 1$
(\textit{2}, график сжат по оси ординат с~коэффициентом~1/3).

%\vspace*{-6pt}

\section{Заключение}

Полученный в~данной работе результат обобщает результаты исследования
задачи оптимизации стоимостной целевой функции на множестве прос\-тых
пороговых стратегий в~СМО с~управ\-ля\-емой очередью, приведенных
в~работах~\cite{11-ag, 12-ag}. В~настоящей работе использован тот же подход,
что и~в~\cite{11-ag, 12-ag}, и~доказано, что целевая функция удовле\-тво\-ря\-ет
условиям теоремы из работы~\cite{13-ag}, где приведены достаточные условия
унимодальности целевой функции. Отметим также, что приведенное в~работе
утверждение доказано с~применением этого подхода и~в~рамках задачи,
рассмотренной в~работе~\cite{9-ag} для СМО с~нетерпеливыми заявками
(заявка после поступления в~очередь через экспоненциально распределенный
интервал времени покидает систему, если она за это время не обслужилась,
и~за это система платит штраф), т.\,е.\ для этой задачи выводятся необходимые
и~достаточные условия конечности оптимального порога, из которых следует
точное алгоритмическое ее решение.

В дальнейшем представляет интерес применение данного подхода
к~исследованию свойства унимодальности целевой функции для СМО с~более\linebreak
общими параметрами и~пороговой стратегией управ\-ле\-ния очередью.

%\vspace*{-6pt}

{\small\frenchspacing
{%\baselineskip=10.8pt
%\addcontentsline{toc}{section}{References}
\begin{thebibliography}{99}

%\vspace*{-2pt}

\bibitem{1-ag}
\Au{Клейнрок Л.} Вычислительные системы с~очередями~/ Пер. с~англ.~--- М.:
Мир, 1979. 600~с. (\Au{Kleinrock~L.} Queueing systems. Vol.~II: Computer
applications.~--- New York, NY, USA: Wiley, 1976. 576~p.)
\bibitem{2-ag}
\Au{Stidham Sh., Jr., Weber~R.} A~survey of Markov decision models for control of
networks of queues~// Queueing Syst., 1993. Vol.~13. P.~291--314.
\bibitem{3-ag}
\Au{Рыков В.\,В.} Об условиях монотонности оптимальных политик управления
системами массового обслуживания~// Автоматика и~телемеханика, 1999. №\,9.
С.~92--106.

%\pagebreak

\bibitem{4-ag}
\Au{Rykov V., Efrosinin~D.} Optimal control of queueing systems with
heterogeneous servers~// Queueing Syst., 2004. Vol.~46. P.~389--407.
\bibitem{5-ag}
\Au{Efrosinin~D.} Queueing model of a~hybrid channel with faster link subject to
partial and complete failures~// Ann. Oper. Res., 2013. Vol.~202. Iss.~1. P.~75--102.
\bibitem{6-ag}
\Au{$\ddot{\mbox{O}}$zkan E., Kharoufeh~J.\,P.} Optimal control of a~two-server
queueing system with failures~// Probab. Eng. Inform. Sc.,
2014. Vol.~28. Iss.~4. P.~489--527.
\bibitem{7-ag}
\Au{Ata B., Shneorson~S.} Dynamic control of an $M/M/1$ service system with
adjustable arrival and service rates~// Manage. Sci., 2006. Vol.~52. Iss.~11.
P.~1778--1791.
\bibitem{8-ag}
\Au{Ward A., Kumar~S.} Asymptotically optimal admission control of a~queue with
impatient customers~// Math. Oper. Res., 2008. Vol.~33. Iss.~1. P.~167--202.
\bibitem{9-ag}
\Au{Kocaga Y.\,L., Ward~A.\,R.} Admission control for a~multi-server queue with
abandonment~// Queueing Syst., 2010. Vol.~65. Iss.~3. P.~275--323.
\bibitem{10-ag}
\Au{Ghosh A.\,P., Weerasinghe~A.\,P.} Optimal buffer size and dynamic rate control
for a~queueing system with impatient customers in heavy traffic~// Stoch. Proc.
Appl., 2010. Vol.~120. Iss.~11. P.~2103--2141.
\bibitem{11-ag}
\Au{Агаларов Я.\,М., Ушаков~В.\,Г.} Об унимодальности функции дохода СМО
типа $G/M/s$ с~управляемой очередью~// Информатика и~её применения, 2019.
Т.~13. Вып.~1. С.~55--61.
\bibitem{12-ag}
\Au{Агаларов Я.\,М., Коновалов~М.\,Г.} Доказательство унимодальности
целевой функции в~задаче порогового управления нагрузкой на сервер~//
Информатика и~её применения, 2019. Т.~13. Вып.~2. С.~2--6.

\bibitem{14-ag}
\Au{Бочаров П.\,П., Печинкин~А.\,В.} Теория массового обслуживания.~--- М.:
РУДН, 1995. 529~с.

\bibitem{13-ag}
\Au{Агаларов Я.\,М.} Признак унимодальности целочисленной функции одной
переменной~// Обозрение прикладной и~промышленной математики, 2019.
Т.~26. Вып.~1. С.~65--66.

\end{thebibliography}

}
}

\end{multicols}

\vspace*{-3pt}

\hfill{\small\textit{Поступила в~редакцию 01.08.2020}}

\vspace*{8pt}

%\pagebreak

%\newpage

%\vspace*{-28pt}

\hrule

\vspace*{2pt}

\hrule

%\vspace*{-2pt}

\def\tit{OPTIMAL THRESHOLD-BASED ADMISSION CONTROL
IN~THE~$M/M/s$ SYSTEM WITH~HETEROGENEOUS SERVERS
AND~A~COMMON QUEUE}

\def\titkol{Optimal threshold-based admission control in~the~$M/M/s$ system
with~heterogeneous servers and~a~common queue}

\def\aut{Ya.\,M. Agalarov}

\def\autkol{Ya.\,M. Agalarov}

\titel{\tit}{\aut}{\autkol}{\titkol}

\vspace*{-11pt}


\noindent
Institute of Informatics Problems, Federal Research Center ``Computer Science and
Control'' of the Russian Academy of Sciences, 44-2~Vavilov Str., Moscow 119333,
Russian Federation

\def\leftfootline{\small{\textbf{\thepage}
\hfill INFORMATIKA I EE PRIMENENIYA~--- INFORMATICS AND
APPLICATIONS\ \ \ 2021\ \ \ volume~15\ \ \ issue\ 1}
}%
\def\rightfootline{\small{INFORMATIKA I EE PRIMENENIYA~---
INFORMATICS AND APPLICATIONS\ \ \ 2021\ \ \ volume~15\ \ \ issue\ 1
\hfill \textbf{\thepage}}}

\vspace*{3pt}

\Abste{The article discusses the $M/M/s$ system with heterogeneous servers and
a~common queue equipped with the mechanism to control the queue length in order
to maximize the average marginal profit. The profit function includes a~fee for
successfully serviced customers, a~fine for each rejected customer, a~fine for idle
period for each server, a~fine for waiting (or for exceeding the allowable waiting
time), and costs associated with queue maintenance. The problem is to maximize the
marginal profit on a~set of simple threshold-based queue length control policies. The
property of convexity of the profit function is proved and conditions for existence of
a~finite optimal threshold of the queue length are obtained.}

%\vspace*{2pt}

\KWE{queuing system; optimization; threshold strategy; queue length}

\DOI{10.14357/19922264210108}

\vspace*{-8pt}

\Ack
\noindent
The reported study was partly funded by RFBR, projects 
Nos.\,18-07-00692, 19-07-00739, and 20-07-00804.

\vspace*{6pt}

  \begin{multicols}{2}

\renewcommand{\bibname}{\protect\rmfamily References}
%\renewcommand{\bibname}{\large\protect\rm References}

{\small\frenchspacing
 {%\baselineskip=10.8pt
 \addcontentsline{toc}{section}{References}
 \begin{thebibliography}{99}

\bibitem{1-ag-1}
\Aue{Kleinrock, L.} 1976. \textit{Queueing systems. Vol.~II: Computer
applications}. New York, NY: Wiley. 576~p.
\bibitem{2-ag-1}
\Aue{Stidham, Sh., Jr., and R.~Weber.} 1993. A~survey of Markov decision models
for control of networks of queues. \textit{Queueing Syst.} 13:291--314.
\bibitem{3-ag-1}
\Aue{Rykov, V.\,V.} 1999. On monotonicity conditions for optimal policies for the
control of queueing systems. \textit{Automat. Rem. Contr.} 60(9):1290--1301.
\bibitem{4-ag-1}
\Aue{Rykov, V., and D.~Efrosinin.} 2004. Optimal control of queueing systems with
heterogeneous servers. \textit{Queueing Syst.} 46:389--407.
\bibitem{5-ag-1}
\Aue{Efrosinin, D.} 2013. Queueing model of a~hybrid channel with faster link
subject to partial and complete failures. \textit{Ann. Oper. Res.} 202(1):75--102.
\bibitem{6-ag-1}
\Aue{$\ddot{\mbox{O}}$zkan, E., and J.\,P.~Kharoufeh.} 2014. Optimal control of
a~two-server queueing system with failures. \textit{Probab. Eng.
Inform. Sc.} 28(4):489--527.
\bibitem{7-ag-1}
\Aue{Ata, B., and S.~Shneorson.} 2006. Dynamic control of an $M/M/1$ service
system with adjustable arrival and service rates. \textit{Manage. Sci.}
52(11):1778--1791.
\bibitem{8-ag-1}
\Aue{Ward, A., and S.~Kumar.} 2008. Asymptotically optimal admission control of
a~queue with impatient customers. \textit{Math. Oper. Res.} 33(1):167--202.
\bibitem{9-ag-1}
\Aue{Kocaga, Y.\,L., and A.\,R.~Ward.} 2010. Admission control for a~multi-server
queue with abandonment.  \textit{Queueing Syst.} 65(3):275--323.
\bibitem{10-ag-1}
\Aue{Ghosh, A.\,P., and A.\,P.~Weerasinghe.} 2010. Optimal buffer size and
dynamic rate control for a~queueing system with impatient customers in heavy
traffic. \textit{Stoch. Proc. Appl.} 120(11):2103--2141.
\bibitem{11-ag-1}
\Aue{Agalarov, Ya.\,M., and V.\,G.~Ushakov}. 2019. Ob unimodal'nosti funktsii
dokhoda SMO tipa $G/M/s$ s~uprav\-lya\-emoy ochered'yu [On the unimodality of the
income function of a~type $G/M/s$ queueing system with controlled queue].
\textit{Informatika i~ee Primeneniya~--- Inform. Appl.} 13(1):55--61.
\bibitem{12-ag-1}
\Aue{Agalarov, Ya.\,M., and M.\,G.~Konovalov.} 2019. Dokazatel'stvo
unimodal'nosti tselevoy funktsii v~zadache porogovogo upravleniya nagruzkoy na
server [Proof of the unimodality of the objective function in $M/M/N$ queue with
threshold-based congestion control]. \textit{Informatika i~ee Primeneniya~---
Inform. Appl.} 13(2):2--6.

\bibitem{14-ag-1}
\Aue{Bocharov, P.\,P., and A.\,V.~Pechinkin.} 1995. \textit{Teoriya massovogo
obsluzhivaniya} [Queueing theory]. Moscow: RUDN. 529~p.

\bibitem{13-ag-1}
\Aue{Agalarov, Ya.\,M.} 2019. Priznak unimodal'nosti tselochislennoy funktsii
odnoy peremennoy [A~sign of unimodality of an integer function of one variable].
\textit{Obozrenie prikladnoy i~promyshlennoy matematiki} [Surveys Applied and
Industrial Mathematics] 26(1):65--66.

\end{thebibliography}

 }
 }

\end{multicols}

\vspace*{-3pt}

  \hfill{\small\textit{Received August~1, 2020}}


%\pagebreak

\vspace*{-8pt}



\Contrl

\noindent
  \textbf{Agalarov Yaver M.} (b.\ 1952)~--- Candidate of Science (PhD) in
technology, associate professor, leading scientist, Institute of Informatics Problems,
Federal Research Center ``Computer Science and Control'' of the Russian Academy
of Sciences, 44-2~Vavilov Str., Moscow 119333, Russian Federation;
\mbox{agglar@yandex.ru}
\label{end\stat}

\renewcommand{\bibname}{\protect\rm Литература}