\def\stat{Kovalev}

\def\tit{МЕТОДЫ ТЕОРИИ КАТЕГОРИЙ\\ В~ЦИФРОВОМ 
ПРОЕКТИРОВАНИИ\\ ГЕТЕРОГЕННЫХ КИБЕРФИЗИЧЕСКИХ СИСТЕМ}

\def\titkol{Методы теории категорий в~цифровом проектировании 
гетерогенных киберфизических систем}

\def\aut{С.\,П.~Ковалёв$^1$}

\def\autkol{С.\,П.~Ковалёв}

\titel{\tit}{\aut}{\autkol}{\titkol}

\index{Ковалёв С.\,П.}
\index{Kovalyov S.\,P.}

%{\renewcommand{\thefootnote}{\fnsymbol{footnote}} \footnotetext[1]
%{Работа выполнена при частичной поддержке РФФИ (проект 19-07-00187-A).}}

\renewcommand{\thefootnote}{\arabic{footnote}}
\footnotetext[1]{Институт проблем управления им.\ В.\,А.~Трапезникова 
Российской академии наук, \mbox{kovalyov@sibnet.ru}}

%\vspace*{-12pt}

\Abst{Развивается предложенный ранее математический аппарат на базе теории 
категорий, предназначенный для формального описания и~строгого 
исследования процедур инженерной деятельности на базе математического 
и~компьютерного моделирования. При помощи аппарата описаны 
и~исследованы высокоавтоматизированные процедуры проектирования 
гетерогенных киберфизических систем на основе цифровых двойников, 
востребованные грядущей четвертой промышленной революцией. Для этого 
впервые введена конструкция категории мультизапятой, объектами которой 
служат архитектурные модели некоторой гетерогенной киберфизической 
системы с~заданной схемой структурной иерархии, представленные 
с~некоторой фиксированной точки зрения, а~морфизмы отвечают действиям по 
подбору составных частей (СЧ) для сборки системы из них. Рассмотрено 
применение категории мультизапятой в~решении прямых и~обратных задач 
проектирования отдельных систем и~состоящих из них так называемых систем 
систем (СС).}

\KW{киберфизическая система; цифровой двойник; порождающее 
проектирование; система систем; теория категорий; категория мультизапятой}

\DOI{10.14357/19922264210104}

\vspace*{12pt}

\vskip 10pt plus 9pt minus 6pt

\thispagestyle{headings}

\begin{multicols}{2}

\label{st\stat}

\section{Введение}

Концепция киберфизической системы (cyber-physical system) состоит 
в~развитии традиционных\linebreak автоматизированных систем в~направлении 
максимально тесной интеграции мира физических объектов с~виртуальным 
миром управления в~целях повышения качества и~оперативности управления 
\mbox{объектами}. Такая сис\-те\-ма вклю\-ча\-ет разнородные физические компоненты, 
оснащенные большим чис\-лом цифровых датчиков и~исполнительных\linebreak 
механизмов, позволяющих скомпоновать и~поддерживать в~системе цифровой 
двойник (digital twin)~--- виртуальную модель, воспроизводящую и~за\-да\-ющую 
состояние и~поведение оригинала в~реальном времени~[1]. Структура 
и~начальное информационное наполнение двойника формируются в~цикле 
высокоавтоматизированного цифрового проектирования системы. 

Примерами 
киберфизических систем служат <<умные>> здания, города, энергетические 
системы, сетецентрические воинские формирования и~т.\,д.

В рамках парадигмы грядущей четвертой промышленной революции 
(Industrie~4.0) степень автоматизации проектирования и~производства 
повышается вплоть до полной замены человека \mbox{искусственным} интеллектом 
в~цикле по\-рож\-да\-юще\-го проектирования (generative design)~\cite{2-kov}. 
Однако традиционные информационные технологии, в~том числе на основе 
машинного обучения, не способны обеспечить искомый уровень автоматизации 
по крайней мере в~двух важных аспектах: 
\begin{enumerate}[(1)]
\item  интеграция 
разноплановых видов и~языков моделирования; 
\item  минимизация 
потребности в~вычислительных ресурсах для решения оптимизационных задач 
проектирования.
\end{enumerate}

 Актуальна разработка новых подходов и~методов цифрового 
проектирования, в~том числе основанных на перспективном математическом 
аппарате.

В настоящей работе, следуя современным тенденциям~[3--5], в~качестве 
основы используется тео\-рия категорий. Модели компонентов и~систем 
рассматриваются как объекты подходящих категорий, морфизмы в~которых 
описывают действия, связанные со сборкой сложных изделий. Строятся 
и~исследуются тео\-ре\-ти\-ко-ка\-те\-гор\-ные конструкции, описывающие 
методы цифрового проектирования киберфизических систем на абстрактном 
концептуальном уровне и~открывающие новые пути к~автоматизации. 
В~частности, впервые введена конструкция категории мультизапятой.

\begin{figure*}[b] %fig1
\vspace*{1pt}
 \begin{center}
 \mbox{%
 \epsfxsize=118mm  
 \epsfbox{kov-1.eps}
 }
  \end{center}
\vspace*{-9pt}
\Caption{Структурная схема <<умного>> здания~\cite{9-kov}}
\end{figure*}

\section{Принципы цифрового проектирования гетерогенных 
киберфизических систем}

Задача проектирования состоит в~формировании архитектуры системы и~ее 
описании в~различных моделях. Согласно основополагающему %\linebreak 
стандарту 
ISO/IEC/IEEE 42010:2011~\cite{6-kov}, описание \mbox{архитектуры} включает 
представления, отра\-жа\-ющие точки зрения различных заинтересованных 
сторон: пользователей, производителей, {проектировщиков}, эксплуатирующего 
персонала и~т.\,д. В~{результате} <<обезличивания>> (устранения субъективных 
факторов заинтересованных сторон) и~<<гранулирования>> (расщепления 
в~целях уменьшения пересечений) выделяются типовые точки зрения, такие 
как пространственное расположение %\linebreak 
(гео\-мет\-ри\-че\-ская форма), поведение, 
технология производства, надежность и~т.\,п. Со\-от\-вет\-ст\-ву\-ющие модели 
в~совокупности образуют элек\-т\-рон\-но-циф\-ро\-вой макет  
системы~\cite{7-kov}~--- основу циф\-ро\-во\-го двойника.
{\looseness=-1

}

По структуре архитектурная модель системы представляет собой 
ориентированный граф, узлы которого помечены моделями составляющих 
сис\-тем\-ных единиц, а~ребра~--- описаниями действий по иерархической сборке 
СЧ от деталей до системы в~целом. Модели задаются 
в~различных формах, таких как геометрические фигуры и~тела, системы 
дифференциальных уравнений, множества с~операциями и~отношениями, базы 
данных и~т.\,д. Все модели некоторого заданного вида и~описания всех 
действий предоставляются проектировщику в~виде (виртуального) каталога, из 
которого он выбирает строительные блоки для архитектуры.

Построение архитектуры усложняется, когда сис\-те\-ма включает множество 
существенно разнородных СЧ. В~этом случае для каж\-дой СЧ имеется свой 
самостоятельный каталог моделей и~априори не ясно, в~каких терминах 
описывать действия по сборке сис\-тем из таких СЧ. Можно зафиксировать лишь 
общую схему структурной иерархии СЧ, обуслов\-лен\-ную их природой. Такая 
<<схема деления>>~\cite{8-kov} формируется в~самом начале жизненного 
цикла изделия, когда определяется его концептуальный облик, и~считается 
неизменной в~процессе проектирования. Киберфизические системы могут 
включать в~себя такие СЧ (подсистемы), как механическая (несущая), 
гидравлическая, электрическая, тепловая, защитная,  
конт\-роль\-но-из\-ме\-ри\-тель\-ная,  
ин\-фор\-ма\-ци\-он\-но-ком\-му\-ни\-ка\-ци\-он\-ная (которая в~свою очередь 
разделяется на программное и~аппаратное обеспечение). Ярким примером 
служит <<умное>> здание, крупноблочная структура которого упрощенно 
показана на рис.~1~\cite{9-kov}.




Чтобы превратить схему деления в~полноценную архитектурную модель, 
необходимо отобразить ее в~рамках некоторой точки зрения на изделие. Такое 
отображение возможно, поскольку каж\-дая точка зрения определяет как 
некоторый аспект каждой СЧ, так и~действия, связанные с~составлением 
в~этом аспекте сложных единиц из более прос\-тых. Например, с~точки зрения 
пространственного расположения сборка сводится к~взаимному расположению 
и~полной либо частичной относительной фиксации геометрических форм СЧ. 
С~точки\linebreak\vspace*{-9pt}

%\pagebreak

{ \begin{center}  %fig2
 \vspace*{-1pt}
   \mbox{%
\epsfxsize=78.413mm 
\epsfbox{kov-2.eps}
}

\end{center}

\noindent
{{\figurename~2}\ \ \small{
Структурно целостный переход между вариантами ЭСИ
}}}

\vspace*{18pt}

\noindent
 зрения поведения СЧ описываются сценариями~--- фрагментами 
предполагаемой истории их поведения, представленного потоком дискретных 
событий, так что описания действий по сборке \mbox{сценариев} поведения сложных 
систем отображают вклад сценариев поведения СЧ.





Представление архитектуры системы с~некоторой точки зрения, заданное 
в~цифровом виде, называется электронной структурой изделия  
(ЭСИ)~\cite{10-kov}. На разных стадиях жизненного цикла системы 
формируются и~прорабатываются разные виды ЭСИ, соответствующие точкам 
зрения различных заинтересованных сторон: конструктивная,  
про\-из\-вод\-ст\-вен\-но-тех\-но\-ло\-ги\-че\-ская, эксплуатационная и~т.\,д. 
Компьютерные инструменты проектирования помогают проектировщику 
подбирать СЧ и~действия в~ЭСИ так, чтобы результирующая система 
удовле\-тво\-ря\-ла требованиям заинтересованных сторон. Классические 
сис\-те\-мы автоматического проектирования 
способны автоматически решать прямые задачи проектирования~--- строить 
виртуальную модель изделия (вершину иерархии ЭСИ), исходя из вручную 
выбранных проектировщиком СЧ и~действий по сборке, рассчитывать ее 
характеристики и~сопоставлять с~требованиями. Однако для технологий типа 
порождающего проектирования этого недостаточно: компьютер должен 
самостоятельно автоматически подбирать из каталога СЧ и~способы их сборки 
так, чтобы в~наибольшей степени удовлетворить требованиям.










Ключевую роль в~порождающем проектировании играет удобное для 
компьютерной навигации пространство проектирования (design space), 
состоящее из всех допустимых вариантов ЭСИ. В~рамках некоторой 
фиксированной точки зрения удобство перехода между вариантами 
обеспечивается, в~частности, когда каждая СЧ исходного варианта переходит 
в~заменяющую ее СЧ целевого варианта посредством некоторого действия. 
Более того, <<хороший>> переход является структурно целостным в~том 
смыс\-ле, что образующие его действия однозначно комбинируются 
с~действиями, со\-став\-ля\-ющи\-ми ЭСИ с~выбранной точки зрения~\cite{11-kov}, 
как показано на рис.~2. Пространства проектирования гетерогенных сис\-тем 
с~такими переходами будут далее описаны на строгом алгебраическом языке, 
что открывает перспективу для применения эффективных алгоритмов 
поисковой оптимизации с~привлечением средств компьютерной ал\-гебры.



Сложность проектирования киберфизических систем усугубляется тем 
обстоятельством, что такие системы редко проектируются по отдельности. 
Напротив, на практике обычно возникает потребность в~целой системе таких 
систем (System of Systems, SoS), в~качестве СЧ которой выступают 
полноценные и~вполне самостоятельные киберфизические системы. В~рамках 
СС они обмениваются информацией и~вступают 
в~коллаборацию в~целях оптимизации управления многошаговыми 
многосубъектными процессами с~помощью средств типа платформы 
<<умных>> сервисов~\cite{1-kov}. Часто встречается такой класс СС, как 
группа однотипных киберфизических объектов, организованная для 
достижения общей цели, например ударная группировка беспилотных 
летательных аппаратов. Синтез таких групп относится к~прямым задачам 
проектирования СС. А~обратные задачи требуют проведения иерархически 
организованных процедур поисковой оптимизации в~пространствах 
проектирования СЧ (имеющих, вообще говоря, разные схемы деления). 
Примером такой обратной задачи служит порождающее проектирование 
<<умной>> городской агломерации, состоящей из зданий и~инфраструктурных 
систем различного назначения. В~качестве подхода к~решению таких задач 
далее рассмотрим приведение к~плоскому поиску путем <<отрисовки>> 
(подстановки в~развернутом виде) ЭСИ СЧ в~ЭСИ~СС.

\section{Категория мультизапятой}

Будем пользоваться теоретико-категорными конструкциями и~обозначениями, 
введенными в~работах~\cite{5-kov, 13-kov}. Каталоги моделей системных 
единиц описываются подходящими категориями. В~свою очередь, все 
возможные модели любых системных единиц в~представлении, выражающем 
некоторую фиксированную точку зрения на систему, также описываются 
категорией, которую будем обозначать через~$C$. Например, при 
пред\-став\-ле\-нии\linebreak сис\-тем с~точки зрения пространственного расположения 
в~качестве~$C$ выступает категория твердотельных геометрических моделей 
\textbf{MBS}, содержащаяся в~категории множеств \textbf{Set}, а~при 
\mbox{представлении} поведения~--- категория дис\-крет\-но-со\-бы\-тий\-ных имитационных 
моделей \textbf{Pomset}, которая является конкретной категорией над 
\textbf{Set}.





Представление архитектуры системы с~точки зрения~$C$ ($C$-ЭСИ) сводится 
к~$C$-диаграмме, в~вершинах которой находятся представления СЧ, 
а~стрелки представляют действия по сборке. Пусть~$I$~--- форма (схема) этой 
диаграммы. На\-пом\-ним, что~$I$~--- это малая категория (и~через \textbf{Cat} 
обозначается полная подкатегория в~<<категории всех категорий>> 
\textbf{CAT}, состоящая из всех малых категорий). Каталоги моделей СЧ 
задаются семейством категорий $D_i$, $i \in \vert I\vert$ (напомним, что через  
$\vert I\vert$ обозначается множество вершин схемы~$I$ и~его можно 
рассматривать как дискретную подкатегорию в~$I$). Для каждой СЧ имеется 
правило пред\-став\-ле\-ния с~точки зрения~$C$, которое по соображениям 
корректности задается функтором вида $F_i \: : \: D_i\hm \to C$ для $i$-й СЧ (будем 
обозначать через~$F$ любое такое семейство функторов, индексированное 
множеством вершин схемы). Таким образом, архитектурная модель некоторой 
конкретной сис\-те\-мы получается, если выбрать по одному объекту $A_i \hm\in D_i$, 
$i \in \vert I\vert$, и~некоторую диаграмму~$\Delta \: : \: I\hm\to C$, 
удовлетворяющую условию $\Delta (i) \hm= F_i(A_i)$, $i \hm\in \vert I\vert$. Заметим, 
что точно такой же формальный вид имеют модели, выражающие комплексное 
представление архитектуры с~нескольких точек зрения одновременно: если 
отдельные точки зрения образуют множество~$Q$ и~им отвечают 
категории~$C_q$, $q\hm\in Q$, то комплексное представление описывается 
произведением категорий $\prod_{q \in Q}C_q$, которое и~выступает 
в~качестве~$C$.
{\looseness=1

}

Процедуры подбора и~замены СЧ в~ходе проек\-ти\-ро\-ва\-ния формально 
описываются преобразованиями архитектурных моделей, не изменяющими\linebreak ни 
схему деления, ни правила представления СЧ. Такими преобразованиями 
очевидным образом служат естественные преобразования диаграмм, 
индуцированные действиями из каталогов СЧ, а~именно: преобразованием 
модели (($A_i$, $i \hm\in \vert I\vert$), $\Delta$) в~модель (($A_i^\prime$,  
$i \hm\in \vert I\vert$), $\Delta^\prime$) является любое семейство морфизмов~$f_i \: : \: A_i \hm\to A^\prime_i$, 
$i \hm\in \vert I\vert$ (где каждый морфизм~$f_i$ принадлежит 
категории~$D_i$) такое, что для любых вершин схемы~$i$, $k \hm\in\vert I\vert$ 
и~стрелки~$h \: : \: i\hm \to k$ выполняется условие 
$$
F_k(f_k) \circ \Delta (h) = \Delta^\prime (h) \circ F_i(f_i)\,.
$$
Именно это условие выражает на языке теории категорий структурную 
целостность перехода между моделями, соответствующего замене  
СЧ~\cite{11-kov}.

Легко проверить, что для любых фиксированных $C$, $I$ и~$F$ совокупность 
всех архитектурных моделей и~всех их преобразований образует категорию. 
В~теории категорий давно известен один частный случай этой конструкции, 
где в~качестве~$I$ выбрана схема вида $0\hm\to1$. Категория архитектурных 
моделей для этого случая встречается во многих задачах, называется 
категорией запятой (comma category)~\cite[\S\,II.6]{13-kov} и~обозначается 
через $F_0 \downarrow F_1$. Поэтому будем называть произвольную категорию 
архитектурных моделей вышеописанного вида \textit{категорией 
мультизапятой} (multicomma) и~обозначать через $\downdownarrows_I F$. 
Пара $\langle I, F\rangle $ называется формой (shape) категории мультизапятой, 
а~категория~$C$~--- пред\-став\-ле\-ни\-ем (representation) категории мультизапятой.
{\looseness=-1

}

Примечательно, что категорию мультизапятой можно получить при помощи 
универсальных конструкций в~<<категории всех категорий>> \textbf{CAT}, 
а~именно: произведения, декартова квад\-ра\-та и~экс\-по\-ненты.

\smallskip

\noindent
\textbf{Теорема~1.}~\textit{Категория муль\-ти\-за\-пя\-той $\downdownarrows_I F$ 
изо\-морф\-на вершине (объекту, находящемуся в~левом верх\-нем углу) 
сле\-ду\-юще\-го декартова квад\-ра\-та в}~\textbf{CAT}:

{ \begin{center} %fig3
\vspace*{1pt}

\mbox{%
\epsfxsize=41.337mm 
\epsfbox{kov-3.eps}
}
\end{center}
\vspace*{3pt}

}

\noindent
Д\,о\,к\,а\,з\,а\,т\,е\,л\,ь\,с\,т\,в\,о\,.\ \  Проверяется непосредственно по правилам 
вы\-чис\-ле\-ния пределов и~экспонент в~\textbf{CAT}.~$\square$

\smallskip

Штриховые стрелки декартова квадрата из тео\-ре\-мы~1 задают два 
канонических <<забывающих>> функтора, определенных на любой категории 
мультизапятой. Первый функтор, заданный левой вертикальной стрелкой, 
извлекает из архитектурной модели набор всех СЧ. Он унивалентен (faithful), 
так что можно трактовать его как функтор выделения <<носителя>> модели, по 
аналогии с~функторами, выделяющими носитель у~алгебраических сис\-тем, 
топологических пространств и~т.\,п. Второй функтор, заданный верхней 
горизонтальной стрелкой, извлекает из модели пред\-став\-ле\-ние структуры 
сис\-те\-мы с~точ\-ки зрения~$C$ и~не имеет прямого аналога\linebreak\vspace*{-12pt}

\pagebreak

\noindent
 в~универсальной 
алгебре. Будем называть его \textit{функтором структуры} и~обозначать через~$\nabla_I^F$, 
так что 
$$
\nabla_I^F :\ \downdownarrows_I F \to C^I : ((A_i, i \in \vert I\vert), \Delta : I \to C) 
\mapsto \Delta \,.
$$


В свою очередь, экспонента $C^I$ допускает каноническое вложение 
в~категорию диаграмм \textbf{D}$C$, представляющую каталог всех 
формально возможных структур систем~\cite{12-kov}, с~точки зрения~$C$. 
При\linebreak помощи этого вложения и~функтора структуры можно представить многие 
процедуры проектирования в~виде функторов, определенных на категории 
мультизапятой. Рас\-смот\-рим в~качестве примера построение копредела 
диаграммы, формально опи\-сы\-ва\-ющее решение классической прямой задачи 
цифрового проектирования~--- формирование модели цельной системы 
известной структуры. Предположим, что любая $C$-диа\-грам\-ма со схемой~$I$ 
имеет копредел. Тогда имеется вложение~$lc$~: $C^I \hookrightarrow LC$, где 
через $LC$ обозначена полная подкатегория в~\textbf{D}$C$, состоящая из 
всех диаграмм, име\-ющих копредел. Вычисление копредела задается функтором 
colim~: $LC \hm\to C$. Таким образом, получается функтор пред\-став\-ле\-ния 
процедуры сборки сис\-тем формы~$I$ с~точ\-ки зрения~$C$:
$$
\mbox{colim}  \circ lc \circ \nabla_I^F :\ \downdownarrows_I F \to C\,.
$$


А для решения обратных задач проектирования киберфизических сис\-тем 
категория $\downdownarrows_I F$ служит естественным <<строительным 
материалом>> для пространства проектирования, поскольку целевые функции, 
определяющие степень соответствия вариантов архитектуры сис\-те\-мы 
требованиям, можно задавать функторами на таком пространстве. 
Действительно, областью значения целевой функции\linebreak
 всегда является линейно 
упорядоченное множество, а~его, как хорошо известно, можно представить  
категорией~\cite[\S\,I.2]{13-kov}: объектами такой\linebreak
 категории служат все 
элементы множества, а~морфизмами~--- все пары ($x, y$) такие, что $x \leq y$ 
(так что между любыми двумя объектами имеется не более одного морфизма). 
Интересна ситуация, когда целевая функция выступает функцией объектов 
функтора, действующего в~такую категорию из нетривиальной (достаточно 
богатой морфизмами) подкатегории в~$\downdownarrows_I F$ или 
в~двойственной категории $(\downdownarrows_I F)^{\mathrm{op}}$. В~этом 
случае можно применить оптимизационные алгоритмы типа градиентного 
спуска, выполняющие навигацию вдоль морфизмов этой подкатегории, 
с~расчетом пути методами компьютерной алгебры.

Обратимся к~задачам проектирования СС. Прямые задачи, пред\-по\-ла\-га\-ющие 
<<подъем>> конструкций в~моделях на уровень структур, могут быть решены 
в~категории мультизапятой <<почленно>>.\linebreak Например, рассмотрим~$I$ как 
схему, вершины которой представляют членов некоторой группы однотипных 
киберфизических объектов, архитектура которых характеризуется формой  
$\langle K, G_k \: : \: D_k \hm\to C$, $k \hm\in \vert K\vert\rangle$, а~стрелки представляют 
внутригрупповые коллаборационные связи. Предположим, что во всех 
категориях $D_k$, $k \hm\in \vert K\vert$, все диаграммы формы~$I$ имеют 
копределы и~все функторы $G_k$ сохраняют их. Тогда произвольная  
($\downdownarrows_K G$)-диа\-грам\-ма формы~$I$, пред\-став\-ля\-ющая 
возможную структуру группы, также имеет копредел: компоненты его носителя 
строятся по от\-дель\-ности на объектах и~морфизмах каждой СЧ $D_k$, затем 
переносятся в~$C$ посредством функторов $G_k$, после чего дополняются 
универсальными стрелками копределов до $C$-диа\-грам\-мы формы~$K$, 
образуя ($\downdownarrows_K G$)-объект, представляющий про\-ек\-ти\-ру\-емую 
группу как единый цельный объект.

Конструкция мультизапятой хорошо подходит и~для решения обратных задач 
проектирования СС, поскольку ведет себя естественно относительно 
процедуры отрисовки диаграмм, со\-сто\-ящих из диаграмм. Действительно, 
формально структура СС описывается диаграммой, в~вершинах которой 
находятся сис\-те\-мы, структуры которых в~свою очередь описываются 
подходящими диаграммами (вообще говоря, имеющими различные схемы). Тем 
самым схема СС задается диаграммой вида~$\Xi \: : \: I \hm\to \mathbf{Cat}$. 
Отрисовка такой диаграммы~--- это преобразование в~малую категорию, 
порожденную заменой каждой\linebreak вершины $i \hm\in \vert I\vert$ схемой~$\Xi (i)$, 
а~каждой стрелки $h \: : \: i \hm\to l$~--- совокупностью стрелок, по одной для 
каждой вершины~$s$ схемы~$\Xi (i)$, на\-прав\-лен\-ной из~$s$\linebreak в~вершину~$\Xi (h)(s)$ 
схемы~$\Xi (l)$, с~наложением подходящих условий 
естественности. Вместе с~\textbf{Cat}-диа-\linebreak грам\-ма\-ми можно отрисовывать 
и~их морфизмы, так что имеется функтор отрисовки~\textbf{K}$_{\text{\bfseries\textit{1}}} \: : \: \mathbf{D(Cat)} 
\hm\to \mathbf{Cat}$ (это част\-ный случай общей конструкции 
отрисовки~--- умножения в~монаде диаграмм~\textbf{D}~\cite{12-kov}).

\smallskip

\noindent
\textbf{Теорема~2.}\ \textit{Пусть заданы произвольные схема~$I$, 
диаграмма~$\Xi \: : \: I \hm\to \mathbf{Cat}$ и~семейство функ\-то\-ров~$G^{(i)}_k \: : \:  
D^{(i)}_k \hm\to C$, $k \hm\in \vert \Xi (i)\vert$, $i \hm\in \vert I\vert$. Категория 
муль\-ти\-за\-пя\-той $\downdownarrows_{\mathbf{K}_{\text{\bfseries\textit{1}}}\Xi} G$ изо\-морф\-на вершине 
сле\-ду\-юще\-го декартова квад\-ра\-та в}~\textbf{CAT}:

{ \begin{center} %fig4
\vspace*{-1pt}

\mbox{%
\epsfxsize=63.716mm 
\epsfbox{kov-4.eps}
}
\vspace*{3pt}

\end{center}
}

\noindent
Д\,о\,к\,а\,з\,а\,т\,е\,л\,ь\,с\,т\,в\,о\,.\ \ Сформируем <<почленное>> произведение 
декартовых квадратов, за\-да\-ющих согласно тео\-ре\-ме~1 все категории 
мультизапятой $\downdownarrows_{\Xi (i)} G^{(i)}$, $i \in \vert I\vert$. Если 
заменить в~нем правое вертикальное реб\-ро экспонентой суммы, то можно 
пристроить его снизу к~декартову квад\-ра\-ту, указанному в~условии тео\-ре\-мы:

{ \begin{center} %fig5
\vspace*{3pt}
\mbox{%
\epsfxsize=69.179mm 
 \epsfbox{kov-5.eps}
}
\end{center}
\vspace*{3pt}

}

\noindent
Здесь наружный прямоугольник является  
декартовым~\cite[предложение 11.10]{14-kov}. А~поскольку $\vert 
\mathbf{K}_{\text{\bfseries\textit{1}}} \Xi\vert \cong \coprod_{i \in \vert I\vert} \vert \Xi (i)\vert$ для любой 
\textbf{Cat}-диа\-грам\-мы~$\Xi$ со схемой~$I$ (отрисовка не добавляет и~не 
удаляет объекты), по теореме~1 объект в~левом верхнем углу действительно 
изоморфен категории мультизапятой $\downdownarrows_{\mathbf{K}_{\text{\bfseries\textit{1}}}\Xi} G$.~$\square$

\section{Заключение}

Теория категорий обладает большим потенциалом применения в~технологиях 
четвертой промышленной революции, в~том числе в~цифровом 
проектировании гетерогенных киберфизических\linebreak \mbox{систем}. В~настоящее время 
возможности применения предложенных тео\-ре\-ти\-ко-ка\-те\-гор\-ных 
методов исследуются на макете программного инструмента разработки 
цифровых двойников энергетических \mbox{систем}~[15]. В~ходе развития 
инструмента до промышленного уровня го\-тов\-ности возникнет много новых 
задач для дальнейших исследований.

%\vspace*{-8pt}

{\small\frenchspacing
{%\baselineskip=10.8pt
%\addcontentsline{toc}{section}{References}
\begin{thebibliography}{99}

%\vspace*{-2pt}

\bibitem{1-kov} 
\Au{Tao~F., Qi~Q., Wang~L., Nee~A.\,Y.\,C.} Digital twins and cyber-physical 
systems toward smart manufacturing and Industry~4.0: Correlation and comparison~// 
Engineering, 2019. Vol.~5. P.~653--661.
\bibitem{2-kov} 
\Au{Kowalski~J.} CAD is a~lie: Generative design to the rescue.~--- San Rafael, CA, 
USA: Autodesk, 2016. {\sf https:// www.autodesk.com/redshift/generative-design}.
\bibitem{3-kov} 
\Au{Baez~J.\,C., Erbele~J.} Categories in control~// Theor. Appl. 
Categ., 2015. Vol.~30. Iss.~24. P.~836--881.
\bibitem{4-kov} 
\Au{Wisnesky~R., Breiner~S., Jones~A., Spivak~D.\,I., Subrahmanian~E.} Using 
category theory to facilitate multiple manufacturing service database integration~// 
J.~Comput. Inf. Sci. Eng., 2017. Vol.~17. Iss.~2. Art. 
ID:~021011.
\bibitem{5-kov} 
\Au{Ковалёв~С.\,П.} Методы теории категорий  
в~мо\-дель\-но-ори\-ен\-ти\-ро\-ван\-ной системной инженерии~// Информатика 
и~её применения, 2017. Т.~11. Вып.~3. С.~42--50.
\bibitem{6-kov} 
ГОСТ Р 57100-2016/ISO/IEC/IEEE 42010:2011. Сис\-тем\-ная и~программная 
инженерия. Описание архитектуры.~--- М.: Стандартинформ, 2016. 32~с.
\bibitem{7-kov} 
\Au{Gherghina~G., Tutunea~D., Popa~D.} About digital mock-up for mechanical 
products~// J.~Industrial Design Engineering Graphics, 2015. Vol.~10. No.\,2. 
P.~19--22.
\bibitem{8-kov} 
\Au{Рафальский~В.\,В., Рафальская~Л.\,Г., Старостина~А.\,В.} 
Информационная модель схемы деления~// Автоматизация процессов 
управления, 2009. №\,3. С.~22--28.
\bibitem{9-kov} 
What is a~smart building and how can it benefit you?~--- Milford, MA, USA: 
Comark, 2016. {\sf https://comarkcorp. com/smart-building-can-benefit.}
\bibitem{10-kov} 
ГОСТ 2.053-2013. Единая система конструкторской документации. 
Электронная структура изделия. Общие положения.~--- М.: Стандартинформ, 
2014. 10~с.
\bibitem{11-kov} 
\Au{Ковалёв~С.\,П.} Алгебраическое моделирование жизненного цикла 
круп\-но\-мас\-штаб\-ных гетерогенных сис\-тем в~аспектах~// Управление развитием 
круп\-но\-мас\-штаб\-ных сис\-тем: Мат-лы X~Междунар. 
конф.~--- М.: ИПУ РАН, 2017.  Т.~II. С.~266--268.

\bibitem{13-kov} 
\Au{Маклейн~С.} Категории для работающего математика~/ Пер. с~англ.~--- 
М.: Физматлит, 2004. 352~с. (\Au{Mac Lane~S.} Categories for the working 
mathematician.~--- New York, NY, USA: Springer, 1978. 317~p.)
\bibitem{12-kov} 
\Au{Ковалёв~С.\,П.} Теория категорий как математическая прагматика  
мо\-дель\-но-ори\-ен\-ти\-ро\-ван\-ной сис\-тем\-ной инженерии~// Информатика 
и~её применения, 2018. Т.~12. Вып.~1. С.~95--104.

\bibitem{14-kov} 
\Au{Ad$\acute{\mbox{a}}$mek~J., Herrlich~H., Strecker~G.\,E.} Abstract and 
concrete categories.~--- New York, NY, USA: John Wiley, 1990. 507~p.
\bibitem{15-kov} 
\Au{Ковалёв~С.\,П.} Проектирование информационного обеспечения цифровых 
двойников энергетических систем~// Системы и~средства информатики, 2020. 
Т.~30. №\,1. С.~66--81.
\end{thebibliography}

}
}

\end{multicols}

\vspace*{-3pt}

\hfill{\small\textit{Поступила в~редакцию 12.10.2019}}

%\vspace*{8pt}

%\pagebreak

\newpage

\vspace*{-30pt}

%\hrule

%\vspace*{2pt}

%\hrule

%\vspace*{-2pt}


\def\tit{METHODS OF~THE~CATEGORY THEORY IN~DIGITAL DESIGN 
OF~HETEROGENEOUS CYBER-PHYSICAL SYSTEMS\\[-4pt]}

\def\titkol{Methods of~the~category theory in~digital design of~heterogeneous 
cyber-physical systems}

\def\aut{S.\,P.~Kovalyov\\[-4pt]}

\def\autkol{S.\,P.~Kovalyov}

\titel{\tit}{\aut}{\autkol}{\titkol}

\vspace*{-20pt}

\noindent
V.\,A.~Trapeznikov Institute of Control Sciences, Russian Academy of Sciences, 
65~Profsoyuznaya Str., Moscow 117997, Russian Federation


\def\leftfootline{\small{\textbf{\thepage}
\hfill INFORMATIKA I EE PRIMENENIYA~--- INFORMATICS AND
APPLICATIONS\ \ \ 2021\ \ \ volume~15\ \ \ issue\ 1}
}%
\def\rightfootline{\small{INFORMATIKA I EE PRIMENENIYA~---
INFORMATICS AND APPLICATIONS\ \ \ 2021\ \ \ volume~15\ \ \ issue\ 1
\hfill \textbf{\thepage}}}

\vspace*{2pt} 

\Abste{A~mathematical device built upon the category theory is developed which 
was previously proposed to formally describe and rigorously explore engineering 
procedures based on mathematical and computer modeling. With the help of the 
device, highly automated procedures for designing heterogeneous cyber-physical 
systems on top of digital twins, demanded by the upcoming fourth industrial 
revolution, are described and explored. For this purpose, the novel construction of the 
multicomma category is introduced, whose objects are the architectural models of 
a~heterogeneous cyber-physical system with a~certain fixed structural hierarchy 
scheme represented from a~certain architecture viewpoint, and morphisms describe 
actions associated with selection of constituents for assembling a~system from them. 
The application of the multicomma category in solving direct and inverse problems 
of designing individual systems and complex systems of systems is considered.}

%\vspace*{2pt}

\KWE{cyber-physical system; digital twin; generative design; system of systems; 
category theory; multicomma category}

\DOI{10.14357/19922264210104}

\vspace*{-6pt}

\begin{multicols}{2}

\renewcommand{\bibname}{\protect\rmfamily References}
%\renewcommand{\bibname}{\large\protect\rm References}

{\small\frenchspacing
{%\baselineskip=10.8pt
\addcontentsline{toc}{section}{References}
\begin{thebibliography}{99}
\bibitem{1-kov-1} 
\Aue{Tao,~F., Q.~Qi, L.~Wang, and A.\,Y.\,C.~Nee.} 2019. Digital twins and  
cyber--physical systems toward smart manufacturing and Industry~4.0: Correlation 
and comparison. \textit{Engineering} 5:653--661.
\bibitem{2-kov-1} 
CAD is a~lie: Generative design to the rescue. Available at: {\sf 
https://www.autodesk.com/redshift/generative-design/} (accessed December~9, 
2020).
\bibitem{3-kov-1} 
\Aue{Baez,~J.\,C., and J.~Erbele}. 2015. Categories in control. \textit{Theor.  
Appl. Categ.} 30(24):836--881.
\bibitem{4-kov-1} 
\Aue{Wisnesky,~R., S.~Breiner, A.~Jones, D.\,I.~Spivak, and E.~Subrahmanian.} 
2017. Using category theory to facilitate multiple manufacturing service database 
integration. \textit{J.~Comput. Inf. Sci. Eng.} 17(2):021011.
\bibitem{5-kov-1} 
\Aue{Kovalyov,~S.\,P.} 2017. Metody teorii kategoriy v model'no-orientirovannoy 
sistemnoy inzhenerii [Methods of category theory in model-based systems 
engineering]. \textit{Informatika i~ee Primeneniya~--- Inform. Appl.} 11(3):42--50.
\bibitem{6-kov-1} 
GOST R ISO/IEC/IEEE 42010:2011. 2016. Sistemnaya i~programmnaya inzheneriya. 
Opisanie arkhitektury [System and software engineering. Description of 
architecture]. Moscow: Standardinform Publs. 32~p.
\bibitem{7-kov-1} 
\Aue{Gherghina,~G., D.~Tutunea, and D.~Popa.} 2015. About digital mock-up for 
mechanical products. \textit{J.~Industrial Design Engineering Graphics} 
10(2):19--22.
\bibitem{8-kov-1} 
\Aue{Rafal'skiy,~V.\,V., L.\,G.~Rafal'skaya, and A.\,V.~Starostina}. 2009. 
Informatsionnaya model' skhemy deleniya [Information model of decomposition scheme]. 
\textit{Avtomatizatsiya protsessov upravleniya} [Automation of Control 
Processes] 3:22--28.
\bibitem{9-kov-1} 
What is a~smart building and how can it benefit you? 2016. Milford, MA: Comark.
Available at: {\sf 
https://comarkcorp. com/smart-building-can-benefit/} (accessed December~9, 2020).
\bibitem{10-kov-1} 
GOST 2.053-2013. 2014. Edinaya sistema konstruktorskoy dokumentatsii. 
Elektronnaya struktura izdeliya. Obshchie polozheniya [Unified system for design 
documentation. Electronic structure of the product. General Provisions]. Moscow: 
Standardinform Publs. 10~p.
\bibitem{11-kov-1} 
\Aue{Kovalyov,~S.\,P.} 2017. Algebraicheskoe modelirovanie zhiznennogo tsikla 
krupnomasshtabnykh geterogennykh sistem v~aspektakh [Aspectwise algebraic 
modeling of large-scale heterogeneous systems life cycle].
 \textit{10th Conference (International) ``Management of Large-Scale Systems 
Development'' Proceedings}. Moscow. 2:266--268.

\bibitem{13-kov-1} 
\Aue{Mac Lane, S.} 1978. \textit{Categories for the working mathematician}. New 
York, NY: Springer. 317~p.

\bibitem{12-kov-1} 
\Aue{Kovalyov, S.\,P.} 2018. Teoriya kategoriy kak ma\-te\-ma\-ti\-che\-skaya pragmatika 
mo\-del'\-no-ori\-yen\-ti\-ro\-van\-noy sis\-tem\-noy inzhenerii [Category theory as a~mathematical 
pragmatics of model-based systems engineering]. \textit{Informatika i~ee 
Primeneniya~--- Inform. Appl.} 12(1):95--104.

\bibitem{14-kov-1} 
\Aue{Ad$\acute{\mbox{a}}$mek,~J., H.~Herrlich, and G.\,E.~Strecker.} 1990. 
\textit{Abstract and concrete categories}. New York, NY: John Wiley. 507~p.
\bibitem{15-kov-1} 
\Aue{Kovalyov,~S.\,P.} 2020. Proektirovanie informatsionnogo obespecheniya 
tsifrovykh dvoynikov energeticheskikh sistem [Information architecture of the power 
system digital twin]. \textit{Sistemy i~Sredstva Informatiki~--- Systems and Means of 
Informatics} 30(1):66--81.
\end{thebibliography}

}
}

\end{multicols}

\vspace*{-8pt}

\hfill{\small\textit{Received October~22, 2019}}


%\pagebreak

\vspace*{-26pt}

\Contrl

\vspace*{-4pt}

\noindent
\textbf{Kovalyov Sergey P.} (b.\ 1972)~--- Doctor of Science in physics and 
mathematics, leading scientist, V\, A.~Trapeznikov Institute of Control Sciences, 
Russian Academy of Sciences, 65~Profsoyuznaya Str., Moscow 117997, Russian 
Federation; \mbox{kovalyov@sibnet.ru}
\label{end\stat}

\renewcommand{\bibname}{\protect\rm Литература} 