\documentclass[10pt]{book}
\usepackage[utf8]{inputenc}

\usepackage{latexsym,amssymb,amsfonts,amsmath,amsxtra,dsfont,
indentfirst,shapepar,%fleqn,%
picinpar,shadow,floatflt,enumerate,multicol,colortbl,moreverb,cite,ipi}

\usepackage{rotating}
\usepackage{mathrsfs}
\usepackage[noend]{algorithmic}
\usepackage{ulem}
\usepackage{graphicx}
%\usepackage{algorithm2e}
\usepackage[linesnumbered,boxed,ruled]{algorithm2e}
%\usepackage{xypic}
\usepackage{oldgerm}
\usepackage{epic}
\usepackage{eepic}

\SetAlgorithmName{Algorithm}{алгоритм}{Список алгоритмов}

%из Дюковой

\newcommand{\algKeyword}[1]{{\bf #1}}
\newcommand{\Proc}[1]{\text{\tt #1}}
\def\CALL{\algKeyword{call}~}

\newenvironment{AlgProcedure}[1]
{
\small
\medskip
%    \hrule
\medskip
\algKeyword{PROCEDURE} #1
\begin{algorithmic}[1]}
{\end{algorithmic}
%    \hrule
\bigskip
}

\def\CALL{\algKeyword{call}~}

%конец для Дюковой

%\RequirePackage[ruled]{algorithm}


\input{epsf}

%\nofiles

%\includeonly{avtor}             %+pdf+
%\includeonly{obchak,avtor}
%\includeonly{pred}                 %+
%\includeonly{podgot-rus-site,podgot-eng-site}  
%\includeonly{podgot-rus,podgot-eng}  
%\includeonly{ocherk} 
%\includeonly{ipi-ind} 
%\includeonly{index14}
%\includeonly{toc-rus, toc-en}
%\includeonly{toc-rus}
%\includeonly{toc-en} 


      
%\includeonly{agalarov}                  %+pdf+авт+
%\includeonly{kudr}                      %+pdf+авт+
%\includeonly{sinits}                    %+pdf+авт+
%\includeonly{pavlov}                    %+pdf+авт+
%\includeonly{lipatyev}                  %+pdf+авт+
%\includeonly{zatsman}                   %+pdf+авт+
%\includeonly{kovalev}                   %+pdf+авт+
%\includeonly{stup-bruhov}               %+pdf+авт+
%\includeonly{flerov}               %pdf
%\includeonly{bosov}                     %pdf+авт+
%\includeonly{hvatova}                   %pdf+авт+
%\includeonly{gonch-zats}                %pdf+авт+
%\includeonly{strijov}                   %pdf+авт+
%\includeonly{chehovich}                 %pdf+авт+
%\includeonly{dorofeeva}                 %pdf+авт+
%\includeonly{grinchenko}      %pdf





%\includeonly{obchak}
%\includeonly{rekl}
%\includeonly{rekl-1}
%\includeonly{reshal}  %
%\includeonly{cover3}

\usepackage{acad}
%\usepackage{courier}
\usepackage{decor}
\usepackage{newton}
\usepackage{pragmatica}
\usepackage{zapfchan}
\usepackage{petrotex}
\usepackage{bm}                     % полужирные греческие буквы
\usepackage{upgreek}                % прямые греческие буквы \upalpha
\usepackage{eufrak}
\usepackage{verbatim}

\renewcommand{\bottomfraction}{0.99}
\renewcommand{\topfraction}{0.99}
\renewcommand{\textfraction}{0.01}

\setcounter{secnumdepth}{1} %здесь - 3 + chapter = 4

\arraycolsep=1.5pt

%\usepackage[pdftex]{graphicx}

%\usepackage{oz}

%NEW COMMANDS


\renewcommand*{\hm}[1]{#1\nobreak\discretionary{}%
            {\hbox{$\mathsurround=0pt #1$}}{}} %% Дублирует знаки операций
                               %при переносе в формуле (перед знаком, который
                               %надо продублировать ставится команда \hm)

%\newcommand{\endproof}{\hfill$\Box$}
\renewcommand{\r}{\mathbb{R}}
%\newcommand{\I}{{\rm I\hspace{-0.7mm}I}}
%\newcommand{\Ikl}{{\tt{1}}\hspace*{-1.44mm}\mathtt{1}}
\newcommand{\Ik}{\mbox{{\small \tt {1}}\hspace{-1.3mm}{\tt 1}}}
\newcommand{\argmin}{\mathop{\mathrm{arg}\,\mathrm{min}}}
\newcommand{\argmax}{\mathop{\mathrm{arg}\,\mathrm{max}}}
%\newcommand{\capr}{\mathop{\cap\,}}
%\newcommand{\cupr}{\mathop{\cup\,}}
%\def\argmin{\mathop{arg\,min}}

\def\vrp{\varphi}
\def\prt{\partial}
\def\mm{{\sf M}}
\def\modnop#1{\mathop{#1}\limits_{n}}
\def\eam{\mathbin{{\mathop{=}\limits^{\mathrm{def}}}}}
\def\dey#1#2{#1 (#2)}
\def\deyc#1#2{#1 \cdot  #2}
\def\ra#1{\;\mathop{\to}\limits^{#1}\;}
\def\raz#1{\;\mathop{\longrightarrow}\limits^{\!\!\!#1}\;}
\def\ral#1{\;\mathop{\longrightarrow}\limits^{#1}\;}

\newcommand{\Nor}{\mathcal{N}}
\newcommand{\T}{\mathbb{T}}
\newcommand{\Z}{\mathbb{Z}}



\newcommand{\il}[2]{\int\limits_{#1}^{#2}}%интеграл с пределами #1 и #2

\def\sm2{\mathop {\sum\limits^{n^\Theta}\sum\limits^{n^\Theta}}}
\def\sss{\sum\limits}
\def\tr{,\,\ldots\,,\,}
\def\rk{\right]}
\def\lk{\left[}
\def\rf{\right\}}
\def\lf{\left\{}
\def\lv{\,\left\vert}
\def\rv{\right\vert\,}
\def\iii{\int\limits}
\def\iin{\int\limits_{-\infty}^\infty}
\def\rrv{\right\vert}


\def\ee{{\cal E}}
\def\ww{{\cal W}}
\def\yy{{\cal Y}}
\def\vv{{\cal V}}

\newcommand{\R}{\mathbb R}
\newcommand{\E}{\mathbb E}
\newcommand{\N}{\mathbb N}

\renewcommand{\P}{\mathbb{P}}

\newcommand{\h}{{\bf H}}
\newcommand{\p}{{\sf P}}  % вероятность

\newcommand{\e}{{\sf E}}  % мат. ожидание
\newcommand{\D}{{\sf D}}  % дисперсия
\newcommand{\eps}{\varepsilon}
\newcommand{\vp}{{\mathbf p}}
\newcommand{\vz}{{\mathbf z}}
\newcommand{\vx}{{\mathbf x}}
\newcommand{\vf}{{\mathbf f}}
\newcommand{\F}{{\mathcal F}}
\def\ap{{\mathrm{ЭР}}}
\newcommand{\ud}{\Delta_n} %uniform ditance
\newcommand{\nud}{\Delta_n(x)}
%\renewcommand{\Re}{\mathrm{Re}\,}

\newcommand{\abs}[1]{\left\vert#1\right\vert}

\newcommand{\norm}[1]{\left\Vert#1\right\Vert}
\def\da{(\Delta_t,A)}

\newcommand{\corr}{\mathrm{corr}}

\newcommand{\cov}{\mathrm{cov}}
\newcommand{\Expect}{\mathbb{E}}

\def\w{\omega}
\def\W{\Omega}

\def\inh{\int\limits_{nh}^{(n+1)h}}

\def\sumin{\sum_{i=1}^N}


\def\bxt{(Y,t)}
\def\xt{(y,t)}

\def\ovth{{\fr{\tau-nh}{h}}}
\def\ov{\overline}
\def\tm{\tilde m}
\def\tl{\tilde\lambda}
\def\tB{\widetilde B}
\def\tb{\tilde b}
\def\ld{\ldots}
\def\cd{\cdots}


\DeclareMathOperator{\sign}{sign}

%\newcommand{\gr}{{\geqslant}}


\newcommand{\g}{\mbox{\textit{g}}}

\renewcommand{\la}{\lambda}
\newcommand{\si}{\sigma}
\newcommand{\alp}{\alpha}

\newcommand{\pto}{\stackrel{P}{\longrightarrow}} % сходимость по веpоятности

\newcommand{\eqd}{\stackrel{\mathrm{d}}{=}} % равенство по pаспpеделению
\newcommand{\eqdelta}{\stackrel{\triangle}{=}} % равенство по pаспpеделению

\def\be#1{\begin{equation}\label{#1}}
\def\ee{\end{equation}}
\def\re#1{(\ref{#1})}

\def\bn{\begin{enumerate}}
\def\en{\end{enumerate}}
\def\bi{\begin{itemize}}
\def\ei{\end{itemize}}
%\def\i{\item}

%\newcommand{\kp}{\kappa}
%\def\Q{{\cal Q}} \def\H{{\cal H}}
%\newcommand{\bet}{\beta_{2+\delta}}


%\newtheorem{definition}{Определение}
%\renewcommand{\thedefinition}{\arabic{definition}.}
%END NEW COMMANDS

%\renewcommand{\baselinestretch}{1.2}

%\pagestyle{myheadings}

\setlength{\textwidth}{167mm}      % 122mm
\setlength{\textheight}{658pt}
%\setlength{\textheight}{635.6pt}
\setlength{\columnsep}{4.5mm}

\setcounter{secnumdepth}{4}

%\addtolength{\headheight}{2pt}
%\addtolength{\headsep}{-2mm}

\addtolength{\topmargin}{-7mm}  % for printing


%\hoffset=-30mm  % From Yap
\hoffset=-23mm  % From Acrobat

%\voffset=0mm % From Yap
\voffset=-5mm   % From Acrobat

%\addtolength{\evensidemargin}{-2.5mm} % for printing
%\addtolength{\oddsidemargin}{2.5mm}  % for printing

\addtolength{\evensidemargin}{-12mm} % for printing
\addtolength{\oddsidemargin}{8mm}  % for printing

%\renewcommand{\thefootnote}{\fnsymbol{footnote}}
%\renewcommand{\thefootnote}{\arabic{footnote}}
\renewcommand{\figurename}{\protect\bf Рис.}
\renewcommand{\tablename}{\protect\bf Таблица}

\newcommand{\Caption}[1]{\caption{\protect\small %\baselineskip=2.5ex
#1}}

\renewcommand{\thefigure}{\arabic{figure}}
\renewcommand{\thetable}{\arabic{table}}
\renewcommand{\theequation}{\arabic{equation}}
\renewcommand{\thesection}{\arabic{section}}

\renewcommand{\contentsname}{СОДЕРЖАНИЕ}
\newcommand{\fr}[2]{\displaystyle\frac{\displaystyle #1\mathstrut}{\displaystyle #2\mathstrut}}

%\renewcommand{\thefootnote}{\fnsymbol{footnote}}
%\newcommand{\g}{\mbox{\textit{g}}}

%\newcommand{\Caption}[1]{\caption{\protect\small\baselineskip=2ex #1}}
\newcounter{razdel}
\setcounter{razdel}{0}

\def\god{2021}
\def\tom{15}
\def\vyp{1}


\newcommand{\titel}[4]{%
\

\vspace*{5pt}

\ifodd\therazdel {\raggedright\noindent\Large\textrm\textbf
 \lineskip .75em
  \baselineskip=3.2ex #1 \par}
\vskip 1em {\noindent\large\textrm\textbf #2 \par}
\addcontentsline{toc}{subsection}{{\textrm\textbf #1}\protect\newline #2}
\def\rightheadline{\underline{\noindent\hbox to \textwidth{\hfill\small\textrm{#4}
%\hfill \large\bf\thepage
}}}
\def\leftheadline{\underline{\noindent\parbox{\textwidth}{
%\raggedleft\large\bf\thepage \hfill
\small\textit{#3}\hfill}}}
\def\leftfootline{\small{\textbf{\thepage}
\hfill ИНФОРМАТИКА И ЕЁ ПРИМЕНЕНИЯ\ \ \ том~\tom\ \ \ выпуск~\vyp\ \ \ \god}
}%
 \def\rightfootline{\small{ИНФОРМАТИКА И ЕЁ ПРИМЕНЕНИЯ\ \ \ том~\tom\ \ \ выпуск~\vyp\ \ \ \god
\hfill \textbf{\thepage}}}
\vskip 2em \setcounter{figure}{0}
\setcounter{table}{0}
\setcounter{equation}{0}
\setcounter{section}{0}
\setcounter{subsection}{0}
\setcounter{subsubsection}{0}
\setcounter{footnote}{0}
\setcounter{razdel}{0}
%\end{flushleft}
\else {
 \raggedright\noindent\Large\textrm\textbf
 \lineskip .75em
\baselineskip=3.2ex #1 \par} \vskip 1em
%\begin{flushleft}
{\noindent\large\textrm\textbf #2 \par}
\addcontentsline{toc}{subsection}{{\textrm\textbf #1}\protect\newline #2}
\def\rightheadline{\underline{\noindent\hbox to \textwidth{\hfill\small\textrm{#4}
%\hfill \large\bf\thepage
}}}
\def\leftheadline{\underline{\noindent\parbox{\textwidth}{%\raggedleft\large\bf\thepage \hfill
\small\textit{#3}\hfill}}}
\def\leftfootline{\small{\textbf{\thepage}
\hfill ИНФОРМАТИКА И ЕЁ ПРИМЕНЕНИЯ\ \ \ том~\tom\ \ \ выпуск~\vyp\ \ \ \god}
}%
 \def\rightfootline{\small{ИНФОРМАТИКА И ЕЁ ПРИМЕНЕНИЯ\ \ \ том~15\ \ \ выпуск~\vyp\ \ \ 2021
\hfill \textbf{\thepage}}} \vskip 2em \setcounter{figure}{0}
\setcounter{table}{0} \setcounter{equation}{0} \setcounter{section}{0}
\setcounter{subsection}{0} \setcounter{subsubsection}{0}
\setcounter{footnote}{0}
%\end{flushleft}
\fi}

\newcommand{\titelr}[2]{%
\

\vspace*{5pt}

\ifodd\therazdel {\raggedright\noindent%\Large\textrm\textbf
 \lineskip .75em
  \baselineskip=3.2ex #1 \par}
\vskip 1em {\noindent\normalsize\textrm\textbf #2 \par}
\else {
 \raggedright\noindent\Large\textrm\textbf
 \lineskip .75em
\baselineskip=3.2ex #1 \par} \vskip 1em
%\begin{flushleft}
{\noindent\large\textrm\textbf #2 \par
%\noindent\normalsize\textrm\textbf #2 \par
} \fi}

\newcommand{\titele}[5]{%
\

%\vspace*{5pt}

\ifodd\therazdel {\raggedright\noindent\large
\textrm\textbf
 \lineskip .75em
%  \baselineskip=3.2ex
#1 \par}
\vskip .5em {\noindent\large\textrm\textbf #2 \par}
\vskip .5em
 {\noindent\textrm #3 \par}
\addcontentsline{toc}{subsection}{{\textrm\textbf #1}\protect\newline #2}
\def\rightheadline{\underline{\noindent\hbox to \textwidth{\hfill\small\textrm{#4}
%\hfill \large\bf\thepage
}}}
\def\leftheadline{\underline{\noindent\parbox{\textwidth}{
%\raggedleft\large\bf\thepage \hfill
\small\textrm{#5}\hfill}}}
\def\leftfootline{\small{\textbf{\thepage}
\hfill ИНФОРМАТИКА И ЕЁ ПРИМЕНЕНИЯ\ \ \ том~15\ \ \ выпуск~1\ \ \ 2021}
}%
 \def\rightfootline{\small{ИНФОРМАТИКА И ЕЁ ПРИМЕНЕНИЯ\ \ \ том~15\ \ \ выпуск~1\ \ \ 2021
\hfill \textbf{\thepage}}} \vskip 1em \setcounter{figure}{0}
\setcounter{table}{0} \setcounter{equation}{0} \setcounter{section}{0}
\setcounter{subsection}{0} \setcounter{subsubsection}{0}
\setcounter{footnote}{0} \setcounter{razdel}{0}
%\end{flushleft}
\else {
 \raggedright\noindent\large
 \textrm\textbf
 \lineskip .75em
%\baselineskip=3.2ex
#1 \par} \vskip .5em
%\begin{flushleft}
{\noindent\large\textrm\textbf #2 \par} \vskip .5em
 {\noindent\textrm #3 \par}
\addcontentsline{toc}{subsection}{{\textrm\textbf #1}\protect\newline #2}
\def\rightheadline{\underline{\noindent\hbox to \textwidth{\hfill\small\textrm{#4}
%\hfill \large\bf\thepage
}}}
\def\leftheadline{\underline{\noindent\parbox{\textwidth}{%\raggedleft\large\bf\thepage \hfill
\small\textrm{#5}\hfill}}}
\def\leftfootline{\small{\textbf{\thepage}
\hfill ИНФОРМАТИКА И ЕЁ ПРИМЕНЕНИЯ\ \ \ том~15\ \ \ выпуск~1\ \ \ 2021}
}%
 \def\rightfootline{\small{ИНФОРМАТИКА И ЕЁ ПРИМЕНЕНИЯ\ \ \ том~15\ \ \ выпуск~1\ \ \ 2021
\hfill \textbf{\thepage}}} \vskip 1em \setcounter{figure}{0}
\setcounter{table}{0} \setcounter{equation}{0} \setcounter{section}{0}
\setcounter{subsection}{0} \setcounter{subsubsection}{0}
\setcounter{footnote}{0}
%\end{flushleft}
\fi}

\def\Abst#1{
\begin{center}\small\nwt
\parbox{150mm}{%\baselineskip=2.5ex
\textbf{Аннотация:}\ \
%\hspace*{\parindent}
#1}
\end{center}}
\def\Abste#1{
\begin{center}\small\nwt
\parbox{150mm}{%\baselineskip=2.5ex
\textbf{Abstract:}\ \
%\hspace*{\parindent}
#1}
\end{center}}

\def\DOI#1{
\begin{center}\small\nwt
\parbox{150mm}{%\baselineskip=2.5ex
\textbf{DOI:}\ \
%\hspace*{\parindent}
#1}
\end{center}}

\def\Abstend#1{
\begin{center}\small\nwt
\parbox{150mm}{%\baselineskip=2.5ex
%\hspace*{\parindent}
#1}
\end{center}}


\def\KW#1{
\begin{center}\small\nwt
\parbox{150mm}{%\baselineskip=2.5ex
\textbf{Ключевые слова:}\ \ #1}
\end{center}}

\def\KWE#1{
\begin{center}\small\nwt
\parbox{150mm}{%\baselineskip=2.5ex
\textbf{Keywords:}\ \ #1}
\end{center}}


\def\KWN#1{
%\begin{center}
%\small
%\parbox{150mm}\end{center}
}

\newcommand{\Avtors}[1]{%\smallskip
%\vspace*{.5pt}
\hangindent=23pt\noindent
%\nwt
{\bfseries#1}\
}


\renewcommand{\thesubsection}{\thesection.\arabic{subsection}\hspace*{-5pt}}
\renewcommand{\thesubsubsection}{\thesubsection\hspace*{5pt}.\arabic{subsubsection}\hspace*{-3pt}}

\newcommand{\Ack}{\section*{\protect\rmfamily Acknowledgments}\noindent}
\newcommand{\Contr}{\section*{\protect\rmfamily Contributors}\noindent}
\newcommand{\Contrl}{\section*{\protect\rmfamily Contributor}\noindent}

\makeindex


\begin{document}
\Rus

\nwt
%\ptb


%\renewcommand{\contentsname}{\protect\Large\bf Содержание}

\setcounter{tocdepth}{2}

%\tableofcontents

\renewcommand{\bibname}{\protect\rmfamily Литература}
  \def\Au#1{{\it #1}}
    \def\Aue#1{{#1}}

%\newcommand{\No}{№}
  \newcommand{\tg}{\,\mathrm{tg}\,}
    \newcommand{\ctg}{\,\mathrm{ctg}\,}
  \newcommand{\arctg}{\,\mathrm{arctg}\,}

\def\forallb{\mathop{\forall}}
\def\cupb{\mathop{\cup}}
\def\existsb{\mathop{\exists}}


\newpage
\addtocounter{razdel}{1}
%\def\razd{РЕГУЛИРУЕМЫЙ ЭЛЕКТРОПРИВОД ДЛЯ ЭЛЕКТРОЭНЕРГЕТИКИ}


\setcounter{page}{3}

%   { %\Large  
   { %\baselineskip=16.6pt
   
   \vspace*{-48pt}
   \begin{center}\LARGE
   \textit{Предисловие}
   \end{center}
   
   %\vspace*{2.5mm}
   
   \vspace*{25mm}
   
   \thispagestyle{empty}
   
   { %\small 

    
Вниманию читателей журнала <<Информатика и её применения>> предлагается 
очередной тематический выпуск <<Вероятностно-статистические методы и 
задачи информатики и информационных технологий>>. Предыдущие тематические 
выпуски журнала по данному направлению вышли в 2008~г.\ (т.~2, вып.~2), 
в 2009~г.\ (т.~3, вып.~3) и в 2010~г.\ (т.~4, вып.~2). 

Статьи, собранные в данном журнале, посвящены разработке новых вероятностно-статистических 
методов, ориентированных на применение к решению конкретных задач информатики и информационных 
технологий, а также~--- в ряде случаев~--- и других прикладных задач. Проблематика, охватываемая 
публикуемыми работами, развивается в рамках научного сотрудничества между Институтом проблем 
информатики Российской академии наук (ИПИ РАН) и Факультетом вычислительной математики и 
кибернетики Московского государственного университета им.\ М.\,В.~Ломоносова в ходе работ 
над совместными научными проектами (в том числе в рамках функционирования 
Научно-образовательного центра <<Вероятностно-статистические методы анализа рисков>>). 
Многие из авторов статей, включенных в данный номер журнала, являются активными участниками 
традиционного международного семинара по проблемам устойчивости стохастических моделей, 
руководимого В.\,М.~Золотаревым и В.\,Ю.~Королевым; регулярные сессии этого семинара 
проводятся под эгидой МГУ и ИПИ РАН (в 2011~г.\ указанный семинар проводится в октябре 
в Калининградской области РФ). 

Наряду с представителями ИПИ РАН и МГУ в число авторов данного выпуска журнала входят 
ученые из Научно-исследовательского института системных исследований РАН, Института 
проблем технологии микроэлектроники и особочистых материалов РАН, Института 
прикладных математических исследований Карельского НЦ РАН, Московского 
авиационного института, Вологодского государственного педагогического университета, 
НИИММ им.\ Н.\,Г.~Чеботарева, Казанского государственного университета, Дебреценского 
университета (Венгрия).

Несколько статей выпуска посвящено разработке и применению стохастических методов и 
информационных технологий для решения различных прикладных задач. В~работе В.\,Г.~Ушакова 
и О.\,В.~Шестакова рассмотрена задача определения вероятностных характеристик случайных 
функций по распределениям интегральных преобразований, возникающих в задачах эмиссионной 
томографии. В~статье Д.\,О.~Яковенко и М.\,А.~Целищева рассмотрены некоторые вопросы 
математической теории риска и предложен новый подход к диверсификации инвестиционных 
портфелей. Работа И.\,А.~Кудрявцевой и А.\,В.~Пантелеева посвящена построению и 
исследованию математической модели, описывающей динамику сильноионизованной плазмы. 
В~статье П.\,П.~Кольцова изучается качество работы ряда алгоритмов сегментации изображений. 
Статья А.\,Н.~Чупрунова и И.~Фазекаша посвящена вероятностному анализу числа без\-оши\-бочных 
блоков при помехоустойчивом кодировании; получены усиленные законы больших чисел для указанных 
величин.

В данном выпуске традиционно присутствует тематика, весьма активно разрабатываемая в течение 
многих лет специалистами ИПИ РАН и МГУ,~--- методы моделирования и управления для 
информационно-телекоммуникационных и вычислительных систем, в частности методы 
теории массового обслуживания. В~статье А.\,И.~Зейфмана с соавторами рассматриваются 
модели обслуживания, описываемые марковскими цепями с непрерывным временем в случае 
наличия катастроф. В~работе М.\,М.~Лери и И.\,А.~Чеплюковой рассматриваются случайные 
графы Интернет-типа, т.\,е.\ графы, степени вершин которых имеют степенные распределения; 
такие задачи находят применение при исследовании глобальных сетей передачи данных. 
Работа Р.\,В.~Разумчика посвящена исследованию систем массового обслуживания специального 
вида~--- с отрицательными заявками и хранением вытесненных заявок.

Ряд статей посвящен развитию перспективных теоретических 
вероятностно-статистических методов, которые находят широкое применение в различных 
задачах информатики и информационных технологий. В~работе В.\,Е.~Бенинга, А.\,К.~Горшенина 
и В.\,Ю.~Королева рассмотрена задача статистической проверки гипотез о числе компонент 
смеси вероятностных распределений, приводится конструкция асимптотически наиболее мощного 
критерия. Результаты этой работы найдут применение в ряде прикладных задач, использующих 
математическую модель смеси вероятностных распределений (в информатике, моделировании 
финансовых рынков, физике турбулентной плазмы и~т.\,д.). В~статье В.\,Ю.~Королева, 
И.\,Г.~Шевцовой и С.\,Я.~Шоргина строится новая, улучшенная оценка точности нормальной 
аппроксимации для пуассоновских случайных сумм; как известно, указанные случайные суммы 
широко используются в качестве моделей многих реальных объектов, в том числе в информатике, 
физике и других прикладных областях. Работа В.\,Г.~Ушакова и Н.\,Г.~Ушакова посвящена 
исследованию ядерной оценки плотности распределения; эти результаты могут применяться, 
в част\-ности, при анализе трафика в телекоммуникационных системах. Серьезные приложения 
в статистике могут получить результаты работы О.\,В.~Шестакова, в которой доказаны оценки 
скорости сходимости распределения выборочного абсолютного медианного отклонения к нормальному 
закону. 

\smallskip

Редакционная коллегия журнала выражает надежду, что данный тематический  выпуск 
будет интересен специалистам в области теории вероятностей и математической статистики 
и их применения к решению задач информатики и информационных технологий.
     
     %\vfill 
     \vspace*{20mm}
     \noindent
     Заместитель главного редактора журнала <<Информатика и её 
применения>>,\\
     директор ИПИ РАН, академик  \hfill
     \textit{И.\,А.~Соколов}\\
     
     \noindent
     Редактор-составитель тематического выпуска,\\
     профессор кафедры математической статистики факультета\\
      вычислительной математики и кибернетики МГУ им.\ М.\,В.~Ломоносова,\\
     ведущий научный сотрудник ИПИ РАН,\\ 
доктор физико-математических наук \hfill
      \textit{В.\,Ю.~Королев}
     
     } }
     }

\def\stat{sinits}

\def\tit{АНАЛИТИЧЕСКОЕ МОДЕЛИРОВАНИЕ
НОРМАЛЬНЫХ ПРОЦЕССОВ В~СТОХАСТИЧЕСКИХ СИСТЕМАХ СО~СЛОЖНЫМИ~НЕЛИНЕЙНОСТЯМИ}

\def\titkol{Аналитическое моделирование
нормальных процессов в~стохастических системах со~сложными нелинейностями}

\def\aut{И.\,Н.~Синицын$^1$, В.\,И.~Синицын$^2$}

\def\autkol{И.\,Н.~Синицын, В.\,И.~Синицын}

\titel{\tit}{\aut}{\autkol}{\titkol}

\renewcommand{\thefootnote}{\arabic{footnote}}
\footnotetext[1]{Институт проблем
информатики Российской академии наук, sinitsin@dol.ru}
\footnotetext[2]{Институт проблем
информатики Российской академии наук, vsinitsin@ipiran.ru}


\Abst{Рассматриваются конечномерные дифференциальные стохастические системы
(ДСтС) и эредитарные (интегродифференциальные) стохастические системы  (ЭСтС)
с винеровскими и пуассоновскими шумами, приводимые к ДСтС со сложными конечными,
дифференциальными и интегральными нелинейностями. Такие модели функционирования
описывают поведение многих современных нано- и кван\-то\-во-оп\-ти\-че\-ских
технических средств информатики. Приводятся уравнения методов нормальной
аппроксимации (МНА) и статистической линеаризации (МСЛ) для аналитического
моделирования нестационарных и стационарных нормальных (гауссовских) процессов
в нелинейных ДСтС и  нелинейных ЭСтС путем аппроксимации эредитарных ядер
линейными операторными уравнениями для дифференцируемых нелинейностей и
сингулярными ядрами для недифференцируемых нелинейностей. Рассматриваются
методы вычисления типовых интегралов МНА (МСЛ) для сложных (многомерных и
векторного аргумента) конечных и дифференциальных нелинейностей. Особое
внимание уделяется иррациональным и дробно-рациональным нелинейностям
скалярного аргумента. Приводятся примеры вычисления интегралов. Подробно
рассматриваются вопросы вычисления типовых интегралов МНА (МСЛ) для сложных
интегральных нелинейностей.}

\KW{аналитическое моделирование;
дифференциальные стохастические системы с винеровскими и пуассоновскими шумами (ДСтС);
метод нормальной аппроксимации (МНА);
метод статистической линеаризации (МСЛ);
сложные иррациональные нелинейности;
сложные конечные, дифференциальные и интегральные нелинейности;
эредитарные стохастические системы (ЭСтС), приводимые к дифференциальным}

\DOI{10.14357/19922264140302}

\vspace*{9pt}

\vskip 16pt plus 9pt minus 6pt

\thispagestyle{headings}

\begin{multicols}{2}

\label{st\stat}


\section{Введение}


Моделями функционирования многих современных технических сис\-тем информатики
служат стохастические системы (СтС), описываемые дифференциальными, интегральными
и интегродифференциальными уравнениями со сложными дроб\-но-ра\-ци\-о\-наль\-ны\-ми,
иррациональными и интегральными нелинейностями. В~[1] дано систематическое
изложение МНА и МСЛ для ДСтС и ЭСтС, приводимых к дифференциальным.

Обобщая~[2--7], рассмотрим развитие МНА и МСЛ для аналитического моделирования
нормальных стохастических процессов (СтП) на случай СтС со сложными конечными,
дифференциальными и интегральными нелинейностями.

Как показано в~\cite{4-sin}, альтернативным подходом к аналитическому моделированию
СтП в ДСтС и ЭСтС служит подход, основанный на дискретизации стохастических
дифференциальных уравнений на основе использования обобщенной формы Ито и
кратных стохастических интегралов от винеровских и пуассоновских СтП с
последующим применением дискретных версий МНА (МСЛ).

Статья состоит из введения, пяти разделов и заключения.

В~разд.~2 и~3
приводятся уравнения МНА и МСЛ для аналитического моделирования одно- и
двумерных распределений стационарных и нестационарных СтП в ДСтС и ЭСтС,
приводимых к ДСтС.

Типовые интегралы МНА и МСЛ рассматриваются в разд.~4.

Особенности аналитического моделирования в ДСтС со сложными конечными и
дифференциальными нелинейностями обсуждаются в разд.~5.

Раздел~6
посвящен аналитическому моделированию СтП в ДСтС со сложными интегральными
нелинейностями.

Приводятся примеры.


\section{Уравнения методов нормальной~аппроксимации и~статистической
линеаризации для~дифференциальных стохастических систем}

Как известно~\cite{2-sin, 3-sin},  уравнения конечномерных непрерывных нелинейных сис\-тем
со стохастическими возмущениями путем расширения вектора состояния ДСтС
могут быть записаны в виде следующего векторного стохастического
дифференциального уравнения Ито:
    \begin{multline}
    dY_t = a(Y_t, t)\, dt + b (Y_t, t) \,dW_0+{}\\
    {}+ \iii_{R_0} c (Y_t, t, v) P^0
    (dt, dv)\,,\enskip Y(t_0) = Y_0\,.\label{e2.1-sin}
    \end{multline}
Здесь $a=a(Y_t, t)$ и $b\hm=b(y_t, t)$~--- известные
$(p\times 1)$-мер\-ная и  $(p\times m)$-мер\-ная функции~$Y_t$ и~$t$;
$W_0\hm= W_0(t)$~--- $r$-мер\-ный винеровский СтП интенсивности
$\nu_0 \hm= \nu_0(t)$; $c(Y_t, t, v)$~--- $(p\times 1)$-мер\-ная функция  $Y_t, t$
и вспомогательного $(q\times 1)$-мер\-но\-го па\-ра\-мет\-ра~$v$;
$\iii_{\Delta} dP^0 (t, A)$~--- центрированная пуассоновская мера,
определяемая
\begin{equation*}
\iii_{\Delta} dP^0 (t, A) = \iii_{\Delta} dP (t,A) =
\iii_{\Delta} \nu_P (t,A)\, dt\,. %\label{e2.2-sin}
\end{equation*}
В~(\ref{e2.1-sin}) принято: $\iii_{\Delta}$~-- число скачков пуассоновского
СтП в интервале времени  $\Delta \hm= (t_1, t_2]$; $\nu_P (t, A)$~---
интенсивность пуассоновского СтП  $P(t,A)$; $A$~--- некоторое борелевское
множество пространства  $R_0^q$ с выколотым началом.
Начальное значение~$Y_0$ представляет собой случайную величину, не зависящую
от приращений СтП  $W_0(t)$ и $P(t,A)$ на интервалах времени, следующих
за~$t_0$, $t_0 \hm\le t_1\hm\le t_2$ для любого множества~$A$.

В случае аддитивных нормальных (гауссовских) и обобщенных
пуассоновских возмущений уравнение~(\ref{e2.1-sin}) имеет вид:
\begin{equation}
\dot Y_t = a(Y_t,t)+ b_0 (t) V\,, \enskip
V = \dot W\,,\enskip Y(t_0) = Y_0\,.\label{e2.3-sin}
\end{equation}
Здесь $W$~--- СтП с независимыми приращениями, представляющий собой
смесь нормального и обобщенного пуассоновского СтП.

Если предположить существование конечных вероятностных
моментов второго порядка для моментов времени~$t_1$ и~$t_2$, то уравнения
МНА примут следующий вид~\cite{2-sin, 3-sin}:
\begin{itemize}
\item  для характеристических функций
    \begin{equation}
    g_1^N (\la;t) =\exp \lk i\la^{\mathrm{T}} m_t - \fr{1}{2}\, \la^{\mathrm{T}} K_t \la\rk\,;\label{e2.4-sin}
    \end{equation}
\begin{equation}
\hspace*{-7.5mm}g_{t_1, t_2}^N (\la_1, \la_2;t_1, t_2 ) =\exp \lk i\bar \la^{\mathrm{T}} \bar m_2 -
\fr{1}{2}\, \bar \la^{\mathrm{T}} \bar K_2 \la\rk\,,\!\!\label{e2.5-sin}
\end{equation}
где
    \begin{gather*}
    \bar \la =\lk \la_1^{\mathrm{T}}\la_2^{\mathrm{T}}\rk^{\mathrm{T}}\,; \quad
        \bar m_2 = \lk m_{t_1}^{\mathrm{T}} m_{t_2}^{\mathrm{T}}\rk^{\mathrm{T}}\,;\\
        \bar K_2= \begin{bmatrix}
    K(t_1, t_1)& K(t_1, t_2)\\
    K(t_2, t_1)& K(t_2, t_2)
    \end{bmatrix}\,;
    \end{gather*}

\item для математических ожиданий  $m_t$, ковариационной матрицы~$K_t$ и
матрицы ковариационных функций $K(t_1, t_2)$:
    \begin{equation}
    \dot m_t = a_1 (m_t, K_t, t)\,,\enskip m_0 = m(t_0)\,;\label{e2.6-sin}
    \end{equation}
\begin{equation}
\dot K_t = a_2 (m_t, K_t, t)\,,\enskip K_0 = K(t_0)\,;\label{e2.7-sin}
\end{equation}

\vspace*{-12pt}

\noindent
\begin{multline}
\fr{\prt K(t_1, t_2)}{\prt t_2 }= K(t_1, t_2) a_{21} (m_{t_2}, K_{t_2}, t_2)^{\mathrm{T}}\,;\\
K(t_1, t_1) = K_{t_1}\,.
\label{e2.8-sin}
\end{multline}
    \end{itemize}
Здесь приняты следующие обозначения:
\begin{equation}
a_1 = a_1 (m_t, K_t, t) = M_N a (Y_t, t)\,;\label{e2.9-sin}
\end{equation}

\vspace*{-12pt}

\noindent
\begin{multline}
a_2 = a_2 (m_t, K_t, t) = a_{21} (m_t, K_t, t)+{}\\
{}+ a_{21} (m_t, K_t, t)^{\mathrm{T}} +
a_{22}(m_t, K_t, t)\,;\label{e2.10-sin}
\end{multline}

\vspace*{-12pt}

\noindent

\begin{equation}
a_{21} = a_{21}(m_t, K_t, t)=  M_N a(Y_t, t) Y_{t}^{0\mathrm{T}}\,;\label{e2.11-sin}
\end{equation}
\begin{equation*}
a_{22} = a_{22}(m_t, K_t, t)= M_N \sigma (Y_t, t)\,;
%\label{e2.12-sin}
\end{equation*}

\vspace*{-12pt}

\noindent
\begin{multline*}
\sigma (Y_t, t) = b(Y_t, t) \nu_0(t) b(Y_t, t)^{\mathrm{T}} +{}\\
{}+
\iii_{R_0^q} c (Y_t, t, v) c(Y_t, t,v)^{\mathrm{T}}
\nu_P (t, dv)\,; %\label{e2.13-sin}
\end{multline*}

\vspace*{-12pt}

\begin{gather*}
m_t = MY_t\,,\quad Y_t^0 = Y_t - m_t\,,\\
K_t = M_N Y_0^0 Y_t^{0\mathrm{T}}\,,\quad K(t_1, t_2) =
M_N Y_{t_1}^0 Y_{t_2}^0\,; %\label{e2.14-sin}
\end{gather*}
$M_N$~--- символ вычисления математического ожидания для нормальных
распределений~(\ref{e2.4-sin}) и~(\ref{e2.5-sin}).

Для стационарных ДСтС нормальные стационарные СтП~--- если они существуют,
то  $m_t \hm=\bar m$, $ K_t \hm=\bar K$, $K(t_1, t_2) \hm= k(\tau)$
$(\tau \hm= t_1\hm-t_2)$,~--- определяются уравнениями~\cite{2-sin, 3-sin}:
   \begin{equation}
   a_1 (\bar m, \bar K) =0\,;\enskip a_2 (\bar m, \bar K)=0\,;\label{e2.15-sin}
   \end{equation}
   \begin{equation}
   \left.
   \hspace*{-2.8mm}\begin{array}{l}
  \dot k_\tau (\tau) = a_{21} (\bar m, \bar K)\bar K^{-1} k(\tau)\,;\\[9pt]
  k(0) =\bar K \enskip (\forall \tau >0)\,, \
  k(\tau) = k(-\tau)^{\mathrm{T}} \enskip
  (\forall\tau <0)\,.
  \end{array}\!\!
  \right\}\!\!
  \label{e2.16-sin}
  \end{equation}
При этом необходимо, чтобы матрица  $a_{21} (\bar m, \bar K)\hm=\bar a_{21}$
была бы асимптотически устойчивой.

Для ДСтС~(\ref{e2.3-sin}) уравнения МНА переходят в уравнения МСЛ
Казакова~\cite{2-sin, 3-sin}, если принять
\begin{equation}
a(Y_t,t) = a_1 (m_t, K_t) + k_1^a (m_t, K_t) Y_t^0\,;\label{e2.17-sin}
\end{equation}
\begin{equation}\left.
\begin{array}{rl}
b(Y_t,t) &= b_0 (t)\,;\\[9pt]
    \si(Y_t, t)&= b_0(t) \nu(t) b_0(t)^{\mathrm{T}} = \si_0(t)\,,
    \end{array}
    \right\}\label{e2.18-sin}
    \end{equation}
    \begin{equation}
k_1^a (m_t, K_t, t) =\lk \left(\fr{\prt}{\prt m_t} \right)
    a_0 (m_t, K_t, t)^{\mathrm{T}}\rk^{\mathrm{T}}\,;\label{e2.19-sin}
    \end{equation}
    \begin{equation}
\dot m_t = a_1 (m_t, K_t, t) \,,\enskip m_0 = m(t_0)\,,\label{e2.20-sin}
\end{equation}

\vspace*{-12pt}

\noindent
\begin{multline}
\dot K_t = k_1^a (m_t, K_t, t) K_t + K_t k_1^a (m_t, K_t, t)^{\mathrm{T}}
    +\si_0(t)\,;\\
    K_0 = K(t_0)\,;
    \label{e2.21-sin}
    \end{multline}

    \vspace*{-12pt}

    \noindent
\begin{multline}
\fr{\prt K(t_1, t_2)}{\prt t_2} =
    K(t_1, t_2) k_{t_2} k_1^a (m_{t_2}, K_{t_2}, t_2)^{\mathrm{T}}\,;\\
    K(t_1, t_2) = K_{t_1}\,.
    \label{e2.22-sin}
\end{multline}

Для стационарных ДСтС~(\ref{e2.3-sin})
при условии асимптотической устойчивости матрицы $k_1^a (\bar m, \bar K)$
в основе МСЛ лежат уравнения~(\ref{e2.15-sin}), записанные в виде:
    \begin{gather}
    a_1 (\bar m, \bar K) =0\,; \label{e2.23-sin}\\
k_1^a (\bar m, \bar K) \bar K + \bar K k_1^a
(\bar m, \bar K)^{\mathrm{T}} +\bar \si_0 =0\,;\label{e2.24-sin}
\end{gather}

\vspace*{-12pt}

\noindent
\begin{multline}
k_\tau (\tau) = k_1^a (\bar m, \bar K)k(\tau)\,,\enskip
k(0) =\bar K \enskip (\forall \tau >0)\,,\\
k(\tau) = k (-\tau)^{\mathrm{T}} \enskip (\forall \tau <0)\,.
\label{e2.25-sin}
\end{multline}

Уравнения~(\ref{e2.4-sin})--(\ref{e2.8-sin})
лежат в основе МНА для ДСтС~(\ref{e2.1-sin}), а уравнения~(\ref{e2.17-sin})--(\ref{e2.22-sin})~---
в основе МСЛ для ДСтС~(\ref{e2.3-sin}). Для определения стационарных СтП
согласно МНА служат соотношения~(\ref{e2.15-sin}) и~(\ref{e2.16-sin}),
а МСЛ~--- (\ref{e2.17-sin})--(\ref{e2.25-sin}).

\section{Уравнения методов нормальной~аппроксимации и~статистической линеаризации
для~эредитарных стохастических систем, приводимых к~дифференциальным}

Рассмотрим ЭСтС, описываемую интегродифференциальным уравнением Ито
следующего вида~\cite{7-sin}:

\noindent
\begin{multline}
dX_t = \lk a(X_t, t) +\iii_{t_0}^t a_1 (X(\tau) ,\tau, t)\,d\tau\rk dt+{}\\
{}+\lk b(X_t, t) +\iii_{t_0}^t b_1 (X(\tau) ,\tau, t)\,d\tau\rk dW_0+{}\\
\hspace*{-1.5mm}{}+\!\!\iii_{R_0^q}\!\!\lk c(X_t, t,v) +\!\iii_{t_0}^t\! c_1 (X(\tau) ,\tau, t,v)\,d\tau\!\rk\! dP^0 (t, dv)
\!\!\!\!\label{e3.1-sin}
\end{multline}
с начальным условием  $X(t_0) = X_0$. В~(\ref{e3.1-sin})
сохранены обозначения разд.~2.

Функции $a=a(X_t, t)$, $a_1\hm = a_1(X (\tau),\tau, t)$,
$b\hm=b(X_t, t)$, $b_1\hm = b_1(X (\tau),\tau, t)$,
$c\hm=c(X_t,t,v)$ и $c_1\hm = c_1(X (\tau),\tau, t,v)$ имеют
соответственно размерности $p\times 1$, $p\times 1$, $p\times r$,
$p\times r$, $p\times 1$ и $p\times 1$ и допускают представления следующего вида:
\begin{equation}
\left.
\begin{array}{rl}
a_1&=A(t,\tau) \vrp (X(\tau) , \tau)\,;\\[9pt]
b_1&=B(t,\tau) \psi (X(\tau) ,  \tau)\,;\\[9pt]
c_1&=C(t,\tau) \chi (X(\tau) ,  \tau, v)\,.
\end{array}
\right\}
\label{e3.2-sin}
\end{equation}
Здесь эредитарные ядра $A\hm=A(t,\tau)\hm=\lk A_{ij}(t,\tau)\rk$
$(i,j\hm=\overline{1,p})$,
$B\hm=B(t,\tau)=\lk B_{i l}(t,\tau)\rk$ $(i\hm=\overline{1,p}$;
$l\hm=\overline{1,r})$ и $C\hm=C(t,\tau)=\lk C_{ij}(t,\tau)\rk$
$(i,j\hm=\overline{1,p})$ имеют соответственно размерности
$p\times p$, $p\times r$ и $p\times p$. Они удовлетворяют следующим условиям
физической реализуемости и асимптотического затухания:
\begin{multline}
A_{ij}(t,\tau)=0;\enskip B_{i l}(t,\tau)=0;\\[1pt]
C_{ij}(t,\tau)=0\enskip \forall \tau >t;\label{e3.3-sin}
\end{multline}

\vspace*{-12pt}

\begin{equation}
\left.
\hspace*{-3mm}\begin{array}{c}
\displaystyle\iin\! \lv A_{ij} (t,\tau) \rv d\tau <\infty\,;\
\displaystyle\iin\! \lv B_{i l} (t,\tau) \rv d\tau <\infty \,;\\[9pt]
\displaystyle\iin \!\lv C_{ij} (t,\tau) \rv d\tau <\infty\,.
\end{array}\!
\right\}\!
\label{e3.4-sin}
\end{equation}

В дальнейшем ограничимся случаем, когда эредитарные ядра удовлетворяют
линейным операторным уравнениям~\cite{6-sin, 5-sin, 7-sin}.

Нелинейные в общем случае функции $\vrp\hm=\vrp(X(\tau),\tau)$,
$\psi \hm=\psi(X(\tau), \tau)$, $\chi \hm=\chi (X(\tau),  \tau, v)$
отражают нелинейные свойства ЭСтС, зависят от  $X(\tau)$ и имеют размерности
$p\times 1$, $p\times p$, $p\times 1$ соответственно.

Важный класс  эредитарных ядер представляют собой
сингулярные (вырожденные) ядра, когда имеют место представления:
\begin{equation}
\left.
\hspace*{-3mm}\begin{array}{rl}
A_{ij} (t,\tau) &= A_{ij}^+(t) A_{ij}^-(\tau)\,;\\[9pt]
B_{i l} (t,\tau)& = B_{il}^+(t) B_{il}^-(\tau)\,;\\[9pt]
C_{ij} (t,\tau) &= C_{ij}^+ ( t) C_{ij}^- (\tau)\
(i,l= \overline{1,p}, j=\overline{1,r}).
\end{array}\!
\right\}\!\!
\label{e3.5-sin}
\end{equation}

В~\cite{6-sin, 5-sin, 7-sin} показано, что для дифференцируемых нелинейных
функций~$\vrp$, $\psi$, $\chi$ путем расширения вектора состояния за счет
инструментальных переменных, аппроксимируемых линейными операторными уравнениями,
определяющими эредитарные ядра в ЭСтС, (\ref{e3.1-sin})--(\ref{e3.4-sin})
приводятся к ДСтС вида~(\ref{e2.1-sin}) или~(\ref{e2.3-sin}).
В~случае недифференцируемых нелинейных функций~$\vrp$, $\psi$, $\chi$
ЭСтС~(\ref{e3.1-sin})--(\ref{e3.4-sin}) приводятся к~(\ref{e2.1-sin}) или~(\ref{e2.3-sin})
на основе аппроксимации вырожденными (сингулярными) ядрами~\cite{6-sin, 5-sin, 7-sin}.

Таким образом, после приведения ЭСтС~(\ref{e3.1-sin}) к ДСтС~(\ref{e2.1-sin})
или~(\ref{e2.3-sin}) можно воспользоваться уравнениями МНА и МСЛ разд.~2.

\section{Типовые интегралы методов нормальной аппроксимации и~статистической
линеаризации}

Как следует из уравнений~(\ref{e2.9-sin})--(\ref{e2.11-sin}),
для МНА необходимо уметь вычислять следующие интегралы:
\begin{multline}
I_0^a = I_0^a (m_t, K_t, t) = a_1 (m_t, K_t, t)={}\\
{}= M_N a(Y_t, t)\,;
\label{e4.1-sin}
\end{multline}

\vspace*{-12pt}

\noindent
\begin{multline}
I_1^a = I_1^a (m_t, K_t, t)= a_{21}(m_t, K_t, t)= {}\\
{}=M_N a(Y_t , t) Y_t^{0\mathrm{T}}\,;\label{e4.2-sin}
\end{multline}

\vspace*{-12pt}

\noindent
\begin{multline}
I_0^{\bar \si} = I_0^{\bar \si} (m_t, K_t, t) = a_{22}(m_t, K_t, t) ={}\\
{}= M_N \bar \si (Y_t, t)\,.\label{e4.3-sin}
\end{multline}
Для МСЛ достаточно вычислить интеграл~(\ref{e4.1-sin}),
причем интеграл~$I_1^a$ вычисляется по формуле~\cite{2-sin, 3-sin, 4-sin}:
\begin{equation*}
k_1^a = k_1^a (m_t, K_t, t)=\lk \left( \fr{\prt}{\prt m_t}\right)
I_0^a (m_t, K_t, t)^{\mathrm{T}}\rk^{\mathrm{T}}. %\label{e4.4-sin}
\end{equation*}

\medskip

\noindent
\textbf{Пример 1.} В~[1] для типовых степенных, тригоно\-мет\-ри\-че\-ских,
показательных и ку\-соч\-но-по\-сто\-ян\-ных нелинейностей $Z_t \hm=\vrp (Y_t, t)$
скалярного и векторного аргумента приведены формулы для интегралов
$I_0^\vrp \hm= I_0^\vrp (m_t^y, K_t^y, t)$, а также
$k_1^\vrp \hm= k_1^\vrp (m_t^y, K_t^y, t)$.

\medskip

\noindent
\textbf{Замечание.}
 Важно иметь в виду, что уравнения МНА (МСЛ) содержат интегралы
 $I_0^a$, $I_1^a$, $I_0^\si$ в виде соответствующих коэффициентов.
 Поэтому процедура вычисления интегралов должна быть согласована с
 методом численного решения обыкновенных дифференциальных уравнений для
 $m_t$, $K_t$ и $K(t_1, t_2)$. Эти коэффициенты допускают дифференцирование
 по~$m_t$ и~$K_t$, так как под интегралом стоит сглаживающая нормальная плотность.

\section{Сложные конечные и~дифференциальные нелинейности}

Важный класс сложных конечных нелинейностей (многомерных и векторного аргумента)
представляют собой сложные функции вида:
    \begin{equation*}
    \xi =\vrp (X_t, Y_t, t)\,,\enskip X_t =\psi (Y_t, t)\,. %\label{e5.1-sin}
    \end{equation*}
В~этом случае вычисление интегралов (см.\ разд.~4) проводится по совокупности
переменных  $\lk X_t^{\mathrm{T}} Y_t^{\mathrm{T}}\rk^{\mathrm{T}}$.
К таким нелинейностям, например, относятся дроб\-но-ра\-ци\-о\-наль\-ные,
иррациональные  нелинейности, выражаемые специальными функциями, многозначные
нелинейности, зависящие от СтП~$X_t$ и его производных~$\dot X_t$,  $\ddot X_t$
и~др.

\medskip

\noindent
\textbf{Пример 2.}
Рассмотрим вычисление интегралов~(\ref{e4.1-sin}) и~(\ref{e4.2-sin})
для сложных одномерных иррациональных нелинейностей скалярного аргумента
\begin{equation}
\vrp (Y_t, t) =\lv Y_t\rrv^{\alpha-1}\, \mathrm{sgn}\, Y_t
\label{e5.2-sin}
\end{equation}
($\alpha$~--- нецелый показатель).

Пользуясь~(\ref{e2.16-sin}) и~(\ref{e2.19-sin}), представим~(\ref{e5.2-sin}) в виде
\begin{equation*}
\vrp(Y_t, t) = \vrp_0 (m_t, D_t, t) + k_1^\vrp(m_t, D_t, t) Y_t^0. %\label{e5.3-sin}
\end{equation*}
Здесь введены следующие обозначения:
\begin{gather*}
\vrp_0(m_t, D_t, t) =\Gamma(\alpha) D_t^{1/2} e^{-\xi^2/4} D_{-\alpha} (\xi)\,;%\label{e5.4-sin}
\\
k_1^a (m_t, D_t, t) =\fr {\prt \vrp_0(m_t, D_t, t)}{\prt m_t}\,,%\label{e5.5-sin}
\end{gather*}
где  $\Gamma(\alpha)$~--- гамма-функция,  $\xi \hm= m_t/\sqrt{D_t}$~---
отношение <<сиг\-нал--шум>>; $D_{-\alpha} (\xi)$~---
функция параболического цилиндра~\cite{9-sin}.
При вычислении были учтены следующие соотношения~\cite{9-sin, 8-sin}:
\begin{multline}
\iii_0^\infty x^{\alpha-1} e^{-\beta x^2 - \gamma x} \,dx ={}\\
{}=
(2\beta)^{-\alpha/2} \Gamma(\alpha) \exp \left(\fr{\gamma^2}{8\beta}\right)
D_{-\alpha} \left(\fr{\gamma}{\sqrt{2\beta}}\right)\,;\label{e5.6-sin}
\end{multline}

\vspace*{-12pt}

\noindent
\begin{multline}
\fr{dD_\rho(\xi)}{d\xi} =
   -\fr{\xi}{2}\, D_\rho (\xi) -\rho D_{\rho-1} (\xi) =
   \fr{\xi}{2}\, D_\rho (\xi) -{}\\
   {}- D_{\rho+1} (\xi) \enskip
   (\mathrm{Re}\, \beta>0\,,\enskip \mathrm{Re}\,\alpha>0\,,\enskip
   \rho=-\alpha)\,.\label{e5.7-sin}
   \end{multline}

Соотношения~(\ref{e5.6-sin}) и~(\ref{e5.7-sin})
могут быть использованы также для вычисления интегралов~(\ref{e4.3-sin}).

\medskip

\noindent
\textbf{Замечание.}
Для вычисления интегралов $I_0^a$, $I_1^a$ и $I_0^{\bar \si}$
применительно к типовым иррациональным нелинейностям вида
    $\lv Y_t\rrv^{\alp-1} e^{\delta Y_t}$, $\lv Y_t\rrv^{\alp-1}  \cos \w Y_t$,
    $\lv Y_t\rrv^{\alp-1}  \sin \w Y_t$
и более общим нелинейностям \mbox{вида}
    \begin{equation*}
    \vrp (Y_t, t) =\Phi^\vrp \left( \lv Y_t\rrv^{\alpha-1}, t\right) %\label{e5.8-sin}
    \end{equation*}
можно рекомендовать известные численные методы вычисления функций на ЭВМ~\cite{8-sin}.

\smallskip

\noindent
\textbf{Пример 3.}
Для нелинейной дроб\-но-ра\-ци\-о\-наль\-ной функции

\noindent
\begin{equation*}
\vrp (Y_t, t) = \fr{a}{(b+Y_t)^2} %\label{e5.9-sin}
\end{equation*}
имеем

\vspace*{-3pt}

\noindent
\begin{gather*}
\vrp_0 (m_t, D_t, t) =a b^{-2} \lk 1+ \chi (m_t, D_t, t)\rk\,; %\label{e5.10-sin}
\\
k_1^\vrp (m_t, D_t, t) =  a b^{-2}\fr{\prt \chi (m_t, D_t, t)}{\prt m_t}\,. %\label{e5.11-sin}
\end{gather*}
Здесь

\vspace*{-3pt}

\noindent
\begin{multline*}
\chi (m_t, D_t, t) ={}\\
{}=\sss_{n=1}^\infty \sss_{l=0}^{E(n/2)}
\fr{(-1)^n (n+1) n!}{(n-2l)! (2l)!}\, b^{-n} m_t^n \left( \fr{D_t}{ 2 m_t^2}
\right)^l, %\label{e5.12-sin}
\end{multline*}
где  $E(n/2)$~--- целая часть~$n/2$; $a\hm=a(t)$; $b\hm= b(t)$.

\vspace*{-6pt}

\section{Сложные интегральные нелинейности}

\vspace*{-2pt}

Пусть сначала векторно-матричная нелинейность имеет эредитарный характер, т.\,е.\
\begin{equation}
\underline{\vrp} (Y_t, t) =\iii_{t_0}^t A(t,\tau) \vrp (Y(\tau), \tau) \,d\tau\,.
\label{e6.1-sin}
\end{equation}
Тогда, как показано в~\cite{6-sin, 5-sin, 7-sin}, следует соответст\-ву\-ющие
интегродифференциальные соотношения путем введения  инструментальных
переменных привести к дифференциальным соотношениям.  Для
дифференцируемых функций~$\vrp$ и асимптотически устойчивых ядер
$A(t,\tau)$ зависимость~(\ref{e3.5-sin}) имеет следующий дифференциальный вид:
\begin{equation*}
F^A (t, D) \underline{\vrp} (Y_t, t) = H^A (t, D) \vrp (Y_t, t)\,. %\label{e6.2-sin}
\end{equation*}
Здесь $F^A (t, D)$ и  $H^A (t, D)$~--- линейные дифференциальные операторы $(D\hm= d/dt)$.

Для недифференцируемых функций~$\vrp$ и асимптотически устойчивых
сингулярных ядер~(\ref{e3.5-sin}) используются соотношения:
\begin{equation*}
\underline{\vrp} (Y_t, t) = A^+ Z\,,\enskip
\dot Z = A^- \vrp\,,\enskip
Z(t_0)=0\,. %\label{e6.3-sin}
\end{equation*}

Многочисленные примеры аналитического моделирования ЭСтС можно найти
в~[1--3, 5, 7, 10, 11].

Как отмечалось в~\cite{3-sin}, часто наряду с интегральными
нелинейностями~(\ref{e6.1-sin}) рассматривают нелинейности вида:

\columnbreak

\noindent
\begin{equation*}
Z_s =\sss_{\rho=1}^R \mathcal{A}_\rho \vrp_\rho (Y_{t_1}\tr Y_{t_r})\,, %\label{e6.2-sin}
\end{equation*}
где $\mathcal{A}_1 \tr \mathcal{A}_R$~--- произвольные линейные операторы,
действующие над функциями~$r$ переменных  $t_1\tr t_r$; $\vrp_\rho
\hm=\vrp_\rho (Y_{t_1} \tr Y_{t_r})$~--- линейные функции отмеченных
переменных. Такие нелинейности называются приводимыми к линейным.
Важным частным случаем~(\ref{e6.1-sin}) являются интегральные нелинейности вида:

\noindent
\begin{gather}
Z_s =\iii_T \vrp (Y_t, t, s)\, dt\,; \notag%\label{e6.3-sin}
\\
Z_s =\!\iii_T \!\cdots\!\iii_T\! \vrp (Y_{t_1}\tr Y_{t_r}; t_1\tr t_r, s)\,dt_1
\ldots dt_r,\notag %\label{e6.4-sin}
\end{gather}
В этом случае используется МСЛ по совокупности переменных  $Y_{t_1} \tr Y_{t_r}$.

\vspace*{-9pt}

\section{Заключение}

\vspace*{-2pt}

Разработаны методы и алгоритмы МНА и МСЛ для ДСтС и ЭСтС,
приводимых к ДСтС со сложными конечными, дроб\-но-ра\-ци\-о\-наль\-ны\-ми,
иррациональными, а также дифференциальными и интегральными нелинейностями.
Приведены примеры.

Результаты допускают обобщение на случай ДСтС и ЭСтС со
стохастическими нелинейностями, заданными каноническими разложениями и
интегральными каноническими  представлениями~\cite{1-sin, 3-sin, 11-sin}.

\vspace*{-9pt}

{\small\frenchspacing
 {%\baselineskip=10.8pt
 \addcontentsline{toc}{section}{References}
 \begin{thebibliography}{99}

 \vspace*{-2pt}

\bibitem{1-sin}
\Au{Синицын И.\,Н.,  Синицын~В.\,И.}
Лекции по нормальной и эллипсоидальной аппроксимации распределений в
стохастических сис\-те\-мах.~--- М.: ТОРУС ПРЕСС, 2013. 488~с.

\bibitem{6-sin} %2
\Au{Синицын И.\,Н. }
Stochastic hereditary control systems~// Проблемы управления и
теории информации, 1986. Т.~15. №\,4. С.~287--298.

\bibitem{2-sin} %3
\Au{Пугачев В.\,С., Синицын~И.\,Н.}
Стохастические дифференциальные сис\-те\-мы. Анализ и фильтрация.~--- М.:
Наука,  1990.  632~с. [Англ. пер.
 Stochastic differential systems.
Analysis and filtering.~--- Chichester, New York: Jonh Wiley, 1987.
549~p.].

\bibitem{5-sin} %4
\Au{Синицын И.\,Н. }
Конечномерные распределения процессов в стохастических интегральных
и интегродифференциальных системах~// Preprints of the 2nd IFAC
Symposium on Stochastic Control.~--- Vilnius: Pergamon Press,
1987.  Vol.~1. P.~144--153.

\bibitem{3-sin} %5
\Au{Пугачев В.\,С., Синицын~И.\,Н.}
Теория стохастических систем.~--- М.: Логос, 2000; 2004. 1000~с.
[Англ. пер.\linebreak\vspace*{-12pt}

\pagebreak

\noindent Stochastic systems. Theory and  applications.~---
Singapore: World Scientific, 2001. 908~p.].

\bibitem{4-sin} %6
\Au{Синицын И.\,Н.}
Параметрическое статистическое и аналитическое моделирование распределений
в нелинейных стохастических сис\-те\-мах на многообразиях~//
Информатика и её применения, 2013. Т.~7. Вып.~2. С.~4--16.

\bibitem{7-sin} %7
\Au{Синицын И.\,Н. }
Анализ и моделирование распределений в эредитарных стохастических
сис\-те\-мах~// Информатика и её применения, 2014. Т.~8. Вып.~1.\linebreak
С.~2--11.



\bibitem{9-sin} %8
\Au{Градштейн И.\,С., Рыжик~И.\,М.}
Таблицы интегралов, сумм, рядов и произведений.~--- М.: ГИФМЛ, 1963. 1100~с.

\bibitem{8-sin} %9
\Au{Попов Б.\,А., Теслер~Г.\,С. }
Вычисление функций на ЭВМ: Справочник.~--- Киев: Наукова Думка, 1984. 599~с.


\bibitem{11-sin} %10
\Au{Синицын И.\,Н.}
Канонические представления случайных функций и их применение в
задачах компьютерной поддержки научных исследований.~--- М.: ТОРУС
ПРЕСС, 2009. 768~с.

\bibitem{10-sin} %11
\Au{Синицын И.\,Н., Синицын~В.\,И., Корепанов~Э.\,Р., Белоусов~В.\,В.,
Сергеев~И.\,В., Басилашвили~Д.\,А.}
Опыт моделирования эредитарных стохастических сис\-тем~//
Кибернетика и высокие технологии XXI века: Сб. докл.  XIII Междунар.
науч.-технич. конф.~--- Воронеж: Саквоее, 2012. Т.~2. C.~346--357.

 \end{thebibliography}

 }
 }

\end{multicols}

\vspace*{-9pt}

\hfill{\small\textit{Поступила в редакцию 05.05.14}}

%\newpage

\vspace*{12pt}

\hrule

\vspace*{2pt}

\hrule

\vspace*{12pt}

\def\tit{ANALYTICAL MODELING OF NORMAL PROCESSES
 IN~STOCHASTIC SYSTEMS WITH~COMPLEX NONLINEARITIES}

\def\titkol{Analytical modeling of normal processes
 in~stochastic systems with~complex nonlinearities}

\def\aut{I.\,N.~Sinitsyn and V.\,I.~Sinitsyn}

\def\autkol{I.\,N.~Sinitsyn and V.\,I.~Sinitsyn}

\titel{\tit}{\aut}{\autkol}{\titkol}

\vspace*{-9pt}

\noindent
Institute of Informatics Problems, Russian Academy of Sciences,
44-2 Vavilov Str., Moscow 119333, Russian Federation


\def\leftfootline{\small{\textbf{\thepage}
\hfill INFORMATIKA I EE PRIMENENIYA~--- INFORMATICS AND
APPLICATIONS\ \ \ 2014\ \ \ volume~8\ \ \ issue\ 3}
}%
 \def\rightfootline{\small{INFORMATIKA I EE PRIMENENIYA~---
INFORMATICS AND APPLICATIONS\ \ \ 2014\ \ \ volume~8\ \ \ issue\ 3
\hfill \textbf{\thepage}}}

\vspace*{6pt}

\Abste{Differential stochastic systems (DStS) with Wiener and Poisson
noises and complex finite, differential, and  integral nonlinearities and
hereditary StS reducible to DStS are considered. Equations and algorithms
of analytical modeling based on the normal approximation method (NAM) and the
statistical linearization method (SLM) are given. The case of complex
continuous and discontinuous nonlinearities of scalar and vector arguments
is considered. Special attention is paid to NAM (SLM) typical integrals
for finite rational and irrational nonlinear and integral scalar and vector
nonlinear functions. The general case of integral nonlinearities reducible to
linear is considered. Test examples are given.}

\KWE{analytical modeling;
complex finite differential and integral nonlinearities;
complex irrational nonlinerarites
differential stochastic system with Wiener and Poisson noises;
method of normal approximation;
method of statistical linearization;
hereditary stochastic systems reducible to differential}

\DOI{10.14357/19922264140302}

  \begin{multicols}{2}

\renewcommand{\bibname}{\protect\rmfamily References}
%\renewcommand{\bibname}{\large\protect\rm References}

{\small\frenchspacing
 {%\baselineskip=10.8pt
 \addcontentsline{toc}{section}{References}
 \begin{thebibliography}{99}



\bibitem{1-sin-1}
\Aue{Sinitsyn, I.\,N., and  V.\,I.~Sinitsyn}.  2013.
Lektsii po normal'noy i ellipsoidal'noy approksimatsii raspredeleniy
v stokhasticheskikh sistemakh [Lectures on normal and ellipsoidal
approximation of distributions in stochastic systems].
Moscow: TORUS PRESS. 488~p.

\bibitem{6-sin-1} %2
\Aue{Sinitsyn, I.\,N.}  1986.
{Stochastic hereditary control systems}.
\textit{Problems Control Inform. Theory} 15(4):287--298.

\bibitem{2-sin-1} %3
\Aue{Pugachev, V.\,S., and  I.\,N.~Sinitsyn}.  1987.
\textit{Stochastic differential systems. Analysis and filtering.}
Chichester, New York: Jonh Wiley. 549~p.

\bibitem{5-sin-1} %4
\Aue{Sinitsyn, I.\,N.}  1987.
Konechnomernye raspredeleniya protsessov v stokhasticheskikh integral'nykh
i in\-teg\-ro\-dif\-fe\-ren\-tsial'nykh sistemakh [Finite dimensional distributions
of processes in stochastic integral and integrodifferential systems].
\textit{2nd  Symposium (International) IFAC on Stochastic Control
Preprints}. Vilnius: Pergamon Press. 1:144--153.

\bibitem{3-sin-1} %5
\Aue{Pugachev, V.\,S., and I.\,N.~Sinitsyn}. 2001.
\textit{Stochastic systems. Theory and  applications}.
Singapore: World Scientific. 908~p.

\bibitem{4-sin-1} %6
\Aue{Sinitsyn, I.\,N.}  2013.
Parametricheskoe statisticheskoe i analiticheskoe modelirovanie
raspredeleniy v nelineynykh stokhasticheskikh sistemakh na mnogoobraziyakh
[Parametric statistical and analytical modeling of distributions in
stochastic systems on manifolds].
\textit{Informatika i ee Primeneniya}~--- \textit{Inform. Appl.} 7(2):4--16.


\bibitem{7-sin-1} %7
\Aue{Sinitsyn, I.\,N.}  2014.
Analiz i modelirovanie raspredeleniy v ereditarnykh stokhasticheskikh sistemakh
[Analysis and modeling of distributions in hereditary stochastic systems].
\textit{Informatika i ee Primeneniya}~--- \textit{Inform. Appl.} 8(1):2--11.

\bibitem{9-sin-1} %8
\Aue{Gradshteyn, I.\,S., and I.\,M.~Ryzhik}.  1963.
\textit{Tablitsy integralov, summ, ryadov i proizvedeniy}
[Tables of integrals, sums, series, and products]. Moscow:  GIFML.   1100~p.

\pagebreak

\bibitem{8-sin-1} %9
\Aue{Popov, B.\,A., and G.\,S.~Tesler}.  1984.
\textit{Vychislenie funktsiy na EVM}. Spravochnik [Computing of functions].
Kiev: Naukova Dumka.  599~p.


\bibitem{11-sin-1} %10
\Au{Sinitsyn, I.\,N.} 2009.
\textit{Kanonicheskie predstavleniya sluchaynykh funktsiy i ikh primenenie v
zadachakh komp'yuternoy podderzhki nauchnykh issledovaniy}
[Canonical expansions of random functions and its application to
scientific computer-aided support]. Moscow: TORUS PRESS. 768~p.

\bibitem{10-sin-1} %11
\Aue{Sinitsyn, I.\,N., V.\,I.~Sinitsyn, E.\,R.~Korepanov,
V.\,V.~Belousov, I.\,V.~Sergeev, and D.\,A.~Basilashvili}.
2012. Opyt modelirovaniya ereditarnykh stokhasticheskikh sistem
[Experience of modeling in hereditary stochastic systems].
\textit{Kibernetika i Vysokie Tekhnologii XXI~Veka:
Sbornik dokladov  XIII Mezhdunar. nauch.-tekhnich. konf.}
[Cybernatics ans High Technologies of the XXI Century: Materials of
XIII  Scientific and Technological Conference (International)].
Voronezh: Sakvoee. 2:346--357.

\end{thebibliography}

 }
 }

\end{multicols}

\vspace*{-6pt}

\hfill{\small\textit{Received May 05, 2014}}

\vspace*{-18pt}

\Contr

\noindent
\textbf{Sinitsyn Igor N.} (b.\ 1940)~---
Doctor of Science in technology, professor, Honored scientist of RF, Head of Department, Institute of
Informatics Problems, Russian Academy of Sciences,
44-2 Vavilov Str., Moscow 119333, Russian
Federation; sinitsin@dol.ru

\vspace*{3pt}

\noindent
\textbf{Sinitsyn Vladimir I.} (b.\ 1968)~--- Doctor of Science in physics
and mathematics, associate professor, Head of Department, Institute of
Information Problems, Russian Academy of Sciences,
44-2 Vavilov Str., Moscow 119333, Russian Federation; VSinitsin@ipiran.ru




\label{end\stat}

\renewcommand{\bibname}{\protect\rm Литература}       %1
\def\stat{bosov+stef}

\def\tit{УПРАВЛЕНИЕ ВЫХОДОМ СТОХАСТИЧЕСКОЙ ДИФФЕРЕНЦИАЛЬНОЙ СИСТЕМЫ 
ПО~КВАДРАТИЧНОМУ КРИТЕРИЮ. I.~ОПТИМАЛЬНОЕ РЕШЕНИЕ МЕТОДОМ 
ДИНАМИЧЕСКОГО ПРОГРАММИРОВАНИЯ$^*$}

\def\titkol{Управление выходом стохастической дифференциальной системы 
по~квадратичному критерию. I}
%.~Оптимальное решение методом 
%динамического программирования}

\def\aut{А.\,В.~Босов$^1$, А.\,И.~Стефанович$^2$}

\def\autkol{А.\,В.~Босов, А.\,И.~Стефанович}

\titel{\tit}{\aut}{\autkol}{\titkol}

\index{Босов А.\,В.}
\index{Стефанович А.\,И.}
\index{Bosov A.\,V.}
\index{Stefanovich A.\,I.}




{\renewcommand{\thefootnote}{\fnsymbol{footnote}} \footnotetext[1]
{Работа выполнена при частичной поддержке РФФИ (проект 16-07-00677).}}


\renewcommand{\thefootnote}{\arabic{footnote}}
\footnotetext[1]{Институт проблем информатики Федерального исследовательского центра <<Информатика 
и~управление>> Российской академии наук, \mbox{AVBosov@ipiran.ru}}
\footnotetext[2]{Институт проблем информатики Федерального исследовательского центра <<Информатика 
и~управление>> Российской академии наук, \mbox{AStefanovich@frccsc.ru}}

%\vspace*{8pt}



  
  \Abst{Решается задача оптимального управления для диффузионного процесса 
Ито и~линейного управ\-ля\-емо\-го выхода. Рассматриваемая постановка близка 
к~классической ли\-ней\-но-квад\-ра\-тич\-ной гауссовской задаче управления 
(linear-quadratic Gaussian (LQG) control). Отличия состоят в~том, что состояние описывается нелинейным 
дифференциальным уравнение Ито $dy_t\hm= A_t(y_t) \,dt\hm+ \Sigma_t(y_t)\,dv_t$ 
и~не зависит от управ\-ле\-ния~$u_t$, оптимизации подлежит управ\-ля\-емый 
линейный выход $dz_t\hm= a_t y_t\,dt\hm+ b_t z_t \,dt\hm+ c_t u_t \,dt\hm+ \sigma_t\, 
dw_t$. Дополнительные обобщения внесены в~квад\-ра\-тич\-ный критерий качества 
с~целью воз\-мож\-ности постановки таких задач, как отслеживание выходом 
состояния или управ\-ле\-ни\-ем~--- линейной комбинации состояния и~выхода. Для 
решения используется метод динамического программирования. Функцию 
Беллмана позволяет найти предположение о~ее структуре вида $V_t(y,z)\hm= 
\alpha_t z^2\hm+ \beta_t(y)z \hm+\gamma_t(y)$. Решение дают три 
дифференциальных уравнения для коэффициентов~$\alpha_t$, $\beta_t(y)$ 
и~$\gamma_t(y)$. Эти уравнения со\-став\-ля\-ют оптимальное решение 
рас\-смат\-ри\-ва\-емой задачи.}
  
  \KW{стохастическое дифференциальное уравнение; оптимальное управ\-ле\-ние; 
динамическое программирование; функция Беллмана; уравнение Риккати; 
линейные уравнения параболического типа}

\DOI{10.14357/19922264180314}
  
%\vspace*{4pt}


\vskip 10pt plus 9pt minus 6pt

\thispagestyle{headings}

\begin{multicols}{2}

\label{st\stat}

\section{Введение}

     Ключевые результаты в~области оптимизации стохастических 
динамических систем, со\-став\-ля\-ющие классическую теорию управления, 
получены более~40~лет назад (такова работа~[1] в~отношении задачи 
управ\-ле\-ния ли\-ней\-но-гаус\-сов\-ски\-ми стохастическими сис\-те\-ма\-ми по 
квад\-ра\-тич\-но\-му критерию). К~классической тео\-рии следует относить 
линейные модели стохастических сис\-тем и~квадратичный критерий качества. 
Это исходный базис, на котором основано множество успешно 
исследованных и~решенных задач стохастического управ\-ле\-ния 
и~оптимизации. 

Дальнейшее развитие~--- это новые модели и~критерии, но 
прежде всего это новые методы: от тео\-рии линейных регуляторов, метода 
динамического программирования и~принципа максимума к~адаптивному 
и~минимаксному подходу, импульсному управ\-ле\-нию и~т.\,д. Множество 
инноваций как в~час\-ти моделей, так и~в~час\-ти математического аппарата, 
имевших мес\-то в~по\-сле\-ду\-ющие годы, существенно обогатили тео\-рию 
управ\-ле\-ния. Но и~до настоящего времени линейные модели и~квадратичный 
критерий, несмотря на всю справедливую критику в~отношении их 
аде\-кват\-ности и~гиб\-кости, сохраняют исследовательский интерес и~находят 
современные области приложения.
     
     Не претендуя на сколь\-ко-ни\-будь полное обосно\-ва\-ние последнего 
тезиса, приведем несколько примеров, показавшихся наиболее ин\-те\-рес\-ными. 

Так, в~[2] решается ли\-ней\-но-квад\-ра\-тич\-ная за\-да\-ча в~игровой 
постановке с~запаздыванием. В~близ\-кой по модели работе~[3] задача 
управ\-ле\-ния ставится в~терминах $H_\infty$-ро\-баст\-ности. Точнее \mbox{называть} 
эту тематику $H_2/H_\infty$-управ\-ле\-ни\-ем, и~работ по этой теме очень 
много. Аккуратности ради следует уточнить, что под линейными 
понимаются модели с~мультипликативными по состоянию воз\-му\-ще\-ниями. 

Совсем другой класс моделей, особо популярных в~по\-след\-ние годы, 
составляют скачкообразные процессы. Например, линейные уравнения 
в~сочетании с~пуассоновскими скачками в~[4] используются в~моделях, 
описывающих различные показатели функционирования сетевых протоколов 
передачи данных транспортного уровня. Телекоммуникации представляют 
в~последние годы самый популярный прикладной материал для 
исследований, работ по этой проб\-ле\-ма\-ти\-ке множество, математические 
техники привлекаются самые разные и~самые современные, но и~линейным 
моделям место находится. Еще один любопытный пример исследования 
скачкообразного процесса и~оптимизации на основе квад\-ра\-тич\-но\-го критерия 
можно найти в~[5] применительно к~задаче инвестирования на финансовом 
рынке. Наконец, упомянем еще работу~[6], подводящую итог исследований 
в~отношении классической детерминированной  
ли\-ней\-но-квад\-ра\-тич\-ной задачи с~использованием техники матричных 
неравенств.
     
     В данной работе также эксплуатируются привлекательные свойства 
линейных моделей и~квад\-ра\-тич\-но\-го критерия, причем в~стохастической 
постановке. На\-прав\-ле\-ни\-ем для обобщения \mbox{выбрана} модель динамики 
сис\-те\-мы: основные усилия на\-прав\-ле\-ны на то, чтобы сделать ее нелинейной. 
Кроме того, пред\-став\-лен\-ная постановка может рас\-смат\-ри\-вать\-ся и~как 
обобщение ранее решенной задачи в~дискретном времени~[7, 8] на время 
непрерывное. В~упомянутых работах помимо собственно модельной 
постановки важна еще и~привлекаемая прикладная об\-ласть~--- 
функционирование сложных программных сис\-тем. Результатов, 
ориентированных непосредственно на такие приложения, к~настоящему 
времени пренебрежимо мало, поэтому~[7, 8]~--- это еще и~прикладное 
обоснование рас\-смат\-ри\-ва\-емой далее задачи.
     
     Оптимизируемая динамическая сис\-те\-ма описывается двумя 
уравнениями. Состояние задается нелинейным стохастическим 
дифференциальным уравнением Ито, не содержащим управ\-ля\-емой 
переменной. Возмущение здесь описывается стандартным винеровским 
процессом, накладываются простые условия существования 
и~един\-ст\-вен\-ности решения. Поскольку состояние не управ\-ля\-ет\-ся, то уместно 
его интерпретировать как слож\-ное внешнее возмущение. Вторая 
переменная~--- управ\-ля\-емый выход~--- задается линейным стохастическим 
дифференциальным уравнением. Цель оптимизации выхода формируется 
квадратичным критерием общего вида. Формальная постановка задачи 
приведена в~сле\-ду\-ющем разделе.
     
     Для решения задачи используется метод динамического 
программирования, решается уравнение Беллмана~[9]. Соответственно, 
в~результате получаются аналитические выражения и~для оптимального 
управ\-ле\-ния, и~для значения функционала качества. Технически 
традиционный, стандартный подход к~задаче обременен, пожалуй, 
единственной проблемой~--- поиском верного пред\-став\-ле\-ния структуры 
функции Беллмана. Справиться с~этой проблемой в~большей степени удается 
за счет результата, полученного при решении дискретного по времени 
аналога рассматриваемой постановки~\cite{8-bos}. Конечные соотношения 
для оптимального решения, как и~во всех подобных задачах, включая 
классическую ли\-ней\-но-квад\-ра\-тич\-ную, содержат решения 
определенных дифференциальных уравнений (обыкновенных и~в~частных 
производных). Вывод этих уравнений и~со\-став\-ля\-ет содержание первой час\-ти 
данной работы. Во второй части будет обсуждаться их приближенное 
чис\-лен\-ное решение и~компьютерные эксперименты.
     
     Кратко обозначим основные положения, при\-вле\-ка\-емые далее 
к~решению задачи, следуя в~основном обозначениям 
и~терминологии~\cite{9-bos}, а~именно: будем рассматривать задачу 
оптимального управления в~стохастической динамической сис\-те\-ме по полной 
информации, применяя метод динамического программирования. В~качестве 
целевого функционала, опре\-де\-ля\-юще\-го качество управ\-ле\-ния $U_0^T\hm= \{ 
u_t,\ 0\leq t\leq T\}$, выступает
     \begin{equation}
     J\left(U_0^T\right)={\sf E}\left\{ \int\limits_0^T L_t \left(x_t, u_t\right)\,dt+ 
l\left(x_T\right)\right\}\,.
     \label{e1-bos}
     \end{equation}
Здесь ${\sf E}\{\cdot\}$~--- оператор математического ожидания; $x_t$~--- 
случайный процесс, описываемый стохастическим дифференциальным 
уравнением Ито
     \begin{equation}
     dx_t=m_t\left( x_t, u_t\right) dt+ \sigma_t\left( x_t\right)dW_t\,,\enskip 
x_0=X\,,
     \label{e2-bos}
     \end{equation}
где $W_t$~--- стандартный винеровский процесс подходящей раз\-мер\-ности; 
$X$~--- случайный вектор.

     $U_0^T$ будем выбирать из класса допустимых неупреждающих (по 
отношению к~$W_t$) управлений~\cite{9-bos}. Соответственно, 
относительно функций сноса и~диффузии~$m_t$ и~$\sigma_t$  
в~(\ref{e2-bos}) будем предполагать выполненными ка\-кие-ли\-бо условия 
существования сильного решения для заданного до\-пус\-ти\-мо\-го управ\-ле\-ния. 
Например, для управ\-ле\-ния с~обратной связью $u_t\hm= u_t(x_t)$ будем 
считать, что $m_t(x,u_t(x))$ и~$\sigma_t(x)$ удовлетворяют условию 
линейного рос\-та и~локальному условию Липшица по~$x$ равномерно 
по~$t$ (т.\,е.\ условиям Ито).
     
     Для поиска оптимального управления, минимизирующего $J(U_0^T)$, 
рас\-смат\-ри\-ва\-ет\-ся функция Беллмана
     \begin{equation}
     V_t(x)=\left.\mathop{\mathrm{inf}}\limits_{U_t^T} {\sf E} \left\{ \int\limits_t^T 
L_t \left( x_t, u_t\right)\,dt+l\left( x_T\right) \right\vert \mathcal{F}_t^x\right\}\,,
     \label{e3-bos}
     \end{equation}
где $\mathcal{F}_t^x$~--- $\sigma$-ал\-геб\-ра, по\-рож\-ден\-ная~$x_\tau$, 
$0\hm\leq \tau\hm\leq t$, ${\sf E}\{\cdot\vert \mathcal{F}\}$~--- оператор условного 
математического ожидания относительно~$\mathcal{F}$. Соответственно, 
в~качестве достаточного условия оп\-ти\-маль\-ности воспользуемся уравнением 
динамического программирования
\begin{multline}
\fr{\partial V_t(x)}{\partial t} +\fr{1}{2}\sum\limits^n_{i,j=1} \sigma^2_{t_{ij}}
\fr{\partial^2 V_t(x)}{\partial x_i \partial x_j}+{}\\
{}+\min\limits_u\left[  
\sum\limits^n_{i=1} m_{t_i} \fr{\partial V_t(x)}{\partial x_i} + L_t(x,u)\right] 
=0\,,\\
V_T(x)=l(x)\,,
\label{e4-bos}
\end{multline}
где $m_{t_i}$~--- $i$-й элемент век\-тор-функ\-ции~$m_t(x,u)$; 
$\sigma^2_{t_{ij}} \hm= \sum\nolimits^m_{k=1} 
\sigma_{t_{ik}}\sigma_{t_{ki}}$, $\sigma_{t_{ij}}$~--- $i$-й по строке, $j$-й 
по столб\-цу элемент мат\-рич\-ной функции~$\sigma_t(x)$; $n$ и~$m$~--- 
размерности~$x_t$ и~$W_t$ соответственно.

     Традиционно в~рамках применения метода динамического 
программирования будем предполагать, что функции~$L_t$, $l$, $m_t$ 
и~$\sigma_t$ обеспечивают существование хотя бы одного решения 
уравнения~(\ref{e4-bos}), а~следовательно, и~оптимального 
управления~$u_t^*$, $0\hm\leq t\hm\leq T$, до\-став\-ля\-юще\-го минимум 
целевому функционалу~(\ref{e1-bos}). Задача оптимизации далее получается 
путем указания конкретных выражений для~$L_t$, $l$, $m_t$ и~$\sigma_t$.

\section{Постановка задачи управления выходом}

     Рассматриваемые далее случайные функции будут предполагаться 
скалярными. Такое упрощение позволит разгрузить выкладки и~итоговые 
выражения от не самых существенных деталей.
     
     Рассмотрим стохастическую дифференциальную сис\-те\-му, со\-сто\-яние 
которой представляет диффузи\-он\-ный процесс~$y_t$, описываемый 
нелинейным стохастическим дифференциальным уравнением Ито
     \begin{equation}
     dy_t=A_t\left( y_t\right) dt +\Sigma_t \left( y_t\right) dv_t\,,\enskip 
y_0=Y\,,
     \label{e5-bos}
     \end{equation}
где $v_t$~--- стандартный (одномерный) винеровский процесс; $Y$~--- 
случайная величина с~конечным вторым моментом; функции~$A_t$ 
и~$\Sigma_t$ удовлетворяют условиям Ито:
\begin{equation*}
\left\vert A_t(y)\right\vert +\left\vert \Sigma_t(y)\right\vert \leq C(1+\vert y\vert )\ 
\mbox{для\ всех } 0\leq t\leq T\,;
\end{equation*}

\vspace*{-12pt}

\noindent
\begin{multline*}
\hspace*{-2.10051pt}\left\vert A_t\left(y_1\right) -A_t \left( y_2\right) \right\vert +\left\vert 
\Sigma_t\left( y_1\right) -\Sigma_t \left(y_2\right)\right\vert \leq
C\left\vert y_1-y_2\right\vert\\
 \mbox{для\ всех\ } 0\leq t\leq T\ \mbox{и } 
y_1,y_2\in \mathbb{R}^1\,,
\end{multline*}
обеспечивающим существование единственного сильного (потраекторного) 
решения уравнения.
     
     Будем считать, что~$y_t$ описывает состояние некоторой 
динамической системы. Соответственно, поведение этой сис\-те\-мы опишем 
выходом, линейно связанным с~со\-сто\-янием:
     \begin{equation}
     dz_t=a_t y_t \,dt+ b_t z_t \,dt+ c_t u_t \,dt+\sigma_t \,dw_t\,,\enskip
     z_0=Z\,.
     \label{e6-bos}
     \end{equation}
Здесь $w_t$~--- не зависящий от~$v_t$, $Y$ и~$Z$ стандартный (одномерный) 
винеровский процесс; $Z$~--- случайная величина с~конечным вторым 
моментом; $u_t$~--- допустимое неупреждающее управ\-ле\-ние, качество 
которого определяется целевым функционалом следующего вида:
\begin{multline}
\!\hspace*{-3.98538pt}J\left( U_0^T\right) ={\sf E}\left\{ \int\limits_0^T \!\left( S_t\left( s_ty_t-g_t z_t -h_t 
u_t\right)^2 +G_t z_t^2+{}\right.\right.\\
\left.\left.{}+ H_t u_t^2
\vphantom{S_t\left( s_ty_t-g_t z_t -h_t 
u_t\right)^2}
\right) dt+S_T\left( s_T y_T -g_T 
z_T\right)^2+G_T z_T^2
\vphantom{\int\limits_0^T}\right\}\,,
\label{e7-bos}
\end{multline}
где $S_t$, $G_t$ и~$H_t$~--- неотрицательные функции\linebreak
$0\hm\leq t\hm\leq T$. 
Такой критерий отражает физический смысл задачи распределения ресурсов 
со\-глас\-но аналогичной~(\ref{e5-bos})--(\ref{e7-bos}) задаче для дис\-крет\-но\-го 
времени, рас\-смот\-рен\-ной в~\cite{7-bos}. В~част\-ности,  
функци\-онал~(\ref{e7-bos}) поз\-во\-ля\-ет ставить задачи отслеживания
 выходом 
со\-сто\-яния сис\-те\-мы, используя сла\-га\-емое $(y_t\hm- z_t)^2$, или 
управлением~--- линейной комбинации со\-сто\-яния и~выхода, сла\-га\-емое типа\linebreak 
$(y_t\hm+ z_t\hm- u_t)^2$. Поскольку задача формулируется 
в~предположении наличия пол\-ной информации о~со\-сто\-янии~$y_t$ 
и~выходе~$z_t$ (соответствующую $\sigma$-ал\-геб\-ру 
обозначим~$\mathcal{F}_t^{y,z}$), то допустимое управ\-ле\-ние ищется 
в~классе~$\mathcal{F}_t^{y,z}$-из\-ме\-ри\-мых неупреждающих функций 
(и,~как будет показано далее, оказывается управ\-ле\-ни\-ем с~обратной связью).

     Функции~$a_t$, $b_t$, $c_t$ и~$\sigma_t$ будем предполагать 
ограниченными: $\vert a_t\vert \hm+ \vert b_t\vert \hm+\vert c_t\vert \hm+ \vert 
\sigma_t \vert \hm\leq C$ для всех $0\hm\leq t\hm\leq T$, процесс  
управления~--- допустимым не\-упреж\-да\-ющим~\cite{9-bos}, обеспечивая, 
таким образом, существование сильного решения урав\-не\-ния~(\ref{e6-bos}) 
для любого допустимого управ\-ления.
     
     Задачу составляет поиск~$u_t^*$~--- допустимого управ\-ле\-ния, 
доставляющего минимум квад\-ра\-тич\-но\-му функционалу~$J(U_0^T)$.
      
     Поставленная задача очевидным образом формулируется в~терминах 
введенных выше в~(\ref{e1-bos})--(\ref{e3-bos}) обозначений, а~именно: 
     требуется обозначить
     \begin{gather*}
      x_t=\begin{pmatrix}
     y_t\\ z_t\end{pmatrix};\quad  m_t(x_t, u_t)=\begin{pmatrix}
     A_t(y_t)\\ a_t y_t +b_t z_t +c_t u_t\end{pmatrix};\\
     \sigma_t(x_t)= \begin{pmatrix}
     \Sigma_t(y_t)& 0\\
     0& \sigma_t\end{pmatrix};\quad W_t=\begin{pmatrix}
     v_t \\ w_t\end{pmatrix}
     %     \label{e8-bos}
     \end{gather*}
для записи уравнения со\-сто\-яния типа~(\ref{e2-bos}) и
\begin{align*}
L_t(x,u)&= L_t(y,z,u) ={}\\
&\hspace*{3mm}{}=S_t\left( s_t y-g_t z -h_t u\right)^2 +G_t z^2 +H_t  u^2\,;\\
l(x)&= l(y,z) =S_T \left( S_T y-g_T z\right)^2 +G_T z^2
%\label{e9-bos}
\end{align*}
для записи целевого функционала в~виде~(\ref{e1-bos}).

     Функция Беллмана~(\ref{e3-bos}) принимает вид 
     $V_t(x)\hm= V_t(y,z)$. Для записи со\-от\-вет\-ст\-ву\-юще\-го~(\ref{e4-bos}) 
уравнения Беллмана для~$V_t(y,z)$ заметим, что
     $$
     \left( \sigma^2_{t_{ij}}\right)_{i,j=1,2}= \begin{pmatrix}
     \Sigma_t^2(y) & 0\\
     0 & \sigma_t^2\end{pmatrix}\,.
     $$
     
     С~учетом перечисленных обозначений урав\-не\-ние динамического 
программирования~(\ref{e4-bos}) принимает вид:
     \begin{multline}
     \fr{\partial V_t(y,z)}{\partial t} +\fr{1}{2}\left( \Sigma_t^2(y) \fr{\partial^2 
V_t(y,z)} {\partial y^2}+\sigma_t^2\fr{\partial^2 V_t(y,z)} {\partial 
z^2}\right)+{}\\
    {}+\min\limits_u\! \left[ A_t(y) \fr{\partial V_t(y,z)}{\partial y}+\left( a_t 
y+b_t z+c_t u\right) \fr{\partial V_t(y,z)}{\partial z} +{}\right.\hspace*{-3pt}\\
\left.{}+ S_t\left( s_t y-g_t z-h_t 
u\right)^2+G_t z^2+H_t u^2
     \vphantom{\fr{\partial V_t(y,z)}{\partial y}}\right] =0\,,\\
     V_T(y,z)=S_T\left( s_T y-g_T z\right)^2+G_T z^2\,.
     \label{e10-bos}
     \end{multline}
     Это и~есть то самое уравнение, которое требуется решить: 
существование решения данного урав\-не\-ния суть достаточное условие 
оптимальности; оптимальное управ\-ле\-ние при этом~--- точ\-ка минимума 
со\-от\-вет\-ст\-ву\-юще\-го сла\-га\-емого.
     
\section{Динамическое программирование и~оптимальное 
управление}

     В рассматриваемой постановке линейность\linebreak выхода и~квадратичность 
критерия дают те же преимущества, что и~в~классической  
ли\-ней\-но-квад\-ра\-тич\-ной задаче управ\-ле\-ния~\cite{1-bos}, а~именно: 
позволяют сразу определить вид оптимального управ\-ле\-ния и~фактические 
условия его существования. Действительно, со\-хра\-няя в~(\ref{e10-bos}) под 
знаком $\min\nolimits_u$ только члены, зависящие от~$u$, получаем
     \begin{multline*}
     \fr{\partial V_t(y,z)}{\partial t} +\fr{1}{2}\left( \Sigma_t^2(y) \fr{\partial^2 
V_t(y,z)} {\partial y^2}+\sigma_t^2\fr{\partial^2 V_t(y,z)} {\partial 
z^2}\right)+{}\\
     {}+A_t(y)\fr{\partial V_t(y,z)}{\partial y}+\left( a_t y+b_t z\right) 
\fr{\partial V_t(y,z)}{\partial z}+{}\\
{}+S_t\left( s_t y-g_t z\right)^2 +G_t z^2+{}
\end{multline*}

\noindent
\begin{multline*}
     {}+\min\limits_u \left[ \left( c_t \fr{\partial V_t(y,z)}{\partial z}-2S_t \left( 
s_t y-g_t z\right) h_t\right)u +{}\right.\\
\left.{}+\left( S_t h_t^2+H_t\right) u^2
\vphantom{\fr{\partial V_t(y,z)}{\partial z}}
\right]=0\,,
     %\label{e11-bos}
     \end{multline*}
откуда в~предположении $S_t h_t^2\hm+ H_t\hm>0$ следует, что существует 
оптимальное управ\-ле\-ние, которое определяется равенством
\begin{multline}
u_t^* = u_t^*(y,z)=-\fr{1}{2}\left( S_t h_t^2 +H_t\right)^{-1} \left( c_t 
\fr{\partial V_t(y,z)}{\partial z}-{}\right.\\
\left.{}-2S_t\left( s_t y-g_t z\right) h_t
\vphantom{\fr{\partial V_t(y,z)}{\partial z}}
\right)
\label{e12-bos}
\end{multline}
и доставляет минимум соответствующему сла\-га\-емо\-му в~урав\-не\-нии Беллмана, 
равный
$-\left( S_t h_t^2\hm+\right.$\linebreak
$\left.{}+H_t\right)^{-1} \left( c_t 
{\partial V_t(y,z)}/{\partial 
z}\hm-2S_t\left( s_t y \hm-g_t z\right) h_t \right)^2/4.
$ 
     
     Отметим, что, как и~в~классической ли\-ней\-но-квад\-ра\-тич\-ной 
задаче, управ\-ле\-ние из класса до\-пус\-ти\-мых не\-упреж\-да\-ющих получилось 
управ\-ле\-ни\-ем с~обратной связью.
     
     Таким образом, функция Беллмана описывается сле\-ду\-ющим 
дифференциальным уравнением:
     \begin{multline}
     \fr{\partial V_t(y,z)}{\partial t} +\fr{1}{2}\left( \Sigma_t^2(y) \fr{\partial^2 
V_t(y,z)} {\partial y^2}+\sigma_t^2\fr{\partial^2 V_t(y,z)} {\partial 
z^2}\right)+{}\\
     {}+ A_t(y) \fr{\partial V_t(y,z)}{\partial y}+\left( a_t y+b_t z\right) 
\fr{\partial V_t(y,z)}{\partial z}+{}\\
{}+ S_t \left( s_t y- g_t z\right)^2 +G_t z^2-
 \fr{1}{4}\left( S_t h_t^2+H_t\right)^{-1}\times{}\\
 {}\times \left( c_t \fr{\partial V_t(y,z)} 
{\partial z}-2S_t\left( s_t y -g_t z\right) h_t \right)^2=0\,.
     \label{e13-bos}
     \end{multline}
     
     Возводя в~квадрат по\-след\-нее сла\-га\-емое в~(\ref{e13-bos}), перепишем 
его в~виде:
     \begin{multline}
     \fr{\partial V_t(y,z)}{\partial t} +\fr{1}{2}\left( \Sigma_t^2(y) \fr{\partial^2 
V_t(y,z)} {\partial y^2}+\sigma_t^2\fr{\partial^2 V_t(y,z)} {\partial 
z^2}\!\right)+{}\\
{}+A_t(y) \fr{\partial V_t(y,z)}{\partial y}
+ \left( 
\vphantom{\left( S_t h_t^2 +H_t\right)^{-1}}
a_t y+b_t z+{}\right.\\
\left.{}+\left( S_t h_t^2 +H_t\right)^{-1}
 c_t S_t \left( s_t y-g_t z\right) h_t
\right) 
     \fr{\partial V_t(y,z)}{\partial z}+{}\\
     {}+\left( S_t-\left( S_t h_t^2 +H_t\right)^{-1} S_t^2 h_t^2\right)\left( s_t y -
g_t z\right)^2+{}\\
     \!\!{}+
     G_t z^2 -\fr{1}{4}\left( S_t h_t^2+H_t\right)^{-1}\! c_t^2
     \left(\fr{\partial V_t(y,z)}{\partial z}\right)^{\!2}=0\,.\!\!
     \label{e14-bos}
     \end{multline}
     
     Рассматривая полученное уравнение, заметим, что его решение может 
быть пред\-став\-ле\-но в~виде:
   \begin{equation}
     V_t(y,z)= \alpha_t z^2+\beta_t(y) z +\gamma_t(y)\,,
     \label{e15-bos}
     \end{equation}
т.\,е.\ будем искать решение при дополнительном предположении 
о~квад\-ра\-тич\-ности функции Белл\-ма\-на по переменной~$z$, и~сведем, таким 
образом, поиск оптимального решения к~уравнениям относительно функций 
$\alpha_t$, $\beta_t(y)$ и~$\gamma_t(y)$. Отметим сразу, что явный вид 
функции~$\gamma_t(y)$ для реализации оптимального управ\-ле\-ния не 
требуется, однако далее будет предложен вариант вы\-чис\-ле\-ния и~этой 
функции, что пред\-став\-ля\-ет\-ся небесполезным, поскольку позволит выполнять 
расчет минимума целевого функционала. Источником для 
предложения~(\ref{e15-bos}) является уже упоминавшаяся аналогичная 
задача для случая дис\-крет\-но\-го времени~\cite{7-bos, 8-bos}. В~той задаче 
выражение для функции Беллмана получается формально без 
дополнительных усилий. При этом форма~(\ref{e15-bos}) обнаруживается 
как свойство оптимального решения. В~рассматриваемом случае 
непрерывного времени~(\ref{e15-bos}) постулируется, а~пра\-виль\-ность 
постулата под\-тверж\-да\-ет\-ся далее ре\-зуль\-ти\-ру\-ющи\-ми уравнениями 
для~$\alpha_t$, $\beta_t(y)$ и~$\gamma_t(y)$ Кроме того, данное 
предположение пред\-став\-ля\-ет\-ся вы\-те\-ка\-ющим из линейной структуры задачи 
в~отношении переменной~$z$, в~част\-ности, тем фактом, что такой вид 
функции Беллмана обеспечивает линейность оптимального 
управ\-ле\-ния~(\ref{e12-bos}) по~$z$.

     Граничное условие при выбранном предположении~(\ref{e15-bos}) 
принимает вид:

\noindent
     \begin{multline*}
     V_T(y,z)= S_T\left( s_T y- g_T z\right)^2+G_T z^2 ={}\\[-0.5pt]
     {}=\alpha_T z^2 
+\beta_T(y) z +\gamma_T(y)\,,
    \end{multline*}
т.\,е.

\noindent
\begin{align*}
\alpha_T&= S_T g_T^2 +G_T\,;\\[-0.5pt]
\beta_T(y)&=-2S_T s_T g_T y\,;\\[-0.5pt]
\gamma_T(y)&=S_T s_T^2 y^2\,.
%\label{e16-bos}
\end{align*}
          При этом само оптимальное управ\-ле\-ние, определенное 
выражением~(\ref{e12-bos}), оказывается управ\-ле\-ни\-ем с~обратной связью 
по~$y_t$ и~$z_t$:

\noindent
     \begin{multline}
     u_t^*=u_t^*(y,z) ={}\\[-0.5pt]
     {}=
     -\fr{1}{2}\left( S_t h_t^2 +H_t\right)^{-1}
     \left( c_t \left( 2\alpha_t z +\beta_t(y)\right) +{}\right.\\[-0.5pt]
    \left. {}+2S_t\left( s_t y-g_t z\right) 
h_t\right)\,.
     \label{e17-bos}
     \end{multline}
          Подставляем $V_t(y,z)\hm= \alpha_t z^2 \hm+ \beta_t(y) 
z\hm+\gamma_t(y)$ в~(\ref{e14-bos}):

\noindent
     \begin{multline*}
     \fr{\partial \alpha_t}{\partial t}\, z^2 +
     \fr{\partial \beta_t(y)}{\partial t}\,z +
     \fr{\partial \gamma_t(y)}{\partial t}+{}\\[-0.5pt]
     {}+\fr{1}{2}\left( \Sigma_t^2(y) \left( 
\fr{\partial^2\beta_t(y)}{\partial y^2}\,z +\fr{\partial^2 \gamma_t(y)}{\partial 
y^2}\right) +2\sigma_t^2\alpha_t\right)+{}\\[-0.5pt]
 {}+A_t(y)\left(\fr{\partial \beta_t(y)}{\partial y}\,z + \fr{\partial 
\gamma_t(y)}{\partial y}\right) +{}\\[-0.5pt]
\hspace*{-0.22987pt}{}+\left( a_t y+b_t z+\left( S_t h_t^2 +H_t\right)^{-1} c_t S_t \left( s_t y-
g_t z\right) h_t\right)\times{}
\end{multline*}

\noindent
\begin{multline*}
         {}\times \left( 2\alpha_t z+\beta_t(y)\right)+{}\\
     {}+\left( S_t-\left( S_t h_t^2 +H_t\right)^{-1} S_t^2 h_t^2\right)\left( s_t y-
g_t z\right)^2+{}\\
     {}+ G_t z^2 -\fr{1}{4}\left( S_t h_t^2 +H_t\right)^{-1} c_t^2 \left( 
2\alpha_t z+\beta_t(y)\right)^2=0\,.
     \end{multline*}
          Далее выделяем слагаемые при~$z^2$, $z$ и~$z^0$
          
          \noindent
     \begin{multline*}
     \fr{\partial \alpha_t}{\partial t}\, z^2 +\fr{\partial \beta_t(y)}{\partial t}\,z +
     \fr{\partial \gamma_t(y)}{\partial 
t}+\fr{1}{2}\,\Sigma_t^2(y)\fr{\partial^2\beta_t(y)}{\partial y^2}\,z+ {}\\
{}+
\fr{1}{2}\,\Sigma_t^2(y)\fr{\partial^2\gamma_t(y)}{\partial 
y^2}+\sigma_t^2\alpha_t+A_t(y)\fr{\partial \beta_t(y)}{\partial y}\,z +{}\\
{}+A_t(y) \fr{\partial 
\gamma_t(y)}{\partial y}+{}\\
{}+ 2\alpha_t \left( b_t -\left( S_t h_t^2+H_t\right)^{-1} c_t 
S_t h_t g_t \right)z^2+{}\\
     {}+
     \left( 2\alpha_t\left( \alpha_t+\left( S_t h_t^2+H_t\right)^{-1} c_t S_t h_t 
s_t\right)y +{}\right.\\
\left.{}+\beta_t(y) \left( b_t-\left( S_t h_t^2+H_t\right)^{-1} c_t S_t h_t 
g_t\right) \right) z+{}\\
     {}+\beta_t(y)\left( a_t +\left( S_t h_t^2+H_t\right)^{-1} c_t S_t h_t s_t\right) 
y+{}\\
{}+ \left( S_t -\left( S_t h_t^2+H_t\right)^{-1} S_t^2 h_t^2\right) g_t^2 z^2-{}\\
     {}- 2\left( S_t -\left( S_t h_t^2+H_t\right)^{-1} S_t^2 h_t^2\right) s_t g_t yz 
+{}\\
{}+
     \left( S_t-\left( S_t h_t^2+H_t\right)^{-1} S_t^2 h_t^2\right) s_t^2 y^2+{}\\
     {}+G_t z^2 -\left( S_t h_t^2 +H_t\right)^{-1} c_t^2 \alpha_t^2 z^2 -{}\\
     {}-\left( 
S_t h_t^2+H_t\right)^{-1} c_t^2 \alpha_t \beta_t(y) z-{}\\
{}-
\fr{1}{4}\left( S_t h_t^2+H_t\right)^{-1}  c_t^2 \beta_t^2(y)=0\,,
     \end{multline*}
группируем их и~получаем сле\-ду\-ющие уравнения:
\begin{itemize}
\item  для~$\alpha_t$:

\noindent
\begin{multline}
\fr{\partial\alpha_t}{\partial t}+2\alpha_t\left( b_t-\left( S_t h_t^2+H_t\right)^{-1} c_t 
S_t h_t g_t\right)+{}\\
{}+ \left( S_t- \left( S_t h_t^2+H_t\right)^{-1} S_t^2 h_t^2\right) 
g_t^2+G_t-{}\\
\hspace*{-8mm}{}-\left( S_t h_t^2+H_t\right)^{-1} c_t^2 \alpha_t^2 =0\,,\enskip \alpha_T=S_T 
g_t^2+G_T\,;\!\!
\label{e18-bos}
\end{multline}
\item для $\beta_t$:

\noindent
\begin{multline}
\fr{\partial\beta_t(y)}{\partial 
t}+\fr{1}{2}\,\Sigma_t^2(y)\fr{\partial^2\beta_t(y)}{\partial y^2} 
+A_t(y)\fr{\partial \beta_t(y)}{\partial y}+{}\\
{}+ 2\alpha_t\left( a_t +\left( S_t h_t^2+H_t\right)^{-1} c_t S_t h_t s_t\right) y+{}\\
{}+
\beta_t(y)\left( b_t -\left( S_t h_t^2 +H_t\right)^{-1} c_t S_t h_t g_t\right)-{}\\
{}-2\left( S_t-\left( S_t h_t^2+H_t\right)^{-1} S_t^2 h_t^2\right) s_t g_t y-{}
\\
{}-
\left( S_t h_t^2+H_t\right)^{-1} c_t^2 \alpha_t \beta_t(y)=0\,,\\
\beta_T(y)=-2S_T s_T g_T y\,;
\label{e19-bos}
\end{multline}
\item  для $\gamma_t$:
\begin{multline}
\hspace*{-0.8pt}\fr{\partial \gamma_t(y)}{\partial t}+\fr{1}{2}\,\Sigma_t^2(y)
\fr{\partial^2 \gamma_t(y)}{\partial y^2} +\sigma_t^2 \alpha_t +A_t(y)
\fr{\partial \gamma_t(y)}{\partial y}+{}\\
{}+ \beta_t(y)\left( a_t +\left( S_t h_t^2+H_t\right)^{-1} c_t S_t h_t s_t\right) y+{}\\
{}+
\left( S_t-\left( S_t h_t^2+H_t\right)^{-1} S_t^2 h_t^2\right)  s_t^2 y^2-{}\\
{}-\fr{1}{4}\left( S_t h_t^2+H_t\right)^{-1} c_t^2 \beta_t^2(y) =0\,,\\
\gamma_T(y)=S_T s_T^2 y^2\,.
\label{e20-bos}
\end{multline}
\end{itemize}
     
     Уравнение~(\ref{e18-bos}), легко заметить, является уравнением 
Риккати, которое в~силу сформулированного выше условия   
имеет единственное неотрицательное решение для всех $0\hm\leq t\hm\leq T$. 
Этот факт требует дополнительного комментария. Для получения 
уравнения~(\ref{e18-bos}) рас\-смот\-рим исходную задачу при дополнительных 
условиях $a_t\hm=0$ и~$s_t\hm=0$ для всех $0\hm\leq t\hm\leq T$. Нетрудно 
видеть, что эти условия рассматриваемую по\-ста\-нов\-ку сводят фактически 
к~классической ли\-ней\-но-квад\-ра\-тич\-ной задаче. Имеющуюся 
в~рассматриваемой формулировке чуть более общую форму целевой 
функции (принципиального значения это обобщение, конечно, не имеет) 
сведем к~классической еще одним предположением: $S_t\hm=0$ для всех 
$0\hm\leq t\hm\leq T$. Теперь уравнение для~$\alpha_t$ принимает хорошо 
известный вид:
     \begin{equation}
     \fr{\partial \alpha_t}{\partial t}+2\alpha_t b_t +G_t- H_t^{-1} c_t^2 
\alpha_t^2=0\,,\enskip \alpha_T=G_T\,.
     \label{e21-bos}
     \end{equation}

     В таком случае, как известно~\cite{10-bos}, существует единственное 
оптимальное управление~--- линейное с~обратной связью по выходу~$z_t$, 
с~коэффициентом усиления, опи\-сы\-ва\-емым уравнением  
Риккати~(\ref{e21-bos}). Именно этот результат дают  
уравнения~(\ref{e18-bos})--(\ref{e20-bos}) и~описываемая ими функция 
Беллмана~(\ref{e15-bos}), так как из $a_t\hm=0$ и~$s_t\hm=0$ немедленно 
следует, что $\beta_t(y)\hm=0$, откуда, в~свою очередь, с~учетом 
не\-за\-ви\-си\-мости решения от~$y_t$ следует, что $\gamma_t(y)\hm=\gamma_t$, 
т.\,е.\ не зависит от~$y$ и~задается уравнением: 
     $$
     \fr{\partial \gamma_t(y)}{\partial t} +\sigma^2_t \alpha_t=0\,,\enskip 
\gamma_T=0\,.
     $$ 
     Оптимальное управ\-ле\-ние при этом 
     $$
     u_t^*= -H_t^{-1} c_t \alpha_t z_t\,,
     $$
      т.\,е.\ все полностью совпадает с~известным классическим решением.
     
     С уравнениями~(\ref{e19-bos}) и~(\ref{e20-bos}) ситуация, естественно, 
обстоит сложнее. Это линейные уравнения второго порядка параболического 
типа, поскольку\linebreak
 $\Sigma_t^2(y)\hm>0$. Фактически отсутствуют 
конструктивные условия, гарантирующие существование их\linebreak
 решений 
(требовать, чтобы все фигурирующие в~уравнениях коэффициенты были 
представлены аналитическими функциями на всем пространстве значений, 
вряд ли целесообразно), поэтому далее будем предполагать, что данные 
уравнения имеют на рас\-смат\-ри\-ва\-емом интервале $0\hm\leq t\hm\leq T$ хотя 
бы одно ограниченное решение и~именно эти условия будем рас\-смат\-ри\-вать 
как достаточные условия существования оптимального решения 
рассматриваемой задачи.
     
     Таким образом, доказана следующая тео\-рема.
     
     \smallskip
     
     \noindent
     \textbf{Теорема.}\ \textit{Пусть для диффузионного 
процесса}~(\ref{e5-bos}) \textit{выполнены условия Ито, для 
     процесса}~(\ref{e6-bos})~--- \textit{ограничены коэффициенты, 
уравнения}~(\ref{e18-bos})--(\ref{e20-bos}) \textit{имеют ограниченные 
решения для $0\hm\leq t\hm\leq T$. Тогда минимум  
функционалу}~(\ref{e7-bos}) \textit{доставляет оптимальное 
управ\-ле\-ние}~(\ref{e17-bos}), \textit{где} $y\hm= y_t$; $z\hm=z_t$.
     
\section{Заключение}

     Рассмотренная задача оптимизации в~целом близка и~по модели, и~по 
критерию к~классической ли\-ней\-но-квад\-ра\-тич\-ной постановке. 
Принципиальным отличием является нелинейная модель для описания 
со\-сто\-яния динамической сис\-те\-мы, в~которой отсутствует управ\-ля\-ющее 
воздействие.\linebreak
 Такую модель наряду с~традиционной интер\-пре\-тацией  
<<со\-сто\-яние--вы\-ход>> мож\-но понимать как\linebreak модель неконтролируемого 
слож\-но\-го внешнего воздействия. Небольшое дополнительное отличие дает 
предложенная форма квад\-ра\-тич\-но\-го критерия, поз\-во\-ля\-ющая, в~част\-ности, 
ставить такие задачи, как отслеживание выходом или управ\-ле\-ни\-ем со\-сто\-яния 
сис\-те\-мы или ее выхода.
     
     Поскольку обсуждать возможности точного решения уравнений, 
определяющих оптимальное управ\-ле\-ние, не имеет смыс\-ла, наиболее 
актуальной далее является задача их приближенного чис\-лен\-но\-го решения 
и~анализа воз\-мож\-ности практической реализации. Этому посвящена вторая 
часть данной работы, пла\-ни\-ру\-емая к~выходу в~ближайшее время.

{\small\frenchspacing
 {%\baselineskip=10.8pt
 \addcontentsline{toc}{section}{References}
 \begin{thebibliography}{99}
\bibitem{1-bos}
\Au{Athans M.} Editorial on the LQG problem~// IEEE~T. Automat. Contr., 1971. Vol.~16. 
No.\,6. P.~528--552. doi: 10.1109/TAC.1971.1099845.
\bibitem{2-bos}
\Au{Wu Z.} Forward-backward stochastic differential equations, linear quadratic stochastic 
optimal control and nonzero sum differential games~// J.~Syst. Sci. Complex., 2005. Vol.~18. 
No.\,2. P.~179--192.
\bibitem{3-bos}
\Au{Chen B.\,S., Zhang~W.} Stochastic H2/H1 control with state-dependent noise~// IEEE 
T.~Automat. Contr., 2004. Vol.~49. No.\,1. P.~45--56. doi: 10.1109/TAC.2003.821400.
\bibitem{4-bos}
\Au{Bohacek S.} A~stochastic model of TCP and fair video transmission~// IEEE 
INFOCOM, 2003. Vol.~2. P.~1134--1144. doi: 10.1109/INFCOM.2003.1208950.
\bibitem{5-bos}
\Au{Домбровский В.\,В., Объедко~Т.\,Ю.} Управление с~прогнозированием системами 
с~марковскими скачками при ограничениях и~применение к~оптимизации 
инвестиционного портфеля~// Автомат. телемех., 2011. №\,5. С.~96--112. doi: 
10.1134/S0005117911050079.
\bibitem{6-bos}
\Au{Баландин Д.\,В., Коган~М.\,М.} Оптимальное линейно-квад\-ра\-тич\-ное управление: от 
матричных уравнений к~линейным матричным неравенствам~// Автомат. телемех., 2011. 
№\,11. С.~60--69. doi: 10.1134/ S0005117911110038.
\bibitem{7-bos}
\Au{Босов А.\,В.} Обобщенная задача распределения ресурсов программной системы~// 
Информатика и~её применения, 2014. Т.~8. Вып.~2. С.~39--47. doi: 
10.14357/19922264140204.
\bibitem{8-bos}
\Au{Босов А.\,В.} Управление линейным выходом дискретной стохастической системы по 
квадратичному критерию~// Изв. РАН. Теория и~системы управления, 2016. №\,3.  
С.~19--35. doi: 10.1134/S1064230716030060.
\bibitem{9-bos}
\Au{Флеминг У., Ришел~Р.} Оптимальное управление детерминированными 
и~стохастическими системами~/ Пер. с~англ.~--- М.: Мир, 1978. 316~с. 
(\Au{Fleming~W.\,H., Rishel~R.\,W.} Deterministic and stochastic optimal control.~--- New 
York, NY, USA: Springer-Verlag, 1975. 222~p.)
\bibitem{10-bos}
\Au{Девис М.\,Х.\,А.} Линейное оценивание и~стохастическое управление~/ Пер. с~англ.~--- 
М.: Наука, 1984. 206~с. (\Au{Davis~M.\,H.\,A.} Linear estimation and stochastic control.~--- 
London: Chapman and Hall, 1977. 224~p.)

 \end{thebibliography}

 }
 }

\end{multicols}

\vspace*{-6pt}

\hfill{\small\textit{Поступила в~редакцию 30.03.18}}

\vspace*{4pt}

%\newpage

%\vspace*{-24pt}

\hrule

\vspace*{2pt}

\hrule

\vspace*{-2pt}


\def\tit{STOCHASTIC DIFFERENTIAL SYSTEM OUTPUT CONTROL 
BY~THE~QUADRATIC CRITERION.~I.~DYNAMIC\\ PROGRAMMING 
OPTIMAL SOLUTION}


\def\titkol{Stochastic differential system output control 
by~the~quadratic criterion. I.~Dynamic programming 
optimal solution}

\def\aut{A.\,V.~Bosov and~A.\,I.~Stefanovich}

\def\autkol{A.\,V.~Bosov and~A.\,I.~Stefanovich}

\titel{\tit}{\aut}{\autkol}{\titkol}

\vspace*{-11pt}


\noindent
Institute of Informatics Problems, Federal Research Center ``Computer Science 
and Control'' of the Russian Academy of Sciences, 44-2~Vavilov Str., Moscow 
119333, Russian Federation


\def\leftfootline{\small{\textbf{\thepage}
\hfill INFORMATIKA I EE PRIMENENIYA~--- INFORMATICS AND
APPLICATIONS\ \ \ 2018\ \ \ volume~12\ \ \ issue\ 3}
}%
 \def\rightfootline{\small{INFORMATIKA I EE PRIMENENIYA~---
INFORMATICS AND APPLICATIONS\ \ \ 2018\ \ \ volume~12\ \ \ issue\ 3
\hfill \textbf{\thepage}}}

\vspace*{3pt}



\Abste{The problem of optimal control for the Ito diffusion 
process and a~controlled linear output is solved. The considered 
statement is close to the classical linear-quadratic Gaussian 
control  (LQG control) problem. Differences consist in the fact 
that the state is described by the nonlinear differential Ito equation  $dy_y = A_t(y_t) 
\,dt+\Sigma_t(y_t)\,dv_t$ and does not depend on the control~$u_t$, 
optimization subject is controlled linear output 
 $dz_t=a_ty_t\,dt +b_tz_t\,dt +c_t u_t\,dt +\sigma_t \,dw_t$. 
Additional generalizations are included in the quadratic 
quality criterion for the purpose of statement such problems 
as state tracking by output or a linear combination of state 
and output tracking by control. The method of dynamic programming 
is used for the solution. 
The assumption about Bellman function in the form  $V_t(y,z)= \alpha_t 
z^2+\beta_t(y) z+\gamma_t(y)$ allows one to find it. 
Three differential equations for the coefficients $\alpha_t$,  $\beta_t(y)$,
and $\gamma_t(y)$ give the solution. 
These equations constitute the optimal solution of the problem under consideration.}

\KWE{stochastic differential equation; optimal control; dynamic programming; 
Bellman function; Riccati equation; linear differential equations of parabolic type}


\DOI{10.14357/19922264180314}

\vspace*{-12pt}

\Ack
\noindent
This work was partially supported by the Russian Science Foundation (grant  
16-07-00677).



%\vspace*{6pt}

  \begin{multicols}{2}

\renewcommand{\bibname}{\protect\rmfamily References}
%\renewcommand{\bibname}{\large\protect\rm References}

{\small\frenchspacing
 {%\baselineskip=10.8pt
 \addcontentsline{toc}{section}{References}
 \begin{thebibliography}{99}
\bibitem{1-bos-1}
\Aue{Athans, M.} 1971. Editorial on the LQG problem. \textit{IEEE~T. 
Automat. Contr.} 16(6):528--552. doi: 10.1109/ TAC.1971.1099845.
\bibitem{2-bos-1}
\Aue{Wu, Z.} 2005. Forward-backward stochastic differential equations, linear 
quadratic stochastic optimal control and\linebreak\vspace*{-12pt}

\columnbreak

\noindent
 nonzero sum differential games. 
\textit{J.~Syst. Sci. Complex.} 18(2):179--192.
\bibitem{3-bos-1}
\Aue{Chen, B.\,S. and W.~Zhang.} 2004. Stochastic H2/H1 control with  
state-dependent noise. \textit{IEEE~T. Automat. Contr.} 49(1):45--56.
doi: 10.1109/TAC.2003.821400.
\bibitem{4-bos-1}
\Aue{Bohacek, S.} 2003. A~stochastic model of TCP and fair video 
transmission. \textit{IEEE INFOCOM}. 2:1134--1144.
doi: 10.1109/INFCOM.2003.1208950.
\bibitem{5-bos-1}
\Aue{Dombrovskii, V.\,V., and T.\,Yu.~Ob''edko.} 2011. Predictive control of 
systems with Markovian jumps under constraints and its application to the 
investment portfolio optimization. \textit{Automat. Rem. Contr.}  
72(5):989--1003.
\bibitem{6-bos-1}
\Aue{Balandin, D.\,V., and M.\,M.~Kogan.} 2011. Optimal linear-quadratic 
control: From matrix equations to linear matrix inequalities. \textit{Automat. 
Rem. Contr.} 72(11):2276--2284.
\bibitem{7-bos-1}
\Aue{Bosov, A.\,V.} 2014. Obobshchennaya zadacha raspredeleniya resursov 
programmnoy sistemy [The generalized problem of software system resources 
distribution]. \textit{Informatika i~ee Primeneniya~--- Inform. Appl.}  
8(2):39--47. doi: 
10.14357/19922264140204.
\bibitem{8-bos-1}
\Aue{Bosov, A.\,V.} 2016. Discrete stochastic system linear output control 
with respect to a quadratic criterion. \textit{J.~Comput. Syst. Sc. 
Int.} 55(3):349--364.
\bibitem{9-bos-1}
\Aue{Fleming, W.\,H., and R.\,W.~Rishel.} 1975. \textit{Deterministic and 
stochastic optimal control.} New York, NY: Springer-Verlag. 222~p.
\bibitem{10-bos-1}
\Aue{Davis, M.\,H.\,A.} 1977. \textit{Linear estimation and stochastic 
control.} London: Chapman and Hall. 224~p.
\end{thebibliography}

 }
 }

\end{multicols}

\vspace*{-6pt}

\hfill{\small\textit{Received March 30, 2018}}

%\pagebreak

%\vspace*{-18pt}
     
     \Contr
     
       \noindent
       \textbf{Bosov Alexey V.} (b.\ 1969)~--- Doctor of Science in technology, 
principal scientist, Institute of Informatics Problems, Federal Research 
Center ``Computer Science and Control'' of the Russian Academy of Sciences, 
44-2~Vavilov Str., Moscow 119333, Russian Federation; 
\mbox{AVBosov@ipiran.ru}
       
       \vspace*{3pt}
       
       \noindent
       \textbf{Stefanovich Alexey I.} (b.\ 1983)~--- principal specialist, 
Institute of Informatics Problems, Federal Research Center ``Computer Science 
and Control'' of the Russian Academy of Sciences, 44-2~Vavilov Str., Moscow 
119333, Russian Federation; \mbox{AStefanovich@frccsc.ru}
\label{end\stat}

\renewcommand{\bibname}{\protect\rm Литература}       

              %2
\def\stat{pavlov}

\def\tit{РАСЧЕТ И ОПТИМИЗАЦИЯ НЕКОТОРЫХ ХАРАКТЕРИСТИК 
ДЛЯ~МОДЕЛИ ВЫЧИСЛИТЕЛЬНОГО КОМПЛЕКСА}

\def\titkol{Расчет и оптимизация некоторых характеристик 
для модели вычислительного комплекса}

\def\autkol{И.\,В.~Павлов}
\def\aut{И.\,В.~Павлов$^1$}

\titel{\tit}{\aut}{\autkol}{\titkol}

%{\renewcommand{\thefootnote}{\fnsymbol{footnote}}\footnotetext[1]
%{Работа выполнена при поддержке РФФИ (гранты 09-07-12098, 09-07-00212-а и
%09-07-00211-а) и Минобрнауки РФ (контракт №\,07.514.11.4001).}}


\renewcommand{\thefootnote}{\arabic{footnote}}
\footnotetext[1]{Московский государственный технический университет им.\ Н.\,Э.~Баумана, 
ipavlov@bmstu.ru}

\Abst{Рассматривается проблема выбора оптимального размера пакетов при обработке 
информационных задач большого объема для модели вычислительного комплекса с 
учетом возможных отказов или сбоев элементов в процессе решения задачи. Получено 
приближенное асимптотическое решение данной проблемы для случая высоконадежных 
элементов и малого времени пересылки (загрузки) пакетов.}

\KW{оптимальный размер пакета; надежность; интенсивность отказов; время пересылки 
пакетов} 

\vskip 14pt plus 9pt minus 6pt

      \thispagestyle{headings}

      \begin{multicols}{2}

            \label{st\stat}

\section{Введение}

     Пусть имеется система, включающая в себя $l$ основных 
вычислительных элементов. В~систему поступают <<задания>>, каждое из 
которых состоит из некоторого (вообще говоря, случайного) числа   
<<элементарных задач>>, каждая из которых может выполняться 
(обрабатываться) независимо от остальных на любом из этих элементов. Для 
выполнения очередного задания, поступившего в систему, необходимо 
выполнить все составляющие его элементарные задачи. При этом в процессе 
выполнения задание разбивается на некоторое количество $n$  блоков 
(<<пакетов>>) элементарных задач равного объема $\upsilon\hm=L/n$, 
$n\hm\in N$, где $N$~--- множество допустимых значений  (например, 
$N$~---  некоторое подмножество целочисленных значений, кратных~2 
и~т.\,п.). Время~$h$ выполнения одной элементарной задачи на любом из 
элементов далее будем считать равным единице: $h\hm=1$. Соответственно, 
время выполнения одного пакета объемом~$\upsilon$ на любом из элементов 
будет численно совпадать с величиной~$\upsilon$. 
{\looseness=1

}
     
     Выполнение задания происходит путем пересылки пакетов на рабочие 
элементы и дальнейшей их обработки на этих элементах. Время пересылки 
(загрузки) пакета на элемент равно величине $\tau\hm>0$, не зависит от 
размера пакета~$\upsilon$ и от состояния других элементов. Обработка пакета 
после его загрузки на данном элементе занимает время~$\tau$ и происходит 
независимо от состояния других элементов. После завершения обработки 
очередного пакета на том или ином элементе снова происходит его загрузка в 
течение времени~$\tau$ следующим пакетом (из общей очереди всех пакетов 
данного задания) независимо от состояния (работы или загрузки) остальных 
элементов и~т.\,д. Задание считается выполненным после выполнения 
(обработки) всех составляющих его пакетов. Близкие по смыслу модели и 
процессы рассматривались ранее в~[1--5].
     
     В процессе работы любой из элементов может отказывать с постоянной 
(не зависящей от времени) функцией интенсивности отказов 
$\lambda(t)\hm\equiv \lambda$~[6, 7]. Заметим, что более близким к 
реальности было бы предположение о монотонном возрастании 
(неубывании) $\lambda(t)$ по времени. Поэтому фактически здесь 
предполагается, что, по крайней мере в течение времени выполнения одного 
задания, функция интенсивности отказов~$\lambda(t)$ меняется 
незначительно и может считаться приближенно постоянной. Такое 
допущение является естественным, по крайней мере в случае высокой 
надежности элементов, когда вероятность отказа элемента за время 
выполнения в системе одного задания достаточно мала. В~указанных 
допущениях время безотказной работы элемента имеет экспоненциальное 
распределение с функцией надежности $P(t)\hm=e^{-\lambda t}$, а вероятность 
отказа элемента за время~$h$ выполнения одной элементарной задачи равна 
величине $\lambda h\hm+ o(\lambda h)$.
     
     Одной из существенных проблем, возникающих в данной ситуации, 
является выбор оптимального размера пакета~$\upsilon$ с учетом 
возможности отказов (сбоев) элементов при выполнении задания.

\section{Модель со сбоями элементов}

     Рассмотрим случай, когда возможные отказы элементов в системе 
имеют характер <<сбоев>>. Другими словами, в результате отказа (сбоя) 
элемент сам по себе не выходит из строя и продолжает работать, но 
находящийся на нем в момент сбоя пакет считается невыполненным и после 
завершения его обработки снова ставится в очередь необработанных пакетов 
и должен быть полностью обработан заново на этом же или любом другом 
элементе.
     
     Рассмотрим сначала более простой частный случай, когда число 
элементов $l\hm=1$. Обозначим через $p\hm=\exp (-\lambda \upsilon)$ 
вероятность обработки пакета объемом~$\upsilon$ без сбоев и $q\hm=1-p$. 
Время~$\eta$ выполнения всего задания объемом~$L$ имеет вид:
     \begin{equation}
     \eta=(\upsilon+\tau) v\,,
     \label{e1p}
     \end{equation}
где $v$~--- момент (номер шага) первого достижения $n$ <<успехов>> в 
классической схеме независимых ис\-пытаний Бернулли при вероятности 
<<успеха>> (на\linebreak
одном шаге) $p\hm=\exp\left( -\lambda \upsilon\right)$. Задача 
выбора оп\-тимального размера пакета~$\upsilon$ далее сводится к 
минимизации математического ожидания E$\eta$ по параметру~$\upsilon$, 
или, учитывая равенство $\upsilon\hm= L/n$, к\linebreak
 минимизации~E$\eta$ по 
переменной $n\hm\in N$, где $n$~---  чис\-ло пакетов, на которое разбивается 
задание. Случайная величина~$v$ имеет распределение Пас\-каля 
\begin{equation}
P\left( v=m\right) = C_{m-1}^{n-1} p^n q^{m-n}\,,\quad m=n, n+1, \ldots ,
\label{e2p}
\end{equation}
с математическим ожиданием E$v=n/p$, откуда с учетом~(\ref{e1p}) следует, 
что выбор оптимального размера пакета сводится к задаче: найти
\begin{equation}
\min \left( L+n\tau\right)\exp\left( \fr{\lambda L}{n}\right)
\label{e3p}
\end{equation}
по $n\in N$. Далее оптимальный размер пакета~$\tilde{\upsilon}$ 
находится по формуле $\tilde{\upsilon} =L/\tilde{n}$, где $\tilde{n}\hm\in 
N$~--- решение задачи~(\ref{e3p}). 
     
     Оптимизационная задача~(\ref{e3p}) является цело\-чис\-лен\-ной, 
поскольку множество $N$ допустимых значений $n$ содержит только 
целочисленные точки. Введем также дополнительную <<непрерывную>> 
задачу: найти
     \begin{equation}
     \min\left( L+n\tau\right) \exp \left( \fr{\lambda L}{n}\right)
     \label{e4p}
     \end{equation}
по всем (не только целочисленным) значениям $n\hm\geq 1$. Далее 
оптимальный размер пакета~$\upsilon^*$ (без ограничения целочисленности 
$n\hm\in N$) находится как $\upsilon^*=L/n^*$, где $n^*$~--- решение 
задачи~(\ref{e4p}).
     
Теорема~1 дает точное решение оптимизационных 
задач~(\ref{e3p}) и~(\ref{e4p}). Теорема~2 дает асимптотическое выражение 
для оптимального размера пакета~$\upsilon^*$.
     
     \medskip
     
     \noindent
     \textbf{Теорема 1.} \textit{Пусть $\lambda\hm>0$, $\tau\hm>0$ и 
выполняется неравенство}
     \begin{equation}
     \tau\leq \lambda L^2\,.
     \label{e5p}
     \end{equation}
\textit{Тогда минимум}~(\ref{e4p}) \textit{достигается в единственной \mbox{точке}}
$$
n^*=L\sqrt{\fr{\lambda}{\tau}}\left[  
\sqrt{1+\fr{\lambda\tau}{4}}+\fr{\sqrt{\lambda\tau}}{2}\right]\,.
$$
\textit{Минимум}~(\ref{e3p}) \textit{достигается в одной из двух ближайших 
(слева или справа) к точке~$n^*$ целочисленных точек $n\hm\in N$.} 

     \smallskip
     
     \noindent
     Д\,о\,к\,а\,з\,а\,т\,е\,л\,ь\,с\,т\,в\,о\,.\ Введем функцию
     \begin{equation}
     f(n) =\left( L+n\tau\right) \exp\left( \fr{\lambda L}{n}\right)
     \label{e6p}
     \end{equation}
от непрерывного аргумента $n\hm\geq 1$. Нетрудно показать, что знак 
производной этой функции совпадает со знаком многочлена 
$Q(n)\hm=n^2\hm-\lambda L n -\lambda L^2/\tau$, который имеет при 
$n\hm\geq 1$ единственный корень в точке $n\hm=n^*$ и для которого 
справедливы неравенства:
\begin{align*}
Q(n)<0 &\ \ \mbox{при}\ \ 1\leq n\leq n^*\,;\\
Q(n)>0 &\  \ \mbox{при}\ \ n>n^*\,,
\end{align*}
если выполняется условие~(\ref{e5p}), откуда далее и следует теорема~1. 
Теорема доказана.

\smallskip

     \noindent
     \textbf{Теорема~2.} \textit{Пусть $\lambda\hm>0$, $\tau\hm>0$, 
$\tau\hm\leq \lambda L^2$ и $\lambda\tau\hm\rightarrow 0$. Тогда} 

\noindent
     \begin{equation}
     \upsilon^*=\sqrt{\fr{\tau}{\lambda}}\left[ 1+o(1)\right]\,.
     \label{e7p}
     \end{equation}
     
     \smallskip
     
     \noindent
     Д\,о\,к\,а\,з\,а\,т\,е\,л\,ь\,с\,т\,в\,о\ следует из теоремы~1 и равенства 
$\upsilon^*\hm=L/n^*$. 
     
     \medskip
     
     Из~(\ref{e7p}) далее следует приближенная формула для оптимального 
размера пакета при $\lambda\tau\hm\ll 1$:
     $$
     \upsilon^*\cong \sqrt{\tau\theta}\,,
     $$
где $\theta=1/\lambda$~--- математическое ожидание \mbox{времени} безотказной 
работы (средний ресурс) элемента. Другими словами, оптимальный размер 
пакета~$\upsilon^*$ приближенно равен среднему геометрическому между 
временем пересылки (загрузки)~$\tau$ и средним ресурсом элемента~$\theta$ 
(при условии $\lambda\tau\hm\ll 1$). Существенно, что оптимальное 
значение~$\upsilon^*$ не зависит от размера всего задания~$L$, который, 
вообще говоря, может быть неизвестным и случайным.
     
     Рассмотрим далее общий случай $l\hm\geq 1$ элементов.
Для рассматриваемой модели время выполнения задания

\noindent
     \begin{equation}
     \eta=\left(\upsilon+\tau\right) \left(\fr{v}{l}\right)^+\,,
     \label{e8p}
     \end{equation}
где $z^+$~---  величина~$z$, округленная вверх до ближайшего целого. 
Задача сводится к вычислению
\begin{equation}
\min E\eta
\label{e9p}
\end{equation}
по $n\in N$, после чего оптимальный размер пакета~$\tilde{\upsilon}$ находится 
по формуле $\tilde\upsilon\hm=L/\tilde{n}$, где $\tilde{n}\hm\in N$~--- решение 
задачи~(\ref{e9p}). 

\pagebreak
     
     В соответствии с~(\ref{e2p}) и (\ref{e8p})
     $$
     E\eta =\left( \fr{L}{n}+\tau\right) \sum\limits_{m=n}^\infty 
     \left (\fr{m}{l}\right)^+ C_{m-1}^{n-
1} p^n q^{m-n}\,,
     $$
откуда, учитывая, что $m C_{m-1}^{n-1}=nC_m^n$,
\begin{multline}
E\eta = \left( \fr{L}{n}+\tau\right) \sum\limits_{m=n}^\infty 
\left (\fr{m}{l}\right)^+ \fr{n}{m}\,C_m^n 
p^n q^{m-n}={}\\
{}=\left( \fr{1}{l}\right) \left( L+n\tau\right) p^n \sum\limits_{m=n}^\infty 
\fr{(m/l)^+}{m/l}\,C_m^n q^{m-n}\,.
\label{e10p}
\end{multline}

В соответствии с~(\ref{e2p}) 
\begin{equation}
E v = \!\sum\limits_{m=n}^\infty m C_{m-1}^{n-1} p^n q^{m-n} =n p^n\! 
\sum\limits_{m=n}^\infty C_m^n q^{m-n}.
\label{e11p}
\end{equation}
С другой стороны, случайная величина~$v$ является суммой~$n$ 
независимых, одинаково распределенных случайных величин, каждая из 
которых имеет геометрическое распределение с параметром~$p$ и 
математическим ожиданием $1/p$. Соответственно, $Ev \hm= n/p$, откуда с 
учетом~(\ref{e11p}) следует 
\begin{equation}
\sum\limits_{m=n}^\infty C_m^n q^{m-n} =\fr{1}{p^{n+1}}\,.
\label{e12p}
\end{equation}
Из~(\ref{e10p}) и~(\ref{e12p}) следует
$$
E\eta = \fr{1}{l}\left( L+n\tau\right) p^n \sum\limits_{m=n}^\infty \left[ 
1+\fr{(m/l)^\prime}{m/l}\right] C_m^n q^{m-n}\,,
$$
где $z^\prime=z^+-z$, откуда
\begin{equation}
E\eta =\fr{1}{l}\left(L+n\tau\right)\exp \left( \fr{\lambda L}{n}\right) 
\left(1+\delta_l(n)\right)\,,
\label{e13p}
\end{equation}
где 
\begin{equation}
\delta_l(n)=\sum\limits_{m=n}^\infty \alpha_{nm} \fr{(m/l)^\prime}{m/l}\,,
\label{e14p}
\end{equation}
где коэффициенты $\alpha_{nm} =p^{n+1} C_m^n q^{m-n}$, $p\hm= e^{-
\lambda L/n}$, $q\hm=1\hm-p$. При этом в соответствии с~(\ref{e12p}) 
\begin{equation}
\sum\limits_{m=n}^\infty \alpha_{nm}=1\,.
\label{e15p}
\end{equation}
Из~(\ref{e14p}) и (\ref{e15p}) видно, что 
\begin{equation}
0<\delta_l(n)<\fr{l}{n}\,.
\label{e16p}
\end{equation}
     
     Целевая функция~(\ref{e13p}) для общего случая $l\hm\geq 1$ 
совпадает с целевой функцией в~(\ref{e3p}) для случая $l\hm=1$ с точностью 
до множителя $(1/l)\left[ 1+\delta_l(n)\right]$, откуда с учетом~(\ref{e16p}) 
видно, что полученное выше решение для случая $l\hm=1$ практически дает 
и решение для случая $l\hm>1$, если оптимальное число пакетов $n$ 
достаточно велико.
     
     Обозначим через
     \begin{equation}
     f_l(n) =\fr{f(n)}{l}\left[ 1+\delta_l(n)\right]
     \label{e17p}
     \end{equation}
целевую функцию~(\ref{e13p})~--- среднее время выполнения задания при 
данных значениях $n$~--- чис\-ле пакетов и $l$~--- чис\-ле элементов, где 
$f(n)\hm=(L+n\tau)\exp\left(\lambda L/n\right)$~--- целевая функция~(\ref{e6p}) 
для случая $l\hm=1$.
     
     Задача выбора оптимального размера пакета~$\tilde{\upsilon}$ сводится к 
нахождению 
     \begin{equation}
     \min f_l(n) =f_l\left(\tilde{n}_l\right)\,.
     \label{e18p}
     \end{equation}
Здесь минимум берется по всем $n\hm\in N$, где $N$~--- множество 
допустимых значений~$n$ (например, $N$~--- множество це\-ло\-чис\-лен\-ных 
значений~$n$, кратных~2, лежащих в некотором допустимом диапазоне, 
и~т.\,п.). Полагаем $\tilde{\upsilon}\hm=L/\tilde{n}_l$, где 
$\tilde{n}_l$~--- решение задачи~(\ref{e18p}). Введем также дополнительную 
задачу нахождения 
\begin{equation}
\min f_l(n) =f_l(n_l^*)\,,
\label{e19p}
\end{equation}
где минимум берется по всем (не только це\-ло\-чис\-лен\-ным) значениям 
$n\hm\geq 1$. Далее оптимальный размер пакета~$\upsilon_l^*$ (без 
ограничений це\-ло\-чис\-лен\-ности $n\hm\in N$) определим по формуле 
$\upsilon_l^*\hm=L/n_l^*$, где $n_l^*$~--- решение задачи~(\ref{e19p}). Из 
выражений~(\ref{e13p})--(\ref{e16p}) далее следует теорема~3.

\medskip

\noindent
\textbf{Теорема~3.} \textit{Решение оптимизационной задачи}~(\ref{e19p}) 
\textit{удовлетворяет неравенствам}
\begin{equation}
\fr{f_l(n^*)}{1+\varepsilon}\leq \min\limits_{n\geq 1} f_l(n)\leq f_l(n^*)\,,
\label{e20p} 
\end{equation}
\textit{где $n^*$~--- решение этой задачи для случая $l\hm=1$, 
$\varepsilon\hm=\delta_l(n^*)\hm<l/n^*$. Решение оптимизационной 
задачи}~(\ref{e18p}) \textit{удовлетворяет аналогичным неравенствам}
\begin{equation}
\fr{f_l(\tilde{n})}{1+\varepsilon}\leq \min\limits_{n\in N} f_l(n)\leq 
f_l(\tilde{n})\,, 
\label{e21p}
\end{equation}
\textit{где $\tilde{n}$~--- решение этой задачи для случая} $l\hm=1$, 
$\varepsilon\hm=\delta_l(\tilde{n})\hm<l/\tilde{n}$.
     
     \medskip
     
\noindent
     Д\,о\,к\,а\,з\,а\,т\,е\,л\,ь\,с\,т\,в\,о\,.\ Равенство~(\ref{e17p}) при 
$n\hm=n^*$ имеет вид:
     \begin{equation}
     f_l(n^*) =\fr{f(n^*)}{l}\left[ 1+\delta_l(n^*)\right]\,.
     \label{e22p}
     \end{equation}
Из этого же равенства, учитывая, что $\delta_l(n)\hm>0$, следует
$$
f_l(n)\geq \fr{f(n)}{l}\,,
$$
откуда с учетом~(\ref{e22p})
$$
\min\limits_{n\geq 1} f_l(n) \geq \fr{1}{l}\min\limits_{n\geq 1} f(n) 
=\fr{f(n^*)}{l}= \fr{f_l(n^*)}{1+\delta_l(n^*)}\,,
$$
что вместе с~(\ref{e16p}) доказывает левое неравенство в~(\ref{e20p}). 
Правое неравенство очевидно. Доказательство неравенств~(\ref{e21p}) 
аналогично. Теорема доказана.

\medskip

     Таким образом, полученное решение для случая $l\hm=1$ 
практически дает решение и в случае $l\hm>1$, если число пакетов много 
больше по сравнению с количеством элементов~$l$.

\section{Заключение}
     
     Получено решение указанной выше основной проблемы (выбора 
оптимального размера пакета) для модели со сбоями элементов в 
естественной с прикладной точки зрения асимптотике, а именно для случая 
высоконадежных элементов и при малом времени пересылки пакетов. 
Существенно, что полученное решение не зависит от общего объема всего 
задания, что, в частности, позволяет использовать его в ситуации 
неопределенности, когда эта величина, вообще говоря, может быть 
неизвестной и случайной. Отметим также, что представляет интерес 
дальнейшее обобщение полученных результатов на ситуацию, когда 
различные элементы могут иметь существенно различные характеристики 
как производительности, так и надежности, а также на модель с отказами и 
восстановлением (заменой) отказавших элементов. 

{\small\frenchspacing
{%\baselineskip=10.8pt
\addcontentsline{toc}{section}{Литература}
\begin{thebibliography}{9}


\bibitem{3p} %1
\Au{Ронжин А.\,Ф., Суриков В.\,Н.}
О~времени полного перебора~// Обозр. прикл. пром. матем., 2007. Т.~14. 
№\,3. С.~506--508.

\bibitem{4p} %2
\Au{Коновалов М.\,Г., Малашенко Ю.\,Е., Назарова~И.\,А.}
Модели и методы управления заданиями в системах распределенных 
вычислительных ресурсов.~--- М.: ВЦ РАН, 2009. 110~с. (Сообщения по 
прикладной математике.)

\bibitem{5p} %3
\Au{Коновалов М.\,Г., Малашенко Ю.\,Е., Назарова~И.\,А.}
Оперативное управление потоком заданий в системе распределенных 
вычислительных ресурсов~// VI Московская междунар. конф. по 
исследованию операций: ORM-2010: Труды.~--- М.: 
МАКС Пресс, 2010. С.~301--302.

\bibitem{6p} %4
\Au{Козлов М.\,В., Малашенко Ю.\,Е., Назарова~И.\,А., Ронжин~А.\,Ф.}
Анализ режимов управления вычислительным комплексом в условиях 
неопределенности.~--- М.: ВЦ РАН, 2011. 63~с. (Сообщения по прикладной 
математике.)

\bibitem{7p} %5
\Au{Коновалов М.\,Г., Малашенко Ю.\,Е., Назарова~И.\,А.}
Управ\-ле\-ние заданиями в гетерогенных вычислительных сис\-те\-мах~// 
Известия РАН. Теория и системы управления, 2011. №\,2. С.~72--90. 

\bibitem{1p} %6
\Au{Гнеденко Б.\,В., Беляев Ю.\,К., Соловьев~А.\,Д.}
Математические методы в теории надежности.~--- М.: Наука, 1965. 524~с. 

\label{end\stat}

\bibitem{2p} %7
\Au{Gnedenko B.\,V., Pavlov I.\,V., Ushakov~I.\,A.}
Statistical reliability engineering.~--- N.Y.: John Wiley, 1999. 514~p.
 \end{thebibliography}
}
}


\end{multicols}       %3
\def\stat{kovalev}

\def\tit{МЕТОДЫ ТЕОРИИ КАТЕГОРИЙ В~МОДЕЛЬНО-ОРИЕНТИРОВАННОЙ СИСТЕМНОЙ 
ИНЖЕНЕРИИ}

\def\titkol{Методы теории категорий в~модельно-ориентированной системной 
инженерии}

\def\aut{С.\,П.~Ковалёв$^1$}

\def\autkol{С.\,П.~Ковалёв}

\titel{\tit}{\aut}{\autkol}{\titkol}

\index{Ковалёв С.\,П.}
\index{Kovalyov S.\,P.}


%{\renewcommand{\thefootnote}{\fnsymbol{footnote}} \footnotetext[1]
%{Исследование выполнено при финансовой поддержке Российского научного фонда (проект 16-11-10227).}}


\renewcommand{\thefootnote}{\arabic{footnote}}
\footnotetext[1]{Институт проблем управления им.\ В.\,А.~Трапезникова 
Российской академии наук,  \mbox{kovalyov@nm.ru}}

%\vspace*{-18pt}

\Abst{Предложен математический аппарат на базе теории категорий, который позволяет 
формально описывать и~строго исследовать процедуры применения моделей в~инженерной 
деятельности, составляющие сущность мо\-дель\-но-ори\-ен\-ти\-ро\-ван\-ной системной 
инженерии (Model-Based Systems Engineering, MBSE). В~основе аппарата лежит 
математическое представление сборочных чертежей (мегамоделей сис\-тем) диаграммами 
в~категориях, объектами которых служат модели, а~морфизмы представляют действия по 
сборке моделей сис\-тем из моделей компонентов. Адекватность аппарата обоснована исходя 
из требований стандартов, регламентирующих описание структуры систем, в~том числе 
IEC~81346. Предложены и~исследованы тео\-ре\-ти\-ко-ка\-те\-гор\-ные методы решения ряда 
практических задач сборки систем. Приведены примеры решения таких задач в~категориях, 
представляющих две ключевые области применения MBSE: гео\-мет\-ри\-че\-ское моделирование 
изделий сложной формы и~дис\-крет\-но-со\-бы\-тий\-ное имитационное моделирование 
поведения технических систем.}

\KW{модельно-ориентированная системная инженерия; мегамодель; теория категорий; 
копредел}



\DOI{10.14357/19922264170305} 


\vspace*{6pt}

\vskip 10pt plus 9pt minus 6pt

\thispagestyle{headings}

\begin{multicols}{2}

\label{st\stat}

\section{Введение}

   Модельно-ориентированная системная инженерия состоит в~формализованном применении моделирования в~
поддержке жизненного цикла сис\-тем, включая сбор требований, 
проектирование, проверку и~приемку, другие стадии~[1]. Модели, 
разрабатываемые в~ходе процедур MBSE, пригодны к~автоматической 
обработке на компьютерах. Это позволяет сначала задавать, верифицировать 
и~оптимизировать проектные решения на моделях <<в циф\-ре~и только потом 
воплощать <<в железе>>, снижая затраты на организацию жизненного цикла 
изделий и~сокращая сроки выполнения работ~[2].
   
   И все же внедрение технологий MBSE в~инженерную деятельность 
происходит медленно. Это связано во многом с~нехваткой единой 
концептуальной базы инженерного моделирования: предлагается много 
частных языков и~технологий, слабо совместимых друг с~другом и~плохо 
приспособленных для совместной разработки моделей большими 
мультидисциплинарными коллективами~[3]. Тем самым затрудняется переход 
от набора электронных чертежей к~полноценному электронно-цифровому 
макету (digital mock-up) промышленного изделия.
   
   Естественный, хотя и~<<трудный>>, подход к~получению результатов 
общего характера, унифи\-ци\-ру\-ющих разнородные технологии, состоит в~том, 
чтобы как можно более строго формализовать процедуры моделирования. 
Формализация позволит совершенствовать процедуры MBSE и~передавать их 
на исполнение компьютеру без пробелов и~искажений. Самый высокий уровень 
строгости достигается при привлечении математического аппарата, поскольку 
математика позволяет надежно доказывать или опровергать утверждения, 
ха\-рак\-те\-ри\-зу\-ющие корректность и~эффективность процедур.
   
   В настоящей работе предложен аппарат, основанный на математическом 
представлении сборочных чертежей (<<мегамоделей>> систем) 
ориенти-\linebreak рованными графами (диаграммами). Узлы такого\linebreak графа помечаются 
обозначениями моделей час\-тей, а~реб\-ра помечаются обозначениями действий\linebreak 
(activities), посредством которых части собираются в~систему. Представление 
структуры систем графами регламентируется, в~частности, стандартом 
IEC~81346~[4]. Естественным источником математических методов 
конструирования и~анализа мегамоделей служит теория категорий (см., 
например,~[5, 6]). Модели рассматриваются как объекты подходящих 
категорий, а~действия формально описываются морфизмами. Строятся 
и~исследу-\linebreak ются тео\-ре\-ти\-ко-ка\-те\-гор\-ные конструкции, опи\-сы\-ва\-ющие процедуры 
MBSE на абстрактном кон-\linebreak цептуальном уровне. Определенный опыт такого\linebreak 
исследования был накоплен в~инженерии программного обеспечения~[7] 
и~теперь может быть обобщен для системной инженерии в~целом. Например, 
сборке системы согласно некоторой мегамодели отвечает построение 
копредела диаграммы~--- универсальной конструкции~\cite{5-kov}.
   
   Статья построена следующим образом. В~разд.~2 приведен обзор 
принципов описания структуры сис\-тем согласно стандарту IEC~81346. 
Раздел~3 посвящен практическим проб\-ле\-мам мегамоделирования и~сборке 
сис\-тем. В~разд.~4 вводятся конструкции тео\-рии категорий, позволяющие 
формально решать задачи мегамоделирования. В~заключении приводятся 
выводы и~намечаются направления дальнейших исследований.

\section{Структура систем и~стандарт~IEC~81346}

   Важной проблемой MBSE, отмеченной во введении, является слабая 
совместимость языков и~инструмен\-тов моделирования от разных поставщиков. 
Основным подходом к~достижению совместимости является стандартизация~--- 
принятие обязывающих документов, устанавливающих требования и~принципы 
взаимозаменяемости инструментов. Многие стандарты определяют конкретные 
форматы машиночитаемой записи моделей, нейтральные относительно 
разработчиков инструментов MBSE. Примером служит формат описания 
твердотельных геометрических моделей STEP, стандартизованный семейством 
ISO~10303. Однако для формализации MBSE в~целом интерес представляют 
в~первую очередь стандарты более общего плана, унифицирующие принципы 
и~методы применения моделей в~жизненном цикле систем независимо от 
способа записи моделей. С~этой точки зрения внимания заслуживает 
международный стандарт IEC 81346-1:2009 <<Промышленные системы, 
установки и~обору\-до\-ва\-ние~--- принципы структурирования и~ссылочные 
обозначения~--- часть~1: основные правила>> (<<Industrial Systems, 
Installations and Equipment and Industrial Products~--- Structuring Principles and 
Reference Designations~--- Part~1: Basic Rules>>)~\cite{4-kov}. Стандарт не 
принят в~России, однако ряду его положений в~области структуры систем 
соответствует российский ГОСТ~2.053-2013 <<ЕСКД. Электронная структура 
изделия. Общие положения>>.
   
   В стандарте IEC~81346 рассматривается ряд вопросов моделирования 
структуры систем и~идентификации отдельных единиц в~составе систем. 
Системная единица названа в~стандарте объектом, причем принципиально не 
проводится различие между объектами реального мира, составляющими 
реально существующие системы, и~объектами мыслительной деятельности~--- 
моделями единиц, составляющими модели систем. Таким образом, стандарт 
выходит за рамки MBSE и~рассматривает ряд вопросов системной инженерии 
вообще. Иерар\-хи\-че\-ская структура системы (холархия~\cite{3-kov}) 
изображается деревом, узлы которого помечены обозначениями объектов. 
Важным достижением стандарта является выявление того факта, что одна и~та 
же система задается не одной, а несколькими в~общем случае различными 
иерархическими структурами, возникающими в~результате декомпозиции 
согласно различным принципам (аспектам). В~их числе:
   \begin{itemize}
\item функциональная (function-oriented) структура, отвечающая разделению 
системных единиц по выполняемым ими функциям в~составе сис\-темы;
\item продуктовая (product-oriented), или модульная, структура, отражающая 
сборочную (технологическую) конфигурацию сис\-темы;
\item структура размещения (location-oriented), в~соответствии с~которой 
единицы располагаются в~физическом пространстве.
\end{itemize}

   Ясно, что один и~тот же объект может входить в~несколько структур и~при 
этом находиться на различных уровнях. В~то же время в~некоторых аспектах 
объект может никак не проявлять себя и~вследствие этого отсутствовать 
в~соответствующих структурах. Полное идентифицирующее ссылочное 
обозначение объекта (reference designation) конструируется путем 
последовательного перечисления всех объектов, находящихся на пути от корня 
дерева рассматриваемой структуры до дан\-ного объекта включительно. 
Наименование каж\-до\-го объекта в~этом перечислении составляется из 
символьного обозначения аспекта, буквенного обозначения класса (типа), 
к~которому относится  объект, и~порядкового номера объекта среди 
экземпляров своего класса. Таким путем обеспечивается\linebreak  уникальность 
наименования любой единицы\linebreak
 в~пределах системы. Например, функциональная 
структура обозначается символом <<=>>, а~функциональный класс 
переключателей потоков ресурсов обозначается буквами QA, так что первая по 
порядку единица, выполняющая функцию переключения, называется =QA1, 
а~ее полное ссылочное обозначение может выглядеть как =WP1=WC1=QA1. 
Если объект присутствует в~нескольких структурах, то он может иметь 
несколько ссылочных обозначений, как показано на рис.~1~\cite{4-kov}.

\begin{figure*} %fig1
    \vspace*{1pt}
\begin{center}
\mbox{%
\epsfxsize=165mm
\epsfbox{kov-1.eps}
}
\end{center}
\vspace*{-9pt}
\Caption{Пример ссылочных обозначений структурных единиц системы}
\vspace*{9pt}
\end{figure*}

   С~точки зрения практики системной инженерии большой интерес 
представляет описание эволюции структурного представления системы по ходу 
жизненного цикла, приведенное в~приложении~B к~стандарту IEC~81346. 
<<Строительный материал>> для структур имеет вид (виртуального) 
справочника или каталога объектов, из которого выбираются объекты для 
включения в~структуру. 

В~начале жизненного цикла системы на основе 
исходных требований к~ней конструктор строит ее функциональную структуру. 
Затем определяется пространственное положение функциональных объектов, 
в~результате чего создается структура размещения. На следующей стадии 
формируются закупочные спецификации, образующие продуктовую структуру. 
В~ходе последующих стадий жизненного цикла эти структуры могут 
трансформироваться. На каждой стадии могут происходить замена, слияние 
и~расщепление объектов. Таким образом, объекты разных структур системы 
связаны отношением вида <<многие ко многим>>, вдоль которого 
прослеживаются (трассируются) исходные требования.
   
   В то же время стандарт не предусматривает указа\-ние способов, какими 
объекты собраны в~сис\-те\-мы. Поэтому структуру сис\-те\-мы можно рас\-смат\-ри\-вать 
как эскизный проект, в~котором отражены лишь факты вхождения системных 
единиц более низкого уровня иерархии в~единицы более высокого уровня. 


Проект такого рода поступает на вход технологу, который определяет 
конкретные операции сборки каждой единицы каждого уровня иерархии. При 
необходимости технолог вносит изменения в~конструкцию объектов (такие как 
нарезка резьбы) и~добавляет связующие интерфейсные объекты (такие как 
клей, трансформатор и~др.). В~результате для каждого составного объекта 
формируется сборочный чертеж, на котором указаны все со\-став\-ля\-ющие 
объекты и~действия по их соединению в~целях получения сис\-те\-мы. 
Технологическая проработка требуется на всех стадиях жизненного цикла, на 
которых формируется либо изменяется ка\-кая-ли\-бо из структур системы.

%\vspace*{-6pt}

\section{Мегамоделирование и~сборка~систем}

   В MBSE объекты, образующие 
структуры\linebreak
 сис\-тем, описываются формализованными ком\-пьютерными моделями 
различных видов: геометрическими фигурами и~телами, численными 
аппроксимациями дифференциальных уравнений, оснащенными графами и~
т.\,д. При этом, как свидетель\-ст\-ву\-ют стандарты типа IEC~81346, для анализа 
структуры систем и~организации сборки необходимо знать не столько 
внутреннюю структуру моделей, сколько ассортимент их возможностей 
соединяться с~другими моделями в~целях формирования моделей составных 
объектов. Иными словами, модели рассматриваются как <<черные ящики>> 
с~известным поведением по отношению к~другим моделям. Каталог объектов, 
упоминавшийся в~предыду\-щем разделе, в~условиях применения \mbox{MBSE} 
составляется из моделей и~описаний действий по их соединению.
   
   Структуры систем и~сборочные чертежи представляют собой частные 
случаи мегамоделей (mega\-mod\-el)~--- моделей, состоящих из моделей и~связей 
между ними~\cite{8-kov}. Мегамодель, в~которой связи описывают соединение 
моделей, образующих некоторую сис\-те\-му, называется конфигурацией этой 
сис\-те\-мы~\cite{5-kov}. Существуют и~другие виды мегамоделей, 
предназначенные для описания других процедур \mbox{MBSE}, таких как 
формирование модели согласно заданной метамодели  
(instantiating)~\cite{9-kov}. Но в~настоящей работе сосредоточимся на 
конфигурациях и~сборке систем.
   
   Например, в~моделировании механических сис\-тем, состоящих из твердых 
тел, моделями деталей и~сборочных единиц служат геометрические тела, 
которые могут быть представлены для компьютерной обработки различными 
способами: конструктивным, воксельным, граничным~\cite{10-kov}. Объекты, 
составляющие механические системы, т.\,е.\ представления экземпляров тел, 
получаются из моделей путем аффинных изометрий и~растяжений. Так, из 
набора цилиндров разных размеров составляется модель штанги (спортивного 
снаряда). В~функциональной структуре штанги по IEC~81346 цилиндры 
представлены разными объектами, поскольку они выполняют разные функции, 
хотя порождаются одной и~той же геометрической моделью. Соответственно, 
в~каталоге моделей содержится тело в~форме цилиндра, допускающее 
несколько разных действий по включению в~состав штанги.
   
   В качестве еще одного примера рассмотрим дис\-крет\-но-со\-бы\-тий\-ное 
имитационное моделирование, поддержка которого относится к~числу 
важнейших достижений MBSE~\cite{1-kov}. Здесь модель имеет вид 
сценария~--- фрагмента предполагаемой истории поведения моделируемой 
системы, пред\-став\-лен\-но\-го потоком дискретных событий различных видов. 
Некоторые события могут вызывать либо запрещать возникновение других 
событий. Описания действий по сборке сценариев поведения систем отражают 
вклад сценариев поведения составляющих. Так, сценарий работы цеха 
составляется из сценариев работы станков, связанных друг с~другом согласно 
маршрутным картам~\cite{11-kov}.
   
   Сформулируем задачу мегамоделирования сборки систем в~общем виде 
следующим образом. По мегамодели, представляющей конфигура\-цию 
некоторой системы, требуется сконструировать модель системы как целого 
и~рассчитать для нее моделируемые параметры, в~том числе эмерджентные~--- 
не присущие никакой из со\-став\-ля\-ющих единиц в~отдельности. Принцип 
конструирования модели системы легко усмотреть из организации 
структур-\linebreak\vspace*{-12pt}

\columnbreak

 { \begin{center}  %fig1
 \vspace*{1pt}
\mbox{%
\epsfxsize=57.246mm
\epsfbox{kov-2.eps}
}


\vspace*{12pt}


\noindent
{{\figurename~2}\ \ \small{Схема склеивания}}
\end{center}
}

\vspace*{18pt}

\addtocounter{figure}{1}

\noindent
ного представления: система должна находиться на иерархическом 
уровне, располагающемся непосредственно над уровнем со\-став\-ля\-ющих ее 
объектов. Иными словами, модель системы должна включать в~себя модели 
всех составляющих с~учетом их конфигурационных связей и~в~то же время 
включаться в~любые модели, включающие в~себя модели всех составляющих 
конфигурации.
   
   Поясним этот принцип на простом примере. Предположим, что нужно 
объединить в~систему два объекта~$P$ и~$S$ и~что технолог решил сделать это 
с~по\-мощью клея~--- третьего объекта~$G$, который может быть соединен 
и~с~$P$, и~с~$S$. Действие клея описывается конфигурацией следующего 
вида: объекты~$G$ и~$P$ порождают в~результате соединения известный 
промежуточный комплексный объект~$P_G$, содержащий их, а~объекты~$G$ 
и~$S$ порождают объект~$S_G$. Система~$R$, полученная путем 
склеивания~$P$ с~$S$ при помощи~$G$, отбирается среди объектов, 
содержащих~$P_G$ и~$S_G$, по следующему структурному критерию: 
объект~$R$ должен содержаться в~любом объекте~$T$, содержащем~$P_G$ 
и~$S_G$. Схематически этот критерий изображен на рис.~2.


   Если объект $R$, удовлетворяющий указанному структурному критерию, 
существует, то он действительно отвечает системе, которая собрана из~$S$ 
и~$P$ путем склеивания посредством~$G$ (и~не содержит ничего 
<<лишнего>>). Более того, легко видеть, что такой объект~$R$ определяется, 
по существу, однозначно в~том смысле, что любые два объекта~$R$ 
и~$R^\prime$, удовлетворяющие структурному критерию, содержатся друг 
в~друге. Если же нужного объекта~$R$ не существует, то делается вывод, что 
технолог ошибся: клей~$G$ не способен соединить объекты~$P$ и~$S$.
   
   В структурное представление, выполненное по стандарту IEC~81346 либо по 
ГОСТу 2.053-2013, входят только объекты~$P$, $S$ и~$R$ и~две композитные 
стрелки: $P\hm\to R$, проходящая через~$P_G$, и~$S\hm\to R$, проходящая 
через~$S_G$ (так что мегамодель склеивания~--- это часть схемы, ограниченная 
треугольником~$PSR$). Кроме того, стрелки на схеме склеивания, в~отличие от 
структуры, представляют не просто факты включения объектов друг в~друга, 
а~конкретные действия по их соединению. При этом соблюдается следующее 
естественное условие структурной корректности: если из одного объекта 
можно прийти в~другой разными путями по схеме, то эти пути задают одно и~то 
же композитное действие. Например, клей~$G$ включается в~состав 
системы~$R$ единственным способом, несмотря на наличие двух путей $G 
\hm\to  P_G \hm\to R$ и~$G \hm\to S_G \hm\to R$: в~действительности не имеет 
значения, через какой промежуточный объект <<прослеживается>> включение 
клея в~систему. Таким образом, мегамодель сборки содержит больше 
информации, чем иерархическая структура системы.
   
   Если модели содержат значения тех или иных параметров, а описание 
действий по их соединению позволяет выявить правила преобразования 
значений, то по мегамодели сборки можно вы\-чис\-лить значения параметров для 
системы. Известны примеры вычислений такого рода в~области разработки 
новых композиционных материалов~\cite{12-kov}. Осредненные 
(эффективные) физические характеристики композитов, такие как модуль Юнга и~коэффициент Пуассона, сложным образом зависят от характеристик 
компонентов и~способов изготовления композита из них. При помощи методов 
теории упру\-гости эти зависимости задаются в~форме линеаризованных 
матричных соотношений, которые приписываются к~стрелкам мегамоделей, 
пред\-став\-ля\-ющим включение компонентов в~композиты. Появляется 
возможность рассчитывать на компьютере свойства композитов по базе данных 
компонентов, без проведения дорогостоящих физических экспериментов.
   
   В заключение раздела отметим, что хотя прямой расчет системы по 
конфигурации имеет большое значение, в~MBSE он играет вспомогательную 
роль. Согласно стандарту IEC~81346 и~практикам системной инженерии, 
система обычно проектируется сверху вниз~--- от корня структурной иерархии 
к~составляющим~\cite{13-kov}. Это означает, что технолог в~основном решает 
не прямую, а~обратную задачу: модель системы, которую нужно собрать, 
известна, а~нужно построить (восстановить) конфигурацию, из которой такая 
система может быть получена путем сборки, с~учетом различных ограничений. 
Формальные математические постановки и~методы решения обратных задач 
мегамоделирования представляют собой крупную перспективную тему 
исследований, выходящую за рамки настоящей статьи.

\section{Теория категорий в~мегамоделировании}

   Как указывалось во введении, естественным источни\-ком математических 
методов кон\-стру\-ирова\-ния и~анализа мегамоделей служит теория категорий. 
Категорией называется коллекция абстрактных объектов, попарно связанных 
морфизмами (стрелками). Точное определение занимает буквально несколько 
строк~\cite{14-kov}: категория~$C$ состоит из совокупности 
объектов~$\mathrm{Ob}\,C$ и~совокупности морфизмов~$\mathrm{Mor}\,C$, 
на которых заданы следующие операции:
\begin{enumerate}[(1)]
\item каждому морфизму~$f$ 
сопоставляется два объекта: область $\mathrm{dom}\,f$ и~кообласть 
$\mathrm{codom}\,f$ (соотношения вида $\mathrm{dom}\,f \hm= A$ и~
$\mathrm{codom}\,f \hm= B$ наглядно записываются в~форме стрелки~$f$: 
$A\hm\to B$, а множество всех морфизмов, удовлетворяющих этим 
соотношениям, обозначается через $\mathrm{Mor}(A, B))$;
\item для 
любой пары морфизмов~$f, g$, удовлетворяющей условию 
$\mathrm{codom}\,f\hm = \mathrm{dom}\,g$, определена композиция~--- 
морфизм $g \circ f : \mathrm{dom}\,f \hm\to  \mathrm{codom}\,g$, причем она 
ассоциативна: для любой тройки морфизмов~$f, g, h$, удовлетворяющей 
условиям $\mathrm{codom}\,f \hm= \mathrm{dom}\,g$ и~$\mathrm{codom}\,g 
\hm= \mathrm{dom}\,h$, выполняется соотношение $h \circ (g \circ f) \hm= (h 
\circ g) \circ f$;
\item любой объект~$A$ обладает тождественным 
морфизмом~$1_A : A \to A$ таким, что для любого морфизма~$f : A\hm\to B$ 
выполняется соотношение $f \circ 1_A \hm= 1_B \circ  f \hm= f$.
\end{enumerate}

Классическим 
примером категории служит $\mathbf{Set}$, состоящая из всех множеств и~всех 
их отображений: закон композиции отображений задается стандартной 
подстановкой, а тождественным морфизмом произвольного множества служит 
его тождественное отображение на себя.
   
   Вместе с~категорией вводится понятие функтора~--- отображения категорий, 
сохраняющего структуру. Функтор $\mathrm{fun}\,: C \hm\to D$, действующий из 
категории~$C$ в~$D$,~--- это пара одноименных отображений $\mathrm{fun}\,: 
\mathrm{Ob}\,C \hm\to \mathrm{Ob}\,D$, $\mathrm{fun}\,: \mathrm{Mor}\,C \hm\to 
\mathrm{Mor}\,D$, удовлетворяющая следующим условиям (для произвольных 
$C$-мор\-физ\-мов~$f, g$ и~$C$-объ\-ек\-та~$A$): 
\begin{enumerate}[(1)]
\item $\mathrm{fun}\,(\mathrm{dom}\,f) 
\hm= \mathrm{dom}\,\mathrm{fun}\,(f), \mathrm{fun}\,(\mathrm{codom}\,f)\hm = 
\mathrm{codom}\,\mathrm{fun}\,(f)$;  
\item $\mathrm{fun}\,(g \circ f) = \mathrm{fun}\,(g) \circ \mathrm{fun}\,(f)$, 
если композиция $g \circ f$ определена; 
\item $\mathrm{fun}\,(1_A) \hm= 1_{\mathrm{fun}\,(A)}$.
\end{enumerate}
 Все категории и~все функторы образуют 
(формальную) категорию~$\mathbf{CAT}$. Чтобы исследовать взаимосвязь 
между функторами, вводится следующее понятие: естественным 
преобразованием~$\varepsilon$ функтора $\mathrm{fun}\, : C\hm\to D$ в~$\mathrm{fun}^\prime\, : C 
\hm\to D$ называется любое семейство $D$-мор\-физ\-мов~$\varepsilon_A : 
\mathrm{fun}\,(A) \hm\to \mathrm{fun}^\prime (A)$, $A \hm\in \mathrm{Ob}\,C$, 
такое что для любого 
\mbox{$C$-мор}\-физ\-ма $f : A\hm\to B$ выполняется соотношение $\varepsilon_B \circ 
\mathrm{fun}\,(f) \hm= \mathrm{fun}^\prime(f) \circ \varepsilon_A$:

%\begin{figure*} %рис
\vspace*{1pt}
\begin{center}
\mbox{%
\epsfxsize=54.473mm
\epsfbox{kov-3.eps}
}
\end{center}
%\vspace*{-9pt}
%\end{figure*}

   Эффективность применения теории категорий в~качестве математического 
аппарата \mbox{MBSE} обуслов\-ле\-на тем, что любой каталог моделей представляет 
собой не что иное, как категорию. Действительно, любая цепочка действий по 
соединению моделей порождает композитное действие (процесс) и, кроме того, 
любая модель допускает пустое действие над самой собою, не 
подразумевающее никаких изменений (процедура <<ничегонеделания>>). 
Например, в~твердотельном моделировании механических систем объектами 
категории\linebreak моделей выступают тела~--- подмножества в~$\mathbb{R}^3$, 
которые являются ограниченными, регулярными\linebreak
 (совпадают с~замыканием 
своей внутренности) и~полуаналитическими (допускают представление 
конечными булевыми комбинациями множеств вида $\{(x, y, z) \vert  F_i(x, y, 
z)\hm\leq 0\}$, где~$F_i : \mathbb{R}^3\hm\to \mathbb{R}$ является 
вещественной аналитической функцией для всех~$i$)~\cite{10-kov}. Чтобы 
было возможно задавать процедуры типа склеивания участков поверхности тел, в~категорию геометрических моделей добавляются ограниченные регулярные 
полуаналитические подмножества в~$\mathbb{R}^n$, $0 \hm\leq n \hm\leq 2$, 
при помощи стандартного вложения~$\mathbb{R}^n$ в~$\mathbb{R}^3$. Далее 
выполняется факторизация: отождествляются друг с~другом все множества, 
переходящие друг в~друга под действием аффинных изометрий. Морфизмы 
таких классов эквивалентности, описывающие действия по сборке составных 
механических сис\-тем, порождаются изометрическими вложениями множеств 
и~растяжениями. Получается подкатегория в~\textbf{Set}, которую будем обозначать 
через $\mathbf{MBS}$ (от Multibody Systems).
   
   Для многих известных технологий MBSE формальное описание каталогов 
поддерживаемых моделей приводит к~категориям множеств со структурой~--- 
алгебраических систем, топологических пространств, графов и~т.\,д. 
Морфизмами в~таких категориях служат отображения множеств, со\-вмес\-ти\-мые 
со структурой. На любой такой категории действует канонический функтор 
в~$\mathbf{Set}$, <<забывающий>> структуру. 

В~качестве примера приведем  
дис\-крет\-но-со\-бы\-тий\-ное моделирование, в~котором математической 
моделью сценария служит множество событий, час-\linebreak тич\-но упорядоченное  
при\-чин\-но-след\-ст\-вен\-ны\-ми зависимостями и~размеченное видами 
событий~\cite{15-kov}. Действия по сборке сложных сценариев задаются 
монотонными отображениями, сохраняющими разметку, поскольку ни 
события, ни зависимости, ни метки не могут быть <<потеряны>> при 
соединении сценариев поведения компонентов в~сценарии поведения систем. 
Получается категория~$\mathbf{Pomset}$, состоящая из всех помеченных 
частично упорядоченных множеств и~всех их монотонных отображений, 
сохраняющих разметку. Имеется функтор $\vert \mbox{--} \vert : 
\mathbf{Pomset}\hm\to \mathbf{Set} : S \mapsto \vert S\vert$, <<забывающий>> 
порядок и~разметку.
   
   Зафиксируем произвольную категорию~$C$, представляющую некоторый 
каталог моделей. Как и~для любой алгебраической системы, определена 
конструкция подкатегории в~$C$~--- это пара, состоящая из подкласса 
в~$\mathrm{Ob}\,C$ и~подкласса в~$\mathrm{Mor}\,C$, замкнутых 
относительно унаследованных из~$C$ операций. Подкатегория в~$C$ 
называется полной, если любой \mbox{$C$-мор}\-физм, область и~кообласть которого 
содержатся в~ней, сам содержится в~ней. Например, подкатегориями 
описываются различные аспекты структурного представления систем согласно 
стандарту IEC~81346. Действительно, композиция двух морфизмов, 
представляющих действия по формированию некоторого аспекта структуры, 
также должна входить в~этот аспект, поскольку стандарт предписывает строить 
цепочки для идентификации объектов в~структуре системы. Кроме того, если 
объект присутствует в~аспекте, то его тождественный морфизм формально 
должен быть включен в~этот аспект. В~то же время подкатегории, 
опи\-сы\-ва\-ющие все аспекты, не обязаны образовывать в~совокупности разбиение 
категории~$C$: как показывает рис.~1, возможны как действия, входящие 
в~несколько аспектов одновременно, так и~композитные действия с~переходом 
между структурами, не входящие ни в~один аспект. Требуется лишь, чтобы 
объединение классов объектов всех этих подкатегорий совпадало 
с~$\mathrm{Ob}\,C$, поскольку не имеет смысла вводить модели, не входящие 
ни в~одну структуру.
   
   Категории можно получать из графов: любой ориентированный мультиграф 
порождает категорию, объектами в~которой служат все узлы, а морфизмами~--- 
все пути. Областью и~кообластью морфизма являются соответственно начало 
и~конец пути, композиция морфизмов действует как конкатенация путей, 
а~тождественным морфизмом узла~$a$ является пустой путь из~$a$ в~$a$, не 
содержащий ни одного ребра. Отсюда получается фундаментальное понятие  
$C$-диа\-грам\-мы~--- это функтор вида~$\Delta : X \hm\to C$, где~$X$~--- 
категория, порожденная некоторым графом и~называемая схемой диаграммы. 
Все $C$-диа\-грам\-мы образуют категорию~$\mathbf{D}C$ (ковариантная 
категория <<сверхзапятой>>~\cite{14-kov}), в~которой морфизмом 
диаграммы~$\Delta : X \hm\to C$ в~$\Xi : Y \hm\to C$ служит любая пара 
вида $\langle\gamma, fd\rangle$, состоящая из функтора~$fd : X\hm\to Y$ 
и~естественного преобразования~$\gamma : \Delta\hm\to \Xi \circ fd$; закон 
композиции морфизмов диаграмм имеет вид:
$$
\langle \gamma, fd\rangle \circ 
\langle \varphi, gd\rangle \hm = \langle \gamma_{gd(-)} \circ \varphi, fd \circ 
gd\rangle\,.
$$ 
В~тео\-рии категорий накоплен богатый арсенал алгебраических 
методов конструирования и~анализа диаграмм.
   
   Любая мегамодель задается $C$-диа\-грам\-мой, так что категорное 
представление каталогов моделей позволяет формально решать задачи 
мегамоделирования. Морфизмы диаграмм описывают структурные 
преобразования мегамоделей, выполняемые при помощи инструментов MBSE. 
Покажем, как решаются средствами теории категорий прямые задачи 
мегамоделирования. Здесь применяется одна из основных  
тео\-ре\-ти\-ко-ка\-те\-гор\-ных конструкций~--- копредел  
диаграммы~\cite{5-kov}, который строится следующим образом. Обозначим 
через~$\mathbf{1}$ категорию,\linebreak состоящую из одного объекта~0 и~одного 
морфизма~$1_0$. Из любой категории~$X$ имеется в~точ\-ности один 
функтор~$!_X : X \hm\to \mathbf{1}$, сопоставляющий объект~0  
любому~$X$-объ\-ек\-ту (иными словами, $\mathbf{1}$ является терминальным 
$\mathbf{CAT}$-объ\-ек\-том). Имеется вложение (инъективный функтор) 
$\ulcorner \mbox{--}\urcorner : C \hookrightarrow \mathbf{D}C$, сопоставляющее 
произвольному $C$-объ\-ек\-ту $Q$~точку~--- диаграмму $\ulcorner Q\urcorner : 
\mathbf{1}\hm\to  C : 0 \mapsto Q$. Коконусом (cocone) называется 
$\mathbf{D}C$-мор\-физм, имеющий точку в~качестве кообласти. Можно 
изобразить коконус $\langle \sigma, !_X\rangle : \Delta\hm\to \ulcorner 
Q\urcorner$ над диаграммой $\Delta : X\hm\to C$ в~виде диаграммы, 
<<пририсовав>> к~$\Delta$ дополнительную вершину, помеченную 
объектом~$Q$, и~набор ребер~--- стрелок, по одной для каждого узла $I\hm\in 
\mathrm{Ob}\,X$, направленной из~$I$ в~вершину и~помеченной морфизмом 
$\sigma_I : \Delta (I) \hm\to Q$. Копределом (colimit) диаграммы~$\Delta$ 
называется коконус $\mathrm{colim}\,\Delta : \Delta\hm\to \ulcorner R\urcorner$, 
универсальный в~том смысле, что для любых \mbox{$C$-объ}\-ек\-та~$T$ 
и~коконуса~$\delta : \Delta\hm\to\ulcorner T\urcorner$ существует единственный 
$C$-мор\-физм~$w : R \hm\to T$ такой, что $\delta\hm= \ulcorner w\urcorner \circ  
\mathrm{colim}\,\Delta$. Легко видеть, что это условие универсальности 
представляет собой в~точности структурный критерий из разд.~3. Таким 
образом, конструирование копредела конфигурации~$\Delta$ описывает на 
строгом математическом языке сборку системы, которой отвечает 
вершина~$R$. В~категориях типа $\mathbf{MBS}$ и~$\mathbf{Pomset}$ 
построение копредела сводится к~факторизации раздельных объединений 
объектов, представляющих компоненты системы, по отношениям 
эквивалентности, индуцированным моделями клея и~других средств сборки.
   
   Копредел любой диаграммы, если он сущест\-вует, определяется однозначно 
   с~точностью до изомор\-физма. Более того, можно описать сборку сис\-тем из 
конфигураций в~виде функтора. Пусть $Cd$~--- некоторый класс  
$C$-диа\-грамм, имеющих копределы. Он порождает полную подкатегорию 
в~$\mathbf{D}C$, из которой в~$C$ действует функтор копредела $\mathrm{colim}$, 
сопоставляя каждой диаграмме из~$Cd$~вершину некоторого ее копредела, а 
каждому \mbox{$\mathbf{D}C$-мор}\-физ\-му~$\theta : \Delta\hm\to \Xi$, 
где~$\Delta, \Xi\hm\in Cd$~--- стрелку копредела $\mathrm{colim}\,(\theta)$ такую, что 
$\mathrm{colim}\,\Xi \circ \theta \hm= \ulcorner \mathrm{colim}\,(\theta)\urcorner \circ 
\mathrm{colim}\,\Delta$.

%\begin{figure*}
\vspace*{1pt}
\begin{center}
\mbox{%
\epsfxsize=56.127mm
\epsfbox{kov-4.eps}
}
\end{center}
%\vspace*{-9pt}
%\end{figure*}

   Например, в~категории \textbf{Set} любая диаграмма имеет 
копредел~\cite[упражнение~5.1.8]{14-kov}, поэтому имеется функтор $\mathrm{colim}\, : 
\mathbf{D}(\mathbf{Set})\hm\to \mathbf{Set}$. Примечательно, что этот функтор 
является рефлектором: он сопряжен слева с~вложением $\ulcorner \mbox{--}\urcorner : 
\mathbf{Set}\hookrightarrow \mathbf{D}(\mathbf{Set})$, причем 
единица рефлексии состоит из $\mathbf{D}(\mathbf{Set})$-мор\-физ\-мов 
$\mathrm{colim}\,\Delta : \Delta\hm\to \ulcorner\mathrm{colim}\,(\Delta)\urcorner$, 
$\Delta\hm\in \mathrm{Ob}\ \mathbf{D}(\mathbf{Set})$. Напомним, что единица 
рефлексии~--- это естественное преобразование тождественного функтора 
в~композицию рефлектора и~вложения (в~данном случае, естественное 
преобразование функтора $1_{\mathbf{D}(\mathbf{Set})}$ в~$\ulcorner \mathrm{colim}\,(  
\mbox{--})\urcorner)$, состоящее из универсальных  
стрелок~\cite[разд.~4.3]{14-kov}. И~для произвольного класса~$Cd$, 
содержащего достаточное количество одноточечных диаграмм, функтор 
$\mathrm{colim}$ сопряжен слева с~ограничением  
вложения~$\ulcorner \mbox{--}\urcorner$ на подходящую полную подкатегорию 
в~$C$. А~поскольку сопряженный функтор задается однозначно с~точностью 
до изоморфизма~\cite[разд.~4.1]{14-kov}, можно сделать вывод, что сборка 
систем в~некотором смысле <<зашифрована>> в~процедуре построения 
одноточечных диаграмм~--- моделей систем как целого без раскрытия 
струк\-туры. 

Так наглядно проявляется двойственность прямых и~обратных задач 
мегамоделирования.

\section{Заключение}

   Аппарат теории категорий обладает большим потенциалом в~области 
повышения полезной отдачи от MBSE, в~том числе путем математически 
строгого решения задач мегамоделирования. Так, базовая процедура системной 
инженерии~--- сборка\linebreak
 системы из заданной конфигурации взаимо\-свя\-занных 
компонентов~--- формально описывается тео\-ретико-ка\-те\-гор\-ной 
конструкцией копредела диа\-граммы. Более сложные конструкции отвечают\linebreak 
сложным процедурам сборки, таким как связывание (weaving) общесистемных 
функций, рассеянных по всем компонентам (crosscutting concerns), например 
мониторинговых или защитных~\cite{16-kov}. Математического представления 
требуют и~другие процедуры MBSE, в~частности коллективная модификация 
мегамоделей и~составляющих моделей, восстановление конфигурации заданной 
системы, оценка взаимозаменяемости компонентов. 

Актуальны вопросы 
внедрения аппарата теории категорий в~практику, в~том числе путем развития 
программных инструментов моделирования и~мегамоделирования. Здесь 
открывается широкий спектр направлений для дальнейших исследований.
   
{\small\frenchspacing
 {%\baselineskip=10.8pt
 \addcontentsline{toc}{section}{References}
 \begin{thebibliography}{99}
\bibitem{1-kov}
Modeling and simulation-based systems engineering handbook~/
Eds.\ D.~Gianni,  A.~D'Ambrogio, A.~Tolk.~--- London: CRC Press, 2014. 513~p.
\bibitem{2-kov}
\Au{Ковалёв С.\,П., Толок~А.\,В.} Применение модельно-ори\-ен\-ти\-ро\-ван\-но\-го подхода 
в~управ\-ле\-нии жизненным циклом технических изделий~// Информационные технологии 
в~проектировании и~производстве, 2015. №\,2. С.~3--9.
\bibitem{3-kov}
\Au{Левенчук А.\,И.} Системноинженерное мышление.~--- М.: TechInvestLab, 2015. 305~с.
\bibitem{4-kov}
IEC 81346-1:2009. Industrial Systems, Installations and Equipment and Industrial Products~--- 
Structuring Principles and Reference Designations~--- Part~1: Basic Rules.~--- Geneva: ISO, 2009. 
168~p.
\bibitem{5-kov}
\Au{Ginali S., Goguen~J.} A~categorical approach to general systems~// 
 Conference (International) on Applied General Systems 
Research Proceedings~/
Ed. G.\,J.~Klir.~--- NATO conference series.~--- New York, NY, USA: Plenum 
Press, 1978. Vol.~5. P.~257--270.
\bibitem{6-kov}
\Au{Mabrok M.\,A., Ryan M.\,J.} Category theory as a~formal mathematical foundation for  
model-based systems engineering~// Appl. Math. Inform. Sci., 2017. Vol.~11. No.\,1. P.~43--51.
\bibitem{7-kov}
\Au{Ковалёв С.\,П.} Тео\-ре\-ти\-ко-ка\-те\-гор\-ный подход к~проектированию программных 
сис\-тем~// Фундаментальная и~прикладная математика, 2014. Т.~19. Вып.~3. С.~111--170.
\bibitem{8-kov}
\Au{B$\acute{\mbox{e}}$zivin J., Jouault~F., Rosenthal~P., Valduriez~P.} Modeling in the large 
and modeling in the small~// Model Driven Architecture: European MDA Workshops on 
Foundations and Applications Proceedings~/
Eds.\ U.~A{\!\ptb{\ss}}mann, M.~Aksit,  A.~Rensink.~--- 
Lecture notes in computer science ser.~--- Springer, 2005. Vol.~3599. 
P.~33--46.
\bibitem{9-kov}
\Au{Diskin Z., Kokaly~S., Maibaum~T.} Mapping-aware mega\-mod\-eling: Design patterns and 
laws~// Software Language Engineering: 6th Conference (International) Proceedings~/
Eds.\ M.~Erwig, R.\,F.~Paige, E.~Van Wyk.~--- 
Lecture notes  in computer science ser.~--- Springer, 2013. Vol.~8225. P.~322--343.
\bibitem{10-kov}
\Au{Requicha A.\,G.} Representations for rigid solids: Theory, methods, and systems~// 
ACM  Comput. Surv., 1980. Vol.~12. Iss.~4. P.~437--464.
\bibitem{11-kov}
\Au{K$\acute{\mbox{a}}$d$\acute{\mbox{a}}$r B., Pfeiffer~A., Monostori~L.} Discrete event 
simulation for supporting production planning and scheduling decisions in digital
 factories~//  37th 
CIRP Seminar (International) on Manufacturing Systems Proceedings.~--- Budapest, 2004.  
P.~444--448.
\bibitem{12-kov}
\Au{Giesa T., Spivak D.\,I., Buehler~M.\,J.} Category theory based solution for the building block 
replacement problem in materials design~// Adv. Eng. Mater., 2012. Vol.~14. 
Iss.~9. P.~810--817.
\bibitem{13-kov}
\Au{Косяков А., Свит У., Сеймур~С., Бимер~С.} Системная инженерия. Принципы 
и~практика~/ Пер. с~англ.~--- М.: ДМК-Пресс, 2014. 636~с. (\Au{Kossiakoff~A., Sweet~W.\,N., 
Seymour~S., Biemer~S.\,M.} Systems engineering principles and practice.~--- 2nd ed.~--- New 
York, NY, USA: John Wiley, 2011. 560~p.)
\bibitem{14-kov}
\Au{Маклейн С.} Категории для работающего математика~/ Пер. с~англ.~--- М.: Физматлит, 
2004. 352~с. (\Au{Mac Lane~S.} Categories for the working mathematician.~--- New York, NY, 
USA: Springer, 1978. 317~p.)
\bibitem{15-kov}
\Au{Pratt V.\,R.} Modeling concurrency with partial orders~// Int. J.~Parallel 
Prog., 1986. Vol.~15. No.\,1. P.~33--71.
\bibitem{16-kov}
\Au{Ковалёв С.\,П.} Семантика ас\-пект\-но-ори\-ен\-ти\-ро\-ван\-но\-го моделирования 
данных и~процессов~// Информатика и~её применения, 2013. Т.~7. Вып.~3. С.~70--80.
 \end{thebibliography}

 }
 }

\end{multicols}

\vspace*{-3pt}

\hfill{\small\textit{Поступила в~редакцию 16.01.17}}

%\vspace*{8pt}

\newpage

\vspace*{-30pt}

%\hrule

%\vspace*{2pt}

%\hrule

%\vspace*{8pt}


\def\tit{METHODS OF CATEGORY THEORY IN~MODEL-BASED SYSTEMS ENGINEERING\\[-7pt]}

\def\titkol{Methods of category theory in~model-based systems engineering}

\def\aut{S.\,P.~Kovalyov\\[-12pt]}

\def\autkol{S.\,P.~Kovalyov}

\titel{\tit}{\aut}{\autkol}{\titkol}

\vspace*{-14pt}


\noindent
Institute of Control Sciences, Russian Academy of Sciences, 65~Profsoyuznaya Str., 
Moscow 117997, Russian Federation



\def\leftfootline{\small{\textbf{\thepage}
\hfill INFORMATIKA I EE PRIMENENIYA~--- INFORMATICS AND
APPLICATIONS\ \ \ 2017\ \ \ volume~11\ \ \ issue\ 3}
}%
 \def\rightfootline{\small{INFORMATIKA I EE PRIMENENIYA~---
INFORMATICS AND APPLICATIONS\ \ \ 2017\ \ \ volume~11\ \ \ issue\ 3
\hfill \textbf{\thepage}}}

\vspace*{1pt}

 

\Abste{A mathematical device based on the category theory is proposed to formally describe and 
rigorously explore procedures of employing models in engineering that constitute the contents of 
model-based systems engineering (MBSE). The essence of the device consists in mathematical 
representation of assembly drawings (megamodels of systems) as diagrams in categories whose 
objects are models, and morphisms represent actions associated with assembling system models 
from component models. The soundness of the device is justified on the basis of standards that 
govern description of the systems' structure such as IEC~81346. Category-theoretical methods for 
solving a number of practical problems of assembling systems are proposed and explored. 
Examples of solving such problems are provided in categories that represent two key application 
areas for MBSE: geometric modeling of complex shapes and discrete-event simulation of the 
behavior of industrial systems.}

\KWE{ model-based systems engineering; megamodel; category theory; colimit}

\DOI{10.14357/19922264170305} 

%\vspace*{-18pt}

%\Ack
%\noindent




\vspace*{-7pt}

  \begin{multicols}{2}

\renewcommand{\bibname}{\protect\rmfamily References}
%\renewcommand{\bibname}{\large\protect\rm References}

{\small\frenchspacing
 {%\baselineskip=10.8pt
 \addcontentsline{toc}{section}{References}
 \begin{thebibliography}{99}
\bibitem{1-kov-1}
Gianni, D., A.~D'Ambrogio, and A.~Tolk, eds. 2014. \textit{Modeling and simulation-based 
systems engineering handbook}. London: CRC Press. 513~p.
\bibitem{2-kov-1}
\Aue{Kovalyov, S.\,P., and A.\,V.~Tolok.} 2015. Primenenie model'no-orientirovannogo podkhoda 
v~upravlenii zhiznennym tsiklom tekhnicheskikh izdeliy [Applying model-based approach 
to product lifecycle management].\linebreak \textit{Informatsionnye tekhnologii v~proektirovanii 
i~proizvod\-st\-ve} [Information Technologies in Design and Industry] 2(158):3--9.
\bibitem{3-kov-1}
\Aue{Levenchuk A.\,I.} 2015. 
\textit{Sistemnoinzhenernoe myshlenie} [Systems engineering thinking]. 
Moscow: TechInvestLab. 305~p.
\bibitem{4-kov-1}
IEC 81346-1:2009. 2009. 
Industrial Systems, Installations and Equipment and Industrial 
Products~--- Structuring Principles and Reference Designations~--- 
Part~1: Basic Rules. Geneva:  ISO. 168~p.
\bibitem{5-kov-1}
\Aue{Ginali, S., and J.~Goguen.} 1978. 
A~categorical approach to general systems. \textit{Conference 
(International) on Applied General Systems Research Proceedings}. Ed.\
 G.\,J.~Klir. \mbox{NATO}  conference ser. Plenum Press. 5:257--270.
\bibitem{6-kov-1}
\Aue{Mabrok, M.\,A., and M.\,J.~Ryan}. 
2017. Category theory as a~formal mathematical foundation for 
model-based systems engineering. \textit{Appl. Math.  Inform. Sci.} 11(1):43--51.
\bibitem{7-kov-1}
\Aue{Kovalyov, S.\,P.} 2016. 
Category-theoretic approach to software systems design. \textit{J.~Math. Sci.} 
214(6):814--853.
\bibitem{8-kov-1}
\Aue{B$\acute{\mbox{e}}$zivin, J., F.~Jouault, P.~Rosenthal, and P.~Valduriez.}
 2005. Modeling in 
the large and modeling in the small. 
\textit{Model Driven Architecture: European MDA Workshops on 
Foundations and Applications Proceedings.} 
Eds.\ U.~\mbox{A{\!\ptb{\ss}}mann}, M.~Aksit, and A.~Rensink. 
Lecture notes in computer science ser. Springer. 3599:33--46.
\bibitem{9-kov-1}
\Aue{Diskin, Z., S.~Kokaly, and T.~Maibaum.} 2013. 
Mapping-aware megamodeling: Design patterns 
and laws. \textit{6th Conference (International) on Software Language Engineering 
Proceedings}. Eds.\ M.~Erwig, R.\,F.~Paige, and E.~Van Wyk. 
Lecture notes in computer science ser. Springer. 
8225:322--343.
\bibitem{10-kov-1}
\Aue{Requicha, A.\,G.} 1980. Representations for rigid solids: 
Theory, methods, and systems. \textit{ACM 
Comput. Surv.} 12(4):437--464.
\bibitem{11-kov-1}
\Aue{K$\acute{\mbox{a}}$d$\acute{\mbox{a}}$r,~B., A.~Pfeiffer, and L.~Monostori.}
2004. Discrete 
event simulation for supporting production planning and scheduling decisions in 
digital factories. \textit{37th CIRP Seminar (International) on Manufacturing 
Systems Proceedings}. Budapest.  444--448.
\bibitem{12-kov-1}
\Aue{Giesa, T., D.\,I.~Spivak, and M.\,J.~Buehler.} 2012. 
Category theory based solution for the building 
block replacement problem in materials design. 
\textit{Adv. Eng. Mater.} 14(9):810--817.
\bibitem{13-kov-1}
\Aue{Kossiakoff, A., W.\,N.~Sweet, S.~Seymour, and S.\,M.~Bie\-mer.}
2011. \textit{Systems engineering 
principles and practice}. 2nd ed. New York, NY: John Wiley. 560~p.
\bibitem{14-kov-1}
\Aue{Mac Lane, S.} 1978. \textit{Categories for the working mathematician}. 
New York, NY: Springer. 317~p.
\bibitem{15-kov-1}
\Aue{Pratt, V.\,R.} 1986. Modeling concurrency with partial orders. 
\textit{Int. J.~Parallel Prog.} 15(1):33--71.
\bibitem{16-kov-1}
\Aue{Kovalyov, S.\,P.} 2013. 
Semantika aspektno-ori\-en\-ti\-ro\-van\-no\-go modelirovaniya dannykh 
i~protsessov [Semantics of aspect-oriented modeling of data and processes]. 
\textit{Informatika i~ee  Primeneniya~--- Inform. Appl.} 7(3):70--80.
\end{thebibliography}

 }
 }

\end{multicols}

\vspace*{-9pt}

\hfill{\small\textit{Received January 16, 2017}}

\vspace*{-18pt}

\Contrl

\noindent
\textbf{Kovalyov Sergey P.} (b.\ 1972)~--- Doctor of Science in physics and 
mathematics, leading scientist, Institute of Control Problems, Russian 
Academy of Sciences, 65~Profsoyuznaya Str., Moscow 117997, Russian 
Federation Federation; \mbox{kovalyov@nm.ru} 

\label{end\stat}


\renewcommand{\bibname}{\protect\rm Литература}       %4 
\def\stat{chehovich}

\def\tit{МЕТОДЫ ОБНАРУЖЕНИЯ ПЕРЕВОДНЫХ ЗАИМСТВОВАНИЙ В~БОЛЬШИХ ТЕКСТОВЫХ 
КОЛЛЕКЦИЯХ$^*$}

\def\titkol{Методы обнаружения переводных заимствований в~больших текстовых 
коллекциях}

\def\aut{Р.\,В.~Кузнецова$^1$, О.\,Ю.~Бахтеев$^2$, Ю.\,В.~Чехович$^3$}

\def\autkol{Р.\,В.~Кузнецова, О.\,Ю.~Бахтеев, Ю.\,В.~Чехович}

\titel{\tit}{\aut}{\autkol}{\titkol}

\index{Кузнецова Р.\,В.}
\index{Бахтеев О.\,Ю.}
\index{Чехович Ю.\,В.}
\index{Kuznetsova R.\,V.}
\index{Bakhteev O.\,Yu.}
\index{Chekhovich Yu.\,V.}

{\renewcommand{\thefootnote}{\fnsymbol{footnote}} \footnotetext[1]
{Работа выполнена при поддержке РФФИ (проект 18-07-01441) 
и~Фонда содействия развитию малых форм предприятий в~на\-уч\-но-тех\-ни\-че\-ской сфере 
(проект~44116).}}

\renewcommand{\thefootnote}{\arabic{footnote}}
\footnotetext[1]{Московский физико-технический институт, 
\mbox{rita.kuznetsova@phystech.edu}}
\footnotetext[2]{Компания 
Антиплагиат; 
Московский фи\-зи\-ко-тех\-ни\-че\-ский институт, \mbox{bakhteev@ap-team.ru}}
\footnotetext[3]{Вычислительный центр им.\ А.\,А.~Дородницына Федерального 
исследовательского центра <<Информатика и~управ\-ле\-ние>> Российской академии наук, 
\mbox{chehovich@ap-team.ru}}


\vspace*{-8pt}


\Abst{Рассматривается задача обнаружения переводных заимствований. 
Для решения предлагается использовать моноязыковой подход~--- свести задачу 
обнаружения заимствований к~одному языку, используя машинный перевод. В~связи со 
спецификой рассматриваемой задачи предлагаемый алгоритм обнаружения должен быть 
устойчив к~неоднозначностям перевода. Предлагается декомпозировать задачу на 
несколько этапов.
Сначала отбираются до\-ку\-мен\-ты-кан\-ди\-да\-ты,  устойчивость к~неоднозначности перевода 
достигается за счет замены слов на метки кластеров, полученных с~по\-мощью 
дистрибутивной модели. Затем происходит сравнение найденных кандидатов 
и~рассматриваемого документа, для этого используется отображение текстовых 
фрагментов документов в~векторное пространство высокой размерности. 
Вычислительный эксперимент проводится для языковой пары 
<<рус\-ский--анг\-лий\-ский>> на двух выборках~--- синтетическом корпусе и~на статьях из 
журналов, входящих в~Российский индекс научного цитирования (РИНЦ).}


\KW{автоматическая обработка текстов; машинный перевод; 
глубокое обучение; переводные заимствования; обнаружение переводных 
заимствований; дистрибутивная семантика}

\DOI{10.14357/19922264210105}

%\vspace*{-2pt}


\vskip 10pt plus 9pt minus 6pt

\thispagestyle{headings}

\begin{multicols}{2}

\label{st\stat}

\section{Введение}

\vspace*{-2pt}

Проблема некорректных текстовых заимствований актуальна для сферы образования 
и~научных исследований~\cite{plag_cheh}. По материалам исследования~\cite{hist}, 
проведенного в~2013~г., более 1500~диссертаций по историческим наукам, 
защищенных в~России после 2000~г., содержат значительные заимствования из 
других диссертаций.

Для задачи обнаружения заимствований в~рамках одного языка высокую полноту 
поиска показывают промышленные инструменты~\cite{plag_cheh}, работа которых 
основана на представлении документов в~виде набора перекрывающих друг друга 
пословных $n$-грамм (шинглов)~\cite{shingles1}. Такой подход позволяет 
эффективно проводить поиск точных текстовых заимствований, но не позволяет 
обнаруживать заимствования с~большой долей перефразированного текста или со 
вставками  текста, переведенного с~другого языка.

Существуют несколько подходов, опи\-сы\-ва\-ющих проб\-ле\-му поиска переводных 
заимствований для некоторых пар языков~\cite{clkga,clfreshonto}, например для 
пары ис\-пан\-ский--анг\-лий\-ский. Настоящая работа посвящена обнаружению переводных 
заимствований для пары языков рус\-ский--анг\-лий\-ский. Данная пара нечасто 
встречается в~литературе и~не является родственной. Выбор пары языков рус\-ский--анг\-лий\-ский 
обуслов\-лен преобладанием англоязычных пуб\-ли\-ка\-ций в~интернете 
и~лучшим знанием этого языка по сравнению с~другими. Аналогично 
работам~\cite{framework1,framework2} в~данной статье предлагается описание 
алгоритма полного цик\-ла поиска заимствований~--- сначала ведется поиск 
до\-ку\-мен\-тов-кан\-ди\-да\-тов по внешней коллекции, затем происходит их детальное 
сравнение с~проверяемым документом. Предлагается алгоритм, основанный на 
моноязыковом анализе документов, схожем с~проведенным 
в~работах~\cite{mono,fruct}~--- про\-ве\-ря\-емый документ переводится на английский 
язык с~использованием системы машинного перевода с~дальнейшим сравнением 
текс\-то\-вых фрагментов внут\-ри документов.

В ряде работ, посвященных поиску переводных заимствований, используются 
дополнительные ресурсы, такие как тезаурусы и~онтологии.
В~работах~\cite{clkga,clfreshonto} авторы предлагают использовать базы знаний 
для извлечения информации о~близости между текстами. В работе~\cite{clfreshonto} 
предлагается алгоритм, основанный на комбинации нейронных сетей и~графов знаний. 
Основной недостаток этого подхода~--- ресурсоемкость: использование 
мультиязычных онтологий и~баз знаний требует больших вычислительных мощностей 
для построения семантических графов для каждого текстового фрагмента, а также 
сравнения полученных семантических графов.

В данной работе предлагается декомпозиция алгоритма обнаружения переводных 
заимствований для поиска по большим текстовым коллекциям.
Общая схема алгоритма включает следующие шаги.
\begin{enumerate}
\item \textit{Машинный перевод}~--- перевод проверяемого документа на английский 
язык. Для этого используется система статистического машинного 
перевода~\cite{moses}.
\item \textit{Поиск до\-ку\-мен\-тов-кан\-ди\-да\-тов}~--- для проверяемого документа 
находятся наиболее релевантные до\-ку\-мен\-ты-кан\-ди\-да\-ты, для этого используется 
модификация алгоритма шинглов.
\item \textit{Сравнение документов}~--- текст разбивается на фрагменты, строится 
отображение каждой фразы в~векторное пространство. Для каждого вектора 
проверяемого документа находятся ближайшие векторы из до\-ку\-мен\-тов-кан\-ди\-да\-тов, 
после чего проводится классификация пар данных векторов на схожие и~несхожие 
пары текстовых фрагментов.
\end{enumerate}

Так как в~предлагаемом алгоритме используется моноязыковой анализ заимствований, 
то задача близка к~задаче обнаружения перефразированного текс\-та.
Ряд подходов~\cite{Socher1,wieting,Iyyer,vbta} к~решению этой задачи используют 
векторные пред\-став\-ле\-ния фраз, полученные с~по\-мощью нейронных сетей глубокого 
обуче\-ния.
В работе~\cite{Iyyer} предлагается  нейронный мешок слов (\textit{англ}.\ Neural Bag-of-Words) 
и~глубокие усред\-ня\-ющие сети (\textit{англ}.\ Deep averaging networks).
В~данной \mbox{статье} предлагается использовать выходы нейронной сети как векторные 
пред\-став\-ле\-ния текс\-то\-вых фрагментов для дальнейшего при\-бли\-жен\-но\-го алгоритма 
поиска бли\-жай\-ше\-го соседа~\cite{ann}.

В работе исследуются свойства предлагаемого метода обнаружения переводных 
заимствований. Проводится анализ моделей глубокого обучения, используемых на 
этапе сравнения документов, а~также составной оптимизируемой функции. Проверка 
качества предложенного метода проводится как на синтетической выборке, так и~на 
статьях из журналов, входящих в~РИНЦ. 
Проводится анализ ошибок. Предложенный метод поиска заимствований сравнивается 
с~базовым алгоритмом поиска заимствований, основанным на использовании машинного 
перевода и~алгоритме шинглов.

\vspace*{-6pt}

\section{Постановка задачи}

Пусть заданы коллекции документов на английском языке 

\noindent
$$
D_e = \{d_e^j\}_{j=1}^N
$$ 
и~русском языке 
$$
D_r = \{d_r^i\}_{i=1}^M.
$$
 Документы на русском и~английском 
языке представимы в~виде конкатенации текстовых фрагментов:
$$
d_e^j = \left[s_{e_1}^j \sqcup \cdots \sqcup s_{e_h}^j\right];\enskip
d_r^i \hm= \left[s_{r_1}^i 
\sqcup \cdots \sqcup s_{r_k}^i\right].
$$

Пусть задана выборка  
$$
\mathcal{D}=\left\{(d_e^l, d_r^l), \mathrm{RL}^l\right\}_{l=1}^L,
$$
 где каждой 
паре документов $(d_e^l, d_r^l)_{d_e^l \in D_e, d_r^l \in D_r}$ сопоставлен 
список пар фрагментов 
$$
\mathrm{RL}= \left[(s_{e_1}^l, s_{r_1}^l), \ldots, (s_{e_{k(l)}}^l, 
s_{r_{k(l)}}^l)\right].
$$
 Для каждой пары $(s_{e_{\tilde{k}}}^l, s_{r_{\tilde{k}}}^l)$ 
известно, что фрагмент~$s_{r_{\tilde{k}}}^l$ является переводом фрагмента~$s_{e_{\tilde{k}}}^l$.


Модель $f$ задается как последовательное выполнение функций {filter} 
и~comparison, где
\begin{align*}
\mbox{filter:}&\ \ (d_r^i, D_e)_{d_r^i \in D_r} \rightarrow D_e^{\mathrm{retrieved}_i} \subset 
D_e,\\
\mbox{comparison:}&\ \  (d_r^i, D_e^{\mathrm{retrieved}_i})_{d_r^i \in D_r} \rightarrow \mathrm{RL}^{i}.
\end{align*}
Здесь $\mathrm{RL}^{i}$~--- список пар фрагментов. Функция {filter} отвечает за 
сужение числа документов коллекции, сравниваемых с~проверяемым документом, 
и~позволяет проводить дальнейшее более детальное сравнение {comparison} 
с~использованием ресурсоемких вычислительных алгоритмов, основанных на моделях 
глубокого обучения.


Качество модели $f$ оценивается с~помощью функций Precision и~Recall:
\begin{align*}
\mbox{Precision}& = \fr{|(\mathop{\cup}\nolimits_{l=1}^L \mathrm{RL}^l) \cap (\mathop{\cup}\nolimits_{i=1}^M 
\mathrm{RL}^{i})|}{|\mathop{\cup}\nolimits_{i=1}^M \mathrm{RL}^i|},\\
\mbox{Recall} &= \fr{|(\mathop{\cup}\nolimits_{l=1}^L \mathrm{RL}^l) \cap (\mathop{\cup}\nolimits_{i=1}^M 
\mathrm{RL}^{i})|}{|\mathop{\cup}\nolimits_{l=1}^L \mathrm{RL}^{l}|}.
\end{align*}

Требуется найти функцию $f$, мак\-си\-ми\-зи\-ру\-ющую~F1, среднее гармоническое 
показателей Precision и~Recall:
\begin{align*}
\hat{f}& = \argmax\limits_{f \in \mathcal{F}}\text{F1}(f, \mathcal{D}), \\[3pt]
 \text{F1} &= \fr{2  \mbox{Precision} \cdot \mbox{Recall} 
}{\mbox{Precision}+\mbox{Recall}},
%\label{opt_frag}
\end{align*}
где $\mathcal{F}$~--- заданное семейство моделей.



\section{Поиск документов-кандидатов}

Одним из алгоритмов поиска до\-ку\-мен\-тов-кан\-ди\-да\-тов в~задачах обнаружения дословных 
заимствований и~поиска \textit{поч\-ти-дуб\-ли\-ка\-тов} текста\linebreak служит алгоритм, 
основанный на построении инвертированного индекса, в~котором каждый документ 
коллекции представляется набором \textit{шинглов}~\cite{shingles1}, т.\,е.\ 
набором перекрывающихся $n$-грамм. Про\-ве\-ря\-емый документ также разбивается на 
шинглы, после чего проводится поиск документов по инвертированному индексу 
с~наибольшим совпадением шинглов. В~данной работе предлагается обобщение алгоритма 
шинглов, позволяющее улучшить качество поиска кандидатов в~случае обнаружения 
переводных заимствований.

Предлагается функция filter следующего вида:
\begin{multline*}
\mbox{filter}\left(d_r^i, D_e\right) ={}\\
{}=\! \!\!\!\!\!\argmax\limits_{D_e^{'} \subset D_e, |D_e^{'}|=k}\sum\limits_{d_e^j \in 
D_e^{'}}\sum\limits_{h \in \mathcal{H}(d_r^i)}\!\!\!\! \mathbf{I}\left[h \in 
\mathcal{H}(d_e^j)\right]\! \Big / \!
\left(|d_e^{j^{'}} \in D_e\!:\right.\\
\left. h \in 
\mathcal{H}(d_e^{j^{'}})|^{\alpha} + \mbox{const}\right)\,.
\end{multline*}
Здесь $\mathcal{H}$~--- множество $n$-грамм документа, упорядоченная 
последовательность $n$ меток кластеров, где процедура формирования кластеров 
описана ниже; $\alpha \hm\in \mathbb{R}$; $k$~--- оптимизируемый гиперпараметр.


Для уменьшения влияния не\-од\-но\-знач\-ности перевода на поиск до\-ку\-мен\-тов-кан\-ди\-да\-тов 
предлагается заменять слова на соответствующие им метки кластеров:
$$
\left\{x_1, \ldots, x_n\right\} \rightarrow \left\{ \mbox{class}\left(x_1\right), \ldots, 
\mbox{class}\left(x_n\right)\right\} = h\,,
$$
где $x_1, \ldots, x_n$~--- слова. Кластеры предварительно выделены из текстового 
корпуса и~содержат семантически близкие слова.
Для уменьшения неоднозначности перевода перед разбиением на $n$-граммы 
предлагается удалять из текста стоп-сло\-ва и~проводить лемматизацию. Для учета 
возможных перестановок слов, возникающих после перевода текста, слова внутри 
каждой $n$-грам\-мы сортируются в~лексикографическом порядке.

В данной работе для получения кластеров используется модель векторного 
представления слов, основанная на дистрибутивной гипотезе. Кластеризация 
проводится с~использованием косинусной функции расстояния
\begin{equation}
\label{eq:cos}
    \cos \left(\mathbf{c}_1, \mathbf{c}_2\right) = \fr{\langle\mathbf{c}_1, 
\mathbf{c}_2\rangle}{{||\mathbf{c}_1||_2||\mathbf{c}_2||_2}},
\end{equation}
где $\mathbf{c}_1$ и~$\mathbf{c}_2$~--- векторы из одного векторного пространства.


Ниже приведены примеры полученных кластеров:
\begin{itemize}
\item $[$beer, beers, brewing, ale, brew, brewery, pint, stout, guinness, 
ipa, brewed, lager, ales, brews, pints, cask$]$;
\item $[$brilliant, excellent, exceptional, finest, outstanding, super, 
terrific$]$.
\end{itemize}


\vspace*{-6pt}

\section{Сравнение документов}

Для сравнения найденных документов-кан\-ди\-да\-тов $D_e^{\mathrm{retrieved}_i}$ 
и~проверяемого документа~$(d_r^i)$\linebreak используется модель векторного пред\-став\-ле\-ния 
фразы~--- текс\-ты разбиваются на фрагменты и~сравниваются соответствующие им 
векторы. Ниже пред\-став\-ле\-ны детали алгоритма сравнения, а~также анализ 
пред\-ла\-га\-емой оптимизационной задачи.

\vspace*{-6pt}

\subsection{Модель векторного представления фразы}

Рассмотрим подробнее этап построения отображения фрагмента в~вектор.
Пусть каждому слову документа на языке коллекции поставлен в~соответствие вектор 
$\mathbf{v} \hm\in \mathbb{R}^u$ размерности~$u$. Для прос\-то\-ты будем полагать, что 
все фрагменты на языке коллекции имеют ограниченную длину $n_{\mathrm{col}}$.
Тогда моделью векторизации фрагмента будем называть отображение
$$
    \mathbf{h}: \mathbb{W} \times \mathbb{R}^{u \times n_{\mathrm{col}}} \to 
\mathbb{R}^u\,,
$$
где $\mathbb{W}$~--- пространство параметров модели. Объекты из множества 
$\mathbb{R}^{u \times n_{\mathrm{col}}}$ являются последовательной конкатенацией 
векторов векторных представлений слов для фрагментов выборки:
$$
\mathbf{x} \in [\mathbf{v}_1, \dots, \mathbf{v}_{n_{\mathrm{col}}}]^{\mathrm{T}}\,,  
\mathbf{x} \in \mathbb{R}^{u \times n_{\mathrm{col}}}.
$$
Для работы с~фрагментами длиной меньше $n_{\mathrm{col}}$ определим некоторый 
вектор, обозначающий пус\-тое слово.


Модель оптимизируется в~режиме частичного обучения с~учителем. В~качестве 
оптимизируемой функции используется составная функция ошибки, представляющая 
собой сумму ошибки реконструкции и~ошибки отступа:
\begin{equation}
\label{eq:alpha}
\alpha E_{\mathrm{rec}}\left(\mathbf{X}_{\mathrm{rec}}, \mathbf{w}\right) + (1 - 
\alpha)E_{\mathrm{me}}(\mathbf{X}_{\mathrm{me}},\mathbf{w}) \to \min_{\mathbf{w} \in 
\mathbb{W}},
\end{equation}
где $E_{\mathrm{rec}}$~--- ошибка реконструкции; $E_{\mathrm{me}}$ --- ошибка отступа; 
$\mathbf{X}_{\mathrm{rec}}$ и~$\mathbf{X}_{\mathrm{me}}$~--- обучающие выборки;  
$\mathbf{w}$~--- параметры модели; $\alpha$~--- настраиваемый гиперпараметр.
Рассмотрим подробнее каждое слагаемое функции ошибки.

Первое слагаемое функции ошибки соответствует модели автокодировщика.
Пусть задана выбор\-ка $\mathbf{X}_{\mathrm{rec}} \subset \mathbb{R}^{u \times 
n_{\mathrm{col}}}$.  Модель~$\mathbf{h}$ выступает в~качестве функции кодирования 
информации о выборке~$\mathbf{X}_{\mathrm{rec}}$. Пусть также задана вспомогательная 
функция декодирования~$\mathbf{g}$, восстанавливающая исходное векторное 
представление~$\mathbf{x}$ по выходам модели~$\mathbf{h}$:
$$
   \mathbf{r}(\mathbf{x}, \mathbf{w}) = \mathbf{g}(\cdot, \mathbf{w}) \circ 
\mathbf{h} (\mathbf{x}, \mathbf{w}) \approx  \mathbf{x},  \mathbf{x} \in 
\mathbb{R}^{u \times n_{\mathrm{col}}}.
$$
Минимизируемая ошибка реконструкции выглядит следующим образом:
\begin{equation}
\label{eq:rec}
E_{\mathrm{rec}}(\mathbf{X}_{\mathrm{rec}}, \mathbf{w}) = 
\fr{1}{|\mathbf{X}_{\mathrm{rec}}|}\sum\limits_{\mathbf{x} \in \mathbf{X}_{\mathrm{rec}}} 
\parallel\mathbf{x}  - \mathbf{r}(\mathbf{x}, \mathbf{w}) \parallel^2_2.
\end{equation}


Выбор ошибки реконструкции в~качестве оптимизируемой функции можно обосновать, 
используя результаты статьи~\cite{ae}. Будем пользоваться результатами, 
доказанными в~работе~\cite{ae}, где было показано, что автокодировщики 
с~регуляризацией специального вида позволяют оценить распределение~$p(\mathbf{X})$ 
объектов, принадлежащих генеральной совокупности.

\smallskip

\noindent
\textbf{Теорема~1}\ 
\cite{ae}. %\label{manifold}
\textit{Пусть $p$~--- дифференцируемая плотность вероятности и~$\forall\, 
\mathbf{x}_i \hm\in \mathbb{R}^{u \times n_{\mathrm{col}}}$ $p(\mathbf{x}_i)\hm\neq 0$.  
Пусть $\mathcal{L}_{\sigma^2}$~--- функция потерь вида}
\begin{multline*}
\mathcal{L}_{\sigma^2} ={}\\
{}=\!\!\!\int\limits_{\mathbb{R}^{u \times n_{\mathrm{col}}}} \!\!\!\!\!\!
p(\mathbf{x}) \left[
\parallel \mathbf{x} - \mathbf{r}(\mathbf{x}, \mathbf{w}) 
\parallel^2_2 + \sigma^2  \left\| \fr{\partial \mathbf{r}(\mathbf{x}, 
\mathbf{w})}{\partial \mathbf{x}} \right\|_{F}^2\right] d\mathbf{x}\,,\hspace*{-7.12155pt}
\end{multline*}
\textit{где $\mathbf{r}$ дважды дифференцируема}; $0 \hm\leqslant \sigma \hm\in \mathbb{R}$. 
\textit{Пусть $\hat{\mathbf{w}}$~--- оптимум функции~$\mathbf{r}$ по параметрам моделей 
кодирования и~декодирования, доставляющий минимум~$\mathcal{L}_{\sigma^2}$. 
Тогда}
$$
\hat{\mathbf{r}}_{\sigma^2}\left(\mathbf{x}, \mathbf{w}\right) = \mathbf{x} + \sigma^2 
\fr{\partial \log p(\mathbf{x})}{\partial \mathbf{x}} + o\left(\sigma^2\right), \quad 
\sigma^2 \rightarrow 0\,.
$$


Используя результаты теоремы~1, можно сделать следующее утверждение.

\smallskip

\noindent
\textbf{Теорема~2.}\ 
\textit{Плотность вероятности представима в~виде}:
$$
\fr{\hat{\mathbf{r}}_{\sigma^2}(\mathbf{x}, \mathbf{w}) - 
\mathbf{x}}{\sigma^2} \approx -\fr{\partial}{\partial 
\mathbf{x}}E(\mathbf{x}),
$$ 
\textit{где} 
$(\mathbf{x}) = ({1}/{Z})\exp(-
E(\mathbf{x}))$, $Z$~--- \textit{нормировочная константа}.

\smallskip

\noindent
Д\,о\,к\,а\,з\,а\,т\,е\,л\,ь\,с\,т\,в\,о\,.
\begin{align*}
{\mathbf{r}}_{\sigma^2}\left(\mathbf{x}, \hat{\mathbf{w}}\right) &= \mathbf{x} + \sigma^2 
\fr{\partial}{\partial \mathbf{x}}\log p(\mathbf{x}) + o\left(\sigma^2\right);
\\
\fr{{\mathbf{r}}_{\sigma^2}(\mathbf{x}, \hat{\mathbf{w}}) - 
\mathbf{x}}{\sigma^2} &= \fr{\partial}{\partial \mathbf{x}}\log p(\mathbf{x}) + o(1);
\\
\fr{{\mathbf{r}}_{\sigma^2}(\mathbf{x}, \hat{\mathbf{w}}) - 
\mathbf{x}}{\sigma^2} &\approx \fr{\partial}{\partial \mathbf{x}}\log 
p(\mathbf{x}).
\end{align*}
Представляя $\log p(\mathbf{x})$ в~форме $-E(\mathbf{x}) \hm- \log Z$, получим 
искомое выражение.

\smallskip

Таким образом, при устремлении регуляризатора~$\sigma$ к~нулю получается  
\textit{языковая модель}, т.\,е.\ распределение вероятностей на множестве~$\mathbf{X}$~--- 
множестве текстовых последовательностей.

Второе слагаемое составной функции ошибки~--- ошибка отступа~\cite{wieting}. Для 
оптимизации этой функции ошибки используется выборка $\mathbf{X}_{\mathrm{me}}\hm = 
\{ (\mathbf{x}_i, \mathbf{x}_j)\}$, состоящая из пар объектов:
$$
    \mathbf{X}_{\mathrm{me}} = \left[\mathbf{X}_{\mathrm{me}}^A; \mathbf{X}_{\mathrm{me}}^B\right]  
\subset \mathbb{R}^{u \times n_{\mathrm{col}}}  \times  \mathbb{R}^{u \times 
n_{\mathrm{col}}};
$$

\vspace*{-14pt}

\noindent
\begin{multline}
\label{eq:me}
E_{\mathrm{me}} = \fr{1}{|\mathbf{X}_{\mathrm{me}}|}\left( \sum\limits_{(\mathbf{x}_i, 
\mathbf{x}_j) \in \mathbf{X}_{\mathrm{me}}}\!\!\!\!\max \left(0, \delta - c_{-}\right) +{}\right.\\
\left.{}+ \max\left(0, \delta 
- c_{+}\right) 
\vphantom{\sum\limits_{(\mathbf{x}_i, 
\mathbf{x}_j) \in \mathbf{X}_{\mathrm{me}}}}
\right),
\end{multline}

\vspace*{-6pt}


\noindent
где 

\vspace*{-6pt}

\noindent
\begin{multline*}
c_{-} = \cos\left(\mathbf{h}\left(\mathbf{x}_i, \mathbf{w}\right), 
\mathbf{h}\left(\mathbf{x}_j, \mathbf{w}\right)\right) -{}\\
{}- \cos\left(\mathbf{h}\left(\mathbf{x}_i, 
\mathbf{w}\right), \mathbf{h}\left(\mathbf{x}_{i^{'}}, \mathbf{w}\right)\right);
\end{multline*}

\vspace*{-12pt}

\noindent
\begin{multline*}
c_{+} = \cos\left(\mathbf{h}\left(\mathbf{x}_i, \mathbf{w}\right), 
\mathbf{h}\left(\mathbf{x}_j, \mathbf{w}\right)\right) - {}\\
{}-\cos\left(\mathbf{h}\left(\mathbf{x}_j, 
\mathbf{w}\right), \mathbf{h}\left(\mathbf{x}_{j'}, \mathbf{w}\right)\right);
\end{multline*}

\vspace*{-2pt}

\noindent
$\delta$~--- отступ; $\cos$~--- функция расстояния~\eqref{eq:cos}, 

\noindent
\begin{align*}
\mathbf{x}_{i^{'}}&=\argmax\limits_{\mathbf{x}_{i^{'}} \in \mathbf{X}^B, 
\mathbf{x}_{i^{'}} \neq \mathbf{x}_{j}}\cos\left(\mathbf{x}_i, 
\mathbf{x}_{i^{'}}\right);
\\
\mathbf{x}_{j^{'}}&=\argmax\limits_{\mathbf{x}_{j^{'}} \in \mathbf{X}^A, 
\mathbf{x}_{j^{'}} \neq \mathbf{x}_{i}}\cos\left(\mathbf{x}_i, 
\mathbf{x}_{i^{'}}\right).
\end{align*}

Следующая теорема объясняет поведение данного слагаемого при проводимой 
оптимизации параметров~$\mathbf{w}$ модели~$\mathbf{h}$.

\smallskip

\noindent
\textbf{Теорема~3.}\ 
\textit{Пусть выполнены следующие условия}.
\begin{enumerate}
\item \textit{Задан гиперпараметр} $\delta \hm\in (0, 2).$
\item \textit{Мощность выборки} $|\mathbf{X}_{\mathrm{me}}|$ \textit{ограничена следующей величиной}:
\begin{multline}
    |\mathbf{X}_{\mathrm{me}}|(|\mathbf{X}_{\mathrm{me}}|-1)  \leq{}\\
    \hspace*{-3pt}{}\leq \sqrt{\pi} 
\fr{\Gamma((u-1)/2)}{\Gamma(u/2)} \left( \int\limits_{0}^{\arccos(1 - \delta)}\!\! \!\!\!\!\!\!
\sin^{u-2} x\, dx \right )^{\!-1}\!\!.\!\!
\label{eq:sphere_code}
\end{multline}
\item \textit{Подвыборки $\mathbf{X}^A_{\mathrm{me}}$ и~$\mathbf{X}^B_{\mathrm{me}}$ содержат все 
элементы в~единственном числе, ни один элемент не встречается в~обеих выборках}.
\end{enumerate}
\textit{Тогда существует непрерывное отображение~$\hat{\mathbf{h}}$ из множества 
векторных представлений слов} $\mathbb{R}^{u \times n_{\mathrm{col}}}$
\textit{в~векторное 
пространство~$\mathbb{R}^{u}$, доставляющее глобальный минимум функции} $E_{\mathrm{me}} \hm= 
0$.

\pagebreak

\noindent
Д\,о\,к\,а\,з\,а\,т\,е\,л\,ь\,с\,т\,в\,о\,.\ \
Построим отображение~$\hat{\mathbf{h}}$ явно.
Положим для каждой пары $(\mathbf{x}_1, \mathbf{x}_2)$: 
$\hat{\mathbf{h}}(\mathbf{x}_1) \hm= \hat{\mathbf{h}}(\mathbf{x}_2).$

Тогда функция $E_{\mathrm{me}}$ выглядит следующим образом с~точностью до множителя:
\begin{multline*}
\!E_{\mathrm{me}}= \!\!\!\!\!\sum\limits_{(\mathbf{x}_i, \mathbf{x}_j) \in \mathbf{X}_{\mathrm{me}}}
\!\!\!\!\!\!\!\!\max\left(0, \delta - 1 
+ \cos\left(\hat{\mathbf{h}}(\mathbf{x}_i), 
\hat{\mathbf{h}}\left(
\mathbf{x}_{i^{'}}\right)\!\right)\!\right) +{}\\
{}+ \max\left(0, \delta - 1  + 
\cos\left(\hat{\mathbf{h}}(\mathbf{x}_j), 
\hat{\mathbf{h}}\left(\mathbf{x}_{j^{'}}\right)\!\right)\!\right).
\end{multline*}

Область значений функции ограничена снизу нулем, который достигается при 
выполнении условий:
$$
   1 - \delta \geq  \cos \left(\hat{\mathbf{h}}(\mathbf{x}), 
\hat{\mathbf{h}}\left(\mathbf{x}'\right)\right)
$$
 для любой пары $\mathbf{x} \hm\in 
\mathbf{X}^A_{\mathrm{me}}$, $\mathbf{x}' \hm\in \mathbf{X}^B_{\mathrm{me}}$,  
$(\mathbf{x}, \mathbf{x}') \notin \mathbf{X}_{\mathrm{me}}$, $(\mathbf{x}', \mathbf{x}) \hm\notin 
\mathbf{X}_{\mathrm{me}}.$
Число пар, описанных выше, в~множестве~$\mathbf{X}_{\mathrm{me}}$ при выполнении 
третьего условия теоремы равно $|\mathbf{X}_{\mathrm{me}}|(|\mathbf{X}_{\mathrm{me}}|\hm-
1)$. Назначим значение отображения~$\hat{\mathbf{h}}$ для каждой такой пары так, 
чтобы $\cos(\hat{\mathbf{h}}(\mathbf{x}), \hat{\mathbf{h}}(\mathbf{x}'))  
\hm\leq 1 \hm- \delta$.

Существование такого отображения следует из задачи о нахождении сферического 
кода максимального размера для сферы в~пространстве размерности~$u$ и~углом 
$\arccos(1 \hm- \delta).$
В~работе~\cite{sphere_code} представлена нижняя оценка для размерности выборки, 
удовлетворяющей заданным условиям. Оценка соответствует правой части 
неравенства~\eqref{eq:sphere_code}. Так как выборка $\mathbf{X}_{\mathrm{me}}$  
конечна, то для построения непрерывной функции, заданной условиями, описанными 
выше, можно использовать интерполяционные полиномы, что и~требовалось доказать.

\smallskip


Заметим, что предложенное в~теореме отображение является непрерывным, поэтому 
для приближения данного отображения можно использовать нейросетевые модели. По 
теореме Цыбенко отображения из класса нейросетевых моделей будут приближать 
непрерывные модели сколь угодно хорошо~\cite{cybenko}.

Таким образом, составная оптимизируемая функция~\eqref{eq:alpha} позволяет 
получить модель, которая, с~одной стороны, обладает обобщающими свойствами, за 
которые отвечает языковая модель~\eqref{eq:rec}, с~другой стороны, эффективно 
разделяет схожие и~несхожие фразы из обучающей выборки~\eqref{eq:me}. 
Гиперпараметр~$\alpha$ отвечает за вклад каждого из оптимизируемых слагаемых в~данную функцию.

\subsection{Классификатор}

Для каждого вектора фразы~$\mathbf{h}(\mathbf{x}_{r_{a}}^i)$ из про\-ве\-ря\-емо\-го 
документа~$d_r^i$ находится~$v$~ближайших векторов по косинусной функции 
расстояния~\eqref{eq:cos} для фрагментов из до\-ку\-мен\-тов-кан\-ди\-да\-тов 
$D_e^{\mathrm{retrieved}_i}$, используя метод приближенного поиска ближайшего соседа.
Основная цель данной процедуры~---~сократить число пар фрагментов для 
классификации для снижения ресурсоемкости этапа сравнения документа.

Для векторного представления пары фрагментов $(\mathbf{h}(\mathbf{x}_{e_{b}}^j), 
\mathbf{h}(\mathbf{x}_{r_{a}}^i))$ рассматривается следующее решающее правило:
\begin{multline}
\label{eq:t1t2}
f_{\mathrm{frag}}\left(\left(\mathbf{h}\left(\mathbf{x}_{e_{b}}^j\right), \mathbf{h}(\mathbf{x}_{r_{a}}^i)\right)\right) ={}\\
{}=
\begin{cases}
1, &\mbox{ если } \cos\left(\mathbf{h}(\mathbf{x}_{e_{b}}^j), 
\mathbf{h}\left(\mathbf{x}_{r_{a}}^i\right)\right)>t_1\\
& \hspace*{5mm}\mbox{ и~} 
p\left(\mathbf{h}\left(\mathbf{x}_{e_{b}}^j\right), \mathbf{h}\left(\mathbf{x}_{r_{a}}^i\right)\right)>t_2;\\
0 &\mbox{ иначе},
\end{cases}
\end{multline}
где $p$~--- вероятность классификатора; $t_1$~--- порог косинусной  функции 
расстояния~\eqref{eq:cos}; $t_2$~--- минимальный порог вероятности 
классификатора. 

В качестве признаков используется конкате\-на\-ция разницы по модулю 
и~покомпонентное\linebreak произведение компонент вектора 
$[|\mathbf{h}(\mathbf{x}_{e_{b}}^j)\hm- 
\mathbf{h}(\mathbf{x}_{r_{a}}^i)|,\mathbf{h}(\mathbf{x}_{e_{b}}^j) \odot 
\mathbf{h}(\mathbf{x}_{r_{a}}^i)]$. В~качестве классификатора выступает модель 
случайного леса.


%\subsection{Анализ предложенного метода}

%Приведем анализ сложности предложенного метода.

%\begin{theorem}
%Пусть заданы:
%\begin{enumerate}
%\item Количество документов в~коллекции $M$.
%\item Сложность поиска одного шингла по коллекции $T_\text{search}(M)$.
%\item Сложность перевода текста длиной $n$ символов $T_\text{tr}(n)$.
%\item Среднее число документов, находящихся в~коллекции по одному шинглу 
%$\alpha$.
%\item Сложность векторизации текста $T_\text{vectorize}(n)$, являющаяся 
%монотонно-возрастающей по $n$ функцией.
%\item Сложность поиска $v$ ближайших соседей по коллекции из $m$ векторов 
%$T_\text{KNN}(m, v)$, являющаяся монотонно возрастаюшей по $m$ функцией.
%\item Максимальный размер текста в~коллекции $n_\text{col}^\text{max}$ в~словах.
%\end{enumerate}


%Тогда средняя сложность проверки документа длиной $n_\text{susp}$ слов 
%оценивается как:
%\begin{equation}
%\label{eq:complexity}
%    T_\text{tr}(n_\text{susp}) + O(n_\text{susp})T_\text{search}(M) + O(\alpha 
%n_\text{susp} \log (\alpha n_\text{susp})) +
%\end{equation}
%\[
%+ T_\text{vectorize}(n_\text{susp}) + 
%KT_\text{vectorize}(n_\text{col}^\text{max}) + 
%n_\text{susp}T_\text{KNN}(O(Kn_\text{susp}), v) + O(vn_\text{susp}).
%\]
%\end{theorem}
%\begin{proof}

%Сложность перевода документа равняется $T_\text{tr}(n_\text{susp})$ 
%в~%соответствии с~заданными обозначениями.
%Поиск документов кандидатов производится по шинглам. Количество шинглов в~
%документе зависит линейно от количества слов, поэтому оценивается как 
%$O(n_\text{susp})$.
%Поиск документов по каждому шинглу осуществляется за $T_\text{search}(M)$, 
%поэтому суммарный поиск всех документов занимает 
%$O(n_\text{susp})T_\text{search}(M)$.
%Количество унникальных документов, найденных по всем шинглам документа можно 
%оценить сверху как $O(n_\text{susp})\alpha$.

%Для дальнейшей обработки документа требуется построить список из $K$ документов-
%кандидатов, сложность построения такого списка равна $O(\alpha n_\text{susp} 
%\log (\alpha n_\text{susp}))$.

%Сложность векторизации проверяемого текста и~$K$ текстов документов-кандидатов 
%можно оценить сверху как $T_\text{vectorize}(n_\text{susp}) + 
%KT_\text{vectorize}(n_\text{col}^\text{max})$ в~силу монотонности функции 
%$T_\text{vectorize}(n)$.
%Полагая, что число предложений в~тексте линейно зависит от длины текста, оценим 
%сложность поиска ближайших соседей как $n_\text{susp} 
%T_\text{KNN}(O(Kn_\text{col}^\text{max}), v)$.

%На итоговом этапе проверки документа для каждой из найденных пар векторов-
%соседей производится классификации. Всего таких пар $O(vn_\text{susp})$,
%Таким образом, средняя сложность проверки документа соответствует 
%формуле~\eqref{eq:complexity}.
%\end{proof}

%Как видно из приведенной теоремы, сложность итогового алгоритма зависит от 
%сложности операций перевода, поиска шинглов, векторизации и~поиска ближайших 
%соседей.
%В качестве следствия теоремы приведем оценку сложности для простого случая.
%\begin{theorem}
%Пусть в~качестве системы машинного перевода используется статистичесекий 
%машинный перевод. Пусть также поиск по шинглам производится с~использованием 
%бинарного дерева поиска~\cite{cormen}, векторизация имеет линейную сложность, а 
%для поиска ближайших соседей используется алгоритм $k-d$-tree~\cite{kd}.
%Тогда сложность:
%\begin{equation}
%\label{eq:complexity}
%    O(n_\text{susp} \log M) + O(\alpha n_\text{susp} \log (\alpha 
%n_\text{susp})) + K O(n_\text{col}^\text{max}) + O(n_\text{susp}) 
%O(Kn_\text{col}^\text{max}) \log (Kn_\text{col}^\text{max})) + 
%O(vn_\text{susp})ю
%\end{equation}
%\end{theorem}
%\begin{proof}
%Сложность статистического перевода линейна по количесту слов, поэтому 
%$T_\text{tr}(n_\text{susp})  = O(n_\text{susp})$.
%Полагая коллекцию фиксированной, сложность поиска в~бинарном дереве можно 
%оценить как $O(\log M)$, тогда сложность поиска шинглов по коллекции равна 
%$O(n_\text{susp}\log M)$.
%Сложность векторизации проверяемого документа и~документов-кандидатов составляет 
%$O(n_\text{susp}) + K O(n_\text{col}^\text{max})$.
%Сложность построения индекса и~поиска составляет в~среднем $O(n_\text{susp}) 
%O(Kn_\text{col}^\text{max}) \log (Kn_\text{col}^\text{max}))$.
%Таким образом, средняя сложность проверки документа соответствует 
%формуле~\eqref{eq:complexity2}.
%\end{proof}

%Как видно из следствия, сложность итогового алгоритма субквадратична по 
%количеству слов в~проверяемом документе и~сублинейная по размеру коллекции 
%текстов. TODO. Заметим, что алгоритм $kd$-tree был приведен в~следствии в~
%качестве примера: для векторных пространств высокой размерности его 
%использование нецелесообразно, вместо него применяются методы со схожими 
%сложностными показателями.

\vspace*{-6pt}

\section{Вычислительный эксперимент}

Для анализа качества предложенного алгоритма был проведен ряд вычислительных 
экспериментов как на синтетической выборке~\cite{dataset}, так и~на реальных 
коллекциях документов.
В~данном разделе приводятся детали порождения синтетических выборок 
и~эксперименты, проведенные на них.

\vspace*{-6pt}

\subsection{Синтетическая коллекция переводных заимствований}

Для порождения переводных заимствований были использованы документы из 
английской и~русской версии сайта Wikipedia.
В качестве коллекции документов~$D_e$ были использованы 100~тыс.\ статей из 
английской версии Wikipedia.
В~качестве коллекции проверяемых документов~$D_r$ использовалась случайная 
подвыборка документов из русской версии Wikipedia. Для порождения заимствований 
для каждого документа $d_r^i \hm\in D_r$ применялся следующий алгоритм.
\begin{enumerate}
\item Выбрать документы-кан\-ди\-да\-ты~$\{d_e^j\}$ из коллекции~$D_e$. Для уменьшения 
разброса лексики\linebreak в~до\-ку\-мен\-тах-кан\-ди\-да\-тах и~проверяемом до\-ку\-менте выбор 
до\-ку\-мен\-тов-кан\-ди\-да\-тов проводился  из подвыборки~500~наиболее релевантных 
документов для проверяемого документа $d_r^i$. Для определения релевантности 
использовалась tf\;$\cdot$\;idf-ме\-ра. Чис\-ло до\-ку\-мен\-тов-кан\-ди\-да\-тов 
выбиралось случайно от~1 до~10.
\item Выбрать предложения из до\-ку\-мен\-тов-кан\-ди\-да\-тов~$\{d_e^j\}$ случайным образом и~перевести их на русский язык.
\item Заменить случайные предложения из проверяемого документа~$d_r^i$ на 
переведенные предложения из до\-ку\-мен\-тов-кан\-ди\-да\-тов. Доля замененных 
предложений из про\-ве\-ря\-емо\-го документа $d_r^i$ выбиралась случайно от~20\% до~80\%.
\end{enumerate}

\vspace*{-6pt}

\subsection{Оптимизация параметров рассматриваемых моделей}

В качестве модели векторного представления слов использовалась библиотека 
\texttt{fastText}~\cite{ft}, оптимизация параметров которой проводилась на 
английской версии Wikipedia. Размерность векторного пространства для векторного 
представления слов и~фрагментов была установлена как~100. Для оптимизации модели 
векторного представления текстовых фрагментов использовался алгоритм AdaDelta 
с~параметрами $\varepsilon\hm=10^{-6}$, $\mu\hm=0{,}95$ и~L2-ре\-гу\-ля\-ри\-за\-ция 
$\lambda_2\hm=10^{-6}$. Для итоговой  функции потерь~\eqref{eq:alpha} были установлены следующие 
значения гиперпараметров: $\delta\hm=0{,}3$; $\alpha\hm=0{,}1$. Пороги 
классификатора~\eqref{eq:t1t2} были подобраны на основе процедуры 
кросс-ва\-ли\-да\-ции: $t_1\hm=0{,}6$; $t_2\hm=0{,}5$. Для построения кластеров была использована 
агломеративная кластеризация на векторах слов. В~качестве меры близости слов 
рассматривалась косинусная  функция расстояния~\eqref{eq:cos} между 
соответствующими векторными представлениями. Итоговая модель содержала 30~тыс.\ 
кластеров для 777 тыс. слов. В качестве моделей кодирования~$\mathbf{h}$ 
и~декодирования~${\mathbf{g}}$ использовалась рекуррентная модель GRU
(gated recurrent unit)~\cite{gru}.
В~качестве системы машинного перевода использовался Moses~\cite{moses}, модель 
которого была была обучена на 18,5~млн параллельных предложений из корпусов 
Opus~\cite{opus}.
В~качестве выборки для минимизации ошибки реконструкции $E_{\mathrm{rec}}$~\eqref{eq:rec} 
использовались 10~млн предложений из английской версии Wikipedia.
Второе слагаемое функции потерь~\eqref{eq:me} использует информацию о похожих 
предложениях $\mathbf{X}_{\mathrm{me}}\hm = \{(\mathbf{x}_i, \mathbf{x}_j)\}$. 
В~качестве выборки таких предложений использовались пары параллельных предложений 
из корпуса OpenSubtitles~\cite{opus}. 

\vspace*{-6pt}

\subsection{Детали вычислительного эксперимента}

Было проведено три эксперимента на синтетических данных.
\begin{enumerate}
\item Поиск кандидатов. В~данном эксперименте анализировалось  качество 
полученной модели кластеров слов. В~качестве базового эксперимента для сравнения 
рассматривался алгоритм, основанный на шинглах без приведения слов к~меткам 
кластеров.
\item Сравнение фрагментов текста. В~данном эксперименте рассматривался случай, 
когда отбор кандидатов был проведен полностью корректно: Recall$@10\hm=1{,}0.$ 
В~качестве базового алгоритма также выступал алгоритм, основанный на шинглах: 
проверяемый документ~$d_r^i$ переводился на английский язык. После этого 
полученный текст проходил лемматизацию и~разбивался на множество перекрывающихся 
4-грамм.
Для учета возможных перестановок слов при переводе слова внутри каждой 4-грам\-мы сортировались. 
Результатом сравнения двух документов выступало множество 
совпавших отсортированных 4-грамм.

\item Эксперимент, оценивающий качество всего алгоритма (поиск кандидатов 
и~сравнение фрагментов текста). Данный эксперимент позволял оценить качество 
представленного алгоритма в~целом.
\end{enumerate}

Результаты эксперимента по поиску кандидатов представлены в~табл.~1.
Представленный алгоритм, основанный на построении кластеров, дает лучшее 
качество, чем базовый алгоритм, основанный на шинглах.



Результаты экспериментов по сравнению фрагментов текста представлены 
в~табл.~2. Пред\-став\-лен\-ный алгоритм показывает точ\-ность, 
сравнимую с~точностью базового алгоритма, и~полноту, значительно превосходящую 
полноту базового алгоритма. Точ\-ность базового алгоритма объясняется тем, что 
данный алгоритм учитывает схожесть только поч\-ти-дуб\-ли\-ка\-тов текста.




В третьем эксперименте, учитывавшем качество представленного алгоритма в~целом, 
были получены следующие показатели: $\mbox{Precision}\hm=0{,}83$; 
$\mbox{Recall}\hm=0{,}79$; $\mbox{F1}\hm=0{,}80$.

%\begin{table*}\small %tabl1
\begin{center}
\noindent
\parbox{138pt}{{{\tablename~1}\ \ \small{
Результаты эксперимента по поиску кандидатов
}}}

\vspace*{6pt}

%\label{table:stage2}
%\vspace*{2ex}

{\small \begin{tabular}{|l|c|}
\hline 
\multicolumn{1}{|c|}{\bf Алгоритм} & \bf  Recall@$10$  \\ 
\hline
Базовый & 0,93\\
%\hline
Представленный&  0,95 \\
\hline
\end{tabular}
}
\end{center}
%\end{table*}
%\begin{table*}\small %tabl2
\begin{center}
\noindent
\parbox{202pt}{{{\tablename~2}\ \ \small{
Результаты экспериментов по поиску схожих фрагментов текста
}}
}


\vspace*{6pt}
%\label{table:stage34}
%\vspace*{2ex}

{\small 
\tabcolsep=7pt
\begin{tabular}{|l|c|c|c|}
\hline 
\multicolumn{1}{|c|}{\bf Алгоритм} &  \bf Precision & \bf Recall  &  $\mathbf{F1}$ \\\hline
Базовый &  0,99 & 0,15 & 0,26 \\
%\hline
Представленный &  0,93 & 0,80 & 0,85 \\
\hline
\end{tabular}
}
\end{center}
%\end{table*}

\section{Результаты экспериментов на~реальной коллекции научных документов}

Для апробации представленного алгоритма был проведен эксперимент по поиску 
переводных заимствований на коллекции документов из электронной биб\-лио\-те\-ки 
{\sf eLibrary.ru}. Данная биб\-лио\-те\-ка содержит научные документы, входящие в~РИНЦ. 
Данный ресурс также содержит дополнительные\linebreak 
метаданные для каждого документа: заголовок,\linebreak авторов документа, язык документа 
и~принадлежность к~тематике, соответствующей Государственному рубрикатору 
на\-уч\-но-тех\-ни\-че\-ской информации (ГРНТИ).
Для апробации алгоритма в~качестве проверяемых документов~$D_r$ были 
подготовлены 2,5~млн документов на русском языке.

В качестве коллекции документов~$D_e$ использовались документы из английской 
версии Wikipedia, документы на английском языке из {\sf eLibrary.ru} и~\mbox{статьи} ресурса 
arXiv.org. Суммарное число полученных документов составило 7,6~млн.

В силу большого числа проверяемых документов~$D_r$ для дальнейшего анализа 
рассматривались документы, содержащие значительное число найден\-ных 
заимствований.
Была получена 21~тыс.\ документов со значительным числом заимствований. Из них 
были проанализированы 7,6~тыс.\ документов, выбранных случайно. Основной \mbox{целью} 
эксперимента было обнаружение переводных заимствований, когда заимствование 
произошло из англоязычного документа в~русскоязычный документ. В~то же время при 
анализе полученных результатов был выявлен ряд других срабатываний 
представленного алгоритма, которые были в~дальнейшем разделены на несколько 
типов:
\begin{itemize}
\item переводные заимствования~--- документ содержит заимствования, 
переведенные с~английского языка, выданные за оригинальный текст;
\item другие заимствования~--- заимствования из русскоязычных ресурсов или 
заимствования, направление которых нельзя определить по датам документов;
\item двуязычные статьи~--- работы одного и~того же автора на двух языках;
\item самоцитирование~--- цитирование автором его англоязычной работы;
\item цитирование законов~--- использование формулировок нормативных актов;
\item ошибочные срабатывания~--- лож\-но-по\-ло\-жи\-тель\-ные срабатывания 
представленного алгоритма;
\end{itemize}

%\setcounter{table}{2}
%\begin{table*}\small %tabl3
\begin{center}
\noindent
\parbox{202pt}{{{\tablename~3}\ \ \small{
езультаты экспериментов для коллекции документов {\sf eLibrary.ru}
}}
}


\vspace*{6pt}

{\small 
\tabcolsep=10pt
\begin{tabular}{|l|c|}
  \hline
 \multicolumn{1}{|c|}{\bf Тип} & \bf Количество  \\
  \hline
Переводные заимствования & 921 \\ 
%\hline
Другие заимствования & 2548\hphantom{9} \\ 
%\hline
Двуязычные статьи & 788 \\ 
%\hline
Самоцитирование & 669 \\ 
%\hline
Цитирование законов & 1567\hphantom{9} \\ 
%\hline
Ошибочные срабатывания & 507 \\ 
%\hline
Другое & 698 \\ 
\hline 
%\hline
Всего & 7689\hphantom{9} \\ 
\hline
\end{tabular}
}
\end{center}
%\end{table*}

%\vspace*{6pt}


\begin{itemize}
\item другое --- срабатывания, которые сложно отнести к~ка\-кой-ли\-бо категории из-за 
нехватки метаданных или плохого качества текстов.
\end{itemize}

Результаты экспериментов представлены в~табл.~3. Заметим, что были 
проанализированы только~36\% всех срабатываний алгоритма, поэтому можно 
предварительно оценить число документов с~переводными заимствованиями по всей 
коллекции в~2,5~тыс., что составляет~0,1\% всех документов. 
Заметим, что результаты были получены в~автоматическом режиме и~требуют 
дальнейшей экспертной верификации.




Распределение доли заимствований в~проанализированных документах 
представлено на рис.~1. Средняя доля заимствований со\-став\-ля\-ет~20\%.

Для анализа научных тематик, в~которых переводные заимствования происходят 
наиболее час\-то, были проанализированы документы, отнесенные к~типу 
\textit{переводные заимствования}.
Около~70\% проанализированных документов были классифицированы по~10~научным 
рубрикам. Наибольшая часть
 документов оказалась распределена между руб\-ри\-ка\-ми 
<<Экономика. Народное хозяйство. Экономические науки>> и~<<Право. Юридические 
науки>>. За-\linebreak\vspace*{-12pt}

{ \begin{center}  %fig1
 \vspace*{9pt}
    \mbox{%
\epsfxsize=79mm
\epsfbox{che-2.eps}
}

\end{center}

\noindent
{{\figurename~1}\ \ \small{
Гистограмма распределения доли заимствования в~тексте
}}}

\pagebreak

\end{multicols}

\setcounter{figure}{1}
\begin{figure*} 
 \vspace*{1pt}
\begin{center}  %fig2
   \mbox{%
\epsfxsize=142.211mm
\epsfbox{che-1.eps}
}

\end{center}
\vspace*{-9pt}
\Caption{Распределение заимствований по рубрикам ГРНТИ для типов 
\textit{переводные заимствования}~(\textit{а}) и~\textit{двуязычные статьи}~(\textit{б})
}
\vspace*{-3pt}

\end{figure*}

\begin{multicols}{2}

\noindent
метим, что распределение по рубрикам заимствований, отнесенных к~типу 
\textit{двуязычные статьи},
 значительно отличается от данного распределения.
Диаграммы десяти наиболее представительных руб\-рик для данных типов срабатываний 
показаны на рис.~2.




\paragraph*{Анализ ложно-от\-ри\-ца\-тель\-ных срабатываний.}
Для анализа лож\-но-от\-ри\-ца\-тель\-ных срабатываний представленного алгоритма была 
проанализирована полнота нахождения двуязычных документов. Оценка полноты была 
проведена с~помощью метаданных, полученных из {\sf eLibrary.ru}. Анализ срабатываний 
алгоритма показал, что только~85$\%$ документов были найдены алгоритмом 
корректно.  Заметим, что представленная оценка полноты является грубой, так как 
учитывает только полные переводы текстов.





Основная причина лож\-но-от\-ри\-ца\-тель\-ных срабатываний~--- низкое качество машинного 
перевода. Другой проб\-ле\-мой, значительно повлиявшей на качество нахождения 
двуязычных статей, является используемый алгоритм поиска кандидатов, позволяющий 
находить только близкие по структуре заимствования. Кроме того, значительная 
часть проанализированных документов имела некорректную кодировку, что также 
повлияло на полноту поиска документов.

\vspace*{-12pt}

\paragraph*{Анализ лож\-но-по\-ло\-жи\-тель\-ных срабатываний.}
Для анализа лож\-но-по\-ло\-жи\-тель\-ных срабатываний были проанализированы вручную 90 
документов, отнесенных к~типу \textit{ошибочные срабатывания}.
\mbox{Основная} проблема лож\-но-по\-ло\-жи\-тель\-ных срабатываний со\-сто\-яла в~некорректном 
векторном представлении предложений, содержащих именованные сущ\-ности, не 
встре\-ча\-емые в~обуча\-ющей \mbox{выборке}, а~также содержащих слова, незнакомые модели 
машинного перевода. Также было замечено, что алгоритм сравнения документов час\-то 
находил общие фразы вида <<Работа посвящена сле\-ду\-ющей проб\-ле\-ме$\ldots$>> и~т.\,п. 
Несмотря на корректность данных срабатываний, общие фразы представленного вида 
встречаются в~большом числе документов и~потому не долж\-ны рассматриваться как 
переводные заимствования.
Общий процент документов с~ложно-положительными срабатываниями составил 7\%.

\section{Заключение}

В работе предложен алгоритм обнаружения переводных заимствований. Предложена 
декомпозиция алгоритма обнаружения переводных заимствований, позволяющая 
проводить эффективный поиск заимствований на больших текстовых коллекциях. 
Проведен анализ предложенного метода обнаружения заимствований, а также 
составной функции ошибки, используемой для оптимизации модели глубокого 
обучения.  Для анализа качества представленного алгоритма были проведены 
эксперименты на синтетических данных для пары языков рус\-ский--анг\-лий\-ский. 
Качество алгоритма было также продемонстрировано на коллекции русскоязычных 
документов, входящих в~РИНЦ.
В~дальнейшем планируется развитие предложенного алгоритма: использование модели 
векторного представления предложений для задачи поиска кандидатов и~улучшение 
качества отображения, ставящего в~соответствие фразе вектор.\\

\bigskip

Авторы выражают свою благодарность Г.\,О.~Еременко, ООО <<Научная электронная 
библиотека>>, за предоставленные материалы.

{\small\frenchspacing
{%\baselineskip=10.8pt
%\addcontentsline{toc}{section}{References}
\begin{thebibliography}{99}

\bibitem{plag_cheh}
\Au{Никитов А.\,В., Орчаков~О.\,А., Чехович~Ю.\,В.} Плагиат в~работах 
студентов и~аспирантов: проблема и~методы противодействия~// Университетское 
управление: практика и~анализ, 2012. Т.~5. С.~61--68.

\bibitem{hist} %2
\Au{Khritankov A., Botov~P., Surovenko~N., Tsarkov~S., Viuchnov~D., 
Chekhovich~Y.} Discovering text reuse in large collections of documents: A~study 
of theses in history sciences~//  Artificial Intelligence and Natural Language 
\& Information Extraction, Social Media and Web Search FRUCT Conference.~--- 
IEEE, 2015. P.~26--32.


\bibitem{shingles1}
\Au{Зеленков И.\,В., Сегалович~И.\,В.} Сравнительный анализ методов 
определения нечетких дубликатов для Web-до\-ку\-мен\-тов~// Электронные библиотеки: 
перспективные методы и~технологии, электронные коллекции: Тр. 9-й 
Всеросс. научн. конф. RCDL.~--- Пе\-ре\-славль-За\-лес\-ский: 
Университет г.~Переславля, 2007. С.~166--174.



\bibitem{clkga} %4
\Au{Franco-Salvador~M., Gupta~P., Rosso~P.} Cross-language plagiarism 
detection using a multilingual semantic network~//  European Conference on 
Information Retrieval~/
Eds. P.~Serdyukov, P.~Braslavski, S.\,O.~Kuznetsov, \textit{et al}.~---
Lecture notes in computer science ser.~--- Berlin--Heidelberg: Springer,  2013. Vol.~7814. P.~710--713.

\bibitem{clfreshonto}
\Au{Franco-Salvador M., Gupta~P., Rosso~P.,  Banchs~R.} Cross-language 
plagiarism detection over continuous-space-and knowledge graph-based 
representations of language~// Knowl.-Based Syst., 2016. Vol.~111. P.~87--99.


\bibitem{framework1} %6
\Au{Grman J.,  Ravas~R.} Improved implementation for finding text 
similarities in large collections of data~// Notebook papers of CLEF 2011 Labs and Workshops~/ 
Eds. V.~Petras, P.~Forner, P.\,D.~Clough.~--- Amsterdam, The Netherlands, 
2011. Vol.~1177. 6~p. {\sf http://ceur-ws.org/Vol-1177/CLEF2011wn-PAN-GrmanEt2011.pdf}.

\bibitem{framework2} %7
\Au{Grozea C.,  Popescu~M.} The encoplot similarity measure for automatic 
detection of plagiarism~// Notebook papers of CLEF 2011 Labs and Workshops~/ Eds. V.~Petras, P.~Forner, P.\,D.~Clough.~--- Amsterdam, The Netherlands, 
2011. Vol.~1177. {\sf http://ceur-ws.org/Vol-1177/CLEF2011wn-PAN-GrozeaEt2011.pdf}. 

\bibitem{mono} %8
\Au{Muhr M., Kern~R., Zechner~M., Granitzer~M.} External and intrinsic 
plagiarism detection using a cross-lingual retrieval and segmentation system~// 
Notebook papers of CLEF 2010 Labs and Workshops~/
Eds. M.~Braschler, D.~Harman, E.~Pianta.~---
Padua, Italy, 2010. Vol.~1176.
{\sf http://ceur-ws.org/Vol-1176/CLEF2010wn-PAN-MuhrEt2010.pdf}.

\bibitem{fruct}
\Au{Bakhteev O., Kuznetsova~R., Romanov~A., Khritankov~A.} A~monolingual 
approach to detection of text reuse in Russian--English collection~// Artificial 
Intelligence and Natural Language \& Information Extraction, Social Media and 
Web Search FRUCT Conference.~--- IEEE, 2015. P.~3--10.

\bibitem{moses} %10
\Au{Koehn P., Hoang Hien, Birch~A., \textit{et al.}} Moses: Open source toolkit for statistical machine 
translation~//  45th Annual Meeting of the Association for 
Computational Linguistics Companion Volume Proceedings of the Demo and Poster 
Sessions Proceedings.~--- ACL, 2007. P.~177--180.


\bibitem{Socher1}
\Au{Tai K., Socher~R., Manning~C.}  Improved semantic representations from 
tree-structured long short-term memory networks~// 53rd 
Annual Meeting of the Association for Computational Linguistics and the 7th 
 Joint Conference (International) on Natural Language Processing Proceedings.~--- ACL, 2015. 
Vol.~1. P.~1556--1566.


\bibitem{wieting}
\Au{Wieting J.,  Bansal~M., Gimpel~K.,  Livescu~K.} Towards universal 
paraphrastic sentence embeddings~// arXiv.org, 2015. arXiv:1511.08198 [cs.CL].



\bibitem{Iyyer}
\Au{Iyyer M., Manjunatha~V.,  Boyd-Graber~J. Daume~H.} Deep unordered 
composition rivals syntactic methods for text classification~// 
53rd Annual Meeting of the Association for Computational Linguistics and the 
7th  Joint Conference (International) on Natural Language Processing Proceedings.~--- ACL, 
2015. Vol.~1. P.~1681--1691.

\bibitem{vbta} %14
\Au{Kuznetsova~R., Bakhteev~O., Ogaltsov~A.} Variational learning across 
domains with triplet information~// 3rd Workshop on Bayesian Deep Learning.~--- Montreal, Canada. 
{\sf http://bayesiandeeplearning.org/2018/papers/65.pdf}.


\bibitem{ann}
\Au{Wang J.,  Shen~H.,  Song~J., Ji~J.} Hashing for similarity search: 
A~survey~// arXiv.org, 2014. 29~p. \mbox{arXiv}:1408.2927 [cs.DS].


\bibitem{ae}
\Au{Alain G., Bengio~Y.} What regularized auto-encoders learn from the data-generating 
distribution~// J.~Mach. Learn. Res., 2014. 
Vol.~15. No.\,1. P.~3563--3593.


\bibitem{sphere_code}
\Au{Jenssen M., Joos~F., Perkins~W.} On kissing numbers and spherical codes 
in high dimensions~// Adv.  Math., 2018. Vol.~335. P.~307--321.


\bibitem{cybenko}
\Au{Cybenko G.} Approximation by superpositions of a~sigmoidal function~// 
Math. Control Signal., 1989. Vol.~2. No.\,4. P.~303--314.

\bibitem{dataset}
Синтетическая выборка для задачи обнаружения переводных заимствований. 
{\sf https://tiny.cc/cl\_ru\_en}.


\bibitem{ft}
\Au{Bojanowski P., Grave~E., Joulin~A., Mikolov~T.} Enriching word vectors 
with subword information~// Transactions Association for Computational 
Linguistics, 2017.  Vol.~5. P.~135--146.


\bibitem{gru}
\Au{Chung J.,  Gulcehre~C.,  Cho~K.,  Bengio~Y.} Empirical evaluation of 
gated recurrent neural networks on sequence modeling~// arXiv.org, 2014. 9~p.
arXiv:1412.3555 [cs.NE].

\bibitem{opus} %22
\Au{Tiedemann~J.} News from OPUS~--- a~collection of multilingual parallel 
corpora with tools and interfaces~// Advances in natural language 
processing.~--- Amsterdam/Philadelphia: John Benjamins, 2009. Vol.~5. P.~237--248.
\end{thebibliography}

}
}

\end{multicols}

\vspace*{-3pt}

\hfill{\small\textit{Поступила в~редакцию 19.03.2020}}

\vspace*{8pt}

%\pagebreak

%\newpage

%\vspace*{-28pt}

\hrule

\vspace*{2pt}

\hrule

%\vspace*{-2pt}

\def\tit{METHODS OF CROSS-LINGUAL TEXT REUSE DETECTION IN~LARGE TEXTUAL COLLECTIONS}

\def\titkol{Methods of cross-lingual text reuse detection in~large textual collections}

\def\aut{R.\,V.~Kuznetsova$^1$, O.\,Yu.~Bakhteev$^{1,2}$, and~Yu.\,V.~Chekhovich$^3$}

\def\autkol{R.\,V.~Kuznetsova, O.\,Yu.~Bakhteev, and~Yu.\,V.~Chekhovich}

\titel{\tit}{\aut}{\autkol}{\titkol}

\vspace*{-11pt}


\noindent
$^1$Moscow Institute of Physics and Technology, 9~Institutskiy Per., 
Dolgoprudny, Moscow Region 141700, Russian\linebreak
$\hphantom{^1}$Federation

\noindent
$^2$Antiplagiat Co., 42-1 Bolshoy Blvd., Moscow 121205, Russian Federation

\noindent
$^3$A.\,A.~Dorodnicyn Computing Center, Federal Research Center ``Computer Science and Control''
 of the Russian\linebreak
 $\hphantom{^1}$Academy of Sciences, 40~Vavilov Str., Moscow 119333, Russian Federation
 
 
\def\leftfootline{\small{\textbf{\thepage}
\hfill INFORMATIKA I EE PRIMENENIYA~--- INFORMATICS AND
APPLICATIONS\ \ \ 2021\ \ \ volume~15\ \ \ issue\ 1}
}%
\def\rightfootline{\small{INFORMATIKA I EE PRIMENENIYA~---
INFORMATICS AND APPLICATIONS\ \ \ 2021\ \ \ volume~15\ \ \ issue\ 1
\hfill \textbf{\thepage}}}

\vspace*{3pt}




\Abste{The paper investigates the cross-lingual text reuse detection problem. 
The paper proposes a~monolingual approach to this problem: to translate 
the suspicious document into the language of the collection for the further monolingual analysis. 
One of the major requirements for the proposed method is robustness to the machine translation ambiguity. 
The further document analysis is divided into two steps. 
At the first step, the authors retrieve documents-candidates which are likely to be the source 
of the text reuse. For the robustness, the authors propose to retrieve the documents 
using word clusters that are constructed using distributional semantics. 
At the second\linebreak\vspace*{-12pt}}

\Abstend{step, the authors compare the suspicious document with candidates 
using sentence embeddings that are obtained by deep learning neural networks.
 The experiment was conducted for the ``English--Russian'' language pair both on the synthetic
  data and on the articles included in the Russian Science Citation Index.}
  
  \KWE{natural language processing; machine translation; deep learning; cross-lingual text 
  reuse detection; distributional semantics}




\DOI{10.14357/19922264210105}

%\vspace*{-15pt}

\Ack
\noindent
This research was supported by RFBR (project 18-07-01441) and Foundation for Assitance to Small Innovative
Enterprises in Science and Technology (project 44116).

\vspace*{12pt}

  \begin{multicols}{2}

\renewcommand{\bibname}{\protect\rmfamily References}
%\renewcommand{\bibname}{\large\protect\rm References}

{\small\frenchspacing
 {%\baselineskip=10.8pt
 \addcontentsline{toc}{section}{References}
 \begin{thebibliography}{99}
\bibitem{1-ce}
\Aue{Nikitov, A.\,V., O.\,A.~Orchakov, and Y.\,V.~Chekhovich.}
 2012. Plagiat v~rabotakh studentov i~aspirantov: problema i~metody protivodeystviya 
 [Plagiarism in papers of students and graduate students: The problem and methods of counteraction].
 \textit{Universitetskoe upravlenie: praktika i~analiz}
  [University Management: Practice and Analysis] 5:61--68.
\bibitem{2-ce}
\Aue{Khritankov, A.\,S., P.\,V.~Botov, N.\,S.~Surovenko, S.\,V.~Tsarkov, D.\,V.~Viuchnov, and 
Y.\,V.~Chekhovich.}
 2015. Discovering text reuse in large collections of documents: 
 A~study of theses in history sciences. 
 \textit{Artificial Intelligence and Natural Language and Information Extraction, 
 Social Media and Web Search FRUCT Conference Proceedings}. IEEE. 26--32.
\bibitem{3-ce}
\Aue{Zelenkov, I.\,V., and I.\,V.~Segalovich.} 
2007. Sravnitel'nyy analiz metodov opredeleniya nechetkikh dublikatov dlya 
Web-dokumentov [Comparative analysis of methods for determining fuzzy duplicates for Web-documents]. 
\textit{Tr. 9-y Vseross. nauchn. konf. ``Elektronnye biblioteki: perspektivnye metody i~tekhnologii, 
elektronnye kollektsii''} [9th All-Russian Scientific Conference ``Digital libraries: 
Advanced Methods and Technologies, Electronic Collections'' Proceedings]. Pereslavl-Zalessky:
Pereslavl-Zalessky University. 166--174.
\bibitem{4-ce}
\Aue{Franco-Salvador, M., P.~Gupta, and P.~Rosso.}
 2013. Cross-language plagiarism detection using a~multilingual semantic network. 
 \textit{European Conference on Information Retrieval}. 
 Eds. P.~Serdyukov, P.~Braslavski, S.\,O.~Kuznetsov, \textit{et al}. Lecture notes in computer science ser. 
 Berlin--Heidelberg: Springer. 7814:710--713.
\bibitem{5-ce}
\Aue{Franco-Salvador, M., P.~Gupta., P.~Rosso, and R.\,E.~Banchs.}
 2016. Cross-language plagiarism detection over continuous-space-and knowledge graph-based 
 representations of language. \textit{Knowl.-Based Syst.} 111:87--99.
\bibitem{6-ce}
\Aue{Grman, J., and R.~Ravas.}
 2011. Improved implementation for finding text similarities in large collections of data.
 \textit{Notebook papers of CLEF 2011 Labs and Workshops}.  Eds. V.~Petras, P.~Forner, and P.\,D.~Clough. 1177. 6~p.
 {\sf http://ceur-ws.org/Vol-1177/CLEF2011wn-PAN-GrmanEt2011.pdf} (accessed January~18, 2021).
\bibitem{7-ce}
\Aue{Grozea, C., and M.~Popescu.} 2011. 
The encoplot similarity measure for automatic detection of plagiarism. 
\textit{Notebook papers of CLEF 2011 Labs and Workshops}.  Eds. V.~Petras, P.~Forner, and P.\,D.~Clough.
Amsterdam, The Netherlands. 1177. Available at: 
{\sf http://ceur-ws.org/\linebreak  Vol-1177/CLEF2011wn-PAN-GrozeaEt2011.pdf} (accessed January~18, 2021).
\bibitem{8-ce}
\Aue{Muhr, M., R.~Kern, M.~Zechner, and M.~Granitzer.}
 2010. External and intrinsic plagiarism detection using 
 a~cross-lingual retrieval and segmentation system. 
 \textit{Notebook paper of CLEF 2010 Labs and Workshops}.
Eds. M.~Braschler, D.~Harman, and  E.~Pianta. Padua, Italy. 1176. 
 Available at: {\sf http://ceur-ws.org/Vol-1176/CLEF2010wn-PAN-MuhrEt2010.pdf} (accessed January~18, 2021).
\bibitem{9-ce}
\Aue{Bakhteev, O., R.~Kuznetsova, A.~Romanov, and A.~Khritankov.}
 2015. A~monolingual approach to detection of text reuse in Russian--English collection. 
 \textit{Artificial Intelligence and Natural Language and Information Extraction, 
 Social Media and Web Search FRUCT Conference Proceedings}. IEEE. 3--10.
\bibitem{10-ce}
\Aue{Koehn, P.,  Hien Hoang, A.~Birch, \textit{et al.}} 2007. Moses: 
Open source toolkit for statistical machine translation. 
\textit{45th Annual Meeting of the Association for Computational Linguistics 
Companion Volume Proceedings of the Demo and Poster Sessions Proceedings}. ACL. 177--180.
\bibitem{11-ce}
\Aue{Tai, K.\,S., R.~Socher, and C.\,D.~Manning.}
 2015. Improved semantic representations from tree-structured long short-term memory networks. 
 \textit{53rd Annual Meeting of the Association for Computational Linguistics and the 
 7th  Joint Conference (International) on Natural Language Processing Proceedings}. ACL. 1:1556--1566.
\bibitem{12-ce}
\Aue{Wieting, J., M.~Bansal, K.~Gimpel, and K.~Livescu.} 2015.
 Towards universal paraphrastic sentence embeddings. 19~p. 
 Available at: {\sf https://arxiv.org/abs/1511.08198} (accessed January~18, 2021).
\bibitem{13-ce}
\Aue{Iyyer, M., V.~Manjunatha, J.~Boyd-Graber, and H.~Daum$\acute{\mbox{e}}$.} 
2015. Deep unordered composition rivals syntactic methods for text classification. 
\textit{53rd Annual Meeting of the Association for Computational Linguistics and the
 7th  Joint Conference (International) on Natural Language Processing Proceedings}. ACL. 1:1681--1691.
\bibitem{14-ce}
\Aue{Kuznetsova, R., O.~Bakhteev, and A.~Ogaltsov.}
 2018. Variational learning across domains with triplet information. 
 \textit{3rd Workshop on Bayesian Deep Learning Proceedings}. 
 Available at: {\sf http://bayesiandeeplearning.org/2018/\linebreak papers/65.pdf} (accessed January~18, 2021).
\bibitem{15-ce}
\Aue{Wang, J., H.\,T.~Shen, J.~Song, and J.~Ji.}
 2014. Hashing for similarity search: A~survey. 29~p. Available at: 
 {\sf https:// arxiv.org/abs/1408.2927} (accessed January~18, 2021).
\bibitem{16-ce}
\Aue{Alain, G., and Y.~Bengio.} 
2014. What regularized auto-encoders learn from the data-generating distribution. 
\textit{J.~Mach. Learn. Res.} 15(1):3563--3593.
\bibitem{17-ce}
\Aue{Jenssen, M., F.~Joos, and W.~Perkins.} 2018. 
On kissing numbers and spherical codes in high dimensions. \textit{Adv. Math.} 335:307--321.
\bibitem{18-ce}
\Aue{Cybenko, G.}
 1989. Approximation by superpositions of a~sigmoidal function. 
 \textit{Math. Control Signal.} 2(4):303--314.
\bibitem{19-ce}
Sinteticheskaya vyborka dlya zadachi obnaruzheniya perevodnykh zaimstvovaniy
 [Synthetic dataset for the cross-lingual text reuse detection problem]. 
 Available at: {\sf https://tiny.cc/cl\_ru\_en} (accessed January~18, 2021).
\bibitem{20-ce}
\Aue{Bojanowski, P., E.~Grave, A.~Joulin, and T.~Mikolov.}
 2017. Enriching word vectors with subword information. 
 \textit{Transactions Association for Computational Linguistics} 5:135--146.
\bibitem{21-ce}
\Aue{Chung, J., C.~Gulcehre, K.~Cho, and Y.~Bengio.}
 2014. Empirical evaluation of gated recurrent neural networks on sequence modeling. 9~p. 
 Available at: {\sf https://\linebreak arxiv.org/abs/1412.3555} (accessed January~18, 2021).
\bibitem{22-ce}
\Aue{Tiedemann, J.}
 2009. News from OPUS~-- a~collection of multilingual parallel corpora with tools and interfaces. 
 \textit{Advances in natural language 
processing}. Amsterdam/Philadelphia: John Benjamins. 5:237--248.
 \end{thebibliography}

 }
 }

\end{multicols}

\vspace*{-3pt}

  \hfill{\small\textit{Received March~19, 2020}}


%\pagebreak

%\vspace*{-8pt}     

\Contr

\noindent
\textbf{Kuznetsova Rita V.} (b.\ 1990)~--- 
PhD student, Moscow Institute of Physics and Technology, 
9~Institutskiy Per., Dolgoprudny, Moscow Region 141701, Russian Federation; 
\mbox{rita.kuznetsova@phystech.edu}

\vspace*{3pt}

\noindent
\textbf{Bakhteev Oleg Yu.} (b.\ 1993)~--- assistant professor, Moscow Institute of Physics and Technology, 
9~Institutskiy Per., Dolgoprudny, Moscow Region 141701, Russian Federation; Head of Research Department, 
Antiplagiat Co., 42-1~Bolshoy Blvd., Moscow 121205, Russian Federation;
\mbox{bakhteev@ap-team.ru}

\vspace*{3pt}

\noindent
\textbf{Chekhovich Yury V.} (b.\ 1976)~--- 
Candidate of Science (PhD) in physics and mathematics, 
Head of Department, A.\,A.~Dorodnicyn Computing Center, 
Federal Research Center ``Computer Science and Control'' of the Russian Academy of Sciences, 
40~Vavilov Str., Moscow 119333, Russian Federation; \mbox{chehovich@ap-team.ru}

\label{end\stat}

\renewcommand{\bibname}{\protect\rm Литература}    %5
\def\stat{strijov}

\def\tit{ВОССТАНОВЛЕНИЕ МАТРИЦЫ СУПЕРПОЗИЦИИ В~ЗАДАЧЕ~СИМВОЛЬНОЙ РЕГРЕССИИ$^*$}

\def\titkol{Восстановление матрицы суперпозиции в~задаче символьной регрессии}

\def\aut{Р.\,Г.~Нейчев$^1$, И.\,А.~Шибаев$^2$, В.\,В.~Стрижов$^3$}

\def\autkol{Р.\,Г.~Нейчев, И.\,А.~Шибаев, В.\,В.~Стрижов}

\titel{\tit}{\aut}{\autkol}{\titkol}

\index{Нейчев Р.\,Г.}
\index{Шибаев И.\,А.}
\index{Стрижов В.\,В.}
\index{Neychev R.\,G.}
\index{Shibaev I.\,A.}
\index{Strijov V.\,V.}


{\renewcommand{\thefootnote}{\fnsymbol{footnote}} \footnotetext[1]
{Работа выполнена при поддержке РФФИ (проекты 20-37-90050 и~20-07-00990).}}


\renewcommand{\thefootnote}{\arabic{footnote}}
\footnotetext[1]{Московский физико-технический институт, 
\mbox{neychevr@gmail.com}}
\footnotetext[2]{Московский физико-технический институт, 
\mbox{shibaev.kesha@gmail.com}}
\footnotetext[3]{Федеральный исследовательский центр <<Информатика 
и~управ\-ле\-ние>> Российской академии наук, \mbox{strijov@phystech.edu}}

\vspace*{-12pt}
 



\Abst{Исследуется проблема порождения структуры регрессионной модели. 
Модель представляет собой суперпозицию базовых функций. Структура модели 
описывается взвешенным цвет\-ным графом. Каждая вершина графа соответствует 
некоторой базовой функции. Ребро задает суперпозицию двух функций. Вес ребра 
равен вероятности суперпозиции. Для создания оптимальной модели необходимо 
восстановить ее структуру по матрице смежности графа. Пред\-ла\-га\-емый алгоритм 
восстанавливает минимальное остовное дерево из взвешенного цветного графа. 
Пред\-став\-ле\-но новое решение, основанное на алгоритме дерева Штейнера. 
Алгоритм сравнивается с~альтернативами.}


\KW{символьная регрессия; линейное программирование; 
суперпозиция; минимальное остовное дерево; мат\-ри\-ца смеж\-ности}

\DOI{10.14357/19922264230105} 
  
\vspace*{-8pt}


\vskip 10pt plus 9pt minus 6pt

\thispagestyle{headings}

\begin{multicols}{2}

\label{st\stat}

\section{Введение}

Символьная регрессия~--- это метод по\-стро\-ения нелинейной модели, 
аппроксимирующей выборку. Структура модели определяется суперпозицией базовых 
функций. Набор базовых функций фиксируется для конкретной прикладной задачи. 
Структуры альтернативных моделей генерируются алгоритмом оптимизации для выбора 
оптимальной модели. В данной статье предлагается определять структуру модели 
с~по\-мощью вероятностного графа. Остовное дерево в~графе определяет некоторую 
суперпозицию. Для выбора оптимальной модели необходимо реконструировать 
минимальное остовное дерево по графу.

Методы генетического программирования~\cite{koza1992genetic} находят оптимальное 
подмножество в~наборе суперпозиций базовых функций, но имеют высокую 
вычислительную сложность. В~\cite{searson2010gptips} описаны методы, понижающие 
сложность. Они используют дополнительные ограничения на суперпозиции, например 
используют линейные комбинации базовых функций. Символьная регрессия, 
описанная~в~\cite{stanley2002evolving}, используется для оптимизации структуры 
суперпозиции. Методы решения задачи символьной регрессии основаны на матричном 
представлении структуры модели~\cite{bochkarev2017generation}. Однако эти методы 
не содержат ограничений на чис\-ло аргументов базовых функций и~на структуру 
графа, обеспечивающую допустимую суперпозицию. В~данной работе решается задача 
построения модели с~помощью символьной регрессии.

Требуется восстановить допустимую суперпозицию из предсказанной мат\-ри\-цы 
смежности с~вероятностями ребер. Решается задача вос\-ста\-нов\-ле\-ния~$k$-минимального 
остовного дерева $k$-MST (\textit{англ.}\ Minimum-cost Spanning Tree). Эта задача NP-слож\-ная, 
поэтому применимы только при\-бли\-жен\-ные решения~\cite{ravi1996spanning}. 
Алгоритм~$k$-MST эквивалентен проб\-ле\-ме дерева Штейнера PCST (\textit{англ.}\ 
Prize-Collecting Steiner Tree) из-за его эквивалентности ослабленной формулировке 
постановки задачи линейного программирования~\cite{chudak2004approximate}. 
В~работах~\cite{ravi1996spanning,awerbuch1998new,arora20062+} пред\-став\-ле\-ны 
приближенные решения задачи \mbox{$k$-MST}.



Предлагаемое решение основано на упрощенной версии задачи~$k$-MST, которая 
трансформируется в~задачу PCST с~постоянными призами, одинаковыми для всех 
вершин. Быст\-рый алгоритм PSCT описан в~\cite{hegde2014fast}. Альтернативное 
решение основано на алгоритме~$(2-\varepsilon)$-аппроксимации для задачи PSCT. 
Она сравнивается с~другими алгоритмами, включая алгоритмы обхода дерева в~глубину, обхода дерева в~ширину, алгоритмы Прима.

\begin{table*}[b]\small  %tabl1
\vspace*{-12pt}
\begin{center}
        \parbox{262pt}{\Caption{Вероятности суперпозиций в~матрице смежности порождают 
ориентированный граф}

}
    \label{restored_adjacency_matrix}
\vspace*{2ex}

        \begin{tabular}{|c|c|ccccccc|}
            \hline
            Арность&Функция&$\ast$&$+$&$\ln$&$\sin$&$\times$&$\exp$&$x$\\
            \hline
            $1$&$\ast$ &0,2&{\bf 0,7}&0,5&0,4&0,5&0,3&0,2\\
            $3$&$+$    &0,3&0,2&{\bf 1,0}&{\bf 0,8}&0,6&0,3&{\bf 0,7}\\
            $1$&$\ln$  &0,3&0,2&0,0&0,0&0,1&0,5&{\bf 0,5}\\
            $1$&$\sin$ &0,1&0,4&0,0&0,5&{\bf 0,9}&0,2&0,5\\
            $2$&$\times$&0,3&0,0&0,3&0,5&0,0&{\bf 0,8}&{\bf 0,6}\\
            $1$&$\exp$ &0,3&0,3&0,4&0,1&0,5&0,4&{\bf 0,4}\\
            \hline
        \end{tabular}
\end{center}
\end{table*}

\vspace*{-12pt}


\section{Задача выбора регрессионной модели}

\vspace*{-3pt}

Требуется выбрать регрессионную модель~$\varphi$ из набора альтернативных 
моделей. Модель описывает выборку~$D=\{(x_i,y_i)\}$ и~минимизирует ошибку

\noindent
\begin{equation}
\hat{\varphi}(D)=\mathop{\argmin}\limits_\varphi\sum\limits_{i=1}^m\left(\varphi(x_i)-
y_i\right)^2.
\label{task_1}
\end{equation}
Модель представляет собой суперпозицию базовых функций из некоторого заданного 
набора. На рис.~1\linebreak\vspace*{-12pt}

{ \begin{center}  %fig1
 \vspace*{-3pt}
    \mbox{%
\epsfxsize=37.447mm
\epsfbox{str-1.eps}
}

\end{center}

\vspace*{-2pt}

\noindent
{{\figurename~1}\ \ \small{Структура регрессионной модели представляет собой ориентированный 
граф
}}}

\vspace*{6pt}

\addtocounter{figure}{1}


\noindent
 показан ее пример. Структура модели~$\varphi$, 
суперпозиция, соответствует графу~$G=(V,E)$, где базовые функции находятся 
в~вершинах~$V$. {Корневая} вершина обозначается через~$\ast$. Модель:

\vspace*{1pt}

\noindent
$$
\varphi(D) =  \ln(x) + x + \sin\left( x\times \exp(x)\right).
$$

\vspace*{-4pt}

\noindent
 Еe структура в~виде матрицы 
смежности графа пред\-став\-ле\-на~в табл.~\ref{restored_adjacency_matrix}.
Базовые функции перечислены в~первой строке. Элементами матрицы являются 
вероятности ребер~$E$ дерева. Жир\-ным шриф\-том выделены ребра восстановленного 
дерева~$M$, образующие суперпозицию~$\varphi$. Для восстановления структуры 
модели~$\varphi$ как суперпозиции, заданной деревом~$M$, необходимы только 
графовое пред\-став\-ле\-ние~$G$~и~базовые функции.



Поставим задачу восстановления структуры модели. Задано множество 
выборок~$\{D_k\}$. Каждой выборке~$D_k$ соответствует своя модель. Эта модель 
имеет структуру~$M_k$. Таким образом, имеется набор пар $\{(D_k, M_k)\}$, 
выборка и~структура.
Обозначим через~$P$ отображение, которое предсказывает вероятности узлов 
в~графе~$G$ по выборке~$D$. Для выбора модели~$\varphi(D)$ необходимо восстановить 
структуру модели~$M$ по графу~$G$. Обозначим алгоритм восстановления дерева 
через~$R$. Регрессионная модель~$\hat{\varphi}(D)$, которая решает 
задачу~(\ref{task_1}), определяется формулой
$
\hat{M}=R\left(P(D)\right).
$
Поскольку дерево~$M$ играет центральную роль в~этой работе, критерий качества 
алгоритма восстановления дерева имеет вид:


\vspace*{-3pt}

\noindent
$$
\min_{M_k \in G} \fr{1}{K}\sum\limits_{k=1}^K \left[ \hat{M_k} = M_k\right].
$$

\vspace*{-4pt}

\noindent
Восстановленное дерево должно быть эквивалентно заданному дереву, следовательно, 
выбранная модель регрессии при\-бли\-жа\-ет выборку.

\vspace*{-10pt}

\section{Задача восстановления дерева суперпозиции}

\vspace*{-3pt}

Требуется восстановить дерево~$M_k$, задающее  суперпозицию и~решающее 
задачу~(\ref{task_1}). Задан ориентированный взвешенный граф~$G\hm=(V,E)$ 
с~раскрашенными вершинами~$v_i$ и~корневой вершиной~$r$. Каждая вершина~$v_i \hm\in 
V$ имеет свой цвет~$t(v_i)\hm=t_i$. Каждое реб\-ро~$e_i\in E$ имеет свой 
вес~\mbox{$w(e_i)\hm=c_i\hm\in[0,1]$}.

Требуется восстановить ориентированное дерево минимального веса с~корнем~$r$. 
Оно должно покрывать не менее~$k$ вершин в~заданном графе~$G$. Чис\-ло ребер, 
выходящих из вершины~$v_i$ дерева, должно быть меньше или равно~$t_i$. 
Корень~$r$ имеет одно ребро,~$t_r=1$.

Сформулируем это условие в~виде задачи линейного программирования 
с~целочисленными ограничениями:

\vspace*{-5pt}

\noindent
\begin{multline}
\underset{\substack{{x_e, z_S} \\ e\in E,\\ S\subseteq V\backslash 
\{r\}}}{\mbox{minimize}}  \displaystyle \sum\limits_{e\in E}c_ex_e \\[-3pt]
\mbox{s.t.}\  \displaystyle  \sum\limits_{\substack{{e\in\delta(S):}\\ e=(\ast,v_i),\\ v_i\in\delta(S)}} \!\!\!\! x_e + 
\sum\limits_{T:T\supseteq S}  \!\!\!\! z_T\geqslant 1,\enskip  S\subseteq 
V\backslash \{r\};\\[-3pt]
 \displaystyle \sum\limits_{e\in E:~e=(\ast,v)} \! x_e\leqslant 1,\enskip v\in V;\\[-3pt]
 \displaystyle \sum\limits_{e\in E:~e=(v,\ast)}x_e\leqslant t_i,\enskip  v\in V;\\[-3pt]
 \displaystyle \sum\limits_{S\subseteq V\backslash \{r\}}|S|z_S \leqslant n-k,\enskip  x_e\in\{0,1\},\enskip 
 z_S\in\{0,1\},\\[-3pt]
  e\in E,\enskip   S\subseteq V\backslash \{r\},
\label{ilp_our}
\end{multline}
где
$$
x_e =\begin{cases}
 1, &\mbox{если\ ребро}\ e\ \mbox{входит\ в~финальную}\\
 &\mbox{суперпозицию};\\
 0 & \mbox{в~противном\ случае};
 \end{cases}
 $$
  $z_S\hm = 1$ для всех вершин, исключенных из финальной 
суперпозиции. Обозначим через~$e\hm=(\ast, v)$ ориентированное ребро с~листом~$v$. 
Обозначим через $e\hm=(v, \ast)$ ориентированное ребро с~вершиной~$v$.

Первое ограничение~(\ref{ilp_our})  определяет структуру графа решения в~виде 
дерева с~корнем~$r$. Второе ограничение определяет ориентацию дерева: каждая 
вершина имеет не более одного входящего ребра. Третье ограничение определяет 
арность используемых базовых функций, поэтому число ребер, имеющих определенную 
вершину в~качестве источника, фиксировано. Четвертое ограничение говорит, что 
итоговое дерево имеет не менее~$k$ вершин. Если все веса неотрицательны, то 
четвертое ограничение на минимальное число вершин принимает более строгий вид: 
число вершин должно быть равно~$k$. Однако более слабое ограничение позволяет 
найти возможные связи с~другими оптимизационными задачами. Ограничения 
в~(\ref{ilp_our}) преследуют ту же цель.

\vspace*{-9pt}

\section{Алгоритмы восстановления дерева $k$-MST и~PCST}

\vspace*{-3pt}

\noindent
\textbf{Определение~1} (\textbf{$\bm{k}$-минимальное остовное дерево,\linebreak $\bm{k}$-MST}).
Задан взвешенный граф~$G\hm=(V,E)$ с~корнем~$r$ и~весами ребер~$w(e_i)\hm=c_i\hm\geqslant 
0$, $e_i\hm\in E$. Требуется построить ориентированное дерево минимального веса 
с~корнем~$r$, покрывающее не менее~$k$ вершин в~$G$.

\smallskip

Если та же задача ставится для ориентированных графов, то конечное дерево 
с~корнем~$r$ должно быть ориентированным. Задача линейного программирования для 
направленного~$k$-MST исключает \mbox{третье} условие в~(\ref{ilp_our}).
В~таком виде задача~$k$-MST отличается от исходной задачи восстановления 
дерева суперпозиций~(\ref{ilp_our}) отсутствием третьего ограничения на арность 
базовых функций. Это эквивалентно ограничению на число ребер, выходящих из 
вершины.

\smallskip

\noindent
\textbf{Определение~2} (\textbf{призовое дерево Штейнера, $\text{PCST}$}).\linebreak
Задан взвешенный граф $G\hm=(V,E)$ с~корнем~$r$ и~весами ребер~$w(e_i)\hm=c_i\hm\geqslant  0$, $e_i\hm\in E$, где каждой вершине~$v_i \hm\in V$ присвоен 
{приз} $\pi(v_i)\hm=\pi_i\geqslant 0$. Требуется построить дерево~$T$ с~корнем~$r$, 
которое \mbox{минимизирует} функционал
$\sum\nolimits_{e\in E}c_ex_e \hm+ \sum\nolimits_{S\subseteq V\backslash\{r\}} 
\pi(S)z_S,$
где~$x_e\in\{0, 1\}$, $x_e\hm=1$, если~$e\hm\in E$ входит в~тройку~$T$; $z_S\hm\in\{0, 1\}$, 
$z_S\hm=1$ для всех вершин, исключенных из дерева~$T$; $S \hm= V\backslash V(T)$; $\pi(S)\hm= \sum\nolimits_{v\in S}\pi(v)$.

\smallskip

В случае ориентированных графов эта задача обобщается до~асимметричной задачи 
A-PCST. Задача линейного программирования для~A-PCST принимает вид:

\vspace*{-4pt}

\noindent
\begin{multline}
\underset{\substack{x_e,z_S \\ e\in E,\\ S\subseteq V\backslash \{r\}}}{\mbox{minimize}} 
\displaystyle \sum\limits_{e\in E} c_e x_e + \sum\limits_{S\subseteq V\backslash\{r\}}  \!\!\!\!\!\pi(S)z_S \\
\mbox{s.t.}\ \displaystyle \sum\limits_{\substack{e\in\delta(S):\\e=(\ast,v_i),\\ v_i\in\delta(S)}} \!\!\!\!\!\! x_e + 
\sum\limits_{T:T\supseteq S}  \!\!\! z_T\geqslant 1,\enskip  S\subseteq  V\backslash \{r\};\\
\displaystyle \sum\limits_{e\in E:~e=(\ast,v)}\!\!\!\!  x_e\leqslant 1,\enskip
x_e\in\{0,1\},\enskip z_S\in\{0,1\},\enskip  v\in V,\\
e\in E,\enskip S\subseteq V\backslash \{r\}.
\label{ilp_pcst_ord}
\end{multline}

\vspace*{-3pt}

\noindent
Если последнее ограничение из~(\ref{ilp_our}) входит в~оптимизируемый 
функционал, задачи $k$-MST и~A-PCST имеют эквивалентные 
ограничения и~отличаются только оптимизируемым функционалом. Такое 
преобразование возможно согласно условиям Ка\-ру\-ша--Ку\-на--Так\-ке\-ра~\cite{ras2017approximate}. Если значения призов 
эквивалентны $\pi(v) \hm=  \lambda$, единственное отличие состоит в~постоянном члене~$\lambda(n\hm-k)$. Таким 
образом, задачи оптимизации~$k$-MST и~A-PCST принимают вид:

\vspace*{-4pt}

\noindent
\begin{align*}
\underset{\substack{x_e,z_S \\ e\in E,\\ S\subseteq V\backslash \{r\}}}{\mbox{minimize}} & 
\sum\limits_{e\in E}c_ex_e + \lambda\left(\sum\limits_{S\subseteq V\backslash \{r\}}|S|z_S - (n-k)\right);\\ 
\underset{\substack{x_e,z_S \\ e\in E,\\ S\subseteq V\backslash \{r\}}}{\mbox{minimize}} & 
\sum\limits_{e\in E}c_ex_e + \lambda\sum\limits_{S\subseteq V\backslash\{r\}}|S|z_S\,. 
\end{align*}
%
Константа~$\lambda$ обозначает неотрицательный множитель Лагранжа в~задаче~$k$-MST и~приз за вершину\linebreak 
в~задаче~A-PCST. 
Существуют несколько алгоритмов для решения проблемы~PCST, но не для 
решения проб\-ле\-мы A-PCST. Возможное решение~--- снять 
ограничения на ориентацию графа, чтобы\linebreak алгоритм~PCST мог позже 
восстановить ориентацию дерева.

\vspace*{-9pt}

\section{Решение задачи восстановления ограниченного леса с~помощью алгоритма 
$(2-\varepsilon)$-приближения}

\vspace*{-3pt}

Обзор методов решения задачи восстановления ограниченного леса представлен 
в~\cite{goemans1995general}. Задан взвешенный неориентированный граф~$G\hm=(V,E)$. 
Все его веса~$w(e_i)\hm=c_i\geqslant 0$, $e_i\hm\in E$. Задана некоторая 
функция~$f:2^{V}\to \{0, 1\}$. Требуется решить задачу линейного 
программирования с~целочисленными ограничениями:

\vspace*{-4pt}

\noindent
\begin{multline}
\underset{x_e:~e\in E}{\mbox{minimize}} \displaystyle \sum\limits_{e\in E}c_ex_e\\
\mbox{s.t.}\  x\left(\delta(S)\right)\geqslant f(S),\enskip  S \subset V, \enskip S \not= \emptyset,\\
 x_e\in\{0,1\},\enskip  e\in E.
\label{ilp_cfp}
\end{multline}

\vspace*{-3pt}

\noindent
Здесь
$$
x(\delta(S))=\sum\limits_{e\in \delta(S)}x_e,
$$
где $x_e\hm=1$, если 
ребро~$e$ входит в~финальное решение. Функция~$\delta(S)$ обозначает все ребра 
из~$E$ такие, что только одна из смежных вершин входит в~$S$.

Предположим, что отображение~$f$ удовлетворяет условиям

\vspace*{-3pt}

\noindent
\begin{gather*}
f(V) = 0,\\
 \underbrace{f(S)=f(V\backslash S)}_{\mathrm{симметричность}},\\
\underbrace{A,B\!\subset\! V\!: A\!\cap\! B\! =\! \emptyset, f(A)\!=\!f(B)\!=\!0\!\to\! f(A\!\cup\! B)\! =\! 0}_{\mathrm{дизъюнктивность}}.
\end{gather*}

\vspace*{-2pt}

\noindent
При выполнении этих условий~$f$ задает число ребер, начинающихся в~множестве 
вершин~$S$.

\smallskip

\noindent
\textbf{Лемма 1.}
\textit{Пусть $B\subseteq S\subset V$. Тогда $f(S) \hm= 0$ и~$f(B) \hm= 0$ приводит к}~$f(S\backslash B) \hm= 0$.

\smallskip

Задача с~таким описанием относится к~\textit{задачам поиска оптимального леса с~ограничениями}. 
Такая постановка задачи~(\ref{ilp_cfp}) с~соответствующим 
отображением~$f$ подходит для многих известных задач взвешенных графов, 
например: минимальный магистральный поиск, $st$-путь, задача Штейнера на 
минимальном дереве. Последняя задача является NP-полной, поэтому применим 
приближенный алгоритм.

\smallskip

\noindent
\textbf{Определение 3} (\textbf{алгоритм $\bm{\alpha}$-аппроксимации}).
Эвристический полиномиальный алгоритм, дающий\linebreak решение некоторой задачи 
оптимизации, называется $\alpha$-ап\-прок\-си\-ма\-ци\-ей, если он гарантирует 
удовлетворяющее ограничениям решение этой задачи оптимизации с~коэффициентом, 
меньшим или равным~$\alpha$, так что решение отличается от оптимального не более 
чем в~$\alpha$ раз по оптимизируемому функционалу.


\smallskip

Чтобы предложить приближенный алгоритм, целочисленные ограничения 
в~(\ref{ilp_cfp}) должны быть ослаблены:

\vspace*{-3pt}

\noindent
\begin{multline*}
\underset{x_e:~e\in E}{\mbox{minimize}}\  \displaystyle \sum\limits_{e\in E}c_ex_e \\
\mbox{s.t.}\  \displaystyle \sum\limits_{e\in \delta(S)}x_e\geqslant f(S),\enskip S \subset V\,, \enskip S \not= \emptyset\,,\\
 x_e>0,\enskip  e\in E,
%\label{rlp_cfp}
\end{multline*}
Двойственная задача принимает вид:

\vspace*{-4pt}

\noindent
\begin{multline}
\underset{y_S:~S \subset V, \; S \not= \emptyset}{\mbox{maximize}}\  
\displaystyle \sum\limits_{S\subset V}f(S)y_S \\
\mbox{s.t.}\  \displaystyle \sum\limits_{S:~e\in \delta(S)}y_S\leqslant c_e,\enskip  e\in E\,,\\
 y_S>0,\enskip  S \subset V, \enskip S \not= \emptyset\,,
\label{rd_cfp}
\end{multline}

\vspace*{-3pt}

\noindent
относительно дополнительного условия
$$
y_S \left(\sum\limits_{e\in \delta(S)}x_e - f(S)\right) = 0\,,\enskip S\subset  V\,.
$$

Обозначим множество вершин $A=\{v\hm\in V: f(\{v\})\hm=1\}$. Предлагается адаптивный 
жадный алгоритм $\left(2-{2}/{\vert A\vert }\right)$-ап\-прок\-си\-ма\-ции для задач 
вида~(\ref{ilp_cfp}). Алгоритм состоит из двух этапов. На первом этапе он жадно 
объединяет кластеры вершин, увеличивая двойственные переменные~$y_S$. Изначально 
каждая вершина принадлежит своему клас\-те\-ру. Если сле\-ду\-ющее реб\-ро~$e$ достигает 
равенства в~ограничениях в~(\ref{rd_cfp}), это ребро добавляется к~множеству~$S$ и~связанные клас\-те\-ры объединяются. Этот этап аналогичен алгоритму минимального 
остовного дерева Крускала. На втором этапе из конечного множества~$S$ удаляются 
некоторые ребра. Если обрезка ребра не нарушает ограничений, то это реб\-ро должно 
быть удалено.


Индекс $Z_{\mathrm{DRLP}}$ в~алгоритме~1 обозначает линейное 
программирование с~двойной релаксацией. Начальное значение $F:=\emptyset$ 
в~алгоритме~1 эквивалентно предположению $x_e \hm= 0$, $ e \hm\in E$. 
По условиям нежесткости $y_S \hm= 0$, $S \hm\subset V$,  $S \hm\not= \emptyset$.

На каждом шаге алгоритма кластер $\mathcal{C}$ содержит две компоненты 
$\mathcal{C} \hm= \mathcal{C}_i \hm\cup \mathcal{C}_a$, где $C\hm\in\mathcal{C }_a$, если 
$f(C) \hm= 1$, и~$C\hm\in\mathcal{C}_i$ в~противном случае. Назовем~$\mathcal{C}_a$ 
активным компонентом.
Переменные~$d(v)$ в~этом алгоритме связаны с~переменными~$y_S$ из~(\ref{rd_cfp}) 
соотношением
$$
d(i) = \sum\limits_{S:i\in S}y_S.
$$ 

Рассмотрим две различные компоненты $C_q$ и~$C_p$, $C_q\cap C_p\hm=\emptyset$, на 
некоторой итерации первого этапа алгоритма. Все~$y_S$ должны быть равномерно 
распределены по некоторому~$\varepsilon$ без нарушения ограничений
$$
\sum\limits_ {S:~e\in \delta(S)}y_S\leqslant c_e. 
$$
В терминах $d(v)$ это условие принимает вид:
$$
\sum\limits_{S:~e\in \delta(S)}y_S = d\left(v_1\right)+d\left(v_2\right),\enskip e=\left( v_1,v_2\right),
$$
поэтому $y_S\hm=0$ для любого~$S$ такого, что $v_1, v_2\hm\in S$, потому что 
компоненты растут только на первом этапе. Увеличение некоторых компонент на~$\varepsilon$ приводит к~уравнению
$$
d(v_1)+d(v_2)+\varepsilon \left(f(C_q)+f(C_p)\right)\leqslant 
c_e,\ e=\left(v_1,v_2\right), 
$$
что приводит к~формуле, используемой в~строке~$10$ алгоритма~1. 
В~случае когда в~состав входит следующее ребро, сумма $\sum\nolimits_{S:~e\in 
\delta (S)}y_S$ не будет увеличиваться, поэтому ограничения выполняются.

Ребра, которые можно удалить из~$F$ без добавления новых активных компонентов, 
удаляются на втором этапе алгоритма. Следующая лемма определяет свойства 
компонент связ\-ности в~$F'$.


\smallskip

\noindent
\textbf{Лемма~2.}\
\textit{Для каждой компоненты связ\-ности~$N$ из~$F'$ выполняется равенство}: $f(N)\hm=0$.

\smallskip

Следующая теорема утверж\-да\-ет, что решение, полученное с~помощью описанного 
алгоритма, удовле\-тво\-ря\-ет ограничениям исходной задачи линейного 
программирования.

\smallskip

\noindent
\textbf{Теорема~1.}
\textit{Набор ребер $F'$, полученный алгоритмом~$1$, удовлетворяет всем 
ограничениям исходной задачи}~(\ref{ilp_cfp}).


\smallskip

\noindent
\textbf{Лемма~3.}\
\textit{Обозначим граф $H$, каждая вершина которого соответствует одной из компонент 
связ\-ности $C\in\mathcal{C}$ на фиксированном шаге алгоритма. Ребро $(v_1,v_2)$ 
присутствует, если существует ребро $\hat{e}$ исходного графа, входящее в~$F'$: 
$\hat{e} \in F'$, поэтому граф $H$~--- это лес. Внут\-ри $H$ нет листовых вершин, 
со\-от\-вет\-ст\-ву\-ющих неактивным вершинам исходного графа}.

\smallskip

\noindent
\textbf{Теорема 2.}
\textit{Алгоритм~$1$ представляет собой $\alpha$-при\-бли\-жен\-ный алгоритм для 
задачи}~(\ref{ilp_cfp}) \textit{с}~$\alpha \hm= 2 - {2}/{|A|}$, \textit{где} $A\hm=\{v\  V: 
f(\{v\})=1\}$.

\smallskip

Несмотря на эту теоретическую основу, не существует подходящей функции $f$ для 
постановки задачи PCST, указанной в~(\ref{ilp_cfp}). Чтобы быть 
применимым в~этих условиях, алгоритм~1 нуждается в~нескольких 
модификациях.

\vspace*{-9pt}

\section{Модифицированная постановка задачи для~PCST}

\vspace*{-3pt}

Как и~в случае A-PCST, упрощенный вид задачи линейного 
программирования PCST принимает вид:
\begin{multline*}
\underset{\substack{x_e,s_v \\ e\in E, v\in V\backslash \{r\}}}{\mbox{minimize}}\  
\displaystyle \sum\limits_{e\in E}c_ex_e + \sum\limits_{v\in V\backslash\{r\}} \left(1-s_v\right)\pi_v \\
\mbox{s.t.}\  \displaystyle \sum\limits_{e\in\delta(S)} \!\! x_e\geqslant s_v,\enskip S\subseteq V\backslash \{r\},\enskip v\in S,\\
x_e\geqslant 0,\enskip e\in E,\enskip s_v\geqslant 0,\enskip v\in V\backslash \{r\}.
%\label{rlp_pcst_inord}
\end{multline*}
Эта постановка задачи отличается от исходной~(\ref{ilp_pcst_ord}) тем, что с~ней 
возможно согласовать задачу $k$-MST. Индикаторы~$s_v$ показывают, что 
вершина~$v$ включена в~дерево.

Двойственная задача принимает вид:

\vspace*{-3pt}

\noindent
\begin{multline*}
\underset{\substack{y_S:~S\subset V\backslash\{r\}}}{\mbox{maximize}}\ 
\displaystyle \sum\limits_{S\in V\backslash\{r\}}y_S \\
\mbox{s.t.}\  \displaystyle \sum\limits_{S:e\in\delta(S)}y_S\leqslant c_e ,\enskip e\in E;\\
 \displaystyle \sum\limits_{S\subseteq T}y_S\leqslant \sum\limits_{v\in T}\pi_v,\enskip  T\subset  V\backslash\{r\},\\
 y_S\geqslant 0,\enskip  S\subset V\backslash\{r\}.
%\label{rd_pcst_inord}
\end{multline*}

\vspace*{-3pt}

Алгоритм~2 решает эту задачу. Он похож на 
алгоритм~1. Двойные переменные должны обновляться равномерно 
с~дополнительными ограничениями. Тогда~$\varepsilon$ примет минимальное из двух 
значений в~соответствии с~обеими группами ограничений.
Более широкий анализ аппроксимационных свойств обновленного алгоритма 
представлен в~\cite{goemans1995general}. Алгоритм~2 представляет 
собой $\alpha$-приближенный алгоритм для задачи PCST с~$\alpha \hm= 2 \hm- 
{2}/({n-1})$, где $n$~--- число вершин в~графе~$G$.

\vspace*{-9pt}

\section{Вычислительный эксперимент}

\vspace*{-3pt}

Основная цель эксперимента~--- восстановить дерево суперпозиции. Алгоритмы, 
используемые для восстановления, перечислены ниже.

\vspace*{-14pt}

\paragraph*{DFS, BFS.}
Алгоритмы жадного дерева обхода в~глубину и~жадного дерева обхода в~ширину. 
Обход ребер с~наибольшим весом эквивалентен выбору наиболее вероятного пути. 
Алгоритм обхода останавливается, когда число ребер, исходящих из некоторой 
вершины, становится равным арности соответствующей функции.

\vspace*{-14pt}

\paragraph*{Алгоритм Прима.}
Алгоритм восстанавливает минимальное остовное дерево для графа с~дополнительными 
ограничениями на арность базовых функций. Эти ограничения задают минимальный вес 
ребра. После добавления вершины все лис\-то\-вые ребра этой вершины исключаются, 
чтобы сохранить направление дерева. Если число ребер, начинающихся в~какой-либо 
вершине, превышает соответствующую арность, то остальные ребра исключаются из 
множества возможных ребер в~этой вершине. Алгоритм не зависит от процедуры 
обхода. В случае небольшого шума в~матрице смежности этот алгоритм способен 
восстановить дерево суперпозиции без ошибок. 


\vspace*{-14pt}

\paragraph*{Алгоритмы на основе PCST.}
Матрица смеж\-ности~$M$ должна быть приведена к~неориентиро-\linebreak\vspace*{-12pt}

\pagebreak

\noindent
ванному виду. 
Использована квад\-рат\-ная мат\-ри\-ца~$M'$ без последнего столбца. PCST 
принимает мат\-ри\-цу смеж\-ности $1 \hm- ({1}/{2})(M' \hm+ M'^{\mathsf{T}})$ с~призовым 
значением~0,5 для каж\-дой вершины.
Призовое значение рав\-но~0,5, поскольку при меньших значениях дерево будет 
обрезано: если шум равен~0,5, некоторые вершины могут быть обрезаны по ошибке. 
В~случае больших призовых значений
дерево PCST может содержать ненужные 
вершины. Дерево восстанавливается по одному из опи-\linebreak\vspace*{-12pt}

{ \begin{center}  %fig2
 \vspace*{9pt}
    \mbox{%
\epsfxsize=79mm
\epsfbox{str-2.eps}
}
\end{center}



\noindent
{{\figurename~2}\ \ \small{Качество алгоритмов восстановления с~базовыми функциями небольших 
арностей: \textit{1}~--- DFS; \textit{2}~--- BFS; \textit{3}~--- алгоритм Прима;
\textit{4}~--- $k$-MST; \textit{5}~--- $k$-MST--DFS; \textit{6}~--- $h$-MST--BFS; \textit{7}~--- $k$-MST\,--\,ал\-го\-ритм Прима
}}}

\vspace*{6pt}

\addtocounter{figure}{1}

%\begin{table*}\small  %tabl2
\begin{center}
\parbox{75mm}{{{\tablename~2}\ \ \small{Качество алгоритмов реконструкции с~равномерным шумом, близким 
к~0,5
}}
}
    
    
\vspace*{6pt}

  {\small  \begin{tabular}{|l|ccccc|}
      \hline
                  & \multicolumn{5}{c|}{Шум}\\%& & Шум & & \\
       \cline{2-6}
        \multicolumn{1}{|c|}{\raisebox{6pt}[0pt][0pt]{Алгоритм}}                          
&0,50&0,52&0,54&0,56&0,58\\
                    \hline
      DFS        &0,20 &0,20 &0,19 &0,18 &0,16\\
      BFS        &0,60 &0,58 &0,51 &0,46 &0,40\\
      Прима    &1,00 &0,94&0,81&0,69&0,57\\
      $k$-MST     &0,17 &0,16 &0,14 &0,12 &0,10\\
      $k$-MST--DFS   &0,17 &0,16 &0,16 &0,14 &0,14 \\
      $k$-MST--BFS   &0,43 &0,40 &0,36 &0,33 &0,29 \\
      $k$-MST--Прима  &0,44 &0,39 &0,34 &0,33 &0,27 \\
      \hline
    \end{tabular}
    }
\end{center}
%\end{table*}




\noindent
 санных алгоритмов. Результаты 
$\text{PCST}$ можно использовать в~качестве априорных для других подходов, $M':=({1}/{2})(M_{\mathrm{PCST}}' + M')$,
поэтому результаты \mbox{PCST} обновляются~$M'$.


Процедура генерации данных имеет следующие допущения: арности функций 
генерируются биномиальным распределением, поэтому существуют много функций 
с~малой арностью, все базовые функции имеют только один вход. Любой случай 
с~частичной реконструкцией считается ошибкой. Качество алгоритмов реконструкции:
$$
\fr{1}{K}\sum\limits_{k=1}^K \left[ R\left( \bar{N}(M_k)\right)=M_k\right],
$$
где~$R$ ~--- алгоритм реконструкции;
$\bar{N}\hm=\left(N - \min(N)\right)/\left(\max(N)\hm-\min(N)\right)$~--- нормированная мат\-ри\-ца шума. 
Мат\-ри\-ца~$N$ генерируется как~$N(M)\hm=M\hm+U(-\alpha,\alpha)$.
Генератор случайных чисел возвращает матрицу того же вида, что и~$M$, где каждый 
элемент является независимой переменной из равномерного распределения 
в~сегменте~$[-\alpha,\alpha]$.

Вот список из семи сравниваемых алгоритмов:
DFS,
BFS,
алгоритм Прима,
$k$-MST через PCST,
$k$-MST\;+\;DFS,
$k$-MST\;+\;BFS,
$k$-MST\;+\;ал\-го\-ритм Прима.
На рис.~2 показана ошибка алгоритмов реконструкции 
с~шумом, близ\-ким к~порогу~0,5. Наилучшие результаты дает алгоритм Прима. Второе по 
точности решение основано на~$\text{BFS}$. Таб\-ли\-ца~2 
соответствует~рис.~2 и~показывает качество реконструкции 
семи алгоритмов для значений граничного шума~0,50--0,58.





\vspace*{-9pt}

\section{Заключение}

\vspace*{-3pt}

Предлагаются и~сравниваются  алгоритмы вос\-ста\-нов\-ле\-ния суперпозиции для задачи 
символьной регрессии. Алгоритм Прима дает наиболее точ\-ные результаты и~устойчив 
к~небольшому шуму в~данных. Пред\-ла\-га\-емый алгоритм дает точные результаты, но он 
более подвержен шуму в~мат\-ри\-це суперпозиции. Алгоритмы, основанные на BFS и~DFS, 
не могут вос\-ста\-но\-вить исходную суперпозицию с~зашумленными мат\-ри\-ца\-ми 
суперпозиции. Алгоритм PCST с~BFS, используемый для реконструкции мат\-ри\-цы 
суперпозиции, показывает приемлемые для практического использования результаты.

{\small\frenchspacing
 {%\baselineskip=10.8pt
 %\addcontentsline{toc}{section}{References}
 \begin{thebibliography}{99}
\bibitem{koza1992genetic}  %1
\Au{Koza J.\,R.} Genetic programming as a means for programming computers by 
natural selection~// Stat. Comput., 1994. Vol.~4. P.~87--112.

\bibitem{searson2010gptips} %2
\Au{Searson~D.\,P., Leahy~D.\,E., Willis~M.\,J.} GPTIPS: An open source 
genetic programming toolbox for multigene  symbolic regression~// 
Multiconference (International) of Engineers and Computer Scientists Proceedings, 
2010. Vol.~1. P.~77--80.

\bibitem{stanley2002evolving} %3
\Au{Stanley~K.\,O., Miikkulainen~R.} Evolving neural networks through 
augmenting topologies~// Evol. Comput., 2002. Vol.~10. 
Iss.~2. P.~99--127.

\bibitem{bochkarev2017generation}
\Au{Бочкарев~А.\,М., Софронов~И.\,Л., Стрижов~В.\,В.} По\-рож\-де\-ние экс\-перт\-но-ин\-тер\-пре\-ти\-ру\-емых 
моделей для прогноза проницаемости горной породы~// Системы и~средства информатики, 2017. Т.~27. №\,3. С.~74--87.
%

\bibitem{ravi1996spanning}
\Au{Ravi~R., Sundaram~R., Marathe~M.\,V., Rosenkrantz~D.\,J., Ravi~S.\,S.} 
Spanning trees~--- short or small~// SIAM J.~Discrete Math., 
1996. Vol.~9. Iss.~2. P.~178--200.

\bibitem{chudak2004approximate}
\Au{Chudak~F.\,A.,  Roughgarden~T., Williamson~D.\,P.} Approximate $k$-MSTS 
and $k$-Steiner trees via the primal-dual method and Lagrangean 
relaxation~// Math. Program., 2004. Vol.~100. Iss.~2. P.~411--421.

\bibitem{awerbuch1998new}
\Au{Awerbuch~B., Azar~Y., Blum~A., Vempala~S.} New approximation guarantees 
for minimum-weight $k$-trees and prize-collecting salesmen~// SIAM J. 
Comput., 1998. Vol.~28. Iss.~1. P.~254--262.

\bibitem{arora20062+}
\Au{Aror~S., Karakostas~G.} A~$2+\varepsilon$ approximation algorithm for the 
$k$-MST problem~// Math. Program., 2006. Vol.~107. 
Iss.~3. P.~491--504.

\bibitem{hegde2014fast}
\Au{Hegde~C., Indyk~P., Schmidt~L.} A~fast, adaptive variant of the 
Goemans--Williamson scheme for the prize-collecting steiner tree problem~// 11th DIMACS Implementation Challenge Workshop Proceedings, 2014. P.~1--32.
{\sf http://people. csail.mit.edu/ludwigs/papers/dimacs14\_fastpcst.pdf}.

\bibitem{ras2017approximate}
\Au{Ras~C., Swanepoel~K., Thomas~D.\,A.} Approximate Euclidean Steiner 
trees~// J.~Optimiz. Theory App., 2017. Vol.~172. 
Iss.~3. P.~845--873.

\bibitem{goemans1995general}
\Au{Goemans~M.\,X., Williamson~D.\,P.} A~general approximation technique for 
constrained forest problems~// SIAM J. Comput., 1995. Vol.~24. 
Iss.~2. P.~296--317.
\end{thebibliography}

 }
 }

\end{multicols}

\vspace*{-6pt}

\hfill{\small\textit{Поступила в~редакцию 23.01.22}}

\vspace*{8pt}

%\pagebreak

%\newpage

%\vspace*{-28pt}

\hrule

\vspace*{2pt}

\hrule

%\vspace*{-2pt}

\def\tit{OPTIMAL SPANNING TREE RECONSTRUCTION IN~SYMBOLIC~REGRESSION}


\def\titkol{Optimal spanning tree reconstruction in~symbolic regression}


\def\aut{R.\,G.~Neychev$^1$, I.\,A.~Shibaev$^1$, and~V.\,V.~Strijov$^2$}

\def\autkol{R.\,G.~Neychev, I.\,A.~Shibaev, and~V.\,V.~Strijov}

\titel{\tit}{\aut}{\autkol}{\titkol}

\vspace*{-8pt}


\noindent
$^1$Moscow Institute of Physics and Technology, 9~Institutskiy Per., Dolgoprudny, Moscow Region 141700, Russian\linebreak
$\hphantom{^1}$Federation

\noindent
$^2$Federal Research Center ``Computer Science and Control'' of the Russian Academy of Sciences, 44-2~Vavilov Str.,\linebreak
$\hphantom{^1}$Moscow 119333, Russian Federation

\def\leftfootline{\small{\textbf{\thepage}
\hfill INFORMATIKA I EE PRIMENENIYA~--- INFORMATICS AND
APPLICATIONS\ \ \ 2023\ \ \ volume~17\ \ \ issue\ 1}
}%
 \def\rightfootline{\small{INFORMATIKA I EE PRIMENENIYA~---
INFORMATICS AND APPLICATIONS\ \ \ 2023\ \ \ volume~17\ \ \ issue\ 1
\hfill \textbf{\thepage}}}

\vspace*{3pt} 



\Abste{The paper investigates the problem of regression model generation. A~model is a~superposition of primitive functions. 
The model structure is described by a~weighted colored graph. Each graph vertex corresponds to a~primitive function. 
An edge assigns a~superposition of two functions. The weight of an edge is equal to the probability of superposition. 
To generate an optimal model, one has to reconstruct its structure from its graph adjacency matrix. 
The proposed algorithm reconstructs the minimum spanning tree from the weighted colored graph. 
The paper presents a~novel solution based on the prize-collecting Steiner tree algorithm. This algorithm is compared with its alternatives.}


\KWE{symbolic regression; linear programming; superposition; minimum spanning tree; adjacency matrix}



\DOI{10.14357/19922264230105} 

\vspace*{-16pt}

\Ack

\vspace*{-3pt}


\noindent
This work was supported by the Russian Foundation for Basic Research, projects 20-37-90050 and 20-07-00990.
  

\vspace*{6pt}

  \begin{multicols}{2}

\renewcommand{\bibname}{\protect\rmfamily References}
%\renewcommand{\bibname}{\large\protect\rm References}

{\small\frenchspacing
 {%\baselineskip=10.8pt
 \addcontentsline{toc}{section}{References}
 \begin{thebibliography}{99} 

\bibitem{1-str}
\Aue{Koza, J.\,R.}
 1994. Genetic programming as a means for programming computers by natural selection. \textit{Stat. Comput.} 4:87--112.

\bibitem{2-str}
\Aue{Searson, D.\,P., D.\,E.~Leahy, and M.\,J.~Willis.}
 2010. \mbox{GPTIPS}: An open source genetic programming toolbox for multigene symbolic regression. 
 \textit{Multiconference (International) of Engineers and Computer Scientists Proceedings}. 1:77--80. 

\bibitem{3-str}
\Aue{Stanley, K.\,O., and R.~Miikkulainen.} 2002. Evolving neural networks through augmenting topologies. 
\textit{Evol. Comput.} 10(2):99--127.

\bibitem{4-str}
\Aue{Bochkarev, A.\,M., I.\,L.~Sofronov, and V.\,V.~Strijov.}
 2017. Po\-rozh\-de\-nie eks\-pert\-no-inter\-pre\-ti\-ru\-emykh mo\-de\-ley dlya prog\-no\-za pro\-ni\-tsa\-emosti gor\-noy po\-ro\-dy 
 [Generation of expertly-interpreted models for prediction of core permeability]. \textit{Sistemy i~Sredstva Informatiki~--- Systems and Means of Informatics}
  27(3):74--87.

\bibitem{5-str}
\Aue{Ravi, R., R.~Sundaram, M.\,V.~Marathe, D.\,J.~Rosenkrantz, and S.\,S.~Ravi.}
 1996. Spanning trees~--- short or small. \textit{SIAM J. Discrete Math.} 9(2):178--200.

\bibitem{6-str}
\Aue{Chudak, F.\,A., T.~Roughgarden, and D.\,P.~Williamson.}
 2004. Approximate k-MSTS and k-Steiner trees via the primal-dual method and Lagrangean relaxation. 
 \textit{Math. Program.} 100(2):411--421.

\bibitem{7-str}
\Aue{Awerbuch, B., Y.~Azar, A.~Blum, and S.~Vempala.}
 1998. New approximation guarantees for minimum-weight \mbox{k-trees} and prize-collecting salesmen.
 \textit{SIAM J. Comput.} 28(1):254--262.

\bibitem{8-str}
\Aue{Arora, S., and G.~Karakostas.} 2006. A~$2+\varepsilon$ approximation algorithm for the $k$-MST problem. 
\textit{Math. Program.} 107(3):491--504.

\bibitem{9-str}
\Aue{Hegde, C., P.~Indyk, and L.~Schmidt.} 2014. 
A~fast, adaptive variant of the Goemans--Williamson scheme for the prize-collecting Steiner tree problem. 
\textit{11th DIMACS Implementation Challenge Workshop Proceedings}. 1--32.
Available at: 
{\sf http://people.csail.mit.edu/ludwigs/papers/\linebreak dimacs14\_fastpcst.pdf} (accessed January~10, 2023).

\bibitem{10-str}
\Aue{Ras, C., K.~Swanepoel, and D.\,A.~Thomas.} 
2017. Approximate Euclidean Steiner trees. \textit{J.~Optimiz. Theory  App.} 172(3):845--873.

\bibitem{11-str}
\Aue{Goemans, M.\,X., and D.\,P.~Williamson.} 1995. 
A~general approximation technique for constrained forest problems. \textit{SIAM J. Comput.} 24(2):296--317.
 \end{thebibliography}

 }
 }

\end{multicols}

\vspace*{-6pt}

\hfill{\small\textit{Received January 23, 2022}}

\Contr

\noindent
\textbf{Neychev Radoslav G.} (b.\ 1994)~--- 
PhD student, Moscow Institute of Physics and Technology, 9~Institutskiy Per., Dolgoprudny, Moscow Region 141701, Russian Federation;
\mbox{neychev@phystech.edu}

\vspace*{3pt}

\noindent
\textbf{Shibaev Innokentii A.} (b.\ 1997)~--- 
PhD student, Moscow Institute of Physics and Technology, 9~Institutskiy Per., Dolgoprudny, Moscow Region 141701, Russian Federation; 
\mbox{shibaev.kesha@gmail.com}

\vspace*{3pt}

\noindent
\textbf{Strijov Vadim V.} (b.\ 1967)~--- 
Doctor of Science in physics and mathematics, leading scientist, A.\,A.~Dorodnicyn Computing Center, 
Federal Research Center ``Computer Science and Control'' of the Russian Academy of Sciences, 40~Vavilov Str., Moscow 119333, Russian Federation;
\mbox{strijov@phystech.edu}


\label{end\stat}

\renewcommand{\bibname}{\protect\rm Литература}       %6
\def\stat{flerov}

\def\tit{АВТОМАТИЗИРОВАННАЯ СИСТЕМА ВЕСОВОГО 
ПРОЕКТИРОВАНИЯ САМОЛЕТОВ}

\def\titkol{Автоматизированная система весового 
проектирования самолетов}

\def\aut{Л.\,Л.~Вышинский$^1$, Ю.\,А.~Флеров$^2$, Н.\,И.~Широков$^1$}

\def\autkol{Л.\,Л.~Вышинский, Ю.\,А.~Флеров, Н.\,И.~Широков}

\titel{\tit}{\aut}{\autkol}{\titkol}

\index{Вышинский Л.\,Л.}
\index{Флеров Ю.\,А.}
\index{Широков Н.\,И.}
\index{Vyshinsky L.\,L.}
\index{Flerov Yu.\,A.}
\index{Shirokov N.\,I.}




%{\renewcommand{\thefootnote}{\fnsymbol{footnote}} \footnotetext[1]
%{Работа выполнена при финансовой поддержке РФФИ (проект 17-01-00816).}}


\renewcommand{\thefootnote}{\arabic{footnote}}
\footnotetext[1]{Вычислительный центр им.\ А.\,А.~Дородницына Федерального исследовательского 
центра <<Информатика и~управ\-ле\-ние>> Российской академии наук, 
\mbox{Wysh@ccas.ru}}
\footnotetext[2]{Вычислительный центр им.\ А.\,А.~Дородницына Федерального исследовательского 
центра <<Информатика и~управ\-ле\-ние>> Российской академии наук, 
fler@ccas.ru}
%\footnotetext[3]{Вычислительный центр им.\ А.\,А.~Дородницына Федерального исследовательского 
%центра <<Информатика и~управ\-ле\-ние>> Российской академии наук, 
%\mbox{Wysh@ccas.ru}}

%\vspace*{-6pt}


 
  \Abst{Статья посвящена вопросам автоматизации задач весового проектирования 
самолетов. Весовые и~мас\-со\-во-инер\-ци\-он\-ные параметры являются одними из основных 
величин, влияющих на эксплуатационные характеристики самолетов. Информационной 
основой системы служит весовая модель самолета. Описывается структура весовой 
модели и~даны характеристики отдельным ее компонентам. Показана программная 
реализация системы, которая выполнена в~рамках архитектуры кли\-ент--сер\-вер. 
Автоматизированная система весового проектирования (АСВП)
реализована с~использованием 
про\-грам\-мно-ин\-стру\-мен\-таль\-но\-го комплекса <<Генератор проектов>> (технология ГП), 
который был разработан в~Вычислительном центре Российской академии наук. Создание 
ин\-фор\-ма\-ци\-он\-но-вы\-чис\-ли\-тель\-ных сис\-тем в~рамках технологии ГП базируется на так 
называемом <<проектном подходе>>, когда по формальному описанию системы автоматически 
генерируются тексты программного кода для клиентских и~серверных компонент системы.}
   
  \KW{математическое моделирование; автоматизация проектирования; самолет; весовое 
проектирование; весовая модель; дерево конструкции; генератор проектов; генерация 
программного кода; архитектура кли\-ент--сер\-вер}

  \DOI{10.14357/19922264180103} 
  
\vspace*{12pt}


\vskip 10pt plus 9pt minus 6pt

\thispagestyle{headings}

\begin{multicols}{2}

\label{st\stat}
   
\section{Введение}

  Развитие и~повсеместное использование информационных технологий за 
последние несколько десятилетий существенно изменили традиционный 
процесс проектирования и~создания различных инженерных систем, 
сооружений, машин. Во многих проектных организациях давно отказались от 
ко\-гда-то привычных инструментов конструктора~--- кульмана 
и~логарифмической линейки. 
%
Сейчас первые эскизы новых проектов 
появляются чаще не на бумаге, как было всегда, а~на экране монитора. Этому 
способствует широкий спектр имеющихся систем автоматизированного 
проектирования. В~российских авиационных конструкторских бюро, например, уже давно 
применяются такие CAD (computer aided design)
сис\-те\-мы, как NX (Unigraphics), CATIA и~др. 
%
Эти развитые системы геометрического трех\-мер\-но\-го (3D) мо\-де\-ли\-ро\-ва\-ния позволяют 
создавать сложные по\-верх\-ности, конструировать любые детали, осуществлять 
сборку узлов, агрегатов и~сложнейших изделий. Однако построение 
геометрических моделей изделий является финальной стадией проектирования, 
за которой следует их реализация <<в~металле>>. Построению электронных 
геометрических макетов предшествует и~сопутствует решение множества 
расчетных задач, а~также задач анализа и~оптимизации в~разных областях инженерных 
знаний. В~авиастроении это аэродинамика, динамика полета, прочность, 
системы управления, двигателестроение и~пр. Все эти задачи 
требуют разработки разноплановых математических моделей и~специальных 
вычислительных программ. 
  
  Одной из важнейших технических характеристик самолета является его вес. 
При решении подавляющего большинства проектных и~конструкторских задач 
весовые параметры в~том или ином виде участвуют в~расчетах. Необходимость 
проведения весовых расчетов возникает на самых ранних шагах 
проектирования и~сопровождает все дальнейшие стадии разработки 
и~эксплуатации. 

В~процессе создания и~эксплуатации самолетов постоянно 
контролируются вес и~другие мас\-со\-во-инер\-ци\-он\-ные характеристики (МИХ)
всех размещаемых на борту систем, агрегатов, узлов и~деталей. Количество 
агрегатов, узлов и~деталей современных самолетов исчисляется 
десятками тысяч, поэтому в~авиастроении весовые расчеты, весовой анализ, 
весовой контроль выливаются в~сложную инженерную проблему и~выделяются 
в~целое направление инженерной деятельности, которое принято называть 
весовым проектированием~[1].
  
  Данная статья посвящена вопросам автоматизации задач весового 
проектирования самолетов. В~разные годы Вычислительным центром РАН\linebreak был 
разработан и~внедрен в~эксплуатацию ряд \mbox{программ}, решающих отдельные 
задачи весовых рас\-че\-тов летательных аппаратов (ЛА)~[2--4]. В~настоящей статье 
представлено описание интегрированной АСВП, предназначенной для использования на всех 
этапах жизненного цикла изделий. Она разработана как интерактивная 
многопользовательская информационная система кли\-ент-сер\-вер\-ной 
архитектуры с~централизованной базой данных. Информационным ядром 
и~основой АСВП является единая струк\-тур\-но-па\-ра\-мет\-ри\-че\-ская весовая модель 
самолета, описание которой дает довольно полное представление о~задачах, 
решаемых с~помощью АСВП.

\section{Структурно-параметрическая весовая модель самолета}

  Самолет является сложным техническим объ\-ектом, состоящим из множества 
различных \mbox{ком\-понентов}, функционально и~конструктивно связанных между 
собой. Под струк\-тур\-но-па\-ра\-мет\-ри\-че\-ской весовой моделью самолета 
здесь понимается база данных, которая содержит всю необходи\-мую 
информацию для проведения комплекса расчетов 
МИХ и~мас\-со\-во-цент\-ро\-воч\-ных данных (МЦД) 
самолета. Весовая модель состоит из нескольких структур, ориентированных на 
определенные группы параметров и~задач весового проектирования. Ниже 
перечислены основные структуры весовой модели, реализованные в~системе 
АСВП:
  \begin{itemize}
\item дерево конструкции самолета;
\item иерархия систем координат, связанных с~самолетом и~его агрегатами;
\item геометрические структуры весовой модели самолета;
\item каталог целевой нагрузки, размещаемой во внут\-рен\-них отсеках и~на 
подвесках;
\item реестр допустимых вариантов загрузки само\-лета;
\item таблицы тарировочных характеристик топливных баков;
\item таблицы характеристик выработки топлива.
\end{itemize}


  \subsection{Дерево конструкции самолета}

  Дерево конструкции самолета является центральной структурой весовой 
модели, которая отражает членение изделия на его составные части~--- 
системы, агрегаты, узлы, детали. В~базе данных весовой модели эта структура 
представлена в~виде многоуровневого корневого дерева $W \hm= (U, V)$, где 
вершинам $U \hm= \{U_i\}$ соответствуют различные\linebreak
 элементы конструкции. 
Ориентированные дуги дере\-ва, идущие из~$U_i$ в~$U_j$, означают вхождение 
конструкции~$U_j$ в~конструкцию~$U_i$ в~качестве ее составной части. 
Терминальными или висячими вершинами дерева конструкции будем называть 
вершины, у которых нет ни одной подчиненной конструкции.
  
  Многолетний опыт самолетостроения выработал устоявшиеся 
конструктивные схемы самолетов различного назначения. Существуют 
отраслевые стандарты и~классификаторы, которые вводят определения 
основных элементов конструкции самолетов. На рис.~1 показан пример 
представления в~АСВП нескольких верхних уровней дерева конструкции 
самолета. 


    

  Существующие классификаторы отражают лишь самые общие принципы 
построения конструкции самолетов. Разумеется, каждый новый проект 
самолета имеет свои конструктивные особенности, которые отражаются на 
структуре весовой модели. Дерево конструкции строится постепенно, сверху 
вниз, в~течение всего процесса проектирования самолета. 

 { \begin{center}  %fig1
 \vspace*{9pt}
\mbox{%
 \epsfxsize=77.216mm 
 \epsfbox{fle-1.eps}
 }

\vspace*{6pt}


\noindent
{{\figurename~1}\ \ \small{Дерево конструкции самолета}}
\end{center}
}

\addtocounter{figure}{1}
  
  Понятие <<конструкции>> в~данном контексте используется и~для 
обозначения любой вершины графа, и~для всего поддерева подчиненных 
конструкций этой вершине. Каждая конструкция дерева имеет уникальное 
в~рамках весовой модели обозначение, которое может быть шифром, кодом, 
идентификатором или чертежным номером конструкции. Разумеется, для более 
полного и~наглядного вербального представления конструкции  
в~струк\-тур\-но-па\-ра\-мет\-ри\-че\-ской модели можно задать ее текстовое 
описание.
  
  \textbf{Масса конструкции.} Основную содержательную и~необходимую 
информацию весовой модели содержит список значений  
МИХ, соответствующих каждой 
вершине дерева конструкций. Центральным параметром является масса. 
  
  На разных стадиях создания самолета, когда неизвестно точное значение 
массы, прибегают к~различным оценкам.  
В~струк\-тур\-но-па\-ра\-мет\-ри\-че\-ской весовой модели фиксируются 
перечисленные ниже оценки массы, которые соответствуют разным этапам 
проектирования:
  \begin{description}
\item[\,]  $M_{\mathrm{теор}}$~--- теоретическая масса~--- оценка массы, вычисленная на 
основании некоторой математической модели конструкции; 
  
\item[\,]  $M_{\mathrm{лим}}$~--- лимитная масса конструкции, уста\-нав\-ли\-ва\-емая на 
основании теоретических оценок и~используемая для весового контроля 
в~процессе детальной разработки конструкции;
  
\item[\,]  $M_{\mathrm{черт}}$~--- чертежная масса конструкции, рассчитанная по чертежу или по 
электронной гео\-мет\-ри\-че\-ской модели конструкции;
  
\item[\,]  $M_{\mathrm{креп}}$~--- масса крепежа конструкции~--- дополнение к~чертежной массе, 
которое учитывает мелкие детали конструкции, предназначенные для 
соединения подчиненных деталей (заклепки, болты, гайки, шайбы и~т.\,п.). 
Введение такой дополнительной массы позволяет избавить дерево конструкции 
от десятков и~сотен тысяч вершин, которые несут относительно небольшую 
нагрузку в~весовых характеристиках, но тем не менее их учет необходим при 
контроле веса. Масса крепежа распределяется по подчиненным конструкциям;  
\item[\,]  $M_{\mathrm{факт}}$~--- фактическая масса изготовленной 
и~взвешенной конструкции. 
Фактическая масса может задаваться не только для изготавливаемых 
конструкций ЛА, но и~для готовых по\-став\-ля\-емых 
изделий при их установке на борту.
\end{description}
  
  Порядок задания оценок массы диктуется логикой развития проекта. 
В~дереве конструкции все оценки массы, кроме $M_{\mathrm{лим}}$ и~$M_{\mathrm{креп}}$, 
суммируются по подчиненным вершинам снизу вверх. Однако если для 
некоторых терминальных значений одна или несколько оценок не определены, 
например некоторые детали конструкции не изготовлены и, стало быть, 
$M_{\mathrm{факт}}$ не определена, то и~для всех вышестоящих конструкций эти оценки не 
определены. При задании $M_{\mathrm{лим}}$ это правило может не соблюдаться. 
  
  На основании оценок массы определяется то расчетное значение массы 
конструкции, которое используется во всех расчетах на текущей стадии 
проекта: 
  $M$~--- текущая масса конструкции. Значение текущей массы \textit{для 
нетерминальных} конструкций определяется суммированием по подчиненным 
конструкциям. \textit{Для терминальных} вершин дерева конструкций 
применяется процедура определения текущей массы по первому известному 
значению из следующего списка в~указанном порядке: $M_{\mathrm{факт}}$, 
$M_{\mathrm{черт}}$\;+\;$M_{\mathrm{креп}}$, $M_{\mathrm{теор}}$, $M_{\mathrm{лим}}$.
  
  \textbf{Геометрия масс конструкции.} Кроме собственно массы в~весовой 
модели задаются или вычисляются значения характеристик, которые принято 
называть характеристиками геометрии масс: 
  \begin{description}
  \item[\,] $X$, $Y$ и $Z$~--- положение центра масс конструкции; 
  \item[\,] $L_x$, $L_y$ и $L_z$~--- габаритные параметры конструкции;
  \item[\,] $I_x$, $I_y$ и $I_z$~--- полные плоскостные моменты инерции;
  \item[\,]  $I_{xy}$, $I_{xz}$ и $I_{yz}$~--- полные центробежные моменты 
инерции;
  \item[\,] $I^c_x$, $I^c_y$ и  $I^c_z$~--- собственные плоскостные моменты 
инерции:
  \begin{align*}
  I^c_x &= I_x - M X^2\,;\\ 
  I^c_y &= I_y - M Y^2\,;\\ 
  I^c_z &= I_z - M Z^2\,;
 \end{align*}
  \item[\,] $I^c_{xy}$, $I^c_{xz}$ и~$I^c_{yz}$~--- собственные центробежные 
моменты инерции:
 \begin{align*}
  I^c_{xy} &= I_{xy}- M X Y\,;\\
   I^c_{xz} &= I_{xz}- M X Z\,;\\
   I^c_{yz} &= I_{yz}- M Y Z\,;
\end{align*}
  \item[\,] $J_x$, $J_y$ и $J_z$~--- собственные осевые моменты инерции 
конструкции:
  \begin{align*}
  J_x &= I^c_y + I^c_z\,;\\ 
  J_y &= I^c_x + I^c_z\,;\\
   J_z &= I^c_y + I^c_x\,;
  \end{align*}
  \item[\,] СК~--- система координат конструкции, в~которой задаются 
характеристики геометрии масс.
  \end{description}
  
  \begin{figure*} %fig2
  \vspace*{1pt}
 \begin{center}
 \mbox{%
 \epsfxsize=162mm 
 \epsfbox{fle-2.eps}
 }
 \end{center}
\vspace*{-9pt}
  \Caption{Основные параметры конструкций весовой модели самолета}
  \end{figure*}
  
  Каждая конструкция привязывается к~одной из систем координат, которые 
описаны в~весовой модели. В~весовой модели изделия для удобства описания 
различных агрегатов может быть описано несколько систем координат. Все 
описанные сис\-те\-мы координат организованы в~иерархическую структуру. 
Считается предописанной глобальная самолетная система координат, в~которой 
могут быть заданы или вычислены координаты всех объектов весовой модели. 
Каждая система координат в~весовой модели задается уникальным именем, 
положением начала координат относительно вышестоящей системы координат 
и~тремя углами поворота относительно вышестоящей. 

Параметр, 
обозначенный как СК,~--- это имя одной из сис\-тем координат весовой модели. 
Если СК не задано, то считается, что характеристики гео\-мет\-рии масс заданы 
в~глобальной системе координат. Каж\-дая сис\-те\-ма координат весовой модели 
содержит матрицу преобразования координат из самолетной (глобальной) 
системы координат в~данную и~обратно. Эта матрица для каждой системы 
координат есть произведение локальных матриц преобразований 
в~соответствии с~положением данной системы в~иерархии систем координат. 
Любое изменение параметров систем координат требует пе\-ре\-вы\-чис\-ле\-ния 
матриц преобразования как измененной сис\-те\-мы, так и~всех подчиненных. На 
рис.~2 показана панель параметрического пред\-став\-ле\-ния конструкций весовой 
модели.
  
  Так же как и~масса, центры тяжести и~моменты инерции вычисляются снизу
вверх от терминальных конструкций к~вышестоящим. При этом осуществляется 
пересчет характеристик по заданной иерархии систем координат от 
нижестоящих к~вышестоящим и~к~самолетной системе координат. Расчет 
МИХ терминальных конструкций 
осуществляется на основании гео\-мет\-ри\-че\-ских моделей. Геометрические модели 
на этапе рабочего проекта строятся в~системах гео\-мет\-ри\-че\-ско\-го 
моделирования. В~процессе их построения автоматически вычисляются 
объемы, массы, положение центра тяжести и~другие характеристики гео\-мет\-рии 
масс. Рассчитанная в~системах гео\-мет\-ри\-че\-ско\-го моделирования масса 
с~по\-мощью соответствующих интерфейсных средств может быть загружена как 
$M_{\mathrm{черт}}$ в~весовую модель. (Раньше документация была представлена в~виде 
чертежей на бумажных носителях и~$M_{\mathrm{черт}}$ вручную вычислялась по этим 
чертежам.) Однако на более ранних этапах проектирования, когда еще не 
проработана гео\-мет\-рия многих элементов конструкции, весовые расчеты 
проводятся на основании эскизов и~наборов гео\-мет\-ри\-че\-ских и~конструктивных 
параметров агрегатов изделия. Для этого в~весовой модели должны быть 
предусмотрены средства параметрического представления гео\-мет\-рии 
конструкций. Геометрическое пред\-став\-ле\-ние конструкций 
в~автоматизированной системе весового проектирования выполняет 
и~немаловажную функцию визуализации конструкций, их компоновки, 
размещения нагрузки и~т.\,д. В~АСВП реализовано несколько форм 
представления гео\-мет\-ри\-че\-ской информации, ориентированных не только на 
расчет МИХ, но и~на визуализацию выполняемых расчетов. Это чертежи 
геометрических проекций изделия, это таб\-лич\-ное задание типовых 
геометрических конструкций, это каркасное представление трехмерных 
геометрических моделей и, наконец, задание объемных конструкций 
триангуляционной (фасеточной) поверхностью. Последний вид представления 
является наиболее перспективным для точного вычисления МИХ. В~АСВП он 
применяется для расчета тарировочных характеристик топливных баков, о~чем 
будет сказано ниже.
  
  \textbf{Классификационные признаки конструкции.} В~весовой модели 
кроме числовых параметров опре\-делен ряд классификационных признаков 
конструкций, по которым проводится весовой анализ.\linebreak
 Таки\-ми маркерами могут 
быть подразделения, ответст\-вен\-ные за разработку конструкции, поставщики 
или изготовители готовых изделий, принадлежность конструкции 
к~определенным функциональным системам, конструкционные материалы 
и~пр.
  
  \textbf{Функциональные подсистемы изделия.} Конст\-рук\-тив\-ное членение 
самолета может не совпадать с~его функциональной структурой. Отдельные\linebreak 
элементы функциональных подсистем самолета удобнее описывать в~составе 
конструкции ка\-ко\-го-ни\-будь агрегата планера. Например, некоторая деталь 
может конструктивно входить в~состав крыла, а принадлежать 
к~функциональной подсистеме гидравлики или электрооборудования. Для того 
чтобы иметь возможность выполнять весовые расчеты, учитывая разные 
подходы к~классификации конструкции самолета, в~АСВП отдельно от дерева 
конструкции ведется реестр подсистем, для которых может быть проведен 
специальный расчет весовых параметров. В~этом реестре ведется полный 
перечень конструкций весовой модели, входящих в~подсистемы реестра, 
независимо от того, в~какой ветви дерева конструкции они находятся. Любая 
конструкция может быть включена только в~одну из подсистем реестра. 
В~зависимости от режима расчетов МИХ
конструкций, входящих в~под\-сис\-те\-му, будут учтены либо в~со\-ста\-ве 
вышестоящих агрегатов дерева конструкции, либо отдельно в~под\-сис\-теме. 
{\looseness=1

}
  
  \textbf{Распределенные характеристики изделия.} Задача вычисления 
распределенных характеристик изделия является родственной задачей 
вычисления характеристик геометрии масс. Основное отличие состоит в~том, 
что в~данной задаче рассчитываются не интегральные характеристики 
распределения материала, а сама функция распределения массы по объему 
конструкции. Такие функции рассчитываются по заданному геометрическому 
разбиению конструкции на пространственные отсеки. Функции распределения 
массы по объему конструкции в~процессе проектирования используются при 
построении динамически подобных моделей для проведения некоторых видов 
испытаний и~продувок, а~также для выполнения прочностных расчетов. 
  
  Каждый отсек разбиения для расчета распределенных характеристик 
представляет собой вы\-пук\-лый многогранник, ограниченный конечным набором 
плоскостей. Задача построения распределенных весовых характеристик состоит 
в~вычислении для каждого отсека массы и~положения центра тяжести той части 
конструкции самолета, которая геометрически расположена внутри этого 
отсека. Эта задача решается путем нахождения геометрического пересечения 
каждой терминальной конструкции с~каждым отсеком разбиения, и~если такое 
пересечение не пусто, то вычисление массы и~центра тяжести той части 
конструкции, которая попадает в~отсек. Некоторые конструкции могут быть 
объявлены сосредоточенными массами. Использование сосредоточенных масс 
позволяет исключить все подчиненные конструкции из распределения по 
отсекам и~рассматривать их отдельно для задания сосредоточенных нагрузок. 
Список сосредоточенных масс с~уникальными именами представляет собой 
отдельную структуру весовой модели. Каждая сосредоточенная масса содержит 
список ссылок на конструкции весовой модели. Любая конструкция может 
быть включена не более чем в~одну сосредоточенную массу.
  
  \textbf{Весовые сводки.} Одной из основных задач \mbox{АСВП} является 
построение так называемых весовых сводок. Весовые сводки являются 
документами, сопровождающими построение весовой модели самолета 
в~процессе его создания. В АСВП реализовано несколько форм весовых 
сводок, которые с~разных сторон отражают дерево конструкции самолета или 
отдельных ветвей этого дерева. Назначение этих сводок и~форма представления 
зависят от ре\-ша\-емых задач. Весовые данные в~сводках могут быть 
представлены либо в~табличном виде, либо в~виде иерархии конструкций. 
Могут содержать информацию в~детализированном или в~укрупненном виде. 
Отдельные виды весовых сводок пред\-став\-ля\-ют распределенные 
характеристики по отсекам. Весовые сводки предназначены для решения задач 
весового контроля и~весового анализа. 
  
  Весовой контроль при проектировании самолетов представляет собой  
ор\-га\-ни\-за\-ци\-он\-но-тех\-ническую сис\-те\-му, нацеленную на создание 
конструк\-ции минимального веса. Для эффективного \mbox{весового} контроля 
необходима оперативная информация о текущей массе изделия и~любой его 
части. Весовая информация для весового контроля в~АСВП представляется 
в~виде оперативных весовых сводок по отдельным подразделениям 
предприятия. В~таких весовых сводках содержится информация о текущей, 
теоретической, лимитной,\linebreak чертежной и~фактической массах конструкций, 
разрабатываемых данным подразделением. Могут также выпускаться 
оперативные сводки по группе подразделений или по всему проекту. Сводки 
весового контроля предназначены для использования руководителями проекта.
  
  Весовой анализ также связан с~выпуском определенного вида весовых 
сводок. Для решения задач весового анализа в~АСВП осуществляется 
сортировка и~выборки конструкций по определенному классификационному 
признаку. Например, могут быть рассчитаны массы силового и~несилового 
набора конструкции, массы продольного и~поперечного набора, массы 
конструкций определенного материала, массы готовых изделий или изделий 
конкретного поставщика и~т.\,д. Весовой контроль и~анализ позволяют 
выявить резервы конструкции, узкие места, тренды в~изменении веса 
кон\-ст\-рук\-ции.
{\looseness=1

}
  
  \subsection{Постоянные и~переменные структуры весовой модели 
самолета}
  
  Дерево конструкции весовой модели готового изделия не является 
статической структурой. Конфигурация самолета зависит от конкретных 
условий его применения. Мас\-со\-во-инер\-ци\-он\-ные характеристики при 
взлете и~посадке отличаются от тех же характеристик в~полете, когда убраны 
стойки шасси. Конфигурация меняется и~в~полете у~самолетов с~изменяемым 
углом стреловидности или с~измененяемым вектором тяги. Текущая 
конфигурация является одним из параметров весовой модели и~параметров 
весовых расчетов. По самому смыс\-лу создания самолета как транспортного 
средства предполагается, что кроме собственно конструкции, которая 
обеспечивает выполнение основных задач, на его  
МИХ существенным образом влияет 
перевозимая нагрузка. Перевозимая нагрузка есть переменная часть структуры 
дерева конструкции. Самолетные весовые классификаторы выделяют 
постоянную часть массы изделия и~переменную, состоящую из снаряжения, 
топлива и~целевой нагрузки:
  \begin{multline*}
{M} = M_{\mathrm{пустого}} + 
M_{\mathrm{снаряжения}} + {}\\
{}+M_{\mathrm{топлива}} + 
M_{\mathrm{целевой\_нагрузки}}\,.
  \end{multline*}
  
  Все переменные и~постоянные компоненты самолета составляют единое 
целое, и~расчет мас\-со\-во-инер\-ци\-он\-ных и~центровочных характеристик 
допусти\-мых конфигураций является одной из главных задач проектирования 
самолетов любого назначения. Переменные структуры в~весовой модели могут 
задаваться альтернативными конструкциями, когда у некоторой вершины 
дерева есть несколько вариантов поддеревьев и~когда любой из вариантов, но 
только один из них, может быть активирован в~конкретный момент времени. 
Существует своя специфика задания переменных структур весовой модели для 
разных содержательных задач. 
  
  \textbf{Пустой самолет}~--- это постоянная часть конструкции самолета, 
которая не меняется в~процессе эксплуатации готового изделия. Компонентами 
пустого самолета являются конструкция планера самолета, силовая установка 
и~ее системы, другие самолетные системы, обеспечивающие управление 
самолетом, а~также специальные системы бортового оборудования, 
предназначенные для решения целевых задач самолета. В~процессе 
проектирования и~при эксплуатации самолетов рассматриваются различные 
варианты отдельных конструкций планера, а~чаще~--- различные варианты 
по\-став\-ля\-емых готовых изделий. В~связи с~этим в~весовой модели АСВП 
рассматриваются возможные комбинации вариантов пустого самолета, 
вариантов снаряжения и~полезной нагрузки. 

\begin{figure*} %fig3
\vspace*{1pt}
 \begin{center}
 \mbox{%
 \epsfxsize=162mm 
 \epsfbox{fle-3.eps}
 }
 \end{center}
\vspace*{-9pt}
\Caption{Тарировочная таблица топливного бака}
\end{figure*}
  
  \textbf{Снаряжение самолета} устанавливается на борту в~процессе 
предполетной подготовки. Снаряжение самолета принято разделять на 
основное и~дополнительное. Основное снаряжение включает несколько 
позиций. Это экипаж и~системы жизнеобеспечения экипажа, системы 
жизнеобеспечения пассажиров, заправляемые компоненты и~расходуемые 
материалы, несливаемый остаток топлива и~другие возможные компоненты. 
Использование различных вариантов экипажа и~другого снаряжения самолета 
связано с~различным характером выполняемых задач. Как правило, существует 
несколько типовых вариантов комплектации экипажа 
и~элементов снаряжения. Весовая модель должна содержать перечень 
альтернативных вариантов снаряжения и~их характеристик. Естественно, что 
этот перечень может модифицироваться. К~дополнительному снаряжению 
относят временное оборудование и~средства, связанные с~установкой на борту 
и~закреплением на подвесках перевозимых грузов. Временно устанавливаемое 
оборудование, как правило, связано со спецификой полетных заданий. Это 
может быть специальная измерительная аппаратура или оборудование, которое 
необходимо проверить в~условиях реального полета. Перечень такого 
оборудования и~его характеристики в~весовой модели должны быть 
пред\-став\-ле\-ны в~специальном реестре, или в~каталоге. Для установки 
оборудования, размещения любой коммерческой нагрузки и~вооружения в~конструкции самолета
должны быть  предусмотрены специальные места 
размещения и~узлы крепления. Точки размещения оборудования и~любых 
элементов целевой нагрузки задаются своими координатами и~установочными 
углами закрепления. 

\begin{figure*} %fig4
  \vspace*{1pt}
 \begin{center}
 \mbox{%
 \epsfxsize=162mm 
 \epsfbox{fle-4.eps}
 }
 \end{center}
\vspace*{-11pt}
\Caption{Варианты размещения целевой нагрузки самолета на подвесках}
\end{figure*}
  
  \textbf{Топливо}~--- величина переменная и~на земле, при подготовке 
самолета к~вылету, и~в~воздухе, при выработке топлива, и, если это 
предусмотрено, при дозаправке в~воздухе. Одной из самых сложных и~важных 
задач построения весовой модели является отражение изменяющихся в~полете  
МИХ топлива, находящегося 
в~топливных баках. Топливные баки современных ЛА
могут иметь довольно сложные геометрические формы. В~процессе выработки 
топлива все характеристики располагаемого запаса топлива меняются. 
Необходимо отслеживать эти изменения в~процессе произвольных допустимых 
эволюций траектории полета. Функции изменения МИХ в~зависимости от 
объема оставшегося топлива задаются тарировочными характеристиками баков. 
Для расчета тарировочных характеристик топливных баков при произвольных 
углах атаки, углах тангажа и~крена в~весовой модели наиболее удобно 
триангуляционное задание баков. В~тарировочной таблице вычисляется масса 
оставшегося топлива в~зависимости от уровня поверхности жидкости 
в~топливном баке. На рис.~3 приведен пример расчета тарировочной таблицы 
крыльевого топливного бака.



  Если МИХ топлива в~конкретном баке по 
мере его выработки определяются тарировочной характеристикой, то 
зависимость МИХ оставшегося топлива определяется последовательностью, 
в~которой осуществляется выработка из разных баков. Топливная система 
самолета состоит из нескольких баков~--- как внутренних, так и~размещенных 
на подвесках, а~также из системы трубопроводов, перекачивающих насосов и~управляющей автоматики. Основой управления расходом топлива является 
программа, определяющая порядок расходования топлива из разных баков. 
Переключение перекачки топлива между разными баками осуществляется для 
обеспечения центровки самолета в~заданных границах. Одним из критериев при 
разработке алгоритмов перекачки является число переключений и~обеспечение 
бесперебойной подачи топлива при любых допустимых параметрах траектории 
полета. Массово-инерционные характеристики топлива в~процессе тарировки 
баков задаются их разбиением плоскопараллельными сечениями на тонкие 
слои. Для каждого слоя указывается масса, координаты центра тяжести 
и~плоскостные моменты инерции. Программа выработки топлива пред\-став\-ля\-ет 
собой последовательность выработки слоев из разных баков в~соответствии 
с~диаграммой переключений. В~весовой модели может быть задано несколько 
вариантов программ расходования топлива. Разумеется, в~процессе выполнения 
полетного задания программа расходования топлива фиксирована. 
Предварительный расчет характеристик для разных вариантов порядка 
выработки топлива необходим для выбора наилучшего, удовле\-тво\-ря\-юще\-го 
всем ограничениям.
  
  \textbf{Целевая нагрузка} зависит от назначения самолета и~от конкретного 
полетного задания. Для пасса\-жирских самолетов целевая нагрузка~--- это 
пассажи\-ры с~багажом, для транспортных са\-мо\-летов~--- это коммерческие 
грузы, для военных~--- подвесное или размещаемое в~специальных \mbox{отсеках} 
вооружение. В~полете возможен сброс и~десантирование целевой нагрузки. 
Комплектация и~установка целевой нагрузки представляет собой довольно 
сложный процесс. Выбор состава грузов и~их размещение могут проходить 
в~несколько этапов. Сложность выбора обусловлена большим количеством 
типов перевозимой нагрузки, наличием большого числа вспомогательных 
специальных устройств закрепления грузов как во внутренних отсеках 
самолета, так и~на внешних подвесках. На рис.~4 приведена панель 
формирования различных расчетных вариантов целевой нагрузки самолета. 
Визуализация этого процесса существенно облегчает решение различных задач 
анализа допустимой нагрузки как на этапе проектирования самолета, так и~при 
эксплуатации во время подготовки полетных заданий.
  
  \begin{figure*} %fig5
\vspace*{1pt}
 \begin{center}
 \mbox{%
 \epsfxsize=162mm 
 \epsfbox{fle-5.eps}
 }
 \end{center}
\vspace*{-9pt}
\Caption{Область допустимых центровок}
\end{figure*}

  Для удобства выбора и~проведения расчетов множества вариантов загрузки 
самолета в~рамках весовой модели реализованы каталоги нагрузки~--- 
специального оборудования, коммерческой нагрузки, вооружения. В~этих 
каталогах ведутся клас\-си\-фи\-ка\-то\-ры, позволяющие в~громадных переч\-нях 
находить нужные позиции и~их характеристики. Кроме  
МИХ размещаемой нагрузки в~каталогах 
даются ссылки на их геометрические модели, задаются габариты, другие 
геометрические па\-ра\-мет\-ры. Эти данные нужны для визуализации размещения 
и~компоновки, для вычисления их МИХ. 
Как правило, существуют довольно жесткие ограничения на 
размещение нагрузки на борту, а~также на внешних узлах крепления. Эти 
ограничения должны указываться в~каталоге и~учитываться в~процессе 
формирования вариантов загрузки самолета. 
  
  Ограничения, которые проверяются при анализе различных вариантов 
снаряжения самолета, программы выработки топлива и~допустимых вариантов 
целевой нагрузки, задают область допустимых центровок самолета. 
  
  \textbf{Область допустимых центровок} является неотъемлемой частью 
весовой модели и~служит одной из основных весовых характеристик самолета, 
особенно важной и~контролируемой в~процессе его эксплуатации. На рис.~5 
проиллюстрированы ограничения, образующие область допустимых центровок, 
и~приведен график изменения центровки самолета при выработке топлива. 



  По оси абсцисс на этом графике откладывается центровка самолета, которая 
определяется как положение центра тяжести самолета на средней 
аэро\-ди\-на\-ми\-че\-ской хорде в~процентах от ее длины. По оси ординат 
откладывается текущая масса самолета с~учетом массы снаряжения, массы 
целевой нагрузки и~текущего запаса топлива. Точки излома на графиках 
центровки соответствуют моментам переключения подачи топлива с~одного 
бака на\linebreak другой, которые определяются программой выработки топлива или 
моментами сброса целевой нагрузки. Двойной график изменения центровки 
соответствует двум полетным конфигурациям~--- с~убранными 
и~выпущенными стойками шасси. Ограничения, которые обеспечивают 
устой\-чи\-вость и~управ\-ля\-емость полета, задаются предельными значениями 
центровки. Предельно передняя и~предельно задняя центровки на графике 
показаны вертикаль\-ными штриховыми линиями. Горизонтальные линии задают 
ограничения на взлетную и~посадочную массы. Ограничения максимальной 
взлетной и~посадочной массы при определенных условиях могут нарушаться, 
но эти нарушения допускаются в~исключительных условиях и~сказываются на 
ресурсных характеристиках самолета.\linebreak Превышение \textbf{предельных} 
значений взлетной и~посадочной массы не допускается. Наклонные штриховые 
линии на графике задают ограничения, связанные с~максимально допустимыми 
нагрузками на переднюю и~главную опоры шасси.  

\begin{figure*} %fig6
\vspace*{1pt}
 \begin{center}
 \mbox{%
 \epsfxsize=165mm 
 \epsfbox{fle-6.eps}
 }
 \end{center}
\vspace*{-9pt}
\Caption{Архитектура программной реализации исполнительных модулей АСВП}
\end{figure*} 

%\vspace*{-12pt}

\section{Программная реализация автоматизированной системы весового
проектирования}

  Представленная здесь струк\-тур\-но-па\-ра\-мет\-ри\-че\-ская весовая модель 
самолета позволяет решать широкий круг задач весового проектирования. 
Весовая модель составляет информационную основу,\linebreak на базе которой могут 
быть построены различные вычислительные программы и~пользовательские 
модули. Рассматриваемая в~данной работе АСВП построена по 
кли\-ент-сер\-вер\-ной архитектуре, где весовая модель служит единым хранилищем 
информации, базой данных системы. Непосредственно с~информацией, 
хранящейся в~этой базе данных, взаимодействуют различные вычислительные, 
расчетные программы~--- серверы, которые кроме расчетных функций 
обеспечивают информационную связь клиентских модулей с~весовой моделью 
самолета. Непосредственными пользователями клиентских модулей являются 
конструкторы и~проектировщики, решающие различные задачи весового 
проектирования.  Построена АСВП как многопользовательская интерактивная 
система. На рис.~6 представлена архитектура АСВП, ее основные программные 
и~информационные компоненты.




  Ниже перечислены основные функции программных модулей АСВП:
 \begin{description}
 \item[\,] 
Сервер ПУСТОЙ ЛА\;+\;Модуль расчета МИХ пус\-то\-го самолета:
\begin{itemize}
\item создание и~модификация дерева конструкции пустого самолета;
\item расчет МИХ пустого изделия, всех его сис\-тем, узлов, агрегатов и~деталей 
на любых уровнях дерева конструкции;
\item весовой анализ и~контроль текущего состояния проекта, выполнения 
лимитных ограничений по весу, осуществление выборок весовой информации 
по различным признакам~--- сис\-те\-мам, агрегатам, типу конструкции 
(си\-ло\-вая/не\-си\-ло\-вая),  материалу конструкции, подразделениям и~т.\,д.;
\item расчет распределения массы самолета по различным разбиениям на 
отсеки; эта информация используется для построения динамически подобных 
моделей и~при прочностных расчетах;
\item расчет МИХ при различных вариантах полетной конфигурации при 
убранных и~выпущенных стойках шасси, при отклонениях консолей крыла для 
самолетов с~из\-ме\-ня\-емой геометрией, при отклонении органов управления.
\end{itemize}
\begin{figure*} %fig7
\vspace*{1pt}
 \begin{center}
 \mbox{%
 \epsfxsize=155.86mm 
 \epsfbox{fle-7.eps}
 }
 \end{center}
\vspace*{-1pt}
\Caption{Проектный подход~--- технология ГП}
\vspace*{6pt}
\end{figure*}
 \item[\,]
Сервер НАГРУЗКА ЛА\;+\;Модуль расчета МИХ самолета с~переменной 
массой:
\begin{itemize}
\item создание и~модификация реестра допустимых вариантов нагрузки 
самолета;
\item расчеты МИХ снаряженного и~загруженного самолета для разных 
вариантов компоновки и~размещения на борту полезной нагрузки;
\item расчет изменения МИХ самолета в~полете при выработке топлива, 
дозаправке в~воздухе, сбросе нагрузки;
\item расчет МИХ самолета в~виде табличных зависимостей для различных 
вариантов снаряжения и~размещения нагрузки;
\item расчет МИХ самолета в~виде графических зависимостей от массы 
самолета и/или от массы топлива;
\item проверка выполнения установленных эксплуатационных ограничений по 
центровке, взлетной и~посадочной массе, нагрузке на опоры шасси для 
различных вариантов снаряжения и~размещения нагрузки; сигнализация 
в~случае нарушения ограничений, а~также для различных вариантов программ 
выработки топлива.
\end{itemize}

\pagebreak

 \item[\,]
Сервер КАТАЛОГ\;+\;Модуль ведения каталога элементов нагрузки:\\[-9pt]
\begin{itemize}
\item создание и~модификация каталога элементов целевой нагрузки самолета;\\[-9pt]
\item создание и~модификация базы данных вариантов размещения 
и~закрепления элементов нагрузки каталога на борту самолета или на подвесках;\\[-9pt]
\item создание и~модификация базы данных вспомогательных элементов 
конструкции установки элементов нагрузки.\\[-9pt]
\end{itemize}
 \item[\,]
Сервер ТОПЛИВО\;+\;Модуль расчета порядка выработки топлива:\\[-9pt]
\begin{itemize}
\item создание и~модификация базы данных различных вариантов программы 
выработки топлива;\\[-9pt]
\item расчет МИХ и~МЦД для различных вариантов переключения выработки 
топлива из внутренних, закладных и~подвесных баков;\\[-9pt]
\item расчет МИХ и~МЦД при различных программах заливки и~дозаправки 
топлива во внутренние, закладные и~подвесные баки.\\[-9pt]
\end{itemize}
 \item[\,]
Сервер БАКИ\;+\;Модуль расчета тарировки топливных баков:\\[-9pt]
\begin{itemize}
\item создание и~модификация базы данных гео\-мет\-рии топливных баков;\\[-9pt]
\item расчет тарировочных характеристик топливных баков при различных 
углах тангажа и~крена.\\[-9pt]
  \end{itemize}
  \end{description}
  
  Программная реализация АСВП велась с~использованием инструментального комплекса 
<<Генератор проектов>> (технология ГП)~\cite{5-fl}. Технология ГП 
обеспечивает возможность разработки приклад\-ных систем многоуровневой  
кли\-ент-сер\-вер\-ной архитектуры с~использованием реляционных и~сетевых 
баз данных со сложным пользовательским и~межпрограммным интерфейсом. 
Создание ин\-фор\-ма\-ци\-он\-но-вы\-чис\-ли\-тель\-ных сис\-тем в~рамках 
технологии ГП базируется на так называемом <<проектном подходе>>. Под 
проектом здесь понимается пакет документов (файлов), содержащий описание 
структуры проекта, описание логической структуры баз данных, спецификации 
пользовательского интерфейса, перечень команд и~сценарии работы 
пользователей, описание функций и~процедур обработки пользовательских 
запросов. Исходное описание проекта подается на вход <<Генератора 
проекта>>, который строит в~памяти модель проекта, осуществляет ее анализ 
на предмет корректности и~целостности, а затем на основании этой модели 
генерирует тексты программного кода для клиентских и~серверных компонент 
системы, а~так\-же ге\-нерирует утилиты, необходимые для сборки, инсталляции 
и~сопровождения системы. 

На рис.~7 показана общая архитектура 
программной конструкции, связанной с~применением технологии ГП.
  


  В приведенной цепочке разработчик прикладной информационной системы 
имеет дело только с~первым ее звеном~--- проектом системы. При этом он 
избавлен от необходимости иметь дело с~системным программным окружением 
вычислительной среды, в~которой должна функционировать разрабатываемая 
прикладная система. Все связи прикладных информационных процессов 
с~конкретной системной вычислительной средой привносит 
в~результирующую рабочую программу <<Генератор проектов>> на стадии 
анализа и~генерации итогового программного кода. Естественно, что при этом 
объем описания проекта оказывается существенно короче программного кода, 
который создается автоматически. Экономия трудозатрат разработчика 
оказывается существенной. В~частности, объем описания проекта АСВП на 
порядок меньше, чем объем сгенерированного программного кода. Даже если 
предположить, что написанный вручную программный код благодаря 
искусству программистов будет весьма экономным, то все равно трудоемкость 
разработки прикладных систем будет в~разы меньше. 

Но главное даже не 
в~числе строк программ, а~прежде всего в~экономии интеллектуальных затрат 
разработчиков прикладных систем и,~в~итоге, автоматически созданные 
программы более надежны и~свободны от нечаянных ошибок и~опечаток.\linebreak 
И~кроме того, разрабатываемые в~рамках технологии ГП прикладные системы 
обеспечивают-\linebreak\vspace*{-12pt}

\pagebreak

\noindent
ся эффективными средствами сопровождения, т.\,е.\linebreak достаточно 
простой процедурой внесения ис\-прав\-ле\-ний и~развития программ в~процессе их 
эксплу\-а\-тации. 

Прикладные программные комплексы в~рамках технологии ГП 
разрабатываются как автономные системы и~не требуют для своей работы 
специальной среды и~дорогостоящих программных продуктов (кроме 
использующихся систем управления базами данных
(СУБД) и~общесистемного обеспечения). Разрабатываемые 
в~рамках технологии ГП прикладные системы допускают масштабирование 
и~портирование на различные вычислительные платформы и~СУБД.
  
  \bigskip
  
  Как уже говорилось, система АСВП разрабатывалась в~течение ряда лет, 
многие ее компоненты и~версии были апробированы и~использовались 
в~реальном проектировании. 
%
Авторы выражают благодарность 
С.\,И.~Скобелеву, М.\,К.~Курьянскому, Д.\,Ю.~Стрель\-цу, П.\,В.~Плунскому 
и~К.\,Н.~Ерасову за плодотворные обсуждения проблем весового проектирования 
самолетов, за постановку многих задач и~за апробацию разработанных 
программ.

%\vspace*{-12pt}

{\small\frenchspacing
 {%\baselineskip=10.8pt
 \addcontentsline{toc}{section}{References}
 \begin{thebibliography}{9}
\bibitem{1-fl}
\Au{Шейнин В.\,М., Козловский~В.\,И.} Весовое проектирование и~эффективность 
пассажирских самолетов.~--- М.: Машиностроение, 1977.   Т.~1. 343~с.

%\columnbreak

\bibitem{2-fl}
\Au{Скобелев С.\,И., Широков~Н.\,И.} Весовой анализ и~контроль в~САПР ЛА~// Задачи 
и~методы автоматизированного проектирования.~--- М.: ВЦ РАН, 1991. С.~92--100.
\bibitem{3-fl}
\Au{Широков Н.\,И.} Автоматизированная система весовых расчетов в~САПР ЛА~// 
Автоматизация проектирования инженерных и~финансовых информационных систем 
средствами Генератора проектов~/ Отв. ред. Ю.\,А.~Флеров.~--- М.: ВЦ РАН, 
2010. С.~55--66.

\vspace*{6pt}

\bibitem{4-fl}
\Au{Вышинский Л.\,Л., Широков~Н.\,И.} Система автоматизации расчетов 
массово-инерционных характеристик ЛА с~переменной массой~// Развитие и~применение 
инструментального комплекса Генератор проектов~/ Отв. ред. Ю.\,А.~Флеров.~--- 
М.: ВЦ РАН, 2014. С.~20--31.
{\looseness=1

}

\vspace*{6pt}

\bibitem{5-fl}
\Au{Вышинский Л.\,Л., Гринев~И.\,Л., Флеров~Ю.\,А., Широков~А.\,Н., Широков~Н.\,И.} 
Генератор проектов~--- инструментальный комплекс для разработки  
<<кли\-ент-сер\-вер\-ных>> сис\-тем~// Информационные технологии и~вычислительные 
системы, 2003. №\,1-2. С.~6--25.
 \end{thebibliography}

 }
 }

\end{multicols}

\vspace*{-6pt}

\hfill{\small\textit{Поступила в~редакцию 24.05.17}}

\vspace*{8pt}

%\newpage

%\vspace*{-24pt}

\hrule

\vspace*{2pt}

\hrule

%\vspace*{8pt}


\def\tit{COMPUTER-AIDED SYSTEM OF~AIRCRAFT WEIGHT DESIGN}

\def\titkol{Computer-aided system of~aircraft weight design}

\def\aut{L.\,L.~Vyshinsky, Yu.\,A.~Flerov, and~N.\,I.~Shirokov}

\def\autkol{L.\,L.~Vyshinsky, Yu.\,A.~Flerov, and~N.\,I.~Shirokov}

\titel{\tit}{\aut}{\autkol}{\titkol}

\vspace*{-9pt}


\noindent
A.\,A.~Dorodnicyn Computing Centre, Federal Research Center ``Computer Science and 
Control'' of the Russian Academy of Sciences,  40~Vavilov Str., Moscow 119333, Russian 
Federation 



\def\leftfootline{\small{\textbf{\thepage}
\hfill INFORMATIKA I EE PRIMENENIYA~--- INFORMATICS AND
APPLICATIONS\ \ \ 2018\ \ \ volume~12\ \ \ issue\ 1}
}%
 \def\rightfootline{\small{INFORMATIKA I EE PRIMENENIYA~---
INFORMATICS AND APPLICATIONS\ \ \ 2018\ \ \ volume~12\ \ \ issue\ 1
\hfill \textbf{\thepage}}}

\vspace*{3pt}
   

\Abste{The article is devoted to the problems of computer-aided weight design of 
aircraft. Weight and mass-inertial parameters are one of the basic values that affect 
the performance characteristics of aircraft. The informational basis of the system is 
the weight model of the aircraft. The paper describes the structure of the weight 
model and its individual components. The program implementation of the system, 
which is executed within the framework of the client-server architecture, is shown. 
The automated system of weight design is implemented using the software tool 
complex ``Project Generator'' (GP technology), which was developed at the 
Computing Centre of the Russian Academy of Sciences. The creation of information 
and computing systems within the framework of the GP technology is based on the 
so-called ``project approach,'' when the formal description of the system 
automatically generates code for the client and server components of the system.}

\KWE{math modeling; design automation; aircraft; weight design; weighting model; 
design tree; project generator; code generation; client-server architecture}

  \DOI{10.14357/19922264180103} 

%\vspace*{-12pt}

%\Ack
%\noindent




%\vspace*{8pt}

  \begin{multicols}{2}

\renewcommand{\bibname}{\protect\rmfamily References}
%\renewcommand{\bibname}{\large\protect\rm References}

{\small\frenchspacing
 {%\baselineskip=10.8pt
 \addcontentsline{toc}{section}{References}
 \begin{thebibliography}{9} 
 
 %\vspace*{-6pt}
 
 \bibitem{1-fl-1}
\Au{Sheynin, V.\,M., and V.\,I.~Kozlovskiy}. 1977. \textit{Vesovoe 
proektirovanie i~effektivnost' passazhirskikh samoletov} [Weight design and 
efficiency of passenger aircraft]. Moscow: Mechanical Engineering. Vol.~1. 343~p.
\bibitem{2-fl-1}
\Aue{Skobelev, S.\,I., and N.\,I.~Shirokov.} 1991. Vesovoy analiz i~kontrol' v~SAPR 
LA [Weight analysis and control in CAD of aircraft]. \textit{Zadachi i~metody 
avtomatizirovannogo proektirovaniya} [Tasks and methods of computer-aided 
design]. Moscow: Computing Centre of the USSR Academy of Sciences.  
92--100.
\bibitem{3-fl-1}
\Aue{Shirokov, N.\,I.} 2010. Avtomatizirovannaya sistema vesovykh raschetov 
v~SAPR LA [Automated system weight calculations in CAD].  
\textit{Avtomatizatsiya proektirovaniya inzhenernykh i~finansovykh 
informatsionnykh system sredsvami Generatora proektov} [Computer 
aided  design of engineering and financial information systems by the means of the 
Project Generator]. Moscow: Computing Centre of RAS. 
55--66.
\bibitem{4-fl-1}
\Aue{Vyshinskiy, L.\,L., and N.\,I.~Shirokov.} 2014. Sistema avtomatizatsii 
raschetov massovo-inertsionnykh kharakteristik LA s~peremennoy massoy [CAD 
system of calculation  aircraft mass-inertial characteristics with variable mass].  
\textit{Razvitie i~primenenie instrumental'nogo kompleksa Generator proektov} 
[The development and application of a tool set Project Generator]. 
Moscow: Computing Centre of RAS. 20--31.
{\looseness=1

}

\bibitem{5-fl-1}
\Aue{Vyshinskiy, L.\,L., I.\,L.~Grinev, Yu.\,A.~Flerov, A.\,N.~Shirokov, and 
N.\,I.~Shirokov.} 2003. Generator proektov~--- instrumental'nyy kompleks dlya 
razrabotki ``klient--servernykh'' sistem [The project generator~--- tool complex for 
development of ``client--server'' systems]. 
\textit{Informatsionnye tekhnologii i~vychislitel'nye sistemy} [Information 
Technologies and Computer Systems] 1-2:6--25.

\end{thebibliography}

 }
 }

\end{multicols}

\vspace*{-6pt}

\hfill{\small\textit{Received May 24, 2017}}

%\vspace*{-10pt}

\Contr

\noindent
\textbf{Vyshinsky Leonid L.} (b.\ 1941)~--- Candidate of Sciences (PhD) in physics and 
mathematics, Head of Laboratory, A.\,A.~Dorodnicyn Computing 
Centre, Federal Research Center ``Computer Science and Control'' of the Russian 
Academy of Sciences, 40~Vavilov Str., Moscow 119333, Russian Federation; 
\mbox{Wysh@ccas.ru} 

\vspace*{3pt}

\noindent
\textbf{Flerov Yuri A.} (b.\ 1942)~--- Corresponding Member of the Russian 
Academy of Science, Doctor of Science in physics and mathematics, professor, 
Deputy Director, A.\,A.~Dorodnicyn Computing Centre, Federal Research Center 
``Computer Science and Control'' of the Russian Academy of Sciences, 40~Vavilov 
Str., Moscow 119333, Russian Federation; \mbox{fler@ccas.ru}

\vspace*{3pt}

\noindent
\textbf{Shirokov Nikolai I.} (b.\ 1963)~--- Candidate of Sciences (PhD) in physics and 
mathematics, senior scientist, A.\,A.~Dorodnicyn Computing Centre, Federal 
Research Center ``Computer Science and Control'' of the Russian Academy of 
Sciences, 40~Vavilov Str., Moscow 119333, Russian Federation; 
\mbox{Wysh@ccas.ru} 



\label{end\stat}


\renewcommand{\bibname}{\protect\rm Литература}        %7
\def\stat{agalarov}


\def\tit{ПРИБЛИЖЕННЫЙ МЕТОД ВЫЧИСЛЕНИЯ ХАРАКТЕРИСТИК УЗЛА 
ТЕЛЕКОММУНИКАЦИОННОЙ СЕТИ С~ПОВТОРНЫМИ ПЕРЕДАЧАМИ}
\def\titkol{Приближенный метод вычисления характеристик узла 
телекоммуникационной сети с~повторными передачами} 

\def\autkol{Я.\,М.~Агаларов}
\def\aut{Я.\,М.~Агаларов$^1$}

\titel{\tit}{\aut}{\autkol}{\titkol}

%{\renewcommand{\thefootnote}{\fnsymbol{footnote}}\footnotetext[1]
%{Работа выполнена при поддержке РФФИ, проекты 08--07--00152 и 08--01--00567.}}

\renewcommand{\thefootnote}{\arabic{footnote}}
\footnotetext[1]{Институт проблем
информатики Российской академии наук, agglar@yandex.ru}

%\vspace*{-6pt}


\Abst{Рассмотрена модель узла коммутации пакетов c повторными передачами для двух 
схем распределения буферной памяти: полнодоступной и полного разделения. Предложен 
приближенный метод вычисления интенсивностей потоков и вероятностей блокировок узла. 
Получены необходимые и достаточные условия существования и единственности решения 
уравнения для потоков в узле при установившемся режиме работы и доказана сходимость 
итерационного метода решения указанного уравнения.}

\KW{узел коммутации пакетов; буферная память; повторные передачи; вероятности 
блокировок; итерационный метод}

      \vskip 18pt plus 9pt minus 6pt

      \thispagestyle{headings}

      \begin{multicols}{2}

      \label{st\stat}


\section{Введение}

    Одной из основных задач предварительного анализа 
телекоммуникационных сетей коммутации пакетов с ограниченной буферной 
памятью является расчет характеристик потоков и вероятностей блокировок в 
узлах связи. Важность указанных характеристик определяется тем, что от их 
значений существенным образом зависят другие основные показатели сети 
(пропускная способность, задержки пакетов и~др.). 

    Существует множество различных моделей узлов коммутации пакетов и 
методов их расчета (см., например,~[1--6]). Для моделей, рассматривающих 
узел с ограниченной буферной памятью как систему массового обслуживания 
(CMO) типа 
$
\begin{matrix}
M \\ \lambda
\end{matrix}
\left |
\begin{matrix}
M \\ \lambda
\end{matrix}
\right |
\overline{m} \vert N
$ или  $\vert PH\vert PH\vert 1\vert r$, в предположении отсутствия повторных 
передач пакетов получены точные методы вычисления характеристик 
узлов~[1, 3, 4, 6]. Приближенные методы расчета узлов, учитывающие повторные 
попытки передачи, используют модели типа $\vert PH\vert PH\vert 1\vert r$ или 
$
\begin{matrix}
M \\ \lambda
\end{matrix}
\left |
\begin{matrix}
M \\ \lambda
\end{matrix}
\right |
1 \vert N
$ и являются 
итерационными~[2, 3, 5, 7]. Для моделей типа 
$BM\!AP\vert PH\vert 1$, $M\vert G\vert 1\vert r$ и $M\!AP\vert 
(PH,PH)\vert 1$ с повторными заявками получены точные методы вычисления 
характеристик (например, в работах~[8--10]), которые также могут быть 
использованы при расчете узлов.

    Ниже будут рассмотрены модели узла коммутации пакетов с повторными 
передачами для двух схем распределения буферной памяти: с 
полнодоступными буферами и с полным разделением буферной памяти. 
Предлагается приближенный метод расчета характеристик, который в качестве 
модели узла использует СМО типа $
\begin{matrix}
M \\ \lambda
\end{matrix}
\left |
\begin{matrix}
M \\ \lambda
\end{matrix}
\right |
\overline{m} \vert N
$ с повторными заявками. Доказаны утверждения о 
достаточных и необходимых условиях существования и единственности 
решения уравнения для вероятности блокировки в установившемся режиме 
работы и сходимости предлагаемого итерационного метода. 

\section{Модель узла}

    Математическая модель узла представляется в виде СМО с ограниченной 
буферной памятью и различными потоками заявок, каждая из которых требует 
обслуживания только на одной из многоканальных линий связи. 

    Пусть $0<N<\infty$~--- число мест хранения в буферной памяти, $u$~--- 
узел связи, $v$~--- линия связи, $\Omega_u^+$~--- множество исходящих из 
узла~$u$ линий, $c_v$~--- канальная емкость линии~$v$. Поток заявок, 
тре\-бу\-ющих обслуживания на линии~$v$, назовем $v$-по\-то\-ком, заявки этого 
потока~--- $v$-за\-яв\-ка\-ми.


    Пусть выполняются следующие предположения: 
\begin{enumerate}[1.]
\item Места в буферной памяти распределяются согласно одной из двух 
схем:
\begin{enumerate}[($i$)]
\item полнодоступная схема~--- каждое свободное место хранения доступно 
любой заявке;
\item схема полного разделения памяти~--- $v$-за\-яв\-кам доступны всего 
$N_v$ мест, где $\sum\limits_{v\in\Omega_u^+} N_v=N$.
\end{enumerate}
\item Если в момент поступления $v$-заявки в буферной памяти есть 
доступное свободное место, то она сразу занимает это место. Если в момент 
поступления $v$-заявки в системе нет свободного доступного места 
хранения, то поступившая заявка через некоторое время повторно поступает 
на систему, оставаясь $v$-заявкой. 
\item Интенсивности первичных потоков $v$-заявок~--- заданные величины 
$0<\Lambda_v<\infty$, $v\in \Omega_u^+$. Суммарные потоки первичных и 
повторных $v$-заявок являются независимыми в совокупности 
пуассоновскими потоками. Для обслуживания $v$-заявки требуется 
одновременно одно место хранения и один канал типа~$v$, $v\in 
\Omega_u^+$.
\item Первичные нагрузки~--- реализуемые, т.\,е.\ в данном случае 
интенсивности входных первичных потоков равны интенсивностям 
выходных потоков выполненных заявок. 
\item Принятые в СМО $v$-заявки обслуживаются линией~$v$ в порядке 
поступления. 
\item Время занятия канала $v$-заявкой~--- экспоненциально 
распределенная случайная величина с параметром $0<\mu_v<\infty$, 
$v\in\Omega_u^+$, независимая от других случайных событий в узле.
\item Выполненная $v$-заявка с вероятностью~$B_v$ повторяется через 
заданное время~$\tau_v$ (тайм-аут) и с вероятностью $1-B_v$ покидает 
систему через время~$t_v$ навсегда, сразу освободив занятый канал и место 
буферной памяти.
\end{enumerate}

   Будем говорить, что узел блокирован для $v$-за\-яв\-ки, если в буферной 
памяти отсутствует доступное место хранения. Ставится задача вычисления 
вероятностей блокировок и интенсивностей потоков в узле.

\section{Вычисление вероятности блокировки и~интенсивностей~потоков} 

   Пусть $\Lambda_v^*$~--- интенсивность суммарного потока внешних 
заявок, требующих передачи по линии~$v$, $\pi_v$~--- вероятность блокировки 
узла для заявок, требующих передачи по исходящей из узла линии~$v$. 

    Пусть в узле используется полнодоступная схема распределения 
буферной памяти. Тогда, как следует из описания модели, $\pi_v 
=\pi_{v^\prime},\,v,\,v^\prime\in \Omega_u^+$, и для 
интенсивностей~$\Lambda_v^*$, $v\in\Omega_u^+$, справедливы соотношения:
\begin{equation*}
\Lambda_v^* = \fr{\Lambda_v}{1-\pi}\,,
%\label{e1aga}
\end{equation*}
    где
    $\pi =\pi_v$, $v\in\Omega_u^+$.

    Пусть 
    $\overline{k} = \{\overline{k}_v$, $v\in\Omega_u^+\}$~--- состояние 
буферной памяти узла, $\overline{k}_v =\left ( k_v,\,k_v^\prime,\,k_v^{\prime\prime}\right )$; 
$k_v$~--- число $v$-заявок в буферной 
памяти, ожидающих выполнения линией~$v$; $k^\prime_v$~--- число 
$v$-заявок в буферной памяти, ожидающих тайм-аут и неуспешно переданных 
в последующий узел; $k_v^{\prime\prime}$~--- число $v$-за\-явок в буферной 
памяти, успешно переданных в последующий узел и ожидающих 
потверждения; 
$A_m = \left \{ \overline{k}:\ \sum\limits_{v\in\Omega_u^+} \left ( 
k_v+k_v^\prime + k_v^{\prime\prime}\right ) =m \right \}$~--- множество различных 
состояний, при которых в памяти узла занято ровно $m$~буферов. Тогда с 
учетом введенных выше обозначений и предположений для ве\-ро\-ят\-ности 
блокировки узла можно написать формулу~\cite{1aga, 2aga}:
\begin{equation}
\pi = \fr{1}{G_N}\sum\limits_{\overline{k}\in A_N} 
p\left (\overline{k},\overline{\rho}^*\right )\,,
\label{e2aga}
\end{equation}
где  
\begin{gather}
p(\overline{k},\overline{\rho}^*) = \prod\limits_{v\in\Omega_u^+} z_v (\pi, 
\rho_v , k_v , k_v^\prime , k_v^{\prime\prime})\,;\\
z_v (\pi, \rho_v , k_v , k_v^\prime , k_v^{\prime\prime}) ={}\notag\\
\!\!{}=
\begin{cases}
 \fr{\rho_v^{\prime *k_v^\prime}}{k_v^{\prime}!}\,
\fr{\rho_v^{\prime\prime * k_v^{\prime\prime}}}{ k_v^{\prime\prime}!}  \,
\fr{\rho_v^{*k_v}}{ k_{v}!} 
&\mbox{при}\ k_v<c_v\,,\\
 \fr{\rho_v^{\prime * k_v^\prime}}{k_v^{\prime}!} \,
\fr{\rho_v^{\prime\prime * k_v^{\prime\prime}}} { k_v^{\prime\prime}!} 
\fr{\rho_v^{*k_v}}{ c_{v}!c_v^{k_v- c_v}} 
& \mbox{при}\ k_v\geq c_v\,;
\end{cases}\\
G_N = \sum\limits_{m=0}^N\sum\limits_{\overline{k}\in A_m}
p(\overline{k},\overline{\rho}^*)\,;\\ 
\overline{\rho}^*=\{\rho_v^*,\,v\in\Omega_u^+\}\,;\\
\rho_v^* = \fr{\rho_v}{1-\pi}\,;\quad \rho_v =\fr{\Lambda_v}{\mu_v(1- B_v)}\,;\\
\rho_v^{\prime *} =\rho_v^*\mu_v\tau_vB_v\,;\quad \rho_v^{\prime\prime *}=
p_v^* \mu_vt_v,\,\quad  v\in \Omega_u^+\,.\label{e3aga}
\end{gather}

Переобозначив $1-\pi$ через $y$, выражение в правой части равенства~(2)~--- через 
$p_{\overline{k}}(\overline{\rho},y)$, выражение в правой части равенства~(4)~--- 
через $g_N(\overline{\rho},y)$, а выражение в правой 
части равенства~(1)~--- через $1-q_N (\overline{\rho},y)$, 
где $\overline{\rho} = (\rho_v,\,v\in \Omega_u^+)$, $\rho_v = \rho_v^*y\;=$\linebreak 
$=\;\Lambda_v/(\mu_v(1-B_v))$, $v\in\Omega_u^+$, получим нелинейное уравнение 
относительно неизвестной переменной~$y$:
\begin{equation}
y=q_N(\overline{\rho},y)\,.
\label{e4aga}
\end{equation}

    Решим уравнение~(8). Как следует из~(2)--(7), верно 
равенство
\begin{equation}
q_N(\overline{\rho},y) = \fr{g_{N-1}(\overline{\rho},y )}{g_N(\overline{\rho},y)}\,.
\label{e5aga}
\end{equation}
Введем функцию  $d_n(\overline{\rho} ,y)$ среднего числа заявок в узле с 
буферной памятью емкости $n\geq 0$:
$$
d_n(\overline{\rho} ,y) = 
\fr{1}{g_n(\overline{\rho},y)}\,\sum\limits_{m=0}^n m\sum\limits_{\overline{k}\in 
A_m} p_{\overline{k}}(\overline{\rho},y)\,.
$$
Заметим, что $g_n$, $d_n$ и $q_n$, 
$n\geq 0$,~--- непрерывно-дифференцируемые функции по $y\in (0,\,1]$. Взяв 
производную функции~$g_n$ по~$y$, из~(2)--(7) получим
\begin{multline}
\fr{\partial g_n(\overline{\rho},y)}{\partial y} ={}\\
{}= -\sum\limits_{m=0}^n m 
\sum\limits_{\overline{k}\in A_m}\fr{\prod\limits_{v\in\Omega_u^+} z_n 
(0,\rho_v, k_v, k_v^\prime , k_v^{\prime\prime})}{y^{m+1}}={}\\
{}= -\fr{1}{y}\,g_n (\overline{\rho},y)d_n(\overline{\rho},y)\,.
\label{e6aga}
\end{multline}
Взяв производную функции $q_N$ по $y$, из~(\ref{e5aga}) и~(\ref{e6aga}) 
получим
\begin{equation}
\fr{\partial q_N(\overline{\rho},y)}{\partial y} = \fr{q_N(\overline{\rho},y)}{y}\left 
[ d_N (\overline{\rho},y)-d_{N-1}(\overline{\rho},y)\right ]\,.
\label{e7aga}
\end{equation}
    Докажем несколько утверждений о свойствах 
функции~$q_N(\overline{\rho},y)$.
\medskip

\noindent
\textbf{Утверждение 1.} \textit{Справедливы неравенства}
\begin{multline}
0<d_{n+1}(\overline{\rho},y)-d_n(\overline{\rho},y) <1\,,\\
\ \ \ \ \ \ \ \ \ \ \ \ \ \ \ \ \ \ \ \ y\in (0,\,1]\,, \ n\geq 0\,.
\label{e8aga}
\end{multline}


\noindent

Д\,о\,к\,а\,з\,а\,т\,е\,л\,ь\,с\,т\,в\,о\,.\ Подставив выражение для функции 
$d_n(\overline{\rho},y)$ и проведя преобразования, получим
\begin{multline*}
d_{n+1}(\overline{\rho},y) -d_n(\overline{\rho},y) = 
\fr{\sum\limits_{m=0}^{n+1}m\sum\limits_{\overline{k}\in A_m} 
p_{\overline{k}}(\overline{\rho},y)}
{\sum\limits_{m=0}^{n+1}
\sum\limits_{\overline{k}\in A_m} p_{\overline{k}}(\overline{\rho},y)} - {}\\
{}-
\fr{\sum\limits_{m=0}^n m \sum\limits_{\overline{k}\in A_m} p_{\overline{k}} 
(\overline{\rho},y)}{\sum\limits_{m=0}^n
\sum\limits_{\overline{k}\in A_m}p_{\overline{k}}(\overline{\rho},y)}={}\\
{}=\fr{\sum\limits_{m=1}^n m \sum\limits_{\overline{k}\in 
A_m}p_{\overline{k}}(\overline{\rho},y)+(n+1)\sum\limits_{\overline{k}\in 
A_{n+1}}  p_{\overline{k}}(\overline{\rho},y)}{\sum\limits_{m=0}^n\sum\limits_{\overline{k
}\in A_m}p_{\overline{k}}(\overline{\rho},y)+\sum\limits_{\overline{k}\in 
A_{n+1}}p_{\overline{k}}(\overline{\rho},y)} -{}
\end{multline*}
\begin{multline}
{}-
\fr{\sum\limits_{m=0}^n m 
\sum\limits_{\overline{k}\in A_m}p_{\overline{k}}(\overline{\rho},y)}
{\sum\limits_{m=0}^n\sum\limits_{\overline{k}\in A_m} 
p_{\overline{k}}(\overline{\rho},y)}={}\\
{}=\fr{(n+1)\sum\limits_{\overline{k}\in 
A_{n+1}}p_{\overline{k}}(\overline{\rho},y)g_n(\overline{\rho},y)}{g_{n+1}(\overline{\rho},y) g_n(\overline{\rho},y)} -{}\\
{}-
\fr{\sum\limits_{\overline{k}\in 
A_{n+1}}p_{\overline{k}}(\overline{\rho},y)\sum\limits_{m=0}^n  m 
\sum\limits_{\overline{k}\in A_m} p_{\overline{k}}(\overline{\rho},y) }
{g_{n+1}(\overline{\rho},y) g_n(\overline{\rho},y)}
={}\\
{}=\left [ 1-q_{n+1}(\overline{\rho},y)\right ] \left [n+1-d_n(\overline{\rho},y)\right ]\,.
\label{e9aga}
\end{multline}


    Докажем утверждение~1 методом индукции. При $n = 0$, как следует 
из~(\ref{e9aga}), имеем
$$
d_2(\overline{\rho},y) - d_1 (\overline{\rho},y) =1-q_1(\overline{\rho},y)\,,
$$
    т.\,е.\ утверждение~1 при $n = 0$ справедливо. 

    Пусть неравенства~(\ref{e8aga}) справедливы для некоторого $n > 0$. 
Докажем, что они справедливы и для $n + 1$. Из~(\ref{e9aga}) получаем
\begin{multline*}
d_{n+1}(\overline{\rho},y)- d_n(\overline{\rho},y)={}\\
{}=\left [ 1-
q_{n+1}(\overline{\rho},y)\right ] \left [n+1-d_n(\overline{\rho},y)\right ] ={}\\
{}= \left [ 1-
1-q_{n+1}(\overline{\rho},y)\right ] \left [ n-{}\right.\\
{}-\left. d_{n-1}(\overline{\rho},y)+d_{n-1}(\overline{\rho},y)-
d_n(\overline{\rho},y)+1\right ] ={}\\
{}=\left [ 1-q_{n+1}(\overline{\rho},y)\right ] 
\left [ n-d_{n-1}(\overline{\rho},y)-{}\right.\\
{}-\left. \left ( d_n(\overline{\rho},y)-d_{n-1}(\overline{\rho},y)\right )+1\right] = {}\\
{}=
\left [ 1-q_{n+1}(\overline{\rho},y)\right ]
\left [ 
\fr{d_n(\overline{\rho},y) -d_{n-1}(\overline{\rho},y)}{1-
q_n(\overline{\rho},y)}\right.-{}\\
{}-\left.
\left ( d_n(\overline{\rho},y)-d_{n-1}(\overline{\rho},y)\right )+1
\vphantom{\fr{d_n(\overline{\rho})}{(q_n)}}
\right ]={}\\
{}=
\left [ 1-q_{n+1}(\overline{\rho},y)\right ]
\left [ 
\vphantom{\fr{d_n(\overline{\rho})}{(q_n)}}
\left ( d_n(\overline{\rho},y\right)\right. -{}\\
 {}-\left.
d_{n-1}\left(\overline{\rho},y)\right )\fr{q_n(\overline{\rho},y)}{1-
q_n(\overline{\rho},y)}+1\right ]\,.
\end{multline*}
Так как по предположению $d_n (\overline{\rho},y) -d_{n-1}(\overline{\rho},y) 
>0$, то правая часть последнего равенства больше нуля; следовательно, 
$d_{n+1}(\overline{\rho},y)-d_n(\overline{\rho},y)>0$. 

    Продолжив преобразование правой части последнего равенства и 
учитывая предположение $d_n(\overline{\rho},y) -d_{n-1}(\overline{\rho},y)<1$, 
получим
\begin{multline*}
d_{n+1}((\overline{\rho},y) -d_n(\overline{\rho},y)<{}\\
{}< \left [ 1-
q_{n+1}(\overline{\rho},y)\right ]
\left ( \fr{q_n(\overline{\rho},y)}{1-q_n(\overline{\rho},y)}+1\right )={}\\
{}=
\fr{1-q_{n+1}(\overline{\rho},y)}{1-q_n(\overline{\rho},y)}<1\,,
\end{multline*}
так как $0< q_n(\overline{\rho},y)<q_{n+1}(\overline{\rho},y)<1$, $n>0$, $y\in 
(0,\,1]$.

Следовательно, утверждение~1 доказано.

\medskip

\noindent
\textbf{Утверждение 2.} $q_N(\overline{\rho},y)$~--- \textit{монотонно-воз\-рас\-та\-ющая 
функция по $y\in (0,\,1]$. При этом $0< q_N(\overline{\rho},y)\;\leq $\linebreak 
$\leq\;q_N(\overline{\rho},1) <1$, $y\in (0,\,1]$,  и $\underset{y\rightarrow 
0}{\mathrm{lim}}\,q_N(\overline{\rho},y) =0$}.

\medskip

\noindent
Д\,о\,к\,а\,з\,а\,т\,е\,л\,ь\,с\,т\,в\,о\,.\  Возрастание функции 
$q_N(\overline{\rho},y)$ следует непосредственно из~(\ref{e7aga}) и 
утверж\-де\-ния~1. Доказательство неравенств в условии утверждения очевидно 
следует из~(\ref{e5aga}) и вида функции $g_n (\overline{\rho},y)$, $n\geq 0$. 
Для предела имеем:
\begin{multline*}
\underset{y\rightarrow 0}{\mathrm{lim}}\,q_N(\overline{\rho},y) 
=\underset{y\rightarrow 0}{\mathrm{lim}}\,\fr{g_{N- 1}(\overline{\rho},y)}{g_N(\overline{\rho},y)} = {}\\
{}= \underset{y\rightarrow 0}{\mathrm{lim}}\,\left (
g_{N-1}(\overline{\rho},y)\Bigg / \left ( 
\vphantom{\prod\limits_{v\in\Omega_u^+}}
g_{N-1}(\overline{\rho},y)\right.\right.+{}\\
{}+\left.\left.\sum\limits_{\overline{k}\in A_N}\prod\limits_{v\in\Omega_u^+} 
\fr{z_v(0,\rho_v,k_v,k^\prime_v,k^{\prime\prime}_v)}{y^N}\right )\right ) = {}\\
{}= \underset{y\rightarrow 0}{\mathrm{lim}}\,\left (
y^N g_{N-1}(\overline{\rho},y)\Bigg / 
\left ( 
\vphantom{\prod\limits_{v\in\Omega_u^+}}
y^N g_{N-1}(\overline{\rho},y)+{}\right.\right.\\
{}+\left.\left.\sum\limits_{\overline{k}\in A_N}
\prod\limits_{v\in\Omega_u^+} z_v(0,\rho_v,k_v,k_v^\prime , k_v^{\prime\prime}) 
\right ) \right )=0\,.
\end{multline*}
    
\medskip

\noindent
\textbf{Утверждение 3.} \textit{Производная функции~$q_N (\overline{\rho},y)$ по 
$y\in (0,\,1]$ удовлетворяет следующим соотношениям}:
\begin{align}
\underset{y\rightarrow 0}{\mathrm{lim}}\fr{\partial q_N(\overline{p},y)}
{\partial  y} &= \fr{\sum\limits_{\overline{k}\in A_{N-1}} 
p_{\overline{k}}(\overline{\rho},1)}{\sum\limits_{\overline{k}\in 
A_N}p_{\overline{k}}(\overline{\rho},1)}\,;\label{e10aga}\\
\fr{\partial q_N(\overline{\rho},y)}{\partial y}\Big |_{y=1}&<1\,.\label{e11aga}
\end{align}

\medskip

\noindent
Д\,о\,к\,а\,з\,а\,т\,е\,л\,ь\,с\,т\,в\,о\,.\ Проведя преобразования 
функции~$q_N(\overline{\rho},y)$, получим:
\begin{multline*}
\underset{y\rightarrow 0}{\mathrm{lim}}\fr{q_N(\overline{\rho},y)}{y} = {}\\
\!\!{}=
\underset{y\rightarrow 0}{\mathrm{lim}}
\fr{\sum\limits_{m=0}^{N-1}\sum\limits_{\overline{k}\in A_m}
\!\!\left (\prod\limits_{v\in\Omega_u^+}\!\! 
z_v(0,\rho_v,k_v,k_v^\prime , k_v^{\prime\prime})\right )\!\!\Bigg /\!\! y^m}
{y\sum\limits_{m=0}^{N}\sum\limits_{\overline{k}\in A_m}
\!\!\left(\prod\limits_{v\in\Omega_u^+}\!\! z_v\left (0,\rho_v,k_v,k_v^\prime , 
k_v^{\prime\prime}\right )\right )\!\!\Bigg /\!\!y^m} = \!\!\!
\end{multline*}
\begin{multline*}
\!\!\!\!\!\!{}=\underset{y\rightarrow 0}{\mathrm{lim}}\,
\fr{\sum\limits_{m=0}^{N-1}\sum\limits_{\overline{k}\in A_m}
y^{N-1-m}\prod\limits_{v\in\Omega_u^+} z_v(0,\rho_v,k_v,k_v^\prime , 
k_v^{\prime\prime})}{\sum\limits_{m=0}^{N}\sum\limits_{\overline{k}
\in A_m} y^{N-m}
\prod\limits_{v\in\Omega_u^+} z_v(0,\rho_v,k_v,k_v^\prime , 
k_v^{\prime\prime})}={}\!\\
{}=\fr{\sum\limits_{\overline{k}\in A_{N-1}} p_{\overline{k}}(\overline{\rho},1)}{ 
\sum\limits_{\overline{k}\in A_{N}} p_{\overline{k}}(\overline{\rho},1)}\,.
\end{multline*}
Очевидно, $\underset{y\rightarrow 0}{\mathrm{lim}} \,[d_N (\overline{\rho},y) -
d_{N-1} (\overline{\rho},y)]=1$, так как $\underset{y\rightarrow 
0}{\mathrm{lim}}\,d_n (\overline{\rho},y)=n$, $n>0$.

Следовательно, учитывая~(\ref{e7aga}), получаем~(\ref{e10aga}). 
Справедливость~(\ref{e11aga}) непосредственно следует из~(\ref{e7aga}) и 
утверждения~1.

\medskip

\noindent
\textbf{Утверждение 4.} \textit{Пусть $y^*\in (0,\,1]$~--- решение 
уравнения}~(\ref{e4aga}). \textit{Тогда}
\begin{equation*}
\fr{\partial q_N(\overline{\rho},y)}{\partial y}\Big |_{y=y^*}<1\,.
%\label{e12aga}
\end{equation*}

\medskip

\noindent
Д\,о\,к\,а\,з\,а\,т\,е\,л\,ь\,с\,т\,в\,о\,.\ \ Доказательство следует из~(\ref{e7aga}), 
так как $q_N(\overline{\rho},y^*)/y^* =1$.
\medskip

\noindent
\textbf{Утверждение 5.} \textit{Уравнение}~(\ref{e4aga}) \textit{имеет решение $y^*\in 
(0,\,1)$ тогда и только тогда, когда} 
\begin{equation}
\fr{\sum\limits_{\overline{k}\in A_{N-1}} p_{\overline{k}}(\overline{\rho},1)}{ 
\sum\limits_{\overline{k}\in A_{N}} p_{\overline{k}}(\overline{\rho},1)} >1\,.
\label{e13aga}
\end{equation}
\textit{Если уравнение}~(\ref{e4aga}) \textit{имеет решение $y^*\in (0,\,1)$, то оно 
единственное положительное решение}.
\medskip

\noindent
Д\,о\,к\,а\,з\,а\,т\,е\,л\,ь\,с\,т\,в\,о\,.\ Пусть выполняется 
неравенство~(\ref{e13aga}). Тогда, как следует из утверждения~3, 
$\underset{y\rightarrow 0}{\mathrm{lim}} (\partial q_N(\overline{\rho},y)/\partial y) 
>1$. Кроме того, как следует из утверждения~2, 
$\underset{y\rightarrow 0}{\mathrm{lim}} q_N(\overline{\rho},y)=0$. Тогда, так 
как $q_N(\overline{\rho},y)$~--- непрерывно-дифференцируемая функция по 
$y\in (0,\,1]$, существует значение $y^\prime \in (0,\,1)$ такое, что 
$q_N(\overline{\rho},y)>y$ для всех $y\in (0,\,y^\prime]$ (следует из теоремы о 
конечном приращении~\cite{11aga}). В то же время, согласно утверждению~2, 
$q_N(\overline{\rho},y)<y$ в окрестности точки $y=1$ (рис.~\ref{f1aga},\,\textit{а}). 
Следовательно, кривая $x=q_N(\overline{\rho},y)$ пересекает прямую $x=y$ 
хотя бы в одной точке $y=y^*\in (0,\,1)$, т.\,е.\ уравнение~(\ref{e4aga}) имеет 
хотя бы одно решение $y^*\in (0,\,1)$.

\begin{figure*}
\vspace*{1pt}
\begin{center}
\vspace*{1pt}
\mbox{%
\epsfxsize=158mm
\epsfbox{aga-1.eps}
}
\end{center}
\vspace*{-9pt}
\Caption{Примеры кривых $x=q_N(\overline{\rho},y)$ и $x=y$ (\textit{а})~при существовании решения 
уравнения~(\ref{e5aga}) и (\textit{б})~при выполнении условий~(17)
\label{f1aga}}
\vspace*{6pt}
\end{figure*}

Пусть уравнение~(\ref{e4aga}) имеет решение $y^*\in (0,\,1)$ и 
\begin{equation}
\fr{\sum\limits_{\overline{k}\in A_{N-1}}p_{\overline{k}}(\overline{\rho},1)}{ 
\sum\limits_{\overline{k}\in A_{N}}p_{\overline{k}}(\overline{\rho},1)}\leq 
1\,.\label{e14aga}
\end{equation}
Тогда из условий утверждений~2 и~3 следует, что 
уравнение~(\ref{e4aga}) в интервале $(0,\,1)$ имеет более одного решения, что 
может быть только при существовании решения $y^\prime \in (0,\,1)$ такого, 
что в окрестности точки $y=y^\prime$ выполняются неравенства: 
$q_N(\overline{\rho},y)<y$ при $y<y^\prime$ и $q_N(\overline{\rho},y)>y$ при 
$y>y^\prime$, где $y$ принадлежит указанной окрест\-ности точки~$y^\prime$ 
(рис.~\ref{f1aga},\,\textit{б}). Тогда в точке $y=y^\prime$ производная функции 
$q_N(\overline{\rho},y)$ по $y$ больше~1, что противоречит утверждению~4. 
Следовательно, неравенство~(\ref{e13aga}) справедливо.


Пусть уравнение~(\ref{e4aga}) имеет более одного положительного 
решения. Рассуждая точно так же, как и выше (в случае выполнения 
условий~(\ref{e14aga})), получим противоречие утверждению~4. 
Следовательно, утверждение~5 справедливо.
\medskip

\noindent
\textbf{Следствие.} \textit{Неравенства}
\begin{gather*}
\fr{\mu_v c_v (1-B_v)}{\Lambda_v}>1\,,\quad \fr{1-B_v}{\Lambda_v \tau_v B_v}>1\,,\\ 
\fr{1-B_v}{\Lambda_v t_v}>1\,,\ v\in\Omega_u^+\,,
\end{gather*}
\textit{являются необходимым условием существования решения 
уравнения}~(\ref{e4aga}).

\medskip
\noindent
Д\,о\,к\,а\,з\,а\,т\,е\,л\,ь\,с\,т\,в\,о\,.\ Пусть $\overline{k}_v$~--- это 
набор~$\overline{k}$, у которого $k_v=0$. Преобразовав левую 
часть~(\ref{e13aga}), получим

\noindent
\begin{multline*}
\fr{\sum\limits_{\overline{k}\in A_{N-1}} p_{\overline{k}} (\overline{\rho},1)}
{ \sum\limits_{\overline{k}\in A_{N}} 
 p_{\overline{k}}(\overline{\rho},1)} 
={}
\\
{}=
\fr{\sum\limits_{\overline{k}\in A_{N-1}}\prod\limits_{v\in \Omega_u^+} 
z_v\left(0,\rho_v,k_v,k_v^\prime , k_v^{\prime\prime}\right)}
{\sum\limits_{\overline{k}\in A_{N}}
\prod\limits_{v\in \Omega_u^+} z_v\left (0,\rho_v,k_v,k_v^\prime , k_v^{\prime\prime}\right )} \leq{}
\\
{}\leq
\left ( 
\vphantom{\prod\limits_{v^\prime\in\Omega_u^+\backslash v}}
\fr{\mu_v c_v(1-B_v)}{\Lambda_v}\right. \times{}\\
{}\times \sum\limits_{k_v=0}^{N-1}\sum\limits_{\overline{k}_v\in A_{N-1-k_v}} z_v\left(0,\rho_v,k_v+1,k_v^\prime , 
k_v^{\prime\prime}\right )\times{}\\
{}\times \left.\prod\limits_{v^\prime\in\Omega_u^+\backslash v} z_v^\prime 
\left(0,\rho_v,k_v,k_v^\prime , k_v^{\prime\prime}\right) \right)
\Bigg /{}\\
\Bigg / \left ( 
\vphantom{\prod\limits_{v^\prime\in\Omega_u^+\backslash v}}
\sum\limits_{k_v=0}^{N-1} \sum\limits_{\overline{k}_v\in A_{N-1-k_v}} z_v 
\left (0,\rho_v,k_v+1,k_v^\prime , 
k_v^{\prime\prime}\right )\right. \times{}\\
{}\times \prod\limits_{v^\prime\in\Omega_u^+\backslash v} 
z_{v^\prime}\left(0,\rho_v,k_v,k^\prime , k_v^{\prime\prime}\right)+{}\\
{}+
\sum\limits_{\overline{k}_v\in A_N} z_v\left (0,\rho_v, 0,k_v^\prime , 
k_v^{\prime\prime}\right)\times{}\\
\left.{}\times \prod\limits_{v^\prime\in\Omega_u^+\backslash v}z_{v^\prime} 
\left(0,\rho_v,k_v,k_v^\prime , k_v^{\prime\prime}\right )\right )\,.
\end{multline*}
Как следует из правой части последнего неравенства, если 
$\mu_v c_v (1-B_v)/\Lambda_v \leq 1$, то она меньше~1. Поэтому, чтобы 
выполнилось условие~(\ref{e13aga}), необходимо выполнение первого 
неравенства в условии следствия для каждого $v\in\Omega_u^+$. Точно так же 
доказывается необходимость выполнения второго и третьего неравенств в 
условии следствия.

    Пусть $y[n]$, $n\geq 0$, последовательность, полученная по формуле 
$y[n+1]=q_N(\overline{\rho},y[n])$, $y[0]=1$.

\medskip

\noindent
\textbf{Утверждение 6.} \textit{Пусть $y^*\in (0,\,1)$~--- решение 
уравнения}~(8). \textit{Тогда последовательность $y[n]$, $n\geq 0$, сходится 
к решению~$y^*$}.

\medskip

\noindent
Д\,о\,к\,а\,з\,а\,т\,е\,л\,ь\,с\,т\,в\,о\,.\ Отметим, что $y[1]<y[0]$ (это следует из 
утверждения~2, так как $y[0]=1$). Пусть для некоторого $n>1$ выполняется 
условие $y[n]<y[n-1]$. Тогда, как следует из утверждения~2, указанное условие 
выполняется и для $n+1$, т.\,е.\ по индукции следует, что последовательность 
$y[n]$, $n\geq 0$, монотонно убывает. 

    Пусть для некоторого $n>0$ $y[n]>y^*$ (существование такого $n$ 
следует из равенства $y[0]=1$). Тогда, как следует из утверждения~2, 
$y[n+1]\;=$\linebreak $=\;q_N(\overline{\rho},y[n])>q_N(\overline{\rho},y^*) =y^*$, т.\,е.\ 
последовательность ограничена снизу величиной~$y^*$. Значит, существует 
$\underset{n\rightarrow \infty}{\mathrm{lim}}\,y[n]=y^0\geq y^*$. Так как 
$q_n(\overline{\rho},y)$~--- непрерывная по~$y$ функция, то можно написать 
$\underset{n\rightarrow 
\infty}{\mathrm{lim}}\,q_N(\overline{\rho},y[n])=q_N(\overline{\rho},y^0)=y^0$, 
т.\,е.\ $y^0$~--- решение уравнения~(\ref{e4aga}). Из единственности 
положительного решения уравнения~(\ref{e4aga}) получаем $y^0=y^*$.

    Пусть в узле используется схема полного разделения буферной памяти. 
Тогда для интенсив\-ностей~$\Lambda_v^*$, $v\in\Omega_u^+$, справедливы 
соотношения:
$$
\Lambda_v^* = \fr{\Lambda_v}{1-\pi_v}\,,
$$
где $v\in\Omega_u^+$.


Фиксируем произвольную линию сети~$v$. Пусть $\overline{k}_v = (k_v, 
k_v^\prime, k_v^{\prime\prime})$~--- состояние буферной памяти линии~$v$; 
$k_v$, $k_v^\prime$, $k_v^{\prime\prime}$ определены выше. Тогда с 
учетом введенных ранее предположений и обозначений для вероятности 
блокировки линии справедлива формула~\cite{4aga}:
\begin{equation}
\pi_v = \fr{1}{G_{N_v}}\sum\limits_{k_v=N_v} 
z_v(\pi_v,\rho_v,\overline{k}_v)\,,
\label{e15aga}
\end{equation}
где 
\begin{multline*}
z_v(\pi_v, \rho_v, \overline{k}_v)={}\\
{}=
\begin{cases}
\fr{\rho_v^{\prime * k_v^\prime}}{k_v^\prime !}\,
 \fr{\rho_v^{\prime\prime * k_v^{\prime\prime}}}{k_v^{\prime\prime}!}\,
 \fr{\rho_v^{*k_v}}{k_v !} & \mbox{при}\ k_v<c_v\,,\\
 \fr{\rho_v^{\prime *k_v^\prime}}{k_v^{\prime }! }
 \fr{\rho_v^{\prime\prime * k_v^{\prime\prime}}}{k_v^{\prime\prime}!}
\fr{\rho_v^{*k_v}}{c_v !c_v^{k_v-c_v}} & \mbox{при}\ k_v\geq c_v\,;
\end{cases}
\end{multline*}
\begin{align*}
G_{N_v} &= \sum\limits_{m=0}^{N_v} z_v (\pi_v ,\rho_v , \overline{k}_v)\,;\\ 
\rho_v^*&=\fr{\rho_v}{1-\pi_v}\,;
\end{align*}
$\rho_v$, $\rho_v^{\prime *}$, 
$\rho_v^{\prime\prime *}$, $v\in\Omega_u^+$ определены выше.
    
Пусть $y_v=1-\pi_v$, а $q_{N_v} (\rho_v, y_v)$~--- выражение в правой 
части~(\ref{e15aga}). Тогда из равенств~(\ref{e15aga}), взяв~$y_v$ в качестве 
неизвестной переменной, получим систему независимых уравнений:
\begin{equation}
y_v = q_{N_v}(\rho_v, y_v)\,, \quad v\in \Omega_u^+\,.
\label{e16aga}
\end{equation}
    
    Заметим, что для фиксированной $v$ и заданных параметров $\Lambda_v$, 
$\mu_v$, $\tau_v$, $t_v$, $N_v$, $v\in\Omega_u^+$, уравнение в~(\ref{e16aga}) 
является частным случаем уравнения~(\ref{e4aga}) и совпадает с последним, 
когда число исходящих линий из узла равно~1. Следовательно, для схемы 
полного разделения памяти также справедливы все приведенные выше 
утверждения~1--6 и следствие. Заметим, что неравенство~(\ref{e13aga}) в 
условии утверждения~5 при $B_v=0$ и $t_v=0$ преобразуется в неравенство 
$\Lambda_v / (\mu_v c_v) >1$, $v\in\Omega_u^+$. Последовательность 
$\overline{y}[n]$, $n\geq 0$, в утверждении~6 определяется по формуле:
    \begin{gather*}
    \overline{y}[n] =\{y_v[n],\ v\in\Omega_u^+\}\,,\
    y_v[n+1]=q_{N_v} (\rho_v,\,y_v[n])\,,\\
    y_v[0] =1\,,\quad n\geq 0\,,\quad v\in \Omega_u^+\,.
    \end{gather*}


\section{Алгоритм расчета} %4

    Для вычисления интенсивностей потоков и вероятностей блокировок в 
узле предлагается следующий алгоритм, описывающий изложенную выше 
итерационную процедуру. Введем обозначения:
$y_u^v$~--- вероятность блокировки узла для заявок, направляемых на 
линию~$v$,
\begin{gather*}
y_u^v  = 
\begin{cases}
y_u & \mbox{для}\ v\in\Omega_u^+\ \mbox{при}\\
&\mbox{полнодоступной схеме},\\
y_v & \mbox{при схеме полного распределения}\\
&\mbox{памяти};
\end{cases}
\\
q_N^v(\overline{\rho}_u^{-v}, y_u^v)  = 
\begin{cases}
q_N(\overline{\rho},y) & \mbox{для}\ v\in\Omega_u^+\ \mbox{при пол-}\\ 
&\mbox{нодоступной схеме},\\
q_{N_v}(\rho_v, y_v) & \mbox{при схеме полного}\\
&\mbox{распределения}\\ 
&\mbox{памяти},  v\in\Omega_u^+\,.
\end{cases}
\end{gather*}



Тогда уравнения~(\ref{e4aga}) и~(\ref{e16aga}) записываются в виде:
$$
y_u^v = q_N^v (\overline{\rho}^v_u, y^v_u)\,,\quad v\in \Omega_u^+\,.
$$
Для значений, вычисляемых на $k$-м шаге алгоритма, к 
обозначениям соответствующих параметров приписывается знак~$[k]$.
\pagebreak

\textbf{Шаг 0.} 
\begin{enumerate}[1.]
\item  \textit{Инициализация}. Вычисление начальных значений 
параметров~$\rho_v$, $v\in\Omega_u^+$: $\Lambda_v[0]=\Lambda_v$, 
$\rho_v[0]=\Lambda_v[0]/(\mu_v(1-B_v))$, $y_u^v[0]=1$.
\item \textit{Проверка условий существования решения}. Если для некоторой 
линии $v\in\Omega_u^+$ выполняется хотя бы одно неравенство $(c_v\mu_v(1-
B_v))/\Lambda_v[0]\;\leq$\linebreak $\leq\;1$, или $(1-B_v)/(\Lambda_v\tau_v B_v) \leq 1$, или 
$(t_v(1\;-$\linebreak $-\;B_v))/\Lambda_v[0] \leq 1$, то алгоритм заканчивает работу с 
результатом <<нагрузка не реализуема>>. Если в узле используется 
полнодоступная схема и $(c_v\mu_v(1-B_v))/\Lambda_v[0] > 1$, $(1-
B_v)/(\Lambda_v\tau_v B_v)\;>$\linebreak $>\;1$, $(t_v(1-B_v))/\Lambda_v[0] > 1$ для всех 
$v\in\Omega_u^+$, то проверяется условие~(\ref{e13aga}) для $\Lambda_v =
\Lambda_v[0]$, $v\in\Omega_u^+$, и при невыполнении этого условия алгоритм 
заканчивает работу с результатом <<нагрузка не реализуема>>.
\end{enumerate}

    При вычислении левой части неравенства~(\ref{e13aga}) рекомендуется 
использовать метод свертки Базена (см.~\cite{12aga}), позволяющий 
производить рекуррентные вычисления (подробно этот метод описан также 
в~[1, 3--6]).



\medskip
\textbf{Шаг~$k$} ($k > 0$):
\begin{enumerate}[1.]
\item \textit{Вычисление вероятностей блокировок}. Используя значения 
параметров $\overline{\rho}_u^v[k-1]$, $y_u^v[k-1]$, $v\in\Omega_u^+$, 
вычисление с помощью формул~(1)--(7) значений 
вероятностей $y[k]=1- \pi [k]$~--- в случае полнодоступной памяти, или 
$y_v[k]=1- \pi_v[k]$, $v\in\Omega_u^+$, с помощью формул~(\ref{e15aga})~--- в 
случае полного разделения памяти. При вычислении этих значений 
рекомендуется использовать метод свертки Базена.
    \item \textit{Проверка условий останова алгоритма}. Если хотя бы для 
одной $v\in\Omega_u^+$ для заданного значения точности   выполняется 
условие
$$
\fr{\vert \Lambda_v^*[k]-\Lambda_v^*[k-1]\vert}{\Lambda_v^*[k]}> \varepsilon\,,
$$
то вычисление параметров $\overline{\rho}_u^v[k]$, $v\in\Omega_u^+$, и 
переход к шагу~$k$, положив $k$ равным $k+1$, иначе алгоритм завершает 
работу. 
\end{enumerate}

    По завершении алгоритма либо выявится, что нагрузка в системе не 
реализуема, либо будут вычислены интенсивности потоков, поступающих на 
линии узла, и стационарные вероятности блокировок для заявок каждого типа. 
    
\section{Примеры расчета}

    Для проверки точности вычисления результатов с помощью 
предложенного выше алгоритма и приемлемости введенных предположений 
были проведены вычислительные эксперименты с использованием 
аналитических и имитационных моделей. Во всех рассмотренных ниже 
примерах потоки внешних заявок считаются пуассоновскими. 
В~табл.~1 приведены значения вероятности блокировок вновь 
поступивших извне заявок, полученные на основании точной формулы, 
приведенной в~\cite{4aga} для СМО типа $M\vert M\vert 1\vert 0$ с повторными 
заявками при экспоненциальном распределении интервала времени между 
повторными попытками (первая строка таблицы), алгоритма из подраздела~5 
настоящей статьи (вторая строка) и имитационной модели при постоянном 
интервале времени между повторными попытками, равном~10 (третья строка). 
Расчет табл.~1 проведен для узла с одной исходящей одноканальной 
линией при интенсивности первичного потока $\Lambda =1$ и емкости 
накопителя $N_v=1$. Таблицы~2 и~3 вычислены с помощью 
алгоритма из подраздела~5 и имитационной модели соответственно при одной 
исходящей линии с числом каналов~10.


    В табл.~\ref{t4aga} и~\ref{t5aga} приведены значения вероятности 
блокировки узла с тремя исходящими линиями канальной емкости~10 каждая 
при $\mu_v =0{,}2$, $v\in\Omega_u^+$,  вычисленные с помощью алгоритма из 
подраздела~5 и имитационной модели с интервалом повторной попытки, 
равным~10, соответственно. В табл.~\ref{t4aga} и~\ref{t5aga} знак <<--->> в 
ячейках означает, что предложенная нагрузка $\Lambda_v$, $v\in\Omega_u^+$, 
не реализуема.



В табл.~\ref{t6aga} отражены вероятности блокировки такого же узла с 
накопителем $N = 35$ при экспоненциальном распределении интервала 
времени между повторными попытками со средним значением~$\tau$. 


Результаты вычислительного эксперимента показывают, что с  увеличением 
длины интервала между повторными попытками  вероятность блокировки 
увеличивается и приближается к значению,\linebreak
вычисленному с помощью 
алгоритма из подраздела~5 (см.\ табл.~\ref{t4aga} и~\ref{t6aga}), т.\,е.\ при 
пуассоновском внешнем потоке заявок предположение, что суммарный 
входной поток заявок  является пуассоновским, вполне приемлемо для 
предварительного анализа характеристик узла (например, при  $\tau c_v\mu_v > 
10$). Как показывают табл.~1--3, вероятность блокировки 
узла существенно зависит от\linebreak 

\vspace*{6pt}
\noindent
%\begin{table*}\small %tabl1
{\small
{{\tablename~1}\ \ \small{Вероятности блокировок при одной исходящей одноканальной линии}}
%\label{t1aga}}
\vspace*{-3pt}

\begin{center}
{\tabcolsep=7.3pt
\begin{tabular}{|c|c|c|c|c|c|}
\hline
&\multicolumn{5}{c|}{$\mu$}\\
\cline{2-6}
\multicolumn{1}{|c|}{\raisebox{4pt}[0pt][0pt]{№}}&1{,}1&1{,}2&2&3&4\\
\hline
1&0,9091&0,8333&0,5000&0,3333&0,2500\\
2&0,9091&0,8333&0,5000&0,3333&0,2500\\
3&0,8867&0,8452&0,4944&0,3167&0,2396\\
\hline
\end{tabular}}
\end{center}
%\vspace*{-6pt}
%\end{table*}
}
%\bigskip
%\medskip
\addtocounter{table}{1}
\pagebreak

\end{multicols}

\renewcommand{\figurename}{\protect\bf Таблица}
%\renewcommand{\tablename}{\protect\bf Рис.}
\begin{figure*}
{\small
\begin{minipage}[t]{76mm}
%\begin{table*}\small %tabl2
\begin{center}
\Caption{Вероятности блокировок при одной исходящей многоканальной линии ($\varepsilon 
=0{,}0001$)
\label{t2aga}}
\vspace*{2ex}

\tabcolsep=6.5pt
\begin{tabular}{|c|c|c|c|c|c|}
\hline
&\multicolumn{5}{c|}{$\mu$}\\
\cline{2-6}
\multicolumn{1}{|c|}{\raisebox{4pt}[0pt][0pt]{$N$}}&0{,}11&0{,}12&0{,}2&0{,}3&0{,}4\\
\hline
10&0,4845&0,2935&0,0204&0,0017&0,0002\\
15&0,1181&0,0545&0,0006&0,0000&0,0000\\
20&0,0489&0,0167&0,0000&0,0000&0,0000\\
\hline
\end{tabular}
\end{center}
%\end{table*}
\end{minipage}
\hfill
\begin{minipage}[t]{76mm}
%\begin{table*}\small %tabl3
\begin{center}
\Caption{Вероятности блокировок при одной исходящей линии
\label{t3aga}}
\vspace*{2ex}

\tabcolsep=6.5pt
\begin{tabular}{|c|c|c|c|c|c|}
\hline
&\multicolumn{5}{c|}{$\mu_v$}\\
\cline{2-6}
\multicolumn{1}{|c|}{\raisebox{4pt}[0pt][0pt]{$N$}}&0{,}11&0{,}12&0{,}2&0{,}3&0{,}4\\
\hline
10&0,5247&0,3238&0,0219&0,0019&0,0001\\
15&0,1726&0,0912&0,0004&0,0001&0,0000\\
20&0,1180&0,0371&0,0000&0,0000&0,0000\\
\hline
\end{tabular}
\end{center}
%\end{table*}
\end{minipage}
}
\vspace*{6pt}
\end{figure*}

\renewcommand{\figurename}{\protect\bf Рис.}
\renewcommand{\tablename}{\protect\bf Таблица}
\addtocounter{table}{2}

\begin{table}\small %tabl4
\begin{center}
\parbox{400pt}{\Caption{Вероятности блокировок при трех исходящих линиях, вычисленные алгоритмом из 
подраздела~5 ($\varepsilon =0{,}0001$)
\label{t4aga}}
}

\vspace*{2ex}

\tabcolsep=8pt
\begin{tabular}{|c|c|c|c|c|c|c|c|c|c|}
\hline
&\multicolumn{9}{c|}{$\Lambda_v$}\\
\cline{2-10}
\multicolumn{1}{|c|}{\raisebox{4pt}[0pt][0pt]{$N$}}&1&1{,}1&1{,}2&1{,}3&1{,}4&1{,}5&1{,}6&1{,}7&1{,}8\\
\hline
20&0,0677&0,1423&0,2975&0,7653&---&---&---&---&---\\
25&0,0065&0,0173&0,0394&0,0827&0.1690&0.3827&---&---&---\\
30&0,0005&0,0019&0,0059&0,0155&0.0361&0.0790&0.1792&0,7259&---\\
35&0,0000&0,0002&0,0008&0,0030&0,0089&0,0234&0,0574&0,1505&---\\
40&0,0000&0,0000&0,0001&0,0005&0,0022&0,0075&0,0220&0,0617&0,2449\\
\hline
\end{tabular}
\end{center}
%\end{table}
\vspace*{6pt}
%\begin{table}\small %tabl5
\begin{center}
\parbox{400pt}{\Caption{Вероятности блокировок при трех исходящих линиях, вычисленные с помощью 
имитационной модели
\label{t5aga}}
}

\vspace*{2ex}

\tabcolsep=8pt
\begin{tabular}{|c|c|c|c|c|c|c|c|c|c|}
\hline
&\multicolumn{9}{c|}{$\Lambda_v$}\\
\cline{2-10}
\multicolumn{1}{|c|}{\raisebox{4pt}[0pt][0pt]{$N$}}&1&1{,}1&1{,}2&1{,}3&1{,}4&1{,}5&1{,}6&1{,}7&1{,}8\\
\hline
20&0,0786&0,1695&0,3549&0,7056&---&---&---&---&---\\
25&0,0069&0,0190&0,0441&0,0998&0,2266&0,4583&---&---&---\\
30&0,0007&0,0024&0,0075&0,0184&0,0462&0,1025&0,2380&0,6931&---\\
35&0,0000&0,0003&0,0007&0,0040&0,0129&0,0307&0,0890&0,2981&---\\
40&0,0000&0,0000&0,0000&0,0011&0,0041&0,0095&0,0346&0,0790&0,3179\\
\hline
\end{tabular}
\end{center}
%\end{table}
\vspace*{6pt}
%\begin{table}\small %tabl6
\begin{center}
\parbox{356pt}{\Caption{Зависимость вероятности блокировки при трех исходящих линиях, вы\-чис\-лен\-ные с 
помощью имитационной модели со случайным интервалом между повторными попытками
\label{t6aga}}
}

\vspace*{2ex}

\tabcolsep=8pt
\begin{tabular}{|c|c|c|c|c|c|c|c|c|}
\hline
&\multicolumn{8}{c|}{$\Lambda_v$}\\
\cline{2-9}
\multicolumn{1}{|c|}{\raisebox{6pt}[0pt][0pt]{$\tau$}}&1&1{,}1&1{,}2&1{,}3&1{,}4&1{,}5&1{,}6&1{,}7\\
\hline
\hphantom{9}1&0.0001&0,0001&0,0017&0,0063&0,0210&0,0733&0,1996&0,4222\\
\hphantom{9}5&0.0000&0,0002&0,0016&0,0036&0,0446&0,0159&0,1360&0,3273\\
10&0.0000&0,0002&0,0011&0,0036&0,0101&0,0430&0,0818&0,2774\\
20&0.0000&0,0003&0,0007&0,0029&0,0089&0,0257&0,0863&0,2045\\
     \hline
\end{tabular}
\end{center}
\end{table}


\begin{multicols}{2}


\noindent
числа каналов в линии при равной суммарной 
производительности. Кроме того, как видно из табл.~\ref{t5aga} и~\ref{t6aga}, 
вероятность блокировки в большей степени зависит от среднего значения 
длины интервала между повторными попытками передачи, чем от закона 
распределения длины интервала. Таким образом, предложенный в работе 
алгоритм позволяет вы\-чис\-лить с достаточной точностью вероятность 
блокировки узла, интенсивности повторных передач и предельную величину 
реализуемой нагрузки. Отметим, что полученные в данной статье результаты 
могут быть использованы для расчета нагрузок в телекоммуникационной сети с 
повторами заявок в предыдущем узле или из источника. 


{\small\frenchspacing
{%\baselineskip=10.8pt
\addcontentsline{toc}{section}{Литература}
\begin{thebibliography}{99}    
\bibitem{1aga}
\Au{Kamoun~F., Kleinrock~L.}
Analysis of shared finite storage in a computer networks node environment under 
general traffic conditions~// IEEE Trans. on Commun., 1980. Vol.~28. No.\,7. 
P.~992--1003.

\bibitem{6aga} %2
\Au{Агаларов~Я.\,М., Шоргин~С.\,Я.}
Рекуррентный метод вычисления параметров сетей связи~// Техника средств 
связи, 1986. Сер. <<Системы связи>>. Вып.~6. С.~42--46.

\bibitem{3aga}
\Au{Башарин Г.\,П., Бочаров~П.\,П., Коган~Я.\,А.}
Анализ очередей в вычислительных сетях.~--- М.: Наука, 1989. 

\bibitem{4aga}
\Au{Бочаров~П.\,П., Печинкин~А.\,В.}
Теория массового обслуживания.~--- М.: Изд-во РУДН, 1995. 

\bibitem{5aga}
\Au{Вишневский~В.\,М.} 
Теоретические основы проектирования компьютерных сетей.~--- М.: 
Техносфера, 2003. 

\bibitem{2aga} %6
\Au{Башарин Г.\,П.}
Лекции по математической теории телетрафика.~--- М.: Изд-во РУДН, 2007. 

\bibitem{7aga}
\Au{Таранцев~А.\,А.}
Инженерные методы теории массового обслуживания.~--- М.: Наука, 2007.

\bibitem{9aga} %8
\Au{D'Apice~C., De~Simone~T., Manzo~R., Rizelian~G.}
$M\vert G\vert 1\vert r$ retrial queueing system with priority service of primary 
customers and a customers-searching server~// Distributed Computer and 
Communication Networks. Stochastic Modelling and Optimization.~--- М.: 
Техносфера, 2003. P.~106--117.

\bibitem{8aga} %9
\Au{Klimenok~V.\,I., Kim~C.\,S.}
$BM\!AP$/$PH$/1 retrial system operating in random environment~// Proceedings of 
the 5th St.-Petersburg Workshop on Simulation, St.-Petersburg, June~26\,--\,July~2, 
2005.~--- St.-Petersburg: NII Chemistry St.-Petersburg University Publs., 
2005. P.~367--372.   

\bibitem{10aga}
\Au{Krishnamoorthy~A., Babu~S.}
$M\!AP\vert (PH,PH)/c$ retrial queue with selegeneration of priorities 
and non-preemptive service~// Proceedings of the 14th International Conference on 
Analytical and Stochastic Modeling Techniques and Applications, June~4--6, 
2007. Prague, Czech Republic.~--- Sbr.-Dudweiler: Digitaldruck Pirrot GmbH, 
2007. P.~70--74.

\bibitem{11aga}
\Au{Корн~Г., Корн~Т.}
Справочник по математике.~--- М.: Наука, 1974.

\label{end\stat}


\bibitem{12aga}
\Au{Buzen~J.\,P.}
Computational algorithm for closed queuing networks with exponential servers~// 
Communications ACM, 1973. Vol.~16. No.\,9. P.~527--531.
 \end{thebibliography}
}
}
\end{multicols}
 
 
      %8
\def\stat{kudr}

\def\tit{ПРИБЛИЖЕННЫЕ МЕТОДЫ РЕШЕНИЯ ЗАДАЧИ ДИАГНОСТИКИ ПЛОСКИМ 
ЗОНДОМ СИЛЬНОИОНИЗОВАННОЙ ПЛАЗМЫ С~УЧЕТОМ КУЛОНОВСКИХ 
СТОЛКНОВЕНИЙ}

\def\titkol{Приближенные методы решения задачи диагностики плоским 
зондом сильноионизованной плазмы} %с~учетом Кулоновских  столкновений}

\def\autkol{И.\,А.~Кудрявцева, А.\,В.~Пантелеев}
\def\aut{И.\,А.~Кудрявцева$^1$, А.\,В.~Пантелеев$^2$}

\titel{\tit}{\aut}{\autkol}{\titkol}

%{\renewcommand{\thefootnote}{\fnsymbol{footnote}}\footnotetext[1]
%{Работа поддержана Российским фондом фундаментальных исследований
%(проекты 11-01-00515а и 11-07-00112а), а также Министерством
%образования и науки РФ в рамках ФЦП <<Научные и
%научно-педагогические кадры инновационной России на 2009--2013~годы>>.}}


\renewcommand{\thefootnote}{\arabic{footnote}}
\footnotetext[1]{Московский авиационный институт, irina.home.mail@mail.ru}
\footnotetext[2]{Московский авиационный институт, avpanteleev@inbox.ru}

\vspace*{-2pt}

\Abst{Сформирована математическая модель, описывающая динамику сильноионизованной 
плазмы с учетом столкновений заряженных частиц вблизи плоского зонда. Модель включает уравнение 
Фоккера--Планка и уравнение Пуассона. Предложено два подхода к решению задачи: на основе метода 
статистических испытаний Мон\-те-Кар\-ло и на основе композиции метода крупных частиц и метода 
расщепления.} 

\vspace*{-2pt}

\KW{телекоммуникационные системы; метод Монте-Карло; метод крупных частиц; метод 
расщепления; зонд; уравнение Фоккера--Планка; уравнение Пуассона} 

\vspace*{-4pt}

 \vskip 8pt plus 9pt minus 6pt

      \thispagestyle{headings}

      \begin{multicols}{2}
      
            \label{st\stat}

\section{Введение}

В настоящее время в области телекоммуникаций все более востребованными становятся 
информационные технологии, основанные на использовании математических моделей и численных 
методов физики плазмы. Поэтому особенно актуальным является решение разнообразных задач анализа 
поведения плазмы, включающих в себя формирование новых моделей и методов их исследования. 
Помимо этого, в разработке телекоммуникационного оборудования эффективно используются 
собственно физические свойства плазмы. В~частности, изготовлена антенна, работа которой основана 
на газовом разряде низкотемпературной плазмы~[1], интенсивно ведутся разработки по созданию и 
усовершенствованию источников бесперебойного питания на основе плазменных элементов~[2, 3]. 
      
      Одним из наиболее перспективных направлений для построения систем оптической 
беспроводной связи является использование лазеров~\cite{4-k, 5-k}. В~этой связи большое внимание 
уделяется использованию плазмы при разработке импульсных сильноточных коммутаторов~\cite{6-k}, 
так как практическое применение подобных разработок требует повышения уровня надежности и 
быстродействия лазерных систем.
      
      Исследования низкотемпературной плазмы также связаны с разработками в области дальней 
космической связи, так как моделирование процессов взаимодействия заряженного тела с верхними 
слоями атмосферы позволяет предлагать способы улучшения существующих систем радиосвязи с 
космическими летательными аппаратами~\cite{7-k}. 
      
      Наряду с этим актуальными также являются задачи диагностики плазмы, поскольку перспективы 
ее использования в области телекоммуникаций после более полного изучения физических свойств 
могут значительно расшириться. 

Для диагностики плазмы применяют зондовые методы исследования~[8--11]. Эти методы относятся к 
классу контактных методов; как следствие, возникает сложность в исследовании пристеночной области 
вблизи зонда, которая характеризуется достаточно сложным распределением потенциала и функциями 
распределения, отличными от максвелловских. 

Данная работа посвящена исследованию переходного режима обтекания заряженного тела плазмой. Для 
переходного режима выполняется следующее условие: длина свободного пробега иона до столкновения 
с нейтральным атомом или другим ионом невелика по сравнению с характерными размерами тела. 
В~этом случае возникает необходимость учета столкновений заряженных частиц с нейтральными 
атомами и кулоновских столкновений. В~работах~\cite{10-k, 11-k} подробно рассмотрена модель с 
учетом столкновений заряженных частиц с нейтральными атомами. В~настоящей статье представлена 
теоретическая модель, описывающая влияния ион-ионных и ион-элек\-т\-рон\-ных столкновений на 
измеряемые характеристики плазмы, что ранее детально не исследовалось.
      
      В~рамках данной работы предлагается модель, описывающая динамику сильноионизованной 
плазмы с учетом кулоновских столкновений. Эта модель учитывает такие процессы взаимодействия, 
как перенос частиц и столкновения между заряженными частицами типа <<ион--ион>> и 
      <<ион--электрон>> под влиянием макроскопического электрического поля. Перечисленные 
процессы описываются самосогласованной системой уравнений, включающей уравнение 
      Фок\-ке\-ра--План\-ка и уравнение Пуассона~[12].
      
      Вычислительная модель задачи строится на основе двух методов: метода статистических 
испытаний Мон\-те-Кар\-ло и композиции метода крупных частиц и метода расщепления. Приведены 
результаты численного моделирования, полученные с использованием вышеперечисленных методов.

\vspace*{-4pt}

\section{Постановка задачи}

\vspace*{-2pt}

Рассматривается следующая физическая постановка зондовой задачи~[11]. В~невозмущенную 
бесконечно протяженную плазму, состоящую из электронов и однозарядных ионов, внесена большая\linebreak 
заряженная до потенциала $\varphi_p$ плоскость. Плоскость, расположенная поперек потока плазмы, 
является идеально поглощающей для электронов. Ионы при ударе о плоскость нейтрализуются. 
Предполагается, что частицы в плазме движутся под действием внешнего электрического поля, 
магнитное поле отсутствует. Концентрации ионов $n_{i\infty}$ и электронов $n_{e\infty}$, а также 
температуры данных час\-тиц~$T_{i\infty}$ 
и~$T_{e\infty}$ в невозмущенной плазме заданы. За начальные 
функции распределения обоих типов час\-тиц принимаются функции распределения Максвелла. 
      
      Требуется с учетом столкновений между заряженными частицами найти напряженность 
самосогласованного электрического поля $\vec{E}(\vec{r},t)$, функции распределения однозарядных 
ионов $f_i(\vec{r}, \vec{v}, t)$ и электронов $f_e(\vec{r}, \vec{v}, t)$, 
а также их моменты (плотности 
токов ионов и электронов  $j_i(\vec{r},t)\hm
=q\int f_i(\vec{r}, \vec{v}, t)\vec{v}\,d\vec{v}$, $j_e(\vec{r},t) 
\hm={\sf e}\int f_e(\vec{r},\vec{v},t)\vec{v}\,d\vec{v}$, где $q=Z_i{\sf e}$, $Z_i=1$~--- заряд иона, ${\sf 
e}$~--- заряд электрона; концентрации ионов и электронов $n_i(\vec{r},t)\hm=\int 
f_i(\vec{r},\vec{v},t)\,d\vec{v}$, $n_e(\vec{r},t)\hm=\int f_e(\vec{r},\vec{v}, t)\,d\vec{v}$). 
Поведение частиц во 
времени~$t$ характеризуется ра\-ди\-ус-век\-то\-ром~$\vec{r}$ и вектором скорости~$\vec{v}$.
      
      Математическая модель, соответствующая данной физической постановке задачи, имеет 
вид~\cite{11-k, 13-k}:

\noindent
      \begin{equation}
      \left.
      \begin{array}{c}
      \fr{\partial f_\alpha (\vec{r},\vec{v},t)}{\partial t}+
      \vec{v}\fr{\partial f_\alpha (\vec{r},\vec{v},t)}{ 
\partial \vec{r}}+
\fr{\vec{F}_\alpha(\vec{r},t)}{m_\alpha}\times{}\\[4pt]
{}\times\fr{\partial f_\alpha(\vec{r},\vec{v},t)}{ \partial 
\vec{v}}=
\left(\fr{\partial f_\alpha(\vec{r},\vec{v},t)}{ \partial t}\right)_{\mathrm{с}}+S_\alpha 
(\vec{r},\vec{v},t)\,;\\[6pt]
      \Delta\varphi(\vec{r},t)=-\fr{{\sf e}}{\varepsilon_0}\left( n_i(\vec{r},t)-n_e(\vec{r},t)\right)\,;\\[6pt]
      \vec{E}(\vec{r},t)=-\nabla \varphi(\vec{r},t)\,.
      \end{array}\!\!
      \right\}\!\!
      \label{e1-k}
      \end{equation}
Здесь первое уравнение~--- уравнение Фок\-ке\-ра--План\-ка для частиц сорта~$\alpha$ ($\alpha=i,e$), 
второе~--- уравнение Пуассона для самосогласованного электрического поля; 
$f_\alpha(\vec{r},\vec{v},t)$~--- функция\linebreak
распределения час\-тиц сорта~$\alpha$; $(\partial 
f_\alpha(\vec{r},\vec{v},t)/\partial t)_{\mathrm{с}}$~--- 
оператор столкновений Фок\-ке\-ра--План\-ка; 
функция~$S_\alpha(\vec{r},\vec{v},t)$ описывает источники или стоки\linebreak
 час\-тиц; 
$\vec{F}_\alpha(\vec{r},t)=q_\alpha\vec{E}(\vec{r},t)$, где $\vec{E}(\vec{r},t)$~--- напряженность 
самосогласованного электрического поля, 
$$
q_\alpha =
\begin{cases}
-{\sf e}\,, & \alpha=e\,,\\
{\sf e}\,, & \alpha=i\,;
\end{cases}
$$
$\varphi(\vec{r},t)$~--- потенциал самосогласованного электрического поля; $n_\alpha(\vec{r},t)$ ($\alpha 
\hm=i,e$)~--- концентрация частиц сорта~$\alpha$; $m_\alpha$~--- масса частицы сорта~$\alpha$; 
$\varepsilon_0$~--- электрическая постоянная. 

Оператор столкновений Фок\-ке\-ра--План\-ка имеет вид~\cite{13-k, 14-k}
\begin{multline*}
\fr{1}{\Gamma_\alpha}\left( \fr{\partial f_\alpha}{\partial t}\right)_{\mathrm{с}} 
=\fr{1}{2}\,\nabla_v\nabla_v:\left(f_\alpha\nabla_v\nabla_vg_\alpha(\vec{r},\vec{v},t)\right)-{}\\
{}-
\nabla_v\cdot\left(f_\alpha\nabla_v h_\alpha\right)\,,
\end{multline*}
где $\nabla_v\nabla_v g_\alpha(\vec{r},\vec{v},t)$~--- ковариантная тензорная производная второго ранга, 
знак двоеточия ($:$) обозначает операцию двойного суммирования:
\begin{gather*}
\Gamma_\alpha=\fr{Z_\alpha^4 {\sf e}^4}{4\pi \varepsilon_0^2 m^2_\alpha}\,\ln D_\alpha\,;
\\
D_\alpha =\fr{12\pi\varepsilon_0 kT_{\alpha\infty}}{Z_\alpha^2 {\sf e}^2}\left( \fr{\varepsilon_0 k 
T_{e\infty}}{n_{e\infty} {\sf e}^2}\right)^{1/2}\,;\\
g_\alpha (\vec{r},\vec{v},t)=\sum\limits_{b=i,e}\left( \fr{Z_b}{Z_\alpha}\right) \int f_b 
(\vec{r},{\vec{v}}^{\,\prime},t)\left\vert \vec{v}-{\vec{v}}^{\,\prime}\right\vert\,d\vec{v}^{\,\prime}\,;\\
h_\alpha (\vec{r},\vec{v},t)=\sum\limits_{b=i,e} \fr{m_\alpha+m_b}{m_b} 
\left(\fr{Z_b}{Z_\alpha}\right)
\int
\fr{f_b(\vec{r},{\vec{v}}^{\,\prime}, t)}{\vert \vec{v}-{\vec{v}}^{\,\prime}\vert}
\,d{\vec{v}}^{\,\prime}\,;\\
Z_\alpha =1\,, \quad \alpha=i,e\,.
\end{gather*}
 
К системе уравнений~(\ref{e1-k}) необходимо добавить начальные и краевые условия:
\begin{equation}
\!\left.
\begin{array}{rrl}
t=0:\ & f_\alpha(\vec{r},\vec{v},0)&=f_\alpha^{\mathrm{maksv}}\,,\enskip \alpha=i,e;\\[9pt]
\vec{r}\in \Omega_p:\ & f_\alpha(\vec{r},\vec{v},t)\big\vert_{\vec{r}\in\Omega_p}&=0\,,\enskip \alpha=i,e\,;\\[9pt]
&\varphi(\vec{r},t)\big\vert_{\vec{r}\in\Omega_p}&=\varphi_p\,;\\[9pt]
\vec{r}\in\Omega_\infty:\ & 
f_\alpha(\vec{r},\vec{v},t)\big\vert_{\vec{r}\in\Omega_\infty}&= %{}\\[9pt]
f_\alpha^{\mathrm{maksv}}\,,\enskip \alpha=i,e\,;\\[9pt]
&\varphi(\vec{r},t)\big\vert_{\vec{r}\in\Omega_\infty}&=0\,,
\end{array}\!\!
\right\}\!\!\!\!
\label{e2-k}
\end{equation}
    где 
    
    \noindent
    \begin{multline*}
    f_\alpha^{\mathrm{maksv}}=n_{\alpha\infty}\left(\fr{m_\alpha}{2k\pi T_{\alpha\infty}}\right)^{3/2}\times{}\\
    {}\times
    \exp\left( -
\fr{m_\alpha}{2kT_{\alpha\infty}}\left\vert\vec{v}-\vec{v}_\infty\right\vert^2\right)\,,
\enskip \alpha=i, e\,;
\end{multline*} 
$\Omega_p$ и $\Omega_\infty$~--- множество радиус-векторов час\-тиц, концы которых принадлежат плоскости зонда и 
границе возмущенной зоны соответственно.

Для решения поставленной задачи введем декартову систему координат таким образом, чтобы 
заряженная плоскость совпала с плоскостью~$0xz$. Тогда положение частицы в пространстве будет 
определяться координатами $x,y,z$, а скорость~--- координатами $v_x, v_y, v_z$. В~силу того что 
плоскость является бесконечно большой в сравнении с характерным размером задачи, функции 
распределения частиц будут зависеть только от переменных $y, v_y, t$.

Поставленную задачу предлагается решать независимо двумя методами. Первый метод основывается на 
методе статистических испытаний Мон\-те-Кар\-ло, второй метод является композицией метода 
расщепления и метода крупных частиц.

\section{Применение метода Монте-Карло}

Запишем самосогласованную систему уравнений~(\ref{e1-k}) и~(\ref{e2-k}) в декартовой системе 
координат с учетом сделанных предположений:
\begin{equation}
\left.
\begin{array}{l}
\fr{\partial f_\alpha}{\partial t}+
v_y\fr{\partial f_\alpha}{\partial y}+\fr{F_y^\alpha}{m_\alpha}\,\fr{\partial 
f_\alpha}{\partial v_y}=\fr{1}{2}\,\fr{\partial^2 }{\partial [v_y]^2}\times{}\\
{}\times \left( 
f_\alpha\fr{\partial^2 g_\alpha  }{\partial [v_y]^2}\right) -
\fr{\partial}{\partial v_y}\left( f_\alpha\fr{\partial h_\alpha}{\partial v_y}\right)\,,
\enskip \alpha=i,e\,;\\[6pt]
    \fr{\partial^2\varphi}{\partial y^2} =-\fr{{\sf e}}{\varepsilon_0}\left(n_i-n_e\right)\,;
    \enskip E_y=-
\fr{\partial\varphi}{\partial y}\,;\\[6pt]
\hspace*{3.1mm}    t=0:\  \hspace*{2.6mm}f_\alpha(y,v_y,0)=f_\alpha^{\mathrm{maksv}}\,,\ \alpha=i,e\,;\\[9pt]
\hspace*{2.9mm} y=0:\ \hspace*{2.8mm}f_\alpha(0,v_y,t)=0\,,\ \alpha=i,e\,;\\[9pt]
\hspace*{24.3mm}\varphi(0,t)=\varphi_p\,;\\[9pt]
y=y_\infty:\ f_\alpha(y_\infty, v_y, t)=f_\alpha^{\mathrm{maksv}}\,,\ \alpha=i,e\,;\\[9pt]
\hspace*{21.5mm}\varphi(y_\infty, t)=0\,.
\end{array}
\right \}
\label{e3-k}
\end{equation}

В полученной системе уравнений~(\ref{e3-k}) перейдем к безразмерным величинам, применив 
соотношение $X=M_X \hat{X}$, где $M_X$~--- масштаб размерной величины~$X$, $\hat{X}$~--- 
безразмерная величина~$X$. В~качестве используемых масштабов были взяты следующие: радиус 
Дебая, скорость теплового движения частиц, концентрация частиц в невозмущенной плазме, потенциал, 
возникающий при разделении зарядов в дебаевской сфере, и производные от них величины.

Система безразмерных уравнений имеет следующий вид:
%\noindent
\begin{equation}
\left.
\begin{array}{l}
\fr{\partial 
\hat{f}_\alpha}{\partial\hat{t}}+A_\alpha\fr{\partial\hat{f}_\alpha}{\partial\hat{y}}+
B_\alpha\hat{E}_y\fr{\partial\hat{f}_\alpha}{\partial \hat{v}_y}={}\\
\!{}=
\fr{\partial^2}{\partial[\hat{v}_y]^2}\left(D_\alpha 
\hat{f}_\alpha\right)-\fr{\partial}{\partial\hat{v}_y}\left(K_\alpha \hat{f}_\alpha\right),\enskip 
\alpha=i,e;\\[9pt]
\fr{\partial^2\hat{\varphi}}{\partial\hat{y}^2}=-\left(\hat{n}_i-\hat{n}_e\right)\,;\enskip \hat{e}_y=-
\fr{\partial\hat\varphi}{\partial\hat{y}}\,;\\[9pt]
\hspace*{3.1mm}\hat{t}=0:\ \hspace*{2.6mm}\hat{f}_\alpha(\hat{y},\hat{v}_y,0)=\hat{f}_\alpha^{\mathrm{maksv}}\,,\enskip \alpha-i,e\,;\\[9pt]
\hspace*{2.9mm}\hat{y}=0:\ \hspace*{2.8mm}\hat{f}_\alpha(0,\hat{v}_y,\hat{t})=0\,,\enskip \alpha=i,e\,;\\[9pt]
\hspace*{24.3mm}\hat\varphi(0,\hat{t})=\hat{\varphi}_p\,;\\[9pt]
\hat{y}=\hat{y}_\infty:\ \hat{f}_\alpha(\hat{y}_\infty, \hat{v}_y, \hat{t})=\hat{f}^{\mathrm{maksv}}_\alpha\,,\enskip 
\alpha=i,e\,;\\[9pt]
\hspace*{21.5mm}\hat\varphi(\hat{y}_\infty,\hat{t})=0\,.
\end{array}
\right\}
\label{e4-k}
\end{equation}
Здесь 

\vspace*{-2pt}

\noindent
\begin{gather*}
A_\alpha=\sqrt{\delta_\alpha }\,\hat{v}_y\,;\enskip 
B_\alpha=\sqrt{\delta_\alpha}\,\fr{z_\alpha}{2\varepsilon_\alpha}\,;\\
\delta_\alpha=\fr{\varepsilon_\alpha}{\mu_\alpha}\,;\enskip 
\varepsilon_\alpha=\fr{T_{\alpha\infty}}{T_{i\infty}}\,;\\
\mu_\alpha=\fr{m_\alpha}{m_i}\,;\enskip 
D_\alpha=A_g^\alpha\fr{\partial^2\hat{g}_\alpha}{\partial  [\hat{v}_y]^2}\,;\\
K_\alpha=A_h^\alpha \fr{\partial \hat{h}_\alpha}{\partial \hat{v}_y}\,,\enskip \alpha=i,e\,,
\end{gather*}
где $A_g^\alpha$ и $A_h^\alpha$~--- коэффициенты, определяемые характерными параметрами 
задачи~\cite{15-k}.

Поиск решения самосогласованной системы уравнений~(\ref{e4-k}) осуществляется по следующей 
схе-\linebreak ме. Вначале находятся значения напряженности\linebreak
 электрического поля по значениям потенциала, 
полученным из граничной задачи для уравнения Пуассона. Далее, используя найденные значения 
напряженности, решается уравнение Фок\-ке\-ра--План\-ка путем перехода к стохастическому 
дифференциальному уравнению (СДУ) Ито:

\noindent
\begin{multline*}
d\Theta_\alpha(\hat{t}) = a_\alpha \left(\hat{t},\Theta_\alpha(\hat{t})\right)+{}\\
{}+\sigma\left(
\hat{t},\Theta_\alpha(\hat{t})\right)\,dW(\hat{t})\,,\quad \alpha=i,e\,,
%\label{e5-k}
\end{multline*}
где 

\noindent
\begin{align*}
\Theta_\alpha(\hat{t})&=\begin{bmatrix}
\hat{y}(\hat{t})\\ \hat{v}_y(\hat{t})
\end{bmatrix}\,;\\
a_\alpha\left(\hat{t},\Theta_\alpha(\hat{t})\right)&=\begin{bmatrix}
-A_\alpha\\ -K_\alpha -B_\alpha \hat{E}_y
\end{bmatrix}\,;\\
\sigma_\alpha\left(\hat{t},\Theta_\alpha(\hat{t})\right)\sigma_\alpha^{\mathrm{T}}\left( 
\hat{t},\Theta_\alpha(\hat{t})\right)&=D_\alpha\,,\enskip \alpha=i,e\,;
\end{align*} 
$W(\hat{t})$~--- стандартный винеровский случайный процесс.
\pagebreak

Для нахождения значений вектора состояния~$\Theta_\alpha(\hat{t})$ применим явную разностную 
схему стохастического метода Эйлера~\cite{16-k}:
\begin{multline*}
\Theta_\alpha^{n+1}=\Theta_\alpha^n +h_\tau a_\alpha \left( \hat{t}_n, \Theta_\alpha^n\right)+\sigma_\alpha 
\left( \hat{t}_n, \Theta_\alpha^n\right)\Delta W_n\,,\\ 
n=0,\ldots , N\,,\ \alpha=i,e\,,
%\label{e6-k}
\end{multline*}
где $\Theta_\alpha^n$, $n=0,\ldots , N$,~--- приближенное значение вектора 
состояния~$\Theta_\alpha(\hat{t})$, $\alpha=i,e$, в момент времени $\hat{t}\hm=\hat{t}_n$, 
$\hat{t}_n\hm=n h_\tau$, $n=0,\ldots , N$; $h_\tau$~--- достаточно малый шаг интегрирования; $\Delta 
W_n$, $n=0,\ldots ,N$,~--- величина приращения винеровского процесса~$W(\hat{t})$ на отрезке $\left[ 
\hat{t}_n,\,\hat{t}_{n+1}\right]$, по определению независимая от~$\Theta_\alpha^0$, 
$\Delta W_0,\ldots , 
\Delta W_{n-1}$: $\Delta W_n\hm=W(\hat{t}_{n-1})\hm-W(\hat{t}_n)$; $\Delta W_n\hm\sim N(0,\,h_\tau)$, 
т.\,е.\ $\Delta W_n$ представляют собой гауссовские случайные величины с нулевыми математическими 
ожиданиями и дисперсиями, равными шагу интегрирования; $\Theta_\alpha^0$~--- значение вектора 
состояния $\Theta_\alpha(\hat{t})$, $\alpha\hm=i,e$, в момент времени $\hat{t}=0$, 
$\Theta_\alpha^0\hm\sim \hat{f}_\alpha^{\mathrm{maksv}}$. 

Частные производные $\partial^2\hat{g}_\alpha/\partial[\hat{v}_y]^2$ и $\partial \hat{h}_\alpha/\partial 
\hat{v}_y$, являющиеся составляющими матрицы $\sigma_\alpha (\hat{t}_n, 
\Theta_\alpha^n)\sigma_\alpha^{\mathrm{T}}(\hat{t}_n,\Theta_\alpha^n)$ и вектора $a_\alpha(\hat{t}_n, 
\Theta_\alpha^n)$ соответственно, аппроксимируются со вторым порядком точности на трехточечном 
шаблоне на основе значений~$\hat{g}_\alpha$ и~$\hat{h}_\alpha$~\cite{17-k}.
      
      В выражения для функций~$\hat{g}_\alpha$ и~$\hat{h}_\alpha$ входят интегралы, которые 
вычисляются методом Мон\-те-Кар\-ло с использованием набора значений скоростной компоненты 
вектора состояния~$\hat{v}_y$, полученных из решения СДУ Ито:
      \begin{equation*}
      \int \hat{f}_\alpha \left\vert \hat{v}_y-
\hat{v}_y^\prime\right\vert\,dv_y^\prime=M\left(\zeta\left(\hat{V}_y\right)\right)\,,
\end{equation*}
где
$$
      \zeta\left(\hat{V}_y\right)=\left\vert \hat{v}_y-\hat{V}_y\right\vert\,,\enskip \hat{V}_y\sim 
\hat{f}_\alpha\,.
  $$
      
      Для вычисления напряженности самосогласованного электрического поля $\hat{E}_y=-
\partial\hat{\varphi}/\partial\hat{y}$, входящей в вектор $a_\alpha(\hat{t}_n, \Theta_\alpha^n)$, необходимо 
аналогично аппроксимировать со вторым порядком точности производную 
$\partial\hat{\varphi}/\partial\hat{y}$ на трехточечном шаблоне с использованием значений 
потенциала~$\hat{\varphi}$~\cite{17-k}. Значения потенциала~$\hat\varphi$ находятся из решения 
уравнения Пуассона. 
      
      Граничную задачу для уравнения Пуассона 
      \begin{align*}
      \fr{\partial^2 \hat\varphi}{\partial \hat{y}^2} & = -\left(\hat{n}_i-\hat{n}_e\right)\,;\\
      \hat{\varphi}\big|_{\hat{y}=0} &=\hat{\varphi}_p\,;\\
      \hat{\varphi}\big|_{\hat{y}_\infty=0} &=0
      \end{align*}
    предлагается решать путем перехода к конечно-разностной системе с последующим ее решением 
методом прогонки~\cite{17-k}:

\noindent
\begin{gather*}
\hat{\varphi}^n_{l-1}+2\hat{\varphi}_l^n+\hat{\varphi}^n_{l+1}=
h_y\hat{\delta}_l^n\,,\enskip l=1,\ldots , 
N_y\,;\\
\hat{\delta}_l^n=-\left( \hat{n}^n_{i,l}-\hat{n}^n_{e,l}\right)\,;\enskip 
\hat{\varphi}_0=\hat{\varphi}_p\,;\enskip \hat{\varphi}_{N_y}=0\,,
\end{gather*}
где $N_y$~--- число шагов по переменной~$\hat{y}$, $h_y$~--- величина шагов разбиения по~$\hat{y}$. 
      
      Концентрации $\hat{n}_\alpha$, $\alpha=i,e$, и плотности токов частиц на зонд~$\hat{f}_\alpha$, 
$\alpha=i,e$, вычисляются согласно описанному выше методу Мон\-те-Карло.

\section{Применение метода расщепления и~метода крупных~частиц}

Решение задачи в данном случае предлагается начать с записи правой части уравнения 
Фок\-ке\-ра--План\-ка в декартовой системе координат в виде:
$$
\mathbf{Q} f_\alpha = \fr{1}{2}\,\fr{\partial^2 f_\alpha}{\partial [v_y]^2}\,\fr{\partial^2 g_\alpha}{\partial 
[v_y]^2}+\fr{\partial f_\alpha}{\partial v_y}\,\fr{\partial C_\alpha}{\partial v_y}+H_\alpha\,,\enskip 
\alpha=i,e\,,
$$  
где 
\begin{align*}
C_\alpha(\vec{r},\vec{v},t)&=
\begin{cases}
\fr{1-\gamma}{Z_i^2}\int\fr{f_e(\vec{r},{\vec{v}}^{\,\prime},t)}{|\vec{v}-{\vec{v}}^{\,\prime} |}\,d{\vec{v}}^{\,\prime}\,, 
&\alpha=i\,;\\[9pt]
\fr{Z_i^2(\gamma-1)}{\gamma}\int \fr{f_i(\vec{r},{\vec{v}}^{\,\prime}, t)}
{|\vec{v}-{\vec{v}}^{\,\prime} 
|}\,d{\vec{v}}^{\,\prime}\,, &\alpha=e\,;
\end{cases} 
\\
H_\alpha&=
\begin{cases}
4\pi \left( \fr{\gamma f_e}{Z_i^2}+f_i\right)f_i\,, & \alpha=i\,;\\[9pt]
4\pi\left(\fr{Z_i^2 f_i}{\gamma}+f_e\right)f_e\,, &\alpha=e\,.
\end{cases}
\end{align*}
Тогда при переходе к безразмерным величинам (см.\ разд.~3) система~(\ref{e1-k}) запишется 
следующим образом:
      \begin{equation}
      \left.
\!\!\begin{array}{l}
      \fr{\partial 
\hat{f}_\alpha}{\partial\hat{t}}+A_\alpha\fr{\partial\hat{f}_\alpha}{\partial\hat{y}}+
B_\alpha  \hat{E}_y
\fr{\partial\hat{f}_\alpha}{\partial\hat{v}_\alpha}=\tilde{\mathbf{Q}}\hat{f}_\alpha\,,\enskip 
\alpha=i,e;\\[9pt]
      \fr{\partial^2\hat{\varphi}}{\partial\hat{y}^2}=-\left( \hat{n}_i-\hat{n}_e\right)\,,\enskip \hat{E}_y=-
\fr{\partial\hat\varphi}{\partial\hat{y}}\,,\\[9pt]
\hspace*{3.1mm}\hat{t}=0:\ \hspace*{2.6mm}\hat{f}_\alpha(\hat{y},\hat{v}_y, 0)=\hat{f}_\alpha^{\mathrm{maksv}}\,,\enskip \alpha=i,e\,,\\[9pt]
\hspace*{2.9mm} \hat{y}=0:\ \hspace*{2.8mm}\hat{f}_\alpha(0,\hat{v}_y,\hat{t})=0\,,\enskip \alpha=i,e\,;\\[9pt]
\hspace*{24.3mm}\hat\varphi(0,\hat{t})=\hat{\varphi}_p\,;\\[9pt]
      \hat{y}=\hat{y}_\infty:\ \hat{f}_\alpha(\hat{y}_\infty, 
\hat{v}_y,\hat{t})=\hat{f}_\alpha^{\mathrm{maksv}}\,,\enskip \alpha=i,e\,;\\[9pt]
\hspace*{21.5mm}\hat{\varphi}(\hat{y}_\infty,\hat{t})=0\,,\\[9pt]
    \end{array}
\right\}\!\!
\label{e7-k}
\end{equation}
где 
\begin{gather*}
\tilde{\mathbf{Q}} \hat{f}_\alpha=D_\alpha\fr{\partial^2\hat{f}_\alpha}{\partial 
[\hat{v}_y]^2}+K_\alpha\fr{\partial\hat{f}_\alpha}{\partial\hat{v}_y}+H_\alpha\,;\\
D_\alpha=A_g^\alpha\fr{\partial^2\hat{g}_\alpha}{\partial [\hat{v}_y]^2}\,;\enskip 
K_\alpha=A_h^\alpha \fr{\partial \hat{h}_\alpha}{\partial\hat{v}_y}\,,\ \alpha=i,e\,.
\end{gather*}

Для решения системы уравнений~(\ref{e7-k}) применяется модификация метода 
расщепления~\cite{17-k}, согласно которой исходная задача разбивается на две вспомогательные. Такое 
разбиение можно осуществить, переписав уравнение Фок\-ке\-ра--План\-ка в следующем виде:
$$
\fr{\partial\hat{f}_\alpha}{\partial\hat{t}} =
\tilde{\mathbf{Q}}_1\hat{f}_\alpha+\tilde{\mathbf{Q}}_2\hat{f}_\alpha\,,
$$
где 
\begin{align*}
\tilde{\mathbf{Q}}_1\hat{f}_\alpha &=-
\left(A_\alpha\fr{\partial\hat{f}_\alpha}{\partial\hat{y}}+
B_\alpha\fr{\partial\hat{f}_\alpha}{\partial\hat{y}}
\right)\,;\\
\tilde{\mathbf{Q}}_2\hat{f}_\alpha 
&=\left(D_\alpha\fr{\partial^2\hat{f}_\alpha}{\partial[\hat{v}_y]^2}+K_\alpha\fr{\partial 
\hat{f}_\alpha}{\partial\hat{v}_y}+H_\alpha\right)\,.
\end{align*}

      Правая часть уравнения Фок\-ке\-ра--План\-ка представляет собой сумму двух операторов, 
первый из которых отвечает за перенос частиц, второй~--- за столкновения заряженных частиц. 
В~результате образуются следующие задачи, которые решаются последовательно:
      \begin{itemize}
\item первая задача:
\begin{align*}
&\fr{\partial w_\alpha(\hat{y},\hat{v}_y,\hat{t})}{\partial\hat{t}} =\mathbf{Q}_1 
w_\alpha(\hat{y},\hat{v}_y,\hat{t})\,,\enskip \alpha=i,e\,;\\[9pt]
&\fr{\partial^2\hat\varphi}{\partial\hat{y}^2}=-\left(\hat{n}_i-\hat{n}_e\right)\,;\enskip
\hat{E}_y=-
\fr{\partial\hat\varphi}{\partial\hat{y}}\,;\\[9pt]
&w_\alpha(\hat{y},\hat{v}_y,\hat{t}^n)=\hat{f}_\alpha(\hat{y},\hat{v}_y,\hat{t}^n)\,,\enskip n=0,\ldots ,N-
1\,;\\[9pt]
&\hspace{2.9mm}\hat{y}=0:\ \hspace*{2.9mm}w_\alpha(0,\hat{v}_y,\hat{t})=0\,,\enskip \alpha=i,e\,;\\[9pt]
&\hspace*{25.1mm}\hat\varphi(0,\hat{t})=\hat{\varphi}_p\,;\\[9pt]
&\hat{y}=\hat{y}_\infty:\ w_\alpha(\hat{y}_\infty, \hat{v}_y, \hat{t})=
\hat{f}_\alpha^{\mathrm{maksv}}\,,\enskip 
\alpha=i,e\,;\\[9pt]
&\hspace*{22.5mm}\hat\varphi(\hat{y}_\infty,\hat{t})=0\,;
\end{align*}
\item вторая задача:
\begin{align*}
\!\!\!\!\!\!\!\fr{\partial s_\alpha(\hat{y},\hat{v}_y,\hat{t})}{\partial \hat{t}} &=\mathbf{Q}_2 
s_\alpha(\hat{y},\hat{v}_y,\hat{t})\,, & \alpha&=i,e\,;\\
\!\!\!\!\!\!\!s_\alpha (\hat{y},\hat{v}_y,\hat{t}^n) &=w_\alpha (\hat{y},\hat{v}_y, \hat{t}^{n+1}),& n&=0,\ldots ,N-
1.
\end{align*}
\end{itemize}

Первая задача представляет собой систему безразмерных уравнений Вла\-со\-ва--Пуас\-со\-на. Для ее 
решения применяется метод крупных частиц~\cite{18-k}. Согласно этому методу решение задачи 
осуществляется путем расщепления на два этапа: на первом этапе не учитываются конвективные члены 
и решение получается обычным интегрированием на неподвижной эйлеровой сетке, а на втором этапе 
рассматривается система, которая описывает перенос частиц в лагранжевой системе координат. Кроме 
того, на первом этапе необходимо решить уравнение Пуассона для получения значений потенциала 
самосогласованного электрического поля. Для этого применяется метод, описанный в разд.~3. 

Вторая задача решается путем перехода к ко\-неч\-но-раз\-ност\-ной сис\-те\-ме. При этом частные 
производные $\partial^2\hat{g}_\alpha/\partial[\hat{v}_y]^2$ и $\partial\hat{h}_\alpha/\partial\hat{v}_y$ 
аппроксимируются со вторым порядком точности с использованием трехточечного шаблона, а 
производная $\partial s_\alpha/\partial\hat{t}$ аппроксимируется на двухточечном шаблоне с первым 
порядком точности~\cite{16-k}. К~полученной системе разностных уравнений предлагается применить 
один из классических методов решения систем линейных уравнений, например метод 
Гаусса~\cite{19-k}.
      
      Решением первой задачи является функция $w_\alpha(\hat{y}, \hat{v}_y, \hat{t}^n)$, 
$n\hm=0,\ldots ,N$, , которая дает начальное условие для второй задачи. Решая вторую задачу, находим 
функцию $s_\alpha(\hat{y},\hat{v}_y,\hat{t}^n)\hm=\hat{f}_\alpha(\hat{y},\hat{v}_y,\hat{t}^n)$, 
$n=1,\ldots ,N$, $\alpha=i,e$, которая определяет решение $\hat{f}_\alpha(\hat{y},\hat{v}_y,\hat{t}^n)$, 
$\alpha=i,e$, исходной системы~(\ref{e7-k}) для рассматриваемых моментов времени $n=1,\ldots ,N$.

Моменты функций распределения $\hat{f}_\alpha$, $\alpha=i,e$, находятся с помощью методов 
численного интегрирования, например метода трапеций~\cite{19-k}.

\section{Результаты численного моделирования}

Для двух описанных выше методов реализованы две отдельные программы в среде {Matlab~7.0}. 
Эти программы позволяют по заданным значениям концентраций и температур частиц $n_{i\infty}$, 
$n_{e\infty}$, $T_{i\infty}$ и~$T_{e\infty}$ в невозмущенной плазме, а также потенциала~$\varphi_p$, 
подаваемого на зонд, изучить эволюцию во времени плотностей тока частиц~$j_i$ и~$j_e$, концентраций 
частиц~$n_i$  и~$n_e$ в произвольной точке пространства в возмущенной зоне, а также динамику 
изменения напряженности~$E_y$ самосогласованного электрического поля во времени и пространстве.

С использованием разработанных программ проведены серии расчетных экспериментов, в которых 
значение концентраций варьировалось в пределах $n_{i\infty} \hm = n_{e\infty}\hm =10^{18}\div 
10^{22}$~м$^{-3}$. Значение температур было выбрано неизменным и равным $T_{i\infty}\hm = 
T_{e\infty}\hm=3000$~K, а значения потенциала, подаваемого на зонд, изменялись в пределах 
$\varphi_p\hm=0\div 2{,}6$~В.

На рис.~1  и~2 приведены графики изменения напряженности самосогласованного электрического
 поля (см.\ рис.~1) и плотности токов ионов (см.\linebreak\vspace*{-12pt}

\pagebreak

\end{multicols}

\begin{figure} %fig1
\vspace*{1pt}
\begin{center}
\mbox{%
\epsfxsize=162.594mm
\epsfbox{kud-1.eps}
}
\end{center}
\vspace*{-9pt}
\Caption{Динамика изменения плотности тока ионов во времени в фиксированной точке возмущенной 
зоны для значений потенциала: \textit{1}~--- $\varphi_p=-6$; 
\textit{2}~--- $\varphi_p=-16$; \textit{3}~--- $\varphi_p=- 30$ 
в случае применения методов Монте-Карло~(\textit{а}) 
и крупных частиц~(\textit{б})}
\end{figure}

\begin{figure} %fig2
\vspace*{1pt}
\begin{center}
\mbox{%
\epsfxsize=162.713mm
\epsfbox{kud-2.eps}
}
\end{center}
\vspace*{-9pt}
\Caption{Динамика изменения напряженности электрического поля во времени в фиксированной точке 
возмущенной зоны для значений потенциала: 
\textit{1}~--- $\varphi_p=-6$; \textit{2}~--- $\varphi_p=-16$; 
\textit{3}~--- $\varphi_p=-30$ в случае применения методов Монте-Карло~(\textit{а}) и
крупных частиц~(\textit{б})
}
\end{figure}

\begin{multicols}{2}

\noindent
 рис.~2) во времени в фиксированной точке пространства 
возмущенной зоны в случае применения обоих разработанных алгоритмов.


На основании полученных результатов можно отметить похожее поведение зависимостей 
напряженности электрического поля и плотности тока от времени в двух рассматриваемых случаях. 
Графики кривых сначала убывают, затем начинают возрастать, выходя в некоторый момент 
времени~$t^\prime$ (момент установления) на стационарные значения. 

Одинаковое поведение 
напряженности и плот\-ности тока можно объяснить из следующих соображений: плотность тока ионов в 
данной области пространства равна произведению концентрации ионов на их направленную скорость и 
на заряд иона. Скорость ионов, в свою очередь, зависит от заряда, массы и напряженности 
электрического поля. 
%\columnbreak

При внесении в плазму отрицательно заряженного зонда возникает электрическое поле, которое 
нарушает квазинейтральность плазмы. Для того чтобы компенсировать действие внешнего 
электрического поля, ионы устремляются к зонду, а электроны~--- от зонда. Это приводит к дисбалансу 
концентраций вблизи зонда и, как следствие, к увеличению разности потенциалов; график 
напряженности электрического поля убывает. Вскоре разделение зарядов компенсирует внешнее 
электрическое поле; график выходит на стационарное значение. 

Также можно отметить, что значения 
напряженности электрического поля и плотности тока частиц на зонд в момент установления для двух 
методов совпадают. 

Момент установления~$t^\prime$ зависит от при\-ме\-ня\-емо\-го метода решения. В~случае метода 
Мон\-те-Кар\-ло $t^\prime=3{,}5\div 4$~ед., а для метода крупных частиц совместно с методом 
расщепления $t^\prime\hm=5\div 5{,}5$~ед. Используя ко\-неч\-но-раз\-ност\-ный метод, можно 
получить динамику изменения функций распределения частиц~$f_\alpha$, $\alpha=i,e$, во времени и 
пространстве. Функции распределения позволяют наглядно представить влияние на картину 
распределения частиц вблизи зонда самой поверхности зонда и электрического поля.

\section{Заключение}
      
      В работе найдено решение задачи диагностики плоским зондом сильноионизованной плазмы с 
учетом столкновений заряженных частиц. Разработана математическая модель исследуемого явления, 
описываемая уравнениями Фок\-ке\-ра--План\-ка и Пуассона. Решение получено двумя методами:\linebreak 
статистическим и ко\-неч\-но-раз\-ност\-ным на основе\linebreak сформированных алгоритмов. Приведены 
резуль-\linebreak таты численного моделирования при различных\linebreak характерных параметрах задачи.
 Из  проведенных 
вычислительных экспериментов вытекает, что искомые величины: напряженность 
электрического поля, плотности токов частиц на зонд, концентрации частиц вблизи зонда~--- как по 
характеру зависимости, так и по числовым значениям совпадают. При применении метода 
      Мон\-те-Кар\-ло момент установления наступает быстрее по сравнению с конечно-разностным 
методом, однако конечно-разностный метод позволяет получить более наглядные результаты.

{\small\frenchspacing
{%\baselineskip=10.8pt
\addcontentsline{toc}{section}{Литература}
\begin{thebibliography}{99}

\bibitem{1-k}
\Au{Alexeff I., Anderson T.}
Experimental and theoretical results with plasma antenna~// IEEE Trans. Plasma Sci., 2006. Vol.~34. 
No.\,2. P.~166--172.

\bibitem{2-k}
\Au{Сысун В.\,И.}
Сильноионизованная низкотемпературная плазма в приборах электронной техники: Методы 
исследования, свойства, применение. Дисс. \ldots д-ра физ.-мат. наук в форме науч. докл.: 
01.04.08.~--- Пет\-ро\-за\-водск, 1996.

\bibitem{3-k}
\Au{Тухас В.\,А.}
Методология создания средств измерений и испытаний на устойчивость к кондуктивным помехам~// 
Мат-лы VI Междунар. симп. по электромагнитной совместимости и 
электромагнитной экологии.~--- СПб., 2005. С.~231--234.

\bibitem{4-k}
\Au{Гудзенко Л.\,И., Яковленко С.\,И.}
Плазменные лазеры.~--- М.: Атомиздат, 1978.  256~с.

\bibitem{5-k}
\Au{Звелто О.}
Принципы лазеров.~--- М.: Мир, 1990.  560~с.

\bibitem{6-k}
\Au{Сысун В.\,И., Хромой Ю.\,Д.}
Расширение канала мощного импульсного разряда в парах ртути~// Электронная техника, 1974. 
Сер.~4. Вып.~10. С.~80--85. 

\bibitem{7-k}
\Au{Винклер Дж.\,Р.}
Искусственные пучки частиц в космической плазме.~--- М.: Мир, 1985.  451~с.

\bibitem{8-k}
\Au{Bernstein I.\,B., Rabinowitz I.\,N.}
Theory of electrostatic probes in low-density plasma~// Phys. Fluids, 1959. Vol.~2. No.\,2. P.~112--121. 

\bibitem{9-k}
\Au{Альперт Я.\,Л., Гуревич А.\,В., Питаевский~Л.\,П.}
Искусственные спутники в разреженной плазме.~--- М.: Наука, 1964.  282~с.

\bibitem{10-k}
\Au{Чан П., Тэлбот Л., Турян~К.}
Электрические зонды в неподвижной и движущейся плазме.~--- М.: Мир, 1978.  202~с.

\bibitem{11-k}
\Au{Алексеев Б.\,В., Котельников В.\,А.}
Зондовый метод диагностики плазмы.~--- М.: Энергоатомиздат, 1989.  240~с.

\bibitem{12-k}
\Au{Пантелеев А.\,В., Кудрявцева И.\,А.}
Формирование математической модели двухкомпонентной плазмы с учетом столкновений 
заряженных частиц в случае плоского зонда~// Теоретические вопросы вычислительной техники и 
программного обеспечения: Межвузовский сб. научн. тр.~--- М.: МИРЭА, 2006. С.~11--21.

\bibitem{13-k}
\Au{Олдер Б.}
Вычислительные методы в физике плазмы.~--- М.: Мир, 1974.  111~с.

\bibitem{14-k}
\Au{Montgomery D.\,C., Tidman D.\,A.}
Plasma kinetic theory.~--- New York, 1964. 

\bibitem{15-k}
\Au{Кудрявцева И.\,А., Пантелеев А.\,В.}
Применение метода Мон\-те-Кар\-ло для анализа поведения двухкомпонентной плазмы с учетом 
столкновений между заряженными частицами~// Теоретические вопросы\linebreak
вычислительной техники и 
программного обеспечения: Межвузовский сб. научн. тр.~--- М.: МИРЭА, 2008. С.~122--128. 

\bibitem{16-k}
\Au{Семенов В.\,В., Пантелеев А.\,В., Руденко~Е.\,А., Бор\-та\-ков\-ский~А.\,С.}
Методы описания, анализа и синтеза нелинейных систем управления.~--- М.: МАИ, 1993.  312~с.

\bibitem{17-k}
\Au{Киреев В.\,И., Пантелеев А.\,В.}
Численные методы в примерах и задачах.~--- М.: Высшая школа, 2006.  480~с.

\bibitem{18-k}
\Au{Белоцерковский О.\,М., Давыдов~Ю.\,М.}
Метод крупных частиц в газовой динамике. Вычислительный эксперимент.~--- М.: Наука, 
Физматгиз, 1982.

\label{end\stat}

\bibitem{19-k}
\Au{Вержбицкий В.\,М.}
Основы численных методов.~--- М.: Высшая школа, 2002.  840~с.
 \end{thebibliography}
}
}


\end{multicols}                %9
%\DeclareMathOperator{\mathrm{tr}\,}{tr}
\def\stat{lipatyev}

\def\tit{НЕАСИМПТОТИЧЕСКИЙ АНАЛИЗ СТАТИСТИКИ БАРТЛЕТТА--НАНДА--ПИЛАЯ 
ДЛЯ~ДАННЫХ\\ БОЛЬШОЙ РАЗМЕРНОСТИ}

\def\titkol{Неасимптотический анализ статистики Бартлетта--Нанда--Пилая 
для~данных большой размерности}

\def\aut{А.\,А.~Липатьев$^1$}

\def\autkol{А.\,А.~Липатьев}

\titel{\tit}{\aut}{\autkol}{\titkol}

\index{Липатьев А.\,А.}
\index{Lipatiev A.\,A.}

%{\renewcommand{\thefootnote}{\fnsymbol{footnote}} \footnotetext[1]
%{Работа выполнена при частичной финансовой поддержке РФФИ
%(проекты 18-07-00692, 19-07-00739 и~20-07-00804).}}

\renewcommand{\thefootnote}{\arabic{footnote}}
\footnotetext[1]{Московский государственный университет имени М.\,В.~Ломоносова, 
факультет вычислительной математики и~кибернетики,
кафедра математической статистики, \mbox{allipatev@cs.msu.ru}}


\vspace*{-10pt}


\Abst{Представлены вычислимые оценки скорости сходимости нормированной статистики Барт\-лет\-та--Нан\-да--Пи\-лая 
к~стандартному нормальному распределению при условии, что размерность данных возрастает 
пропорционально объему выборки. Приведенный результат позволяет корректно вычислять 
p-зна\-че\-ния в~прикладных задачах многомерного анализа данных. Задачи в~постановке, когда 
число анализируемых признаков сравнимо с~объемом выборки, все чаще возникают в~об\-ласти 
обработки сигналов. Доказательство базируется существенным образом на нормальности распределения 
элементов рассматриваемых матриц с~распределением Уишарта. Для случайных величин, представляющих 
собой матричные следы произведения и~квадратов мат\-риц с~нормированным распределением Уишарта, 
находятся удобные оценки сверху для $1-F$, где $F$~--- функция распределения соответствующего 
следа мат\-ри\-цы. Применяя свойства обратных мат\-риц и~неотрицательно определенных мат\-риц, статистика 
Барт\-лет\-та--Нан\-да--Пи\-лая ограничивается сверху комбинацией из упомянутых выше следов матриц.}

\KW{точность приближений; многомерный дисперсионный анализ; вычислимые оценки; 
статистика Барт\-лет\-та--Нан\-да--Пи\-лая; данные большой размерности}

\DOI{10.14357/19922264210110}


\vspace*{-2pt}

\vskip 10pt plus 9pt minus 6pt

\thispagestyle{headings}

\begin{multicols}{2}

\label{st\stat}




\section{Введение}
\label{sec:intro}

В большом числе прикладных задач исследователи анализируют многомерные данные, 
в~которых количество~$p$ признаков сравнимо с~числом~$n$\linebreak наблюдений. Для анализа 
данных фиксированной размерности существует множество статистических процедур, 
уже ставших классическими.\linebreak
Однако час\-то нет возможности использовать традиционную статистическую процедуру, 
лишь устремив в~ней чис\-ло признаков к~бесконечности,
так как при этом изменяется предельное распределение статистики критерия  (см.~[1,  
разд.~6.3.4]).

Цель данной работы~--- нахождение вычислимых оценок точности аппроксимации 
статистики Барт\-лет\-та--Нан\-да--Пи\-лая (Bartlett--Nanda--Pillai test) нормальным 
распределением в~модели многомерного дисперсионного анализа (MANOVA~--- multivariate analysis of variance) 
для данных 
большой размерности,
когда отношение числа \mbox{признаков} к~чис\-лу  наблюдений~$p/n$ стремится к~некоторой 
константе из интервала $(0, 1)$.

Результаты, касающиеся распределений статистик, возникающих в~модели MANOVA при 
условии, что нулевая гипотеза верна, оказываются полезны в~области обработки 
сигналов. \mbox{Например}, в~\cite{lit:Johnstone_RoyStat} показано, каким образом 
результаты из MANOVA и~обработки сигналов могут сводиться к~спектру определенной 
матрицы~$E^{-1}H$. В~\mbox{статье}~\cite{lit:Akbari_AppliedLH} приводится пример 
применения статистики Лоу\-ли--Хо\-тел\-лин\-га, родственной статистике 
Барт\-лет\-та--Нан\-да--Пи\-лая, в~контексте обработки данных радаров с~синтезированной апертурой.

В разд.~\ref{sec:results} сформулирован основной результат работы~--- 
теорема~1.
Теорема~2 является   вспомогательной, но при этом 
представляет самостоятельный интерес.
В~разд.~\ref{sec:proofs} даны доказательства основных теорем, которые 
опираются на леммы из разд.~\ref{sec:lemmas}.

%%%%%%%%%%%%%%%%%
\section{Постановка задачи и~основной результат}
\label{sec:results}


В рамках многомерного дисперсионного анализа исследуется следующая многомерная 
линейная модель:
$
X\hm=Q\mathbb{B}\hm+\mathcal{E},
$
где $X$~--- случайная матрица наблюдений размера $N \times p$; $Q$~--- неслучайная 
матрица плана эксперимента размера $N \times k$;
$\mathbb{B}$~---  неслучайная матрица $k \times p$ регрессионных коэффициентов;
$\mathcal{E}$~--- матрица ошибок  $N \times p$ с~распределением $N_{N\times 
p}\left(O,I_{N}\otimes\Sigma\right)$.

Рассмотрим следующую линейную гипотезу:
$
H_{0} : C\mathbb{B}\hm=O,
$
где $C$~--- известная матрица размера $q \times k$ ранга~$q$.  Статистики 
критериев, инвариантные относительно некоторой группы аффинных преобразований, 
оказываются функциями от ненулевых собственных значений матрицы $S_{h}S_{e}^{-1}$, где
\begin{equation}
\left.
\begin{array}{rl}
S_{h}&=\hat{\mathbb{B}}^{\mathrm{T}}C^{\mathrm{T}}\left(C\left(Q^{\mathrm{T}}Q\right) ^{-1}C^{\mathrm{T}}\right)^{-
1}C\hat{\mathbb{B}} \,;\\[6pt]
 S_{e}&=\left(X-Q\hat{\mathbb{B}}\right)^{\mathrm{T}}\left(X-
Q\hat{\mathbb{B}}\right)
\end{array}
\right\}
\label{S_h_and_S_e}
\end{equation}
при $\hat{\mathbb{B}}=\left(Q^{\mathrm{T}}Q\right)^{-1}Q^{\mathrm{T}}X$ (см.~[4, гл.~8]). 
Одной из наиболее известных инвариантных статистик является 
статистика Барт\-лет\-та--Нан\-да--Пи\-лая:
$
V_{\mathrm{BNP}}\hm=\left(n\hm+q\right)\mathrm{tr}\, S_{h}\left(S_{h}\hm+S_{e}\right)^{-1}.
$
В~дальнейшем предполагаем, что гипотеза~$H_0$ верна.

В~\cite{lit:MuirLargeSamp} рассмотрен случай большого объема выборки, т.\,е.\ 
выполнено \textit{условие}~\textbf{А1}:
$$
{\bf A1}: p\mbox{ и~}q\mbox{ фиксированы},\ n\rightarrow\infty ,
$$
и получены неасимптотические  оценки точности аппроксимации функции 
распределения статистики Барт\-лет\-та--Нан\-да--Пи\-лая:
\begin{multline*}
    \mathbf{P}\{V_{\mathrm{BNP}}<x\}=G_{a}\left(x\right)+\fr{3a}{4n}\{G_{a}\left(x\right)-{}\\
{}-2G_{a+2}\left(x\right)+G_{a+4}\left(x\right)\}+O\left(n^{-2}\right),
\end{multline*}
где $a=pq$; $G_{a}$~--- функция $\chi^{2}$-рас\-пре\-де\-ле\-ния
с~$a$~степенями свободы. В~\cite{lit:LU01} для остаточного члена найде\-на 
оценка сверху.

В~\cite{lit:WFU} рассмотрен случай большой размерности данных, т.\,е.\ выполнено 
\textit{условие} \textbf{А2}:
\begin{multline*}
{\bf A2}: q\mbox{ фиксировано},\enskip p\rightarrow\infty,\enskip n\rightarrow\infty,\\ 
\fr{p}{n}\rightarrow c\in\left(0;1\right),
\end{multline*}
и получено следующее приближение:
\begin{multline*}
\mathbf{P}\left(\fr{1}{\sigma}T_{\mathrm{BNP}}<z\right)={}\\
{}=\Phi(z)-
\phi(z)\left[\fr{1}{\sqrt{p}}\left\{\fr{1}{\sigma}b_{1}+\fr{1}{\sigma^{3}}
b_{3}\,H_{2}\left(z\right)\right\} + \right.\\
\left.{}+\fr{1}{p}\left\{\fr{1}{\sigma^{2}}b_{2}\,H_{1}\left(z\right) + 
\fr{1}{\sigma^{4}}b_{4}\,H_{3}\left(z\right) + 
\fr{1}{\sigma^{6}}b_{6}\,H_{5}\left(z\right)\right\}\right]
+{}\\
{}+O\left(\fr{1}{p\sqrt{p}}\right),
\end{multline*}
где   
$T_{\mathrm{BNP}}=\sqrt{p}\left(1+m^{-1}p\right)\left\{p^{-1}V_{\mathrm{BNP}}-q\right\}$;
$\Phi (z)$ и~$\phi (z)$~--- соответственно функция распределения 
и~плотность распределения стандартного нормального закона;
$m\hm=n\hm-p\hm+q$; $r \hm= p/m$;
$\sigma\hm=\sqrt{2q(1+r)}$; $b_{i}\hm=b_{i}(r,q)$ суть некоторые функции от~$r$ 
и~$q$;
$H_{i}(z)$~--- полиномы Эрмита.
При этом результат имел именно асимптотический вид, верхние оценки остаточного 
члена не находились.

Основной результат данной работы~--- две тео\-ре\-мы,
дающие оценку точности аппроксимации распределения статистики Барт\-лет\-та--Нан\-да--Пи\-лая
нормальным распределением для данных большой размерности, т.\,е.\ при выполнении 
условия~\textbf{A2}:


\smallskip

\noindent
\textbf{Теорема~1.}
\textit{При всех $m\hm>M\hm=M\left(r,q\right)$ справедливо неравенство}
    $$
     \sup\limits_{z}{\left|\mathbf{P}\left(\fr{T_{\mathrm{BNP}}}{\sqrt{2q\left(1+r\right)}}<z\right)-
\Phi\left(z\!\right)\right|}\leqslant
\fr{K_{2}\left(r, q\right)\ln m}{\sqrt{m}}\,,
    $$
   \textit{где $K_{2}\left(r, q\right)$~--- вычислимая функция от~$r$ и~$q$}.

\smallskip

Отметим, что результат теоремы~1 на логарифмический 
множитель уступает результату из~\cite{lit:WFU}, но превосходит последний в~том, 
что для ошибки погрешности дается вычислимая оценка сверху. При этом само 
доказательство является новым.

\smallskip

\noindent
\textbf{Теорема~2.}
\textit{Пусть матрицы $U$ и~$V$ суть нормированные варианты матриц~$B$ и~$W$}:
    \begin{equation}
    \label{U and V}
    U=  \fr{B-pI_{q}}{\sqrt{p}}\,;\quad V= \fr{W-mI_{q}}{\sqrt{m}}\,,
    \end{equation}
     \textit{где $B$ и~$W$ независимы и~имеют распределения Уишарта $W_{q}\left(p, 
I_{q}\right)$ и~$W_{q}\left(m, I_{q}\right)$ с~$m\hm=n\hm-p\hm +q$ соответственно.
    Если $\mathrm{tr}{\left(\sqrt{r}\,U\hm+V\right)^{2}}\hm<\left(r\hm+1\right)m$, то выполнено 
следующее неравенство}:

\noindent
    \begin{multline}
    \label{Theor2}
     \left|\sqrt{m}\left(r+1\right)\left(\left(r+1\right)\mathrm{tr}\,{B\left(B+W\right)^{-1}}-
     rq\right)-{}\right.\\
\left.     {}-\left(\sqrt{r}\,\mathrm{tr}\,{U}-r\mathrm{tr}\,{V}\right)
\vphantom{\sqrt{}W^{-1}}
\right|\leqslant{}\\
{}\leqslant
\fr{ \left(r+1\right)\sqrt{r}\left(\left| \mathrm{tr}\,{UV}\right|+
\sqrt{r}\,\mathrm{tr}\,U^{2}\right)}
{ \left(r+1\right)\sqrt{m}-
{\mathrm{tr}{\left(\sqrt{r}\,U+V\right)^{2}}}/{\sqrt{m}}}+{}\\
{}+
     \left(   rq\left(r+1\right)+\fr{\left(\sqrt{r}\,\mathrm{tr}\,{U}-r\mathrm{tr}\,{V}\right)}
     {\sqrt{m}}\right)\times{}\\
     {}\times 
     \fr{\mathrm{tr}{\left(\sqrt{r}\,U+V\right)^{2}}}
     { \left(r+1\right)\sqrt{m}-
{\mathrm{tr}{\left(\sqrt{r}\,U+V\right)^{2}}}/{\sqrt{m}}}.
    \end{multline}



Заметим, что   вероятность события, противоположного событию 
$\mathrm{tr}{\left(\sqrt{r}\,U\hm+V\right)^{2}}\hm<\left(r\hm+1\right)m$, фигурирующему 
в~теореме~2, имеет порядок $O\left( {1}/{\sqrt{m}}\right)$, 
как это станет ясно из результатов разд.~\ref{sec:lemmas}.

\section{Вспомогательные утверждения}
\label{sec:lemmas}


В этой части приведены вспомогательные утверж\-де\-ния, используемые 
в~доказательствах тео\-рем~1 и~2.

Введем дополнительные случайные величины:
\begin{equation}
\left.
\begin{array}{rlrl}
Z_1 &= \mathrm{tr}\,{UV};&\hspace{1cm} Z_3 &= \mathrm{tr}\,{U}-\sqrt{r}\,\mathrm{tr}\,{V};\\[6pt]
Z_2 &= \mathrm{tr}\,{V^{2}}; &\hspace{1cm} Z_4 &= \mathrm{tr}\,{U^{2}},
\end{array}
\right\}
\label{Z_i}
\end{equation}
где случайные матрицы~$U$ и~$V$ определены в~\eqref{U and V}.

\pagebreak


Положим
\begin{equation*}
%\label{def_B}
    B = B(q, r, m) = 4\,\left(q^2 + \sqrt{r}\right)\left(\sqrt{\ln m} + \sqrt{\ln p}\right)^2.
\end{equation*}

Определим также для $i \hm= 1, 2, 3, 4$ и~натуральных~$m$ случайные события $A_{i, m}$ 
как 
$$
A_{i, m} = \left\{\omega : |Z_i(\omega)|\leqslant B \right\}.
$$

Положим
\begin{multline*}
%\label{Z}
Z =
   %\left( 
   \vphantom{\fr{\left|\mathrm{tr}\,{U}-\sqrt{r}\,\mathrm{tr}\,{V}\right|}{\sqrt{m}}}
  \fr{\left(r+1\right)\sqrt{r}\left(\left|\mathrm{tr}\,{UV}\right|+\sqrt{r}\,\mathrm{tr}\,{U^{2}}\right)}
  {\left(r+1\right)\sqrt{m}-{S_{Z}}/{\sqrt{m}}}+{}\\
  {}+
\fr{
          rq\left(r+1\right)+\sqrt{r/m}\,{\left|\mathrm{tr}\,{U}-\sqrt{r}\,\mathrm{tr}\,{V}\right|}
}{\left(r+1\right)\sqrt{m}-{S_{Z}}/{\sqrt{m}}}\,
S_{Z}\,,
\end{multline*}
где $S_{Z}=\left(r\mathrm{tr}\,{U^{2}}+2\sqrt{r}\left|\mathrm{tr}\,{UV}\right|+\mathrm{tr}\,{V^{2}}\right).$

Ясно, что существует натуральное $M_1 \hm= M_1(r, q, c)$ такое, что при всех $m\hm\geq 
M_1$ и~$\omega \hm\in \mathop{\cap}\nolimits_{i=1}^4 A_{i, m}$ выполняется
\begin{multline}
\label{ineq_Z}
Z(\omega) \leqslant
\left(
(r+1)\sqrt{r}\left(1+\sqrt{r}\right)B+
    \left(
    \vphantom{\fr{B}{\sqrt{m}}}
    rq(r+1)+{}\right.\right.\\
\left.\left.    {}+\sqrt{r}\,\fr{B}{\sqrt{m}}\right)B\left(1+\sqrt{r}\right)^{2}\right)\Bigg/
\left( 
\vphantom{\fr{B\left(1+\sqrt{r}\right)^{2}}{\sqrt{m}}}
(r+1)\sqrt{m}-{}\right.\\
\left.{}-\fr{B\left(1+\sqrt{r}\right)^{2}}{\sqrt{m}}\right)
\leqslant{}\\
{}\leqslant 
16\,\fr{\left(1+\sqrt{r}\right)^{2}\,\left(2\sqrt{r}+r\,(rq+q+1)\right)
\left(q^2+\sqrt{r}\right)}{r+1}\times{}\\
{}\times \fr{\left(\ln m + \ln\sqrt{r}\right)}{\sqrt{m}}\,.
\end{multline}

Оценим вероятности $\mathbf{P}(A_{i, m}^c)$ для $i \hm= 1, 2, 3, 4$. Согласно 
леммам~1 и~2 из~\cite{lit:LU02} справедливо сле\-ду\-ющее неравенство:
\begin{multline}
    \label{L-1, f-la}
\mathbf{P}\left(|Z_1| > B\right) + \mathbf{P}\left(Z_2 > B\right) +
 \mathbf{P}\left(|Z_3| > B\right) + {}\\
 {}+
\mathbf{P}\left(Z_4 > B\right) \leqslant 25{,}8\,q^2\,\fr{1+1/\sqrt{r}}{\sqrt{m}}\,.
\end{multline}

\noindent
\textbf{Лемма~1.}
\textit{Пусть случайные величины $T$, $Y$ и~$Z$ определены на одном вероятностном 
пространстве $\left(\Omega,\mathbf{A},\mathbf{P}\right)$,
при этом распределение~$Y$ является абсолютно непрерывным с~ограниченной 
плотностью $f_{Y}\left(z\right)$.
Предположим, что для некоторого события $A\hm\in \mathbf{A}$ при всех $\omega\hm\in A$ 
выполнено следующее соотношение}:
$$
\left|T(\omega)-Y(\omega)\right|\leqslant Z(\omega) \leqslant a
$$
\textit{с некоторой положительной постоянной~$a$.
Тогда
справедливо неравенство}:
\begin{multline}
\label{uniform}
\sup\limits_{x}\left|\mathbf{P}(T<x)-\mathbf{P}(Y<x)\right| \leqslant{}\\
{}\leqslant \mathbf{P}(A^c) + 
a\sup\limits_{x}{f_{Y}\left(x\right)}.
\end{multline}  

\noindent
Д\,о\,к\,а\,з\,а\,т\,е\,л\,ь\,с\,т\,в\,о\ \ леммы~1. См.\ лемму~3 в~\cite{lit:LU02}.~\hfill $\Box$

В следующих двух леммах приводятся два известных результата о скорости 
сходимости в~центральной предельной теореме для независимых одинаково 
распределенных случайных величин.
Первый из результатов относится к~случайным величинам без ограничений на тип 
распределения.
Второй результат относится к~случайной величине с~распределением $\chi^{2}$, 
рассматриваемой как сумма независимых одинаково распределенных 
случайных величин с~известным распределением. Согласно~\cite{lit:Shevtsova1}, справедлива 
следующая лемма.

\noindent
\textbf{Лемма~2.}
\textit{Пусть случайные величины $\xi_{1},\,\xi_{2},\ldots$ независимы и~одинаково 
распределены, выполнено $\mathbf{D}\xi_{1}\hm=\sigma^{2}\hm>0$ и~cуществует 
$\mathbf{E}\left|\xi_{1}\right|^{3}\hm<\infty$. Тогда  для нормированной суммы}
$
T_{n}\hm=
{(S_{n}-\mathbf{E}{S_{N}})}/{\sqrt{\mathbf{D}\,{S_{N}}}}
$
\textit{выполнено неравенство}:
$$
\sup\limits_{x}{\left|F_{T_{n}}\left(x\right)-\Phi\left(x\right)\right|}\leqslant 
0{,}4748\fr{\mathbf{E}\left|\xi_{1}-
\mathbf{E}\xi_{1}\right|^{3}}{\sigma^{3}\sqrt{n}}\,.
$$


Случайная величина с~функцией распределения $G_{p}(x)$, имеющая   $\chi^{2}$-рас\-пре\-де\-ле\-ние 
с~$p$~степенями свободы, может быть представлена в~виде суммы~$p$~независимых 
одинаково распределенных случайных величин с~$\chi^{2}$-рас\-пре\-де\-ле\-ни\-ем с~одной степенью свободы.
Этот факт позволяет дать более точные оценки точ\-ности аппроксимации нормальным 
распределением, чем те, которые можно получить в~общем случае с~помощью
неравенства Бер\-ри--Ес\-се\-ена, а~именно: имеет место следующий результат (см.\ лемму~2 
в~\cite{lit:Kavaguchi} при $\lambda\hm=0{,}5$).

\smallskip

\noindent
\textbf{Лемма~3.}
\textit{Для всех $\lambda\hm\in\left(0; \sqrt{3}-1\right)$ и~целых $p\hm>1$ выполнено}
    $$
    \sup\limits_{x}{\left|G_{p}\left(p+x\sqrt{2p}\right)-
\Phi\left(x\right)\right|}\leqslant \fr{6{,}22}{\sqrt{p}}\,.
    $$

\smallskip

\noindent
\textbf{Лемма~4.}
\textit{Для любых случайных величин $X$ и~$Y$ и~любого действительного числа $a > 
0$ справедливы неравенства}:
    \begin{equation*}
    %\label{split_formula_1}
    \mathbf{P}(|X+Y|\geq 2a) \leqslant \mathbf{P}(|X|\geq a)  + 
\mathbf{P}(|Y|\geq a)\,;
    \end{equation*}
          \begin{equation*}
  %  \label{split_formula_2}
    \mathbf{P}(|X\cdot Y|\geq a^2) \leqslant \mathbf{P}(|X|\geq a)  + 
\mathbf{P}(|Y|\geq a).
    \end{equation*}

\noindent
Д\,о\,к\,а\,з\,а\,т\,е\,л\,ь\,с\,т\,в\,о\ \ леммы~4 очевидным образом вытекает из рассуждений 
от противного.~\hfill $\Box$

\smallskip

\noindent
\textbf{Лемма~5.}
\textit{Если случайные величины~$X_1, \dots , X_k$ независимы и~таковы, что   $|\mathbf{P}(X_j 
\hm\leqslant x) \hm- \Phi (x)| \hm\leqslant D_j$ при всех~$x$ и~$j \hm= 1, \dots , k$
 с~некоторыми постоянными   $D_1, \dots , D_k$, то}
  \begin{equation*}
  \left| \mathbf{P}\left(\sum\limits_{j=1}^{k}c_j\,X_j \leqslant x\right) - \Phi (x)\right| \leqslant 
\sum\limits_{j=1}^{k} D_j,
  \end{equation*}
\textit{где $c_1,\dots , c_k $ суть произвольные постоянные, для которых}   $c_1^2 + 
\dots + c_k^2 \hm= 1$.

\noindent
Д\,о\,к\,а\,з\,а\,т\,е\,л\,ь\,с\,т\,в\,о\ \ леммы~5 см., например, в~теореме~3.1 
в~~\cite{lit:Letters_2006}. \hfill $\Box$



\section{Доказательства теорем~1 и~2}
\label{sec:proofs}


Начнем с~доказательства теоремы~2, поскольку неравенство~\eqref{Theor2} 
является ключевым в~доказательстве 
теоремы~1.

\noindent
Д\,о\,к\,а\,з\,а\,т\,е\,л\,ь\,с\,т\,в\,о\ \ теоремы~2.
Воспользовавшись матричным равенством
\begin{equation*}
%\label{Matr Geom}
\left(I+A\right)^{-1}-\left(I-A\right)=A^{2}\left(I+A\right)^{-1},
\end{equation*}
из определения \eqref{U and V} получаем
\begin{multline*}
\left(B+W\right)^{-1}=\left(\sqrt{p}\,U+pI_{q}+\sqrt{m}\,V+mI_{q}\right)^{-1}={}\\
{}=\fr{1}{p+m}\left(
\vphantom{\fr{\left(\sqrt{p}\,U+\sqrt{m}\,V\right)^{2}}
{p+m}}
I_{q}-
\fr{1}{p+m}\left(\sqrt{p}\,U+\sqrt{m}\,V\right)+{}\right.\\
\left.{}+\fr{\left(\sqrt{p}\,U+\sqrt{m}\,V\right)^{2}}
{p+m}\left(B+W\right)^{-1}\right),
\end{multline*}
далее
\begin{multline*}
%\label{T2_1}
\sqrt{m}\left(r+1\right)\left(\left(r+1\right)B\left(B+W\right)^{-1}-
rI_{q}\right)-{}\\
{}-\left(\sqrt{r}\,U-rV\right)=\fr{\sqrt{r}}{\sqrt{m}}U\left(\sqrt{r}\,U+V\right)+{}\\
{}+
        \fr{1}{\sqrt{m}}\,B\left(\sqrt{r}\,U+V\right)^{2}\left(B+W\right)^{-1}.
\end{multline*}
Отсюда для следов этих матриц имеем следующее неравенство:
\begin{multline}
\label{trBW}
\left|\sqrt{m}\left(r+1\right)\left(\left(r+1\right)\mathrm{tr}\,{B\left(B+W\right)^{-1}}-
rq\right)-{}\right.\\
\left.{}-\left(\sqrt{r}\,\mathrm{tr}\,{U}-r\mathrm{tr}\,{V}\right)
\vphantom{\left(W\right)^{-1}}
\right|
\leqslant{}\\
{}\leqslant \fr{1}{\sqrt{m}}\sqrt{r}\left(\left|\mathrm{tr}\,{UV}\right|+\sqrt{r}\,\mathrm{tr}\,{U^{2}}\right)+{}\\
{}+
\fr{1}{\sqrt{m}}\left|\mathrm{tr}\,{\left[B\left(\sqrt{r}\,U+V\right)^{2}\left(B+W\right)^{-1}\right]}
\right|\leqslant{}\\
{}\leqslant
\fr{1}{\sqrt{m}}\sqrt{r}\left(\left|\mathrm{tr}\,{UV}\right|+\sqrt{r}\,\mathrm{tr}\,{U^{2}}\right)+{}\\
{}+
\fr{1}{\sqrt{m}}\,\mathrm{tr}\,{\left(\sqrt{r}\,U+V\right)^{2}}\mathrm{tr}\,{B\left(B+W\right)^{-1}}.
\end{multline}
Для получения предпоследнего неравенства использованы симметричность 
и~неотрицательная определенность обеих случайных матриц 
$\left(\sqrt{r}\,U+V\right)^{2}$ и~$B\left(B+W\right)^{-1}$, поскольку для 
сим\-мет\-рич\-ных неотрицательно определенных матриц~$X$ и~$Y$ выполнено   
соотношение (см.~\cite{lit:TraceIneq})
$\mathrm{tr}\,{XY}\hm\leqslant\mathrm{tr}\,{X}\mathrm{tr}\,{Y}.$

Видно, что случайная величина $\mathrm{tr}\,{B\left(B+W\right)^{-1}}$ фигурирует в~крайней 
левой и~крайней правой час\-тях неравенства~\eqref{trBW}.
Преобразуя полученное неравенство, получаем, что при 
$\mathrm{tr}\,{\left(\sqrt{r}\,U+\hm V\right)^{2}}\hm<\left(r\hm+1\right)m$ выполнено~\eqref{Theor2}. 
Тем самым доказательство теоремы~2 завершено.
\hfill $\Box$

\smallskip

Переходим к~доказательству теоремы~1.

\smallskip

\noindent
Д\,о\,к\,а\,з\,а\,т\,е\,л\,ь\,с\,т\,в\,о\ \ теоремы~1.
Используя лемму~1 из~\cite{lit:WFU}, перейдем к~представлению статистики 
Барт\-лет\-та--Нан\-да--Пи\-лая
\begin{multline*}
T_{\mathrm{BNP}}={}\\
{}=\sqrt{p}\left(\!1+\fr{p}{m}\!\right)\!\left\{\!\left(1+\fr{m}{p}\right)\mathrm{tr}\!\left[S_{h}
\left(S_{h}+S_{e}\right)^{-1}\right]-q\right\}\hspace*{-4.9pt}
\end{multline*}
в терминах матриц~$B$ и~$W$ размера $q \times q$ вместо матриц~$S_{h}$ и~$S_{e}$ 
размера $p \times p$, где $S_{h}$ и~$S_{e}$ определены в~\eqref{S_h_and_S_e}, 
а~матрицы~$B$ и~$W$ независимы и~имеют распределения Уишарта $W_{q}\left(p, 
I_{q}\right)$ и~$W_{q}\left(m, I_{q}\right)$ с~$m\hm=n\hm-p\hm+q$ соответственно.
При этом будем пользоваться сле\-ду\-ющим соотношением (см.~\cite{lit:WFU}):
$$
\mathrm{tr}\,{S_{h}\left(S_{h}+S_{e}\right)^{-1}}=
\mathrm{tr}\,{B\left(B+W\right)^{-1}}.
$$

Согласно~\eqref{Theor2} для $Z_1$, $Z_2$, $Z_3$ и~$Z_4$ (см.\ определение 
в~\eqref{Z_i}) при $rZ_{4}\hm+2\sqrt{r}\left|Z_{1}\right|\hm+Z_{2} \hm< m$ 
имеем:
\begin{multline*}
%\label{proof_T-1-1}
|\sqrt{r}\,T_{\mathrm{BNP}} - \left(\sqrt{r}\,\mathrm{tr}\,{U}-r\mathrm{tr}\,{V}\right)|={}\\
{}= \left|\sqrt{m}\left(r+1\right)\left(\left(r+1\right)\mathrm{tr}\,{B\left(B+W\right)^{-1}}
-rq\right)-{}\right.\\
\left.{}-\left(\sqrt{r}\,\mathrm{tr}\,{U}-r\mathrm{tr}\,{V}\right)
\vphantom{\left(W\right)^{-1}}
\right|\leqslant{}\\
{}\leqslant
  \left( 
   \vphantom{\fr{\sqrt{r}}{\sqrt{v}}}
   \left(r+1\right)\sqrt{r}\left(\left|Z_{1}\right|+\sqrt{r}\,Z_{4}\right)+{}\right.
 \\
 {}+     \left(
\vphantom{\fr{\sqrt{r}\left|Z_{3}\right|}{\sqrt{m}}}
    rq\left(r+1\right)+\fr{\sqrt{r}\left|Z_{3}\right|}{\sqrt{m}}\right)\left(rZ_{4}+
2\sqrt{r}\left|Z_{1}\right|+{}\right.\\
\left.\left.{}+Z_{2}\right)
\vphantom{\fr{\sqrt{r}}{\sqrt{v}}}
\right)\Bigg/
    \left( \left(r+1\right)\sqrt{m}-
\fr{rZ_{4}+2\sqrt{r}\left|Z_{1}\right|+Z_{2}}{\sqrt{m}}\right).\hspace*{-6.32483pt}
\end{multline*}
Следовательно, в~силу~\eqref{ineq_Z} и~\eqref{uniform} при всех $m\hm\geq M_1$ 
получаем
\begin{multline}
\label{interm}
\mathop{\sup}\limits_{z}\left|\mathbf{P}\left (\fr{T_{\mathrm{BNP}}}
{\sqrt{2q\left(1+r\right)}}<z\right)-{}\right.\\
\left.{}-
\mathbf{P}\left(\fr{\mathrm{tr}\,{U}-\sqrt{r}\,\mathrm{tr}\,{V}}{\sqrt{2q\left(1+r\right)}} < z 
\right)\right| \leqslant{}\\
{}\leqslant \sum\limits_{i=1}^{4} \mathbf{P}\left(|Z_i|> B\right) +  K_{4}(r, q)\,
\fr{\ln m}{\sqrt{m}}\, \sup\limits_{x}{f(x)},
\end{multline}
где $f(x)$ есть плотность случайной величины ${(\mathrm{tr}\,{U}\hm-
\sqrt{r}\,\mathrm{tr}\,{V})}/{\sqrt{2q\left(1\hm+r\right)}}$;
$K_{4}\left(r, q\right)$~--- 
некоторая вычислимая функция от~$r$ и~$q$.

Отметим, что, поскольку матрицы~$B$ и~$W$ независимы, матрицы $U$ и~$V$ также 
независимы между собой. Известно (см., например, гл.~2 в~\cite{lit:FUS01}), что 
$\mathrm{tr}\, B$ и~$\mathrm{tr}\, W$ имеют $\chi^{2}$-рас\-пре\-де\-ле\-ния с~$pq$ и~$mq$ степенями свободы 
соответственно. Известно также, что плотность $\chi^{2}$-распределения с~$k\hm\geq 3$ степенями свободы ограничена сверху величиной
$1/(2\sqrt{\pi\left(k-2\right)}).$
Поэтому для плотности $f(x)$ справедлива равномерная оценка
\begin{multline}
\label{density}
f(x) \leqslant{}\\
\hspace*{-1.5mm}{}\leqslant \min{\left( \fr{\sqrt{p}}{ \sqrt{ \left(pq-
2\right)}},\,\fr{\sqrt{m}}{\sqrt{r\left(mq-2\right)}}\right)} 
\fr{\sqrt{q(1+r)}}{\sqrt{2\pi}}\,.\!\!\!
\end{multline}
Объединяя утверждения лемм~3 и~5 и~соотношения~\eqref{U and V}, \eqref{L-1, f-la}, \eqref{interm} 
и~\eqref{density}, получаем утверждение теоремы~1.\hfill $\Box$%\\[2ex]


%\vspace*{-8pt}

{\small\frenchspacing
{%\baselineskip=10.8pt
%\addcontentsline{toc}{section}{References}
\begin{thebibliography}{99}

%\vspace*{-2pt}
\bibitem{lit:FUS01}
\Au{Fujikoshi Y., Ulyanov~V.\,V., Shimizu~R.}  Multivariate statistics: High-dimensional 
and large-sample approximations.~--- Hoboken, NJ, USA: John Wiley \& Sons, 2010. 512~p.

\bibitem{lit:Johnstone_RoyStat}
\Au{Johnstone I.\,M., Nadler~B.} Roy's largest root test under rank-one 
alternatives~// Biometrika, 2017. Vol.~104. No.\,1. P.~181--193.

\bibitem{lit:Akbari_AppliedLH}
\Au{Akbari V., Anfinsen S.\,N., Doulgeris~A.\,P., Eltoft~T., Moser~G., Serpico~S.\,B.} 
Polarimetric SAR change detection with the complex Hotelling--Lawley 
trace statistic~// IEEE T.~Geosci. Remote,  2016. Vol.~54. Iss.~7. 
P.~3953--3966.

\bibitem{lit:And}
\Au{Anderson T.\,W.}  An introduction to multivariate analysis.~--- 3rd ed.~--- Hoboken, NJ,
USA: John Wiley \& Sons,  2003. 742~p.

\bibitem{lit:MuirLargeSamp}
\Au{Muirhead R.\,J.}  Asymptotic distributions of some multivariate tests~// 
Ann. Math. Stat., 1970. Vol.~41. No.\,3. P.~1002--1010.

\bibitem{lit:LU01}
\Au{Липатьев А.\,А., Улья\-нов~В.\,В.} Вы\-чис\-ли\-мые оценки точ\-ности при\-бли\-же\-ний 
для статистики Барт\-лет\-та--Нан\-да--Пил\-лай~//
Математические труды, 2016. Т.~19. №\,2. С.~109--118.

\bibitem{lit:WFU}
\Au{Wakaki~H., Fujikoshi~Y., Ulyanov~V.\,V.}  Asymptotic expansions of the 
distributions of
MANOVA test statistics when the dimension is large~// Hiroshima Math.~J., 2014. 
Vol.~44. No.\,3. P.~247--259.

\bibitem{lit:LU02}
\Au{Липатьев А.\,А., Ульянов~В.\,В.} Неасимптотический анализ статистики 
Лоу\-ли--Хо\-тел\-лин\-га для данных большой раз\-мер\-ности~//
Записки научных семинаров \mbox{ПОМИ}, 2019. Т.~486. С.~178--189.

\bibitem{lit:Shevtsova1}
\Au{Shevtsova I.\,G.}  On the absolute constants in the Berry--Esseen type 
inequalities for identically distributed summands~//
arXiv.org, 2011. arXiv:1111.6554 [math.PR]. 7~p.

\bibitem{lit:Kavaguchi}
\Au{Кавагучи Ю., Ульянов~В.\,В., Фуджикоши~Я.}  Приближения для статистик,
описывающих геометрические свойства данных большой размерности, с~оценками 
ошибок~//
Информатика и~её применения, 2010. Т.~4. Вып.~1. С.~22--27.

\bibitem{lit:Letters_2006}
\Au{Ulyanov V.\,V., Wakaki~H., Fujikoshi~Y.}  Berry--Esseen bound for high 
dimensional asymptotic approximation of Wilks' Lambda distribution~//
Stat. Probabil. Lett., 2006. Vol.~76. No.\,12. P.~1191--1200.

\bibitem{lit:TraceIneq}
\Au{Coope I.\,D.}  On matrix trace inequalities and related topics for 
products of Hermitian matrices~//
J.~Math. Anal. Appl., 1949. Vol.~188. No.\,3. P.~999--1001.
\end{thebibliography}

}
}

\end{multicols}

\vspace*{-3pt}

\hfill{\small\textit{Поступила в~редакцию 07.01.2020}}

\vspace*{8pt}

%\pagebreak

%\newpage

%\vspace*{-28pt}

\hrule

\vspace*{2pt}

\hrule

%\vspace*{-2pt}

\def\tit{NONASYMPTOTIC ANALYSIS OF~BARTLETT--NANDA--PILLAI STATISTIC FOR~HIGH-DIMENSIONAL DATA}

\def\titkol{Nonasymptotic analysis of~Bartlett--Nanda--Pillai statistic for~high-dimensional data}

\def\aut{A.\,A.~Lipatiev}

\def\autkol{A.\,A.~Lipatiev}

\titel{\tit}{\aut}{\autkol}{\titkol}

\vspace*{-11pt}


\noindent
Department of Mathematical Statistics, Faculty of Computational 
Mathematics and Cybernetics, M.\,V.~Lomonosov Moscow State University, 
1-52~Leninskiye Gory, GSP-1, Moscow 119991, Russian Federation


\def\leftfootline{\small{\textbf{\thepage}
\hfill INFORMATIKA I EE PRIMENENIYA~--- INFORMATICS AND
APPLICATIONS\ \ \ 2021\ \ \ volume~15\ \ \ issue\ 1}
}%
\def\rightfootline{\small{INFORMATIKA I EE PRIMENENIYA~---
INFORMATICS AND APPLICATIONS\ \ \ 2021\ \ \ volume~15\ \ \ issue\ 1
\hfill \textbf{\thepage}}}

\vspace*{3pt}




\Abste{The author gets the computable error bounds for normal 
approximation of Bartlett--Nanda--Pillai statistic when dimensionality grows proportionally
 to the sample size. This result enables one to get more precise calculations of the p-values 
 in applications of multivariate analysis. In practice, more and more often, analysts encounter 
 situations when the number of factors is large and comparable with the sample size. 
 The examples include signal processing. The proof is essentially based on the normality 
 of the distribution of the elements of the matrices under consideration with the Wishart 
 distribution. For random variables that are the matrix traces of the product and squares 
 of matrices with the normalized Wishart distribution, convenient upper bounds for $1-F$ are 
 found\linebreak\vspace*{-12pt}}
 
 \Abstend{where $F$ is the distribution function of the corresponding matrix trace. Applying 
 the properties of inverse matrices and positive semidefinite matrices, the Bartlett--Nanda--Pillai 
statistic is bounded from above by a combination of the above-mentioned matrix traces.}

\KWE{computable estimates; accuracy of approximation; MANOVA; computable error bounds; 
Bartlett--Nanda--Pillai statistic; high-dimensional data}

\DOI{10.14357/19922264210110}

%\vspace*{-15pt}

%\Ack
%\noindent


\vspace*{12pt}

  \begin{multicols}{2}

\renewcommand{\bibname}{\protect\rmfamily References}
%\renewcommand{\bibname}{\large\protect\rm References}

{\small\frenchspacing
 {%\baselineskip=10.8pt
 \addcontentsline{toc}{section}{References}
 \begin{thebibliography}{99}
\bibitem{1-lip-1}
\Aue{Fujikoshi, Y., V.\,V.~Ulyanov, and R.~Shimizu.} 2010. 
\textit{Multivariate statistics: High-dimensional and large-sample approximations}.
 Hoboken, NJ: John Wiley \& Sons. 512~p.
\bibitem{2-lip-1}
\Aue{Johnstone, I.\,M., and B.~Nadler.} 2017. Roy's largest root test under rank-one 
alternatives. \textit{Biometrika} 104(1):181--193.
\bibitem{3-lip-1}
\Aue{Akbari, V., S.\,N.~Anfinsen, A.\,P.~Doulgeris, T.~Eltoft, G.~Moser, and S.\,B.~Serpico.}
 2016. Polarimetric SAR change detection with the complex Hotelling--Lawley trace statistic. 
 \textit{IEEE~T. Geosci. Remote} 54(7):3953--3966.
\bibitem{4-lip-1}
\Aue{Anderson, T.\,W.} 2003. \textit{An introduction to multivariate analysis}. 
3rd ed. Hoboken, NJ: John Wiley \& Sons. 742~p.
\bibitem{5-lip-1}
\Aue{Muirhead, R.\,J.} 1970. Asymptotic distributions of some multivariate tests. 
\textit{Ann. Math. Stat.} 41(3):1002--1010.
\bibitem{6-lip-1}
\Aue{Lipatiev, A.\,A., and V.\,V.~Ulyanov.} 2017. On computable estimates for accuracy of approximation
 for the Bartlett--Nanda--Pillai statistic. \textit{Siberian Adv. Math.} 27(3):153--159.
\bibitem{7-lip-1}
\Aue{Wakaki, H., Y.~Fujikoshi, and V.\,V.~Ulyanov.} 2014. Asymptotic expansions of the distributions 
of MANOVA test statistics when the dimension is large. \textit{Hiroshima Math.~J.} 44(3):247--259.
\bibitem{8-lip-1}
\Aue{Lipatiev, A.\,A., and V.\,V.~Ulyanov.}
 2019. Neasimptoticheskiy analiz statistiki Louli--Khotellinga dlya dannykh bol'shoy razmernosti 
 [Nonasymptotic analysis of Lawley--Hotelling statistic for high dimensional data]. 
 \textit{Zapiski nauchnykh seminarov POMI} [POMI Notes of Scientific Seminars] 486:178--189.
\bibitem{9-lip-1}
\Aue{Shevtsova, I.\,G.} 2011. On the absolute constants in the Berry--Esseen type inequalities 
for identically distributed summands. arXiv:1111.6554 [math.PR]. Available at: 
{\sf https://arxiv.org/pdf/1111.6554} (accessed December~16, 2020).
\bibitem{10-lip-1}
\Aue{Kawaguchi, Yu., V.\,V.~Ulyanov, and Ya.~Fujikoshi.}
 2010. Priblizheniya dlya statistik, opisyvayushchikh geo\-met\-ri\-che\-skie svoystva dannykh 
 bol'shoy razmernosti, s~otsenkami oshibok [Asymptotic distributions of basic statistics in 
 geometric representation for high-dimensional data and their error bounds]. 
 \textit{Informatika i~ee Primeneniya~--- Inform. Appl.} 4(1):22--27.
\bibitem{11-lip-1}
\Aue{Ulyanov, V.\,V., H.~Wakaki, and Y.~Fujikoshi.} 2006. 
Berry--Esseen bound for high dimensional asymptotic approximation of Wilks' Lambda distribution. 
\textit{Stat. Probabil. Lett.} 76(12):1191--1200.
\bibitem{12-lip-1}
\Aue{Coope, I.\,D.} 1949. On matrix trace inequalities and related topics for products of Hermitian matrices. 
\textit{J.~Math. Anal. Appl.} 188(3):999--1001.
 \end{thebibliography}

 }
 }

\end{multicols}

\vspace*{-3pt}

  \hfill{\small\textit{Received January~7, 2020}}


%\pagebreak

%\vspace*{-8pt}

\Contrl

\noindent
\textbf{Lipatiev Alexander A.} (b.\ 1988)~--- 
PhD student,  Faculty of Computational Mathematics and Cybernetics,
 M.\,V.~Lomonosov Moscow State University, 1-52~Leninskiye Gory, GSP-1, Moscow 119991, 
 Russian Federation; \mbox{allipatev@cs.msu.ru}


\label{end\stat}

\renewcommand{\bibname}{\protect\rm Литература}     %10
\include{stup-bruhov}  %11
\def\stat{gon-zats}

\def\tit{ПРЕДСТАВЛЕНИЕ НОВЫХ ЛЕКСИКОГРАФИЧЕСКИХ ЗНАНИЙ В~ДИНАМИЧЕСКИХ 
КЛАССИФИКАЦИОННЫХ СИСТЕМАХ$^*$}

\def\titkol{Представление новых лексикографических знаний в~динамических 
классификационных системах}

\def\aut{А.\,А.~Гончаров$^1$, И.\,М.~Зацман$^2$, М.\,Г.~Кружков$^3$}

\def\autkol{А.\,А.~Гончаров, И.\,М.~Зацман, М.\,Г.~Кружков}

\titel{\tit}{\aut}{\autkol}{\titkol}

\index{Гончаров А.\,А.}
\index{Зацман И.\,М.}
\index{Кружков М.\,Г.}
\index{Goncharov A.\,A.}
\index{Zatsman I.\,M.}
\index{Kruzhkov M.\,G.}

{\renewcommand{\thefootnote}{\fnsymbol{footnote}} \footnotetext[1]
{Работа выполнена в~Институте проблем информатики ФИЦ ИУ РАН при поддержке РФФИ (проект  
20-012-00166).}}

\renewcommand{\thefootnote}{\arabic{footnote}}
\footnotetext[1]{Институт проблем информатики Федерального исследовательского центра <<Информатика и~управление>> 
Российской академии наук, \mbox{a.gonch48@gmail.com}}
\footnotetext[2]{Институт проблем информатики Федерального исследовательского центра <<Информатика и~управление>> 
Российской академии наук, \mbox{izatsman@yandex.ru}}
\footnotetext[3]{Институт проблем информатики Федерального исследовательского центра <<Информатика и~управление>> 
Российской академии наук, \mbox{magnit75@yandex.ru}}


%\vspace*{-12pt}


     
     \Abst{Характерная особенность динамических классификационных систем (ДКС) состоит 
     в~том, что в~процессе применения этих систем в~них в~любой момент времени могут 
добавляться новые рубрики и/или изменяться дефиниции существующих рубрик, включая 
перераспределение смыслового содержания между ними. С~одной стороны, эта 
особенность ДКС дает возможность оперативно 
отражать в~них новое знание и~сразу начинать его использовать, например в~процессе 
лингвистического аннотирования. С~другой стороны, если некоторая рубрика 
использовалась при аннотировании, а~затем была изменена, то аннотации с~этой 
рубрикой, сформированные до внесения изменений, в~ряде случаев должны быть 
реклассифицированы. Статья преследует двоякую цель, которая состоит, во-пер\-вых, 
в~сопоставлении подходов к~классификации сущностей на основе (1)~ДКС 
и~(2)~онтологий, изменяемых во времени, а~во-вто\-рых, 
в~описании специфики представления новых лексикографических знаний в~ДКС.}
     
     \KW{динамическая классификационная система; версионные онтологии; 
лингвистическое аннотирование; реклассификация аннотаций}

\DOI{10.14357/19922264210112}


\vskip 10pt plus 9pt minus 6pt

\thispagestyle{headings}

\begin{multicols}{2}

\label{st\stat}
     
\section{Введение}

Характерная особенность ДКС 
состоит в~том, что в~процессе применения в~них могут добавляться новые 
рубрики и/или изменяться дефиниции существующих рубрик, включая 
перераспределение смыслового содержания между ними. В~отличие от 
версионных классификационных систем, для которых установлен период их 
обновления (например, рубрики Международной патентной классификации 
могут меняться не чаще, чем раз в~квартал~[1]), добавления и~изменения 
в~ДКС в~случае необходимости могут быть сделаны в~любой момент 
времени. С~одной стороны, эта особенность ДКС позволяет оперативно 
отражать в~них новое знание и~сразу начинать его использовать, например 
в~процессе лингвистического аннотирования~[2]. С~другой стороны, если 
некоторая рубрика использовалась при аннотировании, а~затем была 
изменена, то аннотации с~этой рубрикой, сформированные до внесения 
изменений, в~ряде случаев должны быть реклассифицированы~[3].

Цель статьи состоит в~сопоставлении подходов к~классификации сущностей 
на основе (1)~ДКС и~(2)~онтологий, изменяемых во времени, а также 
в~описании специфики представления новых лексикографических знаний 
в~ДКС. В~качестве примера ДКС в~статье рассматривается фасетная 
классификация (ФК) надкорпусной базы данных (НБД)~[4--6]. В~проекте по 
гранту №\,20-012-00166\linebreak НБД используется для аннотирования таких 
сущностей, как употребления немецких модальных глаго\-лов (НМГ) 
в~параллельных текстах~[7--9], в~процессе которого могут 
(1)~обнаруживаться новые значения НМГ, (2)~добавляться новые рубрики 
для этих значений в~момент их обнаружения~[10, 11] и~(3)~изменяться 
дефиниции рубрик ФК~[3].

\vspace*{-14pt}

\section{Онтологии, изменяемые во~времени}

\vspace*{-4pt}

Методы аннотирования сущностей с~использованием, с~одной стороны, ДКС 
и,~с~другой стороны, онтологий, изменяемых во времени, во многом схожи. 
При аннотировании сущностей \mbox{с~по\-мощью} ДКС им присваиваются рубрики, в~том числе 
до\-бав\-лен\-ные или измененные непосредственно в~процессе 
формирования ка\-кой-ли\-бо аннотации. При классификации сущностей 
с~по\-мощью онтологии для них устанавливаются атрибуты, пред\-став\-ля\-ющие 
собой ссылки на концепты онтологии. С~течением времени могут меняться 
структура и~наполнение как ДКС, так и~онтологии. Это может быть связано 
с~изменениями (1)~самой предметной об\-ласти, (2)~экспертных знаний об этой 
области и/или (3)~стандартов ее описания~[12].

Кроме этого, при проведении исследований научные коллективы могут 
пользоваться теми классификационными системами и~онтологиями для \mbox{своей} 
предметной области, которые находятся в~откры\-том доступе, а~затем 
модифицировать их в~зависимости от целей и~задач выполняемых проектов. 
В~силу перечисленных сходств работы, связанные с~эволюцией 
и~версионностью онтологий, представляют значительный интерес для 
разработки средств актуализации ДКС.

В~[12] описывается подход к~управлению версионностью онтологий 
и~предлагается система нумерации версий онтологий, позволяющая 
определять, обладают ли версии свойством обратной \mbox{совместимости}, или 
эквивалентности (с~точ\-ностью до синтаксических различий). Обратная 
совместимость важна потому, что в~случае ее сохранения снимается 
необходимость в~реклассификации ранее сформированных аннотаций (об 
этой проблеме применительно к~ДКС на примере ФК НБД см.~[3]). 
Сохранять историю изменений понятий онтологии предлагается либо 
в~отдельной онтологии, либо в~выделенных для этого классах исходной 
онтологии. В~\cite{13-gon} представлены инструменты, позволяющие 
выделять и~визуализировать структурные различия между версиями одной 
и~той же онтологии.

В~\cite{14-gon} показано, как с~помощью онтологий могут фиксироваться 
временн$\acute{\mbox{ы}}$е данные, знания, правила и~отношения, 
и~рассмотрены подходы к~описанию изменений в~онтологиях. Темпоральная 
дескриптивная логика (Temporal Description Logics) служит основой 
дополнения онтологии логическими функциями для работы со временем, 
поз\-во\-ля\-ющи\-ми выражать и~обрабатывать такие концепты,\linebreak как <<постоянно 
в~прошлом>> или <<в~некоторый момент в~будущем>>. Для регистрации 
фактов, относящихся к~определенному периоду времени, используются 
подходы, позволяющие обходить \mbox{ограничения} большинства онтологий, 
опирающихся на язык Web Ontology Language (OWL): реификация 
(Reification), четырехмерные (4D-fluent) онтологии и~переход к~n-ар\-ным 
отношениям (N-ary relations). Поскольку по умолчанию отношения 
в~онтологиях OWL являются бинарными, тогда как для фиксации 
временн$\acute{\mbox{о}}$го интервала события необходим по крайней мере 
один дополнительный аргумент, все перечисленные подходы 
характеризуются тем, что для описания временн$\acute{\mbox{ы}}$х фактов 
в~онтологиях каждый раз создается новая сущность.\linebreak На основе 
вышеназванных подходов разработана структура SOWL (spatio-temporal OWL), пред\-на\-значенная 
для включения %\linebreak 
про\-стран\-ст\-вен\-но-вре\-мен\-н$\acute{\mbox{о}}$й информации 
в~онтологии OWL, а~также\linebreak инструмент CHRONOS, упро\-ща\-ющий создание 
и~редактирование данных об \mbox{изменениях} в~онтологиях, ис\-поль\-зу\-ющих эту 
структуру. Кроме того, для извлечения информации из SOWL-он\-то\-ло\-гий 
создан язык запросов, описанный в~[15].

В~[16] дается обзор процессов и~алгоритмов, связанных с~развитием 
онтологий (ontology evolution). Комплекс таких процессов описан как единый 
и~непрерывный цикл, который можно разделить на несколько этапов: 

\begin{enumerate}[(1)]
\item выявление потребности в~изменениях; 
\item формулировка предлагаемых 
изменений; 
\item оценка адекватности предлагаемых изменений; 
\item оценка 
последствий реализации изменений; 
\item внедрение изменений.
\end{enumerate}

 Каждый этап 
рассматривается отдельно с~учетом опыта, полученного в~ходе других 
исследований.

Следует отметить ряд особенностей подхода, предлагаемого в~настоящей 
работе, в~сравнении с~\cite{12-gon, 13-gon, 14-gon, 15-gon, 16-gon}.  
Во-пер\-вых, он нацелен не на описание любых изменений онтологий во 
времени, а~на фиксирование дополнений, изменений порядка и~смыслового 
содержания рубрик самой ДКС. Таким образом, он ближе всего к~работам, 
где описываются подходы к~управлению различными версиями онтологий 
и~развитию онтологий; работы же, посвященные описанию темпоральных 
сущностей в~онтологиях, как правило, имеют иную целевую направленность.

\begin{figure*}[b] %fig1
\vspace*{1pt}
\begin{center}
\mbox{%
\epsfxsize=88mm
\epsfbox{gon-1.eps}
}
\end{center}

\vspace*{3pt}

\noindent
{\small %\begin{center}
{Таблицы для хранения истории изменений рубрик ФК в~НБД.}
%\end{center} %\newline
%
%\vspace*{-6pt}
%
%\noindent
Поля таблицы \textbf{PropHistory}: %\protect\newline
%\begin{itemize}
%\item 
\textbf{PropId} --- уникальный идентификатор рубрики (соответствующей значению НМГ);
\textbf{Code}~--- краткое обозначение рубрики (для значений НМГ обычно имеет следующий вид: 
модальный глагол, дефис, номер значения, например, <<sollen-01>>);
%\item 
\textbf{Name}~--- дефиниция рубрики;
\textbf{isCurrent}~--- признак актуальности данного состояния рубрики (1~--- актуально, 0~--- не 
актуально);
%\item
\textbf{OperationId}~--- идентификатор операции, в~результате которой рубрика приняла вид, 
соответствующий данной строке таблицы;
%\item
\textbf{OperationAttr}~--- атрибут, который присваивается рубрике в~рамках операции с~Id, 
указанным в~поле OperationId (см.\ об атрибутах и~их значениях в~операциях в~табл.~3).
%\end{itemize}
Поля таблицы \textbf{PropOperation}:
%\begin{itemize}
%\item 
\textbf{Id}~--- идентификатор операции (обеспечивает связь между операциями и~рубриками, 
которые они затрагивают);
%\item 
\textbf{Operation}~--- задает тип операции (допустимые значения: CREATE, DELETE, MERGE, 
SPLIT, REORDER, REDISTR, REVISE);
%\item 
\textbf{UserId}~--- идентификатор пользователя, осуществившего операцию;
%\item 
\textbf{TimeDate}~--- дата и~время выполнения операции
%\end{itemize}
}
\end{figure*}

Во-вто\-рых, хотя структура онтологии обычно сложнее, чем структура ДКС, 
это отличие не имеет принципиального значения для задач  
классификации~--- описываемую здесь ФК в~качестве примера ДКС можно 
рассматривать как потенциальную составляющую онтологии знаний по 
лексикографии немецкого языка. Узкая направленность ФК позволяет 
уделить больше внимания непосредственно теме исследования в~рамках 
упомянутого проекта, что немаловажно, поскольку выделение значений НМГ 
является нетривиальной задачей, и~помимо таких операций, как слияние или 
разделение значений, порой требуется перенести часть компонентов 
смыслового содержания из одного значения в~другое, изменить нумерацию 
значений в~соответствии с~последовательностью их описания в~словарных 
статьях, предназначенных для включения в~словарь~\cite{17-gon}, и~т.\,д.

В-третьих, описываемая ФК вложена в~НБД~--- объекты аннотирования 
и~сама ФК физически располагаются в~одной и~той же базе данных, 
благодаря чему разработчикам не нужно беспокоиться о~том, что  
ка\-кие-ли\-бо внешние данные окажутся не\-со\-вмес\-ти\-мы\-ми с~той или иной 
версией ФК. Однако при внесении изменений в~ФК важно в~случае 
необходимости сразу же вносить соответствующие изменения в~аннотации, 
сформированные ранее. Если возможно, это делается автоматически, 
в~противном случае система должна помечать аннотации, затронутые 
изменениями, чтобы их реклассифицировали эксперты~\cite{3-gon}.

\vspace*{-9pt}

\section{Динамическая классификационная система}

\vspace*{-2pt}

В отличие от упомянутых выше решений по поддержке версионности 
онтологий в~предлагаемом подходе и~его реализации в~проекте не 
предусмот\-ре\-на нумерация версий или создание новой версии ФК после 
вносимых изменений. Вместо этого для каждой рубрики ФК в~НБД 
фиксируются все ее изменения и~для них проставляются временн$\acute{\mbox{ы}}$е штампы, 
а~все старые версии сохраняются. Поэтому можно проследить историю 
эволюции каждой рубрики, включая ее взаимодействие с~другими 
руб\-ри\-ка\-ми, а~также восстановить состояние ФК на любой момент времени.

В структуре НБД за сохранение изменений отвечают две связанные между 
собой таблицы: в~первой (PropHistory) хранятся все~--- текущие 
и~устаревшие~--- состояния рубрик ФК (которые в~проекте соответствуют 
значениям НМГ), а~во второй (\mbox{PropOperation})~--- все операции изменения, 
которые применялись к~этим рубрикам. Структура таблиц и~связь между 
ними показаны на рисунке.



Различаются следующие виды операций. 
\begin{enumerate}[1.]
\item CREATE~--- создание новой рубрики.
\item REORDER~--- изменение кода рубрики (соответствует номеру 
значения НМГ по словарю~\cite{17-gon}).
\item REVISE~--- изменение дефиниции рубрики (не затрагивающее 
никакие другие рубрики).
\item MERGE~--- слияние дефиниций двух рубрик, в~результате которого 
одна из рубрик удаляется.
\item DELETE~--- удаление рубрики.
\item SPLIT~--- разделение дефиниции одной рубрики на две, в~результате 
чего создается новая рубрика, а~дефиниция исходной рубрики 
перераспределяется между старой и~новой рубриками.
\item REDISTR (от англ.\ \textit{redistribute})~--- изменение дефиниций 
двух рубрик, предусматривающее перенос части компонентов смыслового 
содержания из одной рубрики в~другую.
\end{enumerate}

\begin{table*}\small %tabl1
\begin{center}
\Caption{Пример выделения компонентов смыслового содержания дефиниции рубрики}
\vspace*{2ex}

\begin{tabular}{|c|c|c|p{52mm}|p{64mm}|}
\hline
Рубрика&Id&Код&\multicolumn{1}{c|}{\tabcolsep=0pt\begin{tabular}{c}
Дефиниция рубрики\\ без структурного выделения\\ компонентов 
смыслового\\ содержания\end{tabular}} &
\multicolumn{1}{c|}{\tabcolsep=0pt\begin{tabular}{c}
Дефиниция рубрики,\\ в~которой выделены компоненты\\ ее 
смыслового содержания\end{tabular}}\\
\hline
X&482&sollen-01&Обязанность что-л. делать по чье\-\mbox{му-л.}\ указанию, по закону, по 
правилам и~т.\,п.: должен. Моральный запрет (под отрицанием): \mbox{нельзя}&$a$.~Обязанность 
что-л.\ делать по чьему-л.\ указанию, по закону, по правилам и~т.\,п.: должен\newline
$b$.~Моральный запрет (под отрицанием): \mbox{нельзя}\\
\hline
\end{tabular}
\end{center}
\end{table*}


В то время как для понимания операций~\mbox{1--6} достаточно приведенных 
определений, операция REDISTR заслуживает более детального 
рассмотрения. Для этого понадобятся условные обозначения, введенные 
в~\cite{3-gon}, а~именно:
\begin{itemize}
\item X, Y,\ \ldots~--- рубрики ФК, обозначающие смыс\-ло\-вые значения 
НМГ;
\item def$_{\mathrm{X}}$, def$_{\mathrm{Y}}$,\ \ldots~---  дефиниции 
рубрик;
\item $\mathbf{S}_{\mathrm{def}_{\mathrm{X}}}$, 
$\mathbf{S}_{\mathrm{def}_{\mathrm{Y}}}$, \ldots ~--- смысловое 
содержание дефиниций рубрик.
\end{itemize}
Кроме того, для обозначения сущностей, которые были каким-то образом 
изменены в~результате выполнения операции, используется индекс <<ch>> 
(от англ.\ \textit{changed}).

Операция REDISTR выполняется только в~том случае, если 
def$_{\mathrm{X}}$ такова, что 
в~$\mathbf{S}_{\mathrm{def}_{\mathrm{X}}}$~--- смысловом содержании 
дефиниции рубрики (соответствующей значению НМГ)~--- выделены 
компоненты (соответствующие подзначениям внутри значения 
НМГ\footnote{С~точки зрения смыслового содержания в~дефиниции рубрики могут быть 
выделены и~более мелкие части, соответствующие частям подзначений. Данная ситуация не 
рассматривается в~рамках настоящей статьи.}). В табл.~1 приводится пример  
руб\-ри\-ки~X, в~дефиниции которой структурно выделены компоненты ее 
смыслового содержания, соответствующие подзначениям,~--- $a$ и~$b$.




Если представить $\mathbf{S}_{\mathrm{def_X}}$, компонентами которого 
являются~$a$ и~$b$, в~виде множества 
$\mathbf{S}_{\mathrm{def_X}}\{a,b\}$, то изменение def$_{\mathrm{X}}$ 
может быть таким, что набор элементов этого множества: (1)~сократится: 
$\mathbf{S}_{\mathrm{def_X}}^1\hm=\{a\}$; (2)~увеличится: 
$\mathbf{S}_{\mathrm{def_X}}^2\hm=\{a, b, c, \ldots\}$.



Более того, возможна ситуация, когда изменение набора элементов затронет 
не только def$_{\mathrm{X}}$, а~одновременно def$_{\mathrm{X}}$ 
и~def$_{\mathrm{Y}}$. Рубрики до внесения изменений обозначим как~X 
и~Y, а~после их внесения~--- как X$_{\mathrm{ch}}$ и~Y$_{\mathrm{ch}}$. 
Если изменение руб\-рик~X и~Y таково, что  $\mathbf{S}_{\mathrm{def_X}} 
\cap \mathbf{S}_{\mathrm{def_Y}}^{\mathrm{ch}}\not= \emptyset$ 
(смысловое содержание def$_{\mathrm{X}}$ и~смысловое содержание 
измененной def$_{\mathrm{Y}}$ имеют один или более общих компонентов) 
и/или $\mathbf{S}_{\mathrm{def_Y}} \cap 
\mathbf{S}_{\mathrm{def_X}}^{\mathrm{ch}}\not= \emptyset$ (смысловое 
содержание def$_{\mathrm{Y}}$ и~смысловое содержание измененной 
def$_{\mathrm{X}}$ имеют один или более общих компонентов), то оно может 
быть описано с~по\-мощью операции \mbox{REDISTR\,(X, Y)}. Экспертная 
реклассификация потребуется для тех аннотаций, которые до выполнения 
операции REDISTR содержали: (1)~руб\-ри\-ку~X, смысловое содержание 
которой в~результате выполнения операции REDISTR сужается;  
(2)~руб\-ри\-ку~X или~Y, если перенос компонентов смыслового содержания 
дефиниций осуществляется как из~X в~Y, так и~из~Y в~X.


Ниже приводится пример выполнения операции REDISTR для рубрик 
с~постоянными номерами (id) 482 и~484, которые позволяют отслеживать 
историю изменений рубрики (табл.~2). В~данном примере: X~---  
руб\-ри\-ка~482 до выполнения операции, причем 
$\mathbf{S}_{\mathrm{def_X}}\hm= \{a, b\}$; Y~--- руб\-ри\-ка~484 до 
выполнения операции, причем $\mathbf{S}_{\mathrm{def_Y}}\hm= 
\{m,n,o\}$; X$_{\mathrm{ch}}$~--- руб\-ри\-ка~482 после выполнения операции, 
причем $\mathbf{S}_{\mathrm{def_X}}^{\mathrm{ch}}\hm= \{a, b, n, o\}$; 
Y$_{\mathrm{ch}}$~--- руб\-ри\-ка~484 после выполнения операции, причем 
$\mathbf{S}_{\mathrm{def_Y}}^{\mathrm{ch}}\hm=\{m\}$.


Поскольку $\mathbf{S}_{\mathrm{def_Y}}\cap 
\mathbf{S}_{\mathrm{def_X}}^{\mathrm{ch}}\hm=\{n, o\}$ 
и~$\mathbf{S}_{\mathrm{def_X}}\cap 
\mathbf{S}_{\mathrm{def_Y}}^{\mathrm{ch}}\hm=\emptyset$, смысловое 
содержание руб\-ри\-ки~482 после внесения изменения расширяется, 
а~смысловое содержание руб\-ри\-ки~484~--- сужается. Следовательно,\linebreak 
аннотации, которые до выполнения операции \mbox{REDISTR} содержали 
 руб\-ри\-ку~482, после ее выполнения не требуют экспертной 
реклассификации и~автоматически обозначаются кодом <<sollen-01>>, тогда 
как аннотации, которые до выполнения операции REDISTR содержали  
руб\-ри\-ку~484, после ее выполнения требуют экспертной реклассификации 
и~поэтому автоматически помечаются тегом <<\mbox{TBR-R}>> (от англ. \textit{To 
Be Reclassified because of Redistribution}).

\begin{table*}\small %tabl2
\begin{center}
\Caption{Исходные данные и~результат выполнения операции REDISTR}
\vspace*{2ex}

\begin{tabular}{|c|c|c|p{120mm}|}
\hline
Рубрика&Id&Код&Дефиниции двух рубрик, в~которых структурно выделены 
компоненты их смыслового содержания, до и~после выполнения операции 
\mbox{REDISTR}\\
\hline
X&482&sollen-01&$a$.~Обязанность что-л.\ делать по чьему-л.\ указанию, по закону, по 
правилам и~т.\,п.: должен\\
&&&$b$.~Моральный запрет (под отрицанием): нельзя\\
\hline
Y&484&sollen-03&$m$.~Желательность по мнению говорящего (в~формах praet conj 
и~pqp conj): следовало (бы), нужно было (бы), должно было (бы)\\
&&&$n$. Совет, рекомендация (только в~формах praet conj)\\
&&&$o$. Нежелательность (под отрицанием): не следовало (бы), нельзя\\
\hline
X$_{\mathrm{ch}}$&482&sollen-01&$a$. Обязанность что-л.\ делать по чьему-л.\ 
указанию, по закону, по правилам и~т.\,п.: должен\\
&&&$b$.~Моральный запрет (под отрицанием): нельзя\\
&&&$n$.~Совет, рекомендация (только в~формах praet conj)\\
&&&$o$.~Нежелательность (под отрицанием): не следовало (бы), нельзя\\
\hline
Y$_{\mathrm{ch}}$&484&sollen-03&$m$.~Желательность по мнению говорящего (в 
формах praet conj и~pqp conj): следовало (бы), нужно было (бы), должно было (бы)\\
\hline
\multicolumn{4}{p{162.7mm}}{\footnotesize\hspace*{3mm}\textbf{Примечания.}
Расшифровка используемых в~таблице сокращений:
\begin{itemize}
\addtolength{\itemsep}{-4pt}
\item praet conj~--- форма прошедшего времени (лат.\ \textit{praeteritum}) сослагательного 
наклонения (лат.\ \textit{conjunctivus});
\item pqp conj~--- форма предпрошедшего времени (лат.\ \textit{plusquamperfectum}) 
сослагательного наклонения (лат.\ \textit{conjunctivus}).
\end{itemize}
Примеры употребления глагола \textit{sollen} по словарю~\cite{17-gon}, иллюстрирующие каждый 
из компонентов смыслового содержания дефиниций (формы глагола \textit{sollen} выделены 
полужирным шрифтом):
\begin{itemize}
\addtolength{\itemsep}{-4pt}
\item[$a$.] ich \textbf{soll} heute noch in die Stadt fahren~--- я должен сегодня еще поехать 
в~город;
\item[$b$.] du \textbf{sollst} nicht t$\ddot{\mbox{o}}$ten!~--- не убий! 
(\textit{библейская заповедь});
\item[$m$.] das \textbf{sollte} sie doch wissen~--- это она (вообще-то) должна была (бы) 
знать;
\item[$n$.] Sie \textbf{sollten} mit dem Rauchen aufh$\ddot{\mbox{o}}$ren~--- вам следует 
бросить курить;
\item[$o$.] das \textbf{sollte} man nie tun~--- этого не следует делать.
\end{itemize}
}
\end{tabular}
\end{center}
\vspace*{-23pt}
%\end{table*}
%\begin{table*}\small %tabl3  %\multicolumn{1}{|c|}{\raisebox{-6pt}[0pt][0pt]{
\begin{center}
\Caption{Соответствия между операциями, атрибутами и~рубриками}
\vspace*{2ex}

\tabcolsep=4.2pt
\begin{tabular}{|l|c|p{105mm}|}
\hline
\multicolumn{1}{|c|}{Операция}&Атрибут&\multicolumn{1}{c|}{Рубрика}\\
\hline
{\raisebox{-6pt}[0pt][0pt]{\tabcolsep=0pt\begin{tabular}{l}CREATE (создается новая\\ 
рубрика X)\end{tabular}}}&A&Создаваемая рубрика X\\
\cline{2-3}
&C&Рубрики, которые надо перенумеровать после выполнения операции, чтобы 
освободить в~нумерации нужную позицию для X\\
\hline
{\raisebox{-6pt}[0pt][0pt]{\tabcolsep=0pt\begin{tabular}{l}REORDER (код рубрики~X\\ изменяется)\end{tabular}}}&A&Рубрика X, код которой изменяется\\
\cline{2-3}
&C&Рубрики, которые надо перенумеровать после выполнения операции, чтобы 
освободить в~нумерации нужную позицию для X\\
\hline
REVISE (изменяется def$_{\mathrm{X}}$)&A&Рубрика X, дефиниция которой изменяется\\
\hline
{\raisebox{-6pt}[0pt][0pt]{\tabcolsep=0pt\begin{tabular}{l}MERGE (объединяются\\ def$_{\mathrm{X}}$ 
и~def$_{\mathrm{Y}}$)\end{tabular}}}&А&Рубрика X, причем 
def$_{\mathrm{X}}$ поглощает def$_{\mathrm{Y}}$, а X остается в~базе данных\\
\cline{2-3}
&B&Рубрика Y, причем def$_{\mathrm{Y}}$ включается в~def$_{\mathrm{X}}$, а~Y 
удаляется из базы данных\\
\cline{2-3}
&C&Рубрики, которые напрямую не затрагиваются операцией MERGE, но которые надо 
перенумеровать после выполнения операции, чтобы заполнить пробелы в~нумерации, 
образовавшиеся из-за удаления Y\\
\hline
{\raisebox{-6pt}[0pt][0pt]{\tabcolsep=0pt\begin{tabular}{l}DELETE (рубрика~X удаля-\\ ется)\end{tabular}}}&A&Удаляемая рубрика X\\
\cline{2-3}
&C&Рубрики, которые надо перенумеровать после выполнения операции, чтобы 
заполнить пробелы в~нумерации, образовавшиеся из-за удаления X\\
\hline
{\raisebox{-6pt}[0pt][0pt]{\tabcolsep=0pt\begin{tabular}{l}SPLIT (def$_{\mathrm{X}}$ делится на две\\
 части~--- def$_{\mathrm{X}}1$ 
и~def$_{\mathrm{X}}2$)\end{tabular}}}&A&Рубрика~X, дефиниция которой делится на две части, причем 
def$_{\mathrm{X}}1$ становится новой дефиницией X\\
\cline{2-3}
&B&Рубрика~Y, которая создается в~базе данных, причем def$_{\mathrm{X}}2$ 
становится дефиницией~Y\\
\hline
{\raisebox{-24pt}[0pt][0pt]{\tabcolsep=0pt\begin{tabular}{l}REDISTR (def$_{\mathrm{X}}$ и~def$_{\mathrm{Y}}$ изме-\\
няются так, что происходит\\ 
перераспределение компо-\\ нентов смыс\-ло\-во\-го содер-\\ жания 
между~$\mathbf{S}_{\mathrm{def_X}}$ 
и~~$\mathbf{S}_{\mathrm{def_Y}}$)\end{tabular}}}&A&Рубрика X, если 
$\mathbf{S}_{\mathrm{def_X}}$ расширяется за счет переноса компонентов 
смыс\-ло\-во\-го содержания из~$\mathbf{S}_{\mathrm{def_Y}}$, причем ни один 
компонент~$\mathbf{S}_{\mathrm{def_X}}$ не переносится 
в~$\mathbf{S}_{\mathrm{def_Y}}$\\
\cline{2-3}
&B&Рубрика Y, если ~$\mathbf{S}_{\mathrm{def_Y}}$ сужается за счет переноса 
компонентов смыс\-ло\-во\-го содержания в~~$\mathbf{S}_{\mathrm{def_X}}$, причем 
ни один компонент~$\mathbf{S}_{\mathrm{def_X}}$ не переносится 
в~~$\mathbf{S}_{\mathrm{def_Y}}$\\
\cline{2-3}
&AB&Рубрики~X и~Y, если одновременно осуществляется перенос компонентов 
смыс\-ло\-во\-го содержания из~$\mathbf{S}_{\mathrm{def_X}}$ 
в~~$\mathbf{S}_{\mathrm{def_Y}}$ и~из~$\mathbf{S}_{\mathrm{def_Y}}$ 
в~~$\mathbf{S}_{\mathrm{def_X}}$\\
\hline
\end{tabular}
\end{center}
\end{table*}

Может показаться, что введение операции \mbox{REDISTR} не оправданно, так как 
рассмотренное перераспределение компонентов значений между~482 и~484 
можно описать последовательностью операций SPLIT\,(MERGE\,(X, Y)). 
Однако при таком подходе объем реклассификации может существенно 
возрасти. Объем реклассификации будет тот же, лишь если верно 
одновременно и~$\mathbf{S}_{\mathrm{def_X}}\hm\cap 
\mathbf{S}_{\mathrm{def_Y}}^{\mathrm{ch}}\hm\not= \emptyset$, 
и~$\mathbf{S}_{\mathrm{def_Y}} \cap 
\mathbf{S}_{\mathrm{def_X}}^{\mathrm{ch}}\not= \emptyset$, т.\,е.\ 
осуществляется перенос компонентов смыслового содержания одновременно 
и~из $\mathbf{S}_{\mathrm{def_X}}$ в~$\mathbf{S}_{\mathrm{def_Y}}$, 
и~из~$\mathbf{S}_{\mathrm{def_Y}}$ в~$\mathbf{S}_{\mathrm{def_X}}$. Если 
же верно или $\mathbf{S}_{\mathrm{def_X}} \cap 
\mathbf{S}_{\mathrm{def_Y}}^{\mathrm{ch}}\not= \emptyset$, 
а~$\mathbf{S}_{\mathrm{def_Y}} \cap 
\mathbf{S}_{\mathrm{def_X}}^{\mathrm{ch}} = \emptyset$, или 
$\mathbf{S}_{\mathrm{def_Y}} \cap 
\mathbf{S}_{\mathrm{def_X}}^{\mathrm{ch}}\not= \emptyset$, 
а~$\mathbf{S}_{\mathrm{def_X}} \cap 
\mathbf{S}_{\mathrm{def_Y}}^{\mathrm{ch}}=\emptyset$ (т.\,е.\ смысловое 
содержание дефиниции одной рубрики расширяется, а~другой~--- сужается), 
то в~таком случае реклассификация нужна только для аннотаций, 
содержавших рубрику с~дефиницией, смысловое
содержание которой 
в~результате изменения сужается (в~примере из табл.~2 это аннотации, 
содержавшие руб\-ри\-ку~Y). Таким образом, введение операции \mbox{REDISTR} 
оправданно, так как позволяет сократить объем реклассификации.

Рассмотренный пример (см.\ табл.~2) показывает, что операция REDISTR  
по-раз\-но\-му влияет на руб\-ри\-ки~482 и~484: смысловое содержание 
дефиниции руб\-ри\-ки~482 расширяется, а~руб\-ри\-ки~484~--- сужается. 
Для того чтобы на основе таб\-лиц НБД с~историей изменения руб\-рик иметь 
возможность определять, как именно операция повлияла на некоторую 
руб\-ри\-ку, каждой из руб\-рик, затронутых операцией, присваивается атрибут. 
В~этом примере руб\-ри\-ке~482 будет присвоен атрибут~A,  
а~руб\-ри\-ке~484~--- B. Все возможные атрибуты и~их смыс\-ло\-вое 
содержание для каждой из семи операций приведены и~расшифрованы 
в~табл.~3.

\vspace*{-3pt}

\section{Заключение}

Предлагаемый подход к~ведению ДКС дает возможность сохранять всю 
информацию об изменениях рубрик ФК, фиксировать виды изменений, время 
их внесения и~данные о~пользователе, который их внес. Он позволяет 
отслеживать совершенные изменения в~хронологическом порядке, а~также, 
при необходимости, восстанавливать состояние ДКС на любой момент 
времени в~прошлом.

\vspace*{-3pt}

{\small\frenchspacing
{%\baselineskip=10.8pt
%\addcontentsline{toc}{section}{References}
\begin{thebibliography}{99}

%\vspace*{-2pt}
\bibitem{1-gon}
\Au{Зацман И.\,М., Косарик~В.\,В., Курчавова~О.\,А.} Задачи представления личностных 
и~коллективных концептов в~цифровой среде~// Информатика и~её применения, 2008. 
Т.~2. Вып.~3. С.~54--69.
\bibitem{2-gon}
Handbook of linguistic annotation~/ Eds. N.~Ide, J.~Pustejovsky.~--- Dordrecht, The 
Netherlands: Springer Science\;+\;Business Media, 2017. 1468~p.
\bibitem{3-gon}
\Au{Гончаров А.\,А., Зацман~И.\,М., Кружков~М.\,Г.} Эволюция классификаций 
в~надкорпусных базах данных~// Информатика и~её применения, 2020. Т.~14. Вып.~4. 
С.~108--116.
\bibitem{4-gon}
\Au{Зацман И.\,М., Инькова~О.\,Ю., Кружков~М.\,Г., Попкова~Н.\,А.} Представление 
кроссязыковых знаний о~коннекторах в~надкорпусных базах данных~// Информатика 
и~её применения, 2016. Т.~10. Вып.~1. С.~106--118.
\bibitem{5-gon}
\Au{Зализняк А., Зацман~И.\,М., Инькова~О.\,Ю.} Надкорпусная база данных коннекторов: 
построение системы терминов~// Информатика и~её применения, 2017. Т.~11. Вып.~1. 
С.~100--108.
\bibitem{6-gon}
\Au{Зацман И.\,М., Кружков~М.\,Г.} Надкорпусная база данных коннекторов: развитие 
системы терминов проектирования~// Системы и~средства информатики, 2018. Т.~28. 
№\,4. С.~156--167.
\bibitem{7-gon}
\Au{Добровольский Д.\,О., Зализняк~Анна~А.} Немецкие конструкции с~модальными 
глаголами и~их русские соответствия: проект надкорпусной базы данных~//\linebreak 
Компьютерная лингвистика и~интеллектуальные технологии: По мат-лам Междунар. 
конф. <<Диалог>>.~--- М.: РГГУ, 2018. С.~172--184.
\bibitem{8-gon}
\Au{Добровольский Д.\,О.} Немецкие модальные глаголы в~параллельном корпусе и~задачи 
двуязычной лексикографии~// Германские языки: текст, корпус, перевод.~--- М.: Институт 
языкознания РАН, 2020. С.~103--116.
\bibitem{9-gon}
\Au{Добровольский Д.\,О., Зализняк~Анна~А.} Русские конструкции с~потенциально 
модальным значением по данным параллельных корпусов~// Труды Института русского 
языка им.\ В.\,В.~Виноградова, 2020. №\,3. С.~35--49.
\bibitem{10-gon}
\Au{Zatsman I.} Finding and filling lacunas in linguistic typologies~// 15th  Forum 
(International) on Knowledge Asset Dynamics Proceedings.~--- Matera: Institute of 
Knowledge Asset Management, 2020. P.~780--793.
\bibitem{11-gon}
\Au{Zatsman I.} Three-dimensional encoding of emerging meanings in AI-systems~// 21st 
European Conference on Knowledge Management Proceedings.~--- Reading: Academic 
Publishing International Ltd., 2020. P.~878--887.
\bibitem{12-gon}
\Au{Klein M., Fensel~D., De~A.} Ontology versioning on the Semantic Web~// 1st  Conference 
(International)  on Semantic Web Working Proceedings.~--- Stanford, CA, USA: Stanford 
University, 2001. P.~75--91.
\bibitem{13-gon}
\Au{Noy N., Kunnatur~S., Klein~M., Musen~M.} Tracking changes during ontology evolution~// 
International Semantic Web Conference~/ Eds. S.\,A.~McIlraith, D.~Plexousakis, F.~van Harmelen.~--- 
Lecture notes in computer science ser.~--- Springer, 2004. Vol.~3298. P.~259--273.
\bibitem{14-gon}
\Au{Preventis A., Petrakis~E.\,G.\,M., Batsakis~S.} CHRONOS Ed: A~tool for handling 
temporal ontologies in prot$\acute{\mbox{e}}$g$\acute{\mbox{e}}$~// Int. J.~Artif. 
Intell.~T., 2014. Vol.~23. No.\,4. P.~1460018-1--1460018-26. doi: 
10.1142/S0218213014600185.
\bibitem{15-gon}
\Au{Stravoskoufos K., Petrakis~E., Mainas~N., Batsakis~S., Samoladas~V.} SOWL QL: 
Querying spatio-temporal ontologies in OWL~// J.~Data Semantics, 2016. Vol.~5. No.\,4. 
P.~249--269.
\bibitem{16-gon}
\Au{Zablith F., Antoniou~G., D'Aquin~M., Flouris~G., Kondylakis~H., Motta~E., 
Plexousakis~D., Sabou~M.} Ontology evolution: A~process-centric survey~// Knowl. 
Eng. Rev., 2015. Vol.~30. No.\,1. P.~45--75.
\bibitem{17-gon}
Немецко-русский словарь актуальной лексики~/ Под ред. Д.\,О.~Добровольского.~--- М.: 
Лексрус, 2021 (в~печати).
\end{thebibliography}

}
}

\end{multicols}

\vspace*{-6pt}

\hfill{\small\textit{Поступила в~редакцию 12.01.2021}}

%\vspace*{8pt}

%\pagebreak

\newpage

\vspace*{-28pt}

%\hrule

%\vspace*{2pt}

%\hrule

%\vspace*{-2pt}

\def\tit{REPRESENTATION OF~NEW LEXICOGRAPHICAL KNOWLEDGE IN~DYNAMIC CLASSIFICATION 
SYSTEMS}

\def\titkol{Representation of new lexicographical knowledge in~dynamic classification 
systems}

\def\aut{A.\,A.~Goncharov, I.\,M.~Zatsman, and~M.\,G.~Kruzhkov}

\def\autkol{A.\,A.~Goncharov, I.\,M.~Zatsman, and~M.\,G.~Kruzhkov}

\titel{\tit}{\aut}{\autkol}{\titkol}

\vspace*{-11pt}


\noindent
Institute of Informatics Problems, Federal Research Center ``Computer Science and
Control'' of the Russian Academy of Sciences, 44-2~Vavilov Str., Moscow 119333,
Russian Federation

\def\leftfootline{\small{\textbf{\thepage}
\hfill INFORMATIKA I EE PRIMENENIYA~--- INFORMATICS AND
APPLICATIONS\ \ \ 2021\ \ \ volume~15\ \ \ issue\ 1}
}%
\def\rightfootline{\small{INFORMATIKA I EE PRIMENENIYA~---
INFORMATICS AND APPLICATIONS\ \ \ 2021\ \ \ volume~15\ \ \ issue\ 1
\hfill \textbf{\thepage}}}

\vspace*{3pt}


\Abste{The distinctive feature of dynamic classification systems is that new categories may be introduced 
in the course of their use or definitions of existing categories may be modified, including cases of 
rearranging semantic content between categories. On one hand, this feature of dynamic classification 
systems provides a possibility to integrate new knowledge on-the-fly and to start using it immediately for 
linguistic annotation. On the other hand, if a category is changed, then, in some cases, the annotations it 
has been previously applied to will have to be reclassified. This paper has a twofold purpose, which is, 
first, to compare approaches to classification of entities based on ($i$)~dynamic classification systems and 
($ii$)~ontologies that change over time; and then, second, to describe how new lexicographical knowledge is 
represented in dynamic classification systems.}

\KWE{dynamic classification system; ontology versioning; linguistic annotation; reclassification of 
annotations}




\DOI{10.14357/19922264210112}

\vspace*{-15pt}

\Ack
\noindent
The study has been conducted at the Institute of Informatics Problems, Federal Research Center 
``Computer Science and Control'' of the Russian Academy of Sciences (FRC CSC RAS) with financial 
support of the Russian Foundation for Basic Research (grant No.\,20-012-00166).
%\vspace*{6pt}

  \begin{multicols}{2}

\renewcommand{\bibname}{\protect\rmfamily References}
%\renewcommand{\bibname}{\large\protect\rm References}

{\small\frenchspacing
 {%\baselineskip=10.8pt
 \addcontentsline{toc}{section}{References}
 \begin{thebibliography}{99}
\bibitem{1-gon-1}
\Aue{Zatsman, I.\,M., V.\,V.~Kosarik, and O.\,A.~Kurchavova.} 2008. Zadachi predstavleniya 
lichnostnykh i~kollektivnykh kontseptov v~tsifrovoy srede [Representation of individual and collective 
concepts in digital medium]. \textit{Informatika i~ee Primeneniya~--- Inform. Appl.} 2(3):54--69.
\bibitem{2-gon-1}
Ide, N., and J.~Pustejovsky, eds. 2017. \textit{Handbook of linguistic annotation}. Dordrecht, The 
Netherlands: Springer Science\;+\;Business Media. 1468~p.
\bibitem{3-gon-1}
\Aue{Goncharov, A.\,A., I.\,M.~Zatsman, and M.\,G.~Kruzhkov.} 2020. Evolyutsiya klassifikatsiy 
v~nadkorpusnykh ba\-zakh dannykh [Evolution of classifications in supracorpora databases]. 
\textit{Informatika i~ee Primeneniya~--- Inform. Appl.} 14(4):108--116.
\bibitem{4-gon-1}
\Aue{Zatsman, I.\,M., O.\,Yu.~Inkova, M.\,G.~Kruzhkov, and N.\,A.~Popkova.}
 2016. Predstavlenie kross-yazykovykh znaniy o~konnektorakh v~nadkorpusnykh 
 bazakh dannykh [Representation of cross-lingual 
knowledge about connectors in suprocorpora databases]. 
\textit{Informatika i~ee Primeneniya~--- Inform. Appl.} 10(1):106--118.
\bibitem{5-gon-1}
\Aue{Zaliznyak, A.\,A., I.\,M.~Zatsman, and O.\,Yu.~In'kova.} 2017. Nadkorpusnaya basa dannykh 
konnektorov: postroenie sistemy terminov [Supracorpora database of connectives: Term system 
development]. \textit{Informatika i~ee Primeneniya~--- Inform. Appl.} 11(1):100--108.
\bibitem{6-gon-1}
\Aue{Zatsman, I.\,M., and M.\,G.~Kruzhkov.} 2018. Nadkorpusnaya baza dannykh konnektorov: razvitie 
sistemy terminov proektirovaniya [Supracorpora database of connectives: Design-oriented evolution of 
the term system]. \textit{Sistemy i~Sredstva Informatiki~--- Systems and Means of Informatics} 
28(4):156--167.
\bibitem{7-gon-1}
\Aue{Dobrovol'skiy, D.\,O., and Anna A.~Zaliznyak.} 2018. Ne\-mets\-kie konstruktsii s~modal'nymi 
glagolami i~ikh russkie sootvetstviya: proekt nadkorpusnoy bazy dannykh [German constructions with 
modal verbs and their Russian correlates: A~supracorpora database project]. \textit{Komp'yuternaya 
lingvistika i~intellektual'nyye tekhnologii: po mat-lam Mezhdunar. konf. ``Dialog'}' [Computational 
Linguistics and Intellectual Technologies. Papers from the Annual Conference (International) 
``Dialogue'']. Moscow: RSHI. 17(24):172--184. 
\bibitem{8-gon-1}
\Aue{Dobrovol'skiy, D.\,O.} 2020. Nemetskie modal'nye glagoly v~parallel'nom korpuse i~zadachi 
dvuyazychnoy leksikografii [German modal verbs in a parallel corpus and bilingual lexicography tasks]. 
\textit{Germanskie yazyki: tekst, korpus, perevod} [German languages: Text, corpus, translation].  Moscow:
Institute of Linguistics RAS. 103--116.
\bibitem{9-gon-1}
\Aue{Dobrovol'skiy, D.\,O., and Anna A.~Zaliznyak.} 2020. Russkie konstruktsii s~potentsial'no 
modal'nym znacheniem po dannym parallel'nykh korpusov [Russian constructions with potentially modal 
meanings: An analysis based on parallel corpus data]. \textit{Trudy Instituta russkogo yazyka im.\ 
V.\,V.~Vinogradova} [V.\,V.~Vinogradov Russian Language Institute Proceedings]. 35--49.
\bibitem{10-gon-1}
\Aue{Zatsman, I.} 2020. Finding and filling lacunas in linguistic typologies. \textit{15th Forum 
(International) on Knowledge Asset Dynamics Proceedings}. Matera: Institute of Knowledge Asset 
Management. 780--793.
\bibitem{11-gon-1}
\Aue{Zatsman, I.} 2020. Three-dimensional encoding of emerging meanings in AI-systems. \textit{21st 
European Conference on Knowledge Management Proceedings}. Reading: Academic Publishing 
International Ltd. 878--887.
\bibitem{12-gon-1}
\Aue{Klein, M., D.~Fensel, and A.~De.} 2001. Ontology versioning on the semantic web. \textit{1st 
Conference (International) on Semantic Web Working Proceedings}. Stanford, CA: Stanford University. 
75--91.
\bibitem{13-gon-1}
\Aue{Noy, N., S.~Kunnatur, M.~Klein, and M.~Musen.} 2004. Tracking changes during ontology 
evolution. \textit{International Semantic Web Conference}. Eds. S.\,A.~McIlraith, D.~Plexousakis, and 
F.~van Harmelen. Lecture notes in computer science ser. Springer. 3298:259--273.
\bibitem{14-gon-1}
\Aue{Preventis, A., E.\,G.\,M.~Petrakis, and S.~Batsakis.} 2014. CHRONOS Ed: A~tool for handling 
temporal ontologies in prot$\acute{\mbox{e}}$g$\acute{\mbox{e}}$. \textit{Int. J.~Artif. 
Intell.~T.} 23(4):1460018. 26~p.
doi: 
10.1142/S0218213014600185.
\bibitem{15-gon-1}
\Aue{Stravoskoufos, K., E.~Petrakis, N.~Mainas, S.~Batsakis, and V.~Samoladas.} 2016. SOWL QL: 
Querying spatio-temporal ontologies in OWL. \textit{J.~Data Semantics} 5(4):249--269.
\bibitem{16-gon-1}
\Aue{Zablith F., G.~Antoniou, M.~D'Aquin, G.~Flouris, H.~Kondylakis, E.~Motta, D.~Plexousakis, and 
M.~Sabou.} 2015. Ontology evolution: A~process-centric survey. \textit{Knowl. Eng. 
Rev.} 30(1):45--75.
\bibitem{17-gon-1}
Dobrovol'skiy, D.O., ed. 2021 (in press). \textit{Nemetsko-russkiy slovar' aktual'noy leksiki} 
[German--Russian dictionary of actual vocabulary]. Moscow: Leksrus.
\end{thebibliography}

 }
 }

\end{multicols}

\vspace*{-3pt}

  \hfill{\small\textit{Received January~12, 2021}}


%\pagebreak

%\vspace*{-8pt}

\Contr

\noindent
\textbf{Goncharov Alexander A.} (b.\ 1994)~--- junior scientist, Institute of Informatics Problems, 
Federal Research Center ``Computer Science and Control'' of the Russian Academy of Sciences,  
44-2~Vavilov Str., Moscow 119333, Russian Federation; \mbox{a.gonch48@gmail.com}

\vspace*{3pt}

\noindent
\textbf{Zatsman Igor M.} (b.\ 1952)~--- Doctor of Science in technology, Head of Department, Institute 
of Informatics Problems, Federal Research Center ``Computer Science and Control'' of the Russian 
Academy of Sciences, 44-2~Vavilov Str., Moscow 119333, Russian Federation; 
\mbox{izatsman@yandex.ru}

\vspace*{3pt}

\noindent
\textbf{Kruzhkov Mikhail G.} (b.\ 1975)~--- senior scientist, Institute of Informatics Problems, Federal 
Research Center ``Computer Science and Control'' of the Russian Academy of Sciences, 44-2~Vavilov 
Str., Moscow 119333, Russian Federation; \mbox{magnit75@yandex.ru}

\label{end\stat}

\renewcommand{\bibname}{\protect\rm Литература}   %12
\def\stat{zatsman}

\def\tit{ТРАНСФОРМАЦИИ ОБЪЕКТОВ ПЕРВОГО И~ВТОРОГО ПОРЯДКА 
В~ЛЕКСИКОГРАФИЧЕСКОЙ ИНФОРМАЦИОННОЙ СИСТЕМЕ$^*$}

\def\titkol{Трансформации объектов первого и~второго порядка 
в~лексикографической информационной системе}

\def\aut{И.\,М.~Зацман$^1$}

\def\autkol{И.\,М.~Зацман}

\titel{\tit}{\aut}{\autkol}{\titkol}

\index{Зацман И.\,М.}
\index{Zatsman I.\,M.}


{\renewcommand{\thefootnote}{\fnsymbol{footnote}} \footnotetext[1]
{Исследование выполнено в~ФИЦ ИУ РАН за счет гранта Российского научного фонда №\,24-18-00155, {\sf 
https://rscf.ru/project/24-18-00155}. Работа выполнялась с~использованием инфраструктуры Центра 
коллективного пользования <<Высокопроизводительные вычисления и~большие данные>> (ЦКП 
<<Информатика>>) ФИЦ ИУ РАН (г.\ Москва).}}


\renewcommand{\thefootnote}{\arabic{footnote}}
\footnotetext[1]{ Федеральный исследовательский центр <<Информатика и~управление>> Российской академии наук, 
\mbox{izatsman@yandex.ru}}

\vspace*{-12pt}


  
  \Abst{Рассматриваются теоретические основания проектирования информационных 
технологий (ИТ) интеграции двуязычных словарей и~параллельных корпусов. Дано описание 
первых результатов создания третьего уровня классификации трансформаций объектов 
предметной области информатики, которую предполагается использовать при создании 
концепции лексикографической информационной системы, обеспечивающей интеграцию. 
Все сущности информатики в~статье разделены на два глобальных класса: объекты и~их 
трансформации. Для каждого такого класса конструируется своя классификация. Ранее были 
описаны два верхних уровня классификации трансформаций объектов предметной области. 
В~данной статье рассматривается третий уровень этой классификации. Основанием для 
построения самого верхнего ее уровня служило деление предметной области информатики 
на среды (ментальная, сенсорно воспринимаемая, цифровая и~ряд других сред), каждая из 
которых по определению включает объекты одной природы. Основанием для построения 
второго уровня классификации трансформаций объектов служила типология знаковых  
сис\-тем А.~Соломоника. Цель статьи состоит в~систематизации трансформаций первого 
и~второго порядка объектов предметной области на третьем уровне этой классификации. 
Основанием для систематизации служит средовая версия иерархии Акоффа.}
  
  \KW{объекты предметной области; трансформации объектов; классификация; данные; 
информация; знание; лексикографическая информационная сис\-тема}

\DOI{10.14357/19922264240211}{VZTGVV}
  
\vspace*{3pt}


\vskip 10pt plus 9pt minus 6pt

\thispagestyle{headings}

\begin{multicols}{2}

\label{st\stat}
  
\section{Введение}

\vspace*{-9pt}

  Возникновение параллельных корпусов, в~которых предложениям 
оригинального текста со\-по\-став\-ле\-ны предложения его перевода, обеспечило 
возможность контрастивного лингвистического\linebreak \mbox{анализа} на принципиально 
новом уровне полноты и~точности, недостижимом в~докорпусную эпоху. 
Пионерскими в~этой области стали работы \mbox{1990-х~гг}. Стига Йоханссона  
с~анг\-ло-нор\-веж\-ским корпусом~[1]. В России параллельные корпусы стали 
формироваться в~начале XXI~века в~рамках Национального корпуса русского 
языка~[2].
  
  Создатели двуязычных словарей используют параллельные корпусы для 
сбора материала и~эмпирической проверки своих гипотез, касающихся 
межъязы\-ко\-вой эквивалентности. Ценность параллельных корпусов 
определяется тем, что в~лингвистике этап сбора исходного материала считается 
наиболее трудоемким и~наименее творческим, а~параллельные корпусы 
позволяют значительно сэкономить время и~силы для творческого этапа 
создания словарей~[3].
 % 
  При этом двуязычные словари, создаваемые на основе исходного материала, 
извлеченного из параллельных корпусов, сейчас формируются без связей с~их 
текстами. Другими словами, онлайновые связи созданных словарей 
с~параллельными корпусами, которые служили источниками исходного 
материала, отсутствуют. 

Параллельные корпусы постоянно пополняются 
новыми текстами, в~предложениях которых можно обнаружить новые значения 
слов и~устойчивых словосочетаний. Однако при этом отсутствуют методы 
и~средства оперативного обновления словарей по корпусным данным. 
В~настоящее время проблема установления связей между двуязычными 
словарями и~параллельными корпусами (далее~--- проблема интеграции) 
находится на стадии поиска концептуальных подходов к~их интеграции на 
уровне значений.
  
  Подход к~решению проблемы интеграции, предлагаемый в~статье, учитывает 
  и~появление новых значений слов и~устойчивых словосочетаний, и~динамику 
смысловых значений, которая обусловлена развитием и~пополнением знания 
лингвистов, фиксирующих эти значения в~результате семантического анализа 
пополняемых корпусных данных. Проведенные эксперименты показали, что 
обнаружение нового лингвистического знания обусловливает и~формирование 
дефиниций новых значений, и~пересмотр уже существующих дефиниций~[4, 5].
  
  Например, в~проведенных экспериментах с~использованием ЦКП 
<<Информатика>> ФИЦ ИУ РАН фиксировалась эволюция значений немецких 
модальных глаголов, исходное состояние значений которых было описано 
в~не\-мец\-ко-рус\-ском словаре. В~экспериментальном массиве текстов как 
потенциальных источниках нового знания 16\,268 предложений содержали 
немецкие модальные глаголы и~в~2041 из них встречался глагол sollen. 
В~начале эксперимента в~словаре были описаны~12~значений этого модального 
глагола. По окончании эксперимента лингвисты обнаружили два новых его 
значения, согласовали их дефиниции и~описали эволюцию дефиниций~[6, 7].
  
  Таким образом, для решения проблемы интеграции требуется фиксировать 
новое знание, обнаруженное лингвистами в~текстовых данных параллельных 
корпусов, отслеживать эволюцию знания, представленного в~виде дефиниций 
значений слов и~устойчивых словосочетаний, и,~соответственно, 
актуализировать электронные двуязычные словари. Предлагаемый 
концептуальный подход к~интеграции, который планируется реализовать 
в~процессе проектирования лексикографической информационной сис\-те\-мы, 
фиксирующей эволюцию лингвистического знания, основан на решении 
следующих задач:\\[-14pt]
  \begin{itemize}
  \item категоризация трех базовых понятий информатики, включенных 
  в~иерархию Акоффа~[8] (данные, информация, знание), на объекты 
проектируемой сис\-те\-мы, которая необходима, чтобы фиксировать 
<<кванты>> нового знания и~отслеживать его эволюцию в~этой сис\-теме;\\[-15pt]
  \item  систематизация трансформаций объектов этой сис\-темы.\\[-14pt]
  \end{itemize}
  
  Цель статьи и~состоит в~решении двух задач: категоризации трех базовых 
понятий информатики на объекты лексикографической информационной  
сис\-те\-мы и~сис\-те\-ма\-ти\-за\-ции трансформаций первого и~второго порядка 
ее объектов.
  
  Трансформациями первого порядка, о которых сказано в~формулировке цели 
статьи, называются взаимные преобразования между двумя объектами  
сис\-те\-мы одной природы. Например, перевод в~сис\-те\-ме текста с~русского 
языка на английский относится к~ним. Трансформациями второго порядка 
и~выше называются взаимные преобразования между двумя и~более объектами 
разной природы. Например, кодирование символов текс\-та компьютерными 
кодами и~их декодирование относятся по определению к~трансформациям 
второго порядка.

%\vspace*{-9pt}
  
\section{Процессы трансформаций в~информатике}

%\vspace*{-3pt}

Процессы трансформаций, рассматриваемые в~статье, относятся к~теоретическому ядру информатики, а~не 
только к~проектированию лексикографической информационной сис\-те\-мы. Например, из трех основных 
подходов к~описанию предметной об\-ласти информатики\footnote{В статье предметная область информатики 
трактуется согласно концепции полиадического компьютинга Пола Розенблума~\cite{9-zac}.} (объектный, 
трансформационный и~синтетический) сис\-те\-ма\-ти\-за\-ция трансформаций ближе всего ко второму 
подходу. Примерами первого подхода, в~рамках которого основное внимание уделяется объектам предметной 
области информатики и~в~меньшей степени отношениям\linebreak между ними, могут служить  
работы~\cite{8-zac, 10-zac, 11-zac}; \mbox{примерами} второго подхода, в~рамках которого основное внимание 
уделяется трансформациям и~в~меньшей степени трансформируемым объектам,~---  
работы~\cite{12-zac, 13-zac}; примерами третьего, синтетического подхода, в~котором уделяется внимание 
и~объектам предметной об\-ласти информатики, и~отношениям между ними, могут служить работы~\cite{14-zac, 
15-zac, 16-zac, 17-zac, 18-zac}.

  Таким образом, для описания трансформаций объектов лексикографической 
информационной\linebreak системы предпочтительнее всего трансформационный 
подход, который упоминается и~в определениях информатики. Например, 
в~2009~г.\ П.~Деннинг и~П.~Розенблум сформулировали суть \mbox{информатики} как 
компьютинга следующим образом: <<$\ldots$информатика~--- это не просто 
алгоритмы и~структуры данных; это преобразования [трансформации] 
представлений>>~\cite{12-zac}. Чуть позже, в~контексте краткого описания 
парадигмы информатики как компьютинга, П.~Деннинг и~П.~Фриман изменили 
эту формулировку на такую: <<Центральный объект внимания в~информатике 
можно определить как информационные процессы~--- \textit{естественные или 
искусственные процессы, преобразующие информацию} (курсив мой~--- 
И.\,З.)>>~\cite{13-zac}. Согласно парадигме, предлагаемой авторами этой 
статьи, на начальном этапе проектирования автоматизированных систем 
базовыми элементами моделей их функционирования служат 
\textit{информационные про\-цессы}.
  
  Однако если 15~лет назад в~формулировке из работы~\cite{13-zac} шла речь 
о~процессах, преобразующих информацию, то в~последние~10~лет в~спектр 
процессов трансформаций все чаще стали включать процессы, преобразующие 
не только информацию, но также и~другие объекты автоматизированных 
систем, в~первую очередь данные и~знания~[19--21]. Например, Виктория 
Стодден, позиционируя науку о~данных как одну из дисциплин информатики, 
говорит, что центральный объект исследований в~науке о~данных~--- это 
<<изучение обобщаемого извлечения знания из данных>>~\cite{21-zac}. 
Увеличение и~чис\-ла объектов, и~спект\-ра процессов их трансформаций 
в~автоматизированных сис\-те\-мах обуслов\-ли\-ва\-ет не\-об\-хо\-ди\-мость 
систематизации и~объектов, и~процессов их трансформаций на начальном этапе 
проектирования сис\-тем.
  
  Для создания концепции лексикографической информационной сис\-те\-мы 
и~проектирования ИТ, обеспечивающих интеграцию 
двуязычных словарей и~параллельных корпусов, сначала выполним 
категоризацию на объекты этой сис\-те\-мы трех базовых понятий информатики 
(данные, информация, знание) в~контексте построения классификаций 
сущностей ее предметной об\-ласти.
  
  Необходимость использования классификаций информатики в~процессе 
создания концепции проиллюстрируем, используя иерархию  
Акоффа~\cite{8-zac}. Он использовал принцип их вертикального размещения 
в~иерархии снизу вверх: данные, информация и~знание. Еще в~ней есть термин 
<<мудрость>>, который в~статье не рассматривается. Такое размещение Акофф 
прокомментировал так: <<Каждое из пе\-ре\-чис\-лен\-ных понятий [кроме данных] 
содержит в~себе нижестоящие$\ldots$>>~\cite{8-zac}.
  
  Этому принципу размещения и~комментарию Акоффа свойственны 
недостатки, проанализированные, в~частности, в~работе~\cite{10-zac}. Главный 
вывод, к~которому пришла Роули после изучения иерархии Акоффа, 
заключается в~следующем: <<$\ldots$информация определяется в~терминах 
данных, знание~--- в~терминах информации$\ldots$ но существует меньше 
консенсуса в~описании трансформаций, которые преобразуют сущности, 
расположенные ниже в~иерархии, в~те, которые находятся над ними, что 
приводит к~их терминологической неопределенности>>~\cite{10-zac}. Причина 
этой неопределенности, скорее всего, в~том, что базовые понятия информатики 
включены в~иерархию Акоффа изолированно от общего контекста 
классификаций сущностей ее предметной об\-ласти.

%\vspace*{-9pt}
  
\section{Классификации сущностей информатики}


%\vspace*{-2pt}

  Все сущности предметной области информатики в~работах~[22, 23] 
разделены на два глобальных класса: ее объекты и~их трансформации. Для 
каждого такого класса была предложена своя классификация. 
В~работе~\cite{22-zac} дано описание классификации объектов предметной 
области информатики, первый уровень которой содержит базовые понятия ее 
предметной области (данные, информация, знания и~др.).  
В~работе~\cite{23-zac} дано описание двух верхних уровней классификации 
трансформаций объектов предметной об\-ласти (см.\ рисунок 
в~работе~\cite{23-zac}). Основанием для построения самого верхнего ее уровня послужило деление 
предметной области информатики на среды\footnote{В~работе~\cite{24-zac} дано описание пяти сред 
предметной области информатики (ментальная; сенсорно воспринимаемая, или информационная; 
цифровая; нейро- и~ДНК-среда), каждая из которых по определению включает объекты одной и~той же 
природы.} и~степень разнообразия природы объектов, вовлеченных в~трансформации:
\begin{itemize}
\item  первый класс верхнего уровня классификации включает 
трансформации объектов в~пределах среды только одной природы 
(трансформации первого порядка);
\item  второй класс включает трансформации объектов, относящихся 
к~двум средам разной природы (трансформации второго порядка);
\item третий и~последующие классы включают трансформации объектов, 
относящихся к~трем и~более средам разной природы (трансформации 
третьего и~более высоких порядков).
\end{itemize}

  В работе~\cite{23-zac} были приведены примеры для трех первых классов 
трансформаций, включая пример трансформаций объектов, относящихся 
к~двум средам разной природы (компьютерное кодирование символов текстов 
с~по\-мощью таб\-лиц Unicode).
  
Основанием для построения второго уровня классификации трансформаций объектов послужила типология 
знаковых сис\-тем А.~Соломоника~\cite[c.~131]{25-zac}: естественные знаковые сис\-те\-мы, образные,  
ес\-тест\-вен\-но-язы\-ко\-в$\acute{\mbox{ы}}$е,  
вер\-баль\-но-не\-сло\-вес\-ные сис\-те\-мы записи\footnote{Под системой записи понимается знаковая 
система, сочетающая вербальные знаки с~несловесными (языки нотной записи, карт, таблиц и~др.).} 
и~формализованные знаковые сис\-те\-мы, включая математические. Введем понятие обобщенного текста~--- 
это текст, который может быть создан в~любой из перечисленных знаковых систем. Тогда обобщенные тексты 
могут быть естественными, образными, ес\-тест\-вен\-но-язы\-ко\-в$\acute{\mbox{ы}}$\-ми,  
вер\-баль\-но-не\-сло\-вес\-ны\-ми и~формализованными. Второй уровень классификации трансформаций 
охватывает не все виды объектов предметной  
об\-ласти информатики, а~только перечисленные~5~видов текс\-тов и~их представления, вовлеченные 
в~процессы трансформаций в~одной или более средах вместе с~данными, знанием и~его концептами.

\begin{figure*}[b] %fig1
\vspace*{6pt}
      \begin{center}
     \mbox{%
\epsfxsize=121.191mm 
\epsfbox{zac-1.eps}
}
\end{center}
\vspace*{-6pt}
\Caption{Средовая версия иерархии Акоффа}
\end{figure*}

\section{Классификация трансформаций: построение~третьего 
уровня}

  Основанием для систематизации трансформаций первого и~второго порядка 
на третьем уровне этой классификации служит иерархия Акоффа~\cite{8-zac}, 
на основе которой и~была создана ее средов$\acute{\mbox{а}}$я версия~[26, 
27]. Для создания средов$\acute{\mbox{о}}$й версии была выполнена 
категоризация трех базовых понятий информатики (данные, информация, 
знания) на объекты лексикографической информационной сис\-те\-мы 
в~процессе создания ее концепции\linebreak (рис.~1).
  


  В отличие от классической иерархии Акоффа, в~ее 
средов$\acute{\mbox{о}}$й версии различаются три вида данных: сенсорно 
воспринимаемые, цифровые и~те данные, которые генерируются 
искусственными нейронными сетями (ИНС) в~системах искусственного интеллекта 
(далее~--- ИИ-дан\-ные). Последний вид данных необходим, например, для 
различения входа и~выхода процесса применения обученной 
ИНС в~цифровой модели генерации знания, описанию которой 
посвящена работа~\cite{27-zac}.
  
  Также предлагается различать два вида информации: сенсорно 
воспринимаемая и~цифровая. Кроме знания в~средов$\acute{\mbox{у}}$ю 
версию добавлены концепты и~ментальные образы сенсорно воспринимаемых 
данных. Последние служат промежуточной сущностью между сенсорно 
воспринимаемыми данными и~генерируемым знанием при описании процессов 
извлечения знания из текстовых данных лексикографической информационной 
системы. Описание объектов средов$\acute{\mbox{о}}$й версии иерархии 
Акоффа (см.\ рис.~1) и~отношений между ними дано в~работах~\cite{26-zac, 28-zac}.
  
  В средов$\acute{\mbox{о}}$й версии число объектов равно восьми. Если 
учитывать направления трансформаций, то между восемью объектами на 
рис.~1 она включает~16 их видов (трансформации на границе между сенсорно 
воспринимаемыми данными и~информацией, обозначенные символом~<<?>>, 
в~статье не рас\-смат\-ри\-ва\-ют\-ся). В~будущем число объектов 
в~средов$\acute{\mbox{о}}$й версии, которая выбрана как основание для 
сис\-те\-ма\-ти\-за\-ции трансформаций первого и~второго порядка, может быть 
увеличено. Для построения классификации трансформаций 
важ\-но не возможное увеличение числа объектов 
и~трансформаций между ними, а то, что их виды в~средов$\acute{\mbox{о}}$й 
версии распределены между трансформациями первого и~второго порядка. Из 
16~видов на рис.~1 шесть относятся к~трансформациям первого порядка, это\linebreak 
виды с~номерами~7, 8, 13--16 (далее~--- типология трансформаций первого 
порядка), а~десять~--- к~трансформациям второго порядка, это виды 
с~\mbox{номерами}~1--6 и~9--12 (далее~--- типология трансформаций второго 
порядка). Разместим обе типологии на третьем уровне классификации (см.\ ее 
схему на рис.~2). Перечислим виды трансформаций первой типологии, вводя 
в~скобках их краткие названия, используемые ниже на рис.~3:
  \begin{description}
  \item[\,] 7~--- членение знания на концепты с~помощью одной или нескольких 
знаковых систем (далее~--- членение знания);
  \item[\,] 8~--- формирование знания на основе концептов (формирование 
знания);
  \item[\,] 13~--- обучение ИНС;
  \end{description}
  
  \vspace*{-6pt}
  
  \pagebreak
  
  \end{multicols}
  
  \begin{figure*} %fig2
\vspace*{1pt}
      \begin{center}
     \mbox{%
\epsfxsize=127.513mm 
\epsfbox{zac-2.eps}
}
\end{center}
\vspace*{-9pt}
\Caption{Схема трех верхних уровней классификации трансформаций объектов (объединены 
по три слоя и~для второго, и~для третьего уровней этой классификации)}
\end{figure*}
  
  \begin{multicols}{2}
  
  \noindent
  \begin{description}
  \item[\,] 14~--- восстановление обучающей информации на основе 
содержания обученной ИНС (обращение ИНС);
  \item[\,] 15~--- использование обученной ИНС (использование ИНС);



  \item[\,] 16~--- восстановление исходных данных, соответствующих 
полученным результатам работы обучен\-ной ИНС (восстановление исходных данных 
по результатам ИНС).
  \end{description}
  
  
  Не все виды трансформаций 13--16 поддерживаются в~конкретных системах 
искусственного интеллекта, но с~теоретической точки зрения все их 
предлагается включить в~первую типологию для полноты спектра видов 
трансформаций.
  
  Перечислим виды трансформаций второй типологии:
  \begin{description}
  \item[\,] 1~--- декодирование цифровых данных в~компьютерных системах 
(декодирование данных);
  \item[\,]  2~--- кодирование сенсорно воспринимаемых данных (кодирование 
данных);
  \item[\,] 3~--- ментальное копирование сенсорно воспринимаемых данных 
(ментальное копирование);
  \item[\,] 4~--- восстановление сенсорно воспринимаемых данных по 
ментальным образам (восстановление по образам);
  \item[\,] 5~--- смысловая интерпретация без деления на концепты ментальных 
образов сенсорно воспринимаемых данных (смысловая интерпретация);
  \item[\,] 6~--- восстановление ментальных образов (восстановление образов);
  \item[\,] 9~--- представление концептов в~виде сенсорно воспринимаемой 
информации, например текс\-та\-ми, формулами, таблицами, рисунками и~т.\,д.\ 
(представление концептов);
  \item[\,] 10~--- понимание смысла сенсорно воспринимаемой информации 
(понимание смысла);
  \item[\,] 11~--- кодирование сенсорно воспринимаемой информации 
(кодирование информации);
\end{description}

\vspace*{-6pt}

\pagebreak

\end{multicols}

\begin{figure*} %fig3
\vspace*{1pt}
      \begin{center}
     \mbox{%
\epsfxsize=163mm 
\epsfbox{zac-3.eps}
}
\end{center}
\vspace*{-9pt}
\Caption{Схема частного случая классификации трансформаций объектов (трансформации 
пронумерованы согласно рис.~1)}
\end{figure*}

\begin{multicols}{2}

\noindent
\begin{description}

  \item[\,] 12~--- декодирование цифровой информации (декодирование 
информации).
  \end{description}
  
  Отметим, что в~существующих ИТ
  и~компьютерных системах наиболее часто используются виды 
трансформаций~13 и~15 типологии первого порядка и~1, 2, 11 и~12 типологии 
второго порядка. На рис.~2 в~первом слое третьего уровня классификации 
показаны типологии первого порядка без указания числа трансформаций в~них 
и~без детализации трансформируемых объектов.
  
  Во втором слое третьего уровня классификации условно (без названий) 
показаны типологии второго порядка. Также на рис.~2 в~третьем слое третьего 
уровня классификации условно (также без названий) показаны типологии 
третьего порядка, которые планируется рассмотреть в~отдельной статье. По 
определению они должны включать трансформации между тремя объектами 
разной природы, но средов$\acute{\mbox{а}}$я версия иерархии Акоффа 
включает трансформации только между двумя объектами разной природы. 
Поэтому потребуется другое основание для их систематизации (ранее были 
рассмотрены отдельные примеры трансформаций третьего 
порядка\footnote{Далеко не всегда трансформации третьего и~более высоких порядков можно 
рассматривать как последовательность трансформаций второго порядка. Примером этого могут 
служить трансформации в~процессе обучения пациента пользованию роботизированной рукой, 
охватывающие личностные концепты пациента, релевантные его намерениям, сигналы активности 
мозга как объекты нейросреды и~компьютерные коды~\cite{29-zac}.}~\cite{29-zac}).

\section{Классификация трансформаций: частный~случай}

  Выше было отмечено, что в~будущем число объектов 
в~средов$\acute{\mbox{о}}$й версии иерархии Акоффа может быть увеличено. 
Это означает, что увеличатся и~чис\-ло объектов, и~чис\-ло трансформаций между 
ними в~классификации трансформаций, так как эта средов$\acute{\mbox{а}}$я 
версия служит по определению основанием для систематизации 
трансформаций первого и~второго порядка. Поэтому на третьем уровне рис.~2 
указаны типологии без детализации объектов и~без указания числа 
трансформаций в~каждой из них. С~одной стороны, при таком подходе 
получаем достаточно общий вид этой классификации, так как она не зависит от 
числа объектов в~том или ином варианте средов$\acute{\mbox{о}}$й версии 
(и~это существенно упрощает рис.~2). С~другой стороны, на третьем уровне 
такой общей классификации подразумевается, но не эксплицируется природа 
трансформируемых объектов и~их возможные сочетания в~трансформациях. 

При проектировании лексикографической информационной системы важно 
эксплицировать природу трансформируемых объектов и~их возможные 
сочетания.
  %
  Поэтому в~парадигму информатики~\cite{30-zac} кроме общей 
классификации трансформаций предлагается включать и~ее частные случаи, 
эксплицирующие природу трансформируемых объектов. 

В~этом разделе 
рассмотрим один частный случай, когда используются только естественные 
знаковые сис\-те\-мы из типологии А.~Соломоника~\cite{25-zac} вместе 
с~данными, знанием и~его концептами. Чис\-ло естественных языков при этом не 
ограничено. И~этот частный случай классификации включает только три 
класса природных трансформаций (первого, второго и~третьего порядка, см.\ 
схему классификации на рис.~3).
  
  Первый и~второй уровни схемы общей классификации (см.\ рис.~2) можно 
объединить в~один уровень в~этом частном случае. Ниже этого уровня 
приведено содержание типологий первого и~второго порядка без содержания 
типологий третьего по\-рядка.




  Наполнение типологий первого и~второго порядка соответствует 
средов$\acute{\mbox{о}}$й версии иерархии Акоффа на рис.~1, содержащей 
6~видов трансформаций типологии первого порядка и~10~видов 
трансформаций типологии второго порядка (на рис.~3 стрелки указывают 
направления трансформаций согласно средов$\acute{\mbox{о}}$й версии на рис.~1).
  
  Таким образом, частный случай классификации содержит для этих двух 
типологий 16~теоретически возможных трансформаций, 6 из которых 
в~настоящее время в~существующих ИТ применяются наиболее часто: виды 
трансформаций~1, 2, 11 и~12 типологии второго порядка реализуются 
с~помощью тех или иных методов ко\-ди\-ро\-ва\-ния/де\-ко\-ди\-ро\-ва\-ния 
(например, с~использованием таблиц Unicode), а~виды трансформаций~13 и~15
 в~типологии первого порядка реализуются полностью с~по\-мощью процессов 
цифровой обработки компьютерами.
  
  Остальные виды трансформаций или применяются намного реже (это 
виды~3, 5, 7, 9 и~10), или находятся в~стадии поиска и~разработки (14 и~16) или 
в~настоящее время носят только теоретический характер, обеспечивая полноту 
первой и~второй типологий (4, 6 и~8). Знаком~<<?>> обозначены те виды 
трансформаций, которые по определению не существуют в~используемой 
парадигме информатики~\cite{30-zac}. Однако возможно, что в~других 
будущих подходах к~построению ее парадигмы эти виды трансформаций будут 
существовать.
  
\section{Заключение}

  На сегодняшний день процесс построения классификаций объектов 
предметной области информатики~\cite{22-zac} и~их  
трансформаций~\cite{23-zac} еще не завершен. Однако первые результаты их 
построения уже используются для создания концепции лексикографической 
информационной сис\-те\-мы, обеспечивающей интеграцию двуязычных 
словарей и~параллельных корпусов.
  
  \bigskip
  
  
  Автор признателен рецензентам за помощь в~улучшении статьи.
  
{\small\frenchspacing
 { %\baselineskip=10.6pt
 %\addcontentsline{toc}{section}{References}
 \begin{thebibliography}{99}
\bibitem{1-zac}
\Au{Aijmer K., Altenberg~B.} Advances in corpus-based contrastive linguistics. Studies in honour 
of Stig Johansson.~--- Amsterdam: John Benjamins, 2013. 295~p.  doi: 10.1075/scl.54.
\bibitem{2-zac}
\Au{Добровольский Д.\,О., Кретов~А.\, А., Шаров~С.\,А.} Корпус параллельных текстов~// 
Научная и~техническая информация. Сер.~2: Информационные процессы и~сис\-те\-мы, 2005. 
№\,6. С.~16--27.
\bibitem{3-zac}
\Au{Добровольский Д.\,О.} Корпус параллельных текстов и~сопоставительная 
лексикология~// Труды Института русского языка им.\ В.\,В.~Виноградова, 2015. №\,6. 
С.~413--449. EDN: VJQBHP.
\bibitem{4-zac}
\Au{Гончаров А.\,А., Зацман~И.\,М., Кружков~М.\,Г.} Эволюция классификаций 
в~надкорпусных базах данных~// Информатика и~её применения, 2020. Т.~14. Вып.~4. 
С.~108--116. doi: 10.14357/19922264200415.  
EDN: \mbox{GKWBZT}.
\bibitem{5-zac}
\Au{Гончаров А.\, А., Зацман И. \,М., Кружков~М.\, Г}. Представление новых 
лексикографических знаний в~динамических классификационных сис\-те\-мах~// 
Информатика и~её применения, 2021. Т.~15. Вып.~1. С.~86--93.  doi: 10.14357/19922264210112. EDN: OPEFXW.
\bibitem{6-zac}
\Au{Zatsman I.} Finding and filling lacunas in linguistic typologies~// 15th Forum (International) 
on Knowledge Asset Dynamics Proceedings.~--- Matera, Italy: Institute of Knowledge Asset 
Management, 2020. P.~780--793.
\bibitem{7-zac}
\Au{Zatsman I.} Three-dimensional encoding of emerging meanings in AI-systems~// 21st 
European Conference on Knowledge Management Proceedings.~--- Reading, U.K.: Academic 
Publishing International Ltd., 2020. P.~878--887.
\bibitem{8-zac}
\Au{Ackoff R.} From data to wisdom~// J.~Applied Systems Analysis, 1989. Vol.~16. No.\,1. P.~3--9.
\bibitem{9-zac}
\Au{Rosenbloom P.\,S.} On computing: The fourth great scientific domain.~--- Cambridge, MA, 
USA: MIT Press, 2013. 307~p.
\bibitem{10-zac}
\Au{Rowley J.} The wisdom hierarchy: Representations of the DIKW hierarchy~// J.~Inf. 
Sci., 2007. Vol.~33. Iss.~2. P.~163--180. doi: 10.1177/0165551506070706.
\bibitem{11-zac} 
\Au{Frick$\acute{\mbox{e}}$~M.\,H.} Data--Information--Knowledge--Wisdom (DIKW) pyramid, 
framework, continuum~// Encyclopedia of big data~/ Eds. L.~Schintler, C.~McNeely.~--- Cham: 
Springer, 2018. 4~p. doi: 10.1007/978-3-319-32001-4\_331-1.
\bibitem{12-zac}
\Au{Denning P., Rosenbloom~P.} Computing: The fourth great domain of science~// Commun. 
ACM, 2009. Vol.~52. Iss.~9. P.~27--29.
\bibitem{13-zac}
\Au{Denning P., Freeman~P.} Computing's paradigm~// Commun.  ACM, 2009. Vol.~52. 
Iss.~12. P.~28--30. doi: 10.1145/ 1610252.1610265.
\bibitem{17-zac} %14
\Au{Farradane J.} Knowledge, information, and information science~// J.~Inf. Sci., 
1980. Vol.~2. Iss.~2. P.~75--80. doi: 10.1177/01655515800020020.

\bibitem{15-zac}
\Au{Шрейдер Ю.\,А.} Информация и~знание~// Сис\-тем\-ная концепция информационных 
процессов.~--- М.: ВНИИСИ, 1988. С.~47--52.
\bibitem{16-zac}
\Au{Ingwersen P.} Information and information science~// Enclyclopaedie of library and 
information science~/ Eds. J.\,D.~McDonald, 
M.~Levine-Clark.~--- New York, NY, USA: Marcel Dekker Inc., 1992. Vol.~56. Sup.~19. 
P.~137--174.

\bibitem{14-zac} %17
Информатика как наука об информации: Информационный, документальный, 
технологический, экономический, социальный и~организационный аспекты~/ Под ред. 
Р.\,С.~Гиляревского.~--- М.: Фаир-Пресс, 2006. 592~с.

\bibitem{18-zac}
\Au{Hjorland B.} Library and information science: practice, theory, and philosophical basis~// 
Inform. Process. Manag., 2000. Vol.~36. Iss.~3. P.~501--531. doi:  
10.1016/S0306-\mbox{4573(99)00038-2}.
\bibitem{19-zac}
Deep shift~--- technology tipping points and societal impact.~--- Geneva: WE Forum, 2015. 44~p. 
{\sf http://www3.weforum.org/docs/WEF\_GAC15\_ Technological\_Tipping\_Points\_report\_2015.pdf}.
\bibitem{20-zac}
\Au{Berman F., Rutenbar~R., Hailpern~B., Christensen~H., Davidson~S., Estrin~D., 
Franklin~M., Martonosi~M., Raghavan~P., Stodden~V., Szalay~A.\,S.} Realizing the potential of 
data science~// Commun.  ACM, 2018. Vol.~61. Iss.~4. P.~67--72. doi: 10.1145/3188721.

\bibitem{21-zac}
\Au{Stodden V.} The data science life cycle: A~disciplined approach to advancing data science as 
a~science~// Commun.  ACM, 2020. Vol.~63. Iss.~7. P.~58--66. doi: 10.1145/ 3360646.


\bibitem{23-zac} %22
\Au{Зацман И.\,М.} Научная парадигма информатики: классификация трансформаций 
объектов предметной об\-ласти~// Системы и~средства информатики, 2023. Т.~33. №\,4. 
С.~126--138. doi: 10.14357/08696527230412. EDN: ZIKUWO.

\bibitem{22-zac} %23
\Au{Зацман И.\,М.} Научная парадигма информатики: классификация объектов предметной  
об\-ласти~// Информатика и~её применения, 2023. Т.~17. Вып.~4. С.~96--103. doi: 
10.14357/19922264230413. EDN: FIUQAT.

\bibitem{24-zac}
\Au{Зацман И.\,М.} О~научной парадигме информатики: верхний уровень классификации 
объектов ее предметной об\-ласти~// Информатика и~её применения, 2022. Т.~16. Вып.~4. 
С.~73--79. doi: 10.14357/ 19922264220411. EDN: XZNKVI.

\bibitem{25-zac}
\Au{Соломоник А.\,Б.} Философия знаковых систем и~язык.~--- М.: ЛКИ, 2011. 408~с.
\bibitem{26-zac}
\Au{Зацман И.\,М.} Трансформация иерархии Акоффа в~научной парадигме информатики~// 
Информатика и~её применения, 2023. Т.~17. Вып.~3. С.~107--113. doi: 
10.14357/19922264230315. EDN: UMVRRV.

\bibitem{27-zac}
\Au{Zatsman I.} Building digital spiral models of knowledge generation~// 19th Forum 
(International) on Knowledge Asset Dynamics Proceedings.~--- Matera, Italy: Arts for Business 
Institute, 2024. P.~2185--2196.
\bibitem{28-zac}
\Au{Zatsman I.} Digital spiral model of knowledge creation and encoding its dynamics~// 18th 
Forum (International) on Knowledge Asset Dynamics Proceedings.~--- Matera, Italy: Arts for 
Business Institute, 2023. P.~581--596.
\bibitem{29-zac}
\Au{Зацман И.\,М.} Интерфейсы третьего порядка в~информатике~// Информатика и~её 
применения, 2019. Т.~13. Вып.~3. С.~82--89. doi: 10.14357/19922264190312. EDN: 
EHRQLF.

\bibitem{30-zac}
\Au{Зацман И.\,М.} Научная парадигма информатики как третьей культуры~//  
На\-уч\-но-тех\-ни\-че\-ская информация. Сер.~1: Организация и~методика информационной 
работы, 2023. №\,11. С.~1--14.

\end{thebibliography}

 }
 }

\end{multicols}

\vspace*{-9pt}

\hfill{\small\textit{Поступила в~редакцию 14.04.24}}

\vspace*{4pt}

%\pagebreak

%\newpage

%\vspace*{-28pt}

\hrule

\vspace*{2pt}

\hrule



\def\tit{OBJECT TRANSFORMATIONS OF~THE~FIRST AND~SECOND ORDER
IN~A~LEXICOGRAPHIC INFORMATION SYSTEM\\[-5pt]}


\def\titkol{Object transformations of~the~first and~second order
in~a~lexicographic information system}


\def\aut{I.\,M.~Zatsman}

\def\autkol{I.\,M.~Zatsman}

\titel{\tit}{\aut}{\autkol}{\titkol}

\vspace*{-13pt}


\noindent
Federal Research Center ``Computer Science and Control'' of the Russian Academy of Sciences, 
44-2~Vavilov Str., Moscow 119133, Russian Federation


\def\leftfootline{\small{\textbf{\thepage}
\hfill INFORMATIKA I EE PRIMENENIYA~--- INFORMATICS AND
APPLICATIONS\ \ \ 2024\ \ \ volume~18\ \ \ issue\ 2}
}%
 \def\rightfootline{\small{INFORMATIKA I EE PRIMENENIYA~---
INFORMATICS AND APPLICATIONS\ \ \ 2024\ \ \ volume~18\ \ \ issue\ 2
\hfill \textbf{\thepage}}}

\vspace*{2pt}



\Abste{The theoretical foundations of the design of information technologies used for 
the integration of bilingual dictionaries and parallel corpora are considered. The 
description of the first outcomes of the creation of the third\linebreak\vspace*{-12pt}}

\Abstend{ level of object 
transformations classification in the subject domain of informatics, which is supposed 
to be used
in creating the lexicographic information system providing integration, is 
given. All the entities of informatics are divided into two global classes: objects and 
their transformations. For each such class, its own classification is constructed. 
Previously, the two upper levels of the object transformation classification in the subject 
domain have been described. The present paper discusses the third level of this classification. The 
basis for the construction of its highest level was the division of the subject domain of 
informatics into media (mental, sensory, digital, and a~number of other media), each 
of which by definition includes objects of the same nature. The Solomonick's 
typology of sign systems served as the basis for constructing the second level of the 
object transformation classification. The aim of the paper is to systematize object 
transformations of the first and second orders at the third level of this classification. 
The basis for systematization is the medium version of the Ackoff's hierarchy.}

\KWE{subject domain objects; object transformations; classification; data; 
information; knowledge; lexicographic information system}


\DOI{10.14357/19922264240211}{VZTGVV}

\vspace*{-12pt}

\Ack

\vspace*{-3pt}


\noindent
The reported study was funded by the Russian Science Foundation, project  
No.\,24-18-00155, {\sf 
https://rscf.ru/project/24-18-00155}. The research was carried out using the infrastructure of the Shared 
Research Facilities ``High Performance Computing and Big Data'' (CKP 
``Informatics'') of FRC CSC RAS (Moscow) .
   


  \begin{multicols}{2}

\renewcommand{\bibname}{\protect\rmfamily References}
%\renewcommand{\bibname}{\large\protect\rm References}

{\small\frenchspacing
 {%\baselineskip=10.8pt
 \addcontentsline{toc}{section}{References}
 \begin{thebibliography}{99} 
\bibitem{1-zac-1}
\Aue{Aijmer, K., and B.~Altenberg.} 2013. \textit{Advances in corpus-based 
contrastive linguistics. Studies in honour of Stig Johansson}. Amsterdam: John 
Benjamins. 295~p. doi: 10.1075/scl.54.
\bibitem{2-zac-1}
\Aue{Dobrovolskiy, D.\,O., A.\,A.~Kretov, and S.\,A.~Sharov.} 2005. Korpus 
parallel'nykh tekstov [Corpus of parallel texts]. \textit{Nauchnaya i~tekhnicheskaya 
informatsiya. Ser. 2. Informatsionnye protsessy i~sistemy} [Scientific and Technical 
Information. Ser.~2: Information Processes and Systems] 6:16--27.
\bibitem{3-zac-1}
\Aue{Dobrovolskiy, D.\,O.} 2015. Korpus parallel'nykh tekstov i~sopostavitel'naya 
leksikologiya [The corpus of parallel texts and contrastive lexicology]. \textit{Trudy 
Instituta russkogo yazyka im. V.\,V.~Vinogradova} [Proceedings of the 
V.\,V.~Vinogradov Russian Language Institute] 6:413--449. EDN: VJQBHP.
\bibitem{4-zac-1}
\Aue{Goncharov, A.\,A., I.\,M.~Zatsman, and M.\,G.~Kruzhkov.} 2020. Evolyutsiya 
klassifikatsiy v~nadkorpusnykh ba\-zakh dannykh [Evolution of classifications in 
supracorpora databases]. \textit{Informatika i~ee Primeneniya~--- Inform. \mbox{Appl.}}  
14(4):108--116. doi: 10.14357/19922264200415.  
EDN: GKWBZT.
\bibitem{5-zac-1}
\Aue{Goncharov, A.\,A., I.\,M.~Zatsman, and M.\,G.~Kruzhkov.} 2021. 
Predstavlenie novykh leksikograficheskikh znaniy v~dinamicheskikh 
klassifikatsionnykh sistemakh [Representation of new lexicographical knowledge in 
dynamic classification systems]. \textit{Informatika i~ee Primeneniya~--- Inform. 
Appl.} 15(1):86--93. doi: 10.14357/19922264210112. EDN: OPEFXW.
\bibitem{6-zac-1}
\Aue{Zatsman, I.} 2020. Finding and filling lacunas in linguistic typologies. 
\textit{15th Forum (International) on Knowledge Asset Dynamics Proceedings}. 
Matera, Italy: Institute of Knowledge Asset Management. 780--793.
\bibitem{7-zac-1}
\Aue{Zatsman, I.} 2020. Three-dimensional encoding of emerging meanings in  
AI-systems. \textit{21st European Conference on Knowledge Management 
Proceedings}. Reading, U.K.: Academic Publishing International Ltd. 878--887.
\bibitem{8-zac-1}
\Aue{Ackoff, R.} 1989. From data to wisdom. \textit{J.~Applied Systems Analysis} 
16(1):3--9.
\bibitem{9-zac-1}
\Aue{Rosenbloom, P.\,S.} 2013. \textit{On computing: The fourth great scientific 
domain}. Cambridge, MA: MIT Press. 307~p.
\bibitem{10-zac-1}
\Aue{Rowley, J.} 2007. The wisdom hierarchy: Representations of the DIKW 
hierarchy. \textit{J.~Inf. Sci.} 33(2):163--180. doi: 10.1177/0165551506070706.
\bibitem{11-zac-1}
\Aue{Frick$\acute{\mbox{e}}$, M.\,H.} 2018.  
Data-Information-Knowledge-Wisdom (DIKW) pyramid, framework, continuum. 
\textit{Encyclopedia of big data}. Eds. L.~Schintler and C.~McNeely. Cham: 
Springer. 4~p. doi: 10.1007/978-3-319-32001- 4\_331-1.
\bibitem{12-zac-1}
\Aue{Denning, P., and P.~Rosenbloom.} 2009. Computing: The fourth great domain 
of science. \textit{Commun. ACM} 52(9):27--29.
\bibitem{13-zac-1}
\Aue{Denning, P., and P.~Freeman.} 2009. Computing's paradigm. \textit{Commun. 
ACM} 52(12):28--30. doi: 10.1145/ 1610252.1610265.

\bibitem{17-zac-1} %14
\Aue{Farradane, J.} 1980. Knowledge, information, and information science. 
\textit{J.~Inf. Sci.} 2(2):75--80. doi: 10.1177/ 01655515800020020.

\bibitem{15-zac-1}
\Aue{Shreyder, Yu.\,A.} 1988. Informatsiya i~znanie [Information and knowledge]. 
\textit{Sistemnaya kontseptsiya in\-for\-ma\-tsi\-on\-nykh protsessov} [System concept of 
information processes]. Moscow: VNIISI. 47--52.
\bibitem{16-zac-1}
\Aue{Ingwersen, P.} 1995. Information and information science. 
\textit{Encyclopedia of library and information science}. Eds. J.\,D.~McDonald and 
M.~Levine-Clark. New York, NY: Marcel Dekker Inc. 56(19):137--174.

\bibitem{14-zac-1} %17
Gilyarevskiy, R.\,S., ed. 2006. \textit{Informatika kak nauka ob informatsii: 
informatsionnyy, dokumental'nyy, tekh\-no\-lo\-gi\-che\-skiy, ekonomicheskiy, sotsial'nyy 
i~organizatsionnyy aspekty} [Informatics as information science: Informational, 
documentary, technological, economic, social, and organizational dimensions]. 
Moscow: FAIR-PRESS. 592~p.

\bibitem{18-zac-1}
\Aue{Hjorland, B.} 2000. Library and information science: Practice, theory, and 
philosophical basis. \textit{Inform. Process. Manag.} 36(3):501--531. doi:  
10.1016/S0306-\mbox{4573(99)00038-2}.
\bibitem{19-zac-1}
Deep shift~--- technology tipping points and societal impact. 2015. \textit{World Economic 
Forum}. Geneva. 44~p. Available at: {\sf 
http://www3.weforum.org/docs/WEF\_ GAC15\_Technological\_Tipping\_Points\_report\_2015.pdf} (accessed May~20, 
2024).
\bibitem{20-zac-1}
\Aue{Berman, F., R.~Rutenbar, B.~Hailpern, H.~Christensen, S.~Davidson, 
D.~Estrin, M.~Franklin, M.~Martonosi, P.~Raghavan, V.~Stodden, and 
A.\,S.~Szalay.} 2018. Realizing the potential of data science. \textit{Commun. ACM} 
61(4):67--72. doi: 10.1145/3188721.
\bibitem{21-zac-1}
\Aue{Stodden, V.} 2020. The data science life cycle: A~disciplined approach to 
advancing data science as a~science. \textit{Commun. ACM} 
 63(7):58--66. doi: 10.1145/3360646.

\bibitem{23-zac-1} %22
\Aue{Zatsman, I.\,M.} 2023. Nauchnaya paradigma informatiki: klassifikatsiya 
transformatsiy ob''ektov predmetnoy oblasti [Scientific paradigm of informatics: 
Transformation classification of domain objects]. \textit{Sistemy i~Sredstva 
Informatiki~--- Systems and Means of Informatics} 33(4):126--138. doi: 
10.14357/08696527230412. EDN: ZIKUWO.

\bibitem{22-zac-1} %23
\Aue{Zatsman, I.\,M.} 2023. Nauchnaya paradigma informatiki: klassifikatsiya 
ob''ektov predmetnoy oblasti [Scientific paradigm of informatics: Classification of 
domain objects]. \textit{Informatika i~ee Primeneniya~--- Inform. Appl.} 
 17(4):96--103. doi: 10.14357/19922264230413. EDN: FIUQAT.
 
\bibitem{24-zac-1}
\Aue{   Zatsman, I.\,M.} 2022. O nauchnoy paradigme informatiki: verkhniy uroven' 
klassifikatsii ob''ektov ee predmetnoy oblasti [On the scientific paradigm of 
informatics: The classification high level of its objects]. \textit{Informatika i~ee 
Primeneniya~--- Inform. Appl.} 16(4):73--79. doi: 10.14357/19922264220411. EDN: 
XZNKVI.
\bibitem{25-zac-1}
\Aue{Solomonick, A.\,B.} 2011. \textit{Filosofiya znakovykh system i~yazyk} 
[Philosophy of sign systems and language]. Moscow: LKI. 408~p.
\bibitem{26-zac-1}
\Aue{Zatsman, I.\,M.} 2023. Transformatsiya ierarkhii Akoffa v~nauchnoy 
paradigme informatiki [Transformation of the Ackoff's hierarchy in the scientific 
paradigm of informatics]. \textit{Informatika i~ee Primeneniya~--- Inform. \mbox{Appl.}} 
17(3):107--113. doi: 10.14357/19922264230315. EDN: UMVRRV.
\bibitem{27-zac-1}
\Aue{Zatsman, I.} 2024. Building digital spiral models of knowledge 
generation. \textit{19th Forum (International) on Knowledge Asset Dynamics 
Proceedings}. Matera, Italy: Arts for Business Institute. 2185--2196.
\bibitem{28-zac-1}
\Aue{Zatsman, I.} 2023. Digital spiral model of knowledge creation and encoding its 
dynamics. \textit{18th Forum (International) on Knowledge Asset Dynamics 
Proceedings}. Matera, Italy: Arts for Business Institute. 581--596.
\bibitem{29-zac-1}
\Aue{Zatsman, I.\,M.} 2019. Interfeysy tret'ego poryadka v~informatike 
 [Third-order interfaces in informatics]. \textit{Informatika i~ee Primeneniya~--- 
Inform. Appl.} 13(3):82--89. doi: 10.14357/19922264190312. EDN: EHRQLF.
\bibitem{30-zac-1}
\Aue{Zatsman, I.} 2023. Scientific paradigm of informatics as a~third culture. 
\textit{Scientific Technical Information Processing} 50(4):246--258. doi: 
10.3103/S0147688223040111. EDN: CKHMYS.

\end{thebibliography}

 }
 }

\end{multicols}

\vspace*{-6pt}

\hfill{\small\textit{Received April 14, 2024}} 


\vspace*{-12pt}


\Contrl

\vspace*{-3pt}

\noindent
\textbf{Zatsman Igor M.} (b.\ 1952)~--- Doctor of Science in technology, head of 
department, Federal Research Center ``Computer Science and Control'' of the 
Russian Academy of Sciences, 44-2~Vavilov Str., Moscow 119333, Russian 
Federation; \mbox{izatsman@yandex.ru}





\label{end\stat}

\renewcommand{\bibname}{\protect\rm Литература}       %13
\def\stat{hvatova}

\def\tit{МОДЕЛИРОВАНИЕ СТОХАСТИЧЕСКОЙ ДИНАМИКИ ИЗМЕНЕНИЯ СОСТОЯНИЙ УЗЛОВ 
И~ПЕРКОЛЯЦИОННЫХ ПЕРЕХОДОВ В~СОЦИАЛЬНЫХ СЕТЯХ С~УЧЕТОМ 
САМООРГАНИЗАЦИИ И~НАЛИЧИЯ ПАМЯТИ}

\def\titkol{Моделирование стохастической динамики изменения состояний узлов 
и~перколяционных переходов} % в~социальных сетях с~учетом самоорганизации и~наличия памяти}

\def\aut{Д.\,О.~Жуков$^1$, Т.\,Ю.~Хватова$^2$, А.\,Д.~Зальцман$^3$}

\def\autkol{Д.\,О.~Жуков, Т.\,Ю.~Хватова, А.\,Д.~Зальцман}

\titel{\tit}{\aut}{\autkol}{\titkol}

\index{Жуков Д.\,О.}
\index{Хватова Т.\,Ю.}
\index{Зальцман А.\,Д.}
\index{Zhukov D.\,O.}
\index{Khvatova T.\,Yu.}
\index{Zaltcman A.\,D.}

%{\renewcommand{\thefootnote}{\fnsymbol{footnote}} \footnotetext[1]
%{Финансовое обеспечение исследований осуществлялось из средств федерального бюджета на 
%выполнение государственного задания Карельского научного центра Российской академии наук 
%(Институт прикладных математических исследований КарНЦ РАН).}}

\renewcommand{\thefootnote}{\arabic{footnote}}
\footnotetext[1]{МИРЭА~--- Российский технологический университет, zhukov\_do@mirea.ru}
\footnotetext[2]{Санкт-Петербургский политехнический университет Петра Великого, \mbox{khvatova.ty@spbstu.ru}}
\footnotetext[3]{МИРЭА~--- Российский технологический университет, ad.zaltcman@gmail.com}


\vspace*{-6pt}



\Abst{Ообсуждаются вопросы использования подходов теоретической информатики 
и~применение ее приложений для анализа и~моделирования процессов в~социотехнических 
системах (социальных сетях). Разработана стохастическая модель динамики изменения 
состояний (настроений или мнений) пользователей (узлов) и~достижения порога перколяции 
в~социальной сети, имеющей случайные связи между узлами. Модель показывает 
возможность скачкообразных переходов между состояниями (мнений, настроений и~т.\,д.)\ 
узлов в~социальной сетевой структуре в~течение короткого времени без внешнего 
воздействия, что может быть связано с~памятью о предыдущих состояниях 
и~самоорганизацией. При создании модели были рассмотрены схемы вероятностей 
переходов между возможными состояниями узлов с~учетом предыдущих шагов 
(немарковские процессы с~наличием памяти) и~выведено нелинейное дифференциальное 
уравнение второго порядка, которое содержит член, отвечающий за возможность 
самоорганизации, а также сформулирована и~решена граничная задача для определения 
функции плотности вероятности нахождения системы в~определенном состоянии с~течением 
времени. Разработанная модель может быть связана с~полученными ее авторами ранее 
результатами описания процессов в~социальных сетевых структурах с~помощью теории 
перколяции (определение времени достижения пороговых значений доли узлов сети, при 
котором мнения или предпочтения могут беспрепятственно распространяться по сети 
в~целом).}

\KW{стохастическая динамика; состояния узлов социальной сети; самоорганизация систем; 
процессы с~памятью; перколяция в~социальных сетях}

\DOI{10.14357/19922264210114}

\vspace*{-4pt}

\vskip 10pt plus 9pt minus 6pt

\thispagestyle{headings}

\begin{multicols}{2}

\label{st\stat}

\section{Введение}

  Отличительной особенностью динамики явлений в~социотехнических 
и~социальных системах является активное воздействие на них человеческого 
фактора, который, с~одной стороны, вносит неопределенность и~создает 
стохастичность, а~с~другой стороны, создает возможности для 
самоорганизации, позволяет говорить о наличии памяти и~придает динамике 
процессов существенно нелинейный характер.
  
  Для моделирования нелинейной динамики самоорганизующихся социальных 
систем с~памятью можно и~нужно применять методы и~средства теоретической 
информатики и~кибернетики, которая, по определению Роберта Винера, 
является наукой об управлении не только техническими, но и~биологическими 
системами.
  
  Использование методов теоретической информатики, разработанных в~ней 
принципов моделирования и~ее приложений может позволить получить 
качественно новые результаты для описания сложных социальных, 
экономических и~социотехнических систем, а также создать новые методики 
прогнозирования поведения людей в~социальных и~социотехнических системах.

%\vspace*{-6pt}
  
\section{Обзор некоторых моделей описания динамики процессов 
в~социальных сетевых структурах}

%\vspace*{-2pt}

  Многие из существующих теоретических подходов к~описанию социальных 
сетевых систем имеют много общего с~кинетическим описанием физических 
систем и~распространением вирусов в~компьютерных сетях. Однако эти модели 
в~основном рассматривают цепные явления, 
где макроскопическая доля узлов с~определенным состоянием в~сети быстро возникает из некоторого 
микроскопического состояния, захватывающего все новые и~новые узлы.
  
  В более сложных моделях взаимодействие пользователей социальных сетей 
описывается теорией многоагентных систем~[1--3], а также аппаратом теории 
клеточных автоматов~[4, 5]. В этих моделях на основании некоторых правил 
переходов агенты принимают определенные состояния, образуют связанную по 
своим свойствам группу, могут сотрудничать, чтобы решить некую задачу или 
достигнуть определенной цели~\cite{1-hv}, а~временн$\acute{\mbox{а}}$я логика поведения 
агентов может зависеть от динамически меняющихся условий~\cite{2-hv}.
  
  В работе~\cite{4-hv} было изучено влияние структуры сетей (случайные 
структуры, маленькие миры, цикл, колесо, звезда, иерархическая) и~правил 
поведения клеточных автоматов на динамику процессов в~социальных сетях. 
При одинаковых правилах взаимодействия клеток динамика процессов зависит 
от топологии сети (наибольшая скорость наблюдается в~регулярных 
структурах, а наименьшая~--- в~неупорядоченных).
  
  Для описания процессов в~социальных сетях также применяются 
стохастические подходы, учитывающие зависимости изменения состояния 
узлов от времени. В~работе~[6] описана модель смешанного членства 
в~стохастически формирующихся группах, основанная на попарном 
рассмотрении присутствия или отсутствия связей между объектами. Анализ 
вероятностных изменений связей требует специальных предположений, 
например независимости или предположения непостоянства данной связи 
(смешанного членства в~стохастически формирующихся группах). Данная 
модель позволяет описать динамику кластеризации членов по группам 
и~изменение их численности.
  
  Другим направлением анализа и~прогнозирования динамики процессов 
в~сложных социальных системах является использование нестационарных 
временн$\acute{\mbox{ы}}$х рядов. Традиционный подход к~их анализу основан на том, чтобы 
с~помощью применения линейных методов свести их к~стационарным 
(например, авторегрессионные интегрированные модели скользящего 
среднего~--- ARIMA, autoregressive integrated moving average~[7]). Эти модели оперируют не функциями 
распределения, а~непосредственно элементами временн$\acute{\mbox{о}}$го ряда. Ряды, не 
укла\-ды\-ва\-ющи\-еся в~рамки регрессионного анализа, чаще\linebreak всего изучаются 
адаптивными эвристическими методами, в~которых ряды на некоторой длине 
описываются стационарной моделью типа регрессии или авторегрессии, 
а~параметры модели пересчитываются с~учетом новой информации или 
с~учетом сравнения предсказанного значения с~фактом. Недостаток этих 
подходов заключается в~том, что длина участка возможной стационарности не 
является известной величиной. При исследовании стационарных случайных 
процессов, согласно теореме Гливенко~[8] (о~сходимости эмпирической 
вероятности к~теоретическому распределению), чем больше учтено 
наблюдаемых значений, тем точнее будут получены теоретические 
характеристики распределения. Для нестационарных временн$\acute{\mbox{ы}}$х рядов данное 
условие, в~силу их специфики, не может быть выполнено, что затрудняет 
возможности прогнозирования.
  
  Следует отметить, что ни одна из существующих моделей не рассматривает 
самоорганизацию и~возможность наличия памяти. Поэтому можно сделать 
вывод о том, что для прогнозирования динамики процессов в~социотехнических 
системах, имеющих сетевую структуру, требуется продолжение разработки их 
моделей с~учетом структурных свойств, самоорганизации и~наличия памяти.

\vspace*{-3pt}
  
\section{Постановка задач исследования}

   С~позиций структурного подхода социотехнические системы представляют 
собой случайную сеть взаимосвязей и~взаимодействий пользователей, которая 
приводит к~нелинейной динамике изменения состояний узлов. При 
моделировании нелинейных динамических процессов в~социальных сетевых 
структурах необходимо ответить минимум на два важных вопроса.  
Во-пер\-вых, как учесть стохастичность, неопределенность, самоорганизацию 
процессов и~наличие памяти и~как они влияют на наблюдаемые явления.  
Во-вто\-рых, как структура сетей влияет на динамику процессов и~как она 
может быть связана со стохастичностью, неопределенностью, 
самоорганизацией процессов и~наличием памяти. Ответы на эти вопросы могут 
позволить создать эффективные алгоритмы мониторинга состояния социальных 
сетей.
   
  В~теории перколяции (теория вероятностей на графах) изучают решение 
задач узлов и~связей для сетей с~различной структурой. При решении задачи 
связей определяют долю связей, которую нужно разорвать, чтобы сеть 
распалась минимум на две несвязанные части (или, наоборот, долю 
проводящих связей в~сети, когда в~целом между любыми произвольными 
узлами появляется проводимость). В~задаче узлов определяют среднюю долю 
блокированных узлов, при которой сеть распадется на не связанные между 
собой кластеры, внутри которых сохраняются связи (или, наоборот, долю 
проводящих узлов, когда проводимость возникает). Доля блокированных узлов 
(в~задаче узлов) или разорванных связей (в~задаче связей), при которой 
исчезает проводимость (или, наоборот, появляется) между двумя произвольно 
выбранными узлами сети, называется порогом перколяции (протекания)~[9].
  
  Использование понятия долей блокированных узлов или связей эквивалентно 
понятию вероятности нахождения случайно выбранного узла (или связи) 
в~блокированном (разорванном) состоянии. Поэтому величина порога 
перколяции определяет вероятность передачи информации через всю сеть 
в~целом, если задана средняя вероятность блокирования узла или связи.
  
  Величину порога перколяции для случайной сетевой структуры можно 
определить теоретически методами численного моделирования или 
экспериментально при изучении реальных сетей \mbox{найти} с~по\-мощью 
инструментов социального сетевого анализа (SNA~--- social network analysis).
  
  Если для случайной сети социальных связей порог ее перколяции известен, 
то, описав механизмы перехода ее узлов в~блокированное или проводящее 
состояние, можно определить время его достижения, а~следовательно, 
спрогнозировать динамику распространения определенных мнений или 
взглядов.
  
\section{Перколяционные свойства случайных сетевых структур}

  Исследования~\cite{1-hv, 10-hv, 11-hv, 12-hv, 13-hv, 14-hv} показывают, что 
пороги перколяции случайных сетей зависят от среднего числа связей в~расчете 
на один узел (плотности) сети. Для задачи связей имеется линейная 
зависимость: $y\hm= -6{,}581z\hm- 0{,}203$; а~для задачи узлов: $y\hm= 4{,}39 
z\hm- 2{,}41$. Здесь $z\hm=1/x,$ где~$x$~--- плот\-ность связей; $y$~--- натуральный 
логарифм доли разорванных связей (или узлов в~задаче узлов), при которой 
исчезает проводимость всей сети  
в~целом~\cite{11-hv, 12-hv, 13-hv, 14-hv, 15-hv, 16-hv, 17-hv}.
  
  Полученные ранее  
результаты~~\cite{11-hv, 12-hv, 13-hv, 14-hv, 15-hv, 16-hv, 17-hv} 
о~перколяционных свойствах случайных сетей позволяют сделать ряд очень 
важных выводов. Например, о наличии насыщения порога перколяции, о роли 
увеличения плотности связей в~информационном влиянии сети и~ряд других. 
Следует отметить, что динамика изменения состояния узлов сетей 
в~совокупности с~их перколяционными свойствами была рассмотрена 
в~работе~\cite{10-hv}, где исследовалось распространение компьютерных 
вирусов. Однако влияние процессов самоорганизации и~наличия памяти на 
динамику изменения состояния узлов и~достижение порогов перколяции 
с~течением времени исследовано не было.
  
\section{Стохастическая динамика переходов между состояниями 
в~сетях социальных связей и~достижение порога перколяции 
с~учетом памяти и~самоорганизации}

\subsection{Построение разностных вероятностных схем переходов 
между~состояниями} %5.1

  Будем описывать социальную сеть как систему, состояния которой в~любой 
момент времени могут быть заданы параметром, принимающим непрерывные 
случайные значения с~недетерминированным законом распределения. 
Например, это может быть доля пользователей (узлов сети), раз\-де\-ля\-ющих 
и~пропагандирующих определенные взгляды или настроения.
  
  Все множество состояний будем обозначать как~$X$. Состояние, 
наблюдаемое в~момент времени~$t$, можно обозначить как~$x_i$ ($x_i\hm\in  
{X}$).
  
  Введем интервал времени~$\tau_0$, за который возможно изменение 
состояния~$x_i$. В~данном случае любое значение текущего времени 
$t\hm=h\tau_0$, где~$h$~--- номер шага перехода между состояниями 
(процесс перехода между состояниями становится квазинепрерывным 
с~бесконечно малым временным интервалом~$\tau_0$); $h\hm=0, 1, 2, \ldots, 
N$. Текущее состояние~$x_i$ на шаге~$h$ после перехода на шаг $h\hm+1$ 
может \textit{за счет случайно возникающих факторов} увеличиваться на 
некоторую величину~$\varepsilon$ или уменьшаться на величину~$\xi$, т.\,е.\ 
оказаться равным $x_i\hm+\varepsilon$ или $x_i\hm- \xi$ 
соответственно. Введем понятие вероятности нахождения системы в~том или 
ином состоянии: после некоторого числа шагов~$h$ про описываемую систему 
можно сказать, что ${\sf P}(x\hm- \varepsilon, h)$~--- вероятность того, что она 
находится в~состоянии $(x\hm - \varepsilon)$; ${\sf P}(x,h)$~--- вероятность того, что 
она находится в~состоянии~$x$; ${\sf P}(x\hm+\xi, h)$~--- вероятность того, что она 
находится в~состоянии $(x\hm + \xi)$.
  
  После каждого шага состояние~$x_i$ (далее индекс~$i$ для краткости 
опустим) может изменяться на величину~$\varepsilon$ или~$\xi$. Вероятность 
${\sf P}(x, h\hm+1)$ того, что на следующем, $(h\hm+1)$-м, шаге система (или 
процесс) окажется в~состоянии~$x$ описывается уравнением (рис.~1):
  \begin{equation}
  {\sf P}(x, h+1)= {\sf P}(x - \varepsilon, h)+{\sf P}(x+\xi, h) - {\sf P}(x, h)\,.
  \label{e1-hv}
  \end{equation}



  Поясним уравнение~(1) и~представленную на рис.~1 схему. Вероятность 
перехода в~состояние~$x$ на\linebreak\vspace*{-12pt}

{ \begin{center}  %fig1
 \vspace*{-1pt}
    \mbox{%
\epsfxsize=77.502mm
\epsfbox{hva-1.eps}
}

\end{center}

\noindent
{{\figurename~1}\ \ \small{
Схема возможных переходов между состояниями системы (или процесса) на ($h + 1$)-м шаге
}}}


\vspace*{14pt}

\addtocounter{figure}{1}

\noindent
 ($h\hm+1$)-м шаге ${\sf P}(x, h\hm+1)$ определяется 
суммой вероятностей перехода в~это состояние из состояний ($x\hm-
\varepsilon$): ${\sf P}(x-\varepsilon, h)$ и~$(x+\xi)$: ${\sf P}(x+\xi, h)$, в~которых 
находилась система на шаге~$h$ за вычетом вероятности ${\sf P}(x, h)$ перехода 
системы из состояния~$x$ (в~котором она находилась на шаге~$h$) в~любое 
другое состояние на $(h\hm+1)$-м шаге. В~реальности в~социальной сети 
всегда остается память о предыдущих состояниях. Для учета этого определим 
вероятности ${\sf P}(x\hm- \varepsilon,h)$, ${\sf P}(x\hm+\xi, h)$ и~${\sf P}(x, h)$ через 
состояния на $(h\hm-1)$-м шаге. Схемы соответствующих переходов можно 
изобразить аналогично схеме, представленной на рис.~1, и~получить для 
вероятности перехода следующее алгебраическое уравнение:
  \begin{multline*}
  {\sf P}(x,h+2)={}\\
  {}=\{ {\sf P}(x-2\varepsilon, h)+{\sf P}(x-\varepsilon+\xi, h) -{\sf P}(x-\varepsilon, 
h)\}+{}\\
  {}+ \{ {\sf P}(x+\xi-\varepsilon,h)+ {\sf P}(x+\xi,h) -{\sf P}(x+\xi,h)\} -{}\\
  {}-{\sf P}(x-\varepsilon,h)- {\sf P}(x+\xi, h-1)+ {\sf P}(x,h)\,.
 % \label{e2-hv}
\end{multline*}
Далее, учитывая, что $t\hm=h\tau_0$, перейдем от~$h$ к~$t$, а~затем проведем 
соответствующие разложения в~ряд Тейлора и~получим:
\begin{equation}
\fr{d{\sf P}(x,t)}{dt} =a\fr{ d^2{\sf P}(x,t)}{dx^2} -b\fr{ d{\sf P}(x,t)}{dx} -c\fr{ 
d^2{\sf P}(x,t)}{dt^2}\,,
\label{e3-hv}
\end{equation}
где 
$$
a=\fr{\varepsilon^2-\varepsilon\xi+\xi^2}{\tau_0}\,;\enskip b=\fr{\varepsilon -
\xi}{\tau_0}\,;\enskip c-\tau_0\,.
$$
  
  Член уравнения $d{\sf P}(x,t)/dx$ описывает упорядоченный переход либо 
в~состояние, когда оно увеличивается ($\varepsilon\hm > \xi$), либо когда оно 
уменьшается ($\varepsilon \hm <\xi$); член $d^2{\sf P}(x,t)/dx^2$ описывает 
случайное изменение состояния (неопределенность изменения). Член 
$d{\sf P}(x,t)/dt$ определяет скорость общего изменения состояния системы 
с~течением времени; член $d^2{\sf P}(x,t)/dt^2$ описывает процесс, при котором 
состояния сами становятся источниками возникновения других состояний 
(\textit{самоорганизация} и~ускорение упорядоченных и~случайных переходов).
  
  Сравним полученный результат с~су\-щест\-ву\-ющи\-ми моделями анализа 
и~описания поведения нестационарных временн$\acute{\mbox{ы}}$х рядов. В~настоящее время 
в~качестве аппроксимаций выборочных распределений чаще всего 
используются диффузионные уравнения, включая нелинейную 
диффузию~\cite{18-hv}:
  $$
  \fr{\partial \rho(x,t)}{\partial t} =\fr{D(t)\partial^2\rho^{(n-1)/(n+1)}(x,t)}{\partial x^2}\,,
  $$
  где $n$~--- числовой параметр модели; $(n\hm-1)/(n\hm+1)$~--- показатель 
степени функции плотности распределения~$\rho(x,t)$ (это уравнение 
учитывает только случайные переходы); уравнение Лиувилля~\cite{18-hv}: 
  $$
  \fr{\partial \rho(x,t)}{\partial t}=-\fr{\partial \{ U(x,t)\rho(x,t)\}}{\partial x}\,,
  $$
  которое определяет упорядоченный перенос; уравнение  
Фок\-ке\-ра--План\-ка~\cite{19-hv}: 
  $$
  \fr{\partial\rho(x,t)}{\partial t}=\fr{D(t)}{2}\,\fr{\partial^2\rho(x,t)}{\partial 
x^2} - \fr{\partial \{ U(x,t)\rho(x,t)\}}{\partial x}\,,
  $$
  где $U(x,t)$~--- скорость <<сноса>>; $D(t)$~--- коэффициент диффузии (это 
уравнение учитывает не только случайное изменение (член 
$\partial^2\rho(x,t)/\partial x^2$), но и~упорядоченные переходы (член $\partial 
\{ U(x,t)\rho(x,t)\}/\partial x$), или <<снос>>) и~ряд других. Однако ни одна из 
этих моделей не рассматривает самоорганизацию и~память. Разработанная 
авторами модель обобщает другие модели, и~при равенстве нулю некоторых 
коэффициентов в~уравнении~(\ref{e3-hv}) оно переходит в~уравнения 
Лиувилля или  
Фок\-ке\-ра--План\-ка, которые выступают ее частными случаями.
  
\subsection{Формулировка и~решение краевой~задачи}

  Считая функцию~${\sf P}(x,t)$ непрерывной, можно перейти от вероятности 
${\sf P}(x,t)$ (уравнение~(\ref{e3-hv})) к~плотности вероятности $\rho(x,t)\hm= 
=d{\sf P}(x,t)/dx$ и~сформулировать граничную задачу, решение которой и~будет 
описывать процесс перехода между состояниями. Предположим, что 
необходимо, чтобы доля пользователей (узлов) социальной сети, имеющих 
негативное мнение, не превышала определенного значения (т.\,е.\ величина 
доли негативных настроений должна находиться на отрезке от~0 до величины 
порога перколяции~$l$ для данной сети).
  
  \textbf{Первое граничное условие}. Состояние $x\hm=0$ определяет полное 
отсутствие негативных мнений (доля равна~0). Сама вероятность обнаружить 
такое состояние может быть отлична от~0, однако плотность вероятности, 
определяющую поток в~состоянии $x\hm=0$, необходимо положить равной~0 
(состояния системы не могут быть отрицательными): $\rho(x,t)_{x=0}\hm=0$.
  
  \textbf{Второе граничное условие}. Вероятность обнаружить состояние 
с~максимально возможной долей негативно настроенных пользователей 
$x\hm=L\hm=1$ отлична от~0. Однако плотность вероятности, определяющая 
поток в~этом состоянии, необходимо положить равной~0 (величина состояния 
не может быть больше, чем максимально возможная доля): 
$\rho(x,t)_{x=L}\hm=0$.
  
  Поскольку в~момент времени $t\hm=0$ состояние системы уже может быть 
равно некоторому значению~$x_0$, то начальное условие зададим в~виде:
  \begin{multline*}
  \rho(x,t=0)=\delta(x-x_0)= {}\\
  {}=\begin{cases} 
  \displaystyle \int \delta(x-x_0)\,dx=1\,, &x=x_0\,;\\
  0\,, & x\not= x_0\,.
  \end{cases}
  \end{multline*}
  
  Второе начальное условие можно задать в~виде: 
  $$
  \fr{\partial G(x,t)}{\partial t}\Big\vert_{t=0}=0\,,
  $$ 
  так как начальное условие содержит дель\-та-функ\-цию; кроме того, ее 
наличие приводит к~тому, что решение для $\rho(x,t)$ разбивается на две 
области при $x\hm> x_0$ и~при $x\hm\leq x_0$. Используя методы 
операционного исчисления для плотности вероятности $\rho_1(x,t)$ 
и~$\rho_2(x,t)$ обнаружения состояния системы в~одном из значений на отрезке 
от~0 до~$L$, можно получить следующую систему уравнений:
  \begin{multline*}
  \mbox{при } x\geq x_0:\ \rho_1(x,t)= -\fr{2}{L}e^{-t/(2\tau_0)} e^{k(x-x_0)}\times{}\\
  {}\times \sum\limits^\infty_{n=1} \fr{\sin ( \pi 
n x_0/L) \sin (\pi n (L-x)/L)}{\cos(\pi n)}\times{}\\
{}\times \mathrm{ch}\left( 
\fr{t}{\tau_0}\sqrt{\fr{k\varepsilon\xi}{2(\varepsilon-\xi)}- \fr{\pi^2 n^2 
(\varepsilon-\xi)}{2kL^2}}\right)\,;
  \end{multline*}
  
  \vspace*{-12pt}
  
  \noindent
  \begin{multline*}
  \mbox{при } x < x_0:\ \rho_2(x,t)= -\fr{2}{L}e^{-t/(2\tau_0)} e^{k(x-x_0)} \times{}\\
  {}\times \sum\limits^\infty_{n=1} \fr{\sin (\pi 
n (L-x_0)/L)\sin (\pi n x/L)}{\cos (\pi n)}\times{}\\
{}\times \mathrm{ch} 
\left(\fr{t}{\tau_0}\sqrt{\fr{k\varepsilon\xi}{2(\varepsilon-\xi)}- \fr{\pi^2 n^2 
(\varepsilon-\xi)}{2kL^2}}\right)\,,
  \end{multline*}
где 
$$
k=\fr{\varepsilon -\xi}{2(\varepsilon^2 -\varepsilon\xi +\xi^2)}\,.
$$ 
Если вычислить интеграл 
\begin{equation}
{\sf P}(l,t)=\int\limits_0^{x_0} \rho_2(x,t)\,dx+\int\limits^l_{x_0}\rho_1(x,t)\,dx\,,
\label{e4-hv}
\end{equation}
то функция ${\sf P}(l,t)$ будет задавать вероятность того, что состояние системы 
к~моменту времени~$t$ будет находиться на отрезке от~0 до~$l$, т.\,е.\ порог 
перколяции~$l$ не будет достигнут. Соответственно, вероятность~$Q_i(t)$ того, 
что порог перколяции~$l$ окажется к~моменту времени~$t$ достигнутым или 
превзойденным, будет равна:
\begin{equation}
Q(l,t)=1-{\sf P}(l,t)\,.
\label{e5-hv}
\end{equation}

\vspace*{-12pt}

\subsection{Моделирование динамики и~самоорганизации состояний узлов~социальной сети}

  При анализе модели необходимо задать приемлемые величины значений 
порогов перколяции случайной сети. Плотность связей можно определить 
экспериментально, а затем, используя зависимости величины порогов 
перколяции от среднего числа связей, приходящегося на один  
узел~~\cite{11-hv, 12-hv, 13-hv, 14-hv, 15-hv, 16-hv, 17-hv}, рассчитать их 
допустимые величины (см.\ уравнения в~разд.~4).
  
  Для моделирования примем, что начальная доля~$x_0$ негативных мнений 
равна~5\% ($x_0\hm=0{,}05$), величину~$\tau_0$ примем равной одной 
условной единице времени ($\tau_0\hm =1$), $\varepsilon\hm=0{,}02$ (2\%) 
и~$\xi\hm=0{,}01$ (1\%). Результаты моделирования времени достижения 
порога перколяции~(\ref{e5-hv}) с~использованием~(\ref{e4-hv}) при заданном 
выше в~качестве примера наборе параметров модели представлены 
в~графическом виде на рис.~2.  Кривые~\textit{1} и~\textit{2} на рис.~2 показывают, что чем 
ближе значение величины начального со\-сто\-яния сис\-те\-мы~$x_0$ в~момент 
времени $t\hm=0$ к~пороговому значению, тем быстрее возрастает вероятность 
перехода и~тем сильнее вероятность его достижения приближается к~единице. 
Кривая~\textit{4} на рис.~2 показывает, что при большой разности между 
величиной порогового значения и~$x_0$ вероятность его достижения имеет 
осциллирующий характер, при этом она сначала снижается с~течением времени, 
а затем показывает рост, причем чем дальше значение величины~$x_0$ от 
порогового значения, тем сильнее проявляются осцилляции. Кроме того, 
существует и~отличная от нуля вероятность достижения порогового значения 
при $t\hm=0$ (мгновенная реализация).
  

\setcounter{figure}{1}
\begin{figure*} %fig2
\vspace*{1pt}
\begin{center}
\mbox{%
\epsfxsize=163mm
\epsfbox{hva-2.eps}
}
\end{center}
\vspace*{-14pt}
\begin{minipage}[t]{79mm}
\Caption{Графическое представление результатов моделирования преодоления порогов 
перколяции для распространения негативных мнений в~социальной сети: \textit{1}~--- $l\hm=0{,}1$; \textit{2}~--- 0,15;
\textit{3}~--- 0,20; \textit{4}~--- $l\hm=0{,}25$}
\end{minipage}
\hfill
%\end{figure*}
%\begin{figure*} %fig3
%\vspace*{1pt}
%\begin{center}
%\mbox{%
%\epsfxsize=77.502mm
%%\epsfbox{hva-3.eps}
%}
%\end{center}
\vspace*{-14pt}
\begin{minipage}[t]{79mm}
\Caption{Графическое представление результатов моделирования преодоления порогов 
перколяции для распространения негативных мнений в~социальной сети при $\varepsilon\hm=\xi\hm= 0{,}02$:
\textit{1}~--- $l\hm=0{,}1$; \textit{2}~--- 0,15; \textit{3}~--- 0,20; \textit{4}~--- $l\hm=0{,}25$}
\end{minipage}
\vspace*{6pt}
\end{figure*}

  Рост вероятности перехода через пороговое значение имеет ступенчатый 
характер, а~про\-тя\-жен\-ность ступени во времени зависит от того, насколько 
начальная величина со\-сто\-яния сис\-те\-мы~$x_0$
 \mbox{близка} к~пороговому значению. 
Процесс достижения порогового значения имеет протяженное во времени 
плато, величина которого (в~единицах ве\-ро\-ят\-ности) зависит от~$x_0$.
  
  Увеличение значения величин ~$\varepsilon$ и~$\xi$ (при выполнении 
условия $\varepsilon \hm >\xi$) изменяет величину плато (горизонтальный 
участок зависимости ве\-ро\-ят\-ности перехода через пороговое значение до 
\mbox{второго} участка резкого рос\-та) на рис.~2, однако общая за\-ви\-си\-мость 
ве\-ро\-ят\-ности перехода от времени качественно не изменяется.
  
  Ход кривых на рис.~2 показывает возможность роста вероятности 
достижения порогового значения состояния системы практически сразу после 
начала процесса. Вероятность перехода через пороговое значение отлична от 
нуля уже после первого шага и~нелинейно возрастает с~течением времени. Это 
следствие того, что не только величины~$\varepsilon$ и~$\xi$ определяют 
изменение состояния~$x$, но и~сами состояния~$x$ служат источником изменения 
вследствие наличия памяти о предыдущих состояниях и~самоорганизации, за 
которую отвечает в~дифференциальном уравнении член $d^2 {\sf P}(x,t)/dt^2$. 
Арифметический расчет показывает, что число шагов (обозначим его 
как~$q_0$), за которое можно достичь порогового значения~$l$, должно быть 
не меньше чем $q\hm=(l\hm - x_0)/(\varepsilon \hm - \xi)$. Например, для 
пороговых значений состояния системы $l\hm=0{,}1$ и~0,2 при ее начальном 
состоянии $x_0\hm=0{,}05$, $\varepsilon\hm=0{,}02$ и~$\xi\hm=0{,}01$ 
для~$q$ получим соответственно~5 и~15. Результаты (см.\ рис.~2) показывают, 
что это не так, т.\,е.\ происходит самоорганизация.
  
  При равенстве~$\varepsilon$ и~$\xi$ (например, 
$\varepsilon\hm=\xi\hm=0{,}02$)\linebreak характер хода кривых, описывающих 
вероятность достижения пороговых значений, изменяется (рис.~3).  
В~част\-ности, не наблюдается протяженного во времени плато 
с~последующим плавным ростом вероятности достижения пороговых значений 
до единицы, а рост вероятности имеет характер резкого скачка. Это связано 
с~тем, что коэффициент~$b$ в~уравнении~(\ref{e3-hv}) окажется равен нулю 
и~упорядоченные переходы будут невозможны, а член $d^2\rho(x,t)/dt^2$ будет 
ускорять только случайные переходы $d^2\rho(x,t)/dx^2$.

\vspace*{-4pt}

\section{Алгоритм мониторинга состояний социальной сетевой~структуры}

\vspace*{-2pt}

  Разработанная модель позволяет создать практически реализуемый алгоритм 
мониторинга состояния социальной сети.\\[-12pt]
\begin{enumerate}[1.]
\item Определяем с~помощью социологического мониторинга плотность сети 
и~долю узлов~$x_0$ с~определенным мнением или настроением (состояние 
узла) в~данный момент времени $t\hm=0$.\\[-15pt]
\item Спустя одну выбранную условную единицу времени $\tau\hm=1$ 
(например, одна неделя) снова находим долю узлов с~определенным мнением 
или настроением в~данный момент времени~$x_1$ ($t\hm=0\hm +\tau$). 
Находим величину $\varepsilon\hm=x_1 \hm- x_0$, а~величину~$\xi$ считаем 
равной~0. Если $\varepsilon\hm<0$, то считаем $\xi\hm= x_1 \hm- x_0$, 
а~$\varepsilon\hm=0$.\\[-15pt]
\item На основании данных о~среднем числе связей рассчитываем порог 
перколяции данной сети. Используя уравнения~(\ref{e4-hv}) и~(\ref{e5-hv}), по 
определенным в~пп.~1 и~2 значениям параметров $x_0$, $\varepsilon$, 
$\xi$ и~порогу перколяции~$l$ моделируем поведение от условного времени 
вероятности перехода поро-\linebreak\vspace*{-12pt}
\end{enumerate}

\begin{enumerate}
\item[\,]
га перколяции и~определяем допустимый лимит 
времени для изменения ситуации.
\end{enumerate}

\vspace*{-6pt}

\section{Заключение}

  Создана новая модель описания динамики изменения состояния узлов 
и~перколяционных переходов в~социальных сетях с~учетом самоорганизации 
и~наличия памяти, которая вносит \mbox{значительный} вклад в~развитие тео\-рии 
управления сложными системами.
  
  Результаты анализа разработанной модели могут быть связаны 
с~полученными ранее результатами описания процессов в~социальных сетевых 
структурах с~помощью теории перколяции (это необходимо для определения 
времени достижения пороговых значений доли узлов социальной сети, когда 
определенные мнения или предпочтения могут беспрепятственно 
распространяться по сети в~целом).
  
  Полученные результаты существенно отличаются от применяемых 
в~настоящее время моделей для описания нестационарных процессов на 
основе тео\-рии хаоса, диффузионных подходов, уравнений Лиувилля  
и~Фок\-ке\-ра--План\-ка. Все это в~целом представляется абсолютно новым 
и~оригинальным, а также вносит вклад в~развитие теории управления 
сложными системами.
  
  Разработанная модель во взаимосвязи с~методами теории перколяции 
существенно расширяет возможности применения уравнений математической 
физики и~теоретической информатики для моделирования социальных систем.

\vspace*{-6pt}
  
{\small\frenchspacing
{\baselineskip=10.85pt
%\addcontentsline{toc}{section}{References}
\begin{thebibliography}{99}
\bibitem{1-hv}
\Au{Gasser L.} Social conceptions of knowledge and action: DAI foundations and open system 
semantics~// Artif. Intell., 1991. Vol.~47. No.\,1-3. P.~107--138.
\bibitem{2-hv}
\Au{Jennings N.\,R., Faratin~P., Lomuscio~A.\,R., Parsons~S., Sierra~C., Wooldridge~M.} 
Automated negotiation: Prospects, methods and challenges~// Group Decis.
Negot., 2001. Vol.~10. No.\,2. P.~199--215.
\bibitem{3-hv}
\Au{Plikynas D., Raudys~A., Raudys~S.} Agent-based modelling of excitation propagation in 
social media groups~// J.~Exp. Theor. Artif. In., 2015. Vol.~27. No.\,4. 
P.~373--388.

\bibitem{5-hv} %4
\Au{Hay J., Flynn D.} The effect of network structure on individual behavior~// Complex Systems, 
2014. Vol.~23. No.\,4. P.~295--311.

\bibitem{4-hv} %5
\Au{Hay J., Flynn D.} How external environment and internal structure change the behavior of 
discrete systems~// Complex Systems, 2016. Vol.~25. No.\,1. P.~39--49.

\bibitem{6-hv}
\Au{Airoldi~E.\,M., Blei~D.\,M., Fienberg~S.\,E., Xing~E.\,P.} Mixed membership stochastic 
blockmodels~// J.~Mach. Learn. Res, 2008. Vol.~9. P.~1981--2014.
\bibitem{7-hv}
\Au{Бокс Дж., Дженкинс~Г.} Анализ временных рядов. Прогноз и~управление~/ Пер. 
с~англ.~--- М.: Мир, 1974. 553~с. (\Au{Box~G.\,E.\,P., Jenkins~G.\,M.} Time series analysis: 
Forecasting and control.~--- Holden-day, 1970. 553~p.)
\bibitem{8-hv}
\Au{Гнеденко Б.\,В.} Курс теории вероятностей.~--- М.: Физматлит, 1961. 406~с.
\bibitem{9-hv}
\Au{Grimmet G.\,R.} Percolation.~--- New York, NY, USA: Springer-Verlag, 1989. 296~p.

\bibitem{13-hv} %10
\Au{Zhukov D., Lesko S.} Percolation models of information dissemination in social networks~// 
IEEE Conference (International) on Smart City/SocialCom/SustainCom 
Together with DataCom Proceedings.~--- 
IEEE, 2015. P.~213--216.
\bibitem{14-hv} %11
\Au{Khvatova T., Block~M., Zhukov~D., Lesko~S.} Studying the structural topology of the 
knowledge sharing network~// 11th European Conference on Management Leadership and 
Governance Proceedings.~--- Lisbon, Portugal: Academic Conferences and Publishing 
Itnternational Ltd., 2015. P.~20--27.

\bibitem{12-hv} %12
\Au{Khvatova T.\,Yu., Zaltsman~A.\,D., Zhukov~D.\,O.} Information processes in social networks: 
Percolation and stochastic dynamics~// CEUR Workshop Procee., 2017. Vol.~2064. 
P.~277--288.

\bibitem{11-hv} %13
\Au{Zhukov D., Khvatova~T., Lesko~S., Zaltsman~A.} Managing social networks: Applying 
percolation theory methodology to understand individuals' attitudes and moods~// Technol. 
Forecast. Soc., 2018. Vol.~129. P.~297--307.


\bibitem{10-hv} %14
\Au{Лесько С.\,А., Алёшкин~.\,С., Филатов~В.\,В.} Стохастические и~перколяционные 
модели динамики блокировки вычислительных сетей при распространении эпидемий 
эволюционирующих компьютерных вирусов~// Российский технологический~ж., 2019. 
Т.~7. №\,3(29). С.~7--27.

\bibitem{16-hv} %15
\Au{Khvatova T., Block~M., Zhukov~D., Lesko~S.} How to measure trust: The percolation model 
applied to intraorganisational knowledge sharing networks~// J.~Knowl. Manag., 2016. Vol.~20. 
No.\,5. P.~918--935.

\bibitem{15-hv} %16
\Au{Жуков Д.\,О., Хватова~Т.\,Ю., Лесько~С.\,А., Зальцман~А.\,Д.} Влияние плотности 
связей на кластеризацию и~порог перколяции при распространении информации 
в~социальных сетях~// Информатика и~её применения, 2018. Т.~12. Вып.~2. С.~90--97.


\bibitem{17-hv}
\Au{Zhukov D.\,O., Khvatova~T.\,Y., Millar~C., Zaltcman~A.} Modelling the stochastic dynamics 
of transitions between states in social systems incorporating self-organisation and memory~// 
Technol. Forecast. Soc., 2020. Vol.~158. Art. ID: 120134.
\bibitem{18-hv}
\Au{Орлов Ю.\,Н., Федоров~С.\,Л.} Генерация нестационарных траекторий временного ряда 
на основе уравнения Фок\-ке\-ра--План\-ка~// Труды МФТИ, 2016. Т.~8. №\,2(30).  
С.~126--133.
\bibitem{19-hv}
\Au{Fuentes M.} Non-linear diffusion and power law properties of heterogeneous systems: 
Application to financial time series~// Entropy, 2018. Vol.~20. Iss.~9. Art. No.~649.
\end{thebibliography}

}
}

\end{multicols}

\vspace*{-10pt}

\hfill{\small\textit{Поступила в~редакцию 18.06.2019}}

%\vspace*{8pt}

%\pagebreak

\newpage

\vspace*{-28pt}

%\hrule

%\vspace*{2pt}

%\hrule

%\vspace*{-2pt}

\def\tit{MODELING OF~THE~STOCHASTIC DYNAMICS OF~CHANGES IN~NODE STATES AND~PERCOLATION 
TRANSITIONS IN~SOCIAL NETWORKS WITH~SELF-ORGANIZATION AND~MEMORY}

\def\titkol{Modeling of the~stochastic dynamics of~changes in~node states and~percolation 
transitions in~social networks} % with~self-organization and~memory}

\def\aut{D.\,O.~Zhukov$^1$, T.\,Yu.~Khvatova$^2$, and~A.\,D.~Zaltcman$^1$}

\def\autkol{D.\,O.~Zhukov, T.\,Yu.~Khvatova, and~A.\,D.~Zaltcman}

\titel{\tit}{\aut}{\autkol}{\titkol}

\vspace*{-11pt}


\noindent
$^1$Russian Technological University (MIREA), 78~Vernadskogo Ave., Moscow 119454, Russian Federation

\noindent
$^2$Peter the Great St.\ Petersburg Polytechnic University, 29~Polytechnicheskaya Str., St.\ Petersburg 195251, 
Russian\linebreak
$\hphantom{^1}$Federation

\def\leftfootline{\small{\textbf{\thepage}
\hfill INFORMATIKA I EE PRIMENENIYA~--- INFORMATICS AND
APPLICATIONS\ \ \ 2021\ \ \ volume~15\ \ \ issue\ 1}
}%
\def\rightfootline{\small{INFORMATIKA I EE PRIMENENIYA~---
INFORMATICS AND APPLICATIONS\ \ \ 2021\ \ \ volume~15\ \ \ issue\ 1
\hfill \textbf{\thepage}}}

\vspace*{3pt}





\Abste{This paper explores the use of theoretical informatics applied for analyzing and modeling 
the processes in sociotechnical systems (social networks). A~stochastic model of users' (network nodes) dynamic changes of states (opinions 
or moods) and the percolation threshold in a~social network with random connections among nodes was developed. 
This model demonstrates the opportunity for jump-like transitions in states (opinions, moods) of the nodes in a~social 
network over a short period of time without external influence. While developing the model, the probabilistic schemes 
of state-to-state transitions of nodes (users having certain opinions and views) were considered; 
a~second-order  
nonlinear differential equation was derived; the boundary for calculating the probability density function for a~system 
being in a certain state depending on the time interval was formulated. The differential equation of the model contains 
a~member representing the opportunity for self-organization; it also considers the presence of memory. The results of 
analysis of the stochastic model support those previously obtained by the authors when investigating social network 
processes using the percolation theory. This theory was used for defining the time of reaching the threshold values for 
the share of social network nodes when certain opinions or preferences can spread freely within the whole social 
network.}

\KWE{stochastic dynamics; states of social network nodes; system self-organization; processes involving memory; 
percolation in social networks}

\DOI{10.14357/19922264210114}

%\vspace*{-15pt}

%\Ack
%\noindent

%\vspace*{6pt}

  \begin{multicols}{2}

\renewcommand{\bibname}{\protect\rmfamily References}
%\renewcommand{\bibname}{\large\protect\rm References}

{\small\frenchspacing
 {%\baselineskip=10.8pt
 \addcontentsline{toc}{section}{References}
 \begin{thebibliography}{99}

\bibitem{1-hv-1}
\Aue{Gasser, L.} 1991. Social conceptions of knowledge and action: DAI foundations and open 
system semantics. \textit{Artif. Intell.} 47(1-3):107--138.
\bibitem{2-hv-1}
\Aue{Jennings, N.\,R., P.~Faratin, A.\,R.~Lomuscio, S.~Parsons, C.~Sierra, and M.~Wooldridge}. 
2001. Automated negotiation: Prospects, methods and challenges. \textit{Group Decis. 
Negot.} 10(2):199--215.
\bibitem{3-hv-1}
\Aue{Plikynas, D., A.~Raudys, and S.~Raudys.} 2015. Agent-based modelling of excitation 
propagation in social media groups. \textit{J.~Exp. Theor. Artif. In.} 
27(4):373--388.

\bibitem{5-hv-1}
\Aue{Hay, J., and D.~Flynn.} 2014. The effect of network structure on individual behavior. 
\textit{Complex Systems} 23(4):295--311.

\bibitem{4-hv-1}
\Aue{Hay, J., and D.~Flynn.} 2016. How external environment and internal structure change the 
behavior of discrete systems. \textit{Complex Systems} 25(1):39--49.

\bibitem{6-hv-1}
\Aue{Airoldi, E.\,M., D.\,M.~Blei, S.\,E.~Fienberg, and E.\,P.~Xing.} 2008. Mixed membership 
stochastic blockmodels. \textit{J.~Mach. Learn. Res.} 9:1981--2014.
\bibitem{7-hv-1}
\Aue{Box, G.\,E.\,P., and G.\,M.~Jenkins.} 1970. \textit{Time series analysis: Forecasting and 
control.} Holden-day. 553~p.
\bibitem{8-hv-1}
\Aue{Gnedenko, B.\,V.} 1961. \textit{Kurs teorii veroyatnostey} [Probability theory]. Moscow: Fizmatlit. 406~p.
\bibitem{9-hv-1}
\Aue{Grimmet, G.\,R.} 1989. \textit{Percolation}. New York, NY: Springer-Verlag. 296~p.

\bibitem{13-hv-1} %10
\Aue{Lesko, S.\,A., and D.\,O.~Zhukov.} 2015. Percolation models of information dissemination in 
social networks. \textit{IEEE Conference (International) on Smart City/SocialCom/\linebreak SustainCom 
Together with DataCom Proceedings}. IEEE. 213--216.
\bibitem{14-hv-1} %11
\Aue{Block, M., T.~Khvatova, D.~Zhukov, and S.~Lesko.} 2015. Studying the structural topology 
of the knowledge sharing network. \textit{11th European Conference on Management, Leadership 
and Governance Proceedings}. Lisbon, Portugal: Academic Conferences and Publishing 
International Ltd. 20--27.

\bibitem{12-hv-1} %12
\Aue{Khvatova, T.\,Yu., A.\,D.~Zaltcman, and D.\,O.~Zhukov.} 2017. Information processes in 
social networks: Percolation and stochastic dynamics. CEUR Workshop Procee.
2064:277--288.

\bibitem{11-hv-1} %13
\Aue{Zhukov, D., T.~Khvatova, S.~Lesko, and A.~Zaltcman.} 2018. Managing social networks: 
Applying Percolation theory methodology to understand individuals' attitudes and moods. 
\textit{Technol. Forecast. Soc.} 129:297--307.


\bibitem{10-hv-1} %14
\Aue{Lesko, S.\,A., A.\,S.~Alyoshkin, and V.\,V.~Filatov.} 2019. Stokhasticheskie 
i~perkolyatsionnye modeli dinamiki blokirovki vychislitel'nykh setey pri rasprostranenii epidemiy 
evolyutsioniruyushchikh komp'yuternykh virusov [Stochastic and percolating models of blocking 
computer networks dynamics during distribution of epidemics of evolutionary computer viruses]. 
\textit{Rossiyskiy tekh\-no\-lo\-gi\-che\-skiy zh.} [Russian Technological~J.] 7(3):7--27.


\bibitem{16-hv-1} %15
\Aue{Khvatova, T., M.~Block, D.~Zhukov, and S.~Lesko.} 2016. How to measure trust: The 
percolation model applied to intraorganisational knowledge sharing networks. \textit{J.~Knowl. 
Manag.} 20 (5):918--935.

\bibitem{15-hv-1} %16
\Aue{Zhukov, D.\,O., T.\,Yu.~Khvatova, S.\,A.~Les'ko, and A.\,D.~Zal'tsman.} 2018. Vliyanie 
plotnosti svyazey na klasterizatsiyu i~porog perkolyatsii pri rasprostranenii informatsii 
v~sotsial'nykh setyakh [The influence of the connections density on clusterization and percolation 
threshold during information distribution in social networks]. \textit{Informatika i~ee 
Primeneniya~---Inform. Appl.} 12(2):90--97.

\bibitem{17-hv-1}
\Aue{Zhukov, D.\,O., T.\,Y.~Khvatova, C.~Millar, and A.~Zaltcman.} 2020. Modelling the 
stochastic dynamics of transitions between states in social systems incorporating self-organisation 
and memory. \textit{Technol. Forecast. Soc.} 158:120134.
\bibitem{18-hv-1}
\Aue{Orlov, Yu.\,N., and S.\,L.~Fedorov.} 2016. Generatsiya ne\-sta\-tsio\-nar\-nykh traektoriy 
vremennogo ryada na osnove uravneniya Fokkera--Planka [Generating nonstationary trajectories 
of a time series based on Fokker--Plank equation]. \textit{Trudy MFTI} [MIPT Proceedings~J.] 
8(2):126--133.
\bibitem{19-hv-1}
\Aue{Fuentes, M.} 2018. Non-linear diffusion and power law properties of heterogeneous systems: 
Application to financial time series. \textit{Entropy} 20(9):649. 8~p.
\end{thebibliography}

 }
 }

\end{multicols}

\vspace*{-3pt}

  \hfill{\small\textit{Received June~18, 2019}}


%\pagebreak

%\vspace*{-8pt}

\Contr

\noindent
\textbf{Zhukov Dmitry O.} (b.\ 1965)~--- 
Doctor of Science in technology, professor, Head of Department, Russian Technological 
University (MIREA), 78~Vernadskogo Ave., Moscow 119454, Russian Federation; 
\mbox{zhukov\_do@mirea.ru}

\vspace*{6pt}

\noindent
\textbf{Khvatova Tatiana Yu.} (b.\ 1971)~--- Doctor of Science in economics, professor, Peter the 
Great St.\ Petersburg Polytechnic University, 29~Polytechnicheskaya Str., St.\ Petersburg 195251, 
Russian Federation; \mbox{khvatova.ty@spbstu.ru}

\vspace*{6pt}

\noindent
\textbf{Zaltcman Anastasia D.} (b.\ 1989)~--- lecturer, Russian Technological University 
(MIREA), 78~Vernadskogo Ave., Moscow 119454, Russian Federation; 
\mbox{ad.zaltcman@gmail.com}

\label{end\stat}

\renewcommand{\bibname}{\protect\rm Литература}       %14
\def\stat{grinchenko}

\def\tit{О ГЕНЕЗИСЕ ИНФОРМАЦИОННОГО ОБЩЕСТВА:  
ИНФОРМАТИКО-КИБЕРНЕТИЧЕСКОЕ МОДЕЛЬНОЕ ПРЕДСТАВЛЕНИЕ}

\def\titkol{О генезисе информационного общества:  
информатико-кибернетическое модельное представление}

\def\aut{С.\,Н.~Гринченко$^1$}

\def\autkol{С.\,Н.~Гринченко}

\titel{\tit}{\aut}{\autkol}{\titkol}

\index{Гринченко С.\,Н.}
\index{Grinchenko S.\,N.}


%{\renewcommand{\thefootnote}{\fnsymbol{footnote}} \footnotetext[1]
%{Работа выполнена при частичной финансовой 
%поддержке РФФИ (проект 17-07-00577).}}


\renewcommand{\thefootnote}{\arabic{footnote}}
\footnotetext[1]{Институт проблем информатики Федерального исследовательского центра <<Информатика и~управление>> 
Российской академии наук, \mbox{sgrinchenko@ipiran.ru}}

\vspace*{-3.5pt}




  \Abst{Вводится понятие <<генезис информационного общества>>, которое рассматривается 
  с~позиций ин\-фор\-ма\-ти\-ко-ки\-бер\-не\-ти\-че\-ско\-го моделирования (ИКМ)
  процесса развития 
Человечества как са\-мо\-управ\-ля\-ющей\-ся иерар\-хо-се\-те\-вой системы. На этой основе 
получены количественные оценки его типовых про\-стран\-ст\-вен\-но-вре\-мен\-ных характеристик, 
представляющих собой геометрические прогрессии со знаменателем 
<<$e$~в~степени~$e$>> (15,15426$\ldots$), а~также скоординированных с~ними во времени 
и~в~пространстве пси\-хи\-ко-ант\-ро\-по\-ло\-ги\-че\-ских, образовательных  
и~ин\-фор\-ма\-ци\-он\-но-ком\-му\-ни\-ка\-ци\-он\-ных параметров и~возможностей 
включенного в~этот процесс усложняющегося человека и~его сообществ различной 
величины. Это позволило раздвинуть рамки существования информационного общества на 
всю историческую и~даже археологическую эпоху такого развития. Результирующая 
последовательность информационных технологий (ИТ) <<сигнальные  
по\-зы/зву\-ки/дви\-же\-ния\,--\,ми\-ми\-ка/жес\-ты\,--\,речь/язык\,--\,пись\-мен\-ность\,--\,ти\-ра\-жи\-ро\-ва\-ние текстов\,--\,компью\-те\-ры\,--\,те\-ле\-ком\-му\-ни\-ка\-ции\,--\,ин\-фор\-ма\-ци\-он\-ная на\-но\-тех\-но\-ло\-гия\,--\,$\ldots$>> 
позволяет рас\-смат\-ри\-вать генезис 
информационного общества в~широком контексте единой исторической ретроспективы 
и~перспективы.}
  
  \KW{информационное общество; информационные технологии;  
ин\-фор\-ма\-ти\-ко-ки\-бер\-не\-ти\-че\-ская модель; самоуправляющаяся 
 иерар\-хо-се\-те\-вая система Человечества; археологическая эпоха}
 
 \DOI{10.14357/19922264190214}
  
%\vspace*{4pt}


\vskip 10pt plus 9pt minus 6pt

\thispagestyle{headings}

\begin{multicols}{2}

\label{st\stat}
  
  В~литературе, даже энциклопедической, распространена трактовка 
<<информационного общества>> как общества <<современного типа>>, 
в~котором общение людей опирается на компьютерные 
и~телекоммуникационные ИТ\footnote[2]{В~[1] дано следующее определение:
<<\textbf{Информационное общество}, одно из понятий, используемых 
в~социологич.\ теории для обозначения обществ.\ систем <<современного типа>>$\ldots$ 
Важнейшие характеристики~И.\,о.: 1)~лавинообразное распространение информац. 
технологий (прежде всего компьютерных и~телекоммуникационных); 2)~превращение 
информации в~важнейший социальный ресурс, необходимую предпосылку управленч. 
деятельности, развития экономики, образования, сферы услуг, домашнего быта, 
рекреационной сферы и~т.\,д.; по некоторым данным, в~наиболее развитых странах проф. 
деятельность более половины занятых связана исключительно с~производством и~обработкой 
информации; 3)~наделение СМИ статусом <<четвертой ветви власти>>; 4)~расширение 
границ и~<<репертуара>> массовой культуры; 5)~увеличение каналов вертикальной 
и~горизонтальной мобильности; 6)~изменение представлений о~социальном пространстве 
(<<глобализация>> пространства, мгновенная доступность даже периферийных его 
сегментов) и~времени (расширение рамок <<современности>>, когда даже отдаленные 
историч. события воспринимаются как происходящие <<здесь>> и~<<сейчас>>); 
7)~возникновение в~процессе коммуникации особой виртуальной реальности, несводимой 
к~результатам технич. визуализации и~выходящей за пределы воображения и~памяти 
индивида; 8)~превращение информац. технологий в~базу для развития высоких технологий 
(Hi-Tech)>>.}. Такая трактовка этого понятия создает иллюзию 
отстраненности информационного общества от его собственного исторического 
прошлого, когда вышеперечисленных ИТ еще не изобрели, но люди в~составе 
сообществ как-то общались между собой, используя иные ИТ. 

Поскольку от 
этой иллюзии недалеко до недооценки полезности соответствующего 
исторического опыта для современности, попытаюсь развеять ее.
  
Результаты ИКМ процесса развития на Земле 
Человечества как самоуправляющейся ие\-рар\-хо-се\-те\-вой\footnote[3]{<<\textbf{Иерархо-сетевая}>> 
структура~--- иерархическая структура типа <<матрешки>>, но с~существенно большим 
единицы числом вложений на каждом ее иерархическом уровне, которые и~образуют 
соответствующие сетевые структуры.} системы~[2--14] (рис.~1) позволяют раздвинуть рамки 
существования информационного\linebreak общества на всю историческую и~даже археологическую эпоху такого 
развития, что дает возможность выделить ту эволюционную линию этого процесса, которую логично 
определить как \textit{генезис информационного общества}. 


\begin{figure*} %fig1
   \vspace*{1pt}
    \begin{center}  
  \mbox{%
 \epsfxsize=130.287mm 
 \epsfbox{gri-1.eps}
 }
\end{center}
%\vspace*{-9pt}
%\Caption{Схема иерархо-сетевой самоуправляющейся (по алгоритмам случайной поисковой 
%оптимизации целевых критериев энергетического характера с~ограничениями типа 
%равенств и~неравенств) системы лич\-ност\-но-про\-из\-вод\-ст\-вен\-но-со\-ци\-аль\-ной природы 
%(Человечества)~\cite{5-grn}}
\end{figure*}


На рис.~1 используются следующие обозначения:
\begin{itemize}
\item восходящие стрелки (имеющие структуру <<мно\-гие\,--\,к~од\-но\-му>>) 
отражают первую из~5~основных со\-став\-ля\-ющих контура поисковой 
оптимизации~--- \textit{поисковую активность} представителей 
соответствующих ярусов в~иерархии; 
\item нисходящие сплошные (имеющие 
структуру <<один\,--\,ко мно\-гим>>) стрелки отражают вторую 
со\-став\-ля\-ющую~--- \textit{целевые критерии} поисковой оптимизации 
энергетики системы Человечества; 
\item нисходящие пунктирные (<<один\,--\,ко 
многим>>) стрелки отражают третью со\-став\-ля\-ющую~--- 
\textit{оптимизационную системную память}  
лич\-ност\-но-про\-из\-вод\-ст\-вен\-но-со\-ци\-аль\-но\-го (результат 
адаптивных влияний представителей вышележащих иерархических ярусов на 
структуру вложенных в~них нижележащих); 
\item полужирными стрелками 
в~левой части схемы условно показана четвертая со\-став\-ля\-ющая~--- 
\textit{антропогенная ак\-тив\-ность} индивидов и~их групп, трак\-ту\-емая как 
<<трудовая деятельность по созданию со\-от\-вет\-ст\-ву\-юще\-го инструментария 
и~результатов его применения>>; 
\item пунктирными полужирными стрелками 
в~правой части схемы условно показана пятая со\-став\-ля\-ющая~--- 
\textit{антропогенная системная\linebreak память}  
лич\-ност\-но-про\-из\-вод\-ст\-вен\-но-со\-ци\-ального (процессы вовлечения 
результатов антропогенной активности в~структуру со\-от\-вет\-ст\-ву\-ющей  
иерар\-хо-се\-те\-вой под\-сис\-те\-мы Человечества).
\end{itemize}

Рассмотрим этот феномен поэтапно, сведя в~общую таблицу расчетные данные 
о~различных его проявлениях. 
       


\begin{table*}\footnotesize
\begin{center}
\Caption{Свод основных характеристик генезиса информационного общества (как 
проявления развития са\-мо\-управ\-ля\-ющей\-ся и~метаэволюционирующей, т.\,е.\ 
наращивающей чис\-ло своих иерархических уров\-ней/яру\-сов, сис\-те\-мы Человечества) от 
прошлого до модельно прогнозируемого будущего}
\vspace*{2ex}

\tabcolsep=1.5pt
\begin{tabular}{|c|c|l|c|c|c|c|}
\hline
&\tabcolsep=0pt\begin{tabular}{c}Характерный\\ ареал (радиус\\
 круга той же\\ площади); точность\\ антропогенного\\ 
воздействия\,/\\
производственных\\ технологий\end{tabular}&
\tabcolsep=0pt\begin{tabular}{c}Характерные\\ времена\\ старта;\\ кульминации\\ 
развития\\ подсистемы\end{tabular}&
\tabcolsep=0pt\begin{tabular}{c}Уровень\\ развития\\ Homo\\  
(и его пред-\\ шествен-\\ ников)\end{tabular}&
\tabcolsep=0pt\begin{tabular}{c}Носитель системной\\ памяти~---\\ субстрат психики\end{tabular}&
\tabcolsep=0pt\begin{tabular}{c}Лидирующая\\ ИТ\end{tabular}&
\tabcolsep=0pt\begin{tabular}{c}Требуемый уровень\\ образованности Homo;\\
аналогия филогенеза\\ и~онтогенеза:\\ примерный возраст\\ гармонично\\ образовываемого\\ 
(сегодня)\end{tabular}\\
\hline
1&2&\multicolumn{1}{c|}{3}&4&5&6&7\\
\hline
0&$\sim4{,}2$~м&\tabcolsep=0pt\begin{tabular}{c} $\sim428$~млн\\ лет назад;\\
$\sim 140{,}1$~млн\\ лет назад\end{tabular}&
\tabcolsep=0pt\begin{tabular}{c}Цефализация\\ позвоночных\end{tabular}&
\tabcolsep=0pt\begin{tabular}{c}Многоклеточный\\организм в~целом\end{tabular}&
\tabcolsep=0pt\begin{tabular}{c}Формирование\\ головного\\ мозга как основы\\
 для реализации\\ 
будущих ИТ\end{tabular}&\tabcolsep=0pt\begin{tabular}{c} ---\\
$\sim0{,}6$--1,0~год\end{tabular}\\
\hline
1&\tabcolsep=0pt\begin{tabular}{c} $\sim64$~м;\\
$\sim28$~см
\end{tabular}&\tabcolsep=0pt\begin{tabular}{c}$\sim28{,}23$~млн\\ лет назад;\\
$\sim9{,}26$~млн\\ лет назад
\end{tabular}&\tabcolsep=0pt\begin{tabular}{c}Пред-пред-\\
люди\\ Hominoidea\end{tabular}&
\tabcolsep=0pt\begin{tabular}{c}Органы многоклеточного\\ организма (его 
нервной\\ системы в~целом)\end{tabular}&
\tabcolsep=0pt\begin{tabular}{c}Сигнальные позы/\\
движения\\ и~неинтонированные\\ звуки (типа 
рычания,\\ ворчания, писка\\ и~т.\,п.)\end{tabular}&
\tabcolsep=0pt\begin{tabular}{c}Выработка\\ 
(младенцами)\\ сигнальных поз;\\
$\sim1{,}0$--1,6~лет \end{tabular}\\
\hline
2&\tabcolsep=0pt\begin{tabular}{c} $\sim1$~км;\\
$\sim1{,}8$~см\end{tabular}&\tabcolsep=0pt\begin{tabular}{c} $\sim1{,}86$~млн\\ лет 
назад;\\
$\sim612$~тыс.\\ лет назад\end{tabular}&
\tabcolsep=0pt\begin{tabular}{c}Пред-люди\\ Homo ergaster\,/\\
Homo erectus\end{tabular}&
\tabcolsep=0pt\begin{tabular}{c}Ткани 
многоклеточного\\ организма\\ (сетей/ансамблей\\ нейронов и~др.)\end{tabular}&
\tabcolsep=0pt\begin{tabular}{c}Мимика/жесты\\ 
и~интонированные\\ звуки\end{tabular}&
\tabcolsep=0pt\begin{tabular}{c}Овладение (ре-\\ бенком) мимикой/\\ 
жестами,\\
начальное\\ понимание речи; \\ $\sim1{,}6$--2,6~лет \end{tabular}\\
\hline
3&\tabcolsep=0pt\begin{tabular}{c} $\sim15$~км; \\
$\sim1{,}2$~мм
\end{tabular}&
\tabcolsep=0pt\begin{tabular}{c} $\sim123$~тыс.\\ лет назад;\\
$\sim40$~тыс.\\ лет назад\end{tabular}&
\tabcolsep=0pt\begin{tabular}{c}Homo\\ sapiens$^\prime$\end{tabular}&
\tabcolsep=0pt\begin{tabular}{c}Эвкариотические\\ клетки\\ 
многоклеточного\\ организма\\ (отдельные нервные\\ и~глиальные клетки\\ и~др.)\end{tabular}&
\tabcolsep=0pt\begin{tabular}{c}Речь/язык\\ 
(артикулированная\\ устная речь)\end{tabular}&
\tabcolsep=0pt\begin{tabular}{c}Овладение (детьми)\\ 
речью/языком\\ (протообразование); \\ $\sim2{,}6$--4,2~лет \end{tabular}\\
\hline
4&\tabcolsep=0pt\begin{tabular}{c} $\sim222$~км;\\
$\sim 80$~мкм
\end{tabular}&\tabcolsep=0pt\begin{tabular}{c}$\sim8{,}1$~тыс.\\ лет назад;\\
$\sim2{,}7$~тыс.\\ лет назад\end{tabular}&
\tabcolsep=0pt\begin{tabular}{c}Homo\\ sapiens$^{\prime\prime}$\end{tabular}&
\tabcolsep=0pt\begin{tabular}{c}Компартменты\\ 
эвкариотической\\ клетки (отдельные\\ рецепторные,\\ или постсинаптические,\\ зоны нейронов и~т.\,п.)\end{tabular}
&Письменность&\tabcolsep=0pt\begin{tabular}{c}Овладение чтением/ \\ письмом 
(дошкольное\\ образование);\\
$\sim4{,}2$--6,9~лет \end{tabular}\\
\hline
5&\tabcolsep=0pt\begin{tabular}{c}$\sim3370$~км;\\
$\sim5$~мкм
\end{tabular}&\tabcolsep=0pt\begin{tabular}{l}$\sim1446$~г.;\\
$\sim1806$~г.\end{tabular}&
\tabcolsep=0pt\begin{tabular}{c}Homo\\ sapiens$^{\prime\prime\prime}$\end{tabular} &
\tabcolsep=0pt\begin{tabular}{c}Субкомпартменты\\ эвкариотической 
клетки\end{tabular}&
\tabcolsep=0pt\begin{tabular}{c}Тиражирование\\ текстов,\\ или книгопечатание\end{tabular}&
\tabcolsep=0pt\begin{tabular}{c}Начальное\\ образование;\\ 
$\sim6{,}9$--11,1~лет \end{tabular}\\
\hline
6&\tabcolsep=0pt\begin{tabular}{c} $\sim51$~тыс.\ км\\ (общепланетарный);\\
$\sim0{,}35$~мкм\end{tabular}&\tabcolsep=0pt\begin{tabular}{l} $\sim1946$~г.;\\
$\sim 1970$~г.\end{tabular}&\tabcolsep=0pt\begin{tabular}{c}Homo \\
sapiens$^{\prime\prime\prime\prime}$\end{tabular}&
\tabcolsep=0pt\begin{tabular}{c}Ультраструктурные\\ (прокариотические)\\ 
внутриклеточные элементы\\ эвкариотической клетки\\ (типа клеточного ядра,\\ деталей 
эндоплазматической\\ сети и~т.\,п.\ образований)\end{tabular}&
Компьютерная ИТ&\tabcolsep=0pt\begin{tabular}{c}Среднее\\ образование;\\
$\sim11{,}1$--18~лет \end{tabular}\\
\hline
7&\tabcolsep=0pt\begin{tabular}{c} $\sim773$~тыс.\ км\\ (ближний\\ космос);\\
$\sim23$~нм\end{tabular}&\tabcolsep=0pt\begin{tabular}{l} $\sim1979$~г.;\\
$\sim2003$~г.\end{tabular}&\tabcolsep=0pt\begin{tabular}{c}Homo\\
 sapiens$^{\prime\prime\prime\prime\prime}$\end{tabular} 
&
\tabcolsep=0pt\begin{tabular}{c}Макромолекулы/гены\\ (компартменты\\ 
ультраструктурных--\\
прокариотических--\\
внутриклеточных\\ элементов)\end{tabular}&
\tabcolsep=0pt\begin{tabular}{c}Телекоммуника-\\ ционная ИТ\end{tabular}&\tabcolsep=0pt\begin{tabular}{c}Высшее обра-\\
зование\;+\;<<аспи-\\ рантура>>; \\
$\sim18$--29,1~лет \end{tabular}\\
\hline
8&\tabcolsep=0pt\begin{tabular}{c}
$\sim11{,}7$~млн км\\ (промежуточный\\ космос);\\
$\sim1{,}5$~нм\end{tabular}&\tabcolsep=0pt\begin{tabular}{l} $\sim1981$~г.;\\ 
$\sim2341$~г.~(?)\end{tabular}&\tabcolsep=0pt\begin{tabular}{c}Homo\\ 
sapiens$^{\prime\prime\prime\prime\prime\prime}$\end{tabular}&
\tabcolsep=0pt\begin{tabular}{c}Органические молекулы \\
(субкомпартменты\\ ультраструктурных--
\\прокариотических--
\\внутриклеточных \\
элементов)\end{tabular}&
\tabcolsep=0pt\begin{tabular}{c}Нано-ИТ (возможно,\\
 <<наноаппаратно\\ поддерживаемая\\ селективная\\ телепатия>>~\cite{16-grn})\end{tabular}&
 \tabcolsep=0pt\begin{tabular}{c}<<Докторантура>>; \\ 
$\sim29{,}1$--47,1~лет \end{tabular}\\
\hline
9&$\cdots$&\multicolumn{1}{c|}{$\cdots$}&$\cdots$&$\cdots$&$\cdots$&$\cdots$\\
\hline
\end{tabular}
\end{center}
\end{table*}




  Промежутки времени между возникновением новых ие\-рар\-хо-се\-те\-вых 
подсистем Человечества (а~следовательно, и~между стартами новых ИТ) 
подчиняются, согласно ИКМ, простой математической за\-ко\-но\-мер\-ности: 
каж\-дый из них в~$e^e\hm= 15{,}15426$\ldots раз короче 
предыдущего\footnote{Эту геометрическую прогрессию~--- как модель критических 
уровней развития биологических сис\-тем~--- выявили А.\,В.~Жирмунский 
и~В.\,И.~Кузьмин~\cite{17-grn}.} (третий\linebreak
 столбец таблицы). В~свою очередь, этой 
же закономерности подчиняются и~размеры ареалов\linebreak
 (радиусы кругов той же 
площади) устойчивых и~эффективно са\-мо\-управ\-ля\-ющих\-ся сообществ 
человека как базисного элемента сис\-те\-мы Человечества, и~точ\-ности 
доступных услож\-ня\-юще\-му\-ся человеку~--- в~конкретный момент 
исторического времени~--- антропогенных воздействий и/или 
производственных технологий (второй столбец таб\-ли\-цы) (рис.~2).
  
  Эмпирические оценки этих времен и~пространств, сделанные 
и~опуб\-ли\-ко\-ван\-ные палео\-ант\-ро\-по\-ло\-га\-ми, археологами и~историками,~--- 
когда они имеются!~--- не противоречат модельным  
результатам~\cite{14-grn}.
  %
Диапазоны примерного возраста <<образовываемых>>, приведенные 
в~седьмом столб\-це таб\-ли\-цы, рассчитаны, исходя из <<золотого сечения>> 
(соотношения смеж\-ных членов чис\-ло\-во\-го ряда, равного 1,618$\ldots$ при 
увеличении ряда, либо 0,618$\ldots$ при его уменьшении, аде\-кват\-ность 
использования которого при выработке количественных оценок в~самых 
различных областях знания хорошо известна\footnote{Применительно 
к~периодизации истории Человечества в~археологическую эпоху это продемонстрировано 
Ю.\,Л.~Щаповой~\cite{18-grn, 19-grn, 20-grn}, согласование подхода к~такой периодизации на 
основе золотого сечения и~пред\-ла\-га\-емо\-го информатико-ки\-бер\-не\-ти\-че\-ско\-го подхода 
подробно показано в~\cite{10-grn, 12-grn, 13-grn, 14-grn, 15-grn, 21-grn}.}), 
опирающегося на ориентировочную оценку завершения человеком среднего 
образования к~18~годам (на сегодня).


  Базируясь на ИКМ, в~качестве нулевого этапа развития будущего 
информационного общества, как пред\-став\-ля\-ет\-ся, можно рас\-смат\-ри\-вать 
процесс \textit{цефализации} позвоночных, т.\,е.\ возникновения 
и~усложнения у~них головного мозга как основного носителя механизмов 
запоминания и~считывания информации о~результатах их адаптивного 
и~социального поведения, начавшейся около 428~млн лет назад 
с~кульминацией около 140,1~млн лет назад (шестой стол\-бец таб\-ли\-цы) на 
<<территории>> порядка 4,2~м~--- т.\,е.\ в~пределах отдельного 
многоклеточного организма.
  

  
  Далее в~качестве первого этапа такого развития будем рассматривать 
начавшуюся около 28,23~млн лет назад, с~кульминацией около 9,26~млн лет 
назад, на территориях порядка 64~м, ИТ сигнальных поз/дви\-же\-ний 
и~неинтонированных звуков (типа рычания, ворчания, писка и~т.\,п.), 
характерную для стад\-ных/стай\-ных животных, в~том числе  
пред-пред-людей {Hominoidea} (четвертый стол\-бец таб\-ли\-цы), 
способных обеспечивать точность своих воздействий на природу порядка~28~см. 
Субстрат их психики относится к~иерархическому уровню органов 
многоклеточного организма (пятый стол\-бец), а~уровень об\-ра\-зо\-ван\-ности 
соответствует современному младенцу возрастом около~1--1,6~лет (седьмой 
столбец).
  
  Следующий, второй этап развития ИТ~--- ми\-ми\-ки/жес\-тов, начавшийся 
около~1,86~млн лет назад, с~кульминацией около~612~тыс.\ лет назад, на 
территориях порядка~1~км, реализовался далекими\linebreak предками современного 
человека~--- пред-людь\-ми {Homo ergaster/Homo erectus}, способными 
обеспечивать точ\-ность своих воздействий на природу\linebreak порядка~1,8~см, 
с~субстратом психики уров\-ня тканей многоклеточного организма и~уровнем 
обра\-зо\-ван\-ности, соответствующим современному ребенку~1,6--2,6~лет.

\pagebreak

\end{multicols}

\setcounter{figure}{1}
\begin{figure*} %fig2
 \vspace*{1pt}
    \begin{center}  
  \mbox{%
 \epsfxsize=163.101mm 
 \epsfbox{gri-2.eps}
 }
\end{center}
\vspace*{-6pt}
\Caption{Пространственно-временн$\acute{\mbox{ы}}$е характеристики и~тренд ИТ в~процессе генезиса 
информационного общества (по ИКМ, в~двойном логарифмическом масштабе; 
иерархическая слож\-ность~--- число уров\-ней/яру\-сов в~системной иерархии)}
\vspace*{1pt}
\end{figure*}

\begin{multicols}{2}



  
  Все последующие этапы развития ИТ~--- речь/язык, пись\-мен\-ность, 
тиражирование текстов (книгопечатание), компьютеры, телекоммуникации, 
на\-но-ИТ~--- реализовались последовательно усложняющимися формами 
{Homo sapiens}, который при этом образовывал относительно 
устойчивые и~относительно эффективно функционирующие 
и~самоуправляющиеся сообщества на все больших ареалах, одновременно 
повышая точность своих (антропогенных) действий при формировании 
вокруг себя <<второй (рукотворной) природы>>.
  
  Так, третий этап развития ИТ~--- речи/языка, начавшийся около 123~тыс.\ 
лет назад, с~кульминацией (верхнепалеолитической революцией) 
около~40~тыс.\ лет назад, на территориях порядка~15~км, реализовался 
{Homo sapiens}$^\prime$, способными обеспечивать точность своих 
производственных технологий порядка~1,2~мм, с~субстратом психики 
уровня эвкариотических клеток многоклеточного организма и~уровнем 
образованности, соответствующим современному ребенку~2,6--4,2~лет.

\begin{figure*}[b] %fig3
%\vspace*{-4pt}
    \begin{center}  
  \mbox{%
 \epsfxsize=162.821mm 
 \epsfbox{gri-3.eps}
 }
\end{center}
\vspace*{-6pt}
\Caption{Тренд изменения времен запаздывания кульминаций развития под\-сис\-тем  
иерар\-хо-се\-те\-вой сис\-те\-мы Человечества относительно их стартов (по ИКМ, в~двойном 
логарифмическом мас\-штабе)}
\end{figure*}
  
  Четвертый этап развития ИТ~--- письменности, начавшийся 
около~8,1~тыс.\ лет назад, с~кульминацией (городской революцией 
<<осевого времени>>) около 2,7~тыс.\ лет назад, на территориях 
порядка~222~км, реализовался {Homo sapiens}$^{\prime\prime}$, 
способными обеспечивать точность своих производственных технологий 
порядка~80~мкм, с~суб\-стра\-том психики уровня компартментов 
эвкариотических клеток многоклеточного организма и~уровнем 
образованности, соответствующим современному ребенку~4,2--6,9~лет 
(дошкольное образование).
  
  Пятый этап развития ИТ~--- тиражирования\linebreak текс\-тов (книгопечатания), 
начавшийся около 1446~г.\ н.\,э., с~кульминацией (промышленной\linebreak 
революцией) около 1806~г., на территориях порядка~3370~км, реализовался 
{Homo sapiens}$^{\prime\prime\prime}$, способными обеспечивать 
точность своих производственных технологий порядка~5~мкм, с~субстратом 
психики уровня субкомпартментов эвкариотических клеток многоклеточного 
организма и~уровнем об\-ра\-зо\-ван\-ности, соответствующим современному 
ребенку~6,9--11,1~лет (начальное образование).
  
  Шестой этап развития ИТ~--- компьютеров (локальных), начавшийся 
около~1946~г., с~кульминацией (изобретением микропроцессоров) 
около~1970~г., на территориях порядка~51~тыс.\ км (т.\,е.\ 
общепланетарного, или глобального размера), реализовался {Homo 
sapiens}$^{\prime\prime\prime\prime}$, способными обеспечивать точ\-ность 
своих производственных технологий порядка~0,35~мкм, с~субстратом 
психики уровня\linebreak
 ультраструктурных (прокариотических) внутриклеточных 
элементов эвкариотической клетки и~уровнем об\-ра\-зо\-ван\-ности, 
соответствующим современному  
под\-рост\-ку-юно\-ше/де\-вуш\-ке~11,1--18~лет\linebreak (среднее образование).
  
  Седьмой этап развития ИТ~--- телекоммуникаций, начавшийся около 
1979~г., с~кульминацией (пиком ско\-рости распространения на планете 
мобильной телефонии, интернета и~т.\,п.) около\linebreak
 2003~г., в~космическом 
объеме радиусом (шара)\linebreak порядка 773~тыс.\ км (т.\,е.\ в~ближнем космосе), 
реализовался {Homo sapiens}$^{\prime\prime\prime\prime\prime}$, 
способными обеспечивать точ\-ность своих производственных технологий 
порядка~23~нм, с~субстратом психики уровня мак\-ро\-мо\-ле\-кул/ге\-нов 
(компартментов\ ульт\-ра\-струк\-тур\-ных--про\-ка\-рио\-ти\-че\-ских--\linebreak
внут\-ри\-кле\-точ\-ных 
элементов эвкариотической клетки) и~уровнем 
об\-ра\-зо\-ван\-ности, со\-от\-вет\-ст\-ву\-ющим современному молодому  
че\-ло\-ве\-ку~18--29,1~лет (высшее обра\-зо\-ва\-ние\;+\;<<ас\-пи\-ран\-ту\-ра, 
с~защитой диссертации кандидата наук>>).
  
  Восьмой этап развития перспективной нано-ИТ (возможно, <<ИТ 
наноаппаратно поддерживаемой селективной телепатии>>~\cite{16-grn}), 
начавшийся около~1981~г., с~кульминацией (пиком скорости ее 
распространения на планете) около~2341~г.\ (расчетная дата), в~космическом 
объеме радиусом шара порядка~11,7~млн км (т.\,е.\ в~промежуточном 
космосе~\cite{5-grn}), реализовался {Homo 
sapiens}$^{\prime\prime\prime\prime\prime\prime }$, способными обеспечивать 
точность своих производственных технологий порядка~1,5~нм (отсюда 
наименование ИТ), с~субстратом психики уровня органических молекул 
(субкомпартментов ульт\-ра\-струк\-тур\-ных--про\-ка\-риоти\-че\-ских--внут\-ри\-кле\-точ\-ных 
элементов эвкариотической клетки) и~уровнем 
об\-ра\-зо\-ван\-ности,\linebreak соответству\-ющим современному зрелому  
человеку~29,1--47,1~лет (<<докторантура>>).
  
  Важно отметить, что процесс появления всех вышеперечисленных 
подсистем подчиняется кумулятивному принципу: возникновение каждой 
новой подсистемы не отменяет существование предыду\-щей: они все активно 
взаимодействуют между собой, коэволюционируют и~т.\,п., но исторически 
более ранние, естественно, постепенно переходят на второй, третий и~т.\,д.\ 
планы исторической сцены.
  
  Точка сходимости этого ряда находится около\linebreak 1981~г., знаменуя собой 
завершение этапа <<детст\-ва--от\-ро\-че\-ст\-ва--юности>> Человечества как 
целого и~начало этапа его <<зрелости>>~--- до\-сти\-же\-ния его максималь\-ной 
иерархической слож\-ности (чис\-ла уров\-ней/яру\-сов в~сис\-тем\-ной 
иерархии)~\cite{5-grn, 7-grn}.
  
  С позиции прогнозирования генезиса информационного общества на 
будущие времена отмечу, что, согласно ИКМ, тренд изменения времен 
запаздывания кульминаций развития под\-сис\-тем относительно их стартов 
сменился прямо на наших глазах. Если во временн$\acute{\mbox{о}}$м диапазоне с~428~млн 
лет назад и~до 1946~г.\ он со\-стоял в~равномерном (в~логарифмическом 
масштабе) укорочении согласно той же за\-ко\-но\-мер\-ности 
(в~$e\hm=15,15426\ldots$~раз), то в~диапазоне от~1946 по 1979~гг.\ это время 
запаздывания не изменилось, а~начиная с~1979~г.\ начало удлиняться 
(рис.~3). 
  

  
  Таким образом, метаэволюция сис\-те\-мы Человечества завершилась около 
1981~г.\ в~том смыс\-ле, что все воз\-мож\-ные ее ие\-рар\-хо-се\-те\-вые под\-сис\-те\-мы 
\textit{в~потенции} уже созданы. Но их \textit{актуализация}, дальнейшее 
услож\-не\-ние, эволюция и~коэволюция с~ранее возникшими аналогичными 
под\-сис\-те\-ма\-ми будет продолжаться неопределенно длительное время.

\vspace*{-10pt}
  
  \section*{Выводы}
  
  \vspace*{-2pt}
  
  \noindent
  \begin{enumerate}[1.]
\item  Изучение \textit{генезиса информационного общества} во всех его 
последовательных формах~--- от древности до современности и~далее~--- на 
базе\linebreak
 ин\-фор\-ма\-ти\-ко-ки\-бер\-не\-ти\-че\-ско\-го модельного подхода 
и~формализации процесса метаэволю\-ционного развития в~соответствующих 
терминах, позволило получить количественные\linebreak оценки его типовых  
про\-стран\-ст\-вен\-но-вре\-менн$\acute{\mbox{ы}}$х характеристик, 
а~также скоординированных с~ними во времени и~в~пространстве  
психико-ант\-ро\-по\-ло\-ги\-че\-ских, образовательных %\linebreak  
и~ин\-фор\-ма\-ци\-он\-но-ком\-му\-ни\-ка\-ци\-он\-ных параметров 
и~возможностей включенного в~этот процесс усложняющегося человека 
и~его сообществ различной величины.
  \item  Позиционирование ИТ локальных компьютеров и~ИТ 
телекоммуникаций в~качестве неотъемлемых составляющих совокупности\linebreak 
монотонно усложняющихся в~ходе цивилизационного развития~--- 
и~информационного общества!~--- ИТ позволяет 
рассматривать их появление и~функционирование в~широком контексте 
единой исторической ретроспективы и~перспективы, давая возможность 
делать не только теоретические, но и~практические выводы.
  \end{enumerate}
  
{\small\frenchspacing
 {%\baselineskip=10.8pt
 \addcontentsline{toc}{section}{References}
 \begin{thebibliography}{99}
\bibitem{1-grn}
\Au{Мелик-Гайгазян И.\,В.} Информационное общество~// Большая российская 
энциклопедия. Т.~11.~--- М.: Большая Российская энциклопедия, 2008. С.~490.
\bibitem{2-grn}
\Au{Гринченко С.\,Н.} Социальная метаэволюция Человечества как последовательность 
шагов формирования механизмов его системной памяти~// Исследовано в~России: 
Электронный журнал, 2001. Т.~145. С.~1652--1681. {\sf  
https://cyberleninka.ru/article/v/sotsialnaya-metaevolyutsiya-chelovechestva-kak-posledovatelnost-shagov-formirovaniya-mehanizmov-ego-sistemnoy-pamyati}.
\bibitem{3-grn}
\Au{Гринченко С.\,Н.} Системная память живого (как основа его метаэволюции
и~периодической структуры).~--- М.: ИПИ РАН, Мир, 2004. 512~с.
\bibitem{4-grn}
\Au{Grinchenko S.\,N.} Meta-evolution of nature system~--- the framework of history~// Social 
Evolution History, 2006. Vol.~5. No.\,1. P.~42--88.
\bibitem{5-grn}
\Au{Гринченко С.\,Н.} Метаэволюция (сис\-тем неживой, живой  
и~со\-ци\-аль\-но-тех\-но\-ло\-ги\-че\-ской природы).~--- М.: ИПИ РАН, 2007. 456~с.
\bibitem{6-grn}
\Au{Гринченко С.\,Н.} Homo eruditus (человек образованный) как элемент сис\-те\-мы 
Человечества~// Открытое образование, 2009. №\,2. С.~48--55.

\bibitem{10-grn} %7
\Au{Гринченко С.\,Н., Щапова~Ю.\,Л.} История Человечества: модели периодизации~// 
Вестник РАН, 2010. №\,12. С.~1076--1084.

%\bibitem{11-grn}  %8
%\Au{Grinchenko S.\,N., Shchapova~Y.\,L.} Human history periodization models~// Herald of the 
%Russian Academy of Sciences, 2010. Vol.~80. No.\,6. P.~498--506.
\bibitem{7-grn} %9
\Au{Grinchenko S.\,N.} The pre- and post-history of Humankind: What is it?~// Problems of 
contemporary world futurology.~--- Newcastle-upon-Tyne: Cambridge Scholars Publishing, 
2011. P.~341--353.
\bibitem{8-grn} %10
\Au{Гринченко С.\,Н.} Об эволюции психики как иерархической сис\-те\-мы 
(кибернетическое пред\-став\-ле\-ние)~// Историческая психология и~социология истории, 
2012. Т.~5. №\,2. С.~60--76.

\bibitem{12-grn} %11
\Au{Гринченко С.\,Н., Щапова~Ю.\,Л.} Информационные технологии в~истории 
Человечества.~--- М.: Новые технологии, 2013. 32~с. (Приложение к~журналу 
<<Информационные технологии>>, 2013. №\,8.)

\bibitem{9-grn} %12
\Au{Гринченко С.\,Н.} Эволюция темпов жизни людей и~развитие человечества~// Человек, 
2014. №\,5. С.~28--36.



\bibitem{13-grn}
\Au{Grinchenko S.\,N., Shchapova~Y.\,L.} Archaeological epoch as the succession of generations 
of evolutive subject-carrier archaeological sub-epoch~// Philosophy of Nature in Cross-Cultural 
Dimensions: The Result of the International Symposium at the University of Vienna~/ 
Komparative Philosophie und Interdisziplin$\ddot{\mbox{a}}$re Bildung (KoPhil). Band~5.~--- 
Hamburg: Verlag Dr.\ Kova$\Check{\mbox{c}}$, 2017. P.~478--499.
\bibitem{14-grn}
\Au{Щапова Ю.\,Л., Гринченко~С.\,Н.} Введение в~теорию археологической эпохи: 
числовое моделирование и~логарифмические шкалы про\-стран\-ст\-вен\-но-вре\-мен\-ных 
координат.~--- М.: Истфак МГУ, ФИЦ ИУ РАН, 2017. 236~с.
\bibitem{15-grn}
\Au{Grinchenko S.\,N., Shchapova~Yu.\,L.} Communications: Model representations about 
historical retrospective and possible perspective~// Communications Media 
Design Electronic~J., 2018. Vol.~3. No.\,2. P.~65--78.
\bibitem{16-grn}
\Au{Гринченко С.\,Н.} Послесловие~// Мат-лы доклада на Совместном научном семинаре 
ИПИ РАН и~\mbox{ИНИОН} РАН <<Методологические проблемы наук об информации>>.~---
М., 2012. С.~5--8. {\sf 
http://legacy.\linebreak inion.ru/files/File/MPNI\_9\_13\_12\_12\_posl.pdf}.
\bibitem{17-grn}
\Au{Жирмунский А.\,В., Кузьмин~В.\,И.} Критические уровни в~процессах развития 
биологических систем.~--- М.: Наука, 1982. 179~с.
\bibitem{18-grn}
\Au{Щапова Ю.\,Л.} Хронология и~периодизации древнейшей истории как числовая 
последовательность (ряд Фибоначчи)~// Информационный бюллетень Ассоциации 
<<История и~компьютер>>, 2000. №\,25.
\bibitem{19-grn}
\Au{Щапова Ю.\,Л.} Археологическая эпоха: хронология, периодизация, теория,  
модель.~--- М.: КомКнига, 2005. 192~с.
\bibitem{20-grn}
\Au{Щапова Ю.\,Л.} Материальное производство в~археологическую эпоху.~--- СПб.: 
Алетейя, 2011. 244~с.
\bibitem{21-grn}
\Au{Гринченко С.\,Н., Щапова~Ю.\,Л.} Пространство и~время в~археологии. Часть~3. 
О~метрике базисной пространственной структуры человечества в~археологическую 
эпоху~// Пространство и~время, 2014. №\,1(15). С.~78--89.
 \end{thebibliography}

 }
 }

\end{multicols}

\vspace*{-8pt}

\hfill{\small\textit{Поступила в~редакцию 17.10.18}}

\vspace*{6pt}

%\pagebreak

%\newpage

%\vspace*{-29pt}

\hrule

\vspace*{2pt}

\hrule

%\vspace*{-2pt}

\def\tit{ON THE GENESIS OF~THE~INFORMATION SOCIETY: INFORMATICS-CYBERNETIC 
MODEL REPRESENTATION}


\def\titkol{On the genesis of~the~information society: Informatics-cybernetic 
model representation}

\def\aut{S.\,N.~Grinchenko}

\def\autkol{S.\,N.~Grinchenko}

\titel{\tit}{\aut}{\autkol}{\titkol}

\vspace*{-11pt}


\noindent
Institute of Informatics Problems of the Federal Research Center ``Informatics and Control'' of 
the Russian Academy of Sciences, 44-2~Vavilov Str., Moscow 119333, Russian Federation

\def\leftfootline{\small{\textbf{\thepage}
\hfill INFORMATIKA I EE PRIMENENIYA~--- INFORMATICS AND
APPLICATIONS\ \ \ 2019\ \ \ volume~13\ \ \ issue\ 2}
}%
 \def\rightfootline{\small{INFORMATIKA I EE PRIMENENIYA~---
INFORMATICS AND APPLICATIONS\ \ \ 2019\ \ \ volume~13\ \ \ issue\ 2
\hfill \textbf{\thepage}}}

\vspace*{6pt}


  
  \Abste{The concept of the information society genesis is introduced, which is 
viewed from the standpoint of informatics-cybernetic modeling of the development 
of Humankind as a self-controlling hierarchical-networking system. On this basis, 
the author obtained quantitative assessments of its typical spatial-temporal 
characteristics, representing geometric progressions with the denominator ``$e$ to the 
degree~$e$'' (15.15426$\ldots$), as well as coordinated with them in time and space 
of the psychoanthropological, educational, and informational communication 
parameters and possibilities of a person who becomes complicated in this process 
and his communities of various sizes. This allowed us to push the framework of 
the information society for the entire historical and even archaeological epoch of 
such development. The resulting sequence of information technologies ``signal 
poses\,/\,sounds/movements\,--\,mimics/gestures\,--\,speech/language\,--\,writing\,--\,replicating 
texts\,--\,computers\,--\,telecommunications\,--\,information 
nanotechnology\,--\,$\ldots$'' allows us to consider the information society genesis 
in the broad context of a unified historical retrospective and perspective.}
  
  \KWE{information society; information technologies; informatics-cybernetic 
model; self-controlling hierarchical-networking system of Humankind; 
archaeological epoch}
  

\DOI{10.14357/19922264190214}

%\vspace*{-14pt}

%\Ack
%\noindent



%\vspace*{6pt}

  \begin{multicols}{2}

\renewcommand{\bibname}{\protect\rmfamily References}
%\renewcommand{\bibname}{\large\protect\rm References}

{\small\frenchspacing
 {%\baselineskip=10.8pt
 \addcontentsline{toc}{section}{References}
 \begin{thebibliography}{99}

\bibitem{1-grn-1}
\Aue{Melik-Gaygazyan, I.\,V.} 2008. Informatsionnoe ob\-shche\-st\-vo [Information 
society]. \textit{Bol'shaya rossiyskaya entsiklopediya} [Great Russian 
Encyclopedia].  Moscow: Great Russian 
Encyclopedia Publs. 11:490.
\bibitem{2-grn-1}
\Aue{Grinchenko, S.\,N.} 2001. Sotsial'naya me\-ta\-evo\-lyu\-tsiya Chelovechestva kak 
posledovatel'nost' shagov for\-mi\-ro\-va\-niya mekhanizmov ego sistemnoy pamyati 
[Social meta-evolution of Mankind as a~sequence of steps for the formation of the 
mechanisms of its system memory]. \textit{Elektronnyy zhurnal <<Issledovano 
v~Rossii>>} [Electronical J. ``Invstigated in Russia'']. 145:1652--1681. Avalable 
at: {\sf  
https://cyberleninka.ru/article/v/sotsialnaya-metaevolyutsiya-chelovechestva-kak-posledovatelnost-shagov-formirovaniya-mehanizmov-ego-sistemnoy-pamyati} (accessed 
October~5, 2018).
\bibitem{3-grn-1}
\Aue{Grinchenko, S.\,N.} 2004. \textit{Sistemnaya pamyat' zhivogo (kak osnova 
ego metaevolyutsii i~periodicheskoy struktury)} [System memory of the life (as the 
basis of its meta-evolution and periodic structure)]. Moscow: IPIRAN, MIR. 
512~p.
\bibitem{4-grn-1}
\Aue{Grinchenko, S.\,N.} 2006. Meta-evolution of nature system~--- the 
framework of history. \textit{Social Evolution History} 5(1):42--88.
\bibitem{5-grn-1}
\Aue{Grinchenko, S.\,N.} 2007. \textit{Metaevolyutsiya (sistem nezhivoy, zhivoy 
i~sotsial'no-tekhnologicheskoy prirody)} [Meta-evolution (of inanimate, animate, 
and socio-technological nature systems)]. Moscow: IPIRAN. 456~p. 
\bibitem{6-grn-1}
\Aue{Grinchenko, S.\,N.} 2009. Homo eruditus (chelovek obrazovannyy) kak 
element sistemy Chelovechestva [Homo eruditus (educated human) as an element 
of the Humakind's system]. \textit{Otkrytoe obrazovanie} [Open Education]  
2:48--55.

\bibitem{10-grn-1} %7
\Aue{Grinchenko, S.\,N., and Yu.\,I.~Shchapova.} 2010. 
Human history periodization models. \textit{Her. Russ. Acad. Sci.} 80(6):498--506.
%\bibitem{11-grn-1} %8
%\Aue{Grinchenko, S.\,N., and Y.\,I.~Shchapova.}  2010. Human history 
%periodization models. \textit{Herald of the Russian Academy of Sciences} 
%80(6):498--506.

\bibitem{7-grn-1} %9
\Aue{Grinchenko, S.\,N.} 2011.The pre- and post-history of Humankind: What is 
it?  \textit{Problems of contemporary world futurology}. 
 Newcastle-upon-Tyne: Cambridge Scholars 
Publishing.  341--353.
\bibitem{8-grn-1} %10
\Aue{Grinchenko, S.\,N.} 2012. Ob evolyutsii psikhiki kak ie\-rar\-khi\-che\-skoy 
sistemy (kiberneticheskoe predstavlenie) [On the evolution of mind as 
a~hierarchical system (a~cybernetic approach)]. \textit{Istoricheskaya 
psikhologiya i~sotsiologiya istorii} [Historical Psychology \& Sociology of 
History] 6(2):\linebreak 60--77.


\bibitem{12-grn-1} %11
\Aue{Grinchenko, S.\,N., and Y.\,I.~Shchapova.} 2013. \textit{In\-for\-ma\-tsi\-on\-nye 
tekhnologii v~istorii Chelovechestva} [Information technology in the history of 
Humankind]. Moscow: Novye tekhnologii. 32~p. (Prilozhenie k zhurnalu 
<<\textit{Informatsionnye tekhnologii}>> [Supplement to J.~Information Technology] 8.

\bibitem{9-grn-1} %12
\Aue{Grinchenko, S.\,N.} 2014. Evolyutsiya tempov zhizni lyudey i~razvitie 
chelovechestva [The evolution of the pace of human life and human development]. 
\textit{Human Being} 5:28--36.

\bibitem{13-grn-1}
\Aue{Grinchenko, S.\,N., and Y.\,I.~Shchapova.} 2017. Archaeological epoch as 
the succession of generations of evolutive subject-carrier archaeological  
sub-epoch. \textit{Philosophy of Nature in Cross-Cultural Dimensions: The Result of 
the International Symposium at the University of Vienna}~/ Komparative 
Philosophie und Interdisziplin$\ddot{\mbox{a}}$re Bildung (KoPhil), Band~5. 
Hamburg: Verlag Dr.\ Kova$\Check{\mbox{c}}$.  478--499.
\bibitem{14-grn-1}
\Aue{Shchapova, Y.\,L., and S.\,N.~Grinchenko.} 2017. \textit{Vvedenie 
v~teoriyu arkheologicheskoy epokhi: chislovoe modelirovanie i~logarifmicheskie 
shkaly prostranstvenno-vremennykh koordinat} [Introduction to the theory of the 
archaeological epoch: Numerical modeling and logarithmic scales of space--time 
coordinates]. Moscow: Faculty 
of History MSU, FRC CSC RAS]. 236~p. 

\vspace*{1pt}

\bibitem{15-grn-1}
\Aue{Grinchenko, S.\,N., and Y.\,I.~Shchapova}. 2018.  Communications: Model 
representations about historical retrospective and possible perspective. 
\textit{Communications Media Design Electronic~J.}  3(2):65--78. 
Available at: {\sf https://elibrary.ru/item.asp?id=36272286} (accessed October~5, 
2018).

\vspace*{1pt}

\bibitem{16-grn-1}
\Aue{Grinchenko, S.\,N.} 2012. Posleslovie [Afterword]. \textit{Mat-ly doklada 
na Sovmestnom nauchnom seminare IPI \mbox{INION} RAN ``Metodologicheskie 
problemy nauk ob informatsii''}  [Report materials at the Joint Scientific 
Seminar of the Institute of Informatics Problems of the Russian Academy of 
Sciences and the Institute of Scientific Information on Social Sciences of the 
Russian Academy of Sciences ``Methodological problems of information 
sciences''].  Moscow. 5--8.  Available at: {\sf 
http://legacy. inion.ru/files/File/MPNI\_9\_13\_12\_12\_posl.pdf} (accessed 
October~5, 2018).

\vspace*{1pt}

\bibitem{17-grn-1}
\Aue{Zhirmunskiy, A.\,V., and V.\,I.~Kuz'min.} 1982. \textit{Kriticheskie urovni 
v~protsessakh razvitiya biologicheskikh sistem} [Critical levels in the development 
of biological systems]. Moscow: Nauka. 179~p.

\vspace*{1pt}

\bibitem{18-grn-1}
\Aue{Shchapova, Y.\,L.} 2000. Khronologiya i~periodizatsii drev\-ney\-shey istorii 
kak chislovaya posledovatel'nost' (ryad Fibonachchi) [Chronology and 
periodization of ancient history as a numerical sequence (Fibonacci's series)]. 
\textit{Informatsionnyy byulleten' Assotsiatsii ``Istoriya i~komp'yuter''} 
[Newsletter of the Association ``History and Computer'']  25.

\vspace*{1pt}

\bibitem{19-grn-1}
\Aue{Shchapova, Y.\,L.} 2005. \textit{Arkheologicheskaya epokha: khro\-no\-lo\-giya, 
periodizatsiya, teoriya, model'} [Archaeological epoch: Chronology, periodization, 
theory, model]. Moscow: KomKniga, 192~p.

\vspace*{1pt}

\bibitem{20-grn-1}
\Aue{Shchapova, Y.\,L.} 2011. \textit{Material'noe proizvodstvo 
v~arkheologicheskuyu epokhu} [Material production in the archaeological epoch]. 
St.\ Petersburg: Aleteyya. 244~p.

\vspace*{1pt}

\bibitem{21-grn-1}
\Aue{Grinchenko, S.\,N., and Yu.\,I.~Shchapova.} 2014. Prostranstvo i~vremya 
v~arheologii. Chast'~3. O~metrike bazisnoy prostranstvennoy struktury 
chelovechestva v~arkheologicheskuyu epokhu [Space and time in archeology. 
Part~3. About the metric of Humankind basic spatial structure  in  
archaeological epoch]. \textit{Space and Time}  
1(15):\linebreak 78--89.
\end{thebibliography}

 }
 }

\end{multicols}

\vspace*{-6pt}

\hfill{\small\textit{Received October 17, 2018}}

%\pagebreak

%\vspace*{-18pt}


  
  \Contrl
  
  \noindent
   \textbf{Grinchenko Sergey N.} (b.\ 1946)~--- Doctor of Science in technology, professor, principal 
scientist, Institute of Informatics Problems, Federal Research Center ``Computer Science and 
Control'' of the Russian Academy of Sciences, 44-2~Vavilov Str., Moscow 119333, Russian 
Federation; \mbox{sgrinchenko@ipiran.ru}
\label{end\stat}

\renewcommand{\bibname}{\protect\rm Литература}     %15 

\def\stat{dorofeeva}

\def\tit{О ТОЧНОСТИ НОРМАЛЬНОЙ АППРОКСИМАЦИИ
ПРИ~ОТСУТСТВИИ НОРМАЛЬНОЙ СХОДИМОСТИ$^*$}

\def\titkol{О точности нормальной аппроксимации
при~отсутствии нормальной сходимости}

\def\aut{В.\,Ю.~Королев$^1$,  А.\,В.~Дорофеева$^2$}

\def\autkol{В.\,Ю.~Королев,  А.\,В.~Дорофеева}

\titel{\tit}{\aut}{\autkol}{\titkol}

\index{Королев В.\,Ю.}
\index{Дорофеева А.\,В.}
\index{Korolev V.\,Yu.}
\index{Dorofeeva A.\,V.}

{\renewcommand{\thefootnote}{\fnsymbol{footnote}} \footnotetext[1]
{Работа выполнена при поддержке РФФИ (проект 18-07-01405).
  }}

\renewcommand{\thefootnote}{\arabic{footnote}}
\footnotetext[1]{Факультет вычислительной математики и~кибернетики Московского государственного университета имени
 М.\,В.~Ломоносова; Институт проб\-лем информатики Федерального исследовательского цент\-ра 
 <<Информатика и~управ\-ле\-ние>> Российской академии наук, \mbox{vkorolev@cs.msu.ru}}
\footnotetext[2]{Факультет вычислительной математики и~кибернетики Московского государственного 
 университета имени М.\,В.~Ломоносова, \mbox{alex.dorofeyeva@gmail.com}}

%\vspace*{-12pt}

  

\Abst{При решении прикладных задач в~самых разных областях принято использовать 
нормальное распределение в~качестве модели статистических закономерностей в~наблюдаемых 
данных с~аддитивной структурой. В~качестве критерия степени адекватности такой 
модели можно использовать оценки ско\-рости сходимости в~центральной предельной теореме (ЦПТ) 
тео\-рии вероятностей, устанавливающей, что при определенных условиях (например, 
при условии Линдеберга) суммарное воздействие большого числа случайных факторов проявляется в~виде 
случайной величины с~нормальным распределением. 
Классические оценки скорости сходимости в~ЦПТ 
типа неравенства Бер\-ри--Эс\-се\-ена доказаны при условии конечности третьих моментов слагаемых.
 Известны также оценки скорости сходимости при существовании моментов порядка $2\hm+\delta$ с~$0<\delta\hm<1$. 
 Если существуют моменты лишь второго порядка, то сходимость в~ЦПТ может быть как угодно медленной. 
 Если же у слагаемых моменты второго порядка не существуют, то сходимость распределений сумм 
 независимых случайных величин к~нормальному закону не имеет места. 
 Условия, гарантирующие 
 справедливость ЦПТ, практически невозможно достоверно проверить 
 при ограниченном объеме наблюдаемой выборки. Поэтому вопрос о том, какой может быть 
 реальная точ\-ность нормальной аппроксимации, когда она теоретически не применима, 
 но используется в~практических вычислениях, представляет большой интерес. Более того, в~некоторых 
 ситуациях при имитационном моделировании, когда распределения слагаемых принадлежат об\-ласти 
 притяжения устойчивого закона с~характеристическим показателем, меньшим двух, при увеличении числа 
 слагаемых сначала наблюдается убывание расстояния между распределением нормированной суммы и~нормальным 
 законом и~лишь при довольно большом числе слагаемых это расстояние начинает увеличиваться. 
 В~данной заметке предпринята попытка дать ответ на сформулированный выше вопрос и~привести 
 некоторые теоретические объяснения указанному эффекту.}

\KW{центральная предельная теорема; точ\-ность нормальной аппроксимации; тяжелые хвос\-ты; 
равномерное расстояние}

\DOI{10.14357/19922264210116}

\vspace*{3pt}

\vskip 10pt plus 9pt minus 6pt

\thispagestyle{headings}

\begin{multicols}{2}

\label{st\stat}

\section{Введение}


При решении прикладных задач в~самых разных областях принято использовать нормальное 
распределение в~качестве модели статистических закономерностей в~наблюдаемых данных с~аддитивной структурой. 
В~качестве критерия степени адекватности такой модели можно использовать оценки ско\-рости 
сходимости в~ЦПТ тео\-рии вероятностей, устанавливающей, 
что при определенных условиях (например, при условии Линдеберга) суммарное воздействие 
большого числа случайных факторов проявляется в~виде случайной величины с~нормальным распределением. 

Классические оценки скорости сходимости в~ЦПТ типа неравенства Бер\-ри--Эс\-се\-ена доказаны при 
условии конечности третьих моментов сла\-га\-емых. 

Известны также оценки ско\-рости сходимости при 
существовании моментов порядка $2\hm+\delta$ с~$0\hm<\delta\hm<1$ (см.\ подробный обзор в~[1]).
 Если существуют моменты лишь второго порядка, то сходимость в~ЦПТ может быть как угодно медленной~[2, 3]. 
 Если же у слагаемых моменты второго порядка не существуют, то сходимость распределений 
 сумм независимых случайных величин к~нормальному закону не имеет места. 
 
 Условия, 
 гарантирующие справедливость ЦПТ, практически 
 невозможно достоверно проверить при ограниченном объеме наблюдаемой выборки. 
 В~част\-ности, гистограмма, построенная по выборке из имеющего очень тяжелые хвосты 
 распределения Коши (у~которого отсутствует даже математическое ожидание), при умеренном 
 объеме выборки может быть визуально практически неотличимой от нормального распределения. 
 Поэтому вопрос о том, какой может быть реальная точность нормальной аппроксимации, когда 
 она теоретически не применима, но используется в~практических вычислениях, представляет 
 большой интерес. Более того, в~некоторых ситуациях при имитационном моделировании, 
 когда распределения слагаемых принадлежат области притяжения устойчивого закона с~характеристическим 
 показателем, меньшим двух, при увеличении числа слагаемых сначала наблюдается убывание 
 расстояния между распределением нормированной суммы и~нормальным законом и~лишь при довольно 
 большом числе слагаемых это расстояние начинает увеличиваться. 
 
 В~данной заметке предпринята 
 попытка дать ответ на сформулированный выше вопрос и~привести некоторые теоретические объяснения 
 указанному эффекту.


\section{Обозначения и~вспомогательные результаты}


Пусть $n\hm\in\mathbb{N}$, $\xi_1,\ldots,\xi_n$~--- независимые необязательно одинаково 
распределенные случайные величины, заданные на вероятностном пространстве 
$(\Omega,\mathfrak{A},{\sf P})$. Обозначим $F_j(x)\hm={\sf P}(\xi_j<x)$, $x\hm\in\mathbb{R}$, 
$j\hm\in\mathbb{N}$. Без существенного ограничения общ\-ности для удобства будем считать, 
что все функции распределения~$F_j(x)$ непрерывны.

Обозначим $S_n=\xi_1+\cdots+\xi_n$. Индикатор множества (события)~$A$ обозначим $\mathbb{I}(A)$. 
Пусть $u\hm>0$. Очевидно, 
$$
\xi_j=\xi_j\mathbb{I}\left(|\xi_j|\le u\right)+\xi_j\mathbb{I}\left(|\xi_j|> u\right)\,.
$$
 Тогда
\begin{multline*}
S_n=\sum\limits_{j=1}^n \xi_j\mathbb{I}(|\xi_j|\le u)+\sum\limits_{j=1}^n \xi_j\mathbb{I}(|\xi_j|> u)\equiv{}\\
{}\equiv
S_n^{(\le u)}+S_n^{(> u)}.
\end{multline*}
Если условиться считать равенство единице индикатора $\mathbb{I}(|\xi_j|\hm\le u)$ <<успехом>>, 
а~противоположное событие~--- <<неудачей>>, то число~$N_n(u)$ ненулевых слагаемых в~сумме 
$S_n^{(\le u)}$ будет случайной величиной, имеющей пуас\-сон-би\-но\-ми\-аль\-ное распределение 
с~па\-ра\-мет\-ра\-ми~$n$ и~$p_j\hm=p_j(u)\hm={\sf P}(|\xi_j|\hm\le u)\hm=F_j(u)\hm-F_j(-u)$, 
$j\hm=1,\ldots,n$. Заметим, что при неограниченном увеличении~$u$ параметры~$p_j$ стремятся к~единице.

\smallskip

\noindent
\textbf{Лемма~1.}\ \textit{Пусть $A,B\hm\in\mathfrak{A}$. Тогда} 
$$
{\sf P}(A\bigcap B)\ge{\sf P}(A) -{\sf P}\left(\overline{B}\right).
$$

%\smallskip

\noindent
Д\,о\,к\,а\,з\,а\,т\,е\,л\,ь\,с\,т\,в\,о\  элементарно.

\smallskip

Равномерное расстояние между функциями распределения~$F_{\xi}$ и~$F_{\eta}$ случайных величин~$\xi$ 
и~$\eta$ будем обозначать $\rho(F_{\xi},\,F_{\eta})$: 
$$
\rho(F_{\xi},\,F_{\eta})= \sup\limits_x\left\vert F_{\xi}(x)-F_{\eta}(x)\right\vert\,.
$$

 Нормальную функцию распределения со средним $a\hm\in\mathbb{R}$ и~дисперсией 
$\sigma^2\hm>0$ обозначим~$\Phi_{a,\sigma}$:
\begin{multline*}
\Phi_{a,\sigma}(x)=\fr{1}{\sigma\sqrt{2\pi}}
\int\limits_{-\infty}^{x}\exp\left\{-\fr{(z-a)^2}{2\sigma^2}\right\}dz={}\\
{}=
\Phi_{0,1}\left(\fr{x-a}{\sigma}\right)=
\Phi_{0,\sigma}(x-a)\,,\enskip x\in\mathbb{R}\,.
\end{multline*}

\smallskip

\noindent
\textbf{Лемма~2.}\ \textit{Для любых $a\hm\in\mathbb{R}$, $\sigma\hm>0$, $b\hm\in\mathbb{R}$}
$$
\rho\left(\Phi_{a+b,\sigma},\,\Phi_{a,\sigma}\right)= 
2\Phi_{0,\sigma}\left(\fr{|b|}{2}\right)-1\,.
$$

\smallskip

\noindent
Д\,о\,к\,а\,з\,а\,т\,е\,л\,ь\,с\,т\,в\,о\,.\ \
 Заметим, что если $H(x)$ и~$G(x)$~--- дифференцируемые функции распределения, то $\rho(H,G)$ 
 реализуется (т.\,е.\ точная верхняя грань $\sup_x|H(x)\hm-G(x)|$ по~$x$ достигается) в~одной из точек~$x$, 
 где $F'(x)\hm=G'(x)$. Действительно,
\begin{multline*}
\rho(H,G)=\sup\limits_x|H(x)-G(x)|={}\\
{}=\max\left\{\max\limits_x\left[H(x)-G(x)\right], 
\max\limits_x\left[G(x)-H(x)\right]\right\}
\end{multline*}
и экстремум каждого из выражений в~фигурных скобках достигается в~такой точке, где производная 
соответствующего выражения равна нулю, что равносильно равенству производных функций 
распределения~$H$ и~$G$, т.\,е.\ равенству соответствующих плотностей. 
В~рассматриваемом случае нормальных плотностей последнее условие эквивалентно тому, что
\begin{multline*}
\fr{1}{\sigma\sqrt{2\pi}}\exp\left\{-\fr{1}{2}\left(\fr{x-a-b}{\sigma}\right)^2\right\}={}\\
{}=
\fr{1}{\sigma\sqrt{2\pi}}\exp\left\{-\fr{1}{2}\left(\fr{x-a}{\sigma}\right)^2\right\},
\end{multline*}
или $\big(x-(a+b)\big)^2\hm=(x-a)^2$. Решая данное уравнение, получаем $x\hm-a\hm={b}/{2}$, 
откуда с~учетом соотношения $\Phi_{0,\sigma}(-|b|)\hm=1-\Phi_{0,\sigma}(|b|)$ вытекает требуемое утверждение.

\smallskip

Используя формулу Лагранжа, из леммы~2 легко получить известное неравенство:
$$
\rho\left(\Phi_{a+b,\sigma},\Phi_{a,\sigma}\right)\le \fr{|b|}{\sigma\sqrt{2\pi}}
$$
(см., например, неравенство~(3.4) в~книге~[4]).

\smallskip

\noindent
\textbf{Лемма~3.}\ \textit{Пусть $n\hm\in\mathbb{N}$, $\xi_1,\ldots,\xi_n$~--- случайные величины, 
$a_1,\ldots,a_n$~--- положительные числа, такие что
$a_1+\cdots+a_n\hm=1$. Тогда для любого $x\hm>0$
$$
{\sf P}\left(\left|\sum\limits_{j=1}^n\xi_j\right|\ge x\right)\le\sum\limits_{j=1}^n{\sf P}\left(|\xi_j|\ge a_jx\right).
$$
Если дополнительно случайные величины $\xi_1,\ldots,\xi_n$ одинаково распределены, то}
$$
{\sf P}\left(\left|\sum\limits_{j=1}^n\xi_j\right|\ge x\right)\le n{\sf P}\left(\left\vert \xi_1\right\vert \ge 
\fr{x}{n}\right).
$$


\smallskip

\noindent
Д\,о\,к\,а\,з\,а\,т\,е\,л\,ь\,с\,т\,в\,о\,.\ \ Заметим, что
$$
{\sf P}\left(\left|\sum\limits_{j=1}^n\xi_j\right|\ge x\right)\le{\sf P}\left(\sum\limits_{j=1}^n|\xi_j|
\ge x\right).
$$
Из геометрических соображений вытекает, что
\begin{multline*}
\left\{\omega:\, \sum\limits_{j=1}^n|\xi_j(\omega)|\ge x\right\}\subseteq
\left\{\omega:\,\left\vert \xi_1(\omega)\right\vert \ge a_1x\right\}\bigcup{}\\
{}\bigcup
\left\{\omega:\, \sum\limits_{j=2}^n \left\vert \xi_j(\omega)\right\vert \ge 
\left(1-a_1\right)x\right\}\subseteq
{}\\
{}
\subseteq\left\{\omega:\,\left\vert \xi_1(\omega)\right\vert \ge a_1x\right\}\bigcup
\left\{\omega:\,\left\vert \xi_2(\omega)\right\vert \ge a_2x\right\}\bigcup{}\\
{}\bigcup
\left\{\omega:\, \sum\limits_{j=3}^n\left\vert \xi_j(\omega)\right\vert
\ge \left(1-a_1-a_2\right)x\right\}\subseteq\cdots
\\
\cdots\subseteq\bigcup\limits_{j=1}^n\left\{\omega:\,\left\vert \xi_j(\omega)\right\vert
\ge a_jx\right\}.
\end{multline*}
Поэтому
\begin{multline*}
{\sf P}\left(\left\vert \sum\nolimits_{j=1}^n\xi_j\right\vert \ge x\right)
\le{\sf P}\left(\sum\limits_{j=1}^n\left\vert \xi_j\right\vert \ge x\right)\le
{}\\
{}
\le{\sf P}\!\!\left(\bigcup\limits_{j=1}^n\left\{\omega:\,\left\vert \xi_j(\omega)\right\vert
\ge a_jx\right\}\!\right)\!
\le\!\sum\limits_{j=1}^n{\sf P}\left(\left\vert \xi_j\right\vert \ge a_jx\right).\hspace*{-0.93279pt}
\end{multline*}
Лемма доказана.


\section{Основные результаты}


Рассмотрим оценку равномерного расстояния между распределением суммы $S_n\hm=S_n^{(\le u)}
\hm+S_n^{(>u)}$ и~нормальным законом с~соответствующими математическим ожиданием 
$a\hm\in\mathbb{R}$ и~дисперсией $\sigma\hm>0$, конкретный выбор которых прокомментируем \mbox{ниже}.

\smallskip

\noindent
\textbf{Теорема~1.} \textit{ Пусть $\epsilon\hm>0$, $u\hm>0$~--- произвольны. 
Тогда для любых $a\hm\in\mathbb{R}$, $\sigma\hm>0$}
\begin{multline}
\rho\left(F_{S_n},\Phi_{a,\,\sigma}\right)\le
\rho\left(F_{S_n^{(\le u)}},\Phi_{a,\,\sigma}\right)+{}\\
{}+
\sum\limits_{j=1}^n\left[F_j(-u)+1-F_j(u)\right].
\label{e1-dor}
\end{multline}

\smallskip

\noindent
Д\,о\,к\,а\,з\,а\,т\,е\,л\,ь\,с\,т\,в\,о\,.\ \
 Пусть $\epsilon\hm>0$ произвольно. На основании леммы~1 имеем:
\begin{multline}
{\sf P}\left(S_n<x\right)={\sf P}\left(S_n<x;\,\left\vert S_n^{(>u)}\right\vert
\le\epsilon\right)+{}\\
{}+{\sf P}\left(S_n<x;\,\left\vert S_n^{(>u)}\right\vert >\epsilon\right)\ge{}
\\
{}\ge{\sf P}\left(S_n^{(\le u)}<x-S_n^{(>u)};\,|S_n^{(>u)}|\le\epsilon\right)\ge{}\\
{}\ge
{\sf P}\left(S_n^{(\le u)}<x-\epsilon;\,\left\vert S_n^{(>u)}\right\vert \le\epsilon\right)\ge
{}\\
{}\ge{\sf P}\left(S_n^{(\le u)}<x-\epsilon\right)-{\sf P}\left(
\left\vert S_n^{(>u)}\right\vert \ge\epsilon\right).
\label{e2-dor}
\end{multline}
С другой стороны, очевидно:
\begin{multline}
{\sf P}\left(S_n<x\right)={\sf P}\!\left(S_n^{(\le u)}<x-S_n^{(>u)};\,\left\vert S_n^{(>u)}\right\vert
\le\epsilon\right)+{}\\
{}+
{\sf P}\left(S_n<x;\,\left\vert S_n^{(>u)}\right\vert >\epsilon\right)\le
{}\\
{}
\le{\sf P}\left(S_n^{(\le u)}<x+\epsilon;\,\left\vert S_n^{(>u)}\right\vert \le\epsilon\right)+{}\\
{}+
{\sf P}(S_n<x;\,\left\vert S_n^{(>u)}\right\vert >\epsilon)\le
{}\\
{}\le{\sf P}\left(S_n^{(\le u)}<x+\epsilon\right)+{\sf P}\left(\left\vert S_n^{(>u)}\right\vert
>\epsilon\right).
\label{e3-dor}
\end{multline}
Легко видеть, что
\begin{multline}
\hspace*{-9.96pt}\left\vert {\sf P}\left(S_n<x\right)-\Phi_{a,\sigma}(x)\right\vert\!=\!
\max\left\{{\sf P}\left(S_n<x\right)-\Phi_{a,\sigma}(x),\right.\\
\left.\Phi_{a,\sigma}(x)-{\sf P}\left(S_n<x\right)\right\}.
\label{e4-dor}
\end{multline}
Используя~(\ref{e3-dor}) и~лемму~2, получим:
\begin{multline}
{\sf P}\left(S_n<x\right)-\Phi_{a,\sigma}(x)\le 
{\sf P}\left(\left\vert S_n^{(>u)}\right\vert>\epsilon\right)+{}\\
{}+
\left[{\sf P}\left(S_n^{(\le u)}<x+\epsilon\right)-\Phi_{a,\sigma}(x+\epsilon)\right]+{}
\\
{}+\left[\Phi_{a,\sigma}(x+\epsilon)-\Phi_{a,\sigma}(x)\right]\le
{\sf P}\left(\left\vert S_n^{(>u)}\right\vert >\epsilon\right)+{}\\
{}+
\rho\left(F_{S_n^{(\le u)}},\Phi_{a,\,\sigma}\right)+
\left[2\Phi_{0,\sigma}\left({\fr{\epsilon}{2}}\right)-1\right].
\label{e5-dor}
\end{multline}
Используя~(\ref{e2-dor}) и~лемму~2, получим
\begin{multline}
\Phi_{a,\sigma}(x)-{\sf P}\left(S_n<x\right)\le 
\Phi_{a,\sigma}(x)-{}\\
{}-{\sf P}\left(S_n^{(\le u)}<x-\epsilon\right)+{\sf P}\left(\left\vert S_n^{(>u)}\right\vert
>\epsilon\right)={}
\\
{}=\left[\Phi_{a,\sigma}(x)-\Phi_{a,\sigma}(x-\epsilon)\right]-
\left[{\sf P}\left(S_n^{(\le u)}<x-\epsilon\right)-{}\right.\\
\left.{}-\Phi_{a,\sigma}(x-\epsilon)
\vphantom{\left(S_n^{(\le u)}<x-\epsilon\right)}
\right]+ 
{\sf P}\left(\left\vert S_n^{(>u)}\right\vert>\epsilon\right) \le
{}\\
{}\le\left[2\Phi_{0,\sigma}\left({\fr{\epsilon}{2}}\right)-1\right]+
\rho(F_{S_n^{(\le u)}},\Phi_{a,\sigma})+{}\\
{}+{\sf P}\left(\left\vert S_n^{(>u)}\right\vert>\epsilon\right).
\label{e6-dor}
\end{multline}
Подставив оценки~(\ref{e5-dor}) и~(\ref{e6-dor}) в~(\ref{e4-dor}), получим
\begin{multline}
\rho(F_{S_n},\Phi_{a,\sigma})\le\rho\left(F_{S_n^{(\le u)}},\Phi_{a,\sigma}\right)+{}\\
{}+
\left[2\Phi_{0,\sigma}\left({\fr{\epsilon}{2}}\right)-1\right]+
{\sf P}\left(\left\vert S_n^{(>u)}\right\vert >\epsilon\right).
\label{e7-dor}
\end{multline}
Рассмотрим третье слагаемое в~правой части~(\ref{e7-dor}). На основании леммы~3 
по формуле полной ве\-ро\-ят\-ности, принимая во внимание тот факт, что 
${\epsilon}/{n}\hm>0$ и~$|\xi_j(\omega)|\mathbb{I}(|\xi_j(\omega)|>u)\hm=0$ для тех~$\omega$, 
для которых $|\xi_j(\omega)|\hm\le u$, $j\hm=1,\ldots,n$, имеем:
\begin{multline}
{\sf P}\left(\left\vert S_n^{(>u)}\right\vert > \epsilon\right)=
{\sf P}\left(\left\vert \sum\limits_{j=1}^n\xi_j\mathbb{I}
\left(\left\vert \xi_j\right\vert >u\right)\right\vert>\epsilon\right)\le{}\\
{}\le
\sum\limits_{j=1}^n{\sf P}\left(
\left\vert \xi_j\right\vert \mathbb{I}\left(\left\vert \xi_j\right\vert >u\right)>
{\fr{\epsilon}{n}}\right)=
{}\\
{}=\!\sum\limits_{j=1}^n\!\left[ {\sf P}\left(\!|\xi_j|\mathbb{I}(|\xi_j|>u)>
{\fr{\epsilon}{n}}\!\left\vert\!
\vphantom{\fr{\epsilon}{n}} \right.
|\xi_j|>u\!\right)\!{\sf P}(|\xi_j|>u)+{}\right.
\\
\left.{}+{\sf P}\!\left(\left\vert \xi_j\right\vert
\mathbb{I}\left(\left\vert \xi_j\right\vert >u\right)>{\fr{\epsilon}{n}}\!\left\vert\!
\vphantom{\fr{\epsilon}{n}}\right.\,\left\vert \xi_j\right\vert 
\le u\right)
{\sf P}\left(\left\vert \xi_j\right\vert \le u\right)\right]={}
\\
{}=\sum\limits_{j=1}^n\left[1-p_j(u)\right]
{\sf P}\!\left(\!|\xi_j|\mathbb{I}\left(\left\vert \xi_j\right\vert>u\right)>
\left.{\fr{\epsilon}{n}}
\right\vert
\left\vert \xi_j\right\vert >u\right)\le{}\\
{}\le
\sum\limits_{j=1}^n\left[F_j(-u)+1-F_j(u)\right].
\label{e8-dor}
\end{multline}
Подставив~(\ref{e8-dor}) в~(\ref{e7-dor}) и~устремив~$\epsilon$ к~нулю, получим требуемое. Теорема доказана.

\smallskip


На практике в~качестве параметров~$a$ и~$\sigma$ можно брать, например,
\begin{multline*}
a=a(u)={\sf E}S_n^{(\le u)}={\sf E}\sum\limits_{j=1}^n \xi_j\mathbb{I}\left(\left\vert \xi_j\right\vert 
\le u\right)={}\\
{}= \sum\limits_{j=1}^n {\sf E}\left[\xi_j\mathbb{I}\left(\left\vert \xi_j\right\vert
\le u\right)\right]=\sum\limits_{j=1}^np_j(u)\int\limits_{-u}^{u}x\,dF_j(x);
\end{multline*}

%\vspace*{-12pt}

\noindent
\begin{multline*}
\sigma^2=\sigma^2(u)={\sf D}S_n^{(\le u)}={\sf D}\sum\limits_{j=1}^n 
\xi_j\mathbb{I}\left(\left\vert \xi_j\right\vert \le u\right)={}\\
{}=
\sum\limits_{j=1}^n {\sf D}\left[\xi_j\mathbb{I}\left(\left\vert \xi_j\right\vert \le u\right)\right]=
{}\\
{}=\sum\limits_{j=1}^n\left[p_j(u)\!\int\limits_{-u}^{u}\!x^2\,dF_j(x)-
\left(\! p_j(u)\!\int\limits_{-u}^{u}\!x\,dF_j(x)\!
\right)^{\!\!2\,}\!\right].
\end{multline*}

При фиксированном $u\hm>0$ первое слагаемое в~правой части~(\ref{e1-dor}) 
убывает с~увеличением~$n$, тогда как второе возрастает. При этом существует $n_0\hm\ge1$ такое, что при
 $1\hm\le n\hm\le n_0$ вся правая часть~(\ref{e1-dor}) убывает, а~при $n\hm\ge n_0$ воз\-рас\-та\-ет. 
 В~случае одинаково распределенных слагаемых второе слагаемое воз\-рас\-та\-ет как~$kn$, где $k\hm>0$. 
 При этом за счет выбора очень большого~$u$ можно добиться произвольной малости коэффициента~$k$ 
 и,~как следствие, очень медленного роста второго слагаемого. Поэтому параметр~$n_0$ может 
 принимать довольно большие значения.

Задачу определения указанного~$n_0$ конкретизируем для частного случая. Предположим, что
\begin{equation}
\rho\left(F_{S_n^{(\le u)}},\Phi_{a,\sigma}\right)\le C n^{-\gamma}, \label{e9-dor}
\end{equation}
где $\gamma>0$, а коэффициент $C\hm=C(u)$ определяется свойствами распределений слагаемых 
в~сумме $S_n^{(\le u)}$. Некоторые критерии справедливости~(\ref{e9-dor}) 
приведены, например, в~[5]. Предположим также, что
\begin{equation}
\sum\limits_{j=1}^n\left[F_j(-u)+1-F_j(u)\right]\le kn\,,
\label{e10-dor}\end{equation}
где $k=k(u)\hm>0$. Несложно убедиться, что минимум функции
$$
g(x)=\fr{C}{x^{\gamma}}+kx
$$
достигается в~точке
$$
x_0=\left(\fr{C\gamma}{k}\right)^{{1}/({1+\gamma})},
$$
причем
\begin{equation}
\min\limits_{x\ge0}g(x)=g(x_0)=
(\gamma+1)\left(\fr{Ck^{\gamma}}{\gamma^{\gamma}}\right)^{{1}/({\gamma+1})}.
\label{e11-dor}
\end{equation}
Таким образом, в~качестве~$n_0$ выступает либо $[x_0]$, либо $[x_0]+1$. 
При этом правую часть~(\ref{e11-dor}) можно рассматривать как приближенное значение наилучшей 
верхней оценки точности нормальной аппроксимации при справедливости условий~(\ref{e9-dor}) и~(\ref{e10-dor}).

{\small\frenchspacing
{%\baselineskip=10.8pt
%\addcontentsline{toc}{section}{References}
\begin{thebibliography}{9}
\bibitem{Shevtsova2016}
\Au{Шевцова И.\,Г.} Точность нормальной аппроксимации: 
методы оценивания и~новые результаты.~--- М.: Аргамак-Медиа, 2016. 380~с.

\bibitem{Mackevicius1983}
\Au{Мацкявичюс В.\,К.} О~нижней оценке скорости сходимости в~центральной предельной теореме~// 
Теория вероятностей и~ее применения,
1983. Т.~28. Вып.~3. С.~565--569.

\bibitem{KorolevDorofeeva2017}
\Au{Korolev V.\,Yu., Dorofeeva~A.\,V.} Bounds of the accuracy of 
the normal approximation to the distributions of random sums under relaxed moment conditions~// 
Lith. Math.~J., 2017. Vol.~57. No.\,1. P.~38--58.

\bibitem{Petrov1972}
\Au{Петров В.\,В.} Суммы независимых случайных величин.~--- М.: Наука, 1972. 414~с.

%\bibitem{Klebanov1999}
%{\it L. B. Klebanov, S. T. Rachev, G. J. Szekely.} Pre-limit theorems and their applications // Acta Applicandae Mathematicae, 1999. Vol 58. P. %159--174.

\bibitem{Ibragimov1966}
\Au{Ибрагимов И.\,А.} О~точ\-ности аппроксимации функций распределения сумм 
независимых случайных величин нормальным распределением~// Теория вероятностей и~ее применения, 1966. 
Т.~11. Вып.~4. С.~632--655.
\end{thebibliography}

}
}

\end{multicols}

\vspace*{-3pt}

\hfill{\small\textit{Поступила в~редакцию 13.10.2020}}

\vspace*{8pt}

%\pagebreak

%\newpage

%\vspace*{-28pt}

\hrule

\vspace*{2pt}

\hrule

%\vspace*{-2pt}

\def\tit{ON THE ACCURACY OF~THE~NORMAL APPROXIMATION UNDER~THE~VIOLATION OF~THE~NORMAL CONVERGENCE}

\def\titkol{On the accuracy of~the~normal approximation under~the~violation of~the~normal convergence}

\def\aut{V.\,Yu.~Korolev$^{1,2}$ and~A.\,V.~Dorofeeva$^1$}

\def\autkol{V.\,Yu.~Korolev and~A.\,V.~Dorofeeva}

\titel{\tit}{\aut}{\autkol}{\titkol}

\vspace*{-11pt}


\noindent
$^1$Faculty of Computational Mathematics and Cybernetics, 
M.\,V.~Lomonosov Moscow State University, GSP-1,\linebreak
$\hphantom{^1}$Leninskie Gory, Moscow 119991, Russian Federation

\noindent
$^2$Institute of Informatics Problems, Federal Research Center ``Computer Sciences and Control'' 
of the Russian\linebreak
$\hphantom{^1}$Academy of Sciences; 44-2~Vavilov Str., Moscow 119133, Russian Federation

\def\leftfootline{\small{\textbf{\thepage}
\hfill INFORMATIKA I EE PRIMENENIYA~--- INFORMATICS AND
APPLICATIONS\ \ \ 2021\ \ \ volume~15\ \ \ issue\ 1}
}%
\def\rightfootline{\small{INFORMATIKA I EE PRIMENENIYA~---
INFORMATICS AND APPLICATIONS\ \ \ 2021\ \ \ volume~15\ \ \ issue\ 1
\hfill \textbf{\thepage}}}

\vspace*{6pt}





\Abste{When solving applied problems in various fields, it is conventional 
to use the normal approximation to the distribution of data with additive structure. 
As a~criterion of the adequacy of such a~model, it is possible to use bounds for the convergence 
rate in the central limit theorem (CLT) of the probability theory stating that under certain 
conditions (say, under the Lindeberg condition), the total effect of very many random factors acts as 
a~random variable with the normal distribution. The classical bounds for the convergence 
rate in the CLT such as the Berry--Esseen inequality are proved under the condition 
that the third moments of the summands exist. Also, bounds are known that require the
 existence of the moments of orders $2+\delta$ with $0< \delta <1$. 
 If only the moments of the second order exist, then the convergence in the CLT can 
 be arbitrarily slow. But if the moments of the summands of the second order do 
 not exist, then the convergence of the distributions of sums of independent 
 random variables to the normal law does not take place. It is practically impossible 
 to reliably check the conditions of the central limit theorem with the limited size 
 of the available sample. Therefore, the question of what is the real accuracy 
 of the normal approximation if it is theoretically impossible is of great interest. 
 Moreover, in some situations, in computer simulation of sums of 
 random variables whose distributions belong to the domain of attraction of the stable 
 distribution with the characteristic exponent less than two, as the number of summands grows, 
 first, the distance between the distribution of the normalized sum and the normal law 
 decreases and starts to increase only when the number of summands becomes sufficiently large. 
 In this paper, an attempt is undertaken to give some theoretical explanation of this 
 effect and to give an answer to the question posed above.}

       
\KWE{central limit theorem; accuracy of normal approximation; heavy tails; uniform distance}



\DOI{10.14357/19922264210116}

%\vspace*{-15pt}

\Ack
\noindent
Research supported by the Russian Foundation for Basic Research, project 18-07-01405.

\vspace*{4pt}

  \begin{multicols}{2}

\renewcommand{\bibname}{\protect\rmfamily References}
%\renewcommand{\bibname}{\large\protect\rm References}

{\small\frenchspacing
 {%\baselineskip=10.8pt
 \addcontentsline{toc}{section}{References}
 \begin{thebibliography}{9}
\bibitem{1-dor}
\Aue{Shevtsova, I.\,G.} 2016. \textit{Tochnost' normal'noy approksimatsii: metody otsenivaniya 
i~novye resul'taty} [The accuracy of the normal approximation: 
Methods of estimation and new results]. Moscow: Argamak-Media. 380~p.
{\looseness=1

}

\columnbreak


\bibitem{2-dor}
\Aue{Mackevicius, V.\,K.} 1984. A~lower bound for the convergence rate in the central limit theorem. 
\textit{Theor. Probab. Appl.} 28(3):596--601.

\vspace*{-2pt}

\bibitem{3-dor}
\Aue{Korolev, V.\,Yu., and A.\,V.~Dorofeeva.}
 2017. Bounds of the accuracy of the normal approximation to the distributions of random 
 sums under relaxed moment conditions. \textit{Lith. Math.~J.} 57(1):38--58.
\bibitem{4-dor}
\Aue{Petrov, V.\,V.} 1972. \textit{Summy nezavisimykh sluchaynykh velichin} 
[Sums of independent random variables]. Moscow: Nauka. 416~p.
\bibitem{5-dor}
\Aue{Ibragimov, I.\,A.} 1966. On the accuracy of Gaussian approximation to the distribution functions 
of sums of independent variables. \textit{Theor. Probab. Appl.} 11(4):559--579.
 \end{thebibliography}

 }
 }

\end{multicols}

\vspace*{-3pt}

  \hfill{\small\textit{Received October~13, 2020}}


%\pagebreak

%\vspace*{-8pt}     

\Contr

\noindent
\textbf{Korolev Victor Yu.} (b.\ 1954)~--- 
Doctor of Science in physics and mathematics, professor, Head of Department, Faculty of 
Computational Mathematics and Cybernetics, M.\,V.~Lomonosov Moscow State University, 
GSP-1, Leninskie Gory, Moscow 119991, Russian Federation; leading scientist, 
Federal Research Center ``Computer Science and Control'' 
of the Russian Academy of Sciences, 44-2~Vavilov Str.,Moscow 119333, Russian Federation; 
\mbox{vkorolev@cs.msu.ru}

\vspace*{6pt}

\noindent
\textbf{Dorofeeva Alexandra V.} (b.\ 1991)~--- 
graduate PhD student, Faculty of Computational Mathematics and Cybernetics, 
M.\,V.~Lomonosov Moscow State University, GSP-1, Leninskie Gory, Moscow 119991, 
Russian Federation; \mbox{alex.dorofeyeva@gmail.com}

\label{end\stat}

\renewcommand{\bibname}{\protect\rm Литература}  %16














%%%%%%%%%%%%%%%%%%%%%%%%%%%%%%%%%%%%%%%%

%\def\stat{rez}
{%\hrule\par
%\vskip 7pt % 7pt
\raggedleft\Large \bf%\baselineskip=3.2ex
Р\,Е\,Ц\,Е\,Н\,З\,И\,И \vskip 17pt
    \hrule
    \par
\vskip 6pt plus 6pt minus 3pt }

%\thispagestyle{headings} %с верхним колонтитулом
%\thispagestyle{myheadings} %с нижним колонтитулом, но в верхнем РЕЦЕНЗИИ

\def\tit{НОВАЯ КНИГА И.\,Н.~СИНИЦЫНА, А.\,С.~ШАЛАМОВА <<ЛЕКЦИИ ПО ТЕОРИИ 
ИНТЕГРИРОВАННОЙ ЛОГИСТИЧЕСКОЙ ПОДДЕРЖКИ>> (М.: ТОРУС ПРЕСС, 2012. 624~с.)}

%1
\def\aut{Д.ф.-м.н., профессор С.\,Я.~Шоргин}

\def\auf{\ }

\def\leftkol{\ % РЕЦЕНЗИИ
}

\def\rightkol{ \ } 

%\def\leftkol{\ } % ENGLISH ABSTRACTS}

%\def\rightkol{\ } %ENGLISH ABSTRACTS}

%\def\leftkol{РЕЦЕНЗИИ}

%\def\rightkol{РЕЦЕНЗИИ}

\titele{\tit}{\aut}{\auf}{\leftkol}{\rightkol}
\vspace*{-18pt}


     \label{st\stat}

     \begin{multicols}{2}
     {\small
     {\baselineskip=10.1pt
     

      В книге представлено системное изложение теоретических основ одного из новейших 
направлений в \mbox{об\-ласти} экономики послепродажного обслуживания изделий наукоемкой 
продукции (ИНП) длительного пользования~--- интегрированной логистической поддержки
(ИЛП). 
{\looseness=1

}

Приведены также результаты новых работ, выполненных в Институте проблем информатики 
Российской академии наук в рамках научного направления <<Информационные технологии и 
анализ сложных сис\-тем>>.
 {%\looseness=1

}
     
      Излагаемые в книге научные подходы позво\-ляют карди\-наль\-но реформировать 
существующие системы производства и эксплуатации ИНП путем создания и внед\-ре\-ния 
методов рационального и оптимального управ\-ле\-ния процессами расходования 
вре\-мен\-н$\acute{\mbox{ы}}$х, 
мате\-ри\-аль\-ных, трудовых и других ресурсов на всех стадиях жизненного цикла изделий (ЖЦИ) по 
критериям экономической целесообразности и эф\-фек\-тив\-ности.
  {\looseness=1

}
    
      В книге приведен краткий обзор причин возник\-новения и
      развития CALS-методологии как основы 
современных международных стандартов по созданию и функционированию глобальных 
ин\-фор\-ма\-ци\-он\-но-ком\-му\-ни\-ка\-ци\-он\-ных систем, ее ключевых возможностей и эффективности 
результатов ее использования. 
Авторы %\linebreak 
предлагают ряд научных обоснований для разработки 
единой теории проектирования и управления систем ИЛП для полноценного использования 
преимуществ %\linebreak
 суще\-ст\-ву\-ющей методологии, определяют \mbox{общую} структурную схему 
комплексной системы <<ИНП-СППО>> и необходимость разработки для ее описания 
гибридных стохастических моделей.
{%\looseness=1

}

%\columnbreak
      
      Книга состоит из пяти частей, где последовательно излагается материал по каждой из 
следующих тем: <<Интегрированная логистическая поддержка>>, <<Теория гибридных 
стохастических систем и компьютерная поддержка исследований и разработок>>, <<Основы 
математического моделирования, анализа и синтеза систем послепродажного обслуживания>>, 
<<Определение и анализ показателей экспортного потенциала ИНП при проектировании>>, 
<<Задачи управления поддержкой послепродажного обслуживания>>, а также 
<<Моделирование инвестиционных процессов ИЛП в условиях неравновесных финансовых 
рынков>>. 
   
      В конце каждой главы приведены выводы и даны вопросы и задания для 
самоконтроля. В~приложениях содержатся основные определения по программам работ по 
анализу ИЛП, логистическим базам данных и компьютерным решениям, эквивалентной статистической 
линеаризации нелинейных преобразований ИЛП, справочный материал, а также развернутые 
уравнения для вероятностных характеристик.


      \def\leftkol{РЕЦЕНЗИИ}

\def\rightkol{РЕЦЕНЗИИ} 

      
      Книга заинтересует широкий круг специалистов и может быть использована научными 
проектными организациями в сфере промышленного производства ИНП. Большое количество 
иллюстраций, примеров и вопросов, обращенных к читателю, позволяет использовать книгу 
также в качестве учебного пособия для студентов и аспирантов машиностроительных, 
транспортных и~других специальностей, а также для самостоятельного изучения. 
{%\looseness=-1

}

Книга 
представляет несомненный интерес для специалистов и студентов в области прикладной 
математики и информатики.
    

}

}
\end{multicols}

%\newpage

\def\stat{authorsrus}
{%\hrule\par
%\vskip 7pt % 7pt
\raggedleft\Large \bf%\baselineskip=3.2ex
О\,Б\ \ А\,В\,Т\,О\,Р\,А\,Х \vskip 17pt
    \hrule
    \par
\vskip 21pt plus 8pt minus 4pt }


\def\tit{\ }

\def\aut{\ }

\def\auf{\ }

\def\leftkol{\ } % ENGLISH ABSTRACTS}

\def\rightkol{ОБ АВТОРАХ} %ENGLISH ABSTRACTS}

\titele{\tit}{\aut}{\auf}{\leftkol}{\rightkol}
      
            \label{st\stat}



\vspace*{24pt}

\begin{multicols}{2}




\noindent
\textbf{Архипов Олег Петрович} (р.\ 1948)~---
кандидат технических наук, директор Орловского филиала Института проб\-лем информатики
Российской академии наук
%302025, г.Орел, Московское шоссе, д.137

\vspace*{3pt}

\noindent
\textbf{Бирюкова Татьяна Константиновна} (р.\ 1968)~---
кандидат фи\-зи\-ко-ма\-те\-ма\-ти\-че\-ских наук, старший научный сотрудник Института проб\-лем информатики
Российской академии наук

\vspace*{3pt}

\noindent 
\textbf{Бобков  Сергей Геннадьевич} (р.\ 1955)~---
доктор технических наук,  заведующий отделением На\-уч\-но-ис\-сле\-до\-ва\-тель\-ско\-го 
института системных исследований Российской академии наук
%117218, Москва, Нахимовский просп., 36, к.1 

\vspace*{3pt}

\noindent \textbf{Васильев Николай Семенович} (р.\ 1952)~--- доктор 
фи\-зи\-ко-ма\-те\-ма\-ти\-че\-ских наук, профессор, 
МГТУ им.\ Н.\,Э.~Баумана 
%, Москва 105005, 2-я Бауманская ул., д.~5,

\vspace*{3pt}

\noindent
\textbf{Гершкович Максим Михайлович} (р.\ 1968)~---
старший научный сотрудник Института проб\-лем информатики
Российской академии наук

\vspace*{3pt}

\noindent 
\textbf{Дьяченко Юрий Георгиевич} (р.\ 1958)~--- кандидат технических наук, 
старший научный сотрудник Института проб\-лем информатики
Российской академии наук

\vspace*{3pt}

\noindent 
\textbf{Ерошенко Александр Андреевич} (р.\ 1989)~--- аспирант кафедры 
математической статистики факультета вычисли\-тельной математики и кибернетики 
Московского государственного университета им.\ М.\,В.~Ломоносова
%119991, Москва ГСП-1, Ленинские горы, д.\ 1, стр. 52

\vspace*{3pt}
 
\noindent 
\textbf{Захаров Виктор Николаевич} (р.\ 1948)~--- 
доктор технических наук, доцент, ученый секретарь Института проб\-лем информатики
Российской академии наук

\vspace*{3pt}

\noindent
\textbf{Зейфман Александр Израилевич} (р.\ 1954)~---
доктор фи\-зи\-ко-ма\-те\-ма\-ти\-че\-ских наук, профессор, 
заведующий кафедрой Вологодского государственного университета; 
старший научный сотрудник Института проб\-лем информатики
Российской академии наук; главный научный сотрудник ИСЭРТ Российской академии наук

\vspace*{3pt}

\noindent
\textbf{Зыкин Сергей Владимирович} (р.\ 1959)~--- 
доктор технических наук, профессор, заведующий лабораторией Института математики 
им.\ С.\,Л.~Соболева Сибирского отделения Российской академии наук, Новосибирск 
%630090, пр.\ ак.\ Коптюга, 4 

\vspace*{4pt}

\noindent
\textbf{Киреев Владимир Иванович} (р.\ 1938)~---
доктор фи\-зи\-ко-ма\-те\-ма\-ти\-че\-ских наук, профессор Московского 
государственного горного университета
%Адрес: Россия, 119991, г. Москва, Ленинский проспект, д. 6

%\columnbreak

\vspace*{4pt}

\noindent
\textbf{Козеренко Елена Борисовна} (р.\ 1959)~---
кандидат филологических наук, заведующая лабораторией Института проб\-лем информатики
Российской академии наук

\vspace*{4pt}

\noindent
\textbf{Королев Виктор Юрьевич} (р.\ 1954)~--- доктор
фи\-зи\-ко-ма\-те\-ма\-ти\-че\-ских наук, профессор кафедры математической 
статистики факультета вычисли\-тельной математики и кибернетики 
Московского государственного университета; 
ведущий научный сотрудник Института проб\-лем информатики
Российской академии наук

\vspace*{4pt}

\noindent
\textbf{Коротышева Анна Владимировна} (р.\ 1988)~---
старший преподаватель Вологодского государственного университета

\vspace*{4pt}

\noindent 
\textbf{Кун Де Турк} (р.\ 1981)~--- научный сотрудник 
исследовательской группы SMACS факультета телекоммуникаций и обработки информации
Университета Гента, Бельгия
%В-9000 Гент, Бельгия

\vspace*{4pt}

\noindent
\textbf{Лупенцов Олег Сергеевич} (р.\ 1986)~---
аспирант Омского государственного института сервиса
%Омск 644043, ул.\ Певцова 13

\vspace*{4pt}

\noindent
\textbf{Лучко Олег Николаевич} (р.\ 1961)~---
кандидат педагогических наук, профессор, заведующий кафедрой 
Омского государственного института сервиса
%Омск 644043, ул.\ Певцова 13

\vspace*{4pt}

\noindent
\textbf{Малашенко Юрий Евгеньевич} (р.\ 1946)~---
доктор фи\-зи\-ко-ма\-те\-ма\-ти\-че\-ских наук, заведующий сектором 
Вычислительного центра им.\ А.\,А.~Дородницына Российской академии наук
%Адрес: 119333, Москва, ул. Вавилова, 40,

\vspace*{4pt}

\noindent
\textbf{Маньяков Юрий Анатольевич} (р.\ 1984)~---
кандидат технических наук, научный сотрудник Орловского филиала Института проб\-лем информатики
Российской академии наук
%302025, г.Орел, Московское шоссе, д.137

\vspace*{4pt}

\noindent
\textbf{Маренко Валентина Афанасьевна} (р.\ 1951)~---
кандидат технических наук, доцент, старший научный сотрудник 
Института математики им.\ С.\,Л.~Соболева Сибирского отделения Российской академии наук
%Новосибирск 630090, пр. ак. Коптюга, 4 

\vspace*{3pt}

\noindent 
\textbf{Морозов Евсей Викторович} (р.\ 1947)~--- доктор 
фи\-зи\-ко-ма\-те\-ма\-ти\-че\-ских, профессор, ведущий научный сотрудник 
Института прикладных математических исследований Карельского научного центра Российской
академии наук; 
%%185910 Россия, Республика Карелия, г.\ Петрозаводск, ул.\ Пушкинская, 11
профессор Петрозаводского государственного университета, Петрозаводск
%185910 Россия, Республика Карелия, г.\ Петрозаводск, пр.\ Ленина, 33

%\pagebreak

\vspace*{3pt}

\noindent
\textbf{Назарова Ирина Александровна} (р.\ 1966)~---
кандидат фи\-зи\-ко-ма\-те\-ма\-ти\-че\-ских наук, 
научный сотрудник Вычислительного центра им.\ А.\,А.~Дородницына Российской академии наук 
%Адрес: 119333, Москва, ул. Вавилова, 40

\vspace*{3pt}

\noindent
\textbf{Павлов Игорь Валерианович} (р.\ 1945)~--- 
доктор фи\-зи\-ко-ма\-те\-ма\-ти\-че\-ских наук, профессор МГТУ им.\ Н.\,Э.~Баумана 
%Москва 105005, 2-я Бауманская ул., д.~5 

%\pagebreak

\vspace*{3pt}

\noindent 
\textbf{Потахина Любовь Викторовна} (р.\ 1989)~--- аспирантка
Института прикладных математических исследований Карельского научного центра
Российской академии наук; 
%%185910 Россия, Республика Карелия, г.\ Петрозаводск, ул.\ Пушкинская, 11
инженер Петрозаводского государственного университета, Петрозаводск
%185910 Россия, Республика Карелия, г.\ Петрозаводск, пр.\ Ленина, 33

\vspace*{3pt}

\noindent 
\textbf{Рождественский Юрий Владимирович} (р.\ 1952)~--- 
кандидат технических наук, заведующий сектором Института проб\-лем информатики
Российской академии наук

\vspace*{3pt}

\noindent 
\textbf{Синицын Игорь Николаевич} (р.\ 1940)~--- доктор технических наук,
профессор, заслуженный деятель\linebreak\vspace*{-12pt}

\columnbreak

\noindent
 науки РФ, заведующий отделом Института проб\-лем информатики
Российской академии наук

\vspace*{7pt}


\noindent
\textbf{Сиротинин Денис Олегович} (р.\ 1984)~---
кандидат технических наук, научный сотрудник Орловского филиала Института проб\-лем информатики
Российской академии наук
%302025, г.Орел, Московское шоссе, д.137

\vspace*{7pt}

%\columnbreak

\noindent 
\textbf{Соколов  Игорь Анатольевич} (р.\ 1954)~--- академик (действительный член) Российской 
академии наук, доктор технических наук, директор Института проб\-лем информатики
Российской академии наук

\vspace*{7pt}

\noindent
\textbf{Степченков Юрий Афанасьевич} (р.\ 1951)~---
кандидат технических наук, заведующий отделом Института проб\-лем информатики
Российской академии наук

\vspace*{7pt}

\noindent
\textbf{Сурков Алексей Викторович} (р.\ 1978)~--- 
старший научный сотрудник На\-уч\-но-ис\-сле\-до\-ва\-тель\-ско\-го 
института системных исследований Российской академии наук
%117218, Москва, Нахимовский просп., 36, к.1 

\vspace*{7pt}

\noindent 
\textbf{Шестаков Олег Владимирович} (р.\ 1976)~--- доктор 
фи\-зи\-ко-ма\-те\-ма\-ти\-че\-ских, доцент кафедры математической статистики 
факультета вычисли\-тельной математики и кибернетики Московского 
государственного университета им.\ М.\,В.~Ломоносова; 
%119991, Москва ГСП-1, Ленинские горы, д.\ 1, стр. 52
старший научный сотрудник Института проб\-лем информатики
Российской академии наук
%, Москва 119333, ул. Вавилова, д.~44, корп.~2

\vspace*{7pt}

\noindent 
\textbf{Шоргин Сергей Яковлевич} (р.\ 1952.)~--- доктор
фи\-зи\-ко-ма\-те\-ма\-ти\-че\-ских наук, профессор, заместитель директора Института 
проб\-лем информатики Российской академии наук





%%%%%%%%%%%%%%%%%%%%%%%%%%%%%%%%%%%%%%%%%%%%%%%%%%%%%%%%%%%%%%%%%%%%%%%%%%%%%%%




%\def\rightkol{ОБ АВТОРАХ}
%\def\leftkol{ОБ АВТОРАХ}

 \label{end\stat}





%\def\leftfootline{\small{\textbf{\thepage}
%\hfill ИНФОРМАТИКА И ЕЁ ПРИМЕНЕНИЯ\ \ \ том~7\ \ \ выпуск~1\ \ \ 2013}
%}%
% \def\rightfootline{\small{ИНФОРМАТИКА И ЕЁ ПРИМЕНЕНИЯ\ \ \ том~7\ \ \ выпуск~1\ \ \ 2013
%\hfill \textbf{\thepage}}}


%\thispagestyle{myheadings}



\end{multicols}

\newpage  

%\def\stat{cont}
{%\hrule\par
%\vskip 7pt % 7pt
\raggedleft\Large \bf%\baselineskip=3.2ex
А\,В\,Т\,О\,Р\,С\,К\,И\,Й\ \ У\,К\,А\,З\,А\,Т\,Е\,Л\,Ь\ \ З\,А\ \ 2\,0\,0\,7 г. \vskip 17pt
    \hrule
    \par
\vskip 21pt plus 6pt minus 3pt }

\label{st\stat}

\def\tit{\ }

\def\aut{\ }
\def\auf{\ }

\def\leftkol{\ } % ENGLISH ABSTRACTS}

\def\rightkol{\ } %ENGLISH ABSTRACTS}

\titele{\tit}{\aut}{\auf}{\leftkol}{\rightkol}


\contentsline {chapter}{\ }{Выпуск \quad Стр.} 
\contentsline {section}{\textbf{Батракова Д.\,А., Королев В.\,Ю., Шоргин С.\,Я.}\ \ Новый метод вероятностно-ста\-ти\-сти\-че\-ско\-го анализа информационных потоков в\nobreakspace {}телекоммуникационных сетях}{\qquad 1 \qquad 40} 
\contentsline {section}{\textbf{Борисов А.\,В.}\ \ Байесовское оценивание в системах наблюдения с\nobreakspace {}марковскими скачкообразными процессами: игровой подход}{\qquad 2 \qquad 65}
\contentsline {section}{\textbf{Босов А.\,В., Иванов А.\,В.}\ \ Программная инфраструктура информационного Web-пор\-тала}{\qquad 2 \qquad 50}
\contentsline {section}{\textbf{Захаров В.\,Н., Калиниченко Л.\,А., Соколов И.\,А., Ступников С.\,А.}\ \ Конструирование канонических информационных моделей для интегрированных информационных систем}{\qquad 2 \qquad 15}
\contentsline {section}{\textbf{Захаров В.\,Н., Козмидиади В.\,А.}\ \ Средства обеспечения отказоустойчивости при\-ло\-жений}{\qquad 1 \qquad 14} 
\contentsline {section}{\textbf{Иванов А.\,В.}\ \ см. Босов А.\,В.\hfill\hfill\hfill\hfill\hfill\hfill\hfill\hfill\hfill\hfill\hfill\hfill\hfill\hfill\hfill\hfill\hfill\hfill\hfill\hfill\hfill\hfill\hfill\hfill\hfill\hfill\hfill\hfill\hfill\hfill\hfill\hfill\hfill\hfill\hfill}{\ }
\contentsline {section}{\textbf{Ильин В.\,Д., Соколов И.\,А.}\ \ Символьная модель системы знаний информатики в\nobreakspace {}че\-ло\-ве\-ко-автоматной среде}{\qquad 1 \qquad 66} 
\contentsline {section}{\textbf{Калиниченко Л.\,А.}\ \ см. Захаров В.\,Н.\hfill\hfill\hfill\hfill\hfill\hfill\hfill\hfill\hfill\hfill\hfill\hfill\hfill\hfill\hfill\hfill\hfill\hfill\hfill\hfill\hfill\hfill\hfill\hfill\hfill\hfill\hfill\hfill\hfill\hfill\hfill\hfill\hfill\hfill\hfill}{\ }
\contentsline {section}{\textbf{Козеренко Е.\,Б.}\ \ Лингвистическое моделирование для систем машинного перевода и обработки знаний}{\qquad 1 \qquad 54} 
\contentsline {section}{\textbf{Козмидиади В.\,А.}\ \ см. Захаров В.\,Н.\hfill\hfill\hfill\hfill\hfill\hfill\hfill\hfill\hfill\hfill\hfill\hfill\hfill\hfill\hfill\hfill\hfill\hfill\hfill\hfill\hfill\hfill\hfill\hfill\hfill\hfill\hfill\hfill\hfill\hfill\hfill\hfill\hfill\hfill\hfill }{\ } 
\contentsline {section}{\textbf{Королев В.\,Ю.}\ \ см. Батракова Д.\,А.\hfill\hfill\hfill\hfill\hfill\hfill\hfill\hfill\hfill\hfill\hfill\hfill\hfill\hfill\hfill\hfill\hfill\hfill\hfill\hfill\hfill\hfill\hfill\hfill\hfill\hfill\hfill\hfill\hfill\hfill\hfill\hfill\hfill\hfill\hfill}{\ } 
\contentsline {section}{\textbf{Кудрявцев А.\,А., Шоргин С.\,Я.}\ \ Байесовский подход к\nobreakspace {}анализу систем массового обслуживания и\nobreakspace {}показателей надежности}{\qquad 2 \qquad 76}
\contentsline {section}{\textbf{Печинкин А.\,В., Соколов И.\,А., Чаплыгин В.\,В.}\ \ Многолинейная система массового обслуживания с конечным накопителем и ненадежными приборами}{\qquad 1 \qquad 27} 
\contentsline {section}{\textbf{Печинкин А.\,В., Соколов И.\,А., Чаплыгин В.\,В.}\ \ Стационарные характеристики многолинейной\nobreakspace {}системы массового обслуживания с\nobreakspace {}одновременными отказами приборов}{\qquad 2 \qquad 39}
\contentsline {section}{\textbf{Синицын И.\,Н.}\ \ Корреляционные методы построения аналитических информационных моделей флуктуаций полюса Земли по априорным данным}{\qquad 2 \qquad \hphantom{9}2}
\contentsline {section}{\textbf{Синицын И.\,Н.}\ \ Развитие теории фильтров Пугачева для оперативной обработки информации в стохастических системах}{{\qquad 1 \qquad \hphantom{9}3}} 
\contentsline {section}{\textbf{Соколов И.\,А.}\ \ см. Захаров В.\,Н.\hfill\hfill\hfill\hfill\hfill\hfill\hfill\hfill\hfill\hfill\hfill\hfill\hfill\hfill\hfill\hfill\hfill\hfill\hfill\hfill\hfill\hfill\hfill\hfill\hfill\hfill\hfill\hfill\hfill\hfill\hfill\hfill\hfill\hfill\hfill}{\ }
\contentsline {section}{\textbf{Соколов И.\,А.}\ \ см. Ильин В.\,Д.\hfill\hfill\hfill\hfill\hfill\hfill\hfill\hfill\hfill\hfill\hfill\hfill\hfill\hfill\hfill\hfill\hfill\hfill\hfill\hfill\hfill\hfill\hfill\hfill\hfill\hfill\hfill\hfill\hfill\hfill\hfill\hfill\hfill\hfill\hfill}{\ } 
\contentsline {section}{\textbf{Соколов И.\,А.}\ \ см. Печинкин А.\,В.\hfill\hfill\hfill\hfill\hfill\hfill\hfill\hfill\hfill\hfill\hfill\hfill\hfill\hfill\hfill\hfill\hfill\hfill\hfill\hfill\hfill\hfill\hfill\hfill\hfill\hfill\hfill\hfill\hfill\hfill\hfill\hfill\hfill\hfill\hfill}{\ } 
\contentsline {section}{\textbf{Соколов И.\,А.}\ \ см. Печинкин А.\,В.\hfill\hfill\hfill\hfill\hfill\hfill\hfill\hfill\hfill\hfill\hfill\hfill\hfill\hfill\hfill\hfill\hfill\hfill\hfill\hfill\hfill\hfill\hfill\hfill\hfill\hfill\hfill\hfill\hfill\hfill\hfill\hfill\hfill\hfill\hfill}{\ }
\contentsline {section}{\textbf{Ступников С.\,А.}\ \ см. Захаров В.\,Н.\hfill\hfill\hfill\hfill\hfill\hfill\hfill\hfill\hfill\hfill\hfill\hfill\hfill\hfill\hfill\hfill\hfill\hfill\hfill\hfill\hfill\hfill\hfill\hfill\hfill\hfill\hfill\hfill\hfill\hfill\hfill\hfill\hfill\hfill\hfill}{\ }
\contentsline {section}{\textbf{Чаплыгин В.\,В.}\ \ см. Печинкин А.\,В.\hfill\hfill\hfill\hfill\hfill\hfill\hfill\hfill\hfill\hfill\hfill\hfill\hfill\hfill\hfill\hfill\hfill\hfill\hfill\hfill\hfill\hfill\hfill\hfill\hfill\hfill\hfill\hfill\hfill\hfill\hfill\hfill\hfill\hfill\hfill}{\ } 
\contentsline {section}{\textbf{Чаплыгин В.\,В.}\ \ см. Печинкин А.\,В.\hfill\hfill\hfill\hfill\hfill\hfill\hfill\hfill\hfill\hfill\hfill\hfill\hfill\hfill\hfill\hfill\hfill\hfill\hfill\hfill\hfill\hfill\hfill\hfill\hfill\hfill\hfill\hfill\hfill\hfill\hfill\hfill\hfill\hfill\hfill}{\ }
\contentsline {section}{\textbf{Шоргин С.\,Я.}\ \ см. Батракова Д.\,А.\hfill\hfill\hfill\hfill\hfill\hfill\hfill\hfill\hfill\hfill\hfill\hfill\hfill\hfill\hfill\hfill\hfill\hfill\hfill\hfill\hfill\hfill\hfill\hfill\hfill\hfill\hfill\hfill\hfill\hfill\hfill\hfill\hfill\hfill\hfill}{\ } 
\contentsline {section}{\textbf{Шоргин С.\,Я.}\ \ см. Кудрявцев А.\,А.\hfill\hfill\hfill\hfill\hfill\hfill\hfill\hfill\hfill\hfill\hfill\hfill\hfill\hfill\hfill\hfill\hfill\hfill\hfill\hfill\hfill\hfill\hfill\hfill\hfill\hfill\hfill\hfill\hfill\hfill\hfill\hfill\hfill\hfill\hfill}{\ }
%\thispagestyle{myheadings}
\def\leftfootline{\small{\textbf{\thepage}
\hfill ИНФОРМАТИКА И ЕЁ ПРИМЕНЕНИЯ\ \ \ том~1\ \ \ выпуск~2\ \ \ 2007}
}%
 \def\rightfootline{\small{ИНФОРМАТИКА И ЕЁ ПРИМЕНЕНИЯ\ \ \ том~1\ \ \ выпуск~2\ \ \ 2007
 \hfill \textbf{\thepage}}}
 \label{end\stat} 
                     
%\def\stat{cont-e}
{%\hrule\par
%\vskip 7pt % 7pt
\raggedleft\Large \bf%\baselineskip=3.2ex
2\,0\,0\,7\ \ A\,U\,T\,H\,O\,R\ \ I\,N\,D\,E\,X \vskip 17pt
    \hrule
    \par
\vskip 21pt plus 6pt minus 3pt }

\label{st\stat}

\def\tit{\ }

\def\aut{\ }
\def\auf{\ }

\def\leftkol{\ } % ENGLISH ABSTRACTS}

\def\rightkol{\ } %ENGLISH ABSTRACTS}

\titele{\tit}{\aut}{\auf}{\leftkol}{\rightkol}


\contentsline {chapter}{\ }{Issue \quad Page} 
\contentsline {subsection}{\textbf{Batrakova D.\,A., Korolev V.\,Yu., Shorgin S.\,Ya.}\ \ A New Method for the Probabilistic and Statistical Analysis of Information Flows in Telecommunication Networks}{\qquad 1 \qquad 40} 
\contentsline {subsection}{\textbf{Borisov A.\,V.}\ \ Bayesian Estimation in\nobreakspace {}Observation Systems with\nobreakspace {}Markov Jump Processes: Game-Theoretic Approach}{\qquad 2 \qquad 65} 
\contentsline {subsection}{\textbf{Bosov A.\,V., Ivanov A.\,V.}\ \ Linguistic Simulation for Machine Translation and Knowledge Management Systems}{\qquad 2 \qquad 50} 
\contentsline {subsection}{\textbf{Chaplygin V.\,V.} see Pechinkin A.\,V.\hfill\hfill\hfill\hfill\hfill\hfill\hfill\hfill\hfill\hfill\hfill\hfill\hfill\hfill\hfill\hfill\hfill\hfill\hfill\hfill\hfill\hfill\hfill\hfill\hfill\hfill\hfill\hfill\hfill\hfill\hfill\hfill\hfill\hfill\hfill}{\ }
\contentsline {subsection}{\textbf{Chaplygin V.\,V.} see Pechinkin A.\,V.\hfill\hfill\hfill\hfill\hfill\hfill\hfill\hfill\hfill\hfill\hfill\hfill\hfill\hfill\hfill\hfill\hfill\hfill\hfill\hfill\hfill\hfill\hfill\hfill\hfill\hfill\hfill\hfill\hfill\hfill\hfill\hfill\hfill\hfill\hfill}{\ }
\contentsline {subsection}{\textbf{Ilyin V.\,D., Sokolov I.\,A.}\ \ The Symbol Model of Informatics Knowledge System in Human-Automaton Environment}{\qquad 1 \qquad 66} 
\contentsline {subsection}{\textbf{Ivanov A.\,V.} see Bosov A.\,V.\hfill\hfill\hfill\hfill\hfill\hfill\hfill\hfill\hfill\hfill\hfill\hfill\hfill\hfill\hfill\hfill\hfill\hfill\hfill\hfill\hfill\hfill\hfill\hfill\hfill\hfill\hfill\hfill\hfill\hfill\hfill\hfill\hfill\hfill\hfill}{\ }
\contentsline {subsection}{\textbf{Kalinichenko L.\,A.} see Zakharov V.\,N.\hfill\hfill\hfill\hfill\hfill\hfill\hfill\hfill\hfill\hfill\hfill\hfill\hfill\hfill\hfill\hfill\hfill\hfill\hfill\hfill\hfill\hfill\hfill\hfill\hfill\hfill\hfill\hfill\hfill\hfill\hfill\hfill\hfill\hfill\hfill}{\ }
\contentsline {subsection}{\textbf{Korolev V.\,Yu.} see Batrakova D.\,A.\hfill\hfill\hfill\hfill\hfill\hfill\hfill\hfill\hfill\hfill\hfill\hfill\hfill\hfill\hfill\hfill\hfill\hfill\hfill\hfill\hfill\hfill\hfill\hfill\hfill\hfill\hfill\hfill\hfill\hfill\hfill\hfill\hfill\hfill\hfill}{\ }
\contentsline {subsection}{\textbf{Kozerenko E.\,B.}\ \ Linguistic Simulation for Machine Translation and Knowledge Management Systems}{\qquad 1 \qquad 54} 
\contentsline {subsection}{\textbf{Kozmidiady V.\,A.} see Zakharov V.\,N.\hfill\hfill\hfill\hfill\hfill\hfill\hfill\hfill\hfill\hfill\hfill\hfill\hfill\hfill\hfill\hfill\hfill\hfill\hfill\hfill\hfill\hfill\hfill\hfill\hfill\hfill\hfill\hfill\hfill\hfill\hfill\hfill\hfill\hfill\hfill}{\ }
\contentsline {subsection}{\textbf{Kudryavtsev A.\,A., Shorgin S.\,Ya.}\ \ Bayesian Approach to Queueing Systems and Reliability Characteristics}{\qquad 2 \qquad 76} 
\contentsline {subsection}{\textbf{Pechinkin A.\,V., Sokolov I.\,A., Chaplygin V.\,V.}\ \ Multichannel Queuing System with Finite Buffer and Unreliable Servers}{\qquad 1 \qquad 27} 
\contentsline {subsection}{\textbf{Pechinkin A.\,V., Sokolov I.\,A., Chaplygin V.\,V.}\ \ Stationary Characteristics of a Multichannel Queueing System with\nobreakspace {}Simultaneous Refusals of Servers}{\qquad 2 \qquad 39} 
\contentsline {subsection}{\textbf{Shorgin S.\,Ya.} see Batrakova D.\,A.\hfill\hfill\hfill\hfill\hfill\hfill\hfill\hfill\hfill\hfill\hfill\hfill\hfill\hfill\hfill\hfill\hfill\hfill\hfill\hfill\hfill\hfill\hfill\hfill\hfill\hfill\hfill\hfill\hfill\hfill\hfill\hfill\hfill\hfill\hfill}{\ }
\contentsline {subsection}{\textbf{Shorgin S.\,Ya.} see Kudryavtsev A.\,A.\hfill\hfill\hfill\hfill\hfill\hfill\hfill\hfill\hfill\hfill\hfill\hfill\hfill\hfill\hfill\hfill\hfill\hfill\hfill\hfill\hfill\hfill\hfill\hfill\hfill\hfill\hfill\hfill\hfill\hfill\hfill\hfill\hfill\hfill\hfill}{\ }
\contentsline {subsection}{\textbf{Sinitsyn I.\,N.}\ \ Correlational Methods for Analytical Informational Models of the Earth Pole Fluctuations Design Based on a priori Data}{\qquad 2 \qquad \hphantom{9}2}
\contentsline {subsection}{\textbf{Sinitsyn I.\,N.}\ \ Development of Pugachev Filtering for Stochastic Systems}{\qquad 1 \qquad \hphantom{9}3}
\contentsline {subsection}{\textbf{Sokolov I.\,A.} see Ilyin V.\,D.\hfill\hfill\hfill\hfill\hfill\hfill\hfill\hfill\hfill\hfill\hfill\hfill\hfill\hfill\hfill\hfill\hfill\hfill\hfill\hfill\hfill\hfill\hfill\hfill\hfill\hfill\hfill\hfill\hfill\hfill\hfill\hfill\hfill\hfill\hfill}{\ }
\contentsline {subsection}{\textbf{Sokolov I.\,A.} see Pechinkin A.\,V.\hfill\hfill\hfill\hfill\hfill\hfill\hfill\hfill\hfill\hfill\hfill\hfill\hfill\hfill\hfill\hfill\hfill\hfill\hfill\hfill\hfill\hfill\hfill\hfill\hfill\hfill\hfill\hfill\hfill\hfill\hfill\hfill\hfill\hfill\hfill}{\ }
\contentsline {subsection}{\textbf{Sokolov I.\,A.} see Pechinkin A.\,V.\hfill\hfill\hfill\hfill\hfill\hfill\hfill\hfill\hfill\hfill\hfill\hfill\hfill\hfill\hfill\hfill\hfill\hfill\hfill\hfill\hfill\hfill\hfill\hfill\hfill\hfill\hfill\hfill\hfill\hfill\hfill\hfill\hfill\hfill\hfill}{\ }
\contentsline {subsection}{\textbf{Sokolov I.\,A.} see Zakharov V.\,N.\hfill\hfill\hfill\hfill\hfill\hfill\hfill\hfill\hfill\hfill\hfill\hfill\hfill\hfill\hfill\hfill\hfill\hfill\hfill\hfill\hfill\hfill\hfill\hfill\hfill\hfill\hfill\hfill\hfill\hfill\hfill\hfill\hfill\hfill\hfill}{\ }
\contentsline {subsection}{\textbf{Stupnikov S.\,A.} see Zakharov V.\,N.\hfill\hfill\hfill\hfill\hfill\hfill\hfill\hfill\hfill\hfill\hfill\hfill\hfill\hfill\hfill\hfill\hfill\hfill\hfill\hfill\hfill\hfill\hfill\hfill\hfill\hfill\hfill\hfill\hfill\hfill\hfill\hfill\hfill\hfill\hfill}{\ }
\contentsline {subsection}{\textbf{Zakharov V.\,N., Kalinichenko L.\,A., Sokolov I.\,A., Stupnikov S.\,A.}\ \ Development of Canonical Information Models for Integrated Information Systems}{\qquad 2 \qquad 15} 
\contentsline {subsection}{\textbf{Zakharov V.\,N., Kozmidiady V.\,A.}\ \ Means Providing Applications Fault Tolerance}{\qquad 1 \qquad 14} 
\def\leftfootline{\small{\textbf{\thepage}
\hfill ИНФОРМАТИКА И ЕЁ ПРИМЕНЕНИЯ\ \ \ том~1\ \ \ выпуск~2\ \ \ 2007}
}%
 \def\rightfootline{\small{ИНФОРМАТИКА И ЕЁ ПРИМЕНЕНИЯ\ \ \ том~1\ \ \ выпуск~2\ \ \ 2007
 \hfill \textbf{\thepage}}}
 \label{end\stat} 


%\end{document}

%
\def\stat{rekl}
%\label{preobr}

%\def\tit{АКАДЕМИК ПУГАЧЁВ  ВЛАДИМИР СЕМЁНОВИЧ\\
%25.03.1911--25.03.1998}


%   \vspace*{-48pt}
%   \begin{center}\LARGE
%Академик Пугачёв  Владимир Семёнович\\ (25.03.1911--25.03.1998)
%   \end{center}

   %\vspace*{2.5mm}

   \begin{center}

{\prgsh\LARGE
ЮБИЛЕИ}

\end{center}
%\hrule

\vspace*{6pt}


   \vspace*{8mm}

   \thispagestyle{empty}


%\def\stat{emel}


\section*{К 70-летию заместителя директора ИПИ РАН,\\ члена редколлегии журнала
<<Информатика и её применения>>\\ доктора технических наук В.\,И.~Будзко}

\vspace*{18pt}




          \begin{multicols}{2}

%            \label{st\stat}

\begin{center}
\vspace*{1pt}
\mbox{%
\epsfxsize=78mm
\epsfbox{bud-1.eps}
}
\end{center}

\vspace*{12pt}

      14 августа 2014~г.\ исполнилось 70~лет за\-мес\-ти\-те\-лю директора ИПИ РАН по
научной работе доктору технических наук Владимиру Игоревичу Будзко.

      Владимир Игоревич Будзко родился в г.~Москве. Высшее образование получил на факультете
элект\-рон\-но-вы\-чис\-ли\-тель\-ных устройств в Московском
ин\-же\-нер\-но-фи\-зи\-че\-ском институте
(МИФИ), который он окончил в 1968~г., после чего был на\-прав\-лен для прохождения
службы в одну из войс\-ко\-вых частей, где прошел путь от инженера до первого заместителя
командира войсковой части.

      С приходом В.\,И.~Будзко в ИПИ РАН (2001~г.)\ в институте
сформировалось новое научное на\-прав\-ле\-ние теоретических исследований~--- <<Постро\-ение
ин\-фор\-ма\-ци\-он\-но-те\-ле\-ком\-му\-ни\-ка\-ци\-он\-ных\linebreak сис\-тем
высокой до\-ступ\-ности>>. В~рамках этого
направления выполнен широкий круг фундаментальных исследований по поиску подходов и
определению принципов построения средств обеспечения доступности, конфиденциальности
и целостности современных крупномасштабных
ин\-фор\-ма\-ци\-он\-но-те\-ле\-ком\-му\-ни\-ка\-ци\-он\-ных
сис\-тем (ИТС). Разработаны основные сис\-тем\-но-тех\-ни\-че\-ские принципы и базовые
архитектурные решения построения перспективных для условий России ИТС с
централизованной обработкой и хранением информации, сочетающих в себе свойства
высокой доступности, отказо- и катастрофоустойчивости, информационной защищенности.
Определены принципы, методы и математические основы рационального построения и
оптимизации средств восстановления функционирования центров обработки данных (ЦОД)
после возникновения отказов и катастроф, передачи и хранения данных, обеспечения
информационной безопасности при достижении минимальной совокупной стоимости
владения такими системами. Результаты нашли практическое воплощение при реализации
проектов в интересах ряда отечественных государственных и негосударственных
организаций, таких как Банк России (БР), Внешторгбанк, ОАО <<ГМК <<Норильский Никель>>,
<<Газпром>>, Минэкономразвития России, Правительство Москвы, а также ряд силовых
ведомств.

      Под руководством В.\,И.~Будзко начиная с 2001~г.\ выполнен комплекс
      на\-уч\-но-ис\-сле\-до\-ва\-тель\-ских и
      опыт\-но-кон\-ст\-рук\-тор\-ских работ (свыше 100~проектов),
направленных на развитие электронной информационной технологии БР.
Разработаны концепции развития ИТС БР сначала до 2008~г., а затем до 2013~г., которые
были приняты в качестве основы проведения технической политики. За реализацию проекта
<<Катастрофоустойчивая тер\-ри\-то\-ри\-аль\-но-рас\-пре\-де\-лен\-ная
      ин\-фор\-ма\-ци\-он\-но-те\-ле\-ком\-му\-ни\-ка\-ци\-он\-ная сис\-те\-ма централизованной
обработки банковской информации>> В.\,И.~Будзко удостоен Премии Правительства РФ в
области науки и техники за 2010~г.

      В.\,И.~Будзко возглавлял и возглавляет работы по ряду других прикладных проектов,
связанных с созданием, совершенствованием и развитием крупномасштабных ИТС.

      В.\,И.~Будзко~--- генерал-майор, доктор технических наук, член-кор\-рес\-пон\-дент
Академии криптографии РФ, известный ученый в области информатики и применения
информационных технологий при построении территориально распределенных ИТС
различного назначения. Является автором свыше 250~научных работ, опубликованных в
на\-уч\-но-тех\-ни\-че\-ских и специальных изданиях.

    \thispagestyle{empty}

      В.\,И.~Будзко уделяет большое внимание подготовке научных кадров. Под его
руководством защищено 6~диссертаций на соискание ученой степени кандидата
технических наук. Свыше 30~лет он читает лекции в ИКСИ Академии ФСБ, профессор
кафедры НИЯУ МИФИ. Является членом двух диссертационных советов, главным
редактором журнала <<Системы высокой доступности>> и членом редколлегии журнала
<<Информатика и её применения>>.

      \bigskip

      Редакционный совет и Редакционная коллегия журнала <<Информатика и её
применения>> сердечно поздравляют Владимира Игоревича Будзко с 70-ле\-ти\-ем и желают
крепкого здоровья и новых научных достижений.

\end{multicols}

%%Информатика и её применения
%Том 14 Выпуск 1-4 Год 2020

\def\stat{cont}
{%\hrule\par
%\vskip 7pt % 7pt
\raggedleft\Large \bf%\baselineskip=3.2ex
А\,В\,Т\,О\,Р\,С\,К\,И\,Й\ \ У\,К\,А\,З\,А\,Т\,Е\,Л\,Ь\ \ З\,А\ \ 2\,0\,2\,0 г. \vskip 17pt
 \hrule
 \par
\vskip 21pt plus 6pt minus 3pt }

\label{st\stat}

\def\tit{\ }

\def\aut{\ }
\def\auf{\ }

\def\leftkol{\ } % ENGLISH ABSTRACTS}

\def\rightkol{\ } %АВТОРСКИЙ УКАЗАТЕЛЬ ЗА 2020 г.} %ENGLISH ABSTRACTS}

\titele{\tit}{\aut}{\auf}{\leftkol}{\rightkol}
\addcontentsline{toc}{subsection}{\textrm\textbf Авторский указатель за 2020 г.}

\vspace*{-24pt}

\noindent
{\tabcolsep=3pt
\begin{tabular}{p{397pt}cc}
&\textbf{Вып.} & \textbf{Стр.}\\[6pt]
\Avtors{Абгарян~К.\,К., Гаврилов~Е.\,С.} Интеграционная платформа для многомасштабного моде-\linebreak
\\[-12pt]
\hspace*{23pt}лирования нейроморфных систем&2&104--110\\
\Avtors{Абгарян~К.\,К., Колбин~И.\,С.} Применение многомасштабного подхода и методов анализа\linebreak
\\[-12pt]
\hspace*{23pt}данных для моделирования теплопроводности в слоистых структурах&4&91--99\\
\Avtors{Агаларов~Я.\,М.} Оптимизация емкости основного накопителя в системе массового\linebreak
\\[-12pt]
\hspace*{23pt}обслуживания типа $G/M/1/K$ с дополнительным накопителем&2&72--79\\
\Avtors{Агасандян~Г.\,А.} Вычислительные аспекты применения CC-VaR на совокупности рынков&3&62--70\\
\Avtors{Агеев~К.\,А., Сопин~Э.\,С., Яркина~Н.\,В., Самуйлов~К.\,Е., Шоргин~С.\,Я.} Анализ механизмов\linebreak
\\[-12pt]
\hspace*{23pt}нарезки сети с учетом гарантий для различных типов трафика&3&\hphantom{1}94--100\\
\Avtors{Адамова~К.\,А.} см.\ Шнурков~П.\,В.&&\\
\Avtors{Базилевский~М.\,П.} Многофакторные модели полносвязной линейной регрессии без\linebreak
\\[-12pt]
\hspace*{23pt}ограничений на соотношения дисперсий ошибок переменных&2&92--97\\
\Avtors{Бахтеев~О.\,Ю.} см.\ Грабовой~А.\,В.&&\\
\Avtors{Беленков~В.\,Г.} см.\ Будзко~В.\,И.&&\\
\Avtors{Бетелин~В.\,Б., Кушниренко~А.\,Г., Леонов~А.\,Г.} Основные понятия программирования\linebreak
\\[-12pt]
\hspace*{23pt}в изложении для дошкольников&3&55--61\\
\Avtors{Бетелин~В.\,Б., Кушниренко~А.\,Г., Семенов~А.\,Л., Сопрунов~С.\,Ф.} О цифровой грамотности\linebreak
\\[-12pt]
\hspace*{23pt}и средах ее формирования&4&100--107\\
\Avtors{Борисов~А.\,В.} Численные схемы фильтрации марковских скачкообразных процессов по\linebreak
\\[-12pt]
\hspace*{23pt}дискретизованным наблюдениям II: случай аддитивных шумов&1&17--23\\
\Avtors{Борисов~А.\,В.} Численные схемы фильтрации марковских скачкообразных процессов по\linebreak
\\[-12pt]
\hspace*{23pt}дискретизованным наблюдениям III: случай мультипликативных шумов&2&10--18\\
\Avtors{Босов~А.\,В.} Управление выходом стохастической дифференциальной системы по квад-\linebreak
\\[-12pt]
\hspace*{23pt}ратичному критерию. V. Случай неполной информации о состоянии&2&19--25\\
\Avtors{Босов~А.\,В., Мартюшова~Я.\,Г., Наумов~А.\,В., Сапунова~А.\,П.} Байесовский подход к~по\-стро\-ению индивидуальной траектории пользователя в~системе дистанционного\linebreak
\\[-12pt]
\hspace*{23pt}обучения&3&86--93\\
\Avtors{Босов~А.\,В., Стефанович~А.\,И.} Управление выходом стохастической дифференциальной\linebreak
\\[-12pt]
\hspace*{23pt}системы по квадратичному критерию. IV. Альтернативное численное решение&1&24--30\\
\Avtors{Брюхов~Д.\,О., Ступников~С.\,А., Ковалёв~Д.\,Ю., Шанин~И.\,А.} Нейрофизиология как\linebreak
\\[-12pt]
\hspace*{23pt}предметная область для решения задач с интенсивным использованием данных&1&40--47\\
\Avtors{Будзко~В.\,И., Ядринцев~В.\,В., Соченков~И.\,В., Королёв~В.\,И., Беленков~В.\,Г.} Об одном подходе
 к формированию в условиях высокой неопределенности марке-\linebreak
\\[-12pt]
\hspace*{23pt}ров конфиденциальности в системах интенсивного использования данных&4&69--76\\
\Avtors{Вайсер~К.\,О.} см.\ Потанин~М.\,С.&&\\
\Avtors{Вохминцев~А.\,В., Мельников~А.\,В., Пачганов~C.\,А.} Метод навигации и составления карты в трехмерном пространстве на основе комбинированного решения вариационной\linebreak
\\[-12pt]
\hspace*{23pt}подзадачи точка--точка ICP для аффинных преобразований&1&101--112\\
\Avtors{Гаврилов~Е.\,С.} см.\ Абгарян~К.\,К.&&\\
\Avtors{Гайдамака~Ю.\,В.} см.\  Москалева~Ф.\,А.&&\\
\Avtors{Голембиовский~Д.\,Ю.} см.\ Данилишин~А.\,Р.&&\\
\Avtors{Голембиовский~Д.\,Ю.} см.\ Данилишин~А.\,Р.&&\\
\Avtors{Гончаров~А.\,А., Зацман~И.\,М., Кружков~М.\,Г.} Эволюция классификаций в надкорпусных\linebreak
\\[-12pt]
\hspace*{23pt}базах данных&4&108--116\\
\Avtors{Гончаров~А.\,В., Стрижов~В.\,В.} Выравнивание декартовых произведений упорядоченных\linebreak
\\[-12pt]
\hspace*{23pt}множеств&1&31--39\\
\end{tabular}
}

\pagebreak

\def\leftkol{АВТОРСКИЙ УКАЗАТЕЛЬ ЗА 2020 г.} % ENGLISH ABSTRACTS}

\def\rightkol{АВТОРСКИЙ УКАЗАТЕЛЬ ЗА 2020 г.} %ENGLISH ABSTRACTS}

%\thispagestyle{myheadings}
\def\leftfootline{\small{\textbf{\thepage}
\hfill ИНФОРМАТИКА И ЕЁ ПРИМЕНЕНИЯ\ \ \ том~14\ \ \ выпуск~4\ \ \ 2020}
}%
 \def\rightfootline{\small{ИНФОРМАТИКА И ЕЁ ПРИМЕНЕНИЯ\ \ \ том~14\ \ \ выпуск~4\ \ \ 2020
 \hfill \textbf{\thepage}}}


\noindent
{\tabcolsep=3pt
\begin{tabular}{p{394pt}cc}
&\textbf{Вып.} & \textbf{Стр.}\\[3pt]
\Avtors{Горшенин~А.\,К., Королев~В.\,Ю.} Аппроксимация распределений размеров частиц лунного\linebreak
\\[-12pt]
\hspace*{23pt}реголита на основе метода статистической симуляции выборок&2&50--57\\
\Avtors{Горшенин~А.\,К., Королев~В.\,Ю., Щербинина~А.\,А.} Статистическое оценивание распределений случайных коэффициентов стохастического дифференциального уравнения\linebreak
\\[-12pt]
\hspace*{23pt}Ланжевена&3&\hphantom{1}3--12\\
\Avtors{Горшенин~А.\,К., Кузьмин~В.\,Ю.} Анализ конфигураций LSTM-сетей для построения\linebreak
\\[-12pt]
\hspace*{23pt}среднесрочных векторных прогнозов&1&10--16\\
\Avtors{Грабовой~А.\,В., Бахтеев~О.\,Ю., Стрижов~В.\,В.} Введение отношения порядка на множестве\linebreak
\\[-12pt]
\hspace*{23pt}параметров аппроксимирующих моделей&2&58--65\\
\Avtors{Грушо~А.\,А., Забежайло~М.\,И., Смирнов~Д.\,В., Тимонина~Е.\,Е.} О вероятностных оценках\linebreak
\\[-12pt]
\hspace*{23pt}достоверности эмпирических выводов&4&3--8\\
\Avtors{Грушо~А.\,А., Забежайло~М.\,И., Смирнов~Д.\,В., Тимонина~Е.\,Е., Шоргин~С.\,Я.} Методы\linebreak
\\[-12pt]
\hspace*{23pt}математической статистики в задаче поиска инсайдера&3&71--75\\
\Avtors{Грушо~А.\,А., Забежайло~М.\,И., Тимонина~Е.\,Е.} О каузальной репрезентативности обуча-\linebreak
\\[-12pt]
\hspace*{23pt}ющих выборок прецедентов в задачах диагностического типа&1&80--86\\
\Avtors{Грушо~А.\,А., Тимонина~Е.\,Е., Грушо~Н.\,А., Терехина~И.\,Ю.} Выявление аномалий с по-\linebreak
\\[-12pt]
\hspace*{23pt}мощью метаданных&3&76--80\\
\Avtors{Грушо~А.\,А.} см.\ Грушо~Н.\,А.&&\\
\Avtors{Грушо~Н.\,А., Грушо~А.\,А., Забежайло~М.\,И., Тимонина~Е.\,Е.} Методы нахождения причин\linebreak
\\[-12pt]
\hspace*{23pt}сбоев в информационных технологиях  с помощью метаданных&2&33--39\\
\Avtors{Грушо~Н.\,А.} см.\ Грушо~А.\,А.&&\\
\Avtors{Данилишин~А.\,Р., Голембиовский~Д.\,Ю.} Оценка стоимости опционов на основе моделей\linebreak
\\[-12pt]
\hspace*{23pt}ARIMA--GARCH с ошибками, распределенными по закону $S_u$ Джонсона&4&83--90\\
\Avtors{Данилишин~А.\,Р., Голембиовский~Д.\,Ю.} Риск-нейтральная динамика для модели ARIMA-\linebreak
\\[-12pt]
\hspace*{23pt}GARCH с ошибками, распределенными по закону $S_U$ Джонсона&1&48--55\\
\Avtors{Диментов~А.\,В.} см.\ Краснов~Ф.\,В.&&\\
\Avtors{Донской~В.\,И.} Извлечение оптимизационных моделей из данных&3&109--118\\
\Avtors{Дубнов~Ю.\,А.} см.\ Попков~Ю.\,С.&&\\
\Avtors{Дулин~С.\,К., Дулина~Н.\,Г., Ермаков~П.\,В.} Информационный синтез документов&1&128--135\\
\Avtors{Дулина~Н.\,Г.} см.\ Дулин~С.\,К.&&\\
\Avtors{Дьяченко~Ю.\,Г.} см.\ Соколов~И.\,А.&&\\
\Avtors{Ермаков~П.\,В.} см.\ Дулин~С.\,К.&&\\
\Avtors{Ефросинин~Д.\,В.} см.\ Харин~П.\,А.&&\\
\Avtors{Жолобов~В.\,А.} см.\ Потанин~М.\,С.&&\\
\Avtors{Забежайло~М.\,И.} см.\ Грушо~А.\,А.&&\\
\Avtors{Забежайло~М.\,И.} см.\ Грушо~А.\,А.&&\\
\Avtors{Забежайло~М.\,И.} см.\ Грушо~А.\,А.&&\\
\Avtors{Забежайло~М.\,И.} см.\ Грушо~Н.\,А.&&\\
\Avtors{Захаров В. Н.} см.\ Френкель С. Л.&&\\
\Avtors{Зацман~И.\,М.} Проблемно-ориентированная верификация полноты темпоральных\linebreak
\\[-12pt]
\hspace*{23pt}онтологий и заполнение понятийных лакун&3&119--128\\
\Avtors{Зацман~И.\,М.} см.\ Гончаров~А.\,А.&&\\
\Avtors{Зацман~И.\,М.} см.\ Нуриев~В.\,А.&&\\
\Avtors{Зейфман~А.\,И.} см.\ Сатин~Я.\,А.&&\\
\Avtors{Кириков~И.\,А.} см.\ Румовская~С.\,Б.&&\\
\Avtors{Кирилюк~И.\,Л., Сенько~О.\,В.} Выбор моделей оптимальной сложности методами Монте-Карло (на примере моделей производственных функций регионов Российской\linebreak
\\[-12pt]
\hspace*{23pt}Федерации)&2&111--118\\
\Avtors{Ковалёв~Д.\,Ю.} см.\ Брюхов~Д.\,О.&&\\
\Avtors{Козеренко~Е.\,Б., Михеев~М.\,Ю., Сомин~Н.\,В., Эрлих~Л.\,И., Кузнецов~К.\,И.} Аналити\-че\-ская
текс\-тология в системах интеллектуальной обработки неструктурированных\linebreak
\\[-12pt]
\hspace*{23pt}данных&1&113--120\\
\Avtors{Колбин~И.\,С.} см.\ Абгарян~К.\,К.&&\\
\end{tabular}
}

\pagebreak

\def\leftkol{АВТОРСКИЙ УКАЗАТЕЛЬ ЗА 2020 г.} % ENGLISH ABSTRACTS}

\def\rightkol{АВТОРСКИЙ УКАЗАТЕЛЬ ЗА 2020 г.} %ENGLISH ABSTRACTS}

%\thispagestyle{myheadings}
\def\leftfootline{\small{\textbf{\thepage}
\hfill ИНФОРМАТИКА И ЕЁ ПРИМЕНЕНИЯ\ \ \ том~14\ \ \ выпуск~4\ \ \ 2020}
}%
 \def\rightfootline{\small{ИНФОРМАТИКА И ЕЁ ПРИМЕНЕНИЯ\ \ \ том~14\ \ \ выпуск~4\ \ \ 2020
 \hfill \textbf{\thepage}}}


\noindent
{\tabcolsep=3pt
\begin{tabular}{p{394pt}cc}
&\textbf{Вып.} & \textbf{Стр.}\\[3pt]
\Avtors{Королев~В.\,Ю.} О распределении отношения суммы элементов выборки, превосходящих\linebreak
\\[-12pt]
\hspace*{23pt}некоторый порог, к сумме всех элементов выборки.~I&3&35--43\\
\Avtors{Королев~В.\,Ю.} О распределении отношения суммы элементов выборки, превосходящих\linebreak
\\[-12pt]
\hspace*{23pt}некоторый порог, к сумме всех элементов выборки.~II&4&33--36\\
\Avtors{Королев~В.\,Ю.} см.\ Горшенин~А.\,К&&\\
\Avtors{Королев~В.\,Ю.} см.\ Горшенин~А.\,К.&&\\
\Avtors{Королёв~В.\,И.} см.\ Будзко~В.\,И.&&\\
\Avtors{Костина~А.\,А., Мирин~А.\,Ю., Молдовян~Д.\,Н., Фахрутдинов~Р.\,Ш.} Метод задания конечных некоммутативных ассоциативных алгебр произвольной четной размерности\linebreak
\\[-12pt]
\hspace*{23pt}для построения постквантовых криптосхем&1&\hphantom{1}94--100\\
\Avtors{Кочеткова~И.\,А.} см.\ Харин~П.\,А.&&\\
\Avtors{Краснов~Ф.\,В., Диментов~А.\,В., Шварцман~М.\,Е.} Использование тематических моделей\linebreak
\\[-12pt]
\hspace*{23pt}для парного сравнения  коллекций научных статей&3&129--135\\
\Avtors{Кривенко~М.\,П.} Последовательный анализ серий данных на основе многомерных ре-\linebreak
\\[-12pt]
\hspace*{23pt}фе\-рен\-с\-ных регионов&2&86--91\\
\Avtors{Кружков~М.\,Г.} см.\ Гончаров~А.\,А.&&\\
\Avtors{Кудрявцев~А.\,А., Шестаков~О.\,В.} Метод логарифмических моментов для оценивания\linebreak
\\[-12pt]
\hspace*{23pt}параметров гамма-экспоненциального распределения&3&49--54\\
\Avtors{Кузнецов~К.\,И.} см.\ Козеренко~Е.\,Б.&&\\
\Avtors{Кузьмин~В.\,Ю.} см.\ Горшенин~А.\,К.&&\\
\Avtors{Кушниренко~А.\,Г.} см.\ Бетелин~В.\,Б.&&\\
\Avtors{Кушниренко~А.\,Г.} см.\ Бетелин~В.\,Б.&&\\
\Avtors{Леонов~А.\,Г.} см.\ Бетелин~В.\,Б.&&\\
\Avtors{Макеева~Е.\,Д.} см.\ Харин~П.\,А.&&\\
\Avtors{Малашенко~Ю.\,Е., Назарова~И.\,А.} Аппроксимация множества достижимых потоков\linebreak
\\[-12pt]
\hspace*{23pt}многопользовательской сети&3&81--85\\
\Avtors{Мартюшова~Я.\,Г.} см.\ Босов~А.\,В.&&\\
\Avtors{Матюшенко~С.\,И., Разумчик~Р.\,В.} Стационарные характеристики системы Geo$/G/1/\infty $\linebreak
\\[-12pt]
\hspace*{23pt}с неординарным входящим потоком, управляющим размером очереди&4&25--32\\
\Avtors{Мейханаджян~Л.\,А., Разумчик~Р.\,В.} Стационарные характеристики системы $M/G/2/\infty$ с одним частным случаем дисциплины инверсионного порядка обслуживания\linebreak
\\[-12pt]
\hspace*{23pt}с обобщенным  вероятностным приоритетом&2&66--71\\
\Avtors{Мельников~А.\,В.} см.\ Вохминцев~А.\,В.&&\\
\Avtors{Мельников~С.\,Ю., Самуйлов~К.\,Е.} Статистические свойства двоичных неавтономных\linebreak
\\[-12pt]
\hspace*{23pt}регистров сдвига  с внутренним суммированием&2&80--85\\
\Avtors{Милованова~Т.\,А., Разумчик~Р.\,В.} Однолинейная система массового обслуживания с инверсионным порядком обслуживания с вероятностным приоритетом, групповым\linebreak
\\[-12pt]
\hspace*{23pt}пуассоновским потоком и фоновыми заявками&3&26--34\\
\Avtors{Мирин~А.\,Ю.} см.\ Костина~А.\,А.&&\\
\Avtors{Михеев~М.\,Ю.} см.\ Козеренко~Е.\,Б.&&\\
\Avtors{Молдовян~Д.\,Н.} см.\ Костина~А.\,А.&&\\
\Avtors{Москалева~Ф.\,А., Гайдамака~Ю.\,В., Шоргин~В.\,С.} Влияние параметров изоляции на\linebreak
\\[-12pt]
\hspace*{23pt}разделение ресурсов при нарезке сети&4&\hphantom{1}9--16\\
\Avtors{Назарова~И.\,А.} см.\ Малашенко~Ю.\,Е.&&\\
\Avtors{Наумов~А.\,В.} см.\ Босов~А.\,В.&&\\
\Avtors{Наумов~В.\,А., Самуйлов~К.\,Е.} О марковских и рациональных потоках случайных со-\linebreak
\\[-12pt]
\hspace*{23pt}бытий.~I&3&13--19\\
\Avtors{Наумов~В.\,А., Самуйлов~К.\,Е.} О марковских и рациональных потоках случайных со-\linebreak
\\[-12pt]
\hspace*{23pt}бытий.~II&4&37--46\\
\Avtors{Новиков~Д.\,А.} см.\ Шнурков~П.\,В.&&\\
\Avtors{Нуриев~В.\,А., Зацман~И.\,М.} Редуцирование спектра моделей перевода в надкорпусных\linebreak
\\[-12pt]
\hspace*{23pt}базах данных&2&119--126\\
\Avtors{Пачганов~C.\,А.} см.\ Вохминцев~А.\,В.&&\\
\end{tabular}
}

\pagebreak

\def\leftkol{АВТОРСКИЙ УКАЗАТЕЛЬ ЗА 2020 г.} % ENGLISH ABSTRACTS}

\def\rightkol{АВТОРСКИЙ УКАЗАТЕЛЬ ЗА 2020 г.} %ENGLISH ABSTRACTS}

%\thispagestyle{myheadings}
\def\leftfootline{\small{\textbf{\thepage}
\hfill ИНФОРМАТИКА И ЕЁ ПРИМЕНЕНИЯ\ \ \ том~14\ \ \ выпуск~4\ \ \ 2020}
}%
 \def\rightfootline{\small{ИНФОРМАТИКА И ЕЁ ПРИМЕНЕНИЯ\ \ \ том~14\ \ \ выпуск~4\ \ \ 2020
 \hfill \textbf{\thepage}}}


\noindent
{\tabcolsep=3pt
\begin{tabular}{p{394pt}cc}
&\textbf{Вып.} & \textbf{Стр.}\\[3pt]
\Avtors{Попков~А.\,Ю.} см.\ Попков~Ю.\,С.&&\\
\Avtors{Попков~Ю.\,С., Попков~А.\,Ю., Дубнов~Ю.\,А.} Методы детерминированных и рандомизи-\linebreak
\\[-12pt]
\hspace*{23pt}рованных энтропийных проекций для редукции размерности матрицы данных&4&47--54\\
\Avtors{Попов~Г.\,А., Симаворян~С.\,Ж., Симонян~А.\,Р., Улитина~Е.\,И.} Моделирование процесса мониторинга систем информационной безопасности на основе систем массового\linebreak
\\[-12pt]
\hspace*{23pt}обслуживания&1&71--79\\
\Avtors{Попов~М.\,В., Посыпкин~М.\,А.} Аппроксимация множества решений систем нелинейных\linebreak
\\[-12pt]
\hspace*{23pt}неравенств с использованием графических ускорителей&3&20--25\\
\Avtors{Посыпкин~М.\,А.} см.\ Попов~М.\,В.&&\\
\Avtors{Потанин~М.\,С., Вайсер~К.\,О., Жолобов~В.\,А., Стрижов~В.\,В.} Оптимизация структуры\linebreak
\\[-12pt]
\hspace*{23pt}сетей глубокого обучения&4&55--62\\
\Avtors{Разумчик~Р.\,В.} см.\ Матюшенко~С.\,И.&&\\
\Avtors{Разумчик~Р.\,В.} см.\ Мейханаджян~Л.\,А.&&\\
\Avtors{Разумчик~Р.\,В.} см.\ Милованова~Т.\,А.&&\\
\Avtors{Рождественский~Ю.\,В.} см.\ Соколов~И.\,А.&&\\
\Avtors{Румовская~С.\,Б., Кириков~И.\,А.} Метод визуального представления конфликтов в гибрид-\linebreak
\\[-12pt]
\hspace*{23pt}ных интеллектуальных многоагентных системах&4&77--82\\
\Avtors{Самуйлов~К.\,Е.} см.\ Агеев~К.\,А.&&\\
\Avtors{Самуйлов~К.\,Е.} см.\ Мельников~С.\,Ю.&&\\
\Avtors{Самуйлов~К.\,Е.} см.\ Наумов~В.\,А.&&\\
\Avtors{Самуйлов~К.\,Е.} см.\ Наумов~В.\,А.&&\\
\Avtors{Сапунова~А.\,П.} см.\ Босов~А.\,В.&&\\
\Avtors{Сатин~Я.\,А., Зейфман~А.\,И., Шилова~Г.\,Н.} О подходах к построению предельных режимов\linebreak
\\[-12pt]
\hspace*{23pt}для некоторых моделей массового обслуживания&2&3--9\\
\Avtors{Севастьянов~Л.\,А., Щетинин~Е.\,Ю.} О методах повышения точности многоклассовой\linebreak
\\[-12pt]
\hspace*{23pt}классификации на несбалансированных данных&1&63--70\\
\Avtors{Семенов~А.\,Л.} см.\ Бетелин~В.\,Б.&&\\
\Avtors{Сенько~О.\,В.} см.\ Кирилюк~И.\,Л.&&\\
\Avtors{Серебрянский~С.\,М., Тырсин~А.\,Н.} Повышение точности решения обратных задач за\linebreak
\\[-12pt]
\hspace*{23pt}счет уточнения граничных условий&1&56--62\\
\Avtors{Симаворян~С.\,Ж.} см.\ Попов~Г.\,А.&&\\
\Avtors{Симонян~А.\,Р.} см.\ Попов~Г.\,А.&&\\
\Avtors{Смирнов~Д.\,В.} см.\ Грушо~А.\,А.&&\\
\Avtors{Смирнов~Д.\,В.} см.\ Грушо~А.\,А.&&\\
\Avtors{Соколов~И.\,А., Степченков~Ю.\,А., Дьяченко~Ю.\,Г., Рождественский~Ю.\,В.} Повышение\linebreak
\\[-12pt]
\hspace*{23pt}сбоеустойчивости самосинхронных схем&4&63--68\\
\Avtors{Сомин~Н.\,В.} см.\ Козеренко~Е.\,Б.&&\\
\Avtors{Сопин~Э.\,С.} см.\ Агеев~К.\,А.&&\\
\Avtors{Сопрунов~С.\,Ф.} см.\ Бетелин~В.\,Б.&&\\
\Avtors{Соченков~И.\,В.} см.\ Будзко~В.\,И.&&\\
\Avtors{Степченков~Ю.\,А.} см.\ Соколов~И.\,А.&&\\
\Avtors{Стефанович~А.\,И.} см.\ Босов~А.\,В.&&\\
\Avtors{Стрижов~В.\,В.} см.\ Гончаров~А.\,В.&&\\
\Avtors{Стрижов~В.\,В.} см.\ Грабовой~А.\,В.&&\\
\Avtors{Стрижов~В.\,В.} см.\ Потанин~М.\,С.&&\\
\Avtors{Ступников~С.\,А.} см.\ Брюхов~Д.\,О.&&\\
\Avtors{Терехина~И.\,Ю.} см.\ Грушо~А.\,А.&&\\
\Avtors{Тимонина~Е.\,Е.} см.\  Грушо~А.\,А.&&\\
\Avtors{Тимонина~Е.\,Е.} см.\ Грушо~А.\,А.&&\\
\Avtors{Тимонина~Е.\,Е.} см.\ Грушо~А.\,А.&&\\
\Avtors{Тимонина~Е.\,Е.} см.\ Грушо~А.\,А.&&\\
\Avtors{Тимонина~Е.\,Е.} см.\ Грушо~Н.\,А.&&\\
\Avtors{Тырсин~А.\,Н.} см.\ Серебрянский~С.\,М.&&\\
\Avtors{Улитина~Е.\,И.} см.\ Попов~Г.\,А.&&\\
\end{tabular}
}

\pagebreak

\def\leftkol{АВТОРСКИЙ УКАЗАТЕЛЬ ЗА 2020 г.} % ENGLISH ABSTRACTS}

\def\rightkol{АВТОРСКИЙ УКАЗАТЕЛЬ ЗА 2020 г.} %ENGLISH ABSTRACTS}

%\thispagestyle{myheadings}
\def\leftfootline{\small{\textbf{\thepage}
\hfill ИНФОРМАТИКА И ЕЁ ПРИМЕНЕНИЯ\ \ \ том~14\ \ \ выпуск~4\ \ \ 2020}
}%
 \def\rightfootline{\small{ИНФОРМАТИКА И ЕЁ ПРИМЕНЕНИЯ\ \ \ том~14\ \ \ выпуск~4\ \ \ 2020
 \hfill \textbf{\thepage}}}


\noindent
{\tabcolsep=3pt
\begin{tabular}{p{394pt}cc}
&\textbf{Вып.} & \textbf{Стр.}\\[3pt]
\Avtors{Фахрутдинов~Р.\,Ш.} см.\ Костина~А.\,А.&&\\
\Avtors{Френкель С. Л., Захаров В. Н.} Совместная оценка предсказуемости данных и качества\linebreak
\\[-12pt]
\hspace*{23pt}предикторов&2&40--49\\
\Avtors{Харин~П.\,А., Макеева~Е.\,Д., Кочеткова~И.\,А., Ефросинин~Д.\,В., Шоргин~С.\,Я.} 
Система массового обслуживания с орбитами для анализа совместного обслуживания трафика 
с малыми задержками URLLC и~широкополосного доступа eMBB в~беспроводных\linebreak
\\[-12pt]
\hspace*{23pt}сетях пятого поколения&4&17--24\\
\Avtors{Хусаинов~А.\,А.} Производительность ограниченного конвейера&1&87--93\\
\Avtors{Шанин~И.\,А.} см.\ Брюхов~Д.\,О.&&\\
\Avtors{Шварцман~М.\,Е.} см.\ Краснов~Ф.\,В.&&\\
\Avtors{Шестаков~О.\,В.} Асимптотика оценки среднеквадратичного риска в задаче обращения\linebreak
\\[-12pt]
\hspace*{23pt}преобразования Радона по проекциям, регистрируемым на случайной сетке&2&26--32\\
\Avtors{Шестаков~О.\,В.} Асимптотическая регулярность вейвлет-методов обращения линейных однородных операторов по наблюдениям, регистрируемым в случайные моменты\linebreak
\\[-12pt]
\hspace*{23pt}времени&1&3--9\\
\Avtors{Шестаков~О.\,В.} О статистических свойствах оценки риска в задаче обращения преобра-\linebreak
\\[-12pt]
\hspace*{23pt}зования Радона при случайном объеме проекционных данных&3&44--48\\
\Avtors{Шестаков~О.\,В.} см.\ Кудрявцев~А.\,А.&&\\
\Avtors{Шилова~Г.\,Н.} см.\ Сатин~Я.\,А.&&\\
\Avtors{Шихиев~Ф.\,Ш.} см.\ Шихиев~Ш.\,Б.&&\\
\Avtors{Шихиев~Ш.\,Б., Шихиев~Ф.\,Ш.} Инкапсуляция семантических представлений в элементы\linebreak
\\[-12pt]
\hspace*{23pt}грамматики&1&121--127\\
\Avtors{Шнурков~П.\,В., Адамова~К.\,А.} Решение задачи безусловного экстремума для дробно-\linebreak
\\[-12pt]
\hspace*{23pt}линейного интегрального функционала, зависящего от параметра&2&\hphantom{1}98--103\\
\Avtors{Шнурков~П.\,В., Новиков~Д.\,А.} О концепции стохастической модели с управлением в~моменты выхода процесса на границу заданного подмножества множества\linebreak
\\[-12pt]
\hspace*{23pt}состояний&3&101--108\\
\Avtors{Шоргин~В.\,С.} см.\ Москалева~Ф.\,А.&&\\
\Avtors{Шоргин~С.\,Я.} см.\ Агеев~К.\,А.&&\\
\Avtors{Шоргин~С.\,Я.} см.\ Грушо~А.\,А.&&\\
\Avtors{Шоргин~С.\,Я.} см.\ Харин~П.\,А.&&\\
\Avtors{Щербинина~А.\,А.} см.\ Горшенин~А.\,К.&&\\
\Avtors{Щетинин~Е.\,Ю.} см.\ Севастьянов~Л.\,А.&&\\
\Avtors{Эрлих~Л.\,И.} см.\ Козеренко~Е.\,Б.&&\\
\Avtors{Ядринцев~В.\,В.} см.\ Будзко~В.\,И.&&\\
\Avtors{Яркина~Н.\,В.} см.\ Агеев~К.\,А.&&\\
\end{tabular}
}

%\thispagestyle{myheadings}
\def\leftfootline{\small{\textbf{\thepage}
\hfill ИНФОРМАТИКА И ЕЁ ПРИМЕНЕНИЯ\ \ \ том~14\ \ \ выпуск~4\ \ \ 2020}
}%
 \def\rightfootline{\small{ИНФОРМАТИКА И ЕЁ ПРИМЕНЕНИЯ\ \ \ том~14\ \ \ выпуск~4\ \ \ 2020
 \hfill \textbf{\thepage}}}

 \label{end\stat}

\newpage

\def\stat{cont-e}
{%\hrule\par
%\vskip 7pt % 7pt
\raggedleft\Large \bf%\baselineskip=3.2ex
2\,0\,2\,0\ \ A\,U\,T\,H\,O\,R\ \ I\,N\,D\,E\,X \vskip 17pt
 \hrule
 \par
\vskip 21pt plus 6pt minus 3pt }

\label{st\stat}

\def\tit{\ }

\def\aut{\ }
\def\auf{\ }

\def\leftkol{\ } %2020 AUTHOR INDEX} % ENGLISH ABSTRACTS}

\def\rightkol{\ } %2020 AUTHOR INDEX} %ENGLISH ABSTRACTS}

\titele{\tit}{\aut}{\auf}{\leftkol}{\rightkol}
\addcontentsline{toc}{subsection}{\textrm\textbf 2020 Author Index}

\def\leftfootline{\small{\textbf{\thepage}
\hfill INFORMATIKA I EE PRIMENENIYA~--- INFORMATICS AND APPLICATIONS\ \ \ 2020\
\ \ volume~14\ \ \ issue\ 4}
}%
 \def\rightfootline{\small{INFORMATIKA I EE PRIMENENIYA~--- INFORMATICS AND APPLICATIONS\ \ \ 2020\ \ \ volume~14\ \ \ issue\ 4
\hfill \textbf{\thepage}}}

\vspace*{-24pt}

\noindent
{\tabcolsep=3pt
\begin{tabular}{p{395.89pt}cc}
&\textbf{Issue} & \textbf{Page}\\[6pt]
\Avtors{Abgaryan~K.\,K. and Gavrilov~E.\,S.} Integration platform for multiscale modeling of neuromorphic\linebreak
\\[-12pt]
\hspace*{23pt}systems&2&104--110\\
\Avtors{Abgaryan~K.\,K. and Kolbin~I.\,S.} Application of multiscale approach and data sciences for\linebreak
\\[-12pt]
\hspace*{23pt}modeling thermal conductivity in layered structures&4&91--99\\
\Avtors{Adamova~K.\,A.} see Shnurkov~~P.\,V.&&\\
\Avtors{Agalarov~Ya.\,M.} Optimization of the capacity of the main storage in $G/M/1/K$ queueing system\linebreak
\\[-12pt]
\hspace*{23pt}with an additional storage device&2&72--79\\
\Avtors{Agasandyan~G.\,A.} Computational aspects of optimization on CC-VaR in a complex of markets&3&62--70\\
\Avtors{Ageev~K.\,A., Sopin~E.\,S., Yarkina~N.\,V., Samouylov~K.\,E., and Shorgin~S.\,Ya.} Analysis of the\linebreak
\\[-12pt]
\hspace*{23pt}network slicing mechanisms with guaranteed allocated resources for various traffic types&3&\hphantom{1}94--100\\
\Avtors{Bakhteev~O.\,Yu.} see Grabovoy~A.\,V.&&\\
\Avtors{Bazilevskiy~M.\,P.} Multifactor fully connected linear regression models without constraints to the\linebreak
\\[-12pt]
\hspace*{23pt}ratios of variables errors variances&2&92--97\\
\Avtors{Belenkov~V.\,G.} see Budzko~V.\,I.&&\\
\Avtors{Betelin~V.\,B., Kushnirenko~A.\,G., and Leonov~A.\,G.} Basic concepts of programming expounded\linebreak
\\[-12pt]
\hspace*{23pt}for preschoolers&3&55--61\\
\Avtors{Betelin~V.\,B., Kushnirenko~A.\,G., Semenov~A.\,L., and Soprunov~S.\,F.} About digital literacy and\linebreak
\\[-12pt]
\hspace*{23pt}environments for its development&4&100--107\\
\Avtors{Borisov~A.\,V.} Numerical schemes of Markov jump process filtering given discretized observa-\linebreak
\\[-12pt]
\hspace*{23pt}tions~II: Additive noise case&1&17--23\\
\Avtors{Borisov~A.\,V.} Numerical schemes of Markov jump process filtering given discretized observa-\linebreak
\\[-12pt]
\hspace*{23pt}tions III: Multiplicative noises case&2&10--18\\
\Avtors{Bosov~A.\,V.} Stochastic differential system output control by the quadratic criterion. V. Case of\linebreak
\\[-12pt]
\hspace*{23pt}incomplete state information&2&19--28\\
\Avtors{Bosov~A.\,V., Martyushova~Ya.\,G., Naumov~A.\,V., and Sapunova~A.\,P.} Bayesian approach to the\linebreak
\\[-12pt]
\hspace*{23pt}construction of an individual user trajectory in the system of distance learning&3&86--93\\
\Avtors{Bosov~A.\,V. and Stefanovich~A.\,I.} Stochastic differential system output control by the quadratic\linebreak
\\[-12pt]
\hspace*{23pt}criterion. IV. Alternative numerical decision&1&24--30\\
\Avtors{Briukhov~D.\,O., Stupnikov~S.\,A., Kovalev~D.\,Yu., and Shanin~I.\,A.} Neurophysiology as a subject\linebreak
\\[-12pt]
\hspace*{23pt}domain for~data intensive problem solving&1&40--47\\
\Avtors{Budzko~V.\,I., Yadrintsev~V.\,V., Sochenkov~I.\,V., Korolev~V.\,I., and Belenkov~V.\,G.} Extraction of confidentiality markers from texts under conditions of high uncertainty in systems with\linebreak
\\[-12pt]
\hspace*{23pt}data intensive usage&4&69--76\\
\Avtors{Danilishin~A.\,R. and Golembiovsky~D.\,Yu.} Estimating the fair value of options based on\linebreak
\\[-12pt]
\hspace*{23pt}ARIMA--GARCH models with errors distributed according to the Johnson's $S_u$ law&4&83--90\\
\Avtors{Danilishin~A.\,R. and Golembiovsky~D.\,Yu.} Risk-neutral dynamics for the ARIMA-GARCH\linebreak
\\[-12pt]
\hspace*{23pt}random process with errors distributed according to the Johnson's $S_u$ law&1&48--55\\
\Avtors{Diachenko~Yu.\,G.} see Sokolov~I.\,A.&&\\
\Avtors{Dimentov~A.\,V.} see Krasnov~F.\,V.&&\\
\Avtors{Donskoy~V.\,I.} Optimization models extraction from data&3&109--118\\
\Avtors{Dubnov~Y.\,A.} see Popkov~Y.\,S.&&\\
\Avtors{Dulin~S.\,K., Dulina~N.\,G., and Ermakov~P.\,V.} Information fusion of documents&1&128--135\\
\Avtors{Dulina~N.\,G.} see Dulin~S.\,K.&&\\
\Avtors{Efrosinin~D.\,V.} see Kharin~P.\,A.&&\\
\Avtors{Ehrlich~L.\,I.} see Kozerenko~E.\,B.&&\\
\Avtors{Ermakov~P.\,V.} see Dulin~S.\,K.&&\\
\end{tabular}
}
\pagebreak

\def\leftfootline{\small{\textbf{\thepage}
\hfill INFORMATIKA I EE PRIMENENIYA~--- INFORMATICS AND APPLICATIONS\ \ \ 2020\
\ \ volume~14\ \ \ issue\ 4}
}%
 \def\rightfootline{\small{INFORMATIKA I EE PRIMENENIYA~---
INFORMATICS AND APPLICATIONS\ \ \ 2020\ \ \ volume~14\ \ \ issue\ 4
\hfill \textbf{\thepage}}}

\def\leftkol{2020 AUTHOR INDEX} % ENGLISH ABSTRACTS}

\def\rightkol{2020 AUTHOR INDEX} %ENGLISH ABSTRACTS}


\noindent
{\tabcolsep=3pt
\begin{tabular}{p{395.48108pt}cc}
&\textbf{Issue} & \textbf{Page}\\[6pt]
\Avtors{Fahrutdinov~R.\,Sh.} see Kostina~A.\,A.&&\\
\Avtors{Frenkel~S.\,L. and Zakharov~V.\,N.} Joint assessment of data predictability and quality pre-\linebreak
\\[-12pt]
\hspace*{23pt}dictors&2&40--49\\
\Avtors{Gaidamaka~Yu.\,V.} see Moskaleva~F.\,A.&&\\
\Avtors{Gavrilov~E.\,S.} see Abgaryan~K.\,K.&&\\
\Avtors{Golembiovsky~D.\,Yu.} see Danilishin~A.\,R.&&\\
\Avtors{Golembiovsky~D.\,Yu.} see Danilishin~A.\,R.&&\\
\Avtors{Goncharov~A.\,V. and Strijov~V.\,V.} Alignment of ordered set Cartesian product&1&31--39\\
\Avtors{Goncharov~A.\,A., Zatsman~I.\,M., and Kruzhkov~M.\,G.} Evolution of classifications in supracorpora\linebreak
\\[-12pt]
\hspace*{23pt}databases&4&108--116\\
\Avtors{Gorshenin~A.\,K. and Korolev~V.\,Yu.} Approximation of particle size distributions of lunar regolith\linebreak
\\[-12pt]
\hspace*{23pt}based on the resampling&2&50--57\\
\Avtors{Gorshenin~A.\,K., Korolev~V.\,Yu., and Shcherbinina~A.\,A.} Statistical estimation of distributions\linebreak
\\[-12pt]
\hspace*{23pt}of random coefficients in the Langevin stochastic differential equation&3&\hphantom{1}3--12\\
\Avtors{Gorshenin~A.\,K. and Kuzmin~V.\,Yu.} Analysis of configurations of LSTM networks for medium-\linebreak
\\[-12pt]
\hspace*{23pt}term vector forecasting&1&10--16\\
\Avtors{Grabovoy~A.\,V., Bakhteev~O.\,Yu., and Strijov~V.\,V.} Ordering the set of neural network parameters&2&58--65\\
\Avtors{Grusho~A.\,A., Timonina~E.\,E., Grusho~N.\,A., and Teryokhina~I.\,Yu.} Identifying anomalies using\linebreak
\\[-12pt]
\hspace*{23pt}metadata&3&76--80\\
\Avtors{Grusho~A.\,A., Zabezhailo~M.\,I., Smirnov~D.\,V., and Timonina~E.\,E.} On probabilistic estimates of\linebreak
\\[-12pt]
\hspace*{23pt}the validity of empirical conclusions&4&3--8\\
\Avtors{Grusho~A.\,A., Zabezhailo~M.\,I., and Timonina~E.\,E.} On causal representativeness of training\linebreak
\\[-12pt]
\hspace*{23pt}samples of precedents in diagnostic type tasks&1&80--86\\
\Avtors{Grusho~A.\,A.} see Grusho~N.\,A.&&\\
\Avtors{Grusho~N.\,A., Grusho~A.\,A., Zabezhailo~M.\,I., and Timonina~E.\,E.} Methods of finding the causes\linebreak
\\[-12pt]
\hspace*{23pt}of information technology failures by means of metadata&2&33--39\\
\Avtors{Grusho~N.\,A., Zabezhailo~M.\,I., Smirnov~D.\,V., Timonina~E.\,E., and Shorgin~S.\,Ya.} Mathematical\linebreak
\\[-12pt]
\hspace*{23pt}statistics in the task of identifying hostile insiders&3&71--75\\
\Avtors{Grusho~N.\,A.} see Grusho~A.\,A.&&\\
\Avtors{Kharin~P.\,A., Makeeva~E.\,D., Kochetkova~I.\,A., Efrosinin~D.\,V., and Shorgin~S.\,Ya.} Retrial\linebreak
\\[-12pt]
\hspace*{23pt}queuing model for analyzing joint URLLC and eMBB transmission in 5G networks&4&17--24\\
\Avtors{Khusainov~A.\,A.} Performance of the bounded pipeline&1&87--93\\
\Avtors{Kirikov~I.\,A.} see Rumovskaya~S.\,B.&&\\
\Avtors{Kirilyuk~I.\,L. and Sen'ko~O.\,V.} Selection of optimal complexity models by methods of nonparametric statistics (on the example of production function model of regions of the Russian\linebreak
\\[-12pt]
\hspace*{23pt}Federation)&2&111--118\\
\Avtors{Kochetkova~I.\,A.} see Kharin~P.\,A.&&\\
\Avtors{Kolbin~I.\,S.} see Abgaryan~K.\,K.&&\\
\Avtors{Korolev~V.\,I.} see Budzko~V.\,I.&&\\
\Avtors{Korolev~V.\,Yu.} On the distribution of the ratio of the sum of sample elements exceeding\linebreak
\\[-12pt]
\hspace*{23pt}a threshold to the total sum of sample elements.~I&3&35--43\\
\Avtors{Korolev~V.\,Yu.} On the distribution of the ratio of the sum of sample elements exceeding\linebreak
\\[-12pt]
\hspace*{23pt}a threshold to the total sum of sample elements.~II&4&33--36\\
\Avtors{Korolev~V.\,Yu.} see Gorshenin~A.\,K.&&\\
\Avtors{Korolev~V.\,Yu.} see Gorshenin~A.\,K.&&\\
\Avtors{Kostina~A.\,A., Mirin~A.\,Yu., Moldovyan~D.\,N., and Fahrutdinov~R.\,Sh.} Method for defining finite noncommutative associative algebras of arbitrary even dimension for development of the\linebreak
\\[-12pt]
\hspace*{23pt}postquantum cryptoschemes&1&\hphantom{1}94--100\\
\Avtors{Kovalev~D.\,Yu.} see Briukhov~D.\,O.&&\\
\Avtors{Kozerenko~E.\,B., Mikheev~M.\,Y., Somin~N.\,V., Ehrlich~L.\,I., and Kuznetsov~K.\,I.} Analytical\linebreak
\\[-12pt]
\hspace*{23pt}textology in intelligent processing systems for unstructured data&1&113--120\\
\Avtors{Krasnov~F.\,V., Dimentov~A.\,V., and Shvartsman~M.\,E.} Using topic models for pairwise comparison\linebreak
\\[-12pt]
\hspace*{23pt}of collections of scientific papers&3&129--135\\
\end{tabular}
}
\pagebreak

\def\leftfootline{\small{\textbf{\thepage}
\hfill INFORMATIKA I EE PRIMENENIYA~--- INFORMATICS AND APPLICATIONS\ \ \ 2020\
\ \ volume~14\ \ \ issue\ 4}
}%
 \def\rightfootline{\small{INFORMATIKA I EE PRIMENENIYA~---
INFORMATICS AND APPLICATIONS\ \ \ 2020\ \ \ volume~14\ \ \ issue\ 4
\hfill \textbf{\thepage}}}

\def\leftkol{2020 AUTHOR INDEX} % ENGLISH ABSTRACTS}

\def\rightkol{2020 AUTHOR INDEX} %ENGLISH ABSTRACTS}


\noindent
{\tabcolsep=3pt
\begin{tabular}{p{395.48108pt}cc}
&\textbf{Issue} & \textbf{Page}\\[6pt]
\Avtors{Krivenko~M.\,P.} Sequential analysis of serial measurements based on multivariate reference\linebreak
\\[-12pt]
\hspace*{23pt}regions&2&86--91\\
\Avtors{Kruzhkov~M.\,G.} see Goncharov~A.\,A.&&\\
\Avtors{Kudryavtsev~A.\,A. and Shestakov~O.\,V.} Method of logarithmic moments for estimating the\linebreak
\\[-12pt]
\hspace*{23pt}gamma-exponential distribution parameters&3&49--54\\
\Avtors{Kushnirenko~A.\,G.} see Betelin~V.\,B.&&\\
\Avtors{Kushnirenko~A.\,G.} see Betelin~V.\,B.&&\\
\Avtors{Kuzmin~V.\,Yu.} see Gorshenin~A.\,K.&&\\
\Avtors{Kuznetsov~K.\,I.} see Kozerenko~E.\,B.&&\\
\Avtors{Leonov~A.\,G.} see Betelin~V.\,B.&&\\
\Avtors{Makeeva~E.\,D.} see Kharin~P.\,A.&&\\
\Avtors{Malashenko~Yu.\,E. and Nazarova~I.\,A.} Approximation of the multiuser network feasible\linebreak
\\[-12pt]
\hspace*{23pt}flows set&3&81--85\\
\Avtors{Martyushova~Ya.\,G.} see Bosov~A.\,V.&&\\
\Avtors{Matyushenko~S.\,I. and Razumchik~R.\,V.} Stationary characteristics of discrete-time Geo$/G/1/\infty$\linebreak
\\[-12pt]
\hspace*{23pt}queue with batch arrivals and one queue skipping policy&4&25--32\\
\Avtors{Melnikov~A.\,V.} see Vokhmintcev~A.\,V.&&\\
\Avtors{Melnikov~S.\,Yu. and Samouylov~K.\,E.} Statistical properties of binary nonautonomous shift\linebreak
\\[-12pt]
\hspace*{23pt}registers with internal xor&2&80--85\\
\Avtors{Meykhanadzhyan~L.\,A. and Razumchik~R.\,V.} Stationary characteristics of $M/G/2/\infty$ queue\linebreak
\\[-12pt]
\hspace*{23pt}with identical servers, LIFO service, and resampling policy&2&66--71\\
\Avtors{Mikheev~M.\,Y.} see Kozerenko~E.\,B.&&\\
\Avtors{Milovanova~T.\,A. and Razumchik~R.\,V.} A single-server queueing system with LIFO service,\linebreak
\\[-12pt]
\hspace*{23pt}probabilistic priority, batch Poisson arrivals, and background customers&3&26--34\\
\Avtors{Mirin~A.\,Yu.} see Kostina~A.\,A.&&\\
\Avtors{Moldovyan~D.\,N.} see Kostina~A.\,A.&&\\
\Avtors{Moskaleva~F.\,A., Gaidamaka~Yu.\,V., and Shorgin~V.\,S.} Impact of the isolation parameters on\linebreak
\\[-12pt]
\hspace*{23pt}resource allocation in the network slicing model&4&\hphantom{1}9--16\\
\Avtors{Naumov~A.\,V.} see Bosov~A.\,V.&&\\
\Avtors{Naumov~V.\,A. and Samouylov~К.\,Е.} On Markovian and rational arrival processes.~I&3&13--19\\
\Avtors{Naumov~V.\,A. and Samouylov~K.\,E.} On Markovian and rational arrival processes.~II&4&37--46\\
\Avtors{Nazarova~I.\,A.} see Malashenko~Yu.\,E.&&\\
\Avtors{Novikov~D.\,A.} see Shnurkov~P.\,V.&&\\
\Avtors{Nuriev~V.\,A. and Zatsman~I.\,M.} Reducing the spectrum of translation models in supracorpora\linebreak
\\[-12pt]
\hspace*{23pt}databases&2&119--126\\
\Avtors{Pachganov~S.\,A.} see Vokhmintcev~A.\,V.&&\\
\Avtors{Popkov~A.\,Y.} see Popkov~Y.\,S.&&\\
\Avtors{Popkov~Y.\,S., Popkov~A.\,Y., and Dubnov~Y.\,A.} Deterministic and randomized methods of entropy\linebreak
\\[-12pt]
\hspace*{23pt}projection for dimensionality reduction problems&4&47--54\\
\Avtors{Popov~G.\,A., Simavoryan~S.\,Zh., Simonyan~A.\,R., and Ulitina~E.\,I.} Modeling of monitoring of\linebreak
\\[-12pt]
\hspace*{23pt}information security process on the basis of queuing systems&1&71--79\\
\Avtors{Popov~M.\,V. and Posypkin~M.\,A.} Approximation of the set of solutions of systems of nonlinear\linebreak
\\[-12pt]
\hspace*{23pt}inequalities using graphic accelerators&3&20--25\\
\Avtors{Posypkin~M.\,A.} see Popov~M.\,V.&&\\
\Avtors{Potanin~M.\,S., Vayser~K.\,O., Zholobov~V.\,A., and Strijov~V.\,V.} Deep learning neural network\linebreak
\\[-12pt]
\hspace*{23pt}structure optimization&4&55--62\\
\Avtors{Razumchik~R.\,V.} see Matyushenko~S.\,I.&&\\
\Avtors{Razumchik~R.\,V.} see Meykhanadzhyan~L.\,A.&&\\
\Avtors{Razumchik~R.\,V.} see Milovanova~T.\,A.&&\\
\Avtors{Rogdestvenski~Yu.\,V.} see Sokolov~I.\,A.&&\\
\Avtors{Rumovskaya~S.\,B. and Kirikov~I.\,A.} Conflict visual representation method in hybrid intelligent\linebreak
\\[-12pt]
\hspace*{23pt}multiagent systems&4&77--82\\
\Avtors{Samouylov~K.\,E.} see Ageev~K.\,A.&&\\
\end{tabular}
}
\pagebreak

\def\leftfootline{\small{\textbf{\thepage}
\hfill INFORMATIKA I EE PRIMENENIYA~--- INFORMATICS AND APPLICATIONS\ \ \ 2020\
\ \ volume~14\ \ \ issue\ 4}
}%
 \def\rightfootline{\small{INFORMATIKA I EE PRIMENENIYA~---
INFORMATICS AND APPLICATIONS\ \ \ 2020\ \ \ volume~14\ \ \ issue\ 4
\hfill \textbf{\thepage}}}

\def\leftkol{2020 AUTHOR INDEX} % ENGLISH ABSTRACTS}

\def\rightkol{2020 AUTHOR INDEX} %ENGLISH ABSTRACTS}


\noindent
{\tabcolsep=3pt
\begin{tabular}{p{395.48108pt}cc}
&\textbf{Issue} & \textbf{Page}\\[6pt]
\Avtors{Samouylov~K.\,E.} see Melnikov~S.\,Yu.&&\\
\Avtors{Samouylov~K.\,E.} see Naumov~V.\,A.&&\\
\Avtors{Samouylov~K.\,Е.} see Naumov~V.\,A.&&\\
\Avtors{Sapunova~A.\,P.} see Bosov~A.\,V.&&\\
\Avtors{Satin~Ya.\,A., Zeifman~A.\,I., and Shilova~G.\,N.} On approaches to constructing limiting regimes\linebreak
\\[-12pt]
\hspace*{23pt}for some queuing models&2&3--9\\
\Avtors{Semenov~A.\,L.} see Betelin~V.\,B.&&\\
\Avtors{Sen'ko~O.\,V.} see Kirilyuk~I.\,L.&&\\
\Avtors{Serebryanskii~S.\,M. and Tyrsin~A.\,N.} Improvement of the accuracy of solution of tasks for the\linebreak
\\[-12pt]
\hspace*{23pt}account of the construction of boundary conditions&1&56--62\\
\Avtors{Sevastianov~L.\,A. and Shchetinin~E.\,Yu.} On methods for improving the accuracy of multiclass\linebreak
\\[-12pt]
\hspace*{23pt}classification on imbalanced data&1&63--70\\
\Avtors{Shanin~I.\,A.} see Briukhov~D.\,O.&&\\
\Avtors{Shcherbinina~A.\,A.} see Gorshenin~A.\,K.&&\\
\Avtors{Shchetinin~E.\,Yu.} see Sevastianov~L.\,A.&&\\
\Avtors{Shestakov~O.\,V.} Asymptotic regularity of the wavelet methods of inverting linear homogeneous\linebreak
\\[-12pt]
\hspace*{23pt}operators from observations recorded at random times&1&3--9\\
\Avtors{Shestakov~O.\,V.} Asymptotics of the mean-square risk estimate in the problem of inverting the\linebreak
\\[-12pt]
\hspace*{23pt}Radon transform from projections registered on a random grid&2&29--32\\
\Avtors{Shestakov~O.\,V.} On the statistical properties of risk estimate in the problem of inverting the\linebreak
\\[-12pt]
\hspace*{23pt}Radon transform with a random volume of projection data&3&44--48\\
\Avtors{Shestakov~O.\,V.} see Kudryavtsev~A.\,A.&&\\
\Avtors{Shihiev~F.\,Sh.} see Shihiev~Sh.\,B.&&\\
\Avtors{Shihiev~Sh.\,B. and Shihiev~F.\,Sh.} Incapsulation of semantic representations into elements of\linebreak
\\[-12pt]
\hspace*{23pt}a grammar&1&121--127\\
\Avtors{Shilova~G.\,N.} see Satin~Ya.\,A.&&\\
\Avtors{Shnurkov~~P.\,V. and Adamova~K.\,A.} Solution of the unconditional extremal problem for a~linear-\linebreak
\\[-12pt]
\hspace*{23pt}fractional integral functional dependent on the parameter&2&\hphantom{1}98--103\\
\Avtors{Shnurkov~P.\,V. and Novikov~D.\,A.} On the concept of a stochastic model with control at the\linebreak
\\[-12pt]
\hspace*{23pt}moments of the process at the border of a presented subset of multiple states&3&101--108\\
\Avtors{Shorgin~S.\,Ya.} see Ageev~K.\,A.&&\\
\Avtors{Shorgin~S.\,Ya.} see Grusho~N.\,A.&&\\
\Avtors{Shorgin~S.\,Ya.} see Kharin~P.\,A.&&\\
\Avtors{Shorgin~V.\,S.} see Moskaleva~F.\,A.&&\\
\Avtors{Shvartsman~M.\,E.} see Krasnov~F.\,V.&&\\
\Avtors{Simavoryan~S.\,Zh.} see Popov~G.\,A.&&\\
\Avtors{Simonyan~A.\,R.} see Popov~G.\,A.&&\\
\Avtors{Smirnov~D.\,V.} see Grusho~A.\,A.&&\\
\Avtors{Smirnov~D.\,V.} see Grusho~N.\,A.&&\\
\Avtors{Sochenkov~I.\,V.} see Budzko~V.\,I.&&\\
\Avtors{Sokolov~I.\,A., Stepchenkov~Yu.\,A., Diachenko~Yu.\,G., and Rogdestvenski~Yu.\,V.} Improvement of\linebreak
\\[-12pt]
\hspace*{23pt}self-timed circuit soft error tolerance&4&63--68\\
\Avtors{Somin~N.\,V.} see Kozerenko~E.\,B.&&\\
\Avtors{Sopin~E.\,S.} see Ageev~K.\,A.&&\\
\Avtors{Soprunov~S.\,F.} see Betelin~V.\,B.&&\\
\Avtors{Stefanovich~A.\,I.} see Bosov~A.\,V.&&\\
\Avtors{Stepchenkov~Yu.\,A.} see Sokolov~I.\,A.&&\\
\Avtors{Strijov~V.\,V.} see Goncharov~A.\,V.&&\\
\Avtors{Strijov~V.\,V.} see Grabovoy~A.\,V.&&\\
\Avtors{Strijov~V.\,V.} see Potanin~M.\,S.&&\\
\Avtors{Stupnikov~S.\,A.} see Briukhov~D.\,O.&&\\
\Avtors{Teryokhina~I.\,Yu.} see Grusho~A.\,A.&&\\
\Avtors{Timonina~E.\,E.} see Grusho~A.\,A.&&\\
\end{tabular}
}
\pagebreak

\def\leftfootline{\small{\textbf{\thepage}
\hfill INFORMATIKA I EE PRIMENENIYA~--- INFORMATICS AND APPLICATIONS\ \ \ 2020\
\ \ volume~14\ \ \ issue\ 4}
}%
 \def\rightfootline{\small{INFORMATIKA I EE PRIMENENIYA~---
INFORMATICS AND APPLICATIONS\ \ \ 2020\ \ \ volume~14\ \ \ issue\ 4
\hfill \textbf{\thepage}}}

\def\leftkol{2020 AUTHOR INDEX} % ENGLISH ABSTRACTS}

\def\rightkol{2020 AUTHOR INDEX} %ENGLISH ABSTRACTS}


\noindent
{\tabcolsep=3pt
\begin{tabular}{p{395.48108pt}cc}
&\textbf{Issue} & \textbf{Page}\\[6pt]
\Avtors{Timonina~E.\,E.} see Grusho~A.\,A.&&\\
\Avtors{Timonina~E.\,E.} see Grusho~A.\,A.&&\\
\Avtors{Timonina~E.\,E.} see Grusho~N.\,A.&&\\
\Avtors{Timonina~E.\,E.} see Grusho~N.\,A.&&\\
\Avtors{Tyrsin~A.\,N.} see Serebryanskii~S.\,M.&&\\
\Avtors{Ulitina~E.\,I.} see Popov~G.\,A.&&\\
\Avtors{Vayser~K.\,O.} see Potanin~M.\,S.&&\\
\Avtors{Vokhmintcev~A.\,V., Melnikov~A.\,V., and Pachganov~S.\,A.} Simultaneous localization and mapping method in  three-dimensional space based on the combined solution of the  point--point\linebreak
\\[-12pt]
\hspace*{23pt}variation problem ICP for an affine transformation&1&101--112\\
\Avtors{Yadrintsev~V.\,V.} see Budzko~V.\,I.&&\\
\Avtors{Yarkina~N.\,V.} see Ageev~K.\,A.&&\\
\Avtors{Zabezhailo~M.\,I.} see Grusho~A.\,A.&&\\
\Avtors{Zabezhailo~M.\,I.} see Grusho~A.\,A.&&\\
\Avtors{Zabezhailo~M.\,I.} see Grusho~N.\,A.&&\\
\Avtors{Zabezhailo~M.\,I.} see Grusho~N.\,A.&&\\
\Avtors{Zakharov~V.\,N.} see Frenkel~S.\,L.&&\\
\Avtors{Zatsman~I.\,M.} Problem-oriented verifying the completeness  of~temporal ontologies and\linebreak
\\[-12pt]
\hspace*{23pt}filling~conceptual lacunas&3&119--128\\
\Avtors{Zatsman~I.\,M.} see Goncharov~A.\,A.&&\\
\Avtors{Zatsman~I.\,M.} see Nuriev~V.\,A.&&\\
\Avtors{Zeifman~A.\,I.} see Satin~Ya.\,A.&&\\
\Avtors{Zholobov~V.\,A.} see Potanin~M.\,S.&&\\
\end{tabular}
}

%\thispagestyle{myheadings}
\def\leftfootline{\small{\textbf{\thepage}
\hfill INFORMATIKA I EE PRIMENENIYA~--- INFORMATICS AND APPLICATIONS\ \ \ 2020\
\ \ volume~14\ \ \ issue\ 4}
}%
 \def\rightfootline{\small{INFORMATIKA I EE PRIMENENIYA~---
INFORMATICS AND APPLICATIONS\ \ \ 2020\ \ \ volume~14\ \ \ issue\ 4
\hfill \textbf{\thepage}}}

 \label{end\stat}

\newpage


%\linebreak
%\\[-12pt]
%\hspace*{23pt}

%   \vspace*{-48pt}

\begin{center}
\vspace*{6pt}
\mbox{%
%\epsfxsize=50mm %56.519mm  
%\epsfbox{smu-1.eps} 

\epsfxsize=50mm %46.402 mm
\epsfbox{nec-rb.eps}
}
%\end{center}

\vspace*{9pt} %Академик


%   \begin{center}
\fbox{\large\textbf{Рустем Бадриевич Сейфуль-Мулюков}}\\[6pt]
\textbf{\large 1928--2020}
   \end{center}


   %\vspace*{2.5mm}

   \vspace*{5mm}

   \thispagestyle{empty}

%\

%\vspace*{-12pt}

  
      Редакция журнала <<Информатика и~её применения>> с глубоким 
      прискорбием сообщают, что 17~марта 2020~г.\ на 93-м~году жизни 
      скончался заведующий редакцией журнала, главный научный сотрудник Федерального исследовательского центра <<Информатика и~управление>> Российской академии наук
      Рустем Бадриевич Сейфуль-Мулюков.
           
     Всю свою жизнь Рустем Бадриевич посвятил служению науке. Закончив в~1956~г.\ аспирантуру Московского ордена Трудового Красного знамени Нефтяного института им.\ академика
     И.\,М.~Губкина, он прошел путь от заведующего отделом Института геологии зарубежных стран Министерства геологии СССР до заместителя директора ВИНИТИ
     АН СССР, доктора гео\-ло\-го-ми\-не\-ра\-ло\-ги\-че\-ских наук, профессора.
     
     С марта 2002~г.\ Рустем Бадриевич успешно применял свои знания и~организационный талант в ИПИ
     РАН (в~дальнейшем~--- ФИЦ ИУ РАН), в~котором руководил лабораторией и~отделом, занимающимися вопросами технологий информационной технической деятельности. 
Р.\,Б.~Сейфуль-Мулюков, являясь автором значительного количества научных трудов и~монографий по геологии, информационным технологиям и~теоретической информатике, осуществлял организацию издания монографий ИПИ РАН и~ФИЦ ИУ РАН, библиографий научных сотрудников Центра.
     
     Р.\,Б.~Сейфуль-Мулюков являлся заведующим редакцией журналов <<Информатика и~её применения>> и~<<Системы и~средства информатики>>, членом редколлегии журнала <<Системы и~средства информатики>>. Он вложил огромный вклад в становление и~развитие этих журналов, организацию их регистрации, функционирования, редактуры и~издания. Включение этих журналов в ряд отечественных и~зарубежных информационных баз и~систем цитирования во многом является его личной заслугой.
     
     На всех занимаемых должностях Рустем Бадриевич отличался высоким профессионализмом, преданностью делу и~вниманием к коллегам.
     
     \thispagestyle{empty}
     
     Рустема Бадриевича отличали доброта, отзывчивость, неиссякаемый
      оптимизм, простота и~сердечность.
     
     Коллеги Рустема Бадриевича запомнят его как многогранного в~своих увлечениях человека, живописца,
     эрудита и~энциклопедиста, интересующегося историей, литературой и~искусством.
     
     Выражаем глубокое
     соболезнование семье, родственникам, друзьям и~коллегам по работе в~связи с~тяжелой невосполнимой утратой.
     Светлый образ Рустема Бадриевича навсегда сохранится в~нашей памяти.
     

      

%\def\stat{cont}
{%\hrule\par
%\vskip 7pt % 7pt
\raggedleft\Large \bf%\baselineskip=3.2ex
А\,В\,Т\,О\,Р\,С\,К\,И\,Й\ \ У\,К\,А\,З\,А\,Т\,Е\,Л\,Ь\ \ З\,А\ \ 2\,0\,1\,0 г. \vskip 17pt
    \hrule
    \par
\vskip 21pt plus 6pt minus 3pt }

\label{st\stat}

\def\tit{\ }

\def\aut{\ }
\def\auf{\ }

\def\leftkol{\ } % ENGLISH ABSTRACTS}

\def\rightkol{\ } %АВТОРСКИЙ УКАЗАТЕЛЬ ЗА 2010 г.} %ENGLISH ABSTRACTS}

\titele{\tit}{\aut}{\auf}{\leftkol}{\rightkol}

\vspace*{-12pt}

{\tabcolsep=3pt
\begin{tabular}{p{388pt}rr}
&\textbf{Выпуск} & \textbf{Стр.}\\[6pt]
\hangindent=23pt\noindent\textbf{Арутюнян~А.\,Р.} Моделирование влияния деформаций отпечатков пальцев на 
точность\linebreak
\vspace*{-12pt}\\
\hspace*{23pt}дактилоскопической идентификации$\dotfill$&1&51\\
\hangindent=23pt\noindent\textbf{Архипов~О.\,П., Зыкова~З.\,П.} Интеграция гетерогенной информации о цветных 
пикселях\linebreak
\vspace*{-12pt}\\
\hspace*{23pt}и их цветовосприятии$\dotfill$&4&15\\
\hangindent=23pt\noindent\textbf{Баранов~С.\,И., Френкель~С.\,Л., Захаров~В.\,Н.} Полуформальная верификация 
цифрового устройства с конвейером, основанная на использовании алгоритмических машин\linebreak
\vspace*{-12pt}\\
\hspace*{23pt}состояния$\dotfill$&4&49\\
\textbf{Бекетова~И.\,В.} см.~Каратеев~С.\,Л.&&\\
\textbf{Белоусов~В.\,В.} см.~Синицын~И.\,Н.&&\\
\hangindent=23pt\noindent\textbf{Бенинг~В.\,Е., Королев~Р.\,А.} О предельном поведении мощностей критериев в 
случае\linebreak
\vspace*{-12pt}\\
\hspace*{23pt}распределения Лапласа$\dotfill$&2&63\\
\hangindent=23pt\noindent\textbf{Бенинг~В.\,Е., Сипина~А.\,В.} Асимптотическое разложение для мощности 
критерия,\linebreak
\vspace*{-12pt}\\
\hspace*{23pt}основанного на выборочной медиане, в случае распределения Лапласа$\dotfill$&1&18\\
\textbf{Бондаренко~А.\,В.} см.~Каратеев~С.\,Л.&&\\
\hangindent=23pt\noindent\textbf{Бородина~А.\,В., Морозов~Е.\,В.} Об оценивании асимптотики вероятности 
большого\linebreak
\vspace*{-12pt}\\
\hspace*{23pt}уклонения стационарной регенеративной очереди с одним прибором$\dotfill$&3&29\\
\hangindent=23pt\noindent\textbf{Бунтман~Н.\,В., Минель~Ж.-Л., Ле~Пезан~Д., Зацман~И.\,М.} Типология и 
компьютерное\linebreak
\vspace*{-12pt}\\
\hspace*{23pt}моделирование трудностей перевода$\dotfill$&3&77\\
\textbf{Визильтер~Ю.\,В.} см.~Каратеев~С.\,Л.&&\\
\hangindent=23pt\noindent\textbf{Гавриленко~С.\,В.} Оценки скорости сходимости распределений случайных сумм с 
безгранично делимыми индексами к нормальному закону$\dotfill$&4&81\\
\hangindent=23pt\noindent\textbf{Григорьева~М.\,Е., Шевцова~И.\,Г.} Уточнение неравенства 
Каца--Берри--Эссеена$\dotfill$&2&75\\
\hangindent=23pt\noindent\textbf{Грушо~А.\,А., Грушо~Н.\,А., Тимонина~Е.\,Е.} Поиск конфликтов в политиках 
безопасности: модель случайных графов$\dotfill$&3&38\\
\textbf{Грушо~Н.\,А.} см.~Грушо~А.\,А.&&\\
\hangindent=23pt\noindent\textbf{Гудков~В.\,Ю.} Математические модели изображения отпечатка пальца на основе 
описания линий$\dotfill$&1&58\\
\textbf{Гуртов~А.\,В.} см.~Лукьяненко~А.\,С.&&\\
\textbf{Желтов~С.\,Ю.} см.~Каратеев~С.\,Л.&&\\
\hangindent=23pt\noindent\textbf{Захаров~А.\,А., Серебряков~В.\,А.} Система управления электронной библиотекой 
LibMeta$\dotfill$&4&2\\
\textbf{Захаров~В.\,Н.} см.~Баранов~С.\,И.&&\\
\textbf{Захарова~Т.\,В.} см.~Матвеева~С.\,С.&&\\
\hangindent=23pt\noindent\textbf{Зацаринный~А.\,А., Чупраков~К.\,Г.} Некоторые аспекты выбора технологии для 
постро-\linebreak
\vspace*{-12pt}\\
\hspace*{23pt}ения систем отображения информации ситуационного центра$\dotfill$&3&59\\
\textbf{Зацман~И.\,М.} см.~Бунтман~Н.\,В.&&\\
\hangindent=23pt\noindent\textbf{Зейфман~А.\,И., Коротышева~А.\,В., Сатин~Я.\,А., Шоргин~С.\,Я.} Об 
устойчивости нестаци-\linebreak
\vspace*{-12pt}\\
\hspace*{23pt}онарных систем обслуживания с катастрофами$\dotfill$&3&9\\
\textbf{Зыкова~З.\,П.} см.~Архипов~О.\,П.&&\\
\hangindent=23pt\noindent\textbf{Илюшин~Г.\,Я., Соколов~И.\,А.} Организация управляемого доступа пользователей 
к\linebreak
\vspace*{-12pt}\\
\hspace*{23pt}разнородным ведомственным информационным ресурсам$\dotfill$&1&24\\
\hangindent=23pt\noindent\textbf{Кавагучи~Ю., Ульянов~В.\,В., Фуджикоши~Я.} Приближения для статистик, 
описывающих\linebreak
\vspace*{-12pt}\\
\hspace*{23pt}геометрические свойства данных большой размерности, с оценками 
ошибок$\dotfill$&1&12\\
\hangindent=23pt\noindent\textbf{Каратеев~С.\,Л., Бекетова~И.\,В., Ососков~М.\,В., Князь~В.\,А., 
Визильтер~Ю.\,В., Бондаренко~А.\,В., Желтов~С.\,Ю.} Автоматизированный контроль 
качества цифровых\linebreak
\vspace*{-12pt}\\
\hspace*{23pt}изображений для персональных документов$\dotfill$&1&65\\
\end{tabular}
}

\pagebreak

\def\leftkol{АВТОРСКИЙ УКАЗАТЕЛЬ ЗА 2010 г.} % ENGLISH ABSTRACTS}

\def\rightkol{АВТОРСКИЙ УКАЗАТЕЛЬ ЗА 2010 г.} %ENGLISH ABSTRACTS}

{\tabcolsep=3pt
\begin{tabular}{p{388pt}rr}
&\textbf{Выпуск} & \textbf{Стр.}\\[3pt]
\hangindent=23pt\noindent\textbf{Козеренко~Е.\,Б.} Лингвистические фильтры в статистических моделях машинного\linebreak
\vspace*{-12pt}\\
\hspace*{23pt}перевода$\dotfill$&2&83\\
\hangindent=23pt\noindent\textbf{Козеренко~Е.\,Б., Кузнецов~И.\,П.} Когнитивно-лингвистические представления в 
систе-\linebreak
\vspace*{-12pt}\\
\hspace*{23pt}мах обработки текстов$\dotfill$&3&69\\
\textbf{Князь~В.\,А.} см.~Каратеев~С.\,Л.&&\\
\hangindent=23pt\noindent\textbf{Колесников~А.\,В., Солдатов~С.\,А.} Алгоритм координации для гибридной 
интеллектуальной системы решения сложной задачи оперативно-производственного\linebreak
\vspace*{-12pt}\\
\hspace*{23pt}планирования$\dotfill$&4&61\\
\hangindent=23pt\noindent\textbf{Коновалов~М.\,Г.} О планировании потоков в системах вычислительных 
ресурсов$\dotfill$&2&3\\
\textbf{Конушин~А.\,С.} см.~Конушин~В.\,С.&&\\
\hangindent=23pt\noindent\textbf{Конушин~В.\,С., Кривовязь~Г.\,Р., Конушин~А.\,С.} Алгоритм распознавания людей 
в видео-\linebreak
\vspace*{-12pt}\\
\hspace*{23pt}последовательности по одежде$\dotfill$&1&74\\
\textbf{Корепанов~Э.\, Р.} см.~Синицын~И.\,Н.&&\\
\textbf{Королев~В.\,Ю.} см.~Соколов~И.\,А.&&\\
\textbf{Королев~Р.\,А.} см.~Бенинг~В.\,Е.&&\\
\textbf{Коротышева~А.\,В.} см.~Зейфман~А.\,И.&&\\
\hangindent=23pt\noindent\textbf{Кривенко~М.\,П.} Непараметрическое оценивание элементов байесовского 
клас\-си-\linebreak
\vspace*{-12pt}\\
\hspace*{23pt}фикатора$\dotfill$&2&13\\
\textbf{Кривовязь~Г.\,Р.} см.~Конушин~В.\,С.&&\\
\textbf{Крылов~А.\,С.} см.~Павельева~Е.\,А.&&\\
\hangindent=23pt\noindent\textbf{Крылов~В.\,А.} Моделирование и классификация многоканальных дистанционных\linebreak
\vspace*{-12pt}\\
\hspace*{23pt}изображений с использованием копул$\dotfill$&4&34\\
\hangindent=23pt\noindent\textbf{Крючин~О.\,В.} Разработка параллельных эвристических алгоритмов подбора 
весовых\linebreak
\vspace*{-12pt}\\
\hspace*{23pt}коэффициентов искусственной нейтронной сети$\dotfill$&2&53\\
\hangindent=23pt\noindent\textbf{Кудрявцев~А.\,А., Шоргин~С.\,Я.} Байесовские модели массового обслуживания и 
надеж-\linebreak
\vspace*{-12pt}\\
\hspace*{23pt}ности: характеристики среднего числа заявок в системе $M\vert M \vert 1\vert 
\infty$$\dotfill$&3&16\\
\hangindent=23pt\noindent\textbf{Кузнецов~А.\,А.} Связь между временными и структурно-топологическими 
характери-\linebreak
\vspace*{-12pt}\\
\hspace*{23pt}стиками диаграмм ритма сердца здоровых людей$\dotfill$&4&39\\
\textbf{Кузнецов~И.\,П.} см.~Козеренко~Е.\,Б.&&\\
\textbf{Ле~Пезан~Д.} см.~Бунтман~Н.\,В.&&\\
\hangindent=23pt\noindent\textbf{Лукьяненко~А.\,С., Морозов~Е.\,В., Гуртов~А.\,В.} Анализ сетевого протокола с общей 
функ-\linebreak
\vspace*{-12pt}\\
\hspace*{23pt}цией расширения окна передачи сообщения при конфликтах$\dotfill$&2&46\\
\hangindent=23pt\noindent\textbf{Лямин~О.\,О.} О предельном поведении мощностей критериев в случае обобщенного\linebreak
\vspace*{-12pt}\\
\hspace*{23pt}распределения Лапласа$\dotfill$&3&47\\
\hangindent=23pt\noindent\textbf{Маркин~А.\,В., Шестаков~О.\,В.} Асимптотики оценки риска при пороговой 
обработке\linebreak
\vspace*{-12pt}\\
\hspace*{23pt}вейвлет-вейглет коэффициентов в задаче томографии$\dotfill$&2&36\\
\hangindent=23pt\noindent\textbf{Матвеева~С.\,С., Захарова~Т.\,В.} Сети массового обслуживания с наименьшей 
длиной\linebreak
\vspace*{-12pt}\\
\hspace*{23pt}очереди$\dotfill$&3&22\\
\hangindent=23pt\noindent\textbf{Матюшенко~С.\,И.} Стационарные характеристики двухканальной системы 
обслужива-\linebreak
\vspace*{-12pt}\\
\hspace*{23pt}ния с переупорядочиванием заявок и распределениями фазового типа$\dotfill$&4&68\\
\textbf{Минель~Ж.-Л.} см.~Бунтман~Н.\,В.&&\\
\textbf{Морозов~Е.\,В.} см.~Бородина~А.\,В.&&\\
\textbf{Морозов~Е.\,В.} см.~Лукьяненко~А.\,С.&&\\
\textbf{Ососков~М.\,В.} см.~Каратеев~С.\,Л.&&\\
\hangindent=23pt\noindent\textbf{Павельева~Е.\,А., Крылов~А.\,С.} Поиск и анализ ключевых точек радужной 
оболочки\linebreak
\vspace*{-12pt}\\
\hspace*{23pt}глаза методом преобразования Эрмита$\dotfill$&1&79\\
\textbf{Печинкин~А.\,В.} см.~Френкель~С.\,Л.,&&\\
\hangindent=23pt\noindent\textbf{Протасов~В.\,И.} Составление субъективного портрета с использованием 
эволюционно-\linebreak
\vspace*{-12pt}\\
\hspace*{23pt}го морфинга и квалиметрия метода$\dotfill$&1&83\\
\hangindent=23pt\noindent\textbf{Рудаков~К.\,В., Торшин~И.\,Ю.} Вопросы разрешимости задачи распознавания 
вторичной\linebreak
\vspace*{-12pt}\\
\hspace*{23pt}структуры белка$\dotfill$&2&25\\
\textbf{Сатин~Я.\,А.} см.~Зейфман~А.\,И.&&\\
\hangindent=23pt\noindent\textbf{Сейфуль-Мулюков~Р.\,Б.} Нефть как носитель информации о своем 
происхождении,\linebreak
\vspace*{-12pt}\\
\hspace*{23pt}структуре и эволюции$\dotfill$&1&41\\
\end{tabular}
}

{\tabcolsep=3pt
\begin{tabular}{p{388pt}rr}
&\textbf{Выпуск} & \textbf{Стр.}\\[6pt]
\textbf{Семендяев~Н.\,Н.} см.~Синицын~И.\,Н.&&\\
\textbf{Серебряков~В.\,А.} см.~Захаров~А.\,А.&&\\
\textbf{Синицын~В.\,И.} см.~Синицын~И.\,Н.&&\\
\hangindent=23pt\noindent\textbf{Синицын~И.\,Н., Синицын~В.\,И., Корепанов~Э.\, Р., Белоусов~В.\,В., 
Семендяев~Н.\,Н.} Оперативное построение информационных моделей движения полюса 
Земли\linebreak
\vspace*{-12pt}\\
\hspace*{23pt}методами линейных и линеаризованных фильтров$\dotfill$&1&2\\
\textbf{Сипина~А.\,В.} см.~Бенинг~В.\,Е.&&\\
\hangindent=23pt\noindent\textbf{Соколов~И.\,А.} О работах заслуженного деятеля науки Российской Федерации 
И.\,Н.~Синицына в области информационных технологий и автоматизации (к 70-летию\linebreak
\vspace*{-12pt}\\
\hspace*{23pt}со дня рождения)$\dotfill$&3&84\\
\textbf{Соколов~И.\,А.} см.~Илюшин~Г.\,Я.&&\\
\hangindent=23pt\noindent\textbf{Соколов~И.\,А., Королев~В.\,Ю.} Предисловие$\dotfill$&2&2\\
\textbf{Солдатов~С.\,А.} см.~Колесников~А.\,В.&&\\
\hangindent=23pt\noindent\textbf{Степанов~С.\,Ю.} Использование координатного метода фрагментации 
коммутаторной\linebreak
\vspace*{-12pt}\\
\hspace*{23pt}нейронной сети для сокращения трафика$\dotfill$&2&57\\
\textbf{Тимонина~Е.\,Е.} см.~Грушо~А.\,А.&&\\
\textbf{Торшин~И.\,Ю.} см.~Рудаков~К.\,В.&&\\
\textbf{Ульянов~В.\,В.} см.~Кавагучи~Ю.&&\\
\textbf{Фазекаш~И.} см.~Чупрунов~А.\,Н.&&\\
\textbf{Френкель~С.\,Л.} см.~Баранов~С.\,И.&&\\
\hangindent=23pt\noindent\textbf{Френкель~С.\,Л., Печинкин~А.\,В.} Оценка времени самовосстановления в 
цифровых\linebreak
\vspace*{-12pt}\\
\hspace*{23pt}системах после сбоев, вызываемых переходными помехами$\dotfill$&3&2\\
\textbf{Фуджикоши~Я.} см.~Кавагучи~Ю.&&\\
\hangindent=23pt\noindent\textbf{Цискаридзе~А.\,К.} Математическая модель и метод восстановления позы человека 
по\linebreak
\vspace*{-12pt}\\
\hspace*{23pt}стереопаре силуэтных изображений$\dotfill$&4&27\\
\hangindent=23pt\noindent\textbf{Чупраков~К.\,Г.} К вопросу о размещении коллективных средств отображения в 
ситуа-\linebreak
\vspace*{-12pt}\\
\hspace*{23pt}ционном зале с заданными параметрами$\dotfill$&4&89\\
\textbf{Чупраков~К.\,Г.} см.~Зацаринный~А.\,А.&&\\
\hangindent=23pt\noindent\textbf{Чупрунов~А.\,Н., Фазекаш~И.} Законы повторного логарифма для числа 
безошибочных\linebreak
\vspace*{-12pt}\\
\hspace*{23pt}блоков при помехоустойчивом кодировании$\dotfill$&3&42\\
\textbf{Шевцова~И.\,Г.} см.~Григорьева~М.\,Е.&&\\
\hangindent=23pt\noindent\textbf{Шестаков~О.\,В.} Аппроксимация распределения оценки риска пороговой 
обработки вейвлет-коэффициентов нормальным распределением при использовании 
выбо-\linebreak
\vspace*{-12pt}\\
\hspace*{23pt}рочной дисперсии$\dotfill$&4&73\\
\textbf{Шестаков~О.\,В.} см.~Маркин~А.\,В.&&\\
\textbf{Шоргин~С.\,Я.} см.~Зейфман~А.\,И.&&\\
\textbf{Шоргин~С.\,Я.} см.~Кудрявцев~А.\,А.&&\\
\end{tabular}
}

%\thispagestyle{myheadings}
\def\leftfootline{\small{\textbf{\thepage}
\hfill ИНФОРМАТИКА И ЕЁ ПРИМЕНЕНИЯ\ \ \ том~4\ \ \ выпуск~4\ \ \ 2010}
}%
 \def\rightfootline{\small{ИНФОРМАТИКА И ЕЁ ПРИМЕНЕНИЯ\ \ \ том~4\ \ \ выпуск~4\ \ \ 2010
 \hfill \textbf{\thepage}}}
 \label{end\stat}
%
%Том 10 Выпуск 1-4 Год 2016

\def\stat{cont-e}
{%\hrule\par
%\vskip 7pt % 7pt
\raggedleft\Large \bf%\baselineskip=3.2ex
2\,0\,1\,6\ \ A\,U\,T\,H\,O\,R\ \ I\,N\,D\,E\,X \vskip 17pt
 \hrule
 \par
\vskip 21pt plus 6pt minus 3pt }

\label{st\stat}

\def\tit{\ }

\def\aut{\ }
\def\auf{\ }

\def\leftkol{\ } %2016 AUTHOR INDEX} % ENGLISH ABSTRACTS}

\def\rightkol{\ } %2016 AUTHOR INDEX} %ENGLISH ABSTRACTS}

\titele{\tit}{\aut}{\auf}{\leftkol}{\rightkol}

\def\leftfootline{\small{\textbf{\thepage}
\hfill INFORMATIKA I EE PRIMENENIYA~--- INFORMATICS AND APPLICATIONS\ \ \ 2016\
\ \ volume~10\ \ \ issue\ 4}
}%
 \def\rightfootline{\small{INFORMATIKA I EE PRIMENENIYA~--- INFORMATICS AND APPLICATIONS\ \ \ 2016\ \ \ volume~10\ \ \ issue\ 4
\hfill \textbf{\thepage}}}

\vspace*{-12pt}
\vspace*{-18pt}

{\tabcolsep=2.8pt
\begin{tabular}{p{382pt}cc}
&\textbf{Issue} & \textbf{Page}\\[6pt]
\Avtors{Agalarov~M.\,Ya.} see~Agalarov~Ya.\,M.&&\\
\Avtors{Agalarov~Ya.\,M., Agalarov~M.\,Ya., and
Shorgin~V.\,S.} About the optimal threshold of queue\linebreak
\\[-12pt]
\hspace*{23pt}length in a~particular problem of profit maximization
in the $M/G/1$ queuing system&2&70--79\\
\Avtors{Alexeyevsky~D.\,A.} BioNLP ontology extraction from 
a~restricted language corpus with\linebreak
\\[-12pt]
\hspace*{23pt}context-free grammars&1&119--128\\
\Avtors{Andreev~S.\,D.} see~Gaidamaka~Yu.\,V.&&\\
\Avtors{Andreev~S.\,D.} see~Ometov~A.\,Ya.&&\\
\Avtors{Arkhipov~O.\,P., Arkhipov~P.\,O., and Sidorkin~I.\,I.} The
option to create a~local coordinate\linebreak
\\[-12pt]
\hspace*{23pt}system for synchronization of selected images&3&91--97\\
\Avtors{Arkhipov~P.\,O.} see~Arkhipov~O.\,P.&&\\
\Avtors{Belousov~V.\,V.} see~Shnurkov~P.\,V.&&\\
\Avtors{Belousov~V.\,V.} see~Shnurkov~P.\,V.&&\\
\Avtors{Bening~V.\,E.} Calculation of~the~asymptotic deficiency
of~some statistical procedures based\linebreak
\\[-12pt]
\hspace*{23pt}on~samples with~random sizes&4&34--45\\
\Avtors{Borisov~A.\,V., Bosov~A.\,V., and Miller~G.\,B.} Modeling and
monitoring of VoIP connection&2&\hphantom{1}2--13\\
\Avtors{Bosov~A.\,V.} see~Borisov~A.\,V.&&\\
\Avtors{Briukhov~D.\,O.} see~Stupnikov~S.\,A.&&\\
\Avtors{Callaos~N.\,K.\ and Seyful-Mulyukov~R.\,B.} Complexity and
its information content&1&129--139\\
\Avtors{Chertok~A.\,V., Kadaner~A.\,I., Khazeeva~G.\,T., and
Sokolov~I.\,A.} Regime switching detection\linebreak
\\[-12pt]
\hspace*{23pt}for~the~Levy driven
Ornstein--Uhlenbeck process using CUSUM methods&4&46--56\\
\Avtors{Chichagov~V.\,V.} Asymptotic expansions of mean absolute
error of uniformly minimum variance unbiased and maximum likelihood
estimators on the one-parameter exponential\linebreak
\\[-12pt]
\hspace*{23pt}family model of lattice distributions&3&66--76\\
\Avtors{Danishevsky~V.\,I.} see~Kolesnikov A.\,V.&&\\
\Avtors{Fazliev~A.\,Z.} see~Kalinichenko~L.\,A.&&\\
\Avtors{Fedoseev~A.\,A.} What is behind the concept of ``knowledge in
small packages''&3&105--110\\
\Avtors{Gaidamaka~Yu.\,V., Andreev~S.\,D., Sopin~E.\,S.,
Samouylov~K.\,E., and Shorgin~S.\,Ya.} Interference analysis
of~the~device-to-device communications model with~regard to~a~signal\linebreak
\\[-12pt]
\hspace*{23pt}propagation environment&4&\hphantom{1}2--10\\
\Avtors{Gasilov~A.\,V.} see~Yakovlev~O.\,A.&&\\
\Avtors{Goncharov~A.\,V.\ and Strijov~V.\,V.} Metric time series
classification using weighted dynamic\linebreak
\\[-12pt]
\hspace*{23pt}warping relative to centroids of classes&2&36--47\\
\Avtors{Gordov~E.\,P.} see~Kalinichenko~L.\,A.&&\\
\Avtors{Gorshenin~A.\,K.} Concept of online service for stochastic
modeling of real processes&1&72--81\\
\Avtors{Gorshenin~A.\,K.} see~Shnurkov~P.\,V.&&\\
\Avtors{Gorshenin~A.\,K.} see~Shnurkov~P.\,V.&&\\
\Avtors{Grusho~A.\,A., Grusho~N.\,A., Zabezhailo~M.\,I., and
Timonina~E.\,E.} Integration of statistical and\linebreak
\\[-12pt]
\hspace*{23pt}deterministic methods for
analysis of information security&3&2--8\\
\Avtors{Grusho~A.\,A., Zabezhailo~M.\,I., and Zatsarinny~A.\,A.} On
the advanced procedure to reduce\linebreak
\\[-12pt]
\hspace*{23pt}calculation of Galois closures&4&\hphantom{1}96--104\\
\Avtors{Grusho~N.\,A.} see~Grusho~A.\,A.&&\\
\Avtors{Havanskov~V.\,A.} see~Minin~V.\,A.&&\\
\Avtors{Inkova~O.\,Yu.} see~Zatsman~I.\,M.&&\\
\Avtors{Isachenko~R.\,V.\ and Strijov~V.\,V.} Metric learning in
multiclass time series classification\linebreak
\\[-12pt]
\hspace*{23pt}problem&2&48--57\\
\end{tabular}
}
\pagebreak

\def\leftfootline{\small{\textbf{\thepage}
\hfill INFORMATIKA I EE PRIMENENIYA~--- INFORMATICS AND APPLICATIONS\ \ \ 2016\
\ \ volume~10\ \ \ issue\ 4}
}%
 \def\rightfootline{\small{INFORMATIKA I EE PRIMENENIYA~---
INFORMATICS AND APPLICATIONS\ \ \ 2016\ \ \ volume~10\ \ \ issue\ 4
\hfill \textbf{\thepage}}}

\def\leftkol{2016 AUTHOR INDEX} % ENGLISH ABSTRACTS}

\def\rightkol{2016 AUTHOR INDEX} %ENGLISH ABSTRACTS}


{\tabcolsep=2.83pt
\begin{tabular}{p{382pt}cc}
&\textbf{Issue} & \textbf{Page}\\[6pt]
\Avtors{Kadaner~A.\,I.} see~Chertok~A.\,V.&&\\[.255pt]
\Avtors{Kalinichenko~L.\,A., Volnova~A.\,A., Gordov~E.\,P.,
Kiselyova~N.\,N., Kovaleva~D.\,A., Malkov~O.\,Yu., Okladnikov~I.\,G.,
Podkolodnyy~N.\,L., Pozanenko~A.\,S., Ponomareva~N.\,V.,
Stupnikov~S.\,A.,} \textbf{and Fazliev~A.\,Z.} Data access challenges for data
intensive\linebreak
\\[-12pt]
\hspace*{23pt}research in Russia&1& 2--22\\[.255pt]
\Avtors{Karasikov~M.\,E.\ and Strijov~V.\,V.} Feature-based
time-series classification&4&121--131\\[.255pt]
\Avtors{Khazeeva~G.\,T.} see~Chertok~A.\,V.&&\\[.255pt]
\Avtors{Khokhlov~Yu.\,S.} Multivariate fractional Levy motion and its
applications&2&\hphantom{1}98--106\\[.255pt]
\Avtors{Kirikov~I.\,A., Kolesnikov~A.\,V., Listopad~S.\,V., and
Rumovskaya~S.\,B.} Fine-grained hybrid\linebreak
\\[-12pt]
\hspace*{23pt}intelligent systems. Part 2:
Bidirectional hybridization&1&\hphantom{1}96--105\\[.255pt]
\Avtors{Kirikov~I.\,A., Kolesnikov~A.\,V., Listopad~S.\,V., and
Rumovskaya~S.\,B.} ``Virtual council''~---\linebreak
\\[-12pt]
\hspace*{23pt}source environment
supporting complex diagnostic decision making&3&81--90\\[.255pt]
\Avtors{Kiselyova~N.\,N.} see~Kalinichenko~L.\,A.&&\\[.255pt]
\Avtors{Kolesnikov A.\,V., Listopad~S.\,V., Rumovskaya~S.\,B., and
Danishevsky~V.\,I.} Informal axiomatic\linebreak
\\[-12pt]
\hspace*{23pt}theory of~the~role visual models&4&114--120\\[.255pt]
\Avtors{Kolesnikov~A.\,V.} see~Kirikov~I.\,A.&&\\[.255pt]
\Avtors{Kolesnikov~A.\,V.} see~Kirikov~I.\,A.&&\\[.255pt]
\Avtors{Kolin~K.\,K.} Humanitarian aspects of information
security&3&111--121\\[.255pt]
\Avtors{Konovalov~M.\,G.\ and Razumchik~R.\,V.} Dispatching
to~two parallel nonobservable queues using\linebreak
\\[-12pt]
\hspace*{23pt}only static
information&4&57--67\\[.255pt]
\Avtors{Korchagin~A.\,Yu.} see~Korolev~V.\,Yu.&&\\[.255pt]
\Avtors{Korchagin~A.\,Yu.} see~Korolev~V.\,Yu.&&\\[.255pt]
\Avtors{Korepanov~E.\,R.} see~Sinitsyn~I.\,N.&&\\[.255pt]
\Avtors{Korepanov~E.\,R.} see~Sinitsyn~I.\,N.&&\\[.255pt]
\Avtors{Korolev~V.\,Yu., Korchagin~A.\,Yu., and Zeifman~A.\,I.} The
Poisson theorem for Bernoulli trials\linebreak
\\[-12pt]
\hspace*{23pt}with~a~random probability
of~success and~a~discrete analog of~the~Weibull distribution&4&11--20\\[.255pt]
\Avtors{Korolev~V.\,Yu., Zeifman~A.\,I., and Korchagin~A.\,Yu.}
Asymmetric Linnik distributions as~limit\linebreak
\\[-12pt]
\hspace*{23pt}laws for~random sums
of~independent random variables with~finite variances&4&21--33\\[.255pt]
\Avtors{Koucheryavy~E.\,A.} see~Ometov~A.\,Ya.&&\\[.255pt]
\Avtors{Kovaleva~D.\,A.} see~Kalinichenko~L.\,A.&&\\[.255pt]
\Avtors{Kovalyov~S.\,P.} Metaprogramming to increase
manufacturability of large-scale software-\linebreak
\\[-12pt]
\hspace*{23pt}intensive systems&1&56--66\\[.255pt]
\Avtors{Krivenko~M.\,P.} Significance tests of feature selection for
classification&3&32--40\\[.255pt]
\Avtors{Kruzhkov~M.\,G.} see~Zalizniak~Anna~A.&&\\[.255pt]
\Avtors{Kruzhkov~M.\,G.} see~Zatsman~I.\,M.&&\\[.255pt]
\Avtors{Kudryavtsev~A.\,A.} Bayesian queueing and reliability models:
\textit{A~priori} distributions with\linebreak
\\[-12pt]
\hspace*{23pt}compact support&1&67--71\\[.255pt]
\Avtors{Kudryavtsev~A.\,A.} Characteristics dependent on the balance
coefficient in Bayesian models\linebreak
\\[-12pt]
\hspace*{23pt}with compact support of \textit{a priori}
distributions&3&77--80\\[.255pt]
\Avtors{Kudryavtsev~A.\,A.\ and Palionnaia~S.\,I.} Bayesian recurrent
model of reliability growth:\linebreak
\\[-12pt]
\hspace*{23pt}Parabolic distribution of parameters&2&80--83\\[.255pt]
\Avtors{Kudryavtsev~A.\,A.\ and Titova~A.\,I.} Bayesian queuing
and~reliability models: Degenerate-\linebreak
\\[-12pt]
\hspace*{23pt}Weibull case&4&68--71\\[.255pt]
\Avtors{Leontyev~N.\,D.\ and Ushakov~V.\,G.} Analysis of a queueing
system with autoregressive arrivals\linebreak
\\[-12pt]
\hspace*{23pt}and nonpreemptive priority&3&15--22\\[.255pt]
\Avtors{Listopad~S.\,V.} see~Kirikov~I.\,A.&&\\[.255pt]
\Avtors{Listopad~S.\,V.} see~Kirikov~I.\,A.&&\\[.255pt]
\Avtors{Listopad~S.\,V.} see~Kolesnikov A.\,V.&&\\[.255pt]
\Avtors{Malkov~O.\,Yu.} see~Kalinichenko~L.\,A.&&\\[.255pt]
\Avtors{Markov~A.\,S., Monakhov~M.\,M., and
Ulyanov~V.\,V.} Generalized Cornish--Fisher expansions\linebreak
\\[-12pt]
\hspace*{23pt}for distributions of statistics based on samples
of random size&2&84--91\\[.255pt]
\Avtors{Melnikov~A.\,K.\ and Ronzhin~A.\,F.} Generalized statistical
method of~text analysis based\linebreak
\\[-12pt]
\hspace*{23pt}on~calculation of~probability distributions
of~statistical values&4&89--95\\
\end{tabular}
}
\pagebreak

\def\leftfootline{\small{\textbf{\thepage}
\hfill INFORMATIKA I EE PRIMENENIYA~--- INFORMATICS AND APPLICATIONS\ \ \ 2016\
\ \ volume~10\ \ \ issue\ 4}
}%
 \def\rightfootline{\small{INFORMATIKA I EE PRIMENENIYA~---
INFORMATICS AND APPLICATIONS\ \ \ 2016\ \ \ volume~10\ \ \ issue\ 4
\hfill \textbf{\thepage}}}

\def\leftkol{2016 AUTHOR INDEX} % ENGLISH ABSTRACTS}

\def\rightkol{2016 AUTHOR INDEX} %ENGLISH ABSTRACTS}


{\tabcolsep=3pt
\begin{tabular}{p{381pt}cc}
&\textbf{Issue} & \textbf{Page}\\[6pt]
\Avtors{Meykhanadzhyan~L.\,A.} Stationary characteristics of the finite
capacity queueing system with\linebreak
\\[-12pt]
\hspace*{23pt}inverse service order and generalized
probabilistic priority&2&123--131\\[.23pt]
\Avtors{Miller~G.\,B.} see~Borisov~A.\,V.&&\\[.23pt]
\Avtors{Minin~V.\,A., Zatsman~I.\,M., Havanskov~V.\,A., and
Shubnikov~S.\,K.} Intensity of citation of scientific publications in
inventions on information and computer technologies patented\linebreak
\\[-12pt]
\hspace*{23pt}in Russia by domestic and foreign applicants&2&107--122\\[.23pt]
\Avtors{Monakhov~M.\,M.} see~Markov~A.\,S.&&\\[.23pt]
\Avtors{Naumov~V.\,A.\ and Samouylov~K.\,E.} On relationship
between queuing systems with resources\linebreak
\\[-12pt]
\hspace*{23pt}and Erlang networks&3&\hphantom{1}9--14\\[.23pt]
\Avtors{Okladnikov~I.\,G.} see~Kalinichenko~L.\,A.&&\\[.23pt]
\Avtors{Ometov~A.\,Ya., Andreev~S.\,D., Turlikov~A.\,M., and
Koucheryavy~E.\,A.} Performance analysis of\linebreak
\\[-12pt]
\hspace*{23pt}a wireless data
aggregation system with contention for contemporary sensor
networks&3&23--31\\[.23pt]
\Avtors{Palionnaia~S.\,I.} see~Kudryavtsev~A.\,A.&&\\[.23pt]
\Avtors{Podkolodnyy~N.\,L.} see~Kalinichenko~L.\,A.&&\\[.23pt]
\Avtors{Ponomareva~N.\,V.} see~Kalinichenko~L.\,A.&&\\[.23pt]
\Avtors{Popkova~N.\,A.} see~Zatsman~I.\,M.&&\\[.23pt]
\Avtors{Pozanenko~A.\,S.} see~Kalinichenko~L.\,A.&&\\[.23pt]
\Avtors{Razumchik~R.\,V.} see~Konovalov~M.\,G.&&\\[.23pt]
\Avtors{Ronzhin~A.\,F.} see~Melnikov~A.\,K.&&\\[.23pt]
\Avtors{Rumovskaya~S.\,B.} see~Kirikov~I.\,A.&&\\[.23pt]
\Avtors{Rumovskaya~S.\,B.} see~Kirikov~I.\,A.&&\\[.23pt]
\Avtors{Rumovskaya~S.\,B.} see~Kolesnikov A.\,V.&&\\[.23pt]
\Avtors{Samouylov~K.\,E.} see~Gaidamaka~Yu.\,V.&&\\[.23pt]
\Avtors{Samouylov~K.\,E.} see~Naumov~V.\,A.&&\\[.23pt]
\Avtors{Serebryanskii~S.\,M.} see~Tyrsin~A.\,N.&&\\[.23pt]
\Avtors{Seyful-Mulyukov~R.\,B.} see~Callaos~N.\,K.&&\\[.23pt]
\Avtors{Shestakov~O.\,V.} Statistical properties of the denoising method
based on the stabilized hard\linebreak
\\[-12pt]
\hspace*{23pt}thresholding&2&65--69\\[.23pt]
\Avtors{Shestakov~O.\,V.} The strong law of large numbers for the risk
estimate in the problem of\linebreak
\\[-12pt]
\hspace*{23pt}tomographic image reconstruction from
projections with a correlated noise&3&41--45\\[.23pt]
\Avtors{Shestakov~O.\,V.} see~Zakharova~T.\,V.&&\\[.23pt]
\Avtors{Shnurkov~P.\,V., Gorshenin~A.\,K., and Belousov~V.\,V.}
Analytical solution of~the~optimal control\linebreak
\\[-12pt]
\hspace*{23pt}task of~a~semi-Markov
process with~finite set of~states&4&72--88\\[.23pt]
\Avtors{Shnurkov~P.\,V., Zasypko~V.\,V., Belousov~V.\,V., and
Gorshenin~A.\,K.} Development of the algorithm of numerical solution
of the optimal investment control problem\linebreak
\\[-12pt]
\hspace*{23pt}in the closed dynamical model of three-sector economy&1&82--95\\[.23pt]
\Avtors{Shorgin~S.\,Ya.} see~Gaidamaka~Yu.\,V.&&\\[.23pt]
\Avtors{Shorgin~V.\,S.} see~Agalarov~Ya.\,M.&&\\[.23pt]
\Avtors{Shubnikov~S.\,K.} see~Minin~V.\,A.&&\\[.23pt]
\Avtors{Sidorkin~I.\,I.} see~Arkhipov~O.\,P.&&\\[.23pt]
\Avtors{Sinitsyn~I.\,N.} Analytical modeling of processes in stochastic
systems with complex fractional\linebreak
\\[-12pt]
\hspace*{23pt}order Bessel nonlinearities&3&55--65\\[.23pt]
\Avtors{Sinitsyn~I.\,N.} Orthogonal supoptimal filters for nonlinear
stochastic systems on manifolds&1&34--44\\[.23pt]
\Avtors{Sinitsyn~I.\,N.\ and Korepanov~E.\,R.} Normal Pugachev
conditionally-optimal filters and extra-\linebreak
\\[-12pt]
\hspace*{23pt}polators for state linear stochastic systems&2&14--23\\[.23pt]
\Avtors{Sinitsyn~I.\,N.\ and Sinitsyn~V.\,I.} Analytical modeling of
distributions in stochastic systems on\linebreak
\\[-12pt]
\hspace*{23pt}manifolds based on ellipsoidal approximation&1&45--55\\[.23pt]
\Avtors{Sinitsyn~I.\,N., Sinitsyn~V.\,I., and
Korepanov~E.\,R.} Ellipsoidal suboptimal filters for nonlinear\linebreak
\\[-12pt]
\hspace*{23pt}stochastic systems on manifolds&2&24--35\\[.23pt]
\Avtors{Sinitsyn~V.\,I.} see~Sinitsyn~I.\,N.&&\\[.23pt]
\Avtors{Sinitsyn~V.\,I.} see~Sinitsyn~I.\,N.&&\\[.23pt]
\Avtors{Skvortsov~N.\,A.} see~Stupnikov~S.\,A.&&\\[.23pt]
\Avtors{Sokolov~I.\,A.} see~Chertok~A.\,V.&&\\
\end{tabular}
}
\pagebreak

\def\leftfootline{\small{\textbf{\thepage}
\hfill INFORMATIKA I EE PRIMENENIYA~--- INFORMATICS AND APPLICATIONS\ \ \ 2016\
\ \ volume~10\ \ \ issue\ 4}
}%
 \def\rightfootline{\small{INFORMATIKA I EE PRIMENENIYA~---
INFORMATICS AND APPLICATIONS\ \ \ 2016\ \ \ volume~10\ \ \ issue\ 4
\hfill \textbf{\thepage}}}

\def\leftkol{2016 AUTHOR INDEX} % ENGLISH ABSTRACTS}

\def\rightkol{2016 AUTHOR INDEX} %ENGLISH ABSTRACTS}


{\tabcolsep=3pt
\begin{tabular}{p{382pt}cc}
&\textbf{Issue} & \textbf{Page}\\[6pt]
\Avtors{Sopin~E.\,S.} see~Gaidamaka~Yu.\,V.&&\\
\Avtors{Strijov~V.\,V.} see~Goncharov~A.\,V.&&\\
\Avtors{Strijov~V.\,V.} see~Isachenko~R.\,V.&&\\
\Avtors{Strijov~V.\,V.} see~Karasikov~M.\,E.&&\\
\Avtors{Stupnikov~S.\,A., Briukhov~D.\,O., and Skvortsov~N.\,A.}
Co-lending systemic risk analysis over\linebreak
\\[-12pt]
\hspace*{23pt}heterogeneous data collections&1&23--33\\
\Avtors{Stupnikov~S.\,A.} see~Kalinichenko~L.\,A.&&\\
\Avtors{Suchkov~A.\,P.} see~Zatsarinny~A.\,A.&&\\
\Avtors{Timonina~E.\,E.} see~Grusho~A.\,A.&&\\
\Avtors{Titova~A.\,I.} see~Kudryavtsev~A.\,A.&&\\
\Avtors{Turlikov~A.\,M.} see~Ometov~A.\,Ya.&&\\
\Avtors{Tyrsin~A.\,N.\ and Serebryanskii~S.\,M.} Recognition of
dependences on the basis of inverse\linebreak
\\[-12pt]
\hspace*{23pt}mapping&2&58--64\\
\Avtors{Ulyanov~V.\,V.} see~Markov~A.\,S.&&\\
\Avtors{Ushakov~V.\,G.} Queueing system with working vacations and
hyperexponential input stream&2&92--97\\
\Avtors{Ushakov~V.\,G.} see~Leontyev~N.\,D.&&\\
\Avtors{Volnova~A.\,A.} see~Kalinichenko~L.\,A.&&\\
\Avtors{Yakovlev~O.\,A.\ and Gasilov~A.\,V.} Speeded-up stereo
matching using geodesic support weights&3&\hphantom{1}98--104\\
\Avtors{Zabezhailo~M.\,I.} see~Grusho~A.\,A.&&\\
\Avtors{Zabezhailo~M.\,I.} see~Grusho~A.\,A.&&\\
\Avtors{Zakharova~T.\,V.\ and Shestakov~O.\,V.} Precision analysis of
wavelet processing of aerodynamic\linebreak
\\[-12pt]
\hspace*{23pt}flow patterns&3&46--54\\
\Avtors{Zalizniak~Anna~A.\ and Kruzhkov~M.\,G.} Database
of~Russian impersonal verbal constructions&4&132--141\\
\Avtors{Zasypko~V.\,V.} see~Shnurkov~P.\,V.&&\\
\Avtors{Zatsarinny~A.\,A.\ and Suchkov~A.\,P.} Systems engineering
approaches to~the~establishment of\linebreak
\\[-12pt]
\hspace*{23pt}a~system for~decision support based
on~situational analysis&4&105--113\\
\Avtors{Zatsarinny~A.\,A.} see~Grusho~A.\,A.&&\\
\Avtors{Zatsman~I.\,M., Inkova~O.\,Yu., Kruzhkov~M.\,G., and
Popkova~N.\,A.} Representation of cross-\linebreak
\\[-12pt]
\hspace*{23pt}lingual knowledge about
connectors in supracorpora databases&1&106--118\\
\Avtors{Zatsman~I.\,M.} see~Minin~V.\,A.&&\\
\Avtors{Zeifman~A.\,I.} see~Korolev~V.\,Yu.&&\\
\Avtors{Zeifman~A.\,I.} see~Korolev~V.\,Yu.&&\\
\end{tabular}
}

%\thispagestyle{myheadings}
\def\leftfootline{\small{\textbf{\thepage}
\hfill INFORMATIKA I EE PRIMENENIYA~--- INFORMATICS AND APPLICATIONS\ \ \ 2016\
\ \ volume~10\ \ \ issue\ 4}
}%
 \def\rightfootline{\small{INFORMATIKA I EE PRIMENENIYA~---
INFORMATICS AND APPLICATIONS\ \ \ 2016\ \ \ volume~10\ \ \ issue\ 4
\hfill \textbf{\thepage}}}

 \label{end\stat}

\newpage

%\def\stat{rekl}
%\label{preobr}

%\def\tit{АКАДЕМИК ПУГАЧЁВ  ВЛАДИМИР СЕМЁНОВИЧ\\
%25.03.1911--25.03.1998}


%   \vspace*{-48pt}
%   \begin{center}\LARGE
%Академик Пугачёв  Владимир Семёнович\\ (25.03.1911--25.03.1998)
%   \end{center}
   
   %\vspace*{2.5mm}
   
   \begin{center}

{\prgsh\LARGE
ОБЪЯВЛЕНИЯ О КОНФЕРЕНЦИЯХ}

\end{center}
%\hrule

\vspace*{6pt}

   
   \vspace*{10mm}
   
   \thispagestyle{empty}

\noindent
\begin{tabular}{cc}
%\begin{center}
\multicolumn{1}{c}{\raisebox{-40pt}[0pt][0pt]{\mbox{%
\epsfxsize=33mm
\epsfbox{vspu.eps}
}}}
%\end{center}
&
\tabcolsep=0pt\begin{tabular}{c}
{\prg{\Large\textbf{XII Всероссийское совещание}}}\\[6pt]
{\prg{\Large\textbf{по проблемам управления}}}\\[12pt]
{\prg{\large 16--19 июня 2014~г.}}\\[6pt] 
{\prg{\large Институт проблем управления имени В.\,А.~Трапезникова РАН}}\\[6pt]
{\prg{\large Москва, Россия}}
\end{tabular}
\end{tabular}

\vspace*{60pt}

     
 { %\large    
 XII Всероссийское совещание по проблемам управления (ВСПУ XII), посвященное 75-летию 
Института проблем управления (ИПУ) имени В.\,А.~Трапезникова РАН, проводится 16--19~июня 
2014~г.\ 
в ИПУ РАН (г.~Москва, Россия). ВСПУ XII организуется ИПУ РАН при поддержке РФФИ, Отделения 
энергетики, машиностроения, механики и процессов управления Российской академии наук, 
Российского 
национального комитета по автоматическому управлению, Академии навигации и управ\-ле\-ния 
движением, 
Научного совета РАН по комплексным проблемам управления и автоматизации, Совета по 
мехатронике и робототехнике РАН. Официальный язык Совещания~--- русский.

\vspace*{24pt}
     
     \textbf{Направления работы}
     \begin{enumerate}[1.]
\item Теория систем управления
\item Управление подвижными объектами и навигация
\item Интеллектуальные системы управления
\item Управление в промышленности, транспортом и логистикой
\item Управление системами междисциплинарной природы
\item Средства измерения, вычислений и контроля в управлении
\item Системный анализ и принятие решений в задачах управления
\item Информационные технологии в управлении
\item Проблемы образования в области управления: современное содержание и технологии обучения
\end{enumerate}

\vspace*{24pt}

     Подробная информация о Совещании находится на сайте {\sf http://vspu2014.ipu.ru}. Срок 
окончательной подачи докладов через систему подачи докладов на сайте~--- \textbf{30~ноября} 
2013~г.
}

%\include{rekl-1}

%\end{document}

%\include{nekrolog-rb}


%\end{document}

%\include{IPPM-25}

\def\stat{cont-rus}
{%\hrule\par
%\vskip 7pt % 7pt
\vspace*{-24pt}
\raggedleft\Large \bf%\baselineskip=3.2ex
Правила подготовки рукописей  для публикации в журнале
<<Информатика~и~её~применения>> \vskip 8pt
    \hrule
    \par
\vskip 14pt plus 6pt minus 3pt }

\label{st\stat}

\def\tit{\ }

\def\aut{\ }
\def\auf{\ }

\def\leftkol{\ }
% Правила подготовки рукописей  для публикации в журнале
%<<Информатика и её применения>>

\def\rightkol{\ }
%Правила подготовки рукописей  для публикации в журнале
%<<Информатика и её применения>>}


\titele{\tit}{\aut}{\auf}{\leftkol}{\rightkol}


\vspace*{-60pt}
{ %\small

Журнал <<Информатика и её применения>>
публикует теоретические, обзорные и дискуссионные статьи,
посвященные научным исследованиям и разработкам в области
информатики и ее приложений.

Журнал издается на русском языке. По специальному решению
редколлегии отдельные статьи могут печататься на английском языке.

Тематика журнала охватывает следующие направления:
\begin{itemize}
\item теоретические основы информатики;\\[-15pt]
      \item
математические методы исследования сложных систем и процессов;\\[-15pt]
           \item
информационные системы и сети;\\[-15pt]
                \item
информационные технологии;\\[-15pt]
                     \item
архитектура и программное обеспечение вычислительных комплексов и сетей.\\[-15pt]
\end{itemize}


\noindent
\begin{enumerate}[1.]
\item В журнале печатаются статьи, содержащие результаты, ранее не опубликованные и
не предназначенные к одновременной публикации в других изданиях.

%Публикация не должна нарушать закон об авторских правах.
Публикация предоставленной автором(ами) рукописи не должна нарушать 
положений глав~69, 70 раздела~VII части~IV Гражданского кодекса, 
которые определяют права на результаты интеллектуальной деятельности 
и~средства индивидуализации, в~том числе авторские права, в~РФ.

Ответственность за нарушение авторских прав, в~случае предъявления претензий к~редакции журнала,  
несут авторы статей.



Направляя рукопись в редакцию, авторы сохраняют свои права на данную
рукопись и при этом передают учредителям и редколлегии журнала неисключительные права на
издание статьи на русском языке 
(или на языке статьи, если он отличен от рус\-ско\-го) и~на перевод ее на английский
язык, а~также на
ее распространение в России и за рубежом. 
Каждый автор должен представить в~редакцию подписанный 
с~его стороны <<Лицензионный договор о~передаче неисключительных прав 
на использование произведения>>, текст которого размещен по адресу 
{\sf http://www.ipiran.ru/publications/licence.doc}. 
Этот договор может быть пред\-став\-лен в~бумажном (в~2-х экз.)\ 
или в~электронном виде (отсканированная копия заполненного и~подписанного документа).




Редколлегия вправе запросить у авторов экспертное заключение о возможности
пуб\-ли\-ка\-ции пред\-став\-лен\-ной статьи в открытой печати.\\[-13.5pt]

\item К статье прилагаются данные автора (авторов) (см.\ п.~8). При наличии нескольких
авторов указывается фамилия автора, ответственного за переписку с редакцией.\\[-13.5pt]

\item Редакция журнала осуществляет экспертизу присланных статей в соответствии с
принятой в журнале процедурой рецензирования.

Возвращение рукописи на доработку не означает ее принятия к печати.

Доработанный вариант с ответом на замечания рецензента необходимо прислать в
редакцию.\\[-13.5pt]

\item Решение редколлегии о публикации статьи или ее отклонении сообщается авторам.

Редколлегия может также направить авторам текст рецензии на их статью. Дискуссия по
поводу отклоненных статей не ведется.\\[-13.5pt]

%\pagebreak

\item Редактура статей высылается авторам для просмотра. Замечания к редактуре должны
быть присланы авторами в кратчайшие сроки.\\[-13.5pt]

\item Рукопись предоставляется в электронном виде в форматах MS WORD (.doc или
.docx) или \LaTeX\  (.tex), дополнительно~--- в формате .pdf, на дискете, лазерном диске
или электронной почтой. Предоставление бумажной рукописи необязательно.\\[-13.5pt]

\item При подготовке рукописи в MS Word рекомендуется использовать следующие
настройки.

Параметры страницы:
формат~--- А4; ориентация~--- книжная; поля (см): внутри~--- 2,5, снаружи~--- 1,5,
сверху~--- 2, снизу~--- 2, от края до нижнего колонтитула~--- 1,3.

Основной текст: стиль~--- <<Обычный>>, шрифт~--- Times New Roman, размер~---
14~пунк\-тов, абзацный отступ~--- 0,5~см, 1,5~интервала, выравнивание~--- по ширине.

\pagebreak

\def\leftkol{Правила подготовки рукописей  для публикации в журнале
<<Информатика и её применения>>}

\def\rightkol{Правила подготовки рукописей  для публикации в журнале
<<Информатика и её применения>>}



Рекомендуемый объем рукописи~--- не свыше 10~страниц указанного формата.
При превышении указанного объема редколлегия вправе потребовать от 
автора сокращения объема рукописи.


Сокращения слов, помимо стандартных, не допускаются. Допускается минимальное
количество аббревиатур.


Все страницы рукописи нумеруются.

Шаблоны оформления представлены в интернете:

\noindent
 {\sf
http://www.ipiran.ru/journal/template\_iiep\_ssi\_2024.zip}\\[-14pt]

\item Статья должна содержать следующую информацию на {\bfseries\textit{русском и
английском языках}}:\\[-16pt]

\begin{itemize}
\item название статьи;\\[-15pt]
\item Ф.И.О.\ авторов, на английском можно только имя и фамилию;\\[-15pt]
\item место работы, с указанием почтового адреса организации и электронного адреса каждого
автора;\\[-15pt]
\item сведения об авторах, в соответствии с форматом, образцы которого
представлены на страницах:



\def\leftfootline{\small{\textbf{\thepage}
\hfill ИНФОРМАТИКА И ЕЁ ПРИМЕНЕНИЯ\ \ \ том\ 18\ \ \ выпуск\ 3\ \ \ 2024}
}%
 \def\rightfootline{\small{ИНФОРМАТИКА И ЕЁ ПРИМЕНЕНИЯ\ \ \ том\ 18\ \ \ выпуск\ 3\ \ \ 2024
\hfill \textbf{\thepage}}}



{\sf http://www.ipiran.ru/journal/issues/2013\_07\_01/authors.asp} и

{\sf http://www.ipiran.ru/journal/issues/2013\_07\_01\_eng/authors.asp};
\item аннотация (не менее 100~слов на каждом из языков). Аннотация~--- это краткое
резюме работы, которое может публиковаться отдельно. Она является основным
источником информации в~ин\-фор\-ма\-ци\-он\-ных системах и базах данных. Английская
аннотация должна быть оригинальной, может не быть дословным переводом русского
текста и должна быть написана хорошим английским языком. В~аннотации не должно
быть ссылок на литературу и, по возможности, формул;\\[-15pt]
\item ключевые слова~--- желательно из принятых в мировой
на\-уч\-но-тех\-ни\-че\-ской литературе тематических тезаурусов. Предложения не
могут быть ключевыми словами;\\[-15pt]
\item источники финансирования работы (ссылки на гранты, проекты,
поддерживающие организации и~т.\,п.).
\end{itemize}



%\pagebreak

\item  Требования к спискам литературы.\\[-14pt]

Ссылки на литературу в тексте статьи нумеруются (в квадратных скобках) и
располагаются в каждом из списков литературы в порядке  первых упоминаний. Если источник имеет DOI и/или EDN,
то их необходимо указывать.

Списки литературы представляются в двух вариантах:\\[-14pt]


\noindent
\begin{enumerate}[(1)]
\item \textbf{Список литературы к русскоязычной части}. Русские и английские
работы~---  на языке и в алфавите оригинала;\\[-14.5pt]
\item  \textbf{References}. Русские работы и работы на других языках~--- в латинской
транслитерации с переводом на английский язык; английские работы и работы на других
языках~--- на языке оригинала.
\end{enumerate}

Необходимо для составления списка ``References'' пользоваться размещенной на сайте
{\sf http://www. translit.net/ru/bgn/} бесплатной программой транслитерации русского
 текста в~латиницу. %, при этом в~за\-клад\-ке <<варианты\ldots>> следует выбратьопцию BGN.

Список литературы ``References'' приводится полностью отдельным блоком, повторяя все
позиции из списка литературы к русскоязычной части, независимо от того, имеются или
нет в нем иностранные источники. Если в списке литературы к русскоязычной части есть
ссылки на иностранные публикации, набранные латиницей, они полностью повторяются в
списке ``References''.

Ниже приведены примеры ссылок на различные виды публикаций в списке ``References''.

\def\leftfootline{\small{\textbf{\thepage}
\hfill ИНФОРМАТИКА И ЕЁ ПРИМЕНЕНИЯ\ \ \ том\ 18\ \ \ выпуск\ 3\ \ \ 2024}
}%
 \def\rightfootline{\small{ИНФОРМАТИКА И ЕЁ ПРИМЕНЕНИЯ\ \ \ том\ 18\ \ \ выпуск\ 3\ \ \ 2024
\hfill \textbf{\thepage}}}

{\small

\noindent
\textbf{Описание статьи из журнала:}

\Aue{Zagurenko, A.\,G., V.\,A.~Korotovskikh, A.\,A.~Kolesnikov, A.\,V.~Timonov, and D.\,V.~Kardymon}. 2008.
Tekhniko-ekonomicheskaya optimizatsiya dizayna gidrorazryva plasta [Technical and
economic optimization of the design
of hydraulic fracturing]. \textit{Neftyanoe hozyaystvo} [\textit{Oil Industry}] 11:54--57.

\Aue{Zhang, Z., and D.~Zhu}. 2008. Experimental research on the localized
electrochemical micromachining. \textit{Russ. J.~Electrochem.}  44(8):926--930.
{\sf doi:10.1134/S1023193508080077}.

\noindent
\textbf{Описание статьи из электронного журнала:}

\Aue{Swaminathan, V., E.~Lepkoswka-White, and B.\,P.~Rao}. 1999. Browsers or buyers in cyberspace? An
investigation of electronic factors influencing electronic exchange. \textit{JCMC}
5(2). Available at: {\sf http://www.ascusc.org/jcmc/vol5/issue2/} (accessed April~28, 2011).

\def\leftkol{Правила подготовки рукописей  для публикации в журнале
<<Информатика и её применения>>}

\def\rightkol{Правила подготовки рукописей  для публикации в журнале
<<Информатика и её применения>>}


\noindent
\textbf{Описание статьи из продолжающегося издания (сборника трудов):}

\Aue{Astakhov, M.\,V., and T.\,V.~Tagantsev}. 2006. Eksperimental'noe
issledovanie prochnosti soedineniy ``stal'--kompozit'' [Experimental study of
the strength of joints ``steel--composite'']. \textit{Trudy MGTU
``Matematicheskoe modelirovanie slozhnykh tekh\-ni\-che\-skikh sistem''}
[\textit{Bauman MSTU ``Mathematical Modeling of Complex Technical
Systems'' Proceedings}]. 593:125--130.


\pagebreak



\noindent
\textbf{Описание материалов конференций:}

\Aue{Usmanov, T.\,S., A.\,A.~Gusmanov, I.\,Z.~Mullagalin, R.\,Ju.~Muhametshina, A.\,N.~Chervyakova, and
A.\,V.~Sveshnikov}. 2007. Osobennosti proektirovaniya razrabotki mestorozhdeniy
s primeneniem gidrorazryva
plasta [Features of the design of field development with the use of hydraulic fracturing].
\textit{Trudy 6-go
Mezhdu\-na\-rod\-no\-go Simpoziuma ``Novye resursosberegayushchie tekhnologii nedropol'zovaniya i povysheniya
neftegazootdachi''} [\textit{6th  Symposium (International) ``New Energy Saving Subsoil Technologies and
the Increasing of the Oil and Gas Impact'' Proceedings}]. Moscow. 267--272.



\def\leftfootline{\small{\textbf{\thepage}
\hfill ИНФОРМАТИКА И ЕЁ ПРИМЕНЕНИЯ\ \ \ том\ 18\ \ \ выпуск\ 3\ \ \ 2024}
}%
 \def\rightfootline{\small{ИНФОРМАТИКА И ЕЁ ПРИМЕНЕНИЯ\ \ \ том\ 18\ \ \ выпуск\ 3\ \ \ 2024
\hfill \textbf{\thepage}}}



\noindent
\textbf{Описание книги (монографии, сборники):}



Lindorf, L.\,S., and L.\,G.~Mamikoniants, eds. 1972.
\textit{Ekspluatatsiya turbogeneratorov s neposredstvennym
okhlazhdeniem} [\textit{Operation of turbine generators with direct cooling}].
Moscow: Energy Publs. 352~p.


\Aue{Latyshev, V.\,N.} 2009. \textit{Tribologiya rezaniya. Kn.~1: Friktsionnye protsessy
pri rezanii metallov}
[\textit{Tribology of cutting. Vol.~1: Frictional processes in metal cutting}]. Ivanovo: Ivanovskii
State Univ. 108~p.

\def\leftkol{Правила подготовки рукописей  для публикации в журнале
<<Информатика и её применения>>}

\def\rightkol{Правила подготовки рукописей  для публикации в журнале
<<Информатика и её применения>>}

\noindent
\textbf{Описание переводной книги}
(в списке литературы к русскоязычной части необходимо указать:~/ Пер.\ с англ.~---
после названия книги, а в конце ссылки указать оригинал книги в круглых скобках):
\begin{enumerate}[1.]
\item  В русскоязычной части:

\def\leftfootline{\small{\textbf{\thepage}
\hfill ИНФОРМАТИКА И ЕЁ ПРИМЕНЕНИЯ\ \ \ том\ 18\ \ \ выпуск\ 3\ \ \ 2024}
}%
 \def\rightfootline{\small{ИНФОРМАТИКА И ЕЁ ПРИМЕНЕНИЯ\ \ \ том\ 18\ \ \ выпуск\ 3\ \ \ 2024
\hfill \textbf{\thepage}}}

\Au{Тимошенко С.\,П., Янг Д.\,Х., Уивер~У.}
Колебания в инженерном деле~/ Пер.\ с англ.~--- М.: Машиностроение, 1985. 472~с.
(\Au{Timoshenko~S.\,P., Young~D.\,H., Weaver~W.}
Vibration problems in engineering.~--- 4th ed.~--- New York, NY, USA: Wiley, 1974. 521~p.)\\[-13.5pt]
\item  В англоязычной части:

\Aue{Timoshenko, S.\,P., D.\,H.~Young, and W.~Weaver}.
1974. \textit{Vibration problems in engineering}. 4th ed. New York: 
Wiley. 521~p.
\end{enumerate}

\vspace*{-3pt}


\noindent
\textbf{Описание неопубликованного документа:}


\Aue{Latypov, A.\,R., M.\,M.~Khasanov, and V.\,A.~Baikov}.
2004 (unpubl.). Geologiya i~dobycha (NGT GiD) [Geology and production (NGT GiD)]. Certificate on official registration of the computer program
No.\,2004611198. 

\noindent
\textbf{Описание интернет-ресурса:}


Pravila tsitirovaniya istochnikov [Rules for the citing of sources]. Available at: {\sf
http://www.scribd.com/doc/1034528/} (accessed February~7, 2011).

%\pagebreak

\noindent
\textbf{Описание диссертации или автореферата диссертации:}

\Aue{Semenov, V.\,I.}
2003. Matematicheskoe modelirovanie plazmy v sisteme kompaktnyy tor [Mathematical
modeling of the plasma in the compact torus].  Moscow.  D.Sc.\ Diss. 272~p.

\Aue{Kozhunova, O.\,S.} 2009. Tekhnologiya razrabotki semanticheskogo
slovarya informatsionnogo monitoringa [Technology of development of
semantic dictionary of information monitoring system].  Moscow: IPI RAN. PhD Thesis. 23~p.


\noindent
\textbf{Описание ГОСТа:}

GOST 8.586.5-2005. 2007. Metodika vypolneniya izmereniy. Izmerenie raskhoda i~kolichestva zhidkostey i~gazov
s~pomoshch'yu standartnykh suzhayushchikh ustroystv [Method of measurement.
Measurement of flow rate and volume of liquids and gases by means of orifice devices]. Moscow:
Standardinform  Publs. 10~p.

\noindent
\textbf{Описание патента:}

\Aue{Bolshakov, M.\,V., A.\,V.~Kulakov, A.\,N.~Lavrenov, and M.\,V.~Palkin}.
2006. Sposob orientirovaniya po krenu letatel'nogo
apparata s opti\-che\-skoy golovkoy
samonavedeniya [The way to orient on the roll of aircraft with optical homing head].
Patent RF No.\,2280590.
}

\item Присланные в редакцию материалы авторам не возвращаются.\\[-13.5pt]

\item При отправке файлов по электронной почте просим придерживаться следующих
правил:
\begin{itemize}
\item указывать в поле subject (тема) название журнала и фамилию автора;\\[-13.5pt]
\item указывать в тексте письма название статьи, авторов и~журнал, в~который направляется статья;\\[-13.5pt]
\item использовать attach (присоединение);\\[-13.5pt]
\item в состав электронной версии статьи должны входить: файл, содержащий текст
статьи, и файл(ы), содержащий(е) иллюстрации.\\[-13.5pt]
\end{itemize}

\item Журнал <<Информатика и её применения>> является некоммерческим изданием.
Плата за публикацию не взимается, гонорар авторам не выплачивается.
\end{enumerate}



\def\leftfootline{\small{\textbf{\thepage}
\hfill ИНФОРМАТИКА И ЕЁ ПРИМЕНЕНИЯ\ \ \ том\ 18\ \ \ выпуск\ 3\ \ \ 2024}
}%
 \def\rightfootline{\small{ИНФОРМАТИКА И ЕЁ ПРИМЕНЕНИЯ\ \ \ том\ 18\ \ \ выпуск\ 3\ \ \ 2024
\hfill \textbf{\thepage}}}


\vspace*{-1mm}

\begin{center}

\textbf{Адрес редакции журнала <<Информатика и её применения>>:} \\




Москва 119333, ул.~Вавилова, д.~44, корп.~2, ФИЦ ИУ РАН\\[-10pt]

\

Тел.: +7\,(499)\,135-86-92\ \ Факс:  +7\,(495)\,930-45-05\\[-10pt]

 \

e-mail:   {\sf iiep@frccsc.ru} (Стригина Светлана Николаевна)\\[-10pt]

\

{\sf http://www.ipiran.ru/journal/issues/}
\end{center}
}


\def\leftkol{Правила подготовки рукописей  для публикации в журнале
<<Информатика и её применения>>}

\def\rightkol{Правила подготовки рукописей  для публикации в журнале
<<Информатика и её применения>>}


\def\leftfootline{\small{\textbf{\thepage}
\hfill ИНФОРМАТИКА И ЕЁ ПРИМЕНЕНИЯ\ \ \ том\ 18\ \ \ выпуск\ 3\ \ \ 2024}
}%
 \def\rightfootline{\small{ИНФОРМАТИКА И ЕЁ ПРИМЕНЕНИЯ\ \ \ том\ 18\ \ \ выпуск\ 3\ \ \ 2024
\hfill \textbf{\thepage}}} 
\def\stat{podg-e}
{%\hrule\par
%\vskip 7pt % 7pt
\vspace*{-24pt}
\raggedleft\Large \bf%\baselineskip=3.2ex
Requirements for manuscripts submitted to Journal
``Informatics~and~Applications'' \vskip 8pt
    \hrule
    \par
\vskip 21pt plus 6pt minus 3pt }

\label{st\stat}

\def\tit{\ }

\def\aut{\ }
\def\auf{\ }

\def\leftkol{\ }

\def\rightkol{\ }
%Requirements for manuscripts submitted to Journal
%``Informatics~and~Applications''}

\titele{\tit}{\aut}{\auf}{\leftkol}{\rightkol}

\def\leftfootline{\small{\textbf{\thepage}
\hfill INFORMATIKA I EE PRIMENENIYA~--- INFORMATICS AND APPLICATIONS\ \ \ 2019\
\ \ volume~13\ \ \ issue\ 4}
}%
 \def\rightfootline{\small{INFORMATIKA I EE PRIMENENIYA~--- INFORMATICS AND APPLICATIONS\ \ \ 2019\ \ \ volume~13\ \ \ issue\ 4
\hfill \textbf{\thepage}}}

\vspace*{-60pt}

{\small

\noindent
Journal ``Informatics and Applications'' (Inform.\ Appl.)
publishes theoretical, review, and discussion
articles on the research and development in the
field of informatics and its applications.

The journal is published in Russian.
By a special decision of the editorial
board, some articles can be published in English.


The topics covered include the following areas:
\begin{itemize}
               \item
     theoretical fundamentals of informatics; \\[-14pt]
\item
mathematical methods for studying complex systems and processes; \\[-14pt]
\item
information systems and networks;\\[-14pt]
\item
information technologies; and \\[-14pt]
\item
architecture and software of computational complexes and networks. \\[-14pt]
\end{itemize}

\noindent
\begin{enumerate}[1.]
\item The Journal publishes original articles which have not been published before and are not
intended for simultaneous publication in other editions. An article submitted to the Journal must not violate the
Copyright law. Sending the manuscript to the Editorial Board, the authors retain all rights of the
owners of the manuscript and transfer the nonexclusive rights to publish the article in Russian
(or the language of the article, if not Russian) and its distribution in Russia and abroad to the
Founders and the Editorial Board. Authors should submit a letter to the Editorial Board in the
following form:

{\bfseries\textit{Agreement on the transfer of rights to publish:}}

``\textit{We, the undersigned authors of the manuscript ``\ldots'', pass to the
Founder and the Editorial Board of the Journal ``Informatics and Applications''
the nonexclusive right to publish the manuscript of the article in Russian (or
in English) in both print and electronic versions of the Journal. We affirm
that this publication does not violate the Copyright of other persons or
organizations.}

\textit{Author(s) signature(s): (name(s), address(es), date).}

This agreement should be submitted in paper form or in the form of a scanned copy (signed by
the authors).


%The Editorial Board has the right to request from the authors an official expert conclusion that
%the submitted article has no secret data prohibited for publication. \\[-13.5pt]
\item
A submitted article should be attached with \textbf{the data on the author(s)} (see item~8). If
there are several authors, the contact person should be indicated who is responsible for
correspondence with the Editorial Board and other authors about revisions and final approval
of the proofs.\\[-13.5pt]

\item The Editorial Board of the Journal examines the article according to the established
reviewing procedure. If the authors receive their article for correction after reviewing, it does not
mean that the article is approved for publication. The corrected article should be sent to the
Editorial Board for the subsequent review and approval.\\[-13.5pt]

\item The decision on the article publication or its rejection is communicated to the authors. The
Editorial Board may also send the reviews on the submitted articles to the authors. Any
discussion upon the rejected articles is not possible.\\[-13.5pt]

\item The edited articles will be sent to the authors for proofread. The comments of the authors
to the edited text of the article should be sent to the Editorial Board as soon as possible.\\[-13.5pt]

\item The manuscript of the article should be presented electronically in the MS WORD (.doc or
.docx) or \LaTeX\ (.tex) formats, and additionally in the .pdf format. All documents
 may be sent
by e-mail or provided on a CD or diskette. A~hard copy submission is not necessary.\\[-13.5pt]

\item The recommended typesetting instructions for manuscript.

Pages parameters: format A4, portrait orientation, document margins (cm): left~--- 2.5, right~---
1.5, above~--- 2.0, below~--- 2.0, footer 1.3.

Text: font~---Times New Roman, font size~--- 14, paragraph indent~--- 0.5, line spacing~--- 1.5,
justified alignment.

The recommended manuscript size: not more than 15~pages of the specified format.
If the specified size exceeded, the editorial board is entitled to require the author
to reduce the manuscript.

Use only standard abbreviations. Avoid  abbreviations in the title and
abstract. The full term for which an abbreviation stands should precede
its first use in the text unless it is a standard unit of measurement.

All pages of the manuscript should be numbered.

The templates for the manuscript typesetting are presented on site: {\sf
http://www.ipiran.ru/journal/template.doc}.\\[-13.5pt]


%\def\leftkol{Requirements for manuscripts submitted to Journal
%``Informatics~and~Applications''}

\item The articles should enclose data both in \textbf{Russian and English}:
\begin{itemize}
\item title;\\[-13.5pt]
\item author's name and surname;\\[-13.5pt]
\item affiliation~--- organization, its address with ZIP code, city, country, and
official e-mail address;\\[-13.5pt]
\item data on authors according to the format: (see site)

{\sf http://www.ipiran.ru/journal/issues/2013\_07\_01/authors.asp}  and

{\sf  http://www.ipiran.ru/journal/issues/2013\_07\_01\_eng/authors.asp};\\[-13.5pt]

\pagebreak

\def\leftfootline{\small{\textbf{\thepage}
\hfill INFORMATIKA I EE PRIMENENIYA~--- INFORMATICS AND APPLICATIONS\ \ \ 2019\
\ \ volume~13\ \ \ issue\ 4}
}%
 \def\rightfootline{\small{INFORMATIKA I EE PRIMENENIYA~--- INFORMATICS AND APPLICATIONS\ \ \ 2019\ \ \ volume~13\ \ \ issue\ 4
\hfill \textbf{\thepage}}}


%\def\leftkol{Requirements for manuscripts submitted to Journal
%``Informatics~and~Applications''}

%\def\rightkol{Requirements for manuscripts submitted to Journal
%``Informatics~and~Applications''}



\item abstract (not less than 100 words) both in Russian and in English. Abstract is a short
summary of the article that can be published separately. The abstract is the
main source of information on the article and it could be included in leading information
systems and data bases. The abstract in English has to be an original text and should
not be an exact translation of the Russian one. Good English is required.
In abstracts, avoid references and formulae;\\[-13.5pt]
\item indexing is performed on the basis of keywords. The use of keywords from the
internationally accepted thematic Thesauri is recommended.

%\def\leftkol{Requirements for manuscripts submitted to Journal
%``Informatics~and~Applications''}

%\def\rightkol{Requirements for manuscripts submitted to Journal
%``Informatics~and~Applications''}

Important! Keywords must not be sentences;
\item Acknowledgments.
\end{itemize}

\item References. Russian references have to be presented both in English translation and Latin
transliteration (refer {\sf http://www.translit.net/ru/bgn/}).

Please take into account the following examples of Russian references appearance:

\noindent
\textbf{Article in journal:}

\Aue{Zhang, Z., and D.~Zhu}. 2008. Experimental research on the localized electrochemical
micromachining.
\textit{Rus. J.~Electrochem.}  44(8):926--930. {\sf doi:10.1134/S1023193508080077}.


\noindent
\textbf{Journal article in electronic format:}

\Aue{Swaminathan, V., E.~Lepkoswka-White, and B.\,P.~Rao}. 1999. Browsers or buyers in
cyberspace? An
investigation of electronic factors influencing electronic exchange. \textit{JCMC}
5(2). Available at: {\sf http://www.ascusc.org/jcmc/vol5/issue2/} (accessed April~28, 2011).




\noindent
\textbf{Article from the continuing publication (collection of works, proceedings):}

\Aue{Astakhov, M.\,V., and T.\,V.~Tagantsev}. 2006. Eksperimental'noe
issledovanie prochnosti soedineniy ``stal'--kompozit'' [Experimental study of
the strength of joints ``steel--composite'']. \textit{Trudy MGTU
``Matematicheskoe modelirovanie slozhnykh tekh\-ni\-che\-skikh sistem''}
[\textit{Bauman MSTU ``Mathematical Modeling of Complex Technical
Systems'' Proceedings}]. 593:125--130.

\def\leftfootline{\small{\textbf{\thepage}
\hfill INFORMATIKA I EE PRIMENENIYA~--- INFORMATICS AND APPLICATIONS\ \ \ 2019\
\ \ volume~13\ \ \ issue\ 4}
}%
 \def\rightfootline{\small{INFORMATIKA I EE PRIMENENIYA~--- INFORMATICS AND APPLICATIONS\ \ \ 2019\ \ \ volume~13\ \ \ issue\ 4
\hfill \textbf{\thepage}}}

\def\leftkol{Requirements for manuscripts submitted to Journal
``Informatics~and~Applications''}

\def\rightkol{Requirements for manuscripts submitted to Journal
``Informatics~and~Applications''}

\noindent
\textbf{Conference proceedings:}

\Aue{Usmanov, T.\,S., A.\,A.~Gusmanov, I.\,Z.~Mullagalin, R.\,Ju.~Muhametshina,
A.\,N.~Chervyakova, and
A.\,V.~Sveshnikov}. 2007. Osobennosti proektirovaniya razrabotki mestorozhdeniy
s primeneniem gidrorazryva
plasta [Features of the design of field development with the use of hydraulic fracturing].
\textit{Trudy 6-go
Mezhdu\-na\-rod\-no\-go Simpoziuma ``Novye resursosberegayushchie tekhnologii
nedropol'zovaniya i povysheniya
neftegazootdachi''} [\textit{6th  Symposium (International) ``New Energy Saving Subsoil
Technologies and
the Increasing of the Oil and Gas Impact'' Proceedings}]. Moscow. 267--272.


\noindent
\textbf{Books and other monographs:}




Lindorf, L.\,S., and L.\,G.~Mamikoniants, eds. 1972.
\textit{Ekspluatatsiya turbogeneratorov s neposredstvennym
okhlazhdeniem} [\textit{Operation of turbine generators with direct cooling}].
Moscow: Energy Publs. 352~p.


%\Aue{Latyshev, V.\,N.} 2009. \textit{Tribologiya rezaniya. Kn.~1: Frikcionnye prosessy
%pri rezanii metallov}
%[\textit{Tribology of cutting. Vol.~1: Frictional processes in metal cutting}]. Ivanovo: Ivanovskii
%State Univ. 108~p.


%\noindent
%\textbf{Unpublished material:}

%\Aue{Latypov, A.\,R., M.\,M.~Khasanov, and V.\,A.~Baikov}.
%2004. Geology and production (NGT GiD). Certificate on official registration of the computer
%program
%No.\,2004611198. (In Russian, unpubl.)

%\noindent
%\textbf{Internet-source:}

%APA Style. 2011. Available at: {\sf http://www.apastyle.org/apa-style-help.aspx} (accessed
%February~5, 2011).

%Pravila citirovaniya istochnikov [Rules for the citing of sources]. Available at: {\sf
%http://www.scribd.com/doc/1034528/} (accessed February~7, 2011).


\noindent
\textbf{Dissertation and Thesis:}

%\Aue{Semenov, V.\,I.}
%2003. Matematicheskoe modelirovanie plazmy v sisteme kompaktnyy tor. [Mathematical
%modeling of the plasma in the compact torus]. D.Sc.\ Diss. Moscow. 272~p.

\Aue{Kozhunova, O.\,S.} 2009. Tekhnologiya razrabotki semanticheskogo
slovarya informatsionnogo monitoringa [Technology of development of
semantic dictionary of information monitoring system]. PhD Thesis. Moscow: IPI RAN. 23~p.


\noindent
\textbf{State standards and patents:}

GOST 8.586.5-2005. 2007. Metodika vypolneniya izmereniy. Izmerenie raskhoda i~kolichestva
zhidkostey i gazov 
s~pomoshch'yu standartnykh suzhayushchikh ustroystv [Method of measurement.
Measurement of flow rate and volume of liquids and gases by means of orifice devices]. M.:
Standardinform
Publs. 10~p.

%\noindent
%\textbf{Patent:}

\Aue{Bolshakov, M.\,V., A.\,V.~Kulakov, A.\,N.~Lavrenov, and M.\,V.~Palkin}.
2006. Sposob orientirovaniya po krenu letatel'nogo
apparata s opti\-che\-skoy golovkoy
samonavedeniya [The way to orient on the roll of aircraft with optical homing head].
Patent RF No.\,2280590.

References in Latin transcription are presented in the original language.

References in the text are numbered according to the order of their
first appearance; the number is
placed in square brackets. All items from the reference list should be
cited.\\[-13.5pt]

\item Manuscripts and additional materials are not returned to Authors by the Editorial Board.\\[-13.5pt]

\item Submissions of files by e-mail must include:\\[-13.5pt]
\begin{itemize}
\item   the journal title and author's name in the ``Subject'' field; \\[-13.5pt]
\item   an article and additional materials have to be attached using the ``attach'' function;\\[-13.5pt]
\item   an electronic version of the article should contain the file with the text and a separate file
with figures.\\[-13.5pt]
\end{itemize}

\item ``Informatics and Applications'' journal is not a profit publication. There are no
charges for the authors as well as there are no royalties.\\[-13.5pt]
\end{enumerate}

\def\leftfootline{\small{\textbf{\thepage}
\hfill INFORMATIKA I EE PRIMENENIYA~--- INFORMATICS AND APPLICATIONS\ \ \ 2019\
\ \ volume~13\ \ \ issue\ 4}
}%
 \def\rightfootline{\small{INFORMATIKA I EE PRIMENENIYA~--- INFORMATICS AND APPLICATIONS\ \ \ 2019\ \ \ volume~13\ \ \ issue\ 4
\hfill \textbf{\thepage}}}

\def\leftkol{Requirements for manuscripts submitted to Journal
``Informatics~and~Applications''}

\def\rightkol{Requirements for manuscripts submitted to Journal
``Informatics~and~Applications''}


%\vspace*{5mm}


\begin{center}
\textbf{Editorial Board address:} \\

%ABOUT AUTHORS



FRC CSC RAS, 44, block~2, Vavilov Str., Moscow 119333, Russia\\[-10pt]

\

Ph.: +7\,(499)\,135\,86\,92,\ \ Fax: +7\,(495)\,930\,45\,05\\[-10pt]

\

 e-mail: {\sf rust@ipiran.ru} (to Prof.\ Rustem Seyful-Mulyukov)\\[-10pt]

\

 {\sf http://www.ipiran.ru/english/journal.asp}
\end{center}
 }
%\thispagestyle{myheadings}

\def\leftkol{Requirements for manuscripts submitted to Journal
``Informatics~and~Applications''}

\def\rightkol{Requirements for manuscripts submitted to Journal
``Informatics~and~Applications''}

\def\leftfootline{\small{\textbf{\thepage}
\hfill INFORMATIKA I EE PRIMENENIYA~--- INFORMATICS AND APPLICATIONS\ \ \ 2019\
\ \ volume~13\ \ \ issue\ 4}
}%
 \def\rightfootline{\small{INFORMATIKA I EE PRIMENENIYA~--- INFORMATICS AND APPLICATIONS\ \ \ 2019\ \ \ volume~13\ \ \ issue\ 4
\hfill \textbf{\thepage}}}

 \label{end\stat}

\newpage

%\vspace*{-60pt} {\small
{\baselineskip=9.1pt
\section*{Правила подготовки рукописей статей для публикации в журнале
<<Информатика и её применения>>}

\thispagestyle{empty}

 Журнал <<Информатика и её применения>> публикует
теоретические, обзорные и дискуссионные статьи, посвященные научным
исследованиям и разработкам в области информатики и ее приложений. Журнал
издается на русском языке. По специальному решению редколлегии отдельные статьи,
в виде исключения, могут печататься на английском языке.
Тематика журнала охватывает следующие направления:
\begin{itemize}
\item теоретические основы информатики; %\\[-13.5pt]
\item математические методы исследования сложных систем и процессов; %\\[-13.5pt]
\item информационные системы и сети; %\\[-13.5pt]
\item информационные технологии; %\\[-13.5pt]
\item архитектура и программное
обеспечение вычислительных комплексов и сетей.
\end{itemize}
\begin{enumerate}
\item В журнале печатаются результаты, ранее не
опубликованные и не предназначенные к одновременной публикации в других
изданиях. Публикация не должна нарушать закон об авторских правах. Направляя
свою рукопись в редакцию, авторы автоматически передают учредителям и
редколлегии неисключительные права на издание данной статьи на русском языке и
на ее распространение в России и за рубежом. При этом за авторами сохраняются
все права как собственников данной рукописи. В связи с этим авторами должно
быть представлено в редакцию письмо в следующей форме:
Соглашение о передаче права на публикацию:

\textit{<<Мы, нижеподписавшиеся, авторы рукописи <<$\qquad\qquad$>>, передаем
учредителям и редколлегии журнала <<Информатика и её применения>>
неисключительное право опубликовать данную рукопись статьи на русском языке как
в печатной, так и в электронной версиях журнала. Мы подтверждаем, что данная
публикация не нарушает авторского права других лиц или организаций. Подписи
авторов: (ф.\,и.\,о., дата, адрес)>>.}

Указанное соглашение может быть представлено 
как в бумажном виде, так и в виде отсканированной копии (с подписями авторов).


Редколлегия вправе запросить у авторов экспертное заключение о возможности
опубликования представленной статьи в открытой печати. %\\[-13.5pt]
\item Статья
подписывается всеми авторами. На отдельном листе представляются данные автора
(или всех авторов): фамилия, полные имя и отчество, телефон, факс, e-mail,
почтовый адрес. Если работа выполнена несколькими авторами, указывается фамилия
одного из них, ответственного за переписку с редакцией. %\\[-13.5pt]
\item Редакция журнала
осуществляет самостоятельную экспертизу присланных статей. Возвращение рукописи
на доработку не означает, что статья уже принята к печати. Доработанный вариант
с ответом на замечания рецензента необходимо прислать в редакцию. %\\[-13.5pt]
\item Решение
редакционной коллегии о принятии статьи к печати или ее отклонении сообщается
авторам. Редколлегия не обязуется направлять рецензию авторам отклоненной
статьи. %\\[-13.5pt]
\item Корректура статей высылается авторам для просмотра. Редакция
просит авторов присылать свои замечания в кратчайшие сроки. %\\[-13.5pt]
\item При
подготовке рукописи в MS Word рекомендуется использовать следующие настройки.
Параметры страницы: формат~--- А4; ориентация~--- книжная; поля (см): внутри~---
2,5, снаружи~--- 1,5, сверху~--- 2, снизу~--- 2, от края до нижнего
колонтитула~--- 1,3. Основной текст: стиль~--- <<Обычный>>: шрифт Times New
Roman, размер 14~пунктов, абзацный отступ~--- 0,5~см, 1,5 интервала,
выравнивание~--- по ширине. Рекомендуемый объем рукописи~--- не свыше
25~страниц указанного формата. Ознакомиться с шаблонами, содержащими примеры
оформления, можно по адресу в Интернете:
\textsf{http://www.ipiran.ru/journal/template.doc}.
\item К рукописи, предоставляемой в 2-х
экземплярах, обязательно прилагается электронная версия статьи (как правило, в
форматах MS WORD (.doc) или \LaTeX\ (.tex), а также~--- дополнительно~--- в
формате .pdf) на дискете, лазерном диске или по электронной почте. Сокращения
слов, кроме стандартных, не применяются. Все страницы рукописи должны быть
пронумерованы. %\\[-13.5pt]
\item Статья должна содержать следующую информацию на русском и
английском языках: название, Ф.И.О. авторов, места работы авторов и их
электронные адреса, подробные сведения об авторах, оформленные в соответствии с форматом, 
определяемым файлами {\sf http://www.ipiran.ru/journal/issues/2011\_05\_01/authors.asp} и 
{\sf http://www.ipiran.ru/journal/issues/2011\_01\_eng/authors.asp},
аннотация (не более 100~слов), ключевые слова. Ссылки на
литературу в тексте статьи нумеруются (в квадратных скобках) и располагаются в
порядке их первого упоминания. В~списке литературы не должно быть позиций, на которые нет ссылки в тексте статьи.
Все фамилии авторов, заглавия статей, названия
книг, конференций и~т.\,п.\ даются на языке оригинала, если этот язык
использует кириллический или латинский алфавит. %\\[-13.5pt]
\item Присланные в редакцию материалы авторам не возвращаются.
\item При отправке файлов по электронной
почте просим придерживаться следующих правил:
\begin{itemize}
\item указывать в поле subject (тема) название журнала и фамилию автора; %\\[-13.5pt]
\item использовать attach (присоединение); %\\[-13.5pt]
\item в случае больших объемов информации возможно
использование общеизвестных архиваторов (ZIP, RAR); %\\[-13.5pt]
\item в состав электронной версии статьи должны входить: файл, содержащий текст статьи, и файл(ы),
содержащий(е) иллюстрации. %\\[-13.5pt]
\end{itemize}
\item Журнал <<Информатика и её применения>> является некоммерческим изданием. 
Плата за публикацию с авторов не взимается, гонорар авторам не выплачивается.
\end{enumerate}
\thispagestyle{empty}
\textbf{Адрес редакции:} Москва 119333,
ул.~Вавилова, д.~44, корп.~2, ИПИ РАН\\
\hphantom{\textbf{Адрес редакции:} }Тел.: +7 (499) 135-86-92\ \
Факс:  +7 (495) 930-45-05\ \  E-mail:   rust@ipiran.ru }
}

%\include{ipi-ind}

%\tableofcontents

\end{document}

%\tableofcontents

%\end{document}

%\tableofcontents


\end{document}

\newcommand{\Ack}{\subsection*{\protect\large\bf Acknowledgments}}

\vphantom{\int\limits_0^T }

{ \begin{center}  %fig1
 \vspace*{3pt}
    \mbox{%
 \epsfxsize=79mm 
 \epsfbox{gru-1.eps}
 }

\end{center}

\noindent
{{\figurename~1}\ \ \small{
Временные зависимости данные 
}}}

\vspace*{6pt}

\setcounter{figure}{1}

$\acute{\mbox{о}}$

\linebreak