\documentclass[10pt]{book}
\usepackage[utf8]{inputenc}

\usepackage{latexsym,amssymb,amsfonts,amsmath,amsxtra,dsfont,
indentfirst,shapepar,%fleqn,%
picinpar,shadow,floatflt,enumerate,multicol,colortbl,moreverb,cite,ipi}

\usepackage{rotating}
\usepackage{mathrsfs}
\usepackage[noend]{algorithmic}
\usepackage{ulem}
\usepackage{graphicx}
%\usepackage{algorithm2e}
\usepackage[linesnumbered,boxed,ruled]{algorithm2e}
%\usepackage{xypic}
\usepackage{oldgerm}
\usepackage{epic}
\usepackage{eepic}

\SetAlgorithmName{Algorithm}{алгоритм}{Список алгоритмов}

%из Дюковой

\newcommand{\algKeyword}[1]{{\bf #1}}
\newcommand{\Proc}[1]{\text{\tt #1}}
\def\CALL{\algKeyword{call}~}

\newenvironment{AlgProcedure}[1]
{
\small
\medskip
%    \hrule
\medskip
\algKeyword{PROCEDURE} #1
\begin{algorithmic}[1]}
{\end{algorithmic}
%    \hrule
\bigskip
}

\def\CALL{\algKeyword{call}~}

%конец для Дюковой

%\RequirePackage[ruled]{algorithm}


\input{epsf}

%\nofiles

%\includeonly{avtor}             %+pdf+
%\includeonly{obchak,avtor}
%\includeonly{pred}                 %+
%\includeonly{podgot-rus-site,podgot-eng-site}  
%\includeonly{podgot-rus,podgot-eng}  
%\includeonly{ocherk} 
%\includeonly{ipi-ind} 
%\includeonly{index14}
%\includeonly{toc-rus, toc-en}
%\includeonly{toc-rus}
%\includeonly{toc-en} 


      
%\includeonly{agalarov}                  %+pdf+авт+
%\includeonly{kudr}                      %+pdf+авт+
%\includeonly{sinits}                    %+pdf+авт+
%\includeonly{pavlov}                    %+pdf+авт+
%\includeonly{lipatyev}                  %+pdf+авт+
%\includeonly{zatsman}                   %+pdf+авт+
%\includeonly{kovalev}                   %+pdf+авт+
%\includeonly{stup-bruhov}               %+pdf+авт+
%\includeonly{flerov}               %pdf
%\includeonly{bosov}                     %pdf+авт+
%\includeonly{hvatova}                   %pdf+авт+
%\includeonly{gonch-zats}                %pdf+авт+
%\includeonly{strijov}                   %pdf+авт+
%\includeonly{chehovich}                 %pdf+авт+
%\includeonly{dorofeeva}                 %pdf+авт+
%\includeonly{grinchenko}      %pdf





%\includeonly{obchak}
%\includeonly{rekl}
%\includeonly{rekl-1}
%\includeonly{reshal}  %
%\includeonly{cover3}

\usepackage{acad}
%\usepackage{courier}
\usepackage{decor}
\usepackage{newton}
\usepackage{pragmatica}
\usepackage{zapfchan}
\usepackage{petrotex}
\usepackage{bm}                     % полужирные греческие буквы
\usepackage{upgreek}                % прямые греческие буквы \upalpha
\usepackage{eufrak}
\usepackage{verbatim}

\renewcommand{\bottomfraction}{0.99}
\renewcommand{\topfraction}{0.99}
\renewcommand{\textfraction}{0.01}

\setcounter{secnumdepth}{1} %здесь - 3 + chapter = 4

\arraycolsep=1.5pt

%\usepackage[pdftex]{graphicx}

%\usepackage{oz}

%NEW COMMANDS


\renewcommand*{\hm}[1]{#1\nobreak\discretionary{}%
            {\hbox{$\mathsurround=0pt #1$}}{}} %% Дублирует знаки операций
                               %при переносе в формуле (перед знаком, который
                               %надо продублировать ставится команда \hm)

%\newcommand{\endproof}{\hfill$\Box$}
\renewcommand{\r}{\mathbb{R}}
%\newcommand{\I}{{\rm I\hspace{-0.7mm}I}}
%\newcommand{\Ikl}{{\tt{1}}\hspace*{-1.44mm}\mathtt{1}}
\newcommand{\Ik}{\mbox{{\small \tt {1}}\hspace{-1.3mm}{\tt 1}}}
\newcommand{\argmin}{\mathop{\mathrm{arg}\,\mathrm{min}}}
\newcommand{\argmax}{\mathop{\mathrm{arg}\,\mathrm{max}}}
%\newcommand{\capr}{\mathop{\cap\,}}
%\newcommand{\cupr}{\mathop{\cup\,}}
%\def\argmin{\mathop{arg\,min}}

\def\vrp{\varphi}
\def\prt{\partial}
\def\mm{{\sf M}}
\def\modnop#1{\mathop{#1}\limits_{n}}
\def\eam{\mathbin{{\mathop{=}\limits^{\mathrm{def}}}}}
\def\dey#1#2{#1 (#2)}
\def\deyc#1#2{#1 \cdot  #2}
\def\ra#1{\;\mathop{\to}\limits^{#1}\;}
\def\raz#1{\;\mathop{\longrightarrow}\limits^{\!\!\!#1}\;}
\def\ral#1{\;\mathop{\longrightarrow}\limits^{#1}\;}

\newcommand{\Nor}{\mathcal{N}}
\newcommand{\T}{\mathbb{T}}
\newcommand{\Z}{\mathbb{Z}}



\newcommand{\il}[2]{\int\limits_{#1}^{#2}}%интеграл с пределами #1 и #2

\def\sm2{\mathop {\sum\limits^{n^\Theta}\sum\limits^{n^\Theta}}}
\def\sss{\sum\limits}
\def\tr{,\,\ldots\,,\,}
\def\rk{\right]}
\def\lk{\left[}
\def\rf{\right\}}
\def\lf{\left\{}
\def\lv{\,\left\vert}
\def\rv{\right\vert\,}
\def\iii{\int\limits}
\def\iin{\int\limits_{-\infty}^\infty}
\def\rrv{\right\vert}


\def\ee{{\cal E}}
\def\ww{{\cal W}}
\def\yy{{\cal Y}}
\def\vv{{\cal V}}

\newcommand{\R}{\mathbb R}
\newcommand{\E}{\mathbb E}
\newcommand{\N}{\mathbb N}

\renewcommand{\P}{\mathbb{P}}

\newcommand{\h}{{\bf H}}
\newcommand{\p}{{\sf P}}  % вероятность

\newcommand{\e}{{\sf E}}  % мат. ожидание
\newcommand{\D}{{\sf D}}  % дисперсия
\newcommand{\eps}{\varepsilon}
\newcommand{\vp}{{\mathbf p}}
\newcommand{\vz}{{\mathbf z}}
\newcommand{\vx}{{\mathbf x}}
\newcommand{\vf}{{\mathbf f}}
\newcommand{\F}{{\mathcal F}}
\def\ap{{\mathrm{ЭР}}}
\newcommand{\ud}{\Delta_n} %uniform ditance
\newcommand{\nud}{\Delta_n(x)}
%\renewcommand{\Re}{\mathrm{Re}\,}

\newcommand{\abs}[1]{\left\vert#1\right\vert}

\newcommand{\norm}[1]{\left\Vert#1\right\Vert}
\def\da{(\Delta_t,A)}

\newcommand{\corr}{\mathrm{corr}}

\newcommand{\cov}{\mathrm{cov}}
\newcommand{\Expect}{\mathbb{E}}

\def\w{\omega}
\def\W{\Omega}

\def\inh{\int\limits_{nh}^{(n+1)h}}

\def\sumin{\sum_{i=1}^N}


\def\bxt{(Y,t)}
\def\xt{(y,t)}

\def\ovth{{\fr{\tau-nh}{h}}}
\def\ov{\overline}
\def\tm{\tilde m}
\def\tl{\tilde\lambda}
\def\tB{\widetilde B}
\def\tb{\tilde b}
\def\ld{\ldots}
\def\cd{\cdots}


\DeclareMathOperator{\sign}{sign}

%\newcommand{\gr}{{\geqslant}}


\newcommand{\g}{\mbox{\textit{g}}}

\renewcommand{\la}{\lambda}
\newcommand{\si}{\sigma}
\newcommand{\alp}{\alpha}

\newcommand{\pto}{\stackrel{P}{\longrightarrow}} % сходимость по веpоятности

\newcommand{\eqd}{\stackrel{\mathrm{d}}{=}} % равенство по pаспpеделению
\newcommand{\eqdelta}{\stackrel{\triangle}{=}} % равенство по pаспpеделению

\def\be#1{\begin{equation}\label{#1}}
\def\ee{\end{equation}}
\def\re#1{(\ref{#1})}

\def\bn{\begin{enumerate}}
\def\en{\end{enumerate}}
\def\bi{\begin{itemize}}
\def\ei{\end{itemize}}
%\def\i{\item}

%\newcommand{\kp}{\kappa}
%\def\Q{{\cal Q}} \def\H{{\cal H}}
%\newcommand{\bet}{\beta_{2+\delta}}


%\newtheorem{definition}{Определение}
%\renewcommand{\thedefinition}{\arabic{definition}.}
%END NEW COMMANDS

%\renewcommand{\baselinestretch}{1.2}

%\pagestyle{myheadings}

\setlength{\textwidth}{167mm}      % 122mm
\setlength{\textheight}{658pt}
%\setlength{\textheight}{635.6pt}
\setlength{\columnsep}{4.5mm}

\setcounter{secnumdepth}{4}

%\addtolength{\headheight}{2pt}
%\addtolength{\headsep}{-2mm}

\addtolength{\topmargin}{-7mm}  % for printing


%\hoffset=-30mm  % From Yap
\hoffset=-23mm  % From Acrobat

%\voffset=0mm % From Yap
\voffset=-5mm   % From Acrobat

%\addtolength{\evensidemargin}{-2.5mm} % for printing
%\addtolength{\oddsidemargin}{2.5mm}  % for printing

\addtolength{\evensidemargin}{-12mm} % for printing
\addtolength{\oddsidemargin}{8mm}  % for printing

%\renewcommand{\thefootnote}{\fnsymbol{footnote}}
%\renewcommand{\thefootnote}{\arabic{footnote}}
\renewcommand{\figurename}{\protect\bf Рис.}
\renewcommand{\tablename}{\protect\bf Таблица}

\newcommand{\Caption}[1]{\caption{\protect\small %\baselineskip=2.5ex
#1}}

\renewcommand{\thefigure}{\arabic{figure}}
\renewcommand{\thetable}{\arabic{table}}
\renewcommand{\theequation}{\arabic{equation}}
\renewcommand{\thesection}{\arabic{section}}

\renewcommand{\contentsname}{СОДЕРЖАНИЕ}
\newcommand{\fr}[2]{\displaystyle\frac{\displaystyle #1\mathstrut}{\displaystyle #2\mathstrut}}

%\renewcommand{\thefootnote}{\fnsymbol{footnote}}
%\newcommand{\g}{\mbox{\textit{g}}}

%\newcommand{\Caption}[1]{\caption{\protect\small\baselineskip=2ex #1}}
\newcounter{razdel}
\setcounter{razdel}{0}

\def\god{2021}
\def\tom{15}
\def\vyp{1}


\newcommand{\titel}[4]{%
\

\vspace*{5pt}

\ifodd\therazdel {\raggedright\noindent\Large\textrm\textbf
 \lineskip .75em
  \baselineskip=3.2ex #1 \par}
\vskip 1em {\noindent\large\textrm\textbf #2 \par}
\addcontentsline{toc}{subsection}{{\textrm\textbf #1}\protect\newline #2}
\def\rightheadline{\underline{\noindent\hbox to \textwidth{\hfill\small\textrm{#4}
%\hfill \large\bf\thepage
}}}
\def\leftheadline{\underline{\noindent\parbox{\textwidth}{
%\raggedleft\large\bf\thepage \hfill
\small\textit{#3}\hfill}}}
\def\leftfootline{\small{\textbf{\thepage}
\hfill ИНФОРМАТИКА И ЕЁ ПРИМЕНЕНИЯ\ \ \ том~\tom\ \ \ выпуск~\vyp\ \ \ \god}
}%
 \def\rightfootline{\small{ИНФОРМАТИКА И ЕЁ ПРИМЕНЕНИЯ\ \ \ том~\tom\ \ \ выпуск~\vyp\ \ \ \god
\hfill \textbf{\thepage}}}
\vskip 2em \setcounter{figure}{0}
\setcounter{table}{0}
\setcounter{equation}{0}
\setcounter{section}{0}
\setcounter{subsection}{0}
\setcounter{subsubsection}{0}
\setcounter{footnote}{0}
\setcounter{razdel}{0}
%\end{flushleft}
\else {
 \raggedright\noindent\Large\textrm\textbf
 \lineskip .75em
\baselineskip=3.2ex #1 \par} \vskip 1em
%\begin{flushleft}
{\noindent\large\textrm\textbf #2 \par}
\addcontentsline{toc}{subsection}{{\textrm\textbf #1}\protect\newline #2}
\def\rightheadline{\underline{\noindent\hbox to \textwidth{\hfill\small\textrm{#4}
%\hfill \large\bf\thepage
}}}
\def\leftheadline{\underline{\noindent\parbox{\textwidth}{%\raggedleft\large\bf\thepage \hfill
\small\textit{#3}\hfill}}}
\def\leftfootline{\small{\textbf{\thepage}
\hfill ИНФОРМАТИКА И ЕЁ ПРИМЕНЕНИЯ\ \ \ том~\tom\ \ \ выпуск~\vyp\ \ \ \god}
}%
 \def\rightfootline{\small{ИНФОРМАТИКА И ЕЁ ПРИМЕНЕНИЯ\ \ \ том~15\ \ \ выпуск~\vyp\ \ \ 2021
\hfill \textbf{\thepage}}} \vskip 2em \setcounter{figure}{0}
\setcounter{table}{0} \setcounter{equation}{0} \setcounter{section}{0}
\setcounter{subsection}{0} \setcounter{subsubsection}{0}
\setcounter{footnote}{0}
%\end{flushleft}
\fi}

\newcommand{\titelr}[2]{%
\

\vspace*{5pt}

\ifodd\therazdel {\raggedright\noindent%\Large\textrm\textbf
 \lineskip .75em
  \baselineskip=3.2ex #1 \par}
\vskip 1em {\noindent\normalsize\textrm\textbf #2 \par}
\else {
 \raggedright\noindent\Large\textrm\textbf
 \lineskip .75em
\baselineskip=3.2ex #1 \par} \vskip 1em
%\begin{flushleft}
{\noindent\large\textrm\textbf #2 \par
%\noindent\normalsize\textrm\textbf #2 \par
} \fi}

\newcommand{\titele}[5]{%
\

%\vspace*{5pt}

\ifodd\therazdel {\raggedright\noindent\large
\textrm\textbf
 \lineskip .75em
%  \baselineskip=3.2ex
#1 \par}
\vskip .5em {\noindent\large\textrm\textbf #2 \par}
\vskip .5em
 {\noindent\textrm #3 \par}
\addcontentsline{toc}{subsection}{{\textrm\textbf #1}\protect\newline #2}
\def\rightheadline{\underline{\noindent\hbox to \textwidth{\hfill\small\textrm{#4}
%\hfill \large\bf\thepage
}}}
\def\leftheadline{\underline{\noindent\parbox{\textwidth}{
%\raggedleft\large\bf\thepage \hfill
\small\textrm{#5}\hfill}}}
\def\leftfootline{\small{\textbf{\thepage}
\hfill ИНФОРМАТИКА И ЕЁ ПРИМЕНЕНИЯ\ \ \ том~15\ \ \ выпуск~1\ \ \ 2021}
}%
 \def\rightfootline{\small{ИНФОРМАТИКА И ЕЁ ПРИМЕНЕНИЯ\ \ \ том~15\ \ \ выпуск~1\ \ \ 2021
\hfill \textbf{\thepage}}} \vskip 1em \setcounter{figure}{0}
\setcounter{table}{0} \setcounter{equation}{0} \setcounter{section}{0}
\setcounter{subsection}{0} \setcounter{subsubsection}{0}
\setcounter{footnote}{0} \setcounter{razdel}{0}
%\end{flushleft}
\else {
 \raggedright\noindent\large
 \textrm\textbf
 \lineskip .75em
%\baselineskip=3.2ex
#1 \par} \vskip .5em
%\begin{flushleft}
{\noindent\large\textrm\textbf #2 \par} \vskip .5em
 {\noindent\textrm #3 \par}
\addcontentsline{toc}{subsection}{{\textrm\textbf #1}\protect\newline #2}
\def\rightheadline{\underline{\noindent\hbox to \textwidth{\hfill\small\textrm{#4}
%\hfill \large\bf\thepage
}}}
\def\leftheadline{\underline{\noindent\parbox{\textwidth}{%\raggedleft\large\bf\thepage \hfill
\small\textrm{#5}\hfill}}}
\def\leftfootline{\small{\textbf{\thepage}
\hfill ИНФОРМАТИКА И ЕЁ ПРИМЕНЕНИЯ\ \ \ том~15\ \ \ выпуск~1\ \ \ 2021}
}%
 \def\rightfootline{\small{ИНФОРМАТИКА И ЕЁ ПРИМЕНЕНИЯ\ \ \ том~15\ \ \ выпуск~1\ \ \ 2021
\hfill \textbf{\thepage}}} \vskip 1em \setcounter{figure}{0}
\setcounter{table}{0} \setcounter{equation}{0} \setcounter{section}{0}
\setcounter{subsection}{0} \setcounter{subsubsection}{0}
\setcounter{footnote}{0}
%\end{flushleft}
\fi}

\def\Abst#1{
\begin{center}\small\nwt
\parbox{150mm}{%\baselineskip=2.5ex
\textbf{Аннотация:}\ \
%\hspace*{\parindent}
#1}
\end{center}}
\def\Abste#1{
\begin{center}\small\nwt
\parbox{150mm}{%\baselineskip=2.5ex
\textbf{Abstract:}\ \
%\hspace*{\parindent}
#1}
\end{center}}

\def\DOI#1{
\begin{center}\small\nwt
\parbox{150mm}{%\baselineskip=2.5ex
\textbf{DOI:}\ \
%\hspace*{\parindent}
#1}
\end{center}}

\def\Abstend#1{
\begin{center}\small\nwt
\parbox{150mm}{%\baselineskip=2.5ex
%\hspace*{\parindent}
#1}
\end{center}}


\def\KW#1{
\begin{center}\small\nwt
\parbox{150mm}{%\baselineskip=2.5ex
\textbf{Ключевые слова:}\ \ #1}
\end{center}}

\def\KWE#1{
\begin{center}\small\nwt
\parbox{150mm}{%\baselineskip=2.5ex
\textbf{Keywords:}\ \ #1}
\end{center}}


\def\KWN#1{
%\begin{center}
%\small
%\parbox{150mm}\end{center}
}

\newcommand{\Avtors}[1]{%\smallskip
%\vspace*{.5pt}
\hangindent=23pt\noindent
%\nwt
{\bfseries#1}\
}


\renewcommand{\thesubsection}{\thesection.\arabic{subsection}\hspace*{-5pt}}
\renewcommand{\thesubsubsection}{\thesubsection\hspace*{5pt}.\arabic{subsubsection}\hspace*{-3pt}}

\newcommand{\Ack}{\section*{\protect\rmfamily Acknowledgments}\noindent}
\newcommand{\Contr}{\section*{\protect\rmfamily Contributors}\noindent}
\newcommand{\Contrl}{\section*{\protect\rmfamily Contributor}\noindent}

\makeindex


\begin{document}
\Rus

\nwt
%\ptb


%\renewcommand{\contentsname}{\protect\Large\bf Содержание}

\setcounter{tocdepth}{2}

%\tableofcontents

\renewcommand{\bibname}{\protect\rmfamily Литература}
  \def\Au#1{{\it #1}}
    \def\Aue#1{{#1}}

%\newcommand{\No}{№}
  \newcommand{\tg}{\,\mathrm{tg}\,}
    \newcommand{\ctg}{\,\mathrm{ctg}\,}
  \newcommand{\arctg}{\,\mathrm{arctg}\,}

\def\forallb{\mathop{\forall}}
\def\cupb{\mathop{\cup}}
\def\existsb{\mathop{\exists}}


\newpage
\addtocounter{razdel}{1}
%\def\razd{РЕГУЛИРУЕМЫЙ ЭЛЕКТРОПРИВОД ДЛЯ ЭЛЕКТРОЭНЕРГЕТИКИ}


\setcounter{page}{3}

%{ %\Large  
{ %\baselineskip=16.6pt

\vspace*{-48pt}
\begin{center}\LARGE
\textit{Уважаемый читатель!}
\end{center}

%\vspace*{2.5mm}

\vspace*{4mm}

\thispagestyle{empty}

{\small

 
В~2017~г.\ исполняется 10~лет со времени выхода в~свет первого 
номера журнала <<Информатика и~её применения>>~--- 
научного журнала Российской академии наук, издающегося под 
на\-уч\-но-ме\-то\-ди\-че\-ским руководством Отделения нанотехнологий 
и~информационных технологий Российской академии наук. Учредителем журнала 
является Федеральный исследовательский центр <<Информатика и~управ\-ле\-ние>> 
Российской академии наук (ФИЦ ИУ РАН) (до~2015~г.~--- 
Институт проб\-лем информатики РАН).

Необходимость издания такого журнала была вызвана активным развитием 
информатики и~информационных технологий, большой важностью этого научного 
направления для развития страны, проникновением информационных технологий 
во все сферы жизни современного общества.

Тематику журнала определяет тот факт, что информатика~--- это комплексная 
фундаментальная научная дисциплина, опирающаяся на достижения 
ряда других наук, в~том числе математики, физики, лингвистики и~др. 
Одновременно журнал уделяет большое внимание современным информационным технологиям, 
являющимся приложениями результатов информатики как фундаментальной науки.

За прошедшие 10~лет (2007--2016~гг.)\ издано~38~выпусков журнала. В~них 
размещено~452~публикации, в~том числе~430~научных статей и~22~информационных 
публикации (обзоры, рецензии и~др.). Среди авторов журнала представители ведущих 
научных организаций и~университетов страны, в~том числе Московского государственного 
университета им.\ М.\,В.~Ломоносова, ФИЦ ИУ РАН (в~том числе ИПИ РАН, ВЦ 
им.\ А.\,А.~Дородницына РАН, ИСА РАН), Института точной механики и~вычислительной 
техники им.\ С.\,А.~Лебедева РАН, Института космических исследований РАН, 
Института астрономии РАН, ряда институтов Сибирского отделения РАН, МФТИ, МИФИ, 
Высшей школы экономики, Санкт-Пе\-тер\-бург\-ско\-го государственного университета, 
Санкт-Пе\-тер\-бург\-ско\-го государственного политехнического университета 
Петра Великого, Санкт-Пе\-тер\-бург\-ско\-го государственного университета 
телекоммуникаций им.\ проф.\ М.\,А.~Бонч-Бруе\-ви\-ча, 
Российского университета дружбы народов, Балтийского федерального университета 
имени Иммануила Канта, Вологодского государственного университета и~др. 
Публиковались статьи зарубежных авторов, в~том числе ученых из Израиля, 
США, Финляндии, Франции, Швейцарии, Швеции и~других стран. 

В конце настоящего выпуска журнала помещен указатель статей, 
опуб\-ли\-ко\-ван\-ных в~томах~1--10 (2007--2016~гг.).

Журнал включен в~Российский индекс научного цитирования и~в~базу 
данных RSCI Web of Science, перечень ВАК, базу данных CrossRef 
и~информационную систему <<Общероссийский математический портал MathNet>>. 
С~2015~г.\ журнал индексируется в~библиографической и~реферативной базе 
данных SCOPUS.

Мы всегда будем помнить ушедших из жизни членов редакционного совета 
и~редакционной коллегии журнала: академика С.\,К.~Коровина, профессоров 
А.\,В.~Печинкина и~И.\,А.~Ушакова, которые внесли неоценимый вклад в~становление 
и~развитие журнала.

После объединения в~2015~г.\ трех учреждений Российской академии наук~--- 
Института проблем информатики, Вычислительного центра им.\ А.\,А.~Дородницына 
и~Института системного анализа~--- в~Федеральное государственное учреждение 
<<Федеральный исследовательский центр <<Информатика и~управ\-ле\-ние>> 
Российской академии наук>> (ФИЦ ИУ РАН) именно этот Центр стал базовой организацией 
для издания журнала, что существенно расширило как тематику журнала, 
так и~его возможности по привлечению новых авторов, в~том числе и~зарубежных.

В настоящее время тематику журнала в~первую очередь составляют:
\begin{itemize}
\item    теоретические основы информатики;\\[-14.5pt] 
\item    математические методы исследования сложных систем и~процессов;\\[-14.5pt]
\item    информационные системы и~сети;\\[-14.5pt]
\item    информационные технологии;\\[-14.5pt]
\item    архитектура и~программное обеспечение вычислительных комплексов и~сетей. 
\end{itemize}

Эти направления особенно важны в~связи с необходимостью решения задач 
формирования технологической базы инновационного развития, обеспечения 
на\-уч\-но-тех\-но\-ло\-ги\-че\-ско\-го прорыва в~области создания и~развития 
отечественных информационных и~коммуникационных технологий в~интересах 
достижения высокого качества и~стабильности систем управления и~предоставления 
услуг в~экономической и~социальной сферах. 

Мы, как и~ранее, приглашаем авторов представлять для публикации в~журнале 
статьи как с достижениями в~области теоретических проблем информатики, так 
и~с~изложением результатов ее практического приложения, а~также 
рецензии на наиболее интересные книжные новинки в~области информатики 
и~информационных технологий, объявления о~крупнейших международных 
и~всероссийских конференциях, различных научных мероприятиях 
по этой тематике и~другие информационные материалы.

Надеемся, что и~в~дальнейшем содержание статей, помещаемых в~журнале, 
будет вызывать интерес научной общественности. Редакционный совет, редколлегия 
и~редакция журнала, со своей стороны, сделают все для того, 
чтобы журнал и~впредь своевременно и~подробно информировал читателей 
о~новейших достижениях информатики и~ее актуальных практических приложениях.

                

      
\vfill
\noindent
Главный редактор журнала <<Информатика и~её применения>>,\\
академик  РАН\hfill
\textit{И.\,А.~Соколов}\\[-6pt]

%\noindent
%Редактор-составитель тематического выпуска, профессор кафедры математической статистики\\
%факультета вычислительной математики и~кибернетики МГУ им.~М.\,В.~Ломоносова,\\
%ведущий научный сотрудник ИПИ РАН, доктор физико-математических наук\hfill
% \textit{В.\,Ю.~Королев}


} }
}
      

\def\stat{sinit}

\def\tit{АНАЛИТИЧЕСКОЕ МОДЕЛИРОВАНИЕ РАСПРЕДЕЛЕНИЙ 
С~ИНВАРИАНТНОЙ МЕРОЙ В~СТОХАСТИЧЕСКИХ СИСТЕМАХ 
С~АВТОКОРРЕЛИРОВАННЫМИ ШУМАМИ$^*$}

\def\titkol{Аналитическое моделирование распределений 
с~инвариантной мерой в~стохастических системах 
с %автокоррелированными 
шумами}

\def\autkol{И.\,Н.~Синицын}

\def\aut{И.\,Н.~Синицын$^1$}

\titel{\tit}{\aut}{\autkol}{\titkol}

{\renewcommand{\thefootnote}{\fnsymbol{footnote}}\footnotetext[1]
{Работа поддержана РФФИ (проект № 10-07-00021) и Программой <<Интеллектуальные информационные 
технологии, системный анализ и автоматизация>> (проект~1.7).}}

\renewcommand{\thefootnote}{\arabic{footnote}}
\footnotetext[1]{Институт проблем информатики Российской академии наук, sinitsin@dol.ru}     


\vspace*{-9pt}
     
  \Abst{Для многомерных нелинейных гауссовских (нормальных) 
дифференциальных сис\-тем с некоррелированными и автокоррелированными 
помехами на базе метода нормальной аппроксимации разработаны 
корреляционные алгоритмы аналитического моделирования стохастических 
режимов с инвариантной мерой. На тес\-то\-вых примерах с по\-мощью 
инструментального программного обеспечения в среде MATLAB показана 
достаточная для многих приложений точность алгоритмов.}

\vspace*{-3pt}
  
  \KW{автокоррелированная помеха; аналитическое моделирование; 
корреляционный алгоритм; метод нормальной аппроксимации; многомерная 
нелинейная дифференциальная стохастическая сис\-те\-ма; распределение с 
инвариантной мерой}

\vspace*{-6pt}

\vskip 14pt plus 9pt minus 6pt

      \thispagestyle{headings}

      \begin{multicols}{2}

            \label{st\stat}
     
\section{Введение}
     
    Будем рассматривать в общем случае неста\-ционарный стохастический 
режим $Z\hm=Z(t)$,\linebreak явля\-ющий\-ся сильным решением следующего нормального 
(гауссовского) стохастического дифференциального урав\-не\-ния Ито~[1, 2]:
    \begin{equation}
    \dot{Z}=a(Z,t)+b(Z,t)V\,,\enskip Z(t_0)=Z_0\,.
    \label{e1-sin}
    \end{equation}
Здесь $Z$~--- $k$-мер\-ный вектор со\-сто\-яния $Z\hm\in \Delta$ ($\Delta$~--- 
многообразие со\-сто\-яний), $a\hm=a(Z,t)$ и $b\hm=b(Z,t)$~--- 
детерминированные ($k\times1$)- и ($k\times m)$-функ\-ции отмечен\-ных 
аргументов, $V\hm=V(t)$~---\linebreak $m$-мер\-ный вектор нормально распределенных 
\mbox{белых} шумов с нулевыми математическими ожиданиями и 
($m\times m$)-мат\-ри\-цей интенсивностей $v\hm=v(t)$ и пред\-став\-ля\-ющий 
собой среднеквадратичную\linebreak производную винеровского процесса $W\hm=W(t)$, %\linebreak 
$V\hm=\dot{W}$. Начальное со\-сто\-яние~$Z_0$ будем считать нормальной 
(гауссовой) случайной величиной, не зависящей от приращений винеровского 
процесса $W(t)$ для $t\hm>t_0$.
    
    Если существуют все многомерные плот\-ности вектора со\-сто\-яния~$Z$, то, 
определив сначала одномерную плот\-ность $f_1\hm=f_1(z;t)$ и переходную 
плот\-ность $f\hm=f(z;t\vert \xi;\tau)$ путем интегрирования урав\-не\-ния 
    Фок\-ке\-ра--План\-ка--Кол\-мо\-го\-ро\-ва (ФПК) с соответствующими 
начальными услови\-ями~[1, 2]:

\noindent
    \begin{multline}
%    \left.
%    \begin{array}{c}
    \fr{\partial f_1}{\partial t} = -\fr{\partial^{\mathrm{T}}}{\partial z}\left( af_1\right) 
+\fr{1}{2}\,\mathrm{tr}\left[ \fr{\partial}{\partial z}\,\fr{\partial^{\mathrm{T}}}{\partial z}\left( 
\sigma f_1\right) \right]\,,\\
    \sigma=b\nu b^{\mathrm{T}}\,,\enskip f_1(z;t_0)=f_0(z)\,;
 %   \end{array}
%    \right\}
    \label{e2-sin}
    \end{multline}
    
%    \vspace*{-9pt}
    
    \noindent
    \begin{multline}
%    \left.
%    \begin{array}{c}
    \fr{\partial f}{\partial t} = -\fr{\partial^{\mathrm{T}}}{\partial z}\left( af\right) 
+\fr{1}{2}\,\mathrm{tr}\left[ \fr{\partial}{\partial z}\,\fr{\partial^{\mathrm{T}}}{\partial z}\left( 
\sigma f\right)\right]\,,\\
    f(z;t\vert \xi;\tau)=\delta(z-\xi)\,,
 %   \end{array}
%    \right\}
    \label{e3-sin}
    \end{multline}
можно найти все многомерные плот\-ности $f_n\hm= f_n(z_1, \ldots , z_n; 
t_1,\ldots , t_n)$ по рекуррентной фор\-муле:

\vspace*{-3pt}

\noindent
\begin{multline*}
f_n=f_n(z_1,\ldots ,z_n; t_1,\ldots ,t_n) ={}\\
{}=f_1(z_1;t_1)f(z_2;t_2\vert z_1;t)\cdots f(z_n;t_n\vert z_{n-1};t_{n-1})\,,\\
t_1\leq t_2\leq \cdots \leq t_n\,, \ 
n=2,3,\ldots
%\label{e4-sin}
\end{multline*}


    
    Для нахождения стационарных в узком смысле одно- и многомерных 
распределений стохастических режимов в стохастических сис\-те\-мах (СтС), 
определяемых~(\ref{e2-sin}), в~(\ref{e3-sin}) следует положить $\partial  
f_1/\partial  t\hm=0$. 
    
    В~[1--12] рассмотрены точные методы аналитического моделирования 
одно- и многомерных плот\-но\-стей стохастических стационарных и 
нестационарных режимов, основанные на построении интегральных 
инвариантов специально подобранных обыкновенных дифференциальных 
уравнений. В~[13] на основе теории потенциала предложены методы расчета 
стационарных распределений с инвариантной мерой для гамильтоновых СтС в 
стохастической среде.
    
    Стохастические сис\-те\-мы с автокоррелированными помехами обычно 
описывают следующими уравнениями Ито~[1, 2]:
%\noindent
    \begin{equation}
    \dot{Z}=a(Z,t)+b_U(z,t)U\,.
    \label{e5-sin}
    \end{equation}
    
%    \pagebreak
    
    
    \noindent
    Здесь векторная помеха~$U$ удовлетворяет следующему стохастическому 
линейному дифференциальному уравнению фильтра, формирующего~$U$ из 
гауссового (нормального) белого шума~$V$ интенсивности $\nu\hm=\nu(t)$:
    \begin{equation}
    \sum\limits_{i=0}^l \alpha_i U^{(i)} =\sum\limits_{j=0}^l \beta_j 
V^{(j)}\enskip (h<l)\,,
    \label{e6-sin}
    \end{equation}
где $\alpha_i$ и $\beta_j$~--- коэффициенты формирующего фильт\-ра (ФФ), в 
общем случае зависящие от времени.
    
    Поставим задачу разработки на основе метода нормальной аппроксимации 
(МНА)~[1, 2] корреляционных алгоритмов аналитического моделирования 
одно- и двумерных нормальных (гауссовских) плот\-но\-стей стохастических 
режимов $Z\hm=Z(t)$ в СтС~(1) и~(\ref{e5-sin}) с инвариантной мерой, т.\,е.\ 
удовлетворяющих условию: 
    \renewcommand{\theequation}{\arabic{equation}$^\prime$}
    \begin{equation}
    \fr{\partial  f_1(z;t)}{\partial t} +\fr{\partial^{\mathrm{T}}}{\partial z}\left[ 
a(z,t)f_1(z;t)\right]=0\,,
     \label{e7-1-sin}
     \end{equation}
     \setcounter{equation}{5}
     
\noindent
где $f_1=f_1(z;t)$~--- одномерная плот\-ность распределения. Для стационарных 
режимов условие~(\ref{e7-1-sin}) принимает вид 
    \renewcommand{\theequation}{\arabic{equation}$^{\prime\prime}$}
    \begin{equation}
    \fr{\partial^{\mathrm{T}}}{\partial z} \left[ a^*(z)f_1^*(z)\right] =0\,,
    \label{e7-2-sin}
    \end{equation}
    \renewcommand{\theequation}{\arabic{equation}}
    \setcounter{equation}{6}
  
  Рассмотрим случаи некоррелированной помехи (разд.~2) и 
автокоррелированной помехи, явля\-ющей\-ся менее (более) гладкой, чем 
стохастический режим (разд.~3).
  
\section{Аналитическое моделирование при~некоррелированных 
помехах}
  
  Пусть нелинейная СтС (1) допускает применение МНА~[1, 2]. Тогда 
одномерная нормальная плот\-ность $f_1^N(z;t)$, вектор математического 
ожидания $m_t\hm={\sf M}Z(t)$, ковариационная мат\-ри\-ца $K_t\hm= {\sf M}X^{0\mathrm{T}}(t) 
Z^0(t)$ и мат\-ри\-ца ковариационных\linebreak функций 
$K(t_1,t_2)\hm={\sf M}Z^{0\mathrm{T}}(t_1)Z^0(t_2)$ $(t_1\hm<t_2)$ опре\-де\-ля\-ют\-ся 
следующими урав\-не\-ни\-ями~[1, 2]:
  \begin{multline}
  f_1^N(z;t,m_t,K_t) =\left[ (2\pi)^k\vert K_t\vert \right]^{-1/2}\times{}\\
  {}\times \exp \left\{ -
\fr{1}{2} \left( z^{\mathrm{T}}-m_t^{\mathrm{T}}\right) K_t^{-1}(z-m_t)\right\}\,;\label{e8-sin}
  \end{multline}
  
  \vspace*{-12pt}
  
  \noindent
  \begin{multline}
     \dot{m}_t=a_1(m_t,K_t,t)={}\\
     {}=\int\limits_{-\infty}^\infty a(z,t)f_1^N 
(z;t,m_t,K_t)\,dz\,;\label{e9-sin}
     \end{multline}
     
     \noindent
     \begin{multline}
     \dot{K}_t =a_2(m_t,K_t,t)=\int\limits_{-\infty}^\infty \left[ a(z,t)(z^{\mathrm{T}}-
m_t^{\mathrm{T}})+{}\right.\\
\hspace*{-1mm}{}+(z-m_t)a^{\mathrm{T}}(z,t)+ %{}\\
\left.\sigma(z,t)\right] f_1^N(z,t,m_t,K_t)\,dz,\!
     \label{e10-sin}
     \end{multline}
     где
     \begin{equation*}
     \sigma(z,t)=b(z,t)\nu(t) b(z,t)^{\mathrm{T}}\,;
     \end{equation*}
     
     \vspace*{-12pt}
     
     \noindent
     \begin{multline}
     \fr{\partial K(t_1,t_2)}{\partial t_2}=a_3(m_{t_1},m_{t_2}, K(t_1,t_2))={}\\
     {}=\left[ 
(2\pi)^{2k}\vert \overline{K}_2\vert \right]^{-1/2}\int\limits_{-\infty}^\infty  
\int\limits_{-\infty}^\infty (z_1-m_{t_1})a(z_{t_1},t_2)\times{}\\
     {}\times 
     \exp \left\{ -\left(\left [ z_1^{\mathrm{T}}\ z_2^{\mathrm{T}}\right] -
     \overline{m}_2^{\mathrm{T}}\right) 
\overline{K}_2^{-1} \times{}\right.\\
\left.{}\times\left( \left[ z_1^{\mathrm{T}}\ z_2^{\mathrm{T}}\right]^{\mathrm{T}} -\overline{m}_2\right) 
\right\}dz_1dz_2\,,
     \label{e11-sin}
     \end{multline}
где
\begin{gather*}
z_1= z_1(t_1)\,;\  z_2=z_2(t_2)\,;\  \overline{m}_2=\left[ 
m^{\mathrm{T}}_{t_1},m^{\mathrm{T}}_{t_2}\right]\,;\\
\overline{K}_2=\begin{bmatrix}
K(t_1,t_1) & \ \ K(t_1,t_2)\\[6pt]
K(t_2,t_{1})&\ \  K(t_2,t_2)
\end{bmatrix}\,.
\end{gather*}
     
     Для стационарной сис\-те\-мы~(1) уравнения для $m_t\hm=m^*\hm=const$, 
$K_t\hm=K^*\hm=const$ получаются из (\ref{e9-sin})--(\ref{e11-sin}) при 
усло\-ви\-ях $\dot{m}_t\hm=0$, $\dot{K}_t\hm=0$:
     \begin{equation}
     a_1^* (m^*,K^*)=0\,;\quad a_2^*(m^*,K^*)=0\,;\label{e12-sin}
     \end{equation}
     \begin{equation}
     \left.
     \begin{array}{c}
     \fr{dk(\tau)}{d\tau} =k_1^a(m^*, K^*)k(\tau)\,;\\[9pt]
      k(\tau)=k(-\tau)^{\mathrm{T}}\,;\ 
k(0)=K\,.
\end{array}
\right\}
\label{e13-sin}
     \end{equation}
    Здесь $k_1^a=k_1^a(m_t,K_t)$ представляет собой коэффициент 
статистической гауссовской линеаризации нелинейной функ\-ции
  \begin{equation*}
  a(Z,t) =a_1(m_t,K_t)+k_1^a(m_t,K_t)(Z-m_t)\,,
%  \label{e14-sin}
  \end{equation*}
причем при $m_t=m^*$, $K_t\hm=K^*$ он определяется формулой:
\begin{equation*}
k_1^a=a_{21}(m_t,K_t) K_t^{-1}\,,
%\label{e15-sin}
\end{equation*}
где
\begin{multline*}
a_{21}=a_{21}(m_t,K_t) ={}\\
{}=\!\!\!\int\limits_{-\infty}^\infty\!\! a(z,t)(z^{\mathrm{T}}-m_t^{\mathrm{T}}) 
f_1^N(z;t,m_t,K_t)\,dz%={}\\
= \!\left[ \fr{\partial}{\partial m_t}\,a_1^{\mathrm{T}}\right]^{\mathrm{T}}\!\!\!\!.\hspace*{-7.34363pt}
%\label{e16-sin}
\end{multline*}
      \renewcommand{\theequation}{\arabic{equation}$^{\prime}$}
  
  \noindent
  При этом условия (\ref{e7-1-sin}) и~(\ref{e7-2-sin}) примут следующий вид:
  \pagebreak
  
  \noindent
  \begin{multline}
  \fr{\partial f_1^N (z;t,m_t,K_t)}{\partial t}+\fr{\partial^{\mathrm{T}}}{\partial z}\left\{ \left[ 
a_1(t,m_t,K_t, t) +{}\right.\right.\\
\hspace*{-2.5mm}\left.\left.{}+k_1^a(t,m_t,K_t,t)(z-m_t)\right] f_1^N (z; t,m_t,K_t)\right\}=0,\!\!\!\!\!
  \label{e17-1-sin}
  \end{multline}
    \renewcommand{\theequation}{\arabic{equation}$^{\prime\prime}$}
    \setcounter{equation}{12}
    
\vspace*{-12pt}

    \noindent
    \begin{multline}
    \fr{\partial^{\mathrm{T}}}{\partial z}\left\{\left[ a_1^*(m^*,K^*)+{}\right.\right.\\
\hspace*{-3mm}\left.\left.    {}+k_1^a(m^*,K^*)(z-
m^*)\right] f_1^N (z;m^*,K^*)\right\}=0.
    \label{e17-2-sin}
  \end{multline}
    \renewcommand{\theequation}{\arabic{equation}}
    \setcounter{equation}{13}
  
  Таким образом, имеет место утверждение.
  
  \medskip
  
  \noindent
  \textbf{Теорема 1.} \textit{Если мат\-ри\-ца $k_1^a\hm=k_1^a(m_t,K_t)$ 
асимп\-то\-тически устойчива, то корреляционный алгоритм аналитического 
моделирования нестационарных режимов МНА в СтС}~(1) \textit{определяется 
уравнениями}~(\ref{e8-sin})--(\ref{e12-sin}), (\ref{e17-1-sin}), \textit{а для 
стационарных режимов~--- урав\-не\-ни\-ями} (\ref{e8-sin})--(\ref{e10-sin}), 
(\ref{e13-sin}), (\ref{e17-2-sin}).
  

\section{Аналитическое моделирование при~автокоррелированных 
помехах}
  
  Сначала рассмотрим СтС~(\ref{e5-sin}) и (\ref{e6-sin}) при 
  условиях~(\ref{e7-1-sin}) и~(\ref{e7-2-sin}) в предположении, что 
стохастический режим~$Z(t)$ является более гладким, чем помеха (имеет 
производные более высокого порядка, чем помеха). Преобразуем 
урав\-не\-ние~(\ref{e5-sin}), записанное в виде:
  \begin{equation}
  \mathrm{Э}\left( Z,U,t\right) =\dot{Z}=a(Z,t)+b_U U\,,
  \label{e18-sin}
  \end{equation}
сис\-те\-мой, обратной ФФ~(\ref{e6-sin}). Как известно~[1, 2], сис\-те\-ма, обратная к 
ФФ~(\ref{e6-sin}), представляет собой параллельное соединение сис\-те\-мы, 
осуществляющей линейную дифференциальную операцию
\begin{equation*}
L=\sum\limits_{k=0}^{l-h} \gamma_k D^k\,,\enskip D=\fr{d}{dt}\,,
%\label{e19-sin}
\end{equation*}
и сис\-те\-мы, описываемой дифференциальным урав\-не\-нием
\begin{equation}
\sum\limits_{i=0}^h \beta_i y^{(i)} =\sum\limits_{j=0}^{h-l} \tilde{\alpha}_j 
x^{(j)}\,.
\label{e20-sin}
\end{equation}
    Здесь введены следующие обозначения:
  \begin{align*}
%  \left.
%  \begin{array}{c}
  \gamma_{l-h}&=\beta_h^{-1}\alpha_l\,;\\
  \gamma_k&=\beta_h^{-1}\left[ 
\alpha_{h+k}-\hspace*{-5mm}\hspace*{-9.52328pt}\sum\limits^{h-1}_{r=\max(0,k-l+2h)} 
\sum\limits_{j=0}^{h-r} 
C^r_{r+j}\beta_{r+j}\gamma^{(j)}_{h+k-r}\right] \\
&\hspace*{35mm}(k=0,1,\ldots , l-h-1)\,;\\
  \tilde{\alpha}_k&=\alpha_k- \hspace*{-2mm}\sum\limits^k_{r=\max(0,k-l+2h)} \sum\limits_{j=0}^{h-r}
C^r_{r+j} \beta_{r+j} \gamma^{(j)}_{k-r}\\
&\hspace*{35mm}(k=0,1,\ldots , l-h-1)\,.
%  \end{array}
%  \right\}
 % \label{e21-sin}
  \end{align*}
  
  Пропустив сигнал $\mathrm{Э}(Z,t)$, определяемый 
  урав\-не\-ни\-ем~(\ref{e18-sin}), через сис\-те\-му, со\-сто\-ящую из усилителя с 
коэффициентом~$b_U^{-1}$ и сис\-те\-мы, обратной ФФ~(\ref{e6-sin}), получим 
на выходе сигнал
  \begin{equation*}
  \mathrm{Э}_1=L\left[ b_U^{-1} a(Z,t)\right] +Z_1^\prime +V\,,
%  \label{e22-sin}
  \end{equation*}
где $Z_1^\prime$~--- выходной сигнал сис\-те\-мы~(\ref{e20-sin}) при 
$\mathrm{Э}(Z,t)=b_U^{-1} a(Z,t)$:
\begin{equation}
\sum\limits_{i=0}^h \beta_i Z_1^{\prime(i)}= \sum\limits_{j=0}^{h-1} 
\tilde{\alpha}_j\left( b_U^{-1} a(Z,t)\right)^{(i)}\,.
\label{e23-sin}
\end{equation}
  
  Приведя уравнение~(\ref{e23-sin}) к сис\-те\-ме урав\-не\-ний первого порядка 
согласно~[1, 2], получим дифференциальное урав\-не\-ние, определяющее век\-тор 
$Z^\prime\hm= \left[ Z_1^{\prime \mathrm{T}}\cdots Z_h^{\prime \mathrm{T}}\right]^{\mathrm{T}}$:
  \begin{equation*}
  \dot{Z}^\prime = c(Z,Z^\prime)\,.
%  \label{e24-sin}
  \end{equation*}
  
  Наконец, введя расширенный век\-тор со\-сто\-яния $\overline{Z}\hm= \left[ Z^{\mathrm{T}}\, 
Z^{\prime T}\right]^{\mathrm{T}}$, получим урав\-не\-ние вида~(\ref{e1-sin}):
  \begin{equation}
  \dot{\overline{Z}}=\overline{a}\left(\overline{Z},t\right) +\overline{b}V\,,
  \label{e25-sin}
  \end{equation}
где
\begin{equation}
\overline{a}=\begin{bmatrix} a\\ c\end{bmatrix}\,;\qquad
\overline{b}= \begin{bmatrix} b_U\\ 0\end{bmatrix}\,.
\label{e26-sin}
\end{equation}
Таким образом, имеем следующий результат.
  
  \medskip
  
  \noindent
  \textbf{Теорема~2.} \textit{Если мат\-ри\-ца $k_1^{\overline{a}}$ 
асимптотически устойчива, а стохастический режим в}~(\ref{e5-sin}) 
\textit{является более гладким, чем помеха, то корреляционный алгоритм 
аналитического моделирования МНА определяется теоремой}~1 \textit{для 
уравнения}~(\ref{e25-sin}) \textit{при условии}~(\ref{e26-sin}).
  
  \smallskip
  
  В случае, когда стохастический режим не менее гладкий, чем помеха~$U$, 
достаточно применить процедуру расширения вектора со\-сто\-яния путем 
дифференцирования~(\ref{e18-sin}) и исключения помехи~$U$ и ее 
производных, не содержащих белого шума~$V$, из урав\-не\-ний 
  ФФ~(\ref{e6-sin}) с по\-мощью урав\-не\-ния~(\ref{e18-sin}) и урав\-не\-ний, 
полученных из него дифференцированием.
  
  С этой целью продифференцируем уравнение~(\ref{e18-sin}), умноженное 
слева на $b_U^{-1}$, до появления в нем белого шума. Тогда получим:
  \begin{multline}
  \fr{d^i}{dt^i}\left[ b_U^{-1}(t) \mathrm{Э}(Z,U,t)\right] ={}\\
  {}=\fr{d^i}{dt^i}\left[ 
b_U^{-1} (t) a(Z,t)\right] + U_{i+1}\enskip (i=0,1,\ldots ,s)\,,
  \label{e27-sin}
  \end{multline}
если
$$
\fr{d^s}{dt^s}\left[ b_U^{-1}(t) a(Z,t)\right] \ \mbox{при}\ s<l-h
$$
содержит белый шум, и
\begin{equation}
\left.
\begin{array}{rl}
\fr{d^i}{dt^i}\left[ b_U^{-1}(t)\mathrm{Э}(Z,U,t)\right] &={}\\[9pt]
&\hspace*{-56.57764pt}{}= \fr{d^i}{dt^i}\left[ 
b_U^{-1}(t)a(Z,t)\right]+U_{i+1}\\[9pt]
&\hspace*{-19pt}(i=0,1,\ldots , l-h-1)\,;\\[9pt]
\fr{d^{l-h}}{dt^{l-1}}\left[ b_U^{-1}(t)\mathrm{Э}(Z,U,t)\right] &={}\\
&\hspace*{-106.57764pt}{}= \fr{d^{l-
h}}{dt^{l-h}}\left[ b_U^{-1}(t)a(Z,t)\right]+U_{l-h+1}+q_{l-h}V\,,
  \end{array}
  \right\}
  \label{e28-sin}
  \end{equation}
если ни при каком $s\hm<l\hm-h$ производная $d^s\left[ b_U^{-1}(t) 
a(Z,t)\right]/dt^s$ не содержит белого шума~$V$.
  
  В первом случае, если согласно~[1, 2] при\-вес\-ти~(\ref{e6-sin}) к сис\-те\-ме 
уравнений первого порядка:
  \begin{equation}
  \left.
  \begin{array}{rl}
  U_1&=U\,;\\[9pt]
  \dot{U}_i &=U_{i+1}\enskip (i=1,\ldots , l-h-1)\,;\\[9pt]
  \dot{U}_i &= U_{i+1}+q_iV \enskip (i=l-h,\ldots , l-1)\,;\\[9pt]
  \dot{U}_l &=\displaystyle -\alpha_i^{-1}\sum\limits_{i=1}^{l-1} \alpha_i U_{i+1}+q_l V\,,
  \end{array}
  \right\}
  \label{e29-sin}
  \end{equation}
где $q_i$ и $q_l$ определены в~[1, 2] и зависят от~$\alpha_i$ и~$\beta_j$, 
можно исключить помехи $U_1,\ldots , U_s$ с по\-мощью~(\ref{e18-sin}) и 
урав\-не\-ний, полученных ($s-1$)-крат\-ным дифференцированием.
  
  Во втором случае следует из~(\ref{e29-sin}) исключить помехи $U_1, \ldots , 
U_{l-h-1}$ и белый шум~$V$ с по\-мощью~(\ref{e18-sin}) и всех урав\-не\-ний, 
полученных из него дифференцированием.
  
  В обоих случаях получим уравнения $U_{s+1}, \ldots , U_l$ ($s\hm\leq l\hm-
h$). При этом можно рас\-чле\-нить каждую из этих переменных на две, одна из 
которых зависит от~$Z$, а другая зависит от $\mathrm{Э}(Z,t)$ и его 
производных, и тогда будем иметь
  \begin{equation}
  U_{s+k} =Z^\prime_k-Y_k\,,
  \label{e30-sin}
  \end{equation}
где $Z_1^\prime, \ldots , Z^\prime_{l-s}$ определяются дифференциальными 
уравнениями~(\ref{e25-sin}):
\begin{equation}
\dot{Z}^\prime =c^\prime\,,\ c^\prime=c^\prime(Z,t)\,,\ Z^\prime=\left[ Z_1^\prime, 
\ldots , Z^\prime_{l-s}\right]^{\mathrm{T}}\,,
\label{e31-sin}
\end{equation}
а $Y_1^\prime, \ldots Y^\prime_{l-s}$~--- урав\-не\-ни\-ями
\begin{equation}
\left.
\begin{array}{c}
\dot{Y}^\prime=c^{\prime\prime}\,,\enskip c^{\prime\prime}= 
c^{\prime\prime}\left(\mathrm{Э}, \dot{\mathrm{Э}}, \ldots , \mathrm{Э}^{(l-
s)}\right)\,;\\[9pt]
Y^\prime= \left[ Y_1^\prime, \ldots , Y^\prime_{l-s}\right]^{\mathrm{T}}\,.
\end{array}
\right\}
\label{e32-sin}
\end{equation}
Правые части $c^\prime$ и $c^{\prime\prime}$ определяются 
согласно~(\ref{e29-sin}) и~(\ref{e30-sin}).

  Расширив вектор со\-сто\-яния~$Z$ для $\overline{Z}\hm=\left[ Z^{\mathrm{T}}\ Z_1^{\prime 
\mathrm{T}} \cdots Z_{l-s}^{\prime \mathrm{T}}\right]^{\mathrm{T}}$, придем к окончательным 
урав\-не\-ни\-ям~(\ref{e25-sin}), (\ref{e26-sin}) при $c\hm=c^\prime$.
  
  Таким образом, получен следующий результат.
  
  \medskip
  
  \noindent
  \textbf{Теорема~3.} \textit{Если мат\-ри\-ца $k_1^{\overline{a}}$ 
асимптотически устойчива, а стохастический режим в}~(\ref{e5-sin}) 
\textit{не менее гладкий, чем помеха, то корреляционный алгоритм 
аналитического моделирования МНА определяется уравнениями тео\-ре\-мы}~1 
\textit{для сис\-те\-мы}~(\ref{e25-sin}) \textit{при 
  условиях}~(\ref{e27-sin})--(\ref{e32-sin}).
  
\section{Тестовые примеры}
     
     
     \noindent
1.~Стохастическое уравнение Дуффинга~[1, 2]
\begin{equation}
\ddot{X}+\omega^2 X-\mu X^3 =-\delta \dot{X}+U_0+V
\label{e33-sin}
\end{equation}
при $U_0=0$ допускает режим стационарных стохастических колебаний, 
определяемых формулой Гиббса 
\begin{equation*}
f_1(x,\dot{x})=C\exp \left[ -\fr{2\delta}{\gamma}\,H(x,\dot{x})\right]\,,
%\label{e34-sin}
\end{equation*}
где $H(x,\dot{x})\hm= \dot{x}^2/2\hm+\omega^2x^2/2\hm- \mu x^4/4$. В~основе 
корреляционного алгоритма аналитического моделирования стационарных и 
нестационарных режимов лежат следующие урав\-не\-ния~[1, 2]:
\begin{multline*}
X^3\approx m_X(m_X^2+3D_X)+3(m^2_X+D_X)X^0={}\\
{}=k_0 m_X+k_1 X^0\enskip 
(X^0=X-m_X)\,;
%\label{e35-sin}
\end{multline*}
\begin{equation*}
\dot{m}_X = m_{\dot{X}}\,,\enskip \dot{m}_X=U_0-\omega^2_{\mathrm{э}} 
m_X-\delta m_{\dot{X}}\,;
%\label{e36-sin}
\end{equation*}
      \begin{gather*}
     \dot{D}_X=2K_{X\dot{X}}\,,\enskip \dot{D}_X =\nu -2(\omega^2_{1\mathrm{э}} 
K_{X\dot{X}}+\delta D_{\dot{X}})\,;\\
 \dot{K}_{X\dot{X}}=D_{\dot{X}}- 
\omega^2_{1\mathrm{э}} D_X-\delta K_{X\dot{X}}\,,
    \end{gather*}
где
\begin{equation}
\hspace*{-4mm}\left.
\begin{array}{rl}
\omega^2_{\mathrm{э}} &=\omega^2\left[ 1-
\fr{\mu(m_X^2+3D_X)}{\omega^2}\right]\,;\\[9pt]
\omega^2_{1\mathrm{э}} &=\omega^2\left[ 1-
\fr{3\mu(m^2_X+D_X)}{\omega^2}\right] \enskip 
(\omega_{\mathrm{э}}>\omega_{1\mathrm{э}})\,.
\end{array}
\right\}
\label{e37-sin}
\end{equation}

  При $U_0=0$ в стационарном случае $m_X^*\hm=0$, 
$m^*_{\dot{X}}\hm=0$, $K^*_{X\dot{X}}\hm=0$, $D^*_{\dot{X}}\hm = 
\nu/(2\delta)$, а $D_X^*$ определяются путем решения урав\-не\-ния 
$\nu/(2\delta)\hm= \omega_{1\mathrm{э}}^2(D^*_{X})D^*_{X}$.
  
  Таким образом, МНА для $D_{\dot{X}}^*$ дает решение, совпадающее с 
точным, а для $D^*_{{X}}$~--- приближенное, точ\-ность которого 
зависит от коэффициента~$\mu$. Процесс установления стационарных 
\mbox{колебаний} происходит в два этапа: сначала устанавливается $D_{\dot{X}}^*$, а 
затем $D_{{X}}^*$.

\pagebreak
  
  Условие (\ref{e7-2-sin}) требует консерватизма статистически 
линеаризованной сис\-те\-мы левой час\-ти~(\ref{e33-sin}). Для значений~$\mu$, 
отвечающих колебаниям, точ\-ность со\-став\-ля\-ет~10\%. 

\noindent
2.~Для системы 
\begin{equation}
\ddot{X}+\omega^2 X -\mu X^3=-\delta \dot{X}+U_0+U\,,\ \dot{U}=-\gamma 
U+V
\label{e38-sin}
\end{equation}
уравнения МНА для $Z=\left[ X\, \dot{X}\, U\right]^{\mathrm{T}}$ имеют вид~(\ref{e9-sin}) 
и~(\ref{e10-sin}) при
\begin{gather*}
a_1=\begin{bmatrix}
m_{\dot{X}}\\ -\omega^2_{\mathrm{э}} m_X-\delta m_{\dot{X}}+U_0\\ -m_U
\end{bmatrix};\enskip
\alpha = \begin{bmatrix}
0 & 1 & 0\\
-\omega^2_{1\mathrm{э}} & -\delta & 0\\
0 & 0& -\gamma
\end{bmatrix};\hspace*{-0.26428pt}\\
\beta=\begin{bmatrix} 0&0&0\\
0&0&0\\
0&0&1
\end{bmatrix}\,;\  
a_2=\alpha K_t+K_t\alpha^{\mathrm{T}} +\beta\nu \beta^{\mathrm{T}}\,,
\end{gather*}
где $\nu$~--- интенсивность белого шума~$V$; $\omega_{\mathrm{э}}$, 
$\omega_{1\mathrm{э}}$ определяются~(\ref{e37-sin}). Отсюда аналитическим 
моделированием определяются стационарные режимы стохастических 
колебаний, а также режимы установления. Условие~(\ref{e7-2-sin}) 
выполняется в силу консерватизма левой час\-ти~(\ref{e38-sin}).
  
  Сравнивая оба примера, заключаем, что точность метода за счет 
<<профильтрованности>> помех значительно повышается и достигает 
  2\%--4\%.
  
  \noindent
  3.~В инструментальном программном обеспечении CStS-Analysis тестовые 
примеры описаны в~[14, 15].
  
\section{Заключение}
  
  Для многомерных нелинейных дифференциальных нормальных 
(гауссовских) сис\-тем с некоррелированными и автокоррелированными 
гауссовскими помехами на базе метода нормальной аппроксима\-ции 
разработаны корреляционные ал\-го\-рит\-мы аналитического моделирования 
стохастических режимов с инвариантной мерой. Результаты допускают 
обобщение на случай негауссовских помех, а также недифференцируемых 
функций $a\hm=a(Z,t)$ в урав\-не\-ни\-ях~(1) и~(\ref{e5-sin}).
  
  С помощью разработанного в среде MATLAB инструментального 
программного обеспечения на тестовых примерах показана достаточная для 
многих приложений точность корреляционных ал\-го\-ритмов.
  
  
  {\small\frenchspacing
{%\baselineskip=10.8pt
\addcontentsline{toc}{section}{Литература}
\begin{thebibliography}{99}

     \bibitem{1-sin}
     \Au{Пугачев В.\,С., Синицын И.\,Н.} Стохастические дифференциальные 
сис\-те\-мы. Анализ и фильтрация.~--- 2-е изд., доп.~--- М.: Наука, 1990.
     \bibitem{2-sin}
     \Au{Пугачев В.\,С., Синицын И.\,Н.} Теория стохастических сис\-тем.~--- 
     2-е изд.~--- М.: Логос, 2004.
     \bibitem{3-sin}
     \Au{Moshchuk N.\,K., Sinitsyn I.\,N.} On stationary distributions in nonlinear 
stochastic differential systems: Preprint.~--- Coventry, CV4 7AL, UK: Mathematics 
Institute, University of Warwick, 1989. 15~p. 
     \bibitem{4-sin}
     \Au{Moshchuk N.\,K., Sinitsyn I.\,N.} On stochastic nonholonomic systems: 
Preprint.~--- Coventry, CV4 7AL, UK: Mathematics Institute University of Warwick, 
1989. 32~p. 
     \bibitem{5-sin}
     \Au{Мощук Н.\,К., Синицын И.\,Н.} О~стохастических неголономных 
сис\-те\-мах~// Прикладная механика и математика, 1990. Т.~54. Вып.~2. 
     С.~213--223.
     \bibitem{6-sin}
     \Au{Moshchuk N.\,K., Sinitsyn I.\,N.} On stationary distributions in nonlinear 
stochastic differential systems~// Quart. J.~Mech. Appl. Math., 1991. Vol.~44. Pt.~4. 
P.~571--579.
     \bibitem{7-sin}
     \Au{Мощук Н.\,К., Синицын И.\,Н.} О~стационарных и приводимых к 
стационарным режимах в нормальных стохастических сис\-те\-мах~// Прикладная 
механика и математика, 1991. Т.~55. Вып.~6. С.~895--903.
     \bibitem{8-sin}
     \Au{Мощук Н.\,К., Синицын И.\,Н.} Распределения с инвариантной мерой 
в механических стохастических нормальных сис\-те\-мах~// Докл. АН СССР, 1992. 
Т.~322. №\,4. С.~662--667.
     \bibitem{9-sin}
     \Au{Синицын И.\,Н.}
     Конечномерные распределения с инвариантной мерой в стохастических 
механических сис\-те\-мах~// Докл. РАН, 1993. Т.~328. №\,3. С.~308--310.
     \bibitem{10-sin}
     \Au{Синицын И.\,Н.} Конечномерные распределения с инвариантной 
мерой в стохастических нелинейных дифференциальных сис\-те\-мах.~--- М.: 
Диалог МГУ, 1997. С.~129--140.
     \bibitem{11-sin}
     \Au{Синицын И.\,Н., Корепанов~Э.\,Р., Белоусов~В.\,В.} Точные методы 
рас\-че\-та стационарных режимов с инвариантной мерой в стохастических 
сис\-те\-мах управ\-ле\-ния~// Кибернетика и технологии XXI~века (C\&T'2002): Тр. 
II Междунар. научно-технич. конф.~--- Воронеж: Саквое, 2002. С.~124--131.
     \bibitem{12-sin}
     \Au{Синицын И.\,Н., Корепанов Э.\,Р., Белоусов~В.\,В.} Точные 
аналитические методы в статистической динамике нелинейных 
ин\-фор\-ма\-ци\-он\-но-управ\-ля\-ющих сис\-тем~// Сис\-те\-мы и средства информатики. 
Спец. вып. Математическое и алгоритмическое обеспечение 
     ин\-фор\-ма\-ци\-он\-но-те\-ле\-ком\-му\-ни\-ка\-ци\-он\-ных сис\-тем.~--- 
М.: Наука, 2002. С.~112--121.
     \bibitem{13-sin}
     \Au{Soize C.} The Fokker--Plank equation for stochastic dynamical systems 
and its explicit steady state solutions.~--- Singapore: World Scientific, 1994.
     \bibitem{14-sin}
     \Au{Sinitsyn I.\,N.} Lectures on PC-based nonlinear stochastic mechanical 
systems research: U$\hat{\mbox{c}}$ebni Texty $\acute{\mbox{u}}$snavu 
Termomechaniky.~--- Praha: {\ptb {\v{C}}}AV, 1992. 63~p.

\label{end\stat}
     \bibitem{15-sin}
     \Au{Синицын И.\,Н.} Стохастические информационные технологии для 
исследования нелинейных круговых сис\-тем~// Информатика и её применения, 
2011. Т.~5. Вып.~4. С.~78--99.



%     \bibitem{16-sin}
%     \Au{Синицын И.\,Н.} Математическое обеспечение для анализа 
%нелинейных многоканальных круговых сис\-тем, основанное на па\-ра\-мет\-ри\-за\-ции 
%распределений~// Информатика и её применения, 2012. Т.~6. Вып.~1. С.~11--17.
\end{thebibliography}
}
}


\end{multicols}     
     
     
     
            %1
\def\stat{bosov}

\def\tit{УПРАВЛЕНИЕ ЛИНЕЙНЫМ ВЫХОДОМ АВТОНОМНОЙ ДИФФЕРЕНЦИАЛЬНОЙ 
СИСТЕМЫ ПО~КВАДРАТИЧНОМУ КРИТЕРИЮ НА~БЕСКОНЕЧНОМ 
ГОРИЗОНТЕ$^*$}

\def\titkol{Управление линейным выходом автономной дифференциальной 
системы по квадратичному критерию} % на бесконечном  горизонте}

\def\aut{А.\,В.~Босов$^1$}

\def\autkol{А.\,В.~Босов}

\titel{\tit}{\aut}{\autkol}{\titkol}

\index{Босов А.\,В.}
\index{Bosov A.\,V.}


{\renewcommand{\thefootnote}{\fnsymbol{footnote}} \footnotetext[1]
{Работа выполнялась с использованием инфраструктуры Центра коллективного пользования 
<<Высокопроизводительные вы\-чис\-ле\-ния и~большие данные>> (ЦКП <<Информатика>>) ФИЦ ИУ 
РАН (г.\ Москва).}}


\renewcommand{\thefootnote}{\arabic{footnote}}
\footnotetext[1]{Федеральный исследовательский центр <<Информатика и управление>> Российской академии наук, 
\mbox{ABosov@frccsc.ru}}

\vspace*{-6pt}


  

\Abst{Решена задача оптимального управления линейным выходом 
стохастической дифференциальной сис\-те\-мы на бесконечном горизонте. Решение 
получено как предельная форма оптимального управ\-ле\-ния в~со\-от\-вет\-ст\-ву\-ющей 
задаче с~конечным горизонтом. Приведены достаточные условия существования 
управления, состоящие из требований стационарности нелинейной динамики, 
конечности квад\-ра\-тич\-но\-го целевого функционала, стабилизируемости линейного 
выхода и~существования предела в~формуле Фейн\-ма\-на--Ка\-ца, опи\-сы\-ва\-ющей 
нелинейную часть управ\-ле\-ния. Условия для линейной час\-ти управления связаны 
с~классическими результатами существования решения автономного урав\-не\-ния 
Риккати. Существование предела в формуле Фейн\-ма\-на--Ка\-ца~--- с~решением 
параболического уравнения, за\-да\-юще\-го коэффициенты для нелинейной части 
управления. Рас\-смот\-рен част\-ный случай линейного сноса, при котором 
сохраняется нелинейный характер задачи, но оптимальное управ\-ле\-ние 
оказывается линейным и по выходу, и~по переменной со\-сто\-яния. Приведены 
результаты численного эксперимента, поз\-во\-ля\-юще\-го проанализировать 
переходный процесс в~задаче с~конечным горизонтом и~эргодическим процессом 
в~динамике. Для коэффициентов управ\-ле\-ния проиллюстрирован предельный 
переход к~оптимальным значениям соответствующего оптимального автономного 
управ\-ле\-ния.}

\KW{стохастическая дифференциальная система Ито; управление по выходу; 
оптимальное управ\-ле\-ние; квад\-ра\-тич\-ный критерий; параболическое уравнение; 
формула Фейн\-ма\-на--Каца}

\DOI{10.14357/19922264240103}{UEESFO}
  
%\vspace*{-6pt}


\vskip 10pt plus 9pt minus 6pt

\thispagestyle{headings}

\begin{multicols}{2}

\label{st\stat}

\section{Введение}

     Типовым вариантом классической задачи управ\-ле\-ния~[1] линейной 
дифференциальной гауссовской системой по квад\-ра\-тич\-но\-му критерию 
качества (LQG, linear-quadratic-Gaussian) стала \mbox{постановка} с~бесконечным горизонтом  
управ\-ле\-ния~[2, 3]. Естественно, что на сис\-те\-му управ\-ле\-ния в этом 
случае накладываются жесткие ограничения, обес\-пе\-чи\-ва\-ющие ее 
существование бесконечное время. Прежде всего это автономность, т.\,е.\ 
независимость от времени всех функций в~модели динамики и~в~целевом 
функционале, устой\-чи\-вость и~ста\-би\-ли\-зи\-ру\-емость. Последние свойства 
обеспечивают принципиальную воз\-мож\-ность управ\-ле\-ния и~ко\-неч\-ность 
квадратичного критерия. Кроме того, вполне <<понятным>> элементом 
постановки варианта LQG с бесконечным горизонтом становится поиск 
оптимального автономного управ\-ле\-ния в~виде линейной функции выхода. 
Линейный класс допустимых управ\-ле\-ний сводит исследование задачи 
управления к~изуче\-нию свойств уравнения Риккати, так что решение задачи 
управления существует, если имеет решение предельное уравнения Риккати. 
Постановка задачи управ\-ле\-ния, ис\-поль\-зу\-емая в~данной статье, похожа на 
модель LQG, т.\,е.\ используется квад\-ра\-тич\-ный целевой функционал 
и~линейный выход, дающий в~оптимальном решении типовое, т.\,е.\ 
линейное по выходу, сла\-га\-емое. Принципиальное отличие заключается 
в~нелинейной не\-управ\-ля\-емой динамике, из-за которой в оптимальном 
управ\-ле\-нии в~варианте с~конечным горизонтом появляется нелинейная 
часть, для вы\-чис\-ле\-ния которой нужно решать параболическое 
дифференциальное урав\-не\-ние в~част\-ных производных. 

Подробно эта задача 
исследована в~[4], а~цель данной статьи~--- пред\-ста\-вить решение 
аналогичной задачи в~постановке с~бесконечным горизонтом. Эта 
постановка сформулирована в~сле\-ду\-ющем разделе статьи, а~в~разд.~3 
сформулирован основной результат. Ровно так же, как управ\-ле\-ние~[4], 
<<похожее>> на LQG, все-та\-ки оказывается нелинейным, полученное 
оптимальное автономное управ\-ле\-ние также остается нелинейным, но 
вместо параболического урав\-не\-ния описывается теперь обыкновенным 
дифференциальным уравнением, хотя и~нелинейным. Чтобы посмотреть, как 
на практике применится автономное управ\-ле\-ние, в~разд.~4 рассмотрен 
част\-ный случай линейного сноса, когда управ\-ле\-ние оказывается линейным, 
несмотря на оста\-ющу\-юся нелинейной задачу. Этот вариант интересен тем, 
что поз\-во\-ля\-ет еще и посмотреть, как чис\-лен\-ные решения параболического 
уравнения традиционным методом сеток~[5] и~методом имитационного 
моделирования формулы Фейн\-ма\-на--Ка\-ца~[6], которые <<не знают>> 
о~наличии точного линейного решения, ведут себя в~связке при переходе 
обычного решения в~автономный режим.

\vspace*{-3pt}

\section{Постановка задачи}

\vspace*{-3pt}

     На каноническом вероятностном пространстве $(\Omega, 
\mathcal{F},\mathcal{P},\mathcal{F}_t)$, $t\hm\in [0,\infty)$, рас\-смот\-рим 
автономную сто\-ха\-сти\-че\-скую динамическую сис\-те\-му, со\-сто\-яние которой 
пред\-став\-ля\-ет диффузионный процесс $y_t\hm\in \mathbb{R}^{n_y}$, 
опи\-сы\-ва\-емый сис\-те\-мой нелинейных сто\-ха\-сти\-че\-ских дифференциальных 
уравнений Ито:
     \begin{equation}
     dy_t= A(y_t)\,dt+\Sigma(y_t)\,dv_t\,,\enskip y_0=Y,\enskip t\in [0,\infty),
     \label{e1-bos}
     \end{equation}
   где $v_t\in \mathbb{R}^{n_v}$~--- стандартный векторный винеровский 
процесс; $Y\hm\in \mathbb{R}^{n_y}$~--- случайная величина с~конечным 
вторым моментом; векторная функция $A\hm= A(y): \mathbb{R}^{n_y}\hm\to 
\mathbb{R}^{n_y}$ и~мат\-рич\-ная функция $\Sigma\hm= \Sigma(y): 
\mathbb{R}^{n_y}\hm\to \mathbb{R}^{n_y\times n_v}$ удовле\-тво\-ря\-ют 
условиям Ито:
\begin{align*}
\vert A(y)\vert +\vert \Sigma(y)\vert &\leq C(1+\vert y\vert);\\
\vert A(y_1)- A(y_2)\vert +\vert \Sigma(y_1) -\Sigma(y_2)\vert&\leq C\vert y_1-
y_2\vert,\\
& \hspace*{5mm}y_1,y_2\in\mathbb{R}^{n_y}\,,
\end{align*}
обеспечивающим существование единственного потраекторного решения 
уравнения~(1) на любом конечном интервале $t\hm\in [0,T]$~\cite{7-bos} 
(здесь и~далее $\vert\cdot\vert$ обозначает евклидову норму вектора или 
мат\-ри\-цы). Чтобы рассматривать решения~(1) на интервале $[0,\infty)$, 
дополнительно по\-тре\-бу\-ем от процесса~$y_t$ стационарности в широком и 
узком смысле~[8]. Это предположение избыточно, пока рассматривается 
задача с~полной информацией, т.\,е.\ состояние~$y_t$ предполагается 
известным. Но в~дальнейшем при переходе к~постановке задачи 
с~косвенными наблюдениями за~$y_t$ ста\-ци\-о\-нар\-ность примет вполне 
содержательный смысл.

     Состоянием $y_t$ формируется управ\-ля\-емый выход, опи\-сы\-ва\-емый 
процессом $z_t\hm\in\mathbb{R}^{n_z}$, линейно связанным с~со\-сто\-я\-нием:

\vspace*{-3pt}

\noindent
     \begin{multline}
     dz_t=ay_t\, dt +bz_t \,dt+c u_t\, dt+\sigma\, dw_t,\\
     z_0=Z,\ t\in [0,\infty),
     \label{e2-bos}
     \end{multline}
     
     \vspace*{-3pt}

\noindent
где $a\in \mathbb{R}^{n_z\times n_y}$, $b\hm\in \mathbb{R}^{n_z\times n_z}$, 
$c\hm\in \mathbb{R}^{n_z\times n_u}$ и~$\sigma\hm\in \mathbb{R}^{n_z\times n_w}$~--- известные матрицы; $w_t\hm\in \mathbb{R}^{n_w}$~--- не 
зависящий от $v_t$, $Y$ и~$Z$ стандартный векторный винеровский процесс; 
$Z\hm\in \mathbb{R}^{n_z}$~---случайная величина с~конечным вторым 
моментом, не зависящая от $v_t$ и~$Y$; $u_t\hm\in \mathbb{R}^{n_u}$~--- 
управление, целью которого ставится стабилизация выхода~$z_t$ около 
некоторой траектории, формируемой со\-сто\-яни\-ем~$y_t$. Цель управ\-ле\-ния 
формализуется ниже квад\-ра\-тич\-ным критерием качества общего вида.

     Задача управления формулируется в предположении наличия полной 
информации о~со\-сто\-янии~$y_t$ и~выходе~$z_t$ (со\-от\-вет\-ст\-ву\-ющая  
$\sigma$-ал\-геб\-ра обозначается $\mathcal{F}_t^{y,z}$ и~выполнено 
$\mathcal{F}_t^{y,z}\hm\subseteq \mathcal{F}_t\hm\subseteq \mathcal{F}$), 
т.\,е.\ управ\-ле\-ние~$u_t$ предполагается  
$\mathcal{F}_t^{y,z}$-из\-ме\-ри\-мым. Класс~$U_0^\infty$ допустимых 
управ\-ле\-ний составляют автономные (не зависящие прямо от времени) 
управ\-ле\-ния с~пол\-ной обратной связью, т.\,е.\ функции вида\linebreak $u_t\hm= 
u(y,z)\hm\in\mathbb{R}^{n_u}$, $y\hm\in \mathbb{R}^{n_y}$ и~$z\hm\in 
\mathbb{R}^{n_z}$, в~предположении, что со\-от\-вет\-ст\-ву\-ющая реализация 
$u_t\hm= u(y_t,z_t)$ обеспечивает выполнение условий существования~$y_t$ 
и~$z_t$ для $u_t\hm\in U_0^\infty$. Поскольку \mbox{со\-сто\-яние}~$y_t$  не зависит
от  управ\-ле\-ния~$u_t$, а~выход~$z_t$ описывается линейным 
автономным уравнением с~винеровским процессом, то данное формальное 
требование ограничивает допустимые управ\-ле\-ния процессами второго 
порядка, что обеспечивает существование решения~(2) на любом конечном 
интервале $t\hm\in [0,T]$. Для управ\-ле\-ния на интервале $[0,\infty)$ 
дополнительно потребуются типовые условия ста\-би\-ли\-зи\-ру\-емости~[2], 
обсуж\-да\-емые далее.
     
     Качество управления~$U_0^\infty$ определяется целевым 
функционалом сле\-ду\-юще\-го вида:
     \begin{multline}
     \!\! J(U_0^\infty) =\lim\limits_{T\to\infty} J\left(U_0^{{T}}\right),\\
      J\left(U_0^{{T}}\right) =\mathbb{E}\left\{ \fr{1}{T}\int\limits_0^{{T}} \left\| Py_t+Qz_t+Ru_t\right\|_S^2 
dt\right\},
     \label{e3-bos}
     \end{multline}
где $P\in \mathbb{R}^{n_J\times n_y}$, $Q\hm\in \mathbb{R}^{n_J\times n_z}$, $R\hm\in \mathbb{R}^{n_J\times k_u}$ 
и~$S\hm\in \mathbb{R}^{n_J\times n_J}$ ($S\hm\geq 0$, $S\hm= S^\prime$)~--- заданные 
матрицы, весовая функция $\| x\|_S^2\hm= x^\prime Sx$, единичной матрице 
$S\hm= E$ соответствует евклидова норма $\| x\|_E^2\hm= \vert x\vert^2$, 
<<${}^\prime$>>~--- операция транспонирования; $\mathbb{E}\{\cdot\}$~--- 
оператор математического ожидания (далее еще используется обозначение 
$\mathbb{E}\{ \cdot\vert\mathcal{F}\}$ для условного математического 
ожидания относительно $\sigma$-ал\-геб\-ры~$\mathcal{F}$). Кроме того, 
предполагается выполненным обычное условие не\-вы\-рож\-ден\-ности, в данных 
обозначениях принимающее вид $R^\prime SR\hm>0$.
     
     Задача состоит, таким образом, в поиске 
     $(U^*)_0^\infty \hm= \{ u^*(y,z), y\hm\in \mathbb{R}^{n_y}, z\hm\in 
\mathbb{R}^{n_z}\}$, допустимого управления с обратной связью, 
реализации $u_t^*\hm= u^*(y_t, z_t^*)$, $t\hm\in [0,\infty)$, которого 
доставляют минимум квад\-ра\-тич\-но\-му функционалу $J(U_0^\infty)$: 
     \begin{equation}
     (U^*)_0^\infty =\argmin\limits_{u_t\in U_0^\infty} J(U_0^\infty).
     \label{e4-bos}
     \end{equation}
     
     Далее через~$z_t^*$ обозначается решение~(2), от\-ве\-ча\-ющее~$u_t^*$, 
и~учитывается, что $y_t$ от~$u_t$ не зависит.
     
     Следует обратить внимание на некоторые отличия данной постановки 
от классической задачи LQG с бесконечным горизонтом (хорошее описание 
которой дано в~[2]). В~классическом варианте вопрос с~определением класса 
допустимых управ\-ле\-ний однозначно решается его описанием линейными 
функциями со\-сто\-яния и~выхода, а~именно: если обозначить через $u_t^\# 
\hm= u_t^\# (y_t, z_t^\#, T)$ решение (неавтономное, поэтому $u_t^\#(y,z,T)$ 
с~нижним индексом~$t$) LQG-за\-да\-чи с конечным горизонтом $t\hm\in 
[0,T]$, оптимальное на классе нелинейных до\-пус\-ти\-мых управ\-ле\-ний, то оно 
получится линейным: 
$$
u_t^\# = L_t^{y\#} y_t + L_t^{z\#} z_t^\# +  l_t^\#.
$$ 
%
Здесь, как и аналогично выше, через~$z_t^\#$  обозначено 
решение~(2), от\-ве\-ча\-ющее~$u_t^\#$. Соответственно, в описании класса 
допустимых управ\-ле\-ний в~задаче с~бесконечным горизонтом будут 
фигурировать только линейные управ\-ления 
$$
u_t= u\left(y_t,z_t\right)= L^y  y_t+ L^z z_t+l\,,
$$
 что естественным образом вытекает из оп\-ти\-маль\-ности 
в задаче с конечным горизонтом именно линейного управ\-ле\-ния. 
В~рас\-смат\-ри\-ва\-емой задаче этого нет, поскольку и~со\-сто\-яние~(1), 
и,~главное, решение задачи с~конечным горизонтом, полученное в~[4], 
нелинейные. Так что, хотя искомое решение и~находится в~виде предельной 
формы оптимального управ\-ле\-ния с~конечным горизонтом, оно не будет 
относиться к~классу линейных.
     
\section{Основной результат}

     Очевидно, что основу для решения задачи~(\ref{e4-bos}) обеспечивает 
решение со\-от\-вет\-ст\-ву\-ющей задачи с~конечным горизонтом, которое получено 
в~[4]. Для корректного предельного перехода от постановки с~конечным 
горизонтом к~бесконечному времени требуется выполнение ряда условий, 
объединенных в~сле\-ду\-ющем утверж\-де\-нии.
     
     \smallskip
     
     \noindent
     \textbf{Теорема.} \textit{Решение задачи}~(\ref{e4-bos}) \textit{может 
быть записано в~виде}

\vspace*{-4pt}

\noindent
     \begin{multline}
     u^*(y,z)=-\fr{1}{2}\left( R^\prime SR\right)^{-1} \left( c^\prime(2\alpha_* z+\beta_*(y))+{}\right.\\
    \left. {}+2R^\prime S(Py+Qz)\right),
     \label{e5-bos}
     \end{multline}
     
     \vspace*{-4pt}
     
     \noindent
\textit{где симметричная неотрицательно определенная мат\-ри\-ца 
$\alpha_*\hm\in \mathbb{R}^{n_z\times n_z}$ и~век\-тор-функ\-ция 
$\beta_*\hm=\beta_*(y)\hm= (\beta_*^{(1)}(y),\ldots , \beta_*^{(n_z)}(y))^\prime\hm\in 
\mathbb{R}^{n_z}$ пред\-став\-ля\-ют собой решения уравнений}

\noindent
\begin{multline}
\left( b^\prime -Q^\prime SR (R^\prime SR)^{-1} c^\prime\right) \alpha_* 
+{}\\[2pt]
{}+\alpha_* \left(b-c(R^\prime SR)^{-1} R^\prime SQ\right)+{}\\[2pt]
{}+
Q^\prime \left(S-SR(R^\prime SR)^{-1} R^\prime S\right)Q -{}\\[2pt]
{}- \alpha_* c\left(R^\prime SR\right)^{-1} 
c^\prime \alpha_*=0\,;
\label{e6-bos}
\end{multline}

\vspace*{-12pt}

\noindent
\begin{multline}
\fr{1}{2}\mathrm{tr}\left\{ \Sigma^{\prime} \fr{\partial^2\beta_*^{(i)}}{\partial y^2}\,\Sigma\right\} +A^\prime \fr{\partial\beta_*^{(i)}}{\partial y} +{}\\
{}+\displaystyle\sum\limits_{j=1}^{n_y} y^{(i)} [M_*]^{(ji)} +\sum\limits_{j=1}^{n_z} 
\beta_*^{(j)} [N_*]^{(ji)} =0\,,
\label{e7-bos}
\end{multline}
где
\begin{align*}
M_*&=2\left( \left( a^\prime -P^\prime SR(R^\prime SR)^{-1} c^\prime 
\right)\alpha_*+{}\right.\\[2pt]
&\hspace*{20mm}\left.{}+ P^\prime (S-SR(R^\prime SR)^{-1} R^\prime S)Q\right);\\[2pt]
N_*&= b-c(R^\prime SR)^{-1}R^\prime SQ -c(R^\prime SR)^{-1} c^\prime \alpha_*,\\[2pt]
&\hspace*{50mm}i=\overline{1, n_z}\,,
\end{align*}
\textit{если для параметров системы}~(1), (2) \textit{и целевого 
функционала}~(3) \textit{выполнены сле\-ду\-ющие условия}:
\begin{enumerate}[(1)]
\item \textit{матрица $b$ устойчива};
\item \textit{матрицы $(K_b, c)$ стабилизируемы}, $K_b\hm= b\hm- 
c(R^\prime SR)^{-1} R^\prime SQ$;
\item \textit{матрицы ($K_b^\prime, K_Q$) стабилизируемы}, $K_Q\hm= Q^\prime S^{1/2}(E\hm- S^{1/2}R(R^\prime SR)^{-1} R^\prime S^{1/2})$;
\item \textit{для любого $y\hm\in \mathbb{R}^{n_y}$ существует и не 
зависит от~$t$ конечный предел}
\begin{equation}
\lim\limits_{T\to\infty} \mathbb{E}\left\{ I^{-1}(t)\int\limits_t^{{T}} I^{-1}(\tau)M^\prime_* y_\tau\,d\tau \vert \mathcal{F}_t^y \right\},
\label{e8-bos}
\end{equation}
\textit{где $I^{-1}(\tau)\hm=\exp \{ N_*^\prime \tau\}$; $y_\tau$~--- 
решение уравнения}~(1) \textit{с~переменной времени $\tau\hm\in 
[t,\infty)$ и~начальным условием} $y_t\hm= y$.
\end{enumerate}

     \textit{Уравнение}~(\ref{e7-bos}) \textit{записано с~использованием 
обозначений $y\hm= (y^{(1)}, \ldots , y^{(n_y)})^\prime$ для элементов 
вектора~$y$, $[A]^{(ji)}$~--- для элемента $j$-й строки $i$-го столб\-ца 
мат\-ри\-цы~$A$, $\mathrm{tr}\{A\}$~--- для следа мат\-ри\-цы~$A$.}
     
     \smallskip
     
     \noindent
     Д\,о\,к\,а\,з\,а\,т\,е\,л\,ь\,с\,т\,в\,о\,.\ \ Перечисленные в теореме 
условия обеспечивают существование предельного решения 
соответствующей~(4) задачи с конечным горизонтом, т.\,е.\ управ\-ле\-ния 
     $(U^\#)_0^{{T}}\hm= \min_{u_t\in U_0^{{T}}} J(U_0^{{T}})$, где $U_0^{{T}}\hm= \{ 
u_t(y,z,T)$,\linebreak $y\hm\in \mathbb{R}^{n_y},\ z\hm\in \mathbb{R}^{n_z},\ t\hm\in 
[0,T]\}$. Оптимальное управ\-ле\-ние $u_t^\#\hm= u_t^\# (y_t, z_t^\#, T)$ 
получено в~[4] в~виде:
     \begin{multline}
     u_t^\#= u_t^\# (y,z,T) =-\fr{1}{2}(R^\prime SR)^{-1} \times{}\\
     {}\times \left( c^\prime 
(2\alpha_t z+\beta_t)+2R^\prime S(Py+Qz)\right),
     \label{e9-bos}
     \end{multline}
где матричный коэффициент $\alpha_t\hm= \alpha_t(T)\hm\in 
\mathbb{R}^{n_z\times n_z}$ представляет собой решение задачи Коши для 
уравнения Риккати

\vspace*{-3pt}

\noindent
\begin{multline}
\fr{d\alpha_t}{dt} +\left( b^\prime -Q^\prime SR(R^\prime SR)^{-1} c^\prime 
\right)\alpha_t +{}\\[-3pt]
{}+\alpha_t \left (b-c(R^\prime SR)^{-1} R^\prime SQ\right)+{}\\
{}+
Q^\prime \left(S-SR(R^\prime SR)^{-1} R^\prime S\right) Q - {}\\
{}- \alpha_t c 
(R^\prime SR)^{-1}c^\prime \alpha_t=0\,,\enskip \alpha_T=Q^\prime SQ\,,
\label{e10-bos}
\end{multline} 

%\vspace*{-3pt}

\noindent
а векторный коэффициент $\beta_t\hm= \beta_t(y,T)\hm= 
(\beta_t^{(1)}(y,T),\ldots , \beta_t^{(n_z)}(y,T))^\prime \hm\in 
\mathbb{R}^{n_z}$~--- задачи Коши для сис\-те\-мы дифференциальных 
уравнений в~част\-ных производных 

\vspace*{-3pt}

\noindent
\begin{multline}
\fr{\partial \beta_t^{(i)}}{\partial t}+\fr{1}{2}\mathrm{tr}\left\{ \Sigma^{\prime} 
\fr{\partial^2\beta_t^{(i)}}{\partial y^2}\, \Sigma\right\}+A^\prime 
\fr{\partial\beta_t^{(i)}}{\partial y} +{}\\
{}+\displaystyle \sum\limits_{j=1}^{n_y} 
y^{(j)} [M_t]^{(ji)} +\sum\limits^{n_z}_{j=1} \beta_t^{(j)} 
[N_t]^{(ji)}=0\,,
\label{e11-bos}
\end{multline}

\vspace*{-3pt}

\noindent
где
\begin{align*}
M_t&=M_t(T)=2\left((a^\prime - P^\prime SR(R^\prime SR)^{-1} c^\prime)\alpha_t+{}\right.\\
&\hspace*{20mm}\left.{}+P^\prime (S-SR(R^\prime SR)^{-1} R^\prime S)Q\right);\\
N_t&=N_t(T) =b-c(R^\prime SR)^{-1} R^\prime SQ -{}\\[3pt]
&\hspace*{40mm}{}-c(R^\prime SR)^{-1} c^\prime \alpha_t\,;\\
\beta_T^{(i)} &=\displaystyle 2\sum\limits_{j=1}^{n_y} y^{(j)} [Q^\prime SP]^{(ji)},\enskip i=\overline{1, n_z}\,.
\end{align*}
     
     Перечисленные в теореме условия вместе с~исходным предположением 
о~ста\-ци\-о\-нар\-ности~$y_t$ обеспечивают, во-пер\-вых, ко\-неч\-ность целевого 
функционала~(3). Во-вто\-рых, из~(\ref{e10-bos}) для коэффициента 
оптимального управ\-ле\-ния~$\alpha_*$ получается уравнение~(\ref{e6-bos}) 
как предел $\alpha_*\hm= \lim_{T\to\infty} \alpha_t(T)$. Предельное решение 
мат\-рич\-но\-го уравнения Риккати~(\ref{e10-bos}) существует, не зависит от 
граничного условия и~является неотрицательно определенным, поскольку 
сформулированные в~тео\-ре\-ме условия ста\-би\-ли\-зи\-ру\-емости приводят 
коэффициенты уравнения Риккати к~формулировкам условий 
тео\-ре\-мы~12.2~[2]. Действительно, ста\-би\-ли\-зи\-ру\-емость пары мат\-риц $(K_b, 
c)$ ~--- это условие ста\-би\-ли\-зи\-ру\-емости наблюдений, ста\-би\-ли\-зи\-ру\-емость 
пары $(K_b^\prime, K_Q)$ ~--- это условие де\-тек\-ти\-ру\-емости. Последнее 
основано на том факте, что мат\-ри\-ца~$K_Q$ обеспечивает пред\-став\-ление 
$$
Q^\prime \left(S- SR\left(R^\prime SR\right)^{-1} R^\prime S\right)Q= K_Q K_Q^\prime,
$$ 
так как

\vspace*{-3pt}

\noindent
     \begin{multline*}
     \left( E-S^{1/2} R(R^\prime SR)^{-1} R^\prime S^{1/2}\right)\times{}\\
     {}\times \left( E-S^{1/2} R(R^\prime SR)^{-1} R^\prime S^{1/2}\right)^\prime={}\hspace*{10mm}
     \end{multline*}
     
     
     \noindent
     \begin{multline*}
     {}=
     \left( E-S^{1/2} R(R^\prime SR)^{-1} R^\prime S^{1/2}\right)^2 ={}\\
     {}=E- S^{1/2} R(R^\prime SR)^{-1} R^\prime S^{1/2}.
     \end{multline*}
     %
     
     \vspace*{-3pt}
     
     \noindent
     Кроме того, эта же теорема гарантирует устой\-чи\-вость мат\-ри\-цы~$N_*$, 
тре\-бу\-емую для выполнения~(\ref{e8-bos}).
     %
     И~последним из~(\ref{e11-bos}) получается коэффициент~$\beta_*$, 
для которого предельный переход\linebreak $\beta_*(y)\hm= \lim_{T\to\infty} 
\beta_t(y,T)$ обеспечивается условием~4 тео\-ре\-мы. Действительно, 
поскольку решение~(\ref{e11-bos}) может быть представлено с~по\-мощью 
формулы Фейн\-ма\-на--Ка\-ца~[9],

\vspace*{-3pt}
     
     \noindent
     \begin{multline}
     \beta_t(y,T) =\mathbb{E}\left\{ 
     \vphantom{\int\limits_y^{{T}}}
     2I^{-1}(t) I^{-1}(T) Q^\prime SPy(T)+{}\right.\\
\left.     {}+ I^{-1}(t)\int\limits_y^{{T}} I^{-1}(\tau) M^\prime(\tau) y(\tau)d\tau\vert 
\mathcal{F}_t^y\right\},\\
     I^{-1}(t)=\exp \{ N_t^\prime t\}.
     \label{e12-bos}
     \end{multline}
     
     \vspace*{-3pt}
     
     
     Поскольку матрица $N_*\hm= \lim_{T\to\infty} N_t(T)$ (предел 
существует и не зависит от~$t$, так как $N_t(T)$ выражается линейно через 
$\alpha_t(T)$) устойчива, первое сла\-га\-емое в~(\ref{e12-bos}) обращается 
в~ноль, что вместе с~условием~4 тео\-ре\-мы обеспечивает существование 
предела при $T\hm\to \infty$ в~(\ref{e11-bos}), не зависящего от~$t$ 
и~граничного условия $\beta_T^{(i)}(y,T)$, что и дает  
уравнения~(\ref{e7-bos}), завершая доказательство.
     
     \smallskip
     
     Отметим, что использованный результат в~[4] включает еще 
соотношения, опи\-сы\-ва\-ющие решение задачи с~конечным горизонтом 
полностью, т.\,е.\ опре\-де\-ля\-ющие функцию Беллмана. Эти соотношения 
можно трансформировать и~для рассматриваемой задачи~(\ref{e4-bos}), но 
ничего практически содержательного они не дадут, поэтому не 
используются.
     
     Принципиально важно в полученном решении то, что оптимальное 
управление~(\ref{e5-bos}) остается нелинейным, как и~в~допредельной 
по\-ста\-нов\-ке, что и~является, как уже упоминалось, главным отличием 
классической задачи LQG от рас\-смат\-ри\-ва\-емой. Другой вопрос, что 
к~условиям существования решения ничего конструктивного, кроме 
традиционных требований к~коэффициентам уравнения Риккати, добавить не 
удается. Это ожидаемо, так как даже конструктивные условия существования 
решения сис\-те\-мы параболических уравнений~(\ref{e11-bos}) представляют 
проб\-ле\-му, тем более затруднительно изучение свойств самих решений. 
С~другой стороны, аналитические исследования можно впол\-не успеш\-но 
заменить практическими расчетами. В~работах~[5, 6] предложены два 
эффективных метода численного решения~(\ref{e11-bos}), а~при наличии 
готовых расчетов анализ схо\-ди\-мости $\lim_{T\to\infty} \beta_t(y,T)$ тру-\linebreak\vspace*{-12pt}

\pagebreak

\noindent
  да не 
составляет. Другой вопрос, что реальные возможности реализации 
нелинейного управ\-ле\-ния для общей модели со\-сто\-яния~(1) пред\-став\-ля\-ют\-ся 
весьма ограниченными. Более практически интересным будет част\-ный 
случай линейного сноса или сводящийся к~нему случай косвенных 
наблюдений за со\-сто\-яни\-ем дискретной цепи Маркова, которому планируется 
по\-свя\-тить будущую работу.
     
\section{Частный случай линейного сноса}

     Итак, практическая реализация управ\-ле\-ния $(U^*)_0^\infty$ сведена 
к~решению алгебраического уравнения~(\ref{e6-bos}) и~сис\-те\-мы 
обыкновенных дифференциальных уравнений~(\ref{e7-bos}) вместо 
уравнения Риккати~(\ref{e10-bos}) и~сис\-те\-мы параболических  
уравнений~(\ref{e11-bos}). При этом если упро\-ще\-ние в~час\-ти отказа от 
решения уравнения Риккати не представляется сколь-ли\-бо значимым, так 
как с~эффективными численными методами для него нет проб\-лем, то замена 
параболических уравнений на обыкновенные~--- это действительно большое 
упрощение. Тем не менее работа с системой~(\ref{e7-bos}) все еще остается 
довольно слож\-ной, поскольку решать требуется хоть и~обыкновенные 
дифференциальные уравнения второго порядка, но для сетки, по\-кры\-ва\-ющей 
об\-ласть воз\-мож\-ных значений вектора $y\hm\in \mathbb{R}^{n_y}$, что даже 
для значений $n_y\hm= 3, 4$ (это минимальные раз\-мер\-но\-сти для моделей, 
име\-ющих практический смысл) становится уже крайне ресурсоемким. 
Вместе с~тем заменять уравнение~(1) линейным неинтересно, так как тогда 
задача сведется к~классической LQG. Интерес, таким образом, пред\-став\-ля\-ет 
случай линейного сноса, т.\,е.\ линейной функции $A(y)$, а~не\-ли\-ней\-ность 
задачи останется за счет диффузии~$\Sigma(y)$.
      
     Случай неавтономной сис\-те\-мы с~линейным сносом $A_t(y)\hm= A_t^a 
y\hm+ A_t^b$, где мат\-ри\-ца $A_t^a\hm\in \mathbb{R}^{n_y\times n_y}$ и~вектор $A_t^b\hm\in \mathbb{R}^{n_y}$ не зависят от~$y$, рас\-смот\-рен в~[4]. 
В~этом случае $\beta_t(y,T)\hm= \beta_t^a(T)y \hm+ \beta_t^b(T)$ и~для 
коэффициентов получены уравнения:
     \begin{equation}
     \left.
     \begin{array}{c}
     \fr{d\beta_t^a}{dt} +\beta_t^a A^a +M^\prime_* +N^\prime_* \beta_t^a 
=0\,;\\[3pt]
     \fr{d\beta_t^b}{dt} +\beta_t^a A^b +N^\prime_* \beta_t^b=0\,;\\[3pt]
     \beta_T^a(T) =2Q^\prime SP\,;\enskip \beta_T^b(T)=0\,.
     \end{array}
     \right\}
     \label{e13-bos}
     \end{equation}
     
     В условиях сформулированной тео\-ре\-мы существуют пределы 
$\beta_*^a\hm= \lim_{T\to\infty} \beta_t^a$ и~$\beta_*^b\hm= \lim_{T\to \infty} 
\beta_t^b$, не зависящие от~$t$ и~от начального условия $\beta_T^a(T), 
\beta_T^b(T)$ в~задаче Коши, опи\-сы\-ва\-емые уравнениями:
     \begin{equation}
     \beta_*^a A^a +M^\prime_* +N^\prime_* \beta_*^a =0\,;\
     \beta_*^a A^b +N^\prime_* \beta_*^b=0\,.
     \label{e14-bos}
     \end{equation}
     
     Соответственно, коэффициент $\beta_*(y)\hm= \beta_*^a y \hm+ 
\beta_*^b$, а~оптимальное управ\-ле\-ние~(\ref{e5-bos}) принимает вид:
     \begin{multline}
     u^*(y,z) =-\fr{1}{2}\left( R^\prime SR\right)^{-1} \left(2(c^\prime \alpha_* 
+R^\prime SQ)z +{}\right.\\
\left.{}+\left(c^\prime \beta_*^a +2R^\prime SP\right)y +c^\prime 
\beta_*^b\right).
     \label{e15-bos}
     \end{multline}
     
     \vspace*{-6pt}

\section{Численный анализ стационарного режима}

     Для иллюстрации перехода неавтономного управ\-ле\-ния~(\ref{e9-bos}) 
$u_t^\#\hm= u_t^\# (y,z,T)$ в~автономный режим~(\ref{e5-bos}) $u^*\hm= 
u^*(y,z)$, который должен иметь мес\-то при достаточно больших~$T$, 
используем пример, детально рас\-смот\-рен\-ный в~[1]. Простая модель 
эволюции процентных ставок Кок\-са--Ин\-гер\-сол\-ла--Рос\-са  
(Cox--Ingersoll--Ross model)~[10], при\-ме\-ня\-емая так\-же для описания 
показателя RTT (Round-Trip Time) сетевого протокола TCP (Transmission 
Control Protocol)~\cite{11-bos}, имеет вид:
     \begin{equation}
     dy_t=(1-y_t)\,dt+2{,}5\sqrt{y_t}\,dv_t,\ y_0=Y\sim \mathbb{N}(15{,}9). \!
     \label{e16-bos}
     \end{equation}
     
     Выход и целевой функционал описываются уравнениями:
     \begin{gather*}
     dz_t = y_t\,dt-z_t\,dt+u_t\,dt+2{,}5\,dw_t, \ z_0=Z\sim \mathbb{N}(9{,}9);\hspace*{-0.48953pt}\\
     J(U_0^{{T}}) = \mathbb{E}\left\{ \int\limits_0^{{T}} \left( (y_t-z_t)^2 
+z_t^2+u_t^2\right)dt +{}\right.\\
\left.{}+(y_T-z_T)^2 +z_T^2
\vphantom{\int\limits_0^{{T}}}
\right\},\ T=5\,.
     \end{gather*}
     
      \begin{figure*} %fig1
     \vspace*{1pt}
     \begin{minipage}[t]{80mm}
      \begin{center}
     \mbox{%
\epsfxsize=79mm 
\epsfbox{bos-1.eps}
}
\end{center}
\vspace*{-9pt}
     \Caption{Коэффициенты оптимального управ\-ле\-ния: \textit{1}~--- $\alpha_t$; 
\textit{2}~--- $\beta_t^a$;
  \textit{3}~--- $\beta_t^b$}
  \vspace*{3pt}
  \end{minipage}
     % \end{figure*}
   \hfill   
      %\begin{figure*} %fig2
\vspace*{1pt}
\begin{minipage}[t]{80mm}
      \begin{center}
     \mbox{%
\epsfxsize=79.055mm 
\epsfbox{bos-2.eps}
}
\end{center}
\vspace*{-9pt}
\Caption{Примеры сечений поверхности $\beta_t(y,T)$ для разных значений~$y$: сверху вниз~--- 
$y\hm=0; 1; \ldots$; $y\hm=15$}
\vspace*{3pt}
  \end{minipage}
\end{figure*}

 
     
     Процессы $y_t$, $z_t$ и~$u_t$ и~возмущения~$v_t$ и~$w_t$~--- 
скалярные, $\mathbb{N}(M,{\sf D})$ обозначает нормальное распределение со 
средним~$M$ и дисперсией~${\sf D}$. Известно, что траектории~$y_t$ 
неотрицательны, т.\,е.\ $y_t\hm>0$, процесс~--- эргодический, известны 
предельное распределение и~переходная ве\-ро\-ят\-ность. При этом начальные 
условия в~(\ref{e15-bos}) выбраны так, что $M$ и~${\sf D}$ значительно отличаются 
от предельных моментов, так что динамика ис\-сле\-ду\-ет\-ся в~переходном 
процессе. Соответственно, в~стационарном режиме процесс удовле\-тво\-ря\-ет 
предположениям модели~(1), а~также имеет линейный снос, т.\,е.\ мож\-но 
аналитически вычислить коэффициенты~(\ref{e13-bos}) и~оптимальное 
автономное управ\-ле\-ние~(\ref{e14-bos}). Также очевидна устой\-чи\-вость~$z_t$, 
так что вопрос остается только к~выполнению условия~4 тео\-ре\-мы, т.\,е.\ 
к~характеру по\-верх\-ности $\beta_t(y,T)$, опре\-де\-ля\-емой  
уравнением~(\ref{e11-bos}) (в~данном скалярном случае~--- од-\linebreak\vspace*{-12pt}

\pagebreak

\noindent
ним 
параболическим уравнением, а~не сис\-те\-мой) или формулой  
Фейн\-ма\-на--Ка\-ца~(\ref{e12-bos}).

     
     Отметим сначала, что в~рас\-смат\-ри\-ва\-емом примере не\-труд\-но найти 
точное решение уравнения Риккати~(\ref{e10-bos}), а~именно: 
    \begin{multline*}
     \alpha_t= \fr{C_\alpha e^{2\sqrt{3}\,t} (1+\sqrt{3})-1+\sqrt{3}}{1-
C_\alpha e^{2\sqrt{3}t}}\,,\\
     C_\alpha =\fr{3-\sqrt{3}}{3+\sqrt{3}}\,e^{-10\sqrt{3}}\,.
    \end{multline*}
     Уравнение~(\ref{e6-bos}) дает $\alpha_*\hm= \sqrt{3}\hm-1\hm\approx 
0{,}73$, что, как нетрудно видеть, совпадает с $\lim_{t\to -\infty} \alpha_t$. 
Для решения~(\ref{e11-bos}) есть три варианта: не учитывать част\-ный случай 
линейного сноса и~решать~(\ref{e11-bos}) традиционным сеточным методом 
(как предлагается в~\cite{5-bos}), или методом имитационного 
моделирования формулы Фейн\-ма\-на--Ка\-ца (как предлагается в~[4]), или 
решать обыкновенные дифференциальные уравнения~(\ref{e7-bos}). Все три 
расчета пред\-став\-ле\-ны в~[4]. Применительно к~рас\-смат\-ри\-ва\-емой автономной 
задаче все три метода под\-тверж\-да\-ют схо\-ди\-мость $\beta_t(y,T)$ и~дают 
$\beta_*(y)\hm= \beta_*^a y\hm+ \beta_*^b$, где $\beta_*^a\hm\approx -
0{,}19$, $\beta_*^b\hm\approx -0{,}11$, т.\,е.\ при любом варианте расчета 
$\beta_*^a y\hm= \lim_{t\to-\infty} \beta_t^a(y,5)$, $\beta_*^b\hm= \lim_{t\to-
\infty} \beta_t^b(5)$. Значения~$y$ определялись в~результате 
предварительного моделирования~(\ref{e16-bos}) и~заданы отрезком $[0,40]$. 
Иллюстрируют эти сходимости рис.~1 и~2.
     
  
\section{Заключение}

     Представленное исследование дополняет ранее\linebreak решенную задачу 
оптимального управления линейным выходом стохастической сис\-те\-мы по 
квад\-ра\-тич\-но\-му критерию традиционным автономным вариантом 
с~бесконечным временем. Основной \mbox{результат} эксплуатирует классические 
методы исследования уравнения Риккати, как и~в~традиционной 
 LQG-за\-да\-че. Важное отличие при этом заключается в том, что решение 
остается в~классе нелинейных управ\-ле\-ний, что было также важным\linebreak 
свойством задачи с~конечным горизонтом. Завершить изучение данной 
задачи должен случай \mbox{косвенных} наблюдений и~практический анализ 
качества автономного управ\-ле\-ния в~сравнении с~оптимальным на 
конечном горизонте. Эти исследования запланированы на будущее.
     
{\small\frenchspacing
 { %\baselineskip=10.6pt
 %\addcontentsline{toc}{section}{References}
 \begin{thebibliography}{99}
\bibitem{1-bos}
      \Au{Athans M.} The role and use of the stochastic linear-quadratic-Gaussian problem in 
control system design~// IEEE T. Automat. Contr., 1971. Vol.~16. No.\,6. P.~529--552. doi: 
10.1109/TAC.1971.1099818.

\bibitem{2-bos}
\Au{Wonham W.\,M.} Linear multivariable control. A~geometric approach.~--- Lecture notes in 
economics and mathematical systems ser.~--- Berlin: Springer-Verlag, 1974. Vol.~101. 347~p.
\bibitem{3-bos}
\Au{Девис М.\,Х.\,А.} Линейное оценивание и~сто\-ха\-сти\-че\-ское управ\-ле\-ние~/ Пер. с англ.~--- 
М.: Наука, 1984. 206~с. (\Au{Davis~M.\,H.\,A.} Linear estimation and stochastic control.~--- 
London: Chapman and Hall, 1977. 224~p.)
\bibitem{4-bos}
\Au{Босов А.\,В.} Задача управления линейным выходом нелинейной не\-управ\-ля\-емой 
стохастической дифференциальной сис\-те\-мы по квад\-ра\-тич\-но\-му критерию~// Известия 
РАН. Теория и~сис\-те\-мы управ\-ле\-ния, 2021. №\,5. C.~52--73. doi: 
10.31857/S000233882104003X.
\bibitem{5-bos}
\Au{Босов А.\,В., Стефанович~А.\,И.} Управ\-ле\-ние выходом стохастической 
дифференциальной сис\-те\-мы по квад\-ра\-тич\-но\-му критерию. II. Численное решение 
уравнений динамического программирования~// Информатика и её применения, 2019. 
Т.~13. Вып.~1. С.~9--15. doi: 10.14357/19922264190102. EDN: ZASZFR.
\bibitem{6-bos}
\Au{Босов А.\,В., Стефанович~А.\,И.} Управление выходом стохастической 
дифференциальной сис\-те\-мы по квад\-ра\-тич\-но\-му критерию. IV. Альтернативное численное 
решение~// Информатика и~её применения, 2020. Т.~14. Вып.~1. С.~24--30. doi: 
10.14357/19922264200104. EDN: XNHVFT.
\bibitem{7-bos}
\Au{Флеминг У., Ришел~Р.} Оптимальное управление детерминированными 
и~стохастическими сис\-те\-ма\-ми~/ Пер. с~англ.~--- М.: Мир, 1978. 316~с. 
(\Au{Fleming~W.\,H., Rishel~R.\,W.} Deterministic and stochastic optimal control.~--- New 
York, NY, USA: Springer-Verlag, 1975. 222~p.)
\bibitem{8-bos}
\Au{Ширяев А.\,Н.} Вероятность.~--- 2-е изд.~--- М.: Наука, 1989. 640~с.
\bibitem{9-bos}
\Au{{\ptb{\O}}\,\,ksendal~B.} Stochastic differential equations. An introduction with 
applications.~--- New York, NY, USA: Springer-Verlag, 2003. 324~p.
\bibitem{10-bos}
      \Au{Cox J.\,C., Ingersoll~J.\,E., Ross~S.\,A.} A~theory of the term structure of interest 
rates~// Econometrica, 1985. Vol.~53. Iss.~2. P.~385--407. doi: 10.2307/1911242.
\bibitem{11-bos}
      \Au{Bohacek S., Rozovskii~B.} A~diffusion model of roundtrip time~// Comput. 
Stat. Data An., 2004. Vol.~45. Iss.~1. P.~25--50. doi:  
10.1016/S0167-9473(03)00114-2.

\end{thebibliography}

 }
 }

\end{multicols}

\vspace*{-10pt}

\hfill{\small\textit{Поступила в~редакцию 07.12.23}}

\vspace*{8pt}

%\pagebreak

%\newpage

%\vspace*{-28pt}

\hrule

\vspace*{2pt}

\hrule



\def\tit{AUTONOMOUS DIFFERENTIAL SYSTEM LINEAR OUTPUT 
CONTROL BY~SQUARE CRITERION ON~AN~INFINITE HORIZON\\[-5pt]}


\def\titkol{Autonomous differential system linear output 
control by~square criterion on~an~infinite horizon}


\def\aut{A.\,V.~Bosov}

\def\autkol{A.\,V.~Bosov}

\titel{\tit}{\aut}{\autkol}{\titkol}

\vspace*{-15pt}


\noindent
Federal Research Center ``Computer Science and Control'' of the Russian Academy of 
Sciences, 44-2~Vavilov Str., Moscow 119333, Russian Federation

\def\leftfootline{\small{\textbf{\thepage}
\hfill INFORMATIKA I EE PRIMENENIYA~--- INFORMATICS AND
APPLICATIONS\ \ \ 2024\ \ \ volume~18\ \ \ issue\ 1}
}%
 \def\rightfootline{\small{INFORMATIKA I EE PRIMENENIYA~---
INFORMATICS AND APPLICATIONS\ \ \ 2024\ \ \ volume~18\ \ \ issue\ 1
\hfill \textbf{\thepage}}}

\vspace*{1pt}

      
      
     
     \Abste{The problem of optimal control of the stochastic differential system 
linear output on an infinite horizon is solved. The solution is considered as the 
limit form of optimal control in the corresponding problem with a~finite horizon. 
Sufficient conditions for the existence of control are given. They consist of the 
requirements of the stationarity of nonlinear dynamics, the finiteness of the 
quadratic target functional, the stabilizability of the linear output, and the existence 
of a limit in the Feynman--Katz formula describing the nonlinear part of control. 
The conditions for the linear part of the control are related to the classical results of 
the existence of a~solution to the autonomous Riccati equation. The existence of 
a~limit in the Feynman--Katz formula is associated with the solution of a~parabolic 
equation that sets the coefficients for the nonlinear part of the control. A~special 
case of linear drift is considered in which the nonlinear nature of the problem is 
preserved but optimal control turns out to be linear both in output and in the state 
variable. The results of a numerical experiment are presented which makes it 
possible to analyze the transient process in a~problem with a~finite horizon and an 
ergodic process in dynamics. For the control coefficients, the limiting transition to 
the optimal values of the corresponding optimal autonomous control is illustrated.}
     
     \KWE{stochastic differential Ito system; output control; optimal control; 
quadratic criterion; parabolic equation; Feynman--Katz formula}
     
     


\DOI{10.14357/19922264240103}{UEESFO}

\vspace*{-22pt}

\Ack

\vspace*{-3pt}


     \noindent
     The research was carried out using the infrastructure of the Shared Research 
Facilities ``High Performance Computing and Big Data'' (CKP ``Informatics'') of 
FRC CSC RAS (Moscow).


  \begin{multicols}{2}

\renewcommand{\bibname}{\protect\rmfamily References}
%\renewcommand{\bibname}{\large\protect\rm References}

{\small\frenchspacing
 {%\baselineskip=10.8pt
 \addcontentsline{toc}{section}{References}
 \begin{thebibliography}{99} 
\bibitem{1-bos-1}
      \Aue{Athans, M.} 1971. The role and use of the stochastic 
 linear-quadratic-Gaussian problem in control system design. \textit{IEEE T. 
Automat. Contr.} 16(6):529--552. doi: 10.1109/TAC.1971.1099818.
\bibitem{2-bos-1}
      \Aue{Wonham, W.\,M.} 1974. \textit{Linear multivariable control. 
A~geometric approach.} Lecture notes in economics and mathematical systems 
ser. Berlin: Springer-Verlag. 347~p.
\bibitem{3-bos-1}
      \Aue{Davis, M.\,H.\,A.} 1977. \textit{Linear estimation and stochastic 
control}. London: Chapman and Hall. 224~p.



\bibitem{4-bos-1}
      \Aue{Bosov, A.\,V.} 2021. The problem of controlling the linear output of 
a~nonlinear uncontrollable stochastic differential system by the square criterion. 
\textit{J.~Comput. Sys. Sc. Int.} 60(5):719--739. doi: 
10.1134/S1064230721040031.

%\vspace*{-1pt}

\bibitem{5-bos-1}
      \Aue{Bosov, A.\,V., and A.\,I.~Stefanovich.} 2019. Upravlenie vykhodom 
stokhasticheskoy differentsial'noy sistemy po kvadratichnomu kriteriyu. II. 
Chislennoe reshenie uravneniy dinamicheskogo programmirovaniya [Stochastic 
differential system output control by the quadratic criterion. II. Dynamic 
programming equations numerical solution]. \textit{Informatika i~ee 
Primeneniya~--- Inform. \mbox{Appl.}} 13(1):9--15. doi: 10.14357/19922264190102. EDN: 
\mbox{ZASZFR}.



%\vspace*{-1pt}

\bibitem{6-bos-1}
      \Aue{Bosov, A.\,V., and A.\,I.~Stefanovich.} 2020. Upravlenie vykhodom 
sto\-kha\-sti\-che\-skoy differentsial'noy sistemy po
kvadratichnomu kriteriyu. IV. 
Al'ternativnoe chislennoe\linebreak\vspace*{-12pt}

\pagebreak

\noindent
  reshenie [Stochastic differential system output control 
by the quadratic criterion. IV. Alternative numerical decision]. \textit{Informatika 
i~ee Primeneniya~--- Inform. \mbox{Appl.}} 14(1):24--30. doi: 
10.14357/19922264200104. EDN: \mbox{XNHVFT}.
\bibitem{7-bos-1}
      \Aue{Fleming, W.\,H., and R.\,W.~Rishel}. 1975. \textit{Deterministic and 
stochastic optimal control}. New York, NY: Springer-Verlag. 222~p.
\bibitem{8-bos-1}
      \Aue{Shiryaev, A.\,N.} 1996. \textit{Probability}. New York, NY: Springer 
Verlag. 624~p.



\bibitem{9-bos-1}
      \Aue{{\ptb{\O}}ksendal,~B.} 2003. \textit{Stochastic differential equations. An 
introduction with applications}. New York, NY: Springer-Verlag. 324~p.
\bibitem{10-bos-1}
      \Aue{Cox, J.\,C., J.\,E.~Ingersoll, and S.\,A.~Ross.} 1985. A~theory of the 
term structure of interest rates. \textit{Econometrica} 53(2):385--407. doi: 
10.2307/1911242.
\bibitem{11-bos-1}
      \Aue{Bohacek, S., and B.~Rozovskii.} 2004. A~diffusion model of roundtrip 
time. \textit{Comput. Stat. Data An.} 45(1):25--50. doi:  
10.1016/S0167-9473(03)00114-2.

\end{thebibliography}

 }
 }

\end{multicols}

\vspace*{-6pt}

\hfill{\small\textit{Received December 7, 2023}} 

%\vspace*{-18pt}
     
     \Contrl
     
     \vspace*{-3pt}
     
     \noindent
     \textbf{Bosov Alexey V.} (b.\ 1969)~--- Doctor of Science in technology, 
principal scientist, Federal Research Center ``Computer Science and Control'' of 
the Russian Academy of Sciences, 44-2~Vavilov Str., Moscow 119333, Russian 
Federation; \mbox{avbosov@ipiran.ru}


\label{end\stat}

\renewcommand{\bibname}{\protect\rm Литература}         %2
%\def\M{\mathop{\kern\z@\mbox{\bfseries\sffamily\upshape E}}\nolimits}
%\def\P{\mathop{\kern\z@\mbox{\bfseries\sffamily\upshape P}}\nolimits}

\def\stat{pavlov}

\def\tit{СВЯЗНОСТЬ КОНФИГУРАЦИОННЫХ ГРАФОВ\\ В~МОДЕЛЯХ СЛОЖНЫХ СЕТЕЙ$^*$}

\def\titkol{Связность конфигурационных графов в~моделях сложных сетей}

\def\aut{Ю.\,Л.~Павлов$^1$}

\def\autkol{Ю.\,Л.~Павлов}

\titel{\tit}{\aut}{\autkol}{\titkol}

\index{Павлов Ю.\,Л.}
\index{Pavlov Yu.\,L.}

{\renewcommand{\thefootnote}{\fnsymbol{footnote}} \footnotetext[1]
{Финансовое обеспечение исследований осуществлялось из средств федерального бюджета на 
выполнение государственного задания Карельского научного центра Российской академии наук 
(Институт прикладных математических исследований КарНЦ РАН).}}

\renewcommand{\thefootnote}{\arabic{footnote}}
\footnotetext[1]{Институт прикладных математических исследований КарНЦ РАН, 
ФИЦ <<Карельский научный центр РАН>>, \mbox{pavlov@krc.karelia.ru}}


%\vspace*{-12pt}





\Abst{Рассматриваются конфигурационные графы, степени вершин которых являются независимыми 
случайными величинами, одинаково распределенными по обобщенному дискретному степенному закону. 
Связи между вершинами формируются равновероятно в~соответствии со степенями вершин. Эти 
случайные графы часто используются для моделирования сложных сетей коммуникаций, таких как 
интернет и~социальные сети. В статье предполагается, что распределение степеней вершин неизвестно, 
поскольку зависит от медленно меняющейся функции с~неизвестными свойствами. При стремлении числа 
вершин к~бесконечности найдены условия, при выполнении которых граф становится асимптотически 
достоверно связным. При этих условиях получены оценки скорости сходимости к~нулю вероятности 
того, что граф не связен. Для доказательства результатов статьи использовались свойства устойчивых 
распределений и~медленно меняющихся функций.}


\KW{случайные графы; конфигурационные графы; случайные степени вершин; связность графа}

\DOI{10.14357/19922264210103}


\vskip 10pt plus 9pt minus 6pt

\thispagestyle{headings}

\begin{multicols}{2}

\label{st\stat}

\section{Введение}
\label{SC:1}

Случайные графы широко используются при моделировании сложных сетей коммуникаций, таких как 
интернет, транспортные, телефонные, социальные сети и~т.\,д.~\cite{Hof}. 
Такие модели обычно соответствуют известным свойствам реальных сетей, обнаруженным в~ходе 
многочисленных эмпирических исследований и~подробно описанным в~различных публикациях 
(см., например,~\cite{Fal}). Наблюдения показали, что в~графах, описывающих топологию сетей, 
степени вершин можно считать независимыми одинаково распределенными случайными величинами. 
Одно из важнейших свойств, присущих сетям коммуникаций различной природы, состоит в~том, 
что число вершин степени, не меньшей чем~$k,$ при больших~$k$ пропорционально~$k^{-\tau},$ при 
этом значения положительного параметра~$\tau$ могут отличаться для разных сетей. 
Поэтому во многих работах предполагается, что распределение случайной величины~$\xi,$ равной 
степени любой вершины графа, можно задать следующим образом:
\begin{equation}
\label{EQ:1}
\mathsf{P}\{\xi \geqslant k\} = \fr{h(k)}{k^\tau}\,,\enskip k=1,2,\ldots,
\end{equation}
где $h(k)$~--- медленно меняющаяся функция.

В настоящее время одной из наиболее популярных моделей сложных сетей коммуникаций 
служит так называемый конфигурационный граф, определение которого было дано в~статье~\cite{Bol}. 

Обозначим через~$N$ число вершин графа. Степени вершин задаются~$N$~независимыми реализациями 
случайной величины~$\xi$, и~они равны числу инцидентных каждой вершине полуребер, т.\,е.\ 
ребер, для которых смежные вершины еще не определены. Все полуребра различимы (занумерованы).
 Граф строится путем попарного равновероятного соединения полуребер друг с~другом для образования ребер. 
 Разумеется, сумма степеней всех вершин любого графа должна быть четной, в~случае нечетной суммы в~граф 
 вводится вспомогательная вершина единичной степени. В~статье~\cite{RN1} было замечено, что эта 
 вспомогательная вершина не влияет на основные асимптотические свойства графа при $N\hm\rightarrow \infty.$ 
 Поэтому ниже не будем учитывать эту дополнительную вершину, что фактически соответствует 
 предположению о~чет\-ности суммы степеней. Если же это предположение неверно, то, как легко 
 убедиться, следуя доказательству полученных в~данной \mbox{статье} результатов, дополнительная вершина 
 не влияет на эти результаты. Заметим еще, что описанная конструкция конфигурационных графов 
 допускает появление петель и~кратных ребер.

Асимптотические свойства случайных конфигурационных графов при $N\hm\rightarrow \infty$ изучались 
многими авторами, наиболее полные обзоры таких работ можно найти в~\cite{Hof,Dur}. Известно, что 
если $\tau \hm\in (1, 2),$ то асимптотически достоверно, т.\,е.\ с~ве\-ро\-ят\-ностью, стремящейся к~единице, граф содержит единственную гигантскую компоненту связности, число вершин в~которой 
пропорционально~$N,$ в~то время как объемы других компонент имеют порядок~$o(N).$

В~статье~\cite{RN1} рассматривался простейший вариант распределения~(\ref{EQ:1}) 
в~предположении, что $h(k)\hm\equiv 1.$ Для таких графов даны оценки объема 
и~диа\-мет\-ра гигантской компоненты связности и~\mbox{подробно} исследована ее структура.

В \cite{PavCh} впервые рассматривались условные конфигурационные графы с~распределением~(\ref{EQ:1}) 
степеней вершин в~случае $h(k)\hm\equiv 1$ при условии, что число ребер известно. 
В~\cite{PavKh} такие графы исследовались уже при условии, что число ребер ограничено сверху. 
Далее, в~\cite{Pav} в~распределении~(\ref{EQ:1}) степеней вершин условных графов медленно меняющаяся 
функция~$h(k)$ уже не предполагалась известной, но задавалась асимптотика вероятностей 
стремящихся к~бесконечности значений степеней вершин:
\begin{equation}
\label{EQ:2}
\mathsf{P}\{\xi = k\} \sim \fr{d}{k^g (\ln k)^\eta}\,,
\end{equation}
где $k\rightarrow \infty$, $g\hm>1$, $\eta \hm\geqslant 0.$ Нетрудно проверить, что 
распределение~(\ref{EQ:1}) удовлетворяет условию~(\ref{EQ:2}), если $d\hm=\tau$, 
$g\hm=\tau \hm+1$ и~$h(k)\hm=(\ln k)^{-\eta}.$
Хотя значения~$h(k)$ в~(\ref{EQ:1}) известны не для всех~$k$, знание асимптотического поведения 
вероятностей~(\ref{EQ:2}) существенно облегчило  доказательство полученных в~\cite{Pav} результатов.

Обозначим $A_N$ событие, состоящее в~том, что граф не связен.
В~теореме~4.15 книги~\cite{Hof2} показано, что если $\mathsf{P}\{\xi \hm= 1\} \hm= \mathsf{P}\{\xi\hm = 2\}
\hm =0,$ то при $N\hm\rightarrow \infty$ асимптотически достоверно граф состоит из 
единственной компоненты связности, содержащей все~$N$~вершин. 
В~\cite{Hof2} установлено также, что в~этом случае 
$\mathsf{P}\{A_N\}\hm=O(1/N).$ В~статье~\cite{Pav2} были найдены условия асимптотически достоверной 
связ\-ности конфигурационных графов, степени вершин которых обладают свойством~(\ref{EQ:2}), в~том числе 
и~в~случаях $\mathsf{P}\{\xi \hm= 1\}\hm>0$, $\mathsf{P}\{\xi \hm= 2\}>0.$

Таким образом, осталась нерешенной задача нахождения условий, при выполнении которых 
асимптотически достоверно связен случайный конфигурационный граф, 
степени вершин которого заданы распределением~(\ref{EQ:1}) с~неизвестной медленно меняющейся функцией~$h(k),$ 
необязательно обеспечивающей выполнение соотношения~(\ref{EQ:2}). Основным результатом настоящей статьи 
является доказанная ниже тео\-ре\-ма, в~которой найдены такие условия и~даны оценки скорости сходимости 
к~нулю вероятности того, что граф не связен. 
Согласно теореме~4.4~\cite{Hof2}, если 
$\mathsf{M} \xi(\xi\hm-1)/\mathsf{M} \xi\hm >1,$ 
то граф асимптотически достоверно содержит больше одной компоненты связности. 
Нетрудно видеть, что это имеет место в~случае $\tau\hm >1,$ поэтому в~теореме 
рассматривается только случай $0\hm <\tau \hm\leqslant 1.$

Доказательство теоремы основано на идеях, изложенных в~доказательстве теоремы~4.15~\cite{Hof2}, но 
с~существенными изменениями, связанными с~использованием, как и~в~\cite{Pav2}, 
методов исследования локальной сходимости распределений сумм независимых случайных величин к~устойчивым 
законам~\cite{IL}. Кроме того, впервые при решении подобных задач используются общие свойства 
медленно\linebreak меняющихся функций, изложение со\-от\-вет\-ст\-ву\-ющей теории можно найти, например, в~\cite{Bin}.

В следующем разделе в~виде теоремы сформулирован основной результат статьи, доказательство 
этой теоремы приводится в~разд.~3.


\section{Основной результат}
\label{SC:2}

Пусть конфигурационный граф содержит~$N$~вершин, степени которых независимы и~одинаково 
распределены в~соответствии с~(\ref{EQ:1}), где $h(k)$, $k\hm\geqslant 1,$~--- 
измеримая медленно меняющаяся функция. Предположим для простоты, что максимальный шаг 
распределения случайной величины~$\xi$ равен единице. Введем последовательность~$B_N$, 
$N\hm=1,2,\ldots,$ при $N\hm\rightarrow \infty$ удовлетворяющую условию
\begin{equation}
\label{EQ:3}
B_N\sim \left(Nh\left([B_N]\right)\right)^{1/\tau},
\end{equation}
где $[x]$ означает целую часть числа~$x.$ Отсюда и~из свойств медленно меняющихся функций 
очевидным образом вытекает, что $B_N\hm\rightarrow \infty.$ Примером 
построения такой последовательности для распределения, имеющего свойство~(\ref{EQ:2}), 
могут служить величины вида $B_N\hm=N^{1/\tau},$ если $\eta\hm=0,$ а если $0\hm<\eta \hm<1,$ то
\begin{equation}
\label{EQ:4}
B_N=\left(N\left(\fr{\tau}{\ln N}\right)^\eta\right)^{1/\tau}.
\end{equation}

Теперь можно сформулировать основной результат статьи.

\smallskip

\noindent
\textbf{Теорема.}\ \textit{Пусть $N\hm\rightarrow \infty$ и~$\mathsf{P}\{\xi \hm= 2\}\hm>0.$ 
Тогда с~вероятностью, сколь угодно близкой к~единице,
справедливы следующие утверждения}.
\begin{enumerate}
\item \textit{Если $\mathsf{P}\{\xi = 1\}\hm=0$ и~$\tau\hm =1,$ то $\mathsf{P}\{A_N\}\hm=O(1/\ln N)$}.
\item \textit{Если $\mathsf{P}\{\xi = 1\}\hm=0$ и~$0<\tau \hm<1,$ то $\mathsf{P}\{A_N\}\hm=O(N/B_N)$}.
\item \textit{Если $\mathsf{P}\{\xi = 1\}\hm>0$ и~$0<\tau\hm <1/2,$ то $\mathsf{P}\{A_N\}\hm=O(N^2/B_N)$}.
\end{enumerate}

\section{Доказательство теоремы} 
\label{SC:3}

Пусть произошло событие~$A_N.$ Тогда множество~$V$ всех вершин графа можно представить в~виде 
$V\hm=V_1\bigcup V_2,$ где $V_1$ и~$V_2$~--- 
непересекающиеся множества вершин такие, что не существует ребер, соединяющих вершины из~$V_1$ 
с~вершинами из~$V_2.$ Можем считать, что $|V_1| \hm \leqslant |V_2|,$ где $|V_1|$ и~$|V_2|$~--- 
мощности множеств~$V_1$ и~$V_2$ соответственно. Тогда $|V_1| \hm \leqslant N/2.$ 
Обозначим~$\Omega$ множество всех возможных разбиений~$V$ на~$V_1$ и~$V_2.$ 
Пусть случайные величины $\xi_1,\ldots, \xi_N$ равны степеням вершин $1,\ldots, N$ 
соответственно. Понятно, что распределения этих независимых случайных величин совпадают с~(\ref{EQ:1}). 
Представляет интерес предельное поведение их суммы
\begin{equation}
\label{EQ:5}
\zeta_N=\xi_1+\cdots +\xi_N.
\end{equation}
Обозначим также
\begin{equation}
\label{EQ:6}
\zeta_N^{(1)}=\sum\limits_{i\in V_1} \xi_i, \quad  \zeta_N^{(2)}=\sum\limits_{i\in V_2} \xi_i.
\end{equation}

Ясно, что общее число различных графов с~суммой степеней вершин~(\ref{EQ:5}) равно 
$(\zeta_N\hm-1)!!$ Учитывая равновероятность соединения полуребер при образовании ребер, находим, что
$$
\mathsf{P}\{A_N\}\leqslant \sum\limits_{V_1,V_2\in \Omega} 
\fr{(\zeta_N^{(1)}-1)!!(\zeta_N^{(2)}-1)!!}{(\zeta_N-1)!!}\,.
$$
Отсюда следует оценка вероятности того, что граф не связен:
\begin{equation*}
%\label{EQ:7}
\mathsf{P}\{A_N\} \leqslant \sum\limits_{V_1\in \Omega} \prod\limits_{j=1}^{\zeta_N^{(1)}/2} 
\fr{\zeta_N^{(1)}-2j+1}{\zeta_N-2j+1}\,.
\end{equation*}
Верхний предел произведения в~этом выражении равен~$\zeta_N^{(1)}/2,$ 
поскольку, очевидно, $\zeta_N^{(1)},$ как и~$\zeta_N^{(2)},$ принимает четные значения.

Пусть $N\rightarrow \infty.$ Обозначим~$F(x)$ функцию распределения случайной величины~$\xi.$ 
Из~(\ref{EQ:1}) следует, что
\begin{equation}
\label{EQ:8}
F(x)=
\begin{cases}
0\,, & \!\!\!\mbox{если} x<0;\\
1-\fr{h(x)}{x^\tau}\left(1 +o(1)\right)& \!\!\!\mbox{при } x\rightarrow \infty.
\end{cases}
\end{equation}
В силу (\ref{EQ:8}) функция~$F(x)$ удовлетворяет условиям теоремы~2.6.1 книги~\cite{IL}, 
следовательно, она\linebreak принадлежит области притяжения устойчивого\linebreak закона~$G(x)$ с~показателем~$\tau.$ 
Применив тео\-ре\-му~2.2.2~\cite{IL}, находим, что если $0\hm<\tau \hm<1,$ то логарифм 
характеристической функции~$\varphi_G(t)$ устойчивого закона~$G(x)$ имеет вид:
\begin{equation}
\label{EQ:10}
\ln \varphi_G(t)=i\gamma t-c|t|^\tau \left(1+i\beta \fr{t}{|t|}\tan \frac{\pi \tau}{2}\right),
\end{equation}
где $\gamma$~--- некоторая постоянная, 
а~значения па\-ра\-мет\-ров~$\beta$ и~$c$ с~по\-мощью~(\ref{EQ:8}) 
%и~(\ref{EQ:9}) 
определены в~ходе доказательства теоремы~2.2.2~\cite{IL}: $\beta \hm=-1;$
\begin{equation}
\label{EQ:11}
c=\Gamma (1-\tau)\cos \fr{\pi \tau}{2}\,,
\end{equation}
где $\Gamma (x)$~--- гам\-ма-функ\-ция.

Вид логарифма характеристической функции~$\varphi_G(t)$ при $\tau \hm=1$ определяется аналогичным образом:
\begin{equation}
\label{EQ:12}
\ln \varphi_G(t)=i\gamma t-\fr{\pi}{2}\left\vert t\right\vert 
\left(1+i\fr{t}{|t|}\,\fr{2}{\pi}\,\ln |t|\right).
\end{equation}
В выражениях~(\ref{EQ:10}) и~(\ref{EQ:12}) константу~$\gamma$ можно сделать равной нулю,
 подобрав нужным образом нормирующие множители в~определении области притяжения закона~$G(x)$~\cite{IL}. 
 Обозначим~$\varphi (t)$ характеристическую функцию случайной величины~$\xi.$ 
 Из~(\ref{EQ:10}), (\ref{EQ:12}) и~теоремы~2.6.5~\cite{IL} следует, что в~окрестности нуля
\begin{equation}
\label{EQ:13}
\ln \varphi(t)=-c|t|^\tau l(t)\left(1+i\beta\fr{t}{|t|}\,\omega (t,\tau)\right),
\end{equation}
где $\beta \hm=-1,$ а~$c$ определено в~(\ref{EQ:11}) в~случае $0\hm<\tau \hm<1,$ 
а~если $\tau \hm=1,$ то $\beta \hm=1$, $c\hm=\pi/2,$ функция~$l(t)$ является медленно меняющейся 
и
\begin{equation}
\label{14}
\omega (t,\tau)=\begin{cases}
\tan \fr{\pi \tau}{2},& 0<\tau<1\,;\\
\fr{2}{\pi}\ln |t|,& \tau =1\,.
\end{cases}
 \end{equation}
В доказательстве теоремы~2.6.8~[11, с.~104--108] 
показано, что для распределения~(\ref{EQ:1}) при $t\hm\rightarrow 0$
\begin{equation}
\label{EQ:15}
l(t)=h\left(\fr{1}{|t|}\right)\left(1+o(1)\right)\,.
\end{equation}

Очевидно, что $\varphi (0)\hm=1.$ Рассмотрим $\varphi (t)$ при фиксированных $t\hm\neq 0.$ 
Пусть $0\hm<\tau \hm<1.$ Используя~(\ref{EQ:3}), (\ref{EQ:11}), (\ref{EQ:13})--(\ref{EQ:15}) 
и~проводя простые вычисления, находим, что если $N\hm\rightarrow \infty,$ то
$$
\varphi^N\left(\fr{t}{B_N}\right) \rightarrow 
\exp \left\{-c|t|^\tau \left(1-i\fr{t}{|t|}\tan \fr{\pi \tau}{2}\right)\right\}.
$$
Это соотношение показывает, что распределения сумм~$\zeta_N$ слабо сходятся к~устойчивому 
закону с~показателем~$\tau.$ Заметим, что, согласно теореме~4.2.1~\cite{IL}, 
на самом деле имеет место и~локальная сходимость. Это значит, что при достаточно больших~$N$ 
и~любом сколь угодно малом $\varepsilon \hm>0$ существует положительная константа~$L$ такая, что
\begin{equation}
\label{EQ:16}
\mathsf{P}\left\{\fr{B_N}{L}\leqslant \zeta_N \leqslant LB_N\right \}>1-\varepsilon\,.
\end{equation}

Пусть $\tau =1.$ С~по\-мощью~(\ref{EQ:3}), (\ref{EQ:13})--(\ref{EQ:15}) и~известных свойств
 медленно меняющихся функций~\cite{Bin} нетрудно вывести, что при $N\hm\rightarrow \infty$ 
 существует стремящаяся к~нулю последовательность~$q(N)$ такая, что при любом фиксированном $t\hm\neq 0$
\begin{multline}
\label{EQ:17}
\varphi^N\left(\fr{t}{B_N}\right) \exp 
\{-itN(\ln N)(1+q(N))\}\rightarrow{}\\
{}\rightarrow \exp 
\left\{-\fr{\pi}{2}\left\vert t\right\vert
\left(1+i\fr{t}{|t|}\,\fr{2}{\pi}\ln |t|\right)\right\}.
\end{multline}
Из~(\ref{EQ:17}) и~локальной предельной теоремы~4.2.1~\cite{IL} 
теперь вытекает, что асимптотически достоверно
\begin{equation}
\label{EQ:18}
\zeta_N\sim N\ln N\,.
\end{equation}

Рассмотрим предельное поведение~$\zeta_N^{(1)}.$ Если объем множества~$V_1$ конечен, то из~(\ref{EQ:1}) 
и~(\ref{EQ:6}) нетрудно получить, что сумма~$\zeta_N^{(1)}$ асимптотически достоверно конечна.
 Пусть $|V_1|\hm\rightarrow \infty.$ Если $0\hm<\tau \hm<1,$ то по аналогии с~(\ref{EQ:3}) 
 и~(\ref{EQ:16}) находим, что при достаточно большом~$|V_1|$ и~достаточно малом $\varepsilon \hm>0$ 
 существует положительная константа~$L_1$ такая, что
\begin{equation}
\label{EQ:19}
\mathsf{P}\left\{\fr{B_N^{(1)}}{L_1}\leqslant \zeta_N^{(1)} \leqslant L_1B_N^{(1)}\right\}>1-\varepsilon\,,
\end{equation}
где
$$
B_N^{(1)}=(|V_1|h([B_N^{(1)}]))^{1/\tau}.
$$
Если же $\tau \hm=1,$ то замечаем, что, подобно~(\ref{EQ:18}),
\begin{equation}
\label{EQ:20}
\zeta_N^{(1)}\sim \left\vert V_1\right\vert \ln \left\vert V_1\right\vert.
\end{equation}

Теперь для того, чтобы получить утверж\-де\-ния~\mbox{1--3}, достаточно повторить доказательство 
тео\-ре\-мы статьи~\cite{Pav2}, в~котором~(\ref{EQ:4}) заменить на~(\ref{EQ:3}) и~для 
оценки предельного поведения сумм~$\zeta_N$ и~$\zeta_N^{(1)}$ использовать соотношения~(\ref{EQ:16}), 
(\ref{EQ:18})--(\ref{EQ:20}).



{\small\frenchspacing
{%\baselineskip=10.8pt
%\addcontentsline{toc}{section}{References}
\begin{thebibliography}{99}

%\vspace*{-2pt}
\bibitem{Hof}
\Au{Hofstad R.} Random graphs and complex networks.~--- Cambridge:
Cambridge University Press, 2017.  Vol.~1. 337~p.

\bibitem{Fal}
\Au{Faloutsos C., Faloutsos~P., Faloutsos~M.} 
On power-law relationships of the Internet topology~// Comput. Commun. Rev., 1999. Vol.~29. P.~251--262.

\bibitem{Bol}
\Au{Bollobas B.\,A.} A~probabilistic proof of an asymptotic formula
for the number of labelled regular graphs~// Eur. J.~Combin., 1980. Vol.~1. P.~311--316.

\bibitem{RN1}
\Au{Reittu H., Norros~I.} On the power-law random graph model of massive
data networks~// Perform. Evaluation, 2004. Vol.~55. P.~3--23.

\bibitem{Dur}
\Au{Durrett R.} Random graph dynamics.~--- Cambridge:
Cambridge University Press, 2007. 212~p.

\bibitem{PavCh}
\Au{Павлов Ю.\,Л., Чеплюкова~И.\,А.} Случайные графы Ин\-тер\-нет-ти\-па 
и~обобщенная схема размещения~// Дискретная математика, 2008. Т.~20. Вып.~3. С.~3--18.

\bibitem{PavKh}
\Au{Павлов Ю.\,Л., Хворостянская~Е.\,В.} 
О~предельных распределениях степеней вершин конфигурационных графов с~ограниченным числом ребер~// 
Математический сборник, 2016. Т.~207. Вып.~3. С.~93--110.

\bibitem{Pav}
\Au{Павлов Ю.\,Л.} Условные конфигурационные графы со случайным параметром степенного распределения степеней~// 
Математический сборник, 2018. Т.~209. Вып.~2. С.~120--137.

\bibitem{Hof2}
\Au{Hofstad R.} Random graphs and complex networks. Vol.~2~// 
Notes RGCNII, November~16, 2020. 341~p. {\sf https://www.win.tue.nl/$\sim$rhofstad/NotesRGCNII.pdf}.


\bibitem{Pav2}
\Au{Павлов Ю.\,Л.} О~связности конфигурационных графов~// Дискретная математика, 2019. Т.~31. Вып.~2. С.~115--123.

\bibitem{IL}
\Au{Ибрагимов И.\,А., Линник~Ю.\,В.} 
Независимые и~стационарно связанные величины.~--- М: Наука, 1965. 524~с.

\bibitem{Bin}
\Au{Bigham N.\,H., Goldie~C.\,M., Teugels~J.\,L.} 
Regular variations. Encyclopedia of mathematics and its applications. ~--- 
Cambridge: Cambridge University Press, 1987. Vol.~27. 513~p.
\end{thebibliography}

}
}

\end{multicols}

\vspace*{-6pt}

\hfill{\small\textit{Поступила в~редакцию 15.04.2020}}

\vspace*{6pt}

%\pagebreak

%\newpage

%\vspace*{-28pt}

\hrule

\vspace*{2pt}

\hrule

\vspace*{-2pt}

\def\tit{CONNECTIVITY OF CONFIGURATION GRAPHS IN~COMPLEX NETWORK MODELS}

\def\titkol{Connectivity of configuration graphs in~complex network models}

\def\aut{Yu.\,L.~Pavlov}

\def\autkol{Yu.\,L.~Pavlov}

\titel{\tit}{\aut}{\autkol}{\titkol}

\vspace*{-11pt}


\noindent
Institute of Applied Mathematical Research of the Karelian Research Centre
 of the Russian Academy of Sciences, 11~Pushkinskaya Str., Petrozavodsk 185910, Russian Federation

\def\leftfootline{\small{\textbf{\thepage}
\hfill INFORMATIKA I EE PRIMENENIYA~--- INFORMATICS AND
APPLICATIONS\ \ \ 2021\ \ \ volume~15\ \ \ issue\ 1}
}%
\def\rightfootline{\small{INFORMATIKA I EE PRIMENENIYA~---
INFORMATICS AND APPLICATIONS\ \ \ 2021\ \ \ volume~15\ \ \ issue\ 1
\hfill \textbf{\thepage}}}

\vspace*{3pt}




\Abste{The author considers configuration graphs whose degrees of vertices are independent 
and identically distributed according to the generalized power-law distribution. 
Connections between vertices are equiprobably\linebreak\vspace*{-12pt}}

\Abstend{formed in compliance with their degrees. 
Such random graphs are often used for modeling complex communication networks like the 
Internet and social networks. It is assumed that the distribution of vertex degrees 
is unknown because it depends on a slowly varying function with unknown properties. 
The conditions are found under which a~graph is asymptotically almost surely connected
 as the number of vertices tends to infinity. Under these conditions, estimates 
 of the convergence rate to zero of the probability that the graph is not connected 
 are obtained. The results in the present paper are proved using the properties of 
 stable distributions and slowly varying functions.}


\KWE{random graphs; configuration graphs; random vertex degrees; graph connectivity}

\DOI{10.14357/19922264210103}

%\vspace*{-15pt}

\Ack
\noindent
The study was carried out under state order to the Karelian Research Centre 
of the Russian Academy of Sciences
(Institute of Applied Mathematical Research KarRC RAS).

\vspace*{12pt}

  \begin{multicols}{2}

\renewcommand{\bibname}{\protect\rmfamily References}
%\renewcommand{\bibname}{\large\protect\rm References}

{\small\frenchspacing
 {%\baselineskip=10.8pt
 \addcontentsline{toc}{section}{References}
 \begin{thebibliography}{99}

\bibitem{1-pav}
\Aue{Hofstad, R.} 2017. 
\textit{Random graphs and complex networks.} Cambridge:
Cambridge University Press.  Vol.~1. 337~p.

\bibitem{2-pav}
\Aue{Faloutsos, C., P.~Faloutsos, and M.~Faloutsos.}
 1999. On power-law relationships of the Internet topology. 
 \textit{Comput. Commun. Rev.} 29:251--262.

\bibitem{3-pav}
\Aue{Bollobas, B.\,A.} 1980. A~probabilistic proof of an asymptotic formula for the number
of labelled regular graphs. \textit{Eur. J.~Combin.} 1:311--316.

\bibitem{4-pav}
\Aue{Reittu, H., and I.~Norros.} 2004. On the power-law random graph model of massive data
networks. \textit{Perform. Evaluation} 55:3--23.

\bibitem{5-pav}
\Aue{Durrett, R.} 2007. \textit{Random graph dynamics.} Cambridge:
Cambridge University Press. 212~p.

\bibitem{6-pav}
\Aue{Pavlov, Yu.\,L., and I.\,A.~Cheplyukova.} 2008. 
Random graphs of Internet type and the generalised allocation scheme.
\textit{Discrete Mathematics Applications} 18(5):447--463.

\bibitem{7-pav}
\Aue{Pavlov, Yu.\,L., and E.\,V.~Khvorostyanskaya.} 
2016. On the limit distributions of the degrees of vertices in configuration graphs with 
a~bounded number of edges. 
\textit{Sb. Math.} 207(3):400--417.

\bibitem{8-pav}
\Aue{Pavlov, Yu.\,L.} 2018. 
Conditional configuration graphs with discrete power-law distribution of vertex degrees. 
\textit{Sb. Math.} 209(2):258--275.

\bibitem{9-pav}
\Aue{Hofstad, R.} 2020. Random graphs and complex Networks. Vol.~2. 
\textit{Notes RGCNII} Available at: 
{\sf https:// www.win.tue.nl/$\sim$rhofstad/NotesRGCNII.pdf} (accessed January~11, 2021).

\bibitem{10-pav}
\Aue{Pavlov, Yu.\,L.} 2019. O~svyaznosti konfiguratsionnykh grafov 
[On connectivity of configuration graphs]. \textit{Discrete Mathematics Applications} 31(2):115--123.

\bibitem{11-pav}
\Aue{Ibragimov, I.\,A., and Yu.\,V.~Linnik.} 1965.  
\textit{Nezavisimye i~statsionarno svyazannye velichiny} 
[Independent and stationary sequences of random variables]. Moscow: Nauka. 524~p.

\bibitem{12-pav}
\Aue{Bigham, N.\,H., C.\,M.~Goldie, and J.\,L.~Teugels.}
 1987. \textit{Regular variations. Encyclopedia of mathematics and its applications.} 
 Cambridge: Cambridge University Press.  Vol.~27. 513~p.



\end{thebibliography}

 }
 }

\end{multicols}

\vspace*{-3pt}

  \hfill{\small\textit{Received April~15, 2020}}


%\pagebreak

%\vspace*{-8pt}


\Contrl

\noindent
\textbf{Pavlov Yuri L.} (b.\ 1949)~--- 
Doctor of Science in physics and mathematics, principal scientist, Institute 
of Applied Mathematical Research of the Karelian Research Centre 
of the Russian Academy of Sciences, 11~Pushkinskaya Str., Petrozavodsk 185910, Russian Federation; 
\mbox{pavlov@krc.karelia.ru}


\label{end\stat}

\renewcommand{\bibname}{\protect\rm Литература} 
            %3
\def\stat{Kovalev}

\def\tit{МЕТОДЫ ТЕОРИИ КАТЕГОРИЙ\\ В~ЦИФРОВОМ 
ПРОЕКТИРОВАНИИ\\ ГЕТЕРОГЕННЫХ КИБЕРФИЗИЧЕСКИХ СИСТЕМ}

\def\titkol{Методы теории категорий в~цифровом проектировании 
гетерогенных киберфизических систем}

\def\aut{С.\,П.~Ковалёв$^1$}

\def\autkol{С.\,П.~Ковалёв}

\titel{\tit}{\aut}{\autkol}{\titkol}

\index{Ковалёв С.\,П.}
\index{Kovalyov S.\,P.}

%{\renewcommand{\thefootnote}{\fnsymbol{footnote}} \footnotetext[1]
%{Работа выполнена при частичной поддержке РФФИ (проект 19-07-00187-A).}}

\renewcommand{\thefootnote}{\arabic{footnote}}
\footnotetext[1]{Институт проблем управления им.\ В.\,А.~Трапезникова 
Российской академии наук, \mbox{kovalyov@sibnet.ru}}

%\vspace*{-12pt}

\Abst{Развивается предложенный ранее математический аппарат на базе теории 
категорий, предназначенный для формального описания и~строгого 
исследования процедур инженерной деятельности на базе математического 
и~компьютерного моделирования. При помощи аппарата описаны 
и~исследованы высокоавтоматизированные процедуры проектирования 
гетерогенных киберфизических систем на основе цифровых двойников, 
востребованные грядущей четвертой промышленной революцией. Для этого 
впервые введена конструкция категории мультизапятой, объектами которой 
служат архитектурные модели некоторой гетерогенной киберфизической 
системы с~заданной схемой структурной иерархии, представленные 
с~некоторой фиксированной точки зрения, а~морфизмы отвечают действиям по 
подбору составных частей (СЧ) для сборки системы из них. Рассмотрено 
применение категории мультизапятой в~решении прямых и~обратных задач 
проектирования отдельных систем и~состоящих из них так называемых систем 
систем (СС).}

\KW{киберфизическая система; цифровой двойник; порождающее 
проектирование; система систем; теория категорий; категория мультизапятой}

\DOI{10.14357/19922264210104}

\vspace*{12pt}

\vskip 10pt plus 9pt minus 6pt

\thispagestyle{headings}

\begin{multicols}{2}

\label{st\stat}

\section{Введение}

Концепция киберфизической системы (cyber-physical system) состоит 
в~развитии традиционных\linebreak автоматизированных систем в~направлении 
максимально тесной интеграции мира физических объектов с~виртуальным 
миром управления в~целях повышения качества и~оперативности управления 
\mbox{объектами}. Такая сис\-те\-ма вклю\-ча\-ет разнородные физические компоненты, 
оснащенные большим чис\-лом цифровых датчиков и~исполнительных\linebreak 
механизмов, позволяющих скомпоновать и~поддерживать в~системе цифровой 
двойник (digital twin)~--- виртуальную модель, воспроизводящую и~за\-да\-ющую 
состояние и~поведение оригинала в~реальном времени~[1]. Структура 
и~начальное информационное наполнение двойника формируются в~цикле 
высокоавтоматизированного цифрового проектирования системы. 

Примерами 
киберфизических систем служат <<умные>> здания, города, энергетические 
системы, сетецентрические воинские формирования и~т.\,д.

В рамках парадигмы грядущей четвертой промышленной революции 
(Industrie~4.0) степень автоматизации проектирования и~производства 
повышается вплоть до полной замены человека \mbox{искусственным} интеллектом 
в~цикле по\-рож\-да\-юще\-го проектирования (generative design)~\cite{2-kov}. 
Однако традиционные информационные технологии, в~том числе на основе 
машинного обучения, не способны обеспечить искомый уровень автоматизации 
по крайней мере в~двух важных аспектах: 
\begin{enumerate}[(1)]
\item  интеграция 
разноплановых видов и~языков моделирования; 
\item  минимизация 
потребности в~вычислительных ресурсах для решения оптимизационных задач 
проектирования.
\end{enumerate}

 Актуальна разработка новых подходов и~методов цифрового 
проектирования, в~том числе основанных на перспективном математическом 
аппарате.

В настоящей работе, следуя современным тенденциям~[3--5], в~качестве 
основы используется тео\-рия категорий. Модели компонентов и~систем 
рассматриваются как объекты подходящих категорий, морфизмы в~которых 
описывают действия, связанные со сборкой сложных изделий. Строятся 
и~исследуются тео\-ре\-ти\-ко-ка\-те\-гор\-ные конструкции, описывающие 
методы цифрового проектирования киберфизических систем на абстрактном 
концептуальном уровне и~открывающие новые пути к~автоматизации. 
В~частности, впервые введена конструкция категории мультизапятой.

\begin{figure*}[b] %fig1
\vspace*{1pt}
 \begin{center}
 \mbox{%
 \epsfxsize=118mm  
 \epsfbox{kov-1.eps}
 }
  \end{center}
\vspace*{-9pt}
\Caption{Структурная схема <<умного>> здания~\cite{9-kov}}
\end{figure*}

\section{Принципы цифрового проектирования гетерогенных 
киберфизических систем}

Задача проектирования состоит в~формировании архитектуры системы и~ее 
описании в~различных моделях. Согласно основополагающему %\linebreak 
стандарту 
ISO/IEC/IEEE 42010:2011~\cite{6-kov}, описание \mbox{архитектуры} включает 
представления, отра\-жа\-ющие точки зрения различных заинтересованных 
сторон: пользователей, производителей, {проектировщиков}, эксплуатирующего 
персонала и~т.\,д. В~{результате} <<обезличивания>> (устранения субъективных 
факторов заинтересованных сторон) и~<<гранулирования>> (расщепления 
в~целях уменьшения пересечений) выделяются типовые точки зрения, такие 
как пространственное расположение %\linebreak 
(гео\-мет\-ри\-че\-ская форма), поведение, 
технология производства, надежность и~т.\,п. Со\-от\-вет\-ст\-ву\-ющие модели 
в~совокупности образуют элек\-т\-рон\-но-циф\-ро\-вой макет  
системы~\cite{7-kov}~--- основу циф\-ро\-во\-го двойника.
{\looseness=-1

}

По структуре архитектурная модель системы представляет собой 
ориентированный граф, узлы которого помечены моделями составляющих 
сис\-тем\-ных единиц, а~ребра~--- описаниями действий по иерархической сборке 
СЧ от деталей до системы в~целом. Модели задаются 
в~различных формах, таких как геометрические фигуры и~тела, системы 
дифференциальных уравнений, множества с~операциями и~отношениями, базы 
данных и~т.\,д. Все модели некоторого заданного вида и~описания всех 
действий предоставляются проектировщику в~виде (виртуального) каталога, из 
которого он выбирает строительные блоки для архитектуры.

Построение архитектуры усложняется, когда сис\-те\-ма включает множество 
существенно разнородных СЧ. В~этом случае для каж\-дой СЧ имеется свой 
самостоятельный каталог моделей и~априори не ясно, в~каких терминах 
описывать действия по сборке сис\-тем из таких СЧ. Можно зафиксировать лишь 
общую схему структурной иерархии СЧ, обуслов\-лен\-ную их природой. Такая 
<<схема деления>>~\cite{8-kov} формируется в~самом начале жизненного 
цикла изделия, когда определяется его концептуальный облик, и~считается 
неизменной в~процессе проектирования. Киберфизические системы могут 
включать в~себя такие СЧ (подсистемы), как механическая (несущая), 
гидравлическая, электрическая, тепловая, защитная,  
конт\-роль\-но-из\-ме\-ри\-тель\-ная,  
ин\-фор\-ма\-ци\-он\-но-ком\-му\-ни\-ка\-ци\-он\-ная (которая в~свою очередь 
разделяется на программное и~аппаратное обеспечение). Ярким примером 
служит <<умное>> здание, крупноблочная структура которого упрощенно 
показана на рис.~1~\cite{9-kov}.




Чтобы превратить схему деления в~полноценную архитектурную модель, 
необходимо отобразить ее в~рамках некоторой точки зрения на изделие. Такое 
отображение возможно, поскольку каж\-дая точка зрения определяет как 
некоторый аспект каждой СЧ, так и~действия, связанные с~составлением 
в~этом аспекте сложных единиц из более прос\-тых. Например, с~точки зрения 
пространственного расположения сборка сводится к~взаимному расположению 
и~полной либо частичной относительной фиксации геометрических форм СЧ. 
С~точки\linebreak\vspace*{-9pt}

%\pagebreak

{ \begin{center}  %fig2
 \vspace*{-1pt}
   \mbox{%
\epsfxsize=78.413mm 
\epsfbox{kov-2.eps}
}

\end{center}

\noindent
{{\figurename~2}\ \ \small{
Структурно целостный переход между вариантами ЭСИ
}}}

\vspace*{18pt}

\noindent
 зрения поведения СЧ описываются сценариями~--- фрагментами 
предполагаемой истории их поведения, представленного потоком дискретных 
событий, так что описания действий по сборке \mbox{сценариев} поведения сложных 
систем отображают вклад сценариев поведения СЧ.





Представление архитектуры системы с~некоторой точки зрения, заданное 
в~цифровом виде, называется электронной структурой изделия  
(ЭСИ)~\cite{10-kov}. На разных стадиях жизненного цикла системы 
формируются и~прорабатываются разные виды ЭСИ, соответствующие точкам 
зрения различных заинтересованных сторон: конструктивная,  
про\-из\-вод\-ст\-вен\-но-тех\-но\-ло\-ги\-че\-ская, эксплуатационная и~т.\,д. 
Компьютерные инструменты проектирования помогают проектировщику 
подбирать СЧ и~действия в~ЭСИ так, чтобы результирующая система 
удовле\-тво\-ря\-ла требованиям заинтересованных сторон. Классические 
сис\-те\-мы автоматического проектирования 
способны автоматически решать прямые задачи проектирования~--- строить 
виртуальную модель изделия (вершину иерархии ЭСИ), исходя из вручную 
выбранных проектировщиком СЧ и~действий по сборке, рассчитывать ее 
характеристики и~сопоставлять с~требованиями. Однако для технологий типа 
порождающего проектирования этого недостаточно: компьютер должен 
самостоятельно автоматически подбирать из каталога СЧ и~способы их сборки 
так, чтобы в~наибольшей степени удовлетворить требованиям.










Ключевую роль в~порождающем проектировании играет удобное для 
компьютерной навигации пространство проектирования (design space), 
состоящее из всех допустимых вариантов ЭСИ. В~рамках некоторой 
фиксированной точки зрения удобство перехода между вариантами 
обеспечивается, в~частности, когда каждая СЧ исходного варианта переходит 
в~заменяющую ее СЧ целевого варианта посредством некоторого действия. 
Более того, <<хороший>> переход является структурно целостным в~том 
смыс\-ле, что образующие его действия однозначно комбинируются 
с~действиями, со\-став\-ля\-ющи\-ми ЭСИ с~выбранной точки зрения~\cite{11-kov}, 
как показано на рис.~2. Пространства проектирования гетерогенных сис\-тем 
с~такими переходами будут далее описаны на строгом алгебраическом языке, 
что открывает перспективу для применения эффективных алгоритмов 
поисковой оптимизации с~привлечением средств компьютерной ал\-гебры.



Сложность проектирования киберфизических систем усугубляется тем 
обстоятельством, что такие системы редко проектируются по отдельности. 
Напротив, на практике обычно возникает потребность в~целой системе таких 
систем (System of Systems, SoS), в~качестве СЧ которой выступают 
полноценные и~вполне самостоятельные киберфизические системы. В~рамках 
СС они обмениваются информацией и~вступают 
в~коллаборацию в~целях оптимизации управления многошаговыми 
многосубъектными процессами с~помощью средств типа платформы 
<<умных>> сервисов~\cite{1-kov}. Часто встречается такой класс СС, как 
группа однотипных киберфизических объектов, организованная для 
достижения общей цели, например ударная группировка беспилотных 
летательных аппаратов. Синтез таких групп относится к~прямым задачам 
проектирования СС. А~обратные задачи требуют проведения иерархически 
организованных процедур поисковой оптимизации в~пространствах 
проектирования СЧ (имеющих, вообще говоря, разные схемы деления). 
Примером такой обратной задачи служит порождающее проектирование 
<<умной>> городской агломерации, состоящей из зданий и~инфраструктурных 
систем различного назначения. В~качестве подхода к~решению таких задач 
далее рассмотрим приведение к~плоскому поиску путем <<отрисовки>> 
(подстановки в~развернутом виде) ЭСИ СЧ в~ЭСИ~СС.

\section{Категория мультизапятой}

Будем пользоваться теоретико-категорными конструкциями и~обозначениями, 
введенными в~работах~\cite{5-kov, 13-kov}. Каталоги моделей системных 
единиц описываются подходящими категориями. В~свою очередь, все 
возможные модели любых системных единиц в~представлении, выражающем 
некоторую фиксированную точку зрения на систему, также описываются 
категорией, которую будем обозначать через~$C$. Например, при 
пред\-став\-ле\-нии\linebreak сис\-тем с~точки зрения пространственного расположения 
в~качестве~$C$ выступает категория твердотельных геометрических моделей 
\textbf{MBS}, содержащаяся в~категории множеств \textbf{Set}, а~при 
\mbox{представлении} поведения~--- категория дис\-крет\-но-со\-бы\-тий\-ных имитационных 
моделей \textbf{Pomset}, которая является конкретной категорией над 
\textbf{Set}.





Представление архитектуры системы с~точки зрения~$C$ ($C$-ЭСИ) сводится 
к~$C$-диаграмме, в~вершинах которой находятся представления СЧ, 
а~стрелки представляют действия по сборке. Пусть~$I$~--- форма (схема) этой 
диаграммы. На\-пом\-ним, что~$I$~--- это малая категория (и~через \textbf{Cat} 
обозначается полная подкатегория в~<<категории всех категорий>> 
\textbf{CAT}, состоящая из всех малых категорий). Каталоги моделей СЧ 
задаются семейством категорий $D_i$, $i \in \vert I\vert$ (напомним, что через  
$\vert I\vert$ обозначается множество вершин схемы~$I$ и~его можно 
рассматривать как дискретную подкатегорию в~$I$). Для каждой СЧ имеется 
правило пред\-став\-ле\-ния с~точки зрения~$C$, которое по соображениям 
корректности задается функтором вида $F_i \: : \: D_i\hm \to C$ для $i$-й СЧ (будем 
обозначать через~$F$ любое такое семейство функторов, индексированное 
множеством вершин схемы). Таким образом, архитектурная модель некоторой 
конкретной сис\-те\-мы получается, если выбрать по одному объекту $A_i \hm\in D_i$, 
$i \in \vert I\vert$, и~некоторую диаграмму~$\Delta \: : \: I\hm\to C$, 
удовлетворяющую условию $\Delta (i) \hm= F_i(A_i)$, $i \hm\in \vert I\vert$. Заметим, 
что точно такой же формальный вид имеют модели, выражающие комплексное 
представление архитектуры с~нескольких точек зрения одновременно: если 
отдельные точки зрения образуют множество~$Q$ и~им отвечают 
категории~$C_q$, $q\hm\in Q$, то комплексное представление описывается 
произведением категорий $\prod_{q \in Q}C_q$, которое и~выступает 
в~качестве~$C$.
{\looseness=1

}

Процедуры подбора и~замены СЧ в~ходе проек\-ти\-ро\-ва\-ния формально 
описываются преобразованиями архитектурных моделей, не изменяющими\linebreak ни 
схему деления, ни правила представления СЧ. Такими преобразованиями 
очевидным образом служат естественные преобразования диаграмм, 
индуцированные действиями из каталогов СЧ, а~именно: преобразованием 
модели (($A_i$, $i \hm\in \vert I\vert$), $\Delta$) в~модель (($A_i^\prime$,  
$i \hm\in \vert I\vert$), $\Delta^\prime$) является любое семейство морфизмов~$f_i \: : \: A_i \hm\to A^\prime_i$, 
$i \hm\in \vert I\vert$ (где каждый морфизм~$f_i$ принадлежит 
категории~$D_i$) такое, что для любых вершин схемы~$i$, $k \hm\in\vert I\vert$ 
и~стрелки~$h \: : \: i\hm \to k$ выполняется условие 
$$
F_k(f_k) \circ \Delta (h) = \Delta^\prime (h) \circ F_i(f_i)\,.
$$
Именно это условие выражает на языке теории категорий структурную 
целостность перехода между моделями, соответствующего замене  
СЧ~\cite{11-kov}.

Легко проверить, что для любых фиксированных $C$, $I$ и~$F$ совокупность 
всех архитектурных моделей и~всех их преобразований образует категорию. 
В~теории категорий давно известен один частный случай этой конструкции, 
где в~качестве~$I$ выбрана схема вида $0\hm\to1$. Категория архитектурных 
моделей для этого случая встречается во многих задачах, называется 
категорией запятой (comma category)~\cite[\S\,II.6]{13-kov} и~обозначается 
через $F_0 \downarrow F_1$. Поэтому будем называть произвольную категорию 
архитектурных моделей вышеописанного вида \textit{категорией 
мультизапятой} (multicomma) и~обозначать через $\downdownarrows_I F$. 
Пара $\langle I, F\rangle $ называется формой (shape) категории мультизапятой, 
а~категория~$C$~--- пред\-став\-ле\-ни\-ем (representation) категории мультизапятой.
{\looseness=-1

}

Примечательно, что категорию мультизапятой можно получить при помощи 
универсальных конструкций в~<<категории всех категорий>> \textbf{CAT}, 
а~именно: произведения, декартова квад\-ра\-та и~экс\-по\-ненты.

\smallskip

\noindent
\textbf{Теорема~1.}~\textit{Категория муль\-ти\-за\-пя\-той $\downdownarrows_I F$ 
изо\-морф\-на вершине (объекту, находящемуся в~левом верх\-нем углу) 
сле\-ду\-юще\-го декартова квад\-ра\-та в}~\textbf{CAT}:

{ \begin{center} %fig3
\vspace*{1pt}

\mbox{%
\epsfxsize=41.337mm 
\epsfbox{kov-3.eps}
}
\end{center}
\vspace*{3pt}

}

\noindent
Д\,о\,к\,а\,з\,а\,т\,е\,л\,ь\,с\,т\,в\,о\,.\ \  Проверяется непосредственно по правилам 
вы\-чис\-ле\-ния пределов и~экспонент в~\textbf{CAT}.~$\square$

\smallskip

Штриховые стрелки декартова квадрата из тео\-ре\-мы~1 задают два 
канонических <<забывающих>> функтора, определенных на любой категории 
мультизапятой. Первый функтор, заданный левой вертикальной стрелкой, 
извлекает из архитектурной модели набор всех СЧ. Он унивалентен (faithful), 
так что можно трактовать его как функтор выделения <<носителя>> модели, по 
аналогии с~функторами, выделяющими носитель у~алгебраических сис\-тем, 
топологических пространств и~т.\,п. Второй функтор, заданный верхней 
горизонтальной стрелкой, извлекает из модели пред\-став\-ле\-ние структуры 
сис\-те\-мы с~точ\-ки зрения~$C$ и~не имеет прямого аналога\linebreak\vspace*{-12pt}

\pagebreak

\noindent
 в~универсальной 
алгебре. Будем называть его \textit{функтором структуры} и~обозначать через~$\nabla_I^F$, 
так что 
$$
\nabla_I^F :\ \downdownarrows_I F \to C^I : ((A_i, i \in \vert I\vert), \Delta : I \to C) 
\mapsto \Delta \,.
$$


В свою очередь, экспонента $C^I$ допускает каноническое вложение 
в~категорию диаграмм \textbf{D}$C$, представляющую каталог всех 
формально возможных структур систем~\cite{12-kov}, с~точки зрения~$C$. 
При\linebreak помощи этого вложения и~функтора структуры можно представить многие 
процедуры проектирования в~виде функторов, определенных на категории 
мультизапятой. Рас\-смот\-рим в~качестве примера построение копредела 
диаграммы, формально опи\-сы\-ва\-ющее решение классической прямой задачи 
цифрового проектирования~--- формирование модели цельной системы 
известной структуры. Предположим, что любая $C$-диа\-грам\-ма со схемой~$I$ 
имеет копредел. Тогда имеется вложение~$lc$~: $C^I \hookrightarrow LC$, где 
через $LC$ обозначена полная подкатегория в~\textbf{D}$C$, состоящая из 
всех диаграмм, име\-ющих копредел. Вычисление копредела задается функтором 
colim~: $LC \hm\to C$. Таким образом, получается функтор пред\-став\-ле\-ния 
процедуры сборки сис\-тем формы~$I$ с~точ\-ки зрения~$C$:
$$
\mbox{colim}  \circ lc \circ \nabla_I^F :\ \downdownarrows_I F \to C\,.
$$


А для решения обратных задач проектирования киберфизических сис\-тем 
категория $\downdownarrows_I F$ служит естественным <<строительным 
материалом>> для пространства проектирования, поскольку целевые функции, 
определяющие степень соответствия вариантов архитектуры сис\-те\-мы 
требованиям, можно задавать функторами на таком пространстве. 
Действительно, областью значения целевой функции\linebreak
 всегда является линейно 
упорядоченное множество, а~его, как хорошо известно, можно представить  
категорией~\cite[\S\,I.2]{13-kov}: объектами такой\linebreak
 категории служат все 
элементы множества, а~морфизмами~--- все пары ($x, y$) такие, что $x \leq y$ 
(так что между любыми двумя объектами имеется не более одного морфизма). 
Интересна ситуация, когда целевая функция выступает функцией объектов 
функтора, действующего в~такую категорию из нетривиальной (достаточно 
богатой морфизмами) подкатегории в~$\downdownarrows_I F$ или 
в~двойственной категории $(\downdownarrows_I F)^{\mathrm{op}}$. В~этом 
случае можно применить оптимизационные алгоритмы типа градиентного 
спуска, выполняющие навигацию вдоль морфизмов этой подкатегории, 
с~расчетом пути методами компьютерной алгебры.

Обратимся к~задачам проектирования СС. Прямые задачи, пред\-по\-ла\-га\-ющие 
<<подъем>> конструкций в~моделях на уровень структур, могут быть решены 
в~категории мультизапятой <<почленно>>.\linebreak Например, рассмотрим~$I$ как 
схему, вершины которой представляют членов некоторой группы однотипных 
киберфизических объектов, архитектура которых характеризуется формой  
$\langle K, G_k \: : \: D_k \hm\to C$, $k \hm\in \vert K\vert\rangle$, а~стрелки представляют 
внутригрупповые коллаборационные связи. Предположим, что во всех 
категориях $D_k$, $k \hm\in \vert K\vert$, все диаграммы формы~$I$ имеют 
копределы и~все функторы $G_k$ сохраняют их. Тогда произвольная  
($\downdownarrows_K G$)-диа\-грам\-ма формы~$I$, пред\-став\-ля\-ющая 
возможную структуру группы, также имеет копредел: компоненты его носителя 
строятся по от\-дель\-ности на объектах и~морфизмах каждой СЧ $D_k$, затем 
переносятся в~$C$ посредством функторов $G_k$, после чего дополняются 
универсальными стрелками копределов до $C$-диа\-грам\-мы формы~$K$, 
образуя ($\downdownarrows_K G$)-объект, представляющий про\-ек\-ти\-ру\-емую 
группу как единый цельный объект.

Конструкция мультизапятой хорошо подходит и~для решения обратных задач 
проектирования СС, поскольку ведет себя естественно относительно 
процедуры отрисовки диаграмм, со\-сто\-ящих из диаграмм. Действительно, 
формально структура СС описывается диаграммой, в~вершинах которой 
находятся сис\-те\-мы, структуры которых в~свою очередь описываются 
подходящими диаграммами (вообще говоря, имеющими различные схемы). Тем 
самым схема СС задается диаграммой вида~$\Xi \: : \: I \hm\to \mathbf{Cat}$. 
Отрисовка такой диаграммы~--- это преобразование в~малую категорию, 
порожденную заменой каждой\linebreak вершины $i \hm\in \vert I\vert$ схемой~$\Xi (i)$, 
а~каждой стрелки $h \: : \: i \hm\to l$~--- совокупностью стрелок, по одной для 
каждой вершины~$s$ схемы~$\Xi (i)$, на\-прав\-лен\-ной из~$s$\linebreak в~вершину~$\Xi (h)(s)$ 
схемы~$\Xi (l)$, с~наложением подходящих условий 
естественности. Вместе с~\textbf{Cat}-диа-\linebreak грам\-ма\-ми можно отрисовывать 
и~их морфизмы, так что имеется функтор отрисовки~\textbf{K}$_{\text{\bfseries\textit{1}}} \: : \: \mathbf{D(Cat)} 
\hm\to \mathbf{Cat}$ (это част\-ный случай общей конструкции 
отрисовки~--- умножения в~монаде диаграмм~\textbf{D}~\cite{12-kov}).

\smallskip

\noindent
\textbf{Теорема~2.}\ \textit{Пусть заданы произвольные схема~$I$, 
диаграмма~$\Xi \: : \: I \hm\to \mathbf{Cat}$ и~семейство функ\-то\-ров~$G^{(i)}_k \: : \:  
D^{(i)}_k \hm\to C$, $k \hm\in \vert \Xi (i)\vert$, $i \hm\in \vert I\vert$. Категория 
муль\-ти\-за\-пя\-той $\downdownarrows_{\mathbf{K}_{\text{\bfseries\textit{1}}}\Xi} G$ изо\-морф\-на вершине 
сле\-ду\-юще\-го декартова квад\-ра\-та в}~\textbf{CAT}:

{ \begin{center} %fig4
\vspace*{-1pt}

\mbox{%
\epsfxsize=63.716mm 
\epsfbox{kov-4.eps}
}
\vspace*{3pt}

\end{center}
}

\noindent
Д\,о\,к\,а\,з\,а\,т\,е\,л\,ь\,с\,т\,в\,о\,.\ \ Сформируем <<почленное>> произведение 
декартовых квадратов, за\-да\-ющих согласно тео\-ре\-ме~1 все категории 
мультизапятой $\downdownarrows_{\Xi (i)} G^{(i)}$, $i \in \vert I\vert$. Если 
заменить в~нем правое вертикальное реб\-ро экспонентой суммы, то можно 
пристроить его снизу к~декартову квад\-ра\-ту, указанному в~условии тео\-ре\-мы:

{ \begin{center} %fig5
\vspace*{3pt}
\mbox{%
\epsfxsize=69.179mm 
 \epsfbox{kov-5.eps}
}
\end{center}
\vspace*{3pt}

}

\noindent
Здесь наружный прямоугольник является  
декартовым~\cite[предложение 11.10]{14-kov}. А~поскольку $\vert 
\mathbf{K}_{\text{\bfseries\textit{1}}} \Xi\vert \cong \coprod_{i \in \vert I\vert} \vert \Xi (i)\vert$ для любой 
\textbf{Cat}-диа\-грам\-мы~$\Xi$ со схемой~$I$ (отрисовка не добавляет и~не 
удаляет объекты), по теореме~1 объект в~левом верхнем углу действительно 
изоморфен категории мультизапятой $\downdownarrows_{\mathbf{K}_{\text{\bfseries\textit{1}}}\Xi} G$.~$\square$

\section{Заключение}

Теория категорий обладает большим потенциалом применения в~технологиях 
четвертой промышленной революции, в~том числе в~цифровом 
проектировании гетерогенных киберфизических\linebreak \mbox{систем}. В~настоящее время 
возможности применения предложенных тео\-ре\-ти\-ко-ка\-те\-гор\-ных 
методов исследуются на макете программного инструмента разработки 
цифровых двойников энергетических \mbox{систем}~[15]. В~ходе развития 
инструмента до промышленного уровня го\-тов\-ности возникнет много новых 
задач для дальнейших исследований.

%\vspace*{-8pt}

{\small\frenchspacing
{%\baselineskip=10.8pt
%\addcontentsline{toc}{section}{References}
\begin{thebibliography}{99}

%\vspace*{-2pt}

\bibitem{1-kov} 
\Au{Tao~F., Qi~Q., Wang~L., Nee~A.\,Y.\,C.} Digital twins and cyber-physical 
systems toward smart manufacturing and Industry~4.0: Correlation and comparison~// 
Engineering, 2019. Vol.~5. P.~653--661.
\bibitem{2-kov} 
\Au{Kowalski~J.} CAD is a~lie: Generative design to the rescue.~--- San Rafael, CA, 
USA: Autodesk, 2016. {\sf https:// www.autodesk.com/redshift/generative-design}.
\bibitem{3-kov} 
\Au{Baez~J.\,C., Erbele~J.} Categories in control~// Theor. Appl. 
Categ., 2015. Vol.~30. Iss.~24. P.~836--881.
\bibitem{4-kov} 
\Au{Wisnesky~R., Breiner~S., Jones~A., Spivak~D.\,I., Subrahmanian~E.} Using 
category theory to facilitate multiple manufacturing service database integration~// 
J.~Comput. Inf. Sci. Eng., 2017. Vol.~17. Iss.~2. Art. 
ID:~021011.
\bibitem{5-kov} 
\Au{Ковалёв~С.\,П.} Методы теории категорий  
в~мо\-дель\-но-ори\-ен\-ти\-ро\-ван\-ной системной инженерии~// Информатика 
и~её применения, 2017. Т.~11. Вып.~3. С.~42--50.
\bibitem{6-kov} 
ГОСТ Р 57100-2016/ISO/IEC/IEEE 42010:2011. Сис\-тем\-ная и~программная 
инженерия. Описание архитектуры.~--- М.: Стандартинформ, 2016. 32~с.
\bibitem{7-kov} 
\Au{Gherghina~G., Tutunea~D., Popa~D.} About digital mock-up for mechanical 
products~// J.~Industrial Design Engineering Graphics, 2015. Vol.~10. No.\,2. 
P.~19--22.
\bibitem{8-kov} 
\Au{Рафальский~В.\,В., Рафальская~Л.\,Г., Старостина~А.\,В.} 
Информационная модель схемы деления~// Автоматизация процессов 
управления, 2009. №\,3. С.~22--28.
\bibitem{9-kov} 
What is a~smart building and how can it benefit you?~--- Milford, MA, USA: 
Comark, 2016. {\sf https://comarkcorp. com/smart-building-can-benefit.}
\bibitem{10-kov} 
ГОСТ 2.053-2013. Единая система конструкторской документации. 
Электронная структура изделия. Общие положения.~--- М.: Стандартинформ, 
2014. 10~с.
\bibitem{11-kov} 
\Au{Ковалёв~С.\,П.} Алгебраическое моделирование жизненного цикла 
круп\-но\-мас\-штаб\-ных гетерогенных сис\-тем в~аспектах~// Управление развитием 
круп\-но\-мас\-штаб\-ных сис\-тем: Мат-лы X~Междунар. 
конф.~--- М.: ИПУ РАН, 2017.  Т.~II. С.~266--268.

\bibitem{13-kov} 
\Au{Маклейн~С.} Категории для работающего математика~/ Пер. с~англ.~--- 
М.: Физматлит, 2004. 352~с. (\Au{Mac Lane~S.} Categories for the working 
mathematician.~--- New York, NY, USA: Springer, 1978. 317~p.)
\bibitem{12-kov} 
\Au{Ковалёв~С.\,П.} Теория категорий как математическая прагматика  
мо\-дель\-но-ори\-ен\-ти\-ро\-ван\-ной сис\-тем\-ной инженерии~// Информатика 
и~её применения, 2018. Т.~12. Вып.~1. С.~95--104.

\bibitem{14-kov} 
\Au{Ad$\acute{\mbox{a}}$mek~J., Herrlich~H., Strecker~G.\,E.} Abstract and 
concrete categories.~--- New York, NY, USA: John Wiley, 1990. 507~p.
\bibitem{15-kov} 
\Au{Ковалёв~С.\,П.} Проектирование информационного обеспечения цифровых 
двойников энергетических систем~// Системы и~средства информатики, 2020. 
Т.~30. №\,1. С.~66--81.
\end{thebibliography}

}
}

\end{multicols}

\vspace*{-3pt}

\hfill{\small\textit{Поступила в~редакцию 12.10.2019}}

%\vspace*{8pt}

%\pagebreak

\newpage

\vspace*{-30pt}

%\hrule

%\vspace*{2pt}

%\hrule

%\vspace*{-2pt}


\def\tit{METHODS OF~THE~CATEGORY THEORY IN~DIGITAL DESIGN 
OF~HETEROGENEOUS CYBER-PHYSICAL SYSTEMS\\[-4pt]}

\def\titkol{Methods of~the~category theory in~digital design of~heterogeneous 
cyber-physical systems}

\def\aut{S.\,P.~Kovalyov\\[-4pt]}

\def\autkol{S.\,P.~Kovalyov}

\titel{\tit}{\aut}{\autkol}{\titkol}

\vspace*{-20pt}

\noindent
V.\,A.~Trapeznikov Institute of Control Sciences, Russian Academy of Sciences, 
65~Profsoyuznaya Str., Moscow 117997, Russian Federation


\def\leftfootline{\small{\textbf{\thepage}
\hfill INFORMATIKA I EE PRIMENENIYA~--- INFORMATICS AND
APPLICATIONS\ \ \ 2021\ \ \ volume~15\ \ \ issue\ 1}
}%
\def\rightfootline{\small{INFORMATIKA I EE PRIMENENIYA~---
INFORMATICS AND APPLICATIONS\ \ \ 2021\ \ \ volume~15\ \ \ issue\ 1
\hfill \textbf{\thepage}}}

\vspace*{2pt} 

\Abste{A~mathematical device built upon the category theory is developed which 
was previously proposed to formally describe and rigorously explore engineering 
procedures based on mathematical and computer modeling. With the help of the 
device, highly automated procedures for designing heterogeneous cyber-physical 
systems on top of digital twins, demanded by the upcoming fourth industrial 
revolution, are described and explored. For this purpose, the novel construction of the 
multicomma category is introduced, whose objects are the architectural models of 
a~heterogeneous cyber-physical system with a~certain fixed structural hierarchy 
scheme represented from a~certain architecture viewpoint, and morphisms describe 
actions associated with selection of constituents for assembling a~system from them. 
The application of the multicomma category in solving direct and inverse problems 
of designing individual systems and complex systems of systems is considered.}

%\vspace*{2pt}

\KWE{cyber-physical system; digital twin; generative design; system of systems; 
category theory; multicomma category}

\DOI{10.14357/19922264210104}

\vspace*{-6pt}

\begin{multicols}{2}

\renewcommand{\bibname}{\protect\rmfamily References}
%\renewcommand{\bibname}{\large\protect\rm References}

{\small\frenchspacing
{%\baselineskip=10.8pt
\addcontentsline{toc}{section}{References}
\begin{thebibliography}{99}
\bibitem{1-kov-1} 
\Aue{Tao,~F., Q.~Qi, L.~Wang, and A.\,Y.\,C.~Nee.} 2019. Digital twins and  
cyber--physical systems toward smart manufacturing and Industry~4.0: Correlation 
and comparison. \textit{Engineering} 5:653--661.
\bibitem{2-kov-1} 
CAD is a~lie: Generative design to the rescue. Available at: {\sf 
https://www.autodesk.com/redshift/generative-design/} (accessed December~9, 
2020).
\bibitem{3-kov-1} 
\Aue{Baez,~J.\,C., and J.~Erbele}. 2015. Categories in control. \textit{Theor.  
Appl. Categ.} 30(24):836--881.
\bibitem{4-kov-1} 
\Aue{Wisnesky,~R., S.~Breiner, A.~Jones, D.\,I.~Spivak, and E.~Subrahmanian.} 
2017. Using category theory to facilitate multiple manufacturing service database 
integration. \textit{J.~Comput. Inf. Sci. Eng.} 17(2):021011.
\bibitem{5-kov-1} 
\Aue{Kovalyov,~S.\,P.} 2017. Metody teorii kategoriy v model'no-orientirovannoy 
sistemnoy inzhenerii [Methods of category theory in model-based systems 
engineering]. \textit{Informatika i~ee Primeneniya~--- Inform. Appl.} 11(3):42--50.
\bibitem{6-kov-1} 
GOST R ISO/IEC/IEEE 42010:2011. 2016. Sistemnaya i~programmnaya inzheneriya. 
Opisanie arkhitektury [System and software engineering. Description of 
architecture]. Moscow: Standardinform Publs. 32~p.
\bibitem{7-kov-1} 
\Aue{Gherghina,~G., D.~Tutunea, and D.~Popa.} 2015. About digital mock-up for 
mechanical products. \textit{J.~Industrial Design Engineering Graphics} 
10(2):19--22.
\bibitem{8-kov-1} 
\Aue{Rafal'skiy,~V.\,V., L.\,G.~Rafal'skaya, and A.\,V.~Starostina}. 2009. 
Informatsionnaya model' skhemy deleniya [Information model of decomposition scheme]. 
\textit{Avtomatizatsiya protsessov upravleniya} [Automation of Control 
Processes] 3:22--28.
\bibitem{9-kov-1} 
What is a~smart building and how can it benefit you? 2016. Milford, MA: Comark.
Available at: {\sf 
https://comarkcorp. com/smart-building-can-benefit/} (accessed December~9, 2020).
\bibitem{10-kov-1} 
GOST 2.053-2013. 2014. Edinaya sistema konstruktorskoy dokumentatsii. 
Elektronnaya struktura izdeliya. Obshchie polozheniya [Unified system for design 
documentation. Electronic structure of the product. General Provisions]. Moscow: 
Standardinform Publs. 10~p.
\bibitem{11-kov-1} 
\Aue{Kovalyov,~S.\,P.} 2017. Algebraicheskoe modelirovanie zhiznennogo tsikla 
krupnomasshtabnykh geterogennykh sistem v~aspektakh [Aspectwise algebraic 
modeling of large-scale heterogeneous systems life cycle].
 \textit{10th Conference (International) ``Management of Large-Scale Systems 
Development'' Proceedings}. Moscow. 2:266--268.

\bibitem{13-kov-1} 
\Aue{Mac Lane, S.} 1978. \textit{Categories for the working mathematician}. New 
York, NY: Springer. 317~p.

\bibitem{12-kov-1} 
\Aue{Kovalyov, S.\,P.} 2018. Teoriya kategoriy kak ma\-te\-ma\-ti\-che\-skaya pragmatika 
mo\-del'\-no-ori\-yen\-ti\-ro\-van\-noy sis\-tem\-noy inzhenerii [Category theory as a~mathematical 
pragmatics of model-based systems engineering]. \textit{Informatika i~ee 
Primeneniya~--- Inform. Appl.} 12(1):95--104.

\bibitem{14-kov-1} 
\Aue{Ad$\acute{\mbox{a}}$mek,~J., H.~Herrlich, and G.\,E.~Strecker.} 1990. 
\textit{Abstract and concrete categories}. New York, NY: John Wiley. 507~p.
\bibitem{15-kov-1} 
\Aue{Kovalyov,~S.\,P.} 2020. Proektirovanie informatsionnogo obespecheniya 
tsifrovykh dvoynikov energeticheskikh sistem [Information architecture of the power 
system digital twin]. \textit{Sistemy i~Sredstva Informatiki~--- Systems and Means of 
Informatics} 30(1):66--81.
\end{thebibliography}

}
}

\end{multicols}

\vspace*{-8pt}

\hfill{\small\textit{Received October~22, 2019}}


%\pagebreak

\vspace*{-26pt}

\Contrl

\vspace*{-4pt}

\noindent
\textbf{Kovalyov Sergey P.} (b.\ 1972)~--- Doctor of Science in physics and 
mathematics, leading scientist, V\, A.~Trapeznikov Institute of Control Sciences, 
Russian Academy of Sciences, 65~Profsoyuznaya Str., Moscow 117997, Russian 
Federation; \mbox{kovalyov@sibnet.ru}
\label{end\stat}

\renewcommand{\bibname}{\protect\rm Литература}       %4 
\def\stat{chehovich}

\def\tit{МЕТОДЫ ОБНАРУЖЕНИЯ ПЕРЕВОДНЫХ ЗАИМСТВОВАНИЙ В~БОЛЬШИХ ТЕКСТОВЫХ 
КОЛЛЕКЦИЯХ$^*$}

\def\titkol{Методы обнаружения переводных заимствований в~больших текстовых 
коллекциях}

\def\aut{Р.\,В.~Кузнецова$^1$, О.\,Ю.~Бахтеев$^2$, Ю.\,В.~Чехович$^3$}

\def\autkol{Р.\,В.~Кузнецова, О.\,Ю.~Бахтеев, Ю.\,В.~Чехович}

\titel{\tit}{\aut}{\autkol}{\titkol}

\index{Кузнецова Р.\,В.}
\index{Бахтеев О.\,Ю.}
\index{Чехович Ю.\,В.}
\index{Kuznetsova R.\,V.}
\index{Bakhteev O.\,Yu.}
\index{Chekhovich Yu.\,V.}

{\renewcommand{\thefootnote}{\fnsymbol{footnote}} \footnotetext[1]
{Работа выполнена при поддержке РФФИ (проект 18-07-01441) 
и~Фонда содействия развитию малых форм предприятий в~на\-уч\-но-тех\-ни\-че\-ской сфере 
(проект~44116).}}

\renewcommand{\thefootnote}{\arabic{footnote}}
\footnotetext[1]{Московский физико-технический институт, 
\mbox{rita.kuznetsova@phystech.edu}}
\footnotetext[2]{Компания 
Антиплагиат; 
Московский фи\-зи\-ко-тех\-ни\-че\-ский институт, \mbox{bakhteev@ap-team.ru}}
\footnotetext[3]{Вычислительный центр им.\ А.\,А.~Дородницына Федерального 
исследовательского центра <<Информатика и~управ\-ле\-ние>> Российской академии наук, 
\mbox{chehovich@ap-team.ru}}


\vspace*{-8pt}


\Abst{Рассматривается задача обнаружения переводных заимствований. 
Для решения предлагается использовать моноязыковой подход~--- свести задачу 
обнаружения заимствований к~одному языку, используя машинный перевод. В~связи со 
спецификой рассматриваемой задачи предлагаемый алгоритм обнаружения должен быть 
устойчив к~неоднозначностям перевода. Предлагается декомпозировать задачу на 
несколько этапов.
Сначала отбираются до\-ку\-мен\-ты-кан\-ди\-да\-ты,  устойчивость к~неоднозначности перевода 
достигается за счет замены слов на метки кластеров, полученных с~по\-мощью 
дистрибутивной модели. Затем происходит сравнение найденных кандидатов 
и~рассматриваемого документа, для этого используется отображение текстовых 
фрагментов документов в~векторное пространство высокой размерности. 
Вычислительный эксперимент проводится для языковой пары 
<<рус\-ский--анг\-лий\-ский>> на двух выборках~--- синтетическом корпусе и~на статьях из 
журналов, входящих в~Российский индекс научного цитирования (РИНЦ).}


\KW{автоматическая обработка текстов; машинный перевод; 
глубокое обучение; переводные заимствования; обнаружение переводных 
заимствований; дистрибутивная семантика}

\DOI{10.14357/19922264210105}

%\vspace*{-2pt}


\vskip 10pt plus 9pt minus 6pt

\thispagestyle{headings}

\begin{multicols}{2}

\label{st\stat}

\section{Введение}

\vspace*{-2pt}

Проблема некорректных текстовых заимствований актуальна для сферы образования 
и~научных исследований~\cite{plag_cheh}. По материалам исследования~\cite{hist}, 
проведенного в~2013~г., более 1500~диссертаций по историческим наукам, 
защищенных в~России после 2000~г., содержат значительные заимствования из 
других диссертаций.

Для задачи обнаружения заимствований в~рамках одного языка высокую полноту 
поиска показывают промышленные инструменты~\cite{plag_cheh}, работа которых 
основана на представлении документов в~виде набора перекрывающих друг друга 
пословных $n$-грамм (шинглов)~\cite{shingles1}. Такой подход позволяет 
эффективно проводить поиск точных текстовых заимствований, но не позволяет 
обнаруживать заимствования с~большой долей перефразированного текста или со 
вставками  текста, переведенного с~другого языка.

Существуют несколько подходов, опи\-сы\-ва\-ющих проб\-ле\-му поиска переводных 
заимствований для некоторых пар языков~\cite{clkga,clfreshonto}, например для 
пары ис\-пан\-ский--анг\-лий\-ский. Настоящая работа посвящена обнаружению переводных 
заимствований для пары языков рус\-ский--анг\-лий\-ский. Данная пара нечасто 
встречается в~литературе и~не является родственной. Выбор пары языков рус\-ский--анг\-лий\-ский 
обуслов\-лен преобладанием англоязычных пуб\-ли\-ка\-ций в~интернете 
и~лучшим знанием этого языка по сравнению с~другими. Аналогично 
работам~\cite{framework1,framework2} в~данной статье предлагается описание 
алгоритма полного цик\-ла поиска заимствований~--- сначала ведется поиск 
до\-ку\-мен\-тов-кан\-ди\-да\-тов по внешней коллекции, затем происходит их детальное 
сравнение с~проверяемым документом. Предлагается алгоритм, основанный на 
моноязыковом анализе документов, схожем с~проведенным 
в~работах~\cite{mono,fruct}~--- про\-ве\-ря\-емый документ переводится на английский 
язык с~использованием системы машинного перевода с~дальнейшим сравнением 
текс\-то\-вых фрагментов внут\-ри документов.

В ряде работ, посвященных поиску переводных заимствований, используются 
дополнительные ресурсы, такие как тезаурусы и~онтологии.
В~работах~\cite{clkga,clfreshonto} авторы предлагают использовать базы знаний 
для извлечения информации о~близости между текстами. В работе~\cite{clfreshonto} 
предлагается алгоритм, основанный на комбинации нейронных сетей и~графов знаний. 
Основной недостаток этого подхода~--- ресурсоемкость: использование 
мультиязычных онтологий и~баз знаний требует больших вычислительных мощностей 
для построения семантических графов для каждого текстового фрагмента, а также 
сравнения полученных семантических графов.

В данной работе предлагается декомпозиция алгоритма обнаружения переводных 
заимствований для поиска по большим текстовым коллекциям.
Общая схема алгоритма включает следующие шаги.
\begin{enumerate}
\item \textit{Машинный перевод}~--- перевод проверяемого документа на английский 
язык. Для этого используется система статистического машинного 
перевода~\cite{moses}.
\item \textit{Поиск до\-ку\-мен\-тов-кан\-ди\-да\-тов}~--- для проверяемого документа 
находятся наиболее релевантные до\-ку\-мен\-ты-кан\-ди\-да\-ты, для этого используется 
модификация алгоритма шинглов.
\item \textit{Сравнение документов}~--- текст разбивается на фрагменты, строится 
отображение каждой фразы в~векторное пространство. Для каждого вектора 
проверяемого документа находятся ближайшие векторы из до\-ку\-мен\-тов-кан\-ди\-да\-тов, 
после чего проводится классификация пар данных векторов на схожие и~несхожие 
пары текстовых фрагментов.
\end{enumerate}

Так как в~предлагаемом алгоритме используется моноязыковой анализ заимствований, 
то задача близка к~задаче обнаружения перефразированного текс\-та.
Ряд подходов~\cite{Socher1,wieting,Iyyer,vbta} к~решению этой задачи используют 
векторные пред\-став\-ле\-ния фраз, полученные с~по\-мощью нейронных сетей глубокого 
обуче\-ния.
В работе~\cite{Iyyer} предлагается  нейронный мешок слов (\textit{англ}.\ Neural Bag-of-Words) 
и~глубокие усред\-ня\-ющие сети (\textit{англ}.\ Deep averaging networks).
В~данной \mbox{статье} предлагается использовать выходы нейронной сети как векторные 
пред\-став\-ле\-ния текс\-то\-вых фрагментов для дальнейшего при\-бли\-жен\-но\-го алгоритма 
поиска бли\-жай\-ше\-го соседа~\cite{ann}.

В работе исследуются свойства предлагаемого метода обнаружения переводных 
заимствований. Проводится анализ моделей глубокого обучения, используемых на 
этапе сравнения документов, а~также составной оптимизируемой функции. Проверка 
качества предложенного метода проводится как на синтетической выборке, так и~на 
статьях из журналов, входящих в~РИНЦ. 
Проводится анализ ошибок. Предложенный метод поиска заимствований сравнивается 
с~базовым алгоритмом поиска заимствований, основанным на использовании машинного 
перевода и~алгоритме шинглов.

\vspace*{-6pt}

\section{Постановка задачи}

Пусть заданы коллекции документов на английском языке 

\noindent
$$
D_e = \{d_e^j\}_{j=1}^N
$$ 
и~русском языке 
$$
D_r = \{d_r^i\}_{i=1}^M.
$$
 Документы на русском и~английском 
языке представимы в~виде конкатенации текстовых фрагментов:
$$
d_e^j = \left[s_{e_1}^j \sqcup \cdots \sqcup s_{e_h}^j\right];\enskip
d_r^i \hm= \left[s_{r_1}^i 
\sqcup \cdots \sqcup s_{r_k}^i\right].
$$

Пусть задана выборка  
$$
\mathcal{D}=\left\{(d_e^l, d_r^l), \mathrm{RL}^l\right\}_{l=1}^L,
$$
 где каждой 
паре документов $(d_e^l, d_r^l)_{d_e^l \in D_e, d_r^l \in D_r}$ сопоставлен 
список пар фрагментов 
$$
\mathrm{RL}= \left[(s_{e_1}^l, s_{r_1}^l), \ldots, (s_{e_{k(l)}}^l, 
s_{r_{k(l)}}^l)\right].
$$
 Для каждой пары $(s_{e_{\tilde{k}}}^l, s_{r_{\tilde{k}}}^l)$ 
известно, что фрагмент~$s_{r_{\tilde{k}}}^l$ является переводом фрагмента~$s_{e_{\tilde{k}}}^l$.


Модель $f$ задается как последовательное выполнение функций {filter} 
и~comparison, где
\begin{align*}
\mbox{filter:}&\ \ (d_r^i, D_e)_{d_r^i \in D_r} \rightarrow D_e^{\mathrm{retrieved}_i} \subset 
D_e,\\
\mbox{comparison:}&\ \  (d_r^i, D_e^{\mathrm{retrieved}_i})_{d_r^i \in D_r} \rightarrow \mathrm{RL}^{i}.
\end{align*}
Здесь $\mathrm{RL}^{i}$~--- список пар фрагментов. Функция {filter} отвечает за 
сужение числа документов коллекции, сравниваемых с~проверяемым документом, 
и~позволяет проводить дальнейшее более детальное сравнение {comparison} 
с~использованием ресурсоемких вычислительных алгоритмов, основанных на моделях 
глубокого обучения.


Качество модели $f$ оценивается с~помощью функций Precision и~Recall:
\begin{align*}
\mbox{Precision}& = \fr{|(\mathop{\cup}\nolimits_{l=1}^L \mathrm{RL}^l) \cap (\mathop{\cup}\nolimits_{i=1}^M 
\mathrm{RL}^{i})|}{|\mathop{\cup}\nolimits_{i=1}^M \mathrm{RL}^i|},\\
\mbox{Recall} &= \fr{|(\mathop{\cup}\nolimits_{l=1}^L \mathrm{RL}^l) \cap (\mathop{\cup}\nolimits_{i=1}^M 
\mathrm{RL}^{i})|}{|\mathop{\cup}\nolimits_{l=1}^L \mathrm{RL}^{l}|}.
\end{align*}

Требуется найти функцию $f$, мак\-си\-ми\-зи\-ру\-ющую~F1, среднее гармоническое 
показателей Precision и~Recall:
\begin{align*}
\hat{f}& = \argmax\limits_{f \in \mathcal{F}}\text{F1}(f, \mathcal{D}), \\[3pt]
 \text{F1} &= \fr{2  \mbox{Precision} \cdot \mbox{Recall} 
}{\mbox{Precision}+\mbox{Recall}},
%\label{opt_frag}
\end{align*}
где $\mathcal{F}$~--- заданное семейство моделей.



\section{Поиск документов-кандидатов}

Одним из алгоритмов поиска до\-ку\-мен\-тов-кан\-ди\-да\-тов в~задачах обнаружения дословных 
заимствований и~поиска \textit{поч\-ти-дуб\-ли\-ка\-тов} текста\linebreak служит алгоритм, 
основанный на построении инвертированного индекса, в~котором каждый документ 
коллекции представляется набором \textit{шинглов}~\cite{shingles1}, т.\,е.\ 
набором перекрывающихся $n$-грамм. Про\-ве\-ря\-емый документ также разбивается на 
шинглы, после чего проводится поиск документов по инвертированному индексу 
с~наибольшим совпадением шинглов. В~данной работе предлагается обобщение алгоритма 
шинглов, позволяющее улучшить качество поиска кандидатов в~случае обнаружения 
переводных заимствований.

Предлагается функция filter следующего вида:
\begin{multline*}
\mbox{filter}\left(d_r^i, D_e\right) ={}\\
{}=\! \!\!\!\!\!\argmax\limits_{D_e^{'} \subset D_e, |D_e^{'}|=k}\sum\limits_{d_e^j \in 
D_e^{'}}\sum\limits_{h \in \mathcal{H}(d_r^i)}\!\!\!\! \mathbf{I}\left[h \in 
\mathcal{H}(d_e^j)\right]\! \Big / \!
\left(|d_e^{j^{'}} \in D_e\!:\right.\\
\left. h \in 
\mathcal{H}(d_e^{j^{'}})|^{\alpha} + \mbox{const}\right)\,.
\end{multline*}
Здесь $\mathcal{H}$~--- множество $n$-грамм документа, упорядоченная 
последовательность $n$ меток кластеров, где процедура формирования кластеров 
описана ниже; $\alpha \hm\in \mathbb{R}$; $k$~--- оптимизируемый гиперпараметр.


Для уменьшения влияния не\-од\-но\-знач\-ности перевода на поиск до\-ку\-мен\-тов-кан\-ди\-да\-тов 
предлагается заменять слова на соответствующие им метки кластеров:
$$
\left\{x_1, \ldots, x_n\right\} \rightarrow \left\{ \mbox{class}\left(x_1\right), \ldots, 
\mbox{class}\left(x_n\right)\right\} = h\,,
$$
где $x_1, \ldots, x_n$~--- слова. Кластеры предварительно выделены из текстового 
корпуса и~содержат семантически близкие слова.
Для уменьшения неоднозначности перевода перед разбиением на $n$-граммы 
предлагается удалять из текста стоп-сло\-ва и~проводить лемматизацию. Для учета 
возможных перестановок слов, возникающих после перевода текста, слова внутри 
каждой $n$-грам\-мы сортируются в~лексикографическом порядке.

В данной работе для получения кластеров используется модель векторного 
представления слов, основанная на дистрибутивной гипотезе. Кластеризация 
проводится с~использованием косинусной функции расстояния
\begin{equation}
\label{eq:cos}
    \cos \left(\mathbf{c}_1, \mathbf{c}_2\right) = \fr{\langle\mathbf{c}_1, 
\mathbf{c}_2\rangle}{{||\mathbf{c}_1||_2||\mathbf{c}_2||_2}},
\end{equation}
где $\mathbf{c}_1$ и~$\mathbf{c}_2$~--- векторы из одного векторного пространства.


Ниже приведены примеры полученных кластеров:
\begin{itemize}
\item $[$beer, beers, brewing, ale, brew, brewery, pint, stout, guinness, 
ipa, brewed, lager, ales, brews, pints, cask$]$;
\item $[$brilliant, excellent, exceptional, finest, outstanding, super, 
terrific$]$.
\end{itemize}


\vspace*{-6pt}

\section{Сравнение документов}

Для сравнения найденных документов-кан\-ди\-да\-тов $D_e^{\mathrm{retrieved}_i}$ 
и~проверяемого документа~$(d_r^i)$\linebreak используется модель векторного пред\-став\-ле\-ния 
фразы~--- текс\-ты разбиваются на фрагменты и~сравниваются соответствующие им 
векторы. Ниже пред\-став\-ле\-ны детали алгоритма сравнения, а~также анализ 
пред\-ла\-га\-емой оптимизационной задачи.

\vspace*{-6pt}

\subsection{Модель векторного представления фразы}

Рассмотрим подробнее этап построения отображения фрагмента в~вектор.
Пусть каждому слову документа на языке коллекции поставлен в~соответствие вектор 
$\mathbf{v} \hm\in \mathbb{R}^u$ размерности~$u$. Для прос\-то\-ты будем полагать, что 
все фрагменты на языке коллекции имеют ограниченную длину $n_{\mathrm{col}}$.
Тогда моделью векторизации фрагмента будем называть отображение
$$
    \mathbf{h}: \mathbb{W} \times \mathbb{R}^{u \times n_{\mathrm{col}}} \to 
\mathbb{R}^u\,,
$$
где $\mathbb{W}$~--- пространство параметров модели. Объекты из множества 
$\mathbb{R}^{u \times n_{\mathrm{col}}}$ являются последовательной конкатенацией 
векторов векторных представлений слов для фрагментов выборки:
$$
\mathbf{x} \in [\mathbf{v}_1, \dots, \mathbf{v}_{n_{\mathrm{col}}}]^{\mathrm{T}}\,,  
\mathbf{x} \in \mathbb{R}^{u \times n_{\mathrm{col}}}.
$$
Для работы с~фрагментами длиной меньше $n_{\mathrm{col}}$ определим некоторый 
вектор, обозначающий пус\-тое слово.


Модель оптимизируется в~режиме частичного обучения с~учителем. В~качестве 
оптимизируемой функции используется составная функция ошибки, представляющая 
собой сумму ошибки реконструкции и~ошибки отступа:
\begin{equation}
\label{eq:alpha}
\alpha E_{\mathrm{rec}}\left(\mathbf{X}_{\mathrm{rec}}, \mathbf{w}\right) + (1 - 
\alpha)E_{\mathrm{me}}(\mathbf{X}_{\mathrm{me}},\mathbf{w}) \to \min_{\mathbf{w} \in 
\mathbb{W}},
\end{equation}
где $E_{\mathrm{rec}}$~--- ошибка реконструкции; $E_{\mathrm{me}}$ --- ошибка отступа; 
$\mathbf{X}_{\mathrm{rec}}$ и~$\mathbf{X}_{\mathrm{me}}$~--- обучающие выборки;  
$\mathbf{w}$~--- параметры модели; $\alpha$~--- настраиваемый гиперпараметр.
Рассмотрим подробнее каждое слагаемое функции ошибки.

Первое слагаемое функции ошибки соответствует модели автокодировщика.
Пусть задана выбор\-ка $\mathbf{X}_{\mathrm{rec}} \subset \mathbb{R}^{u \times 
n_{\mathrm{col}}}$.  Модель~$\mathbf{h}$ выступает в~качестве функции кодирования 
информации о выборке~$\mathbf{X}_{\mathrm{rec}}$. Пусть также задана вспомогательная 
функция декодирования~$\mathbf{g}$, восстанавливающая исходное векторное 
представление~$\mathbf{x}$ по выходам модели~$\mathbf{h}$:
$$
   \mathbf{r}(\mathbf{x}, \mathbf{w}) = \mathbf{g}(\cdot, \mathbf{w}) \circ 
\mathbf{h} (\mathbf{x}, \mathbf{w}) \approx  \mathbf{x},  \mathbf{x} \in 
\mathbb{R}^{u \times n_{\mathrm{col}}}.
$$
Минимизируемая ошибка реконструкции выглядит следующим образом:
\begin{equation}
\label{eq:rec}
E_{\mathrm{rec}}(\mathbf{X}_{\mathrm{rec}}, \mathbf{w}) = 
\fr{1}{|\mathbf{X}_{\mathrm{rec}}|}\sum\limits_{\mathbf{x} \in \mathbf{X}_{\mathrm{rec}}} 
\parallel\mathbf{x}  - \mathbf{r}(\mathbf{x}, \mathbf{w}) \parallel^2_2.
\end{equation}


Выбор ошибки реконструкции в~качестве оптимизируемой функции можно обосновать, 
используя результаты статьи~\cite{ae}. Будем пользоваться результатами, 
доказанными в~работе~\cite{ae}, где было показано, что автокодировщики 
с~регуляризацией специального вида позволяют оценить распределение~$p(\mathbf{X})$ 
объектов, принадлежащих генеральной совокупности.

\smallskip

\noindent
\textbf{Теорема~1}\ 
\cite{ae}. %\label{manifold}
\textit{Пусть $p$~--- дифференцируемая плотность вероятности и~$\forall\, 
\mathbf{x}_i \hm\in \mathbb{R}^{u \times n_{\mathrm{col}}}$ $p(\mathbf{x}_i)\hm\neq 0$.  
Пусть $\mathcal{L}_{\sigma^2}$~--- функция потерь вида}
\begin{multline*}
\mathcal{L}_{\sigma^2} ={}\\
{}=\!\!\!\int\limits_{\mathbb{R}^{u \times n_{\mathrm{col}}}} \!\!\!\!\!\!
p(\mathbf{x}) \left[
\parallel \mathbf{x} - \mathbf{r}(\mathbf{x}, \mathbf{w}) 
\parallel^2_2 + \sigma^2  \left\| \fr{\partial \mathbf{r}(\mathbf{x}, 
\mathbf{w})}{\partial \mathbf{x}} \right\|_{F}^2\right] d\mathbf{x}\,,\hspace*{-7.12155pt}
\end{multline*}
\textit{где $\mathbf{r}$ дважды дифференцируема}; $0 \hm\leqslant \sigma \hm\in \mathbb{R}$. 
\textit{Пусть $\hat{\mathbf{w}}$~--- оптимум функции~$\mathbf{r}$ по параметрам моделей 
кодирования и~декодирования, доставляющий минимум~$\mathcal{L}_{\sigma^2}$. 
Тогда}
$$
\hat{\mathbf{r}}_{\sigma^2}\left(\mathbf{x}, \mathbf{w}\right) = \mathbf{x} + \sigma^2 
\fr{\partial \log p(\mathbf{x})}{\partial \mathbf{x}} + o\left(\sigma^2\right), \quad 
\sigma^2 \rightarrow 0\,.
$$


Используя результаты теоремы~1, можно сделать следующее утверждение.

\smallskip

\noindent
\textbf{Теорема~2.}\ 
\textit{Плотность вероятности представима в~виде}:
$$
\fr{\hat{\mathbf{r}}_{\sigma^2}(\mathbf{x}, \mathbf{w}) - 
\mathbf{x}}{\sigma^2} \approx -\fr{\partial}{\partial 
\mathbf{x}}E(\mathbf{x}),
$$ 
\textit{где} 
$(\mathbf{x}) = ({1}/{Z})\exp(-
E(\mathbf{x}))$, $Z$~--- \textit{нормировочная константа}.

\smallskip

\noindent
Д\,о\,к\,а\,з\,а\,т\,е\,л\,ь\,с\,т\,в\,о\,.
\begin{align*}
{\mathbf{r}}_{\sigma^2}\left(\mathbf{x}, \hat{\mathbf{w}}\right) &= \mathbf{x} + \sigma^2 
\fr{\partial}{\partial \mathbf{x}}\log p(\mathbf{x}) + o\left(\sigma^2\right);
\\
\fr{{\mathbf{r}}_{\sigma^2}(\mathbf{x}, \hat{\mathbf{w}}) - 
\mathbf{x}}{\sigma^2} &= \fr{\partial}{\partial \mathbf{x}}\log p(\mathbf{x}) + o(1);
\\
\fr{{\mathbf{r}}_{\sigma^2}(\mathbf{x}, \hat{\mathbf{w}}) - 
\mathbf{x}}{\sigma^2} &\approx \fr{\partial}{\partial \mathbf{x}}\log 
p(\mathbf{x}).
\end{align*}
Представляя $\log p(\mathbf{x})$ в~форме $-E(\mathbf{x}) \hm- \log Z$, получим 
искомое выражение.

\smallskip

Таким образом, при устремлении регуляризатора~$\sigma$ к~нулю получается  
\textit{языковая модель}, т.\,е.\ распределение вероятностей на множестве~$\mathbf{X}$~--- 
множестве текстовых последовательностей.

Второе слагаемое составной функции ошибки~--- ошибка отступа~\cite{wieting}. Для 
оптимизации этой функции ошибки используется выборка $\mathbf{X}_{\mathrm{me}}\hm = 
\{ (\mathbf{x}_i, \mathbf{x}_j)\}$, состоящая из пар объектов:
$$
    \mathbf{X}_{\mathrm{me}} = \left[\mathbf{X}_{\mathrm{me}}^A; \mathbf{X}_{\mathrm{me}}^B\right]  
\subset \mathbb{R}^{u \times n_{\mathrm{col}}}  \times  \mathbb{R}^{u \times 
n_{\mathrm{col}}};
$$

\vspace*{-14pt}

\noindent
\begin{multline}
\label{eq:me}
E_{\mathrm{me}} = \fr{1}{|\mathbf{X}_{\mathrm{me}}|}\left( \sum\limits_{(\mathbf{x}_i, 
\mathbf{x}_j) \in \mathbf{X}_{\mathrm{me}}}\!\!\!\!\max \left(0, \delta - c_{-}\right) +{}\right.\\
\left.{}+ \max\left(0, \delta 
- c_{+}\right) 
\vphantom{\sum\limits_{(\mathbf{x}_i, 
\mathbf{x}_j) \in \mathbf{X}_{\mathrm{me}}}}
\right),
\end{multline}

\vspace*{-6pt}


\noindent
где 

\vspace*{-6pt}

\noindent
\begin{multline*}
c_{-} = \cos\left(\mathbf{h}\left(\mathbf{x}_i, \mathbf{w}\right), 
\mathbf{h}\left(\mathbf{x}_j, \mathbf{w}\right)\right) -{}\\
{}- \cos\left(\mathbf{h}\left(\mathbf{x}_i, 
\mathbf{w}\right), \mathbf{h}\left(\mathbf{x}_{i^{'}}, \mathbf{w}\right)\right);
\end{multline*}

\vspace*{-12pt}

\noindent
\begin{multline*}
c_{+} = \cos\left(\mathbf{h}\left(\mathbf{x}_i, \mathbf{w}\right), 
\mathbf{h}\left(\mathbf{x}_j, \mathbf{w}\right)\right) - {}\\
{}-\cos\left(\mathbf{h}\left(\mathbf{x}_j, 
\mathbf{w}\right), \mathbf{h}\left(\mathbf{x}_{j'}, \mathbf{w}\right)\right);
\end{multline*}

\vspace*{-2pt}

\noindent
$\delta$~--- отступ; $\cos$~--- функция расстояния~\eqref{eq:cos}, 

\noindent
\begin{align*}
\mathbf{x}_{i^{'}}&=\argmax\limits_{\mathbf{x}_{i^{'}} \in \mathbf{X}^B, 
\mathbf{x}_{i^{'}} \neq \mathbf{x}_{j}}\cos\left(\mathbf{x}_i, 
\mathbf{x}_{i^{'}}\right);
\\
\mathbf{x}_{j^{'}}&=\argmax\limits_{\mathbf{x}_{j^{'}} \in \mathbf{X}^A, 
\mathbf{x}_{j^{'}} \neq \mathbf{x}_{i}}\cos\left(\mathbf{x}_i, 
\mathbf{x}_{i^{'}}\right).
\end{align*}

Следующая теорема объясняет поведение данного слагаемого при проводимой 
оптимизации параметров~$\mathbf{w}$ модели~$\mathbf{h}$.

\smallskip

\noindent
\textbf{Теорема~3.}\ 
\textit{Пусть выполнены следующие условия}.
\begin{enumerate}
\item \textit{Задан гиперпараметр} $\delta \hm\in (0, 2).$
\item \textit{Мощность выборки} $|\mathbf{X}_{\mathrm{me}}|$ \textit{ограничена следующей величиной}:
\begin{multline}
    |\mathbf{X}_{\mathrm{me}}|(|\mathbf{X}_{\mathrm{me}}|-1)  \leq{}\\
    \hspace*{-3pt}{}\leq \sqrt{\pi} 
\fr{\Gamma((u-1)/2)}{\Gamma(u/2)} \left( \int\limits_{0}^{\arccos(1 - \delta)}\!\! \!\!\!\!\!\!
\sin^{u-2} x\, dx \right )^{\!-1}\!\!.\!\!
\label{eq:sphere_code}
\end{multline}
\item \textit{Подвыборки $\mathbf{X}^A_{\mathrm{me}}$ и~$\mathbf{X}^B_{\mathrm{me}}$ содержат все 
элементы в~единственном числе, ни один элемент не встречается в~обеих выборках}.
\end{enumerate}
\textit{Тогда существует непрерывное отображение~$\hat{\mathbf{h}}$ из множества 
векторных представлений слов} $\mathbb{R}^{u \times n_{\mathrm{col}}}$
\textit{в~векторное 
пространство~$\mathbb{R}^{u}$, доставляющее глобальный минимум функции} $E_{\mathrm{me}} \hm= 
0$.

\pagebreak

\noindent
Д\,о\,к\,а\,з\,а\,т\,е\,л\,ь\,с\,т\,в\,о\,.\ \
Построим отображение~$\hat{\mathbf{h}}$ явно.
Положим для каждой пары $(\mathbf{x}_1, \mathbf{x}_2)$: 
$\hat{\mathbf{h}}(\mathbf{x}_1) \hm= \hat{\mathbf{h}}(\mathbf{x}_2).$

Тогда функция $E_{\mathrm{me}}$ выглядит следующим образом с~точностью до множителя:
\begin{multline*}
\!E_{\mathrm{me}}= \!\!\!\!\!\sum\limits_{(\mathbf{x}_i, \mathbf{x}_j) \in \mathbf{X}_{\mathrm{me}}}
\!\!\!\!\!\!\!\!\max\left(0, \delta - 1 
+ \cos\left(\hat{\mathbf{h}}(\mathbf{x}_i), 
\hat{\mathbf{h}}\left(
\mathbf{x}_{i^{'}}\right)\!\right)\!\right) +{}\\
{}+ \max\left(0, \delta - 1  + 
\cos\left(\hat{\mathbf{h}}(\mathbf{x}_j), 
\hat{\mathbf{h}}\left(\mathbf{x}_{j^{'}}\right)\!\right)\!\right).
\end{multline*}

Область значений функции ограничена снизу нулем, который достигается при 
выполнении условий:
$$
   1 - \delta \geq  \cos \left(\hat{\mathbf{h}}(\mathbf{x}), 
\hat{\mathbf{h}}\left(\mathbf{x}'\right)\right)
$$
 для любой пары $\mathbf{x} \hm\in 
\mathbf{X}^A_{\mathrm{me}}$, $\mathbf{x}' \hm\in \mathbf{X}^B_{\mathrm{me}}$,  
$(\mathbf{x}, \mathbf{x}') \notin \mathbf{X}_{\mathrm{me}}$, $(\mathbf{x}', \mathbf{x}) \hm\notin 
\mathbf{X}_{\mathrm{me}}.$
Число пар, описанных выше, в~множестве~$\mathbf{X}_{\mathrm{me}}$ при выполнении 
третьего условия теоремы равно $|\mathbf{X}_{\mathrm{me}}|(|\mathbf{X}_{\mathrm{me}}|\hm-
1)$. Назначим значение отображения~$\hat{\mathbf{h}}$ для каждой такой пары так, 
чтобы $\cos(\hat{\mathbf{h}}(\mathbf{x}), \hat{\mathbf{h}}(\mathbf{x}'))  
\hm\leq 1 \hm- \delta$.

Существование такого отображения следует из задачи о нахождении сферического 
кода максимального размера для сферы в~пространстве размерности~$u$ и~углом 
$\arccos(1 \hm- \delta).$
В~работе~\cite{sphere_code} представлена нижняя оценка для размерности выборки, 
удовлетворяющей заданным условиям. Оценка соответствует правой части 
неравенства~\eqref{eq:sphere_code}. Так как выборка $\mathbf{X}_{\mathrm{me}}$  
конечна, то для построения непрерывной функции, заданной условиями, описанными 
выше, можно использовать интерполяционные полиномы, что и~требовалось доказать.

\smallskip


Заметим, что предложенное в~теореме отображение является непрерывным, поэтому 
для приближения данного отображения можно использовать нейросетевые модели. По 
теореме Цыбенко отображения из класса нейросетевых моделей будут приближать 
непрерывные модели сколь угодно хорошо~\cite{cybenko}.

Таким образом, составная оптимизируемая функция~\eqref{eq:alpha} позволяет 
получить модель, которая, с~одной стороны, обладает обобщающими свойствами, за 
которые отвечает языковая модель~\eqref{eq:rec}, с~другой стороны, эффективно 
разделяет схожие и~несхожие фразы из обучающей выборки~\eqref{eq:me}. 
Гиперпараметр~$\alpha$ отвечает за вклад каждого из оптимизируемых слагаемых в~данную функцию.

\subsection{Классификатор}

Для каждого вектора фразы~$\mathbf{h}(\mathbf{x}_{r_{a}}^i)$ из про\-ве\-ря\-емо\-го 
документа~$d_r^i$ находится~$v$~ближайших векторов по косинусной функции 
расстояния~\eqref{eq:cos} для фрагментов из до\-ку\-мен\-тов-кан\-ди\-да\-тов 
$D_e^{\mathrm{retrieved}_i}$, используя метод приближенного поиска ближайшего соседа.
Основная цель данной процедуры~---~сократить число пар фрагментов для 
классификации для снижения ресурсоемкости этапа сравнения документа.

Для векторного представления пары фрагментов $(\mathbf{h}(\mathbf{x}_{e_{b}}^j), 
\mathbf{h}(\mathbf{x}_{r_{a}}^i))$ рассматривается следующее решающее правило:
\begin{multline}
\label{eq:t1t2}
f_{\mathrm{frag}}\left(\left(\mathbf{h}\left(\mathbf{x}_{e_{b}}^j\right), \mathbf{h}(\mathbf{x}_{r_{a}}^i)\right)\right) ={}\\
{}=
\begin{cases}
1, &\mbox{ если } \cos\left(\mathbf{h}(\mathbf{x}_{e_{b}}^j), 
\mathbf{h}\left(\mathbf{x}_{r_{a}}^i\right)\right)>t_1\\
& \hspace*{5mm}\mbox{ и~} 
p\left(\mathbf{h}\left(\mathbf{x}_{e_{b}}^j\right), \mathbf{h}\left(\mathbf{x}_{r_{a}}^i\right)\right)>t_2;\\
0 &\mbox{ иначе},
\end{cases}
\end{multline}
где $p$~--- вероятность классификатора; $t_1$~--- порог косинусной  функции 
расстояния~\eqref{eq:cos}; $t_2$~--- минимальный порог вероятности 
классификатора. 

В качестве признаков используется конкате\-на\-ция разницы по модулю 
и~покомпонентное\linebreak произведение компонент вектора 
$[|\mathbf{h}(\mathbf{x}_{e_{b}}^j)\hm- 
\mathbf{h}(\mathbf{x}_{r_{a}}^i)|,\mathbf{h}(\mathbf{x}_{e_{b}}^j) \odot 
\mathbf{h}(\mathbf{x}_{r_{a}}^i)]$. В~качестве классификатора выступает модель 
случайного леса.


%\subsection{Анализ предложенного метода}

%Приведем анализ сложности предложенного метода.

%\begin{theorem}
%Пусть заданы:
%\begin{enumerate}
%\item Количество документов в~коллекции $M$.
%\item Сложность поиска одного шингла по коллекции $T_\text{search}(M)$.
%\item Сложность перевода текста длиной $n$ символов $T_\text{tr}(n)$.
%\item Среднее число документов, находящихся в~коллекции по одному шинглу 
%$\alpha$.
%\item Сложность векторизации текста $T_\text{vectorize}(n)$, являющаяся 
%монотонно-возрастающей по $n$ функцией.
%\item Сложность поиска $v$ ближайших соседей по коллекции из $m$ векторов 
%$T_\text{KNN}(m, v)$, являющаяся монотонно возрастаюшей по $m$ функцией.
%\item Максимальный размер текста в~коллекции $n_\text{col}^\text{max}$ в~словах.
%\end{enumerate}


%Тогда средняя сложность проверки документа длиной $n_\text{susp}$ слов 
%оценивается как:
%\begin{equation}
%\label{eq:complexity}
%    T_\text{tr}(n_\text{susp}) + O(n_\text{susp})T_\text{search}(M) + O(\alpha 
%n_\text{susp} \log (\alpha n_\text{susp})) +
%\end{equation}
%\[
%+ T_\text{vectorize}(n_\text{susp}) + 
%KT_\text{vectorize}(n_\text{col}^\text{max}) + 
%n_\text{susp}T_\text{KNN}(O(Kn_\text{susp}), v) + O(vn_\text{susp}).
%\]
%\end{theorem}
%\begin{proof}

%Сложность перевода документа равняется $T_\text{tr}(n_\text{susp})$ 
%в~%соответствии с~заданными обозначениями.
%Поиск документов кандидатов производится по шинглам. Количество шинглов в~
%документе зависит линейно от количества слов, поэтому оценивается как 
%$O(n_\text{susp})$.
%Поиск документов по каждому шинглу осуществляется за $T_\text{search}(M)$, 
%поэтому суммарный поиск всех документов занимает 
%$O(n_\text{susp})T_\text{search}(M)$.
%Количество унникальных документов, найденных по всем шинглам документа можно 
%оценить сверху как $O(n_\text{susp})\alpha$.

%Для дальнейшей обработки документа требуется построить список из $K$ документов-
%кандидатов, сложность построения такого списка равна $O(\alpha n_\text{susp} 
%\log (\alpha n_\text{susp}))$.

%Сложность векторизации проверяемого текста и~$K$ текстов документов-кандидатов 
%можно оценить сверху как $T_\text{vectorize}(n_\text{susp}) + 
%KT_\text{vectorize}(n_\text{col}^\text{max})$ в~силу монотонности функции 
%$T_\text{vectorize}(n)$.
%Полагая, что число предложений в~тексте линейно зависит от длины текста, оценим 
%сложность поиска ближайших соседей как $n_\text{susp} 
%T_\text{KNN}(O(Kn_\text{col}^\text{max}), v)$.

%На итоговом этапе проверки документа для каждой из найденных пар векторов-
%соседей производится классификации. Всего таких пар $O(vn_\text{susp})$,
%Таким образом, средняя сложность проверки документа соответствует 
%формуле~\eqref{eq:complexity}.
%\end{proof}

%Как видно из приведенной теоремы, сложность итогового алгоритма зависит от 
%сложности операций перевода, поиска шинглов, векторизации и~поиска ближайших 
%соседей.
%В качестве следствия теоремы приведем оценку сложности для простого случая.
%\begin{theorem}
%Пусть в~качестве системы машинного перевода используется статистичесекий 
%машинный перевод. Пусть также поиск по шинглам производится с~использованием 
%бинарного дерева поиска~\cite{cormen}, векторизация имеет линейную сложность, а 
%для поиска ближайших соседей используется алгоритм $k-d$-tree~\cite{kd}.
%Тогда сложность:
%\begin{equation}
%\label{eq:complexity}
%    O(n_\text{susp} \log M) + O(\alpha n_\text{susp} \log (\alpha 
%n_\text{susp})) + K O(n_\text{col}^\text{max}) + O(n_\text{susp}) 
%O(Kn_\text{col}^\text{max}) \log (Kn_\text{col}^\text{max})) + 
%O(vn_\text{susp})ю
%\end{equation}
%\end{theorem}
%\begin{proof}
%Сложность статистического перевода линейна по количесту слов, поэтому 
%$T_\text{tr}(n_\text{susp})  = O(n_\text{susp})$.
%Полагая коллекцию фиксированной, сложность поиска в~бинарном дереве можно 
%оценить как $O(\log M)$, тогда сложность поиска шинглов по коллекции равна 
%$O(n_\text{susp}\log M)$.
%Сложность векторизации проверяемого документа и~документов-кандидатов составляет 
%$O(n_\text{susp}) + K O(n_\text{col}^\text{max})$.
%Сложность построения индекса и~поиска составляет в~среднем $O(n_\text{susp}) 
%O(Kn_\text{col}^\text{max}) \log (Kn_\text{col}^\text{max}))$.
%Таким образом, средняя сложность проверки документа соответствует 
%формуле~\eqref{eq:complexity2}.
%\end{proof}

%Как видно из следствия, сложность итогового алгоритма субквадратична по 
%количеству слов в~проверяемом документе и~сублинейная по размеру коллекции 
%текстов. TODO. Заметим, что алгоритм $kd$-tree был приведен в~следствии в~
%качестве примера: для векторных пространств высокой размерности его 
%использование нецелесообразно, вместо него применяются методы со схожими 
%сложностными показателями.

\vspace*{-6pt}

\section{Вычислительный эксперимент}

Для анализа качества предложенного алгоритма был проведен ряд вычислительных 
экспериментов как на синтетической выборке~\cite{dataset}, так и~на реальных 
коллекциях документов.
В~данном разделе приводятся детали порождения синтетических выборок 
и~эксперименты, проведенные на них.

\vspace*{-6pt}

\subsection{Синтетическая коллекция переводных заимствований}

Для порождения переводных заимствований были использованы документы из 
английской и~русской версии сайта Wikipedia.
В качестве коллекции документов~$D_e$ были использованы 100~тыс.\ статей из 
английской версии Wikipedia.
В~качестве коллекции проверяемых документов~$D_r$ использовалась случайная 
подвыборка документов из русской версии Wikipedia. Для порождения заимствований 
для каждого документа $d_r^i \hm\in D_r$ применялся следующий алгоритм.
\begin{enumerate}
\item Выбрать документы-кан\-ди\-да\-ты~$\{d_e^j\}$ из коллекции~$D_e$. Для уменьшения 
разброса лексики\linebreak в~до\-ку\-мен\-тах-кан\-ди\-да\-тах и~проверяемом до\-ку\-менте выбор 
до\-ку\-мен\-тов-кан\-ди\-да\-тов проводился  из подвыборки~500~наиболее релевантных 
документов для проверяемого документа $d_r^i$. Для определения релевантности 
использовалась tf\;$\cdot$\;idf-ме\-ра. Чис\-ло до\-ку\-мен\-тов-кан\-ди\-да\-тов 
выбиралось случайно от~1 до~10.
\item Выбрать предложения из до\-ку\-мен\-тов-кан\-ди\-да\-тов~$\{d_e^j\}$ случайным образом и~перевести их на русский язык.
\item Заменить случайные предложения из проверяемого документа~$d_r^i$ на 
переведенные предложения из до\-ку\-мен\-тов-кан\-ди\-да\-тов. Доля замененных 
предложений из про\-ве\-ря\-емо\-го документа $d_r^i$ выбиралась случайно от~20\% до~80\%.
\end{enumerate}

\vspace*{-6pt}

\subsection{Оптимизация параметров рассматриваемых моделей}

В качестве модели векторного представления слов использовалась библиотека 
\texttt{fastText}~\cite{ft}, оптимизация параметров которой проводилась на 
английской версии Wikipedia. Размерность векторного пространства для векторного 
представления слов и~фрагментов была установлена как~100. Для оптимизации модели 
векторного представления текстовых фрагментов использовался алгоритм AdaDelta 
с~параметрами $\varepsilon\hm=10^{-6}$, $\mu\hm=0{,}95$ и~L2-ре\-гу\-ля\-ри\-за\-ция 
$\lambda_2\hm=10^{-6}$. Для итоговой  функции потерь~\eqref{eq:alpha} были установлены следующие 
значения гиперпараметров: $\delta\hm=0{,}3$; $\alpha\hm=0{,}1$. Пороги 
классификатора~\eqref{eq:t1t2} были подобраны на основе процедуры 
кросс-ва\-ли\-да\-ции: $t_1\hm=0{,}6$; $t_2\hm=0{,}5$. Для построения кластеров была использована 
агломеративная кластеризация на векторах слов. В~качестве меры близости слов 
рассматривалась косинусная  функция расстояния~\eqref{eq:cos} между 
соответствующими векторными представлениями. Итоговая модель содержала 30~тыс.\ 
кластеров для 777 тыс. слов. В качестве моделей кодирования~$\mathbf{h}$ 
и~декодирования~${\mathbf{g}}$ использовалась рекуррентная модель GRU
(gated recurrent unit)~\cite{gru}.
В~качестве системы машинного перевода использовался Moses~\cite{moses}, модель 
которого была была обучена на 18,5~млн параллельных предложений из корпусов 
Opus~\cite{opus}.
В~качестве выборки для минимизации ошибки реконструкции $E_{\mathrm{rec}}$~\eqref{eq:rec} 
использовались 10~млн предложений из английской версии Wikipedia.
Второе слагаемое функции потерь~\eqref{eq:me} использует информацию о похожих 
предложениях $\mathbf{X}_{\mathrm{me}}\hm = \{(\mathbf{x}_i, \mathbf{x}_j)\}$. 
В~качестве выборки таких предложений использовались пары параллельных предложений 
из корпуса OpenSubtitles~\cite{opus}. 

\vspace*{-6pt}

\subsection{Детали вычислительного эксперимента}

Было проведено три эксперимента на синтетических данных.
\begin{enumerate}
\item Поиск кандидатов. В~данном эксперименте анализировалось  качество 
полученной модели кластеров слов. В~качестве базового эксперимента для сравнения 
рассматривался алгоритм, основанный на шинглах без приведения слов к~меткам 
кластеров.
\item Сравнение фрагментов текста. В~данном эксперименте рассматривался случай, 
когда отбор кандидатов был проведен полностью корректно: Recall$@10\hm=1{,}0.$ 
В~качестве базового алгоритма также выступал алгоритм, основанный на шинглах: 
проверяемый документ~$d_r^i$ переводился на английский язык. После этого 
полученный текст проходил лемматизацию и~разбивался на множество перекрывающихся 
4-грамм.
Для учета возможных перестановок слов при переводе слова внутри каждой 4-грам\-мы сортировались. 
Результатом сравнения двух документов выступало множество 
совпавших отсортированных 4-грамм.

\item Эксперимент, оценивающий качество всего алгоритма (поиск кандидатов 
и~сравнение фрагментов текста). Данный эксперимент позволял оценить качество 
представленного алгоритма в~целом.
\end{enumerate}

Результаты эксперимента по поиску кандидатов представлены в~табл.~1.
Представленный алгоритм, основанный на построении кластеров, дает лучшее 
качество, чем базовый алгоритм, основанный на шинглах.



Результаты экспериментов по сравнению фрагментов текста представлены 
в~табл.~2. Пред\-став\-лен\-ный алгоритм показывает точ\-ность, 
сравнимую с~точностью базового алгоритма, и~полноту, значительно превосходящую 
полноту базового алгоритма. Точ\-ность базового алгоритма объясняется тем, что 
данный алгоритм учитывает схожесть только поч\-ти-дуб\-ли\-ка\-тов текста.




В третьем эксперименте, учитывавшем качество представленного алгоритма в~целом, 
были получены следующие показатели: $\mbox{Precision}\hm=0{,}83$; 
$\mbox{Recall}\hm=0{,}79$; $\mbox{F1}\hm=0{,}80$.

%\begin{table*}\small %tabl1
\begin{center}
\noindent
\parbox{138pt}{{{\tablename~1}\ \ \small{
Результаты эксперимента по поиску кандидатов
}}}

\vspace*{6pt}

%\label{table:stage2}
%\vspace*{2ex}

{\small \begin{tabular}{|l|c|}
\hline 
\multicolumn{1}{|c|}{\bf Алгоритм} & \bf  Recall@$10$  \\ 
\hline
Базовый & 0,93\\
%\hline
Представленный&  0,95 \\
\hline
\end{tabular}
}
\end{center}
%\end{table*}
%\begin{table*}\small %tabl2
\begin{center}
\noindent
\parbox{202pt}{{{\tablename~2}\ \ \small{
Результаты экспериментов по поиску схожих фрагментов текста
}}
}


\vspace*{6pt}
%\label{table:stage34}
%\vspace*{2ex}

{\small 
\tabcolsep=7pt
\begin{tabular}{|l|c|c|c|}
\hline 
\multicolumn{1}{|c|}{\bf Алгоритм} &  \bf Precision & \bf Recall  &  $\mathbf{F1}$ \\\hline
Базовый &  0,99 & 0,15 & 0,26 \\
%\hline
Представленный &  0,93 & 0,80 & 0,85 \\
\hline
\end{tabular}
}
\end{center}
%\end{table*}

\section{Результаты экспериментов на~реальной коллекции научных документов}

Для апробации представленного алгоритма был проведен эксперимент по поиску 
переводных заимствований на коллекции документов из электронной биб\-лио\-те\-ки 
{\sf eLibrary.ru}. Данная биб\-лио\-те\-ка содержит научные документы, входящие в~РИНЦ. 
Данный ресурс также содержит дополнительные\linebreak 
метаданные для каждого документа: заголовок,\linebreak авторов документа, язык документа 
и~принадлежность к~тематике, соответствующей Государственному рубрикатору 
на\-уч\-но-тех\-ни\-че\-ской информации (ГРНТИ).
Для апробации алгоритма в~качестве проверяемых документов~$D_r$ были 
подготовлены 2,5~млн документов на русском языке.

В качестве коллекции документов~$D_e$ использовались документы из английской 
версии Wikipedia, документы на английском языке из {\sf eLibrary.ru} и~\mbox{статьи} ресурса 
arXiv.org. Суммарное число полученных документов составило 7,6~млн.

В силу большого числа проверяемых документов~$D_r$ для дальнейшего анализа 
рассматривались документы, содержащие значительное число найден\-ных 
заимствований.
Была получена 21~тыс.\ документов со значительным числом заимствований. Из них 
были проанализированы 7,6~тыс.\ документов, выбранных случайно. Основной \mbox{целью} 
эксперимента было обнаружение переводных заимствований, когда заимствование 
произошло из англоязычного документа в~русскоязычный документ. В~то же время при 
анализе полученных результатов был выявлен ряд других срабатываний 
представленного алгоритма, которые были в~дальнейшем разделены на несколько 
типов:
\begin{itemize}
\item переводные заимствования~--- документ содержит заимствования, 
переведенные с~английского языка, выданные за оригинальный текст;
\item другие заимствования~--- заимствования из русскоязычных ресурсов или 
заимствования, направление которых нельзя определить по датам документов;
\item двуязычные статьи~--- работы одного и~того же автора на двух языках;
\item самоцитирование~--- цитирование автором его англоязычной работы;
\item цитирование законов~--- использование формулировок нормативных актов;
\item ошибочные срабатывания~--- лож\-но-по\-ло\-жи\-тель\-ные срабатывания 
представленного алгоритма;
\end{itemize}

%\setcounter{table}{2}
%\begin{table*}\small %tabl3
\begin{center}
\noindent
\parbox{202pt}{{{\tablename~3}\ \ \small{
езультаты экспериментов для коллекции документов {\sf eLibrary.ru}
}}
}


\vspace*{6pt}

{\small 
\tabcolsep=10pt
\begin{tabular}{|l|c|}
  \hline
 \multicolumn{1}{|c|}{\bf Тип} & \bf Количество  \\
  \hline
Переводные заимствования & 921 \\ 
%\hline
Другие заимствования & 2548\hphantom{9} \\ 
%\hline
Двуязычные статьи & 788 \\ 
%\hline
Самоцитирование & 669 \\ 
%\hline
Цитирование законов & 1567\hphantom{9} \\ 
%\hline
Ошибочные срабатывания & 507 \\ 
%\hline
Другое & 698 \\ 
\hline 
%\hline
Всего & 7689\hphantom{9} \\ 
\hline
\end{tabular}
}
\end{center}
%\end{table*}

%\vspace*{6pt}


\begin{itemize}
\item другое --- срабатывания, которые сложно отнести к~ка\-кой-ли\-бо категории из-за 
нехватки метаданных или плохого качества текстов.
\end{itemize}

Результаты экспериментов представлены в~табл.~3. Заметим, что были 
проанализированы только~36\% всех срабатываний алгоритма, поэтому можно 
предварительно оценить число документов с~переводными заимствованиями по всей 
коллекции в~2,5~тыс., что составляет~0,1\% всех документов. 
Заметим, что результаты были получены в~автоматическом режиме и~требуют 
дальнейшей экспертной верификации.




Распределение доли заимствований в~проанализированных документах 
представлено на рис.~1. Средняя доля заимствований со\-став\-ля\-ет~20\%.

Для анализа научных тематик, в~которых переводные заимствования происходят 
наиболее час\-то, были проанализированы документы, отнесенные к~типу 
\textit{переводные заимствования}.
Около~70\% проанализированных документов были классифицированы по~10~научным 
рубрикам. Наибольшая часть
 документов оказалась распределена между руб\-ри\-ка\-ми 
<<Экономика. Народное хозяйство. Экономические науки>> и~<<Право. Юридические 
науки>>. За-\linebreak\vspace*{-12pt}

{ \begin{center}  %fig1
 \vspace*{9pt}
    \mbox{%
\epsfxsize=79mm
\epsfbox{che-2.eps}
}

\end{center}

\noindent
{{\figurename~1}\ \ \small{
Гистограмма распределения доли заимствования в~тексте
}}}

\pagebreak

\end{multicols}

\setcounter{figure}{1}
\begin{figure*} 
 \vspace*{1pt}
\begin{center}  %fig2
   \mbox{%
\epsfxsize=142.211mm
\epsfbox{che-1.eps}
}

\end{center}
\vspace*{-9pt}
\Caption{Распределение заимствований по рубрикам ГРНТИ для типов 
\textit{переводные заимствования}~(\textit{а}) и~\textit{двуязычные статьи}~(\textit{б})
}
\vspace*{-3pt}

\end{figure*}

\begin{multicols}{2}

\noindent
метим, что распределение по рубрикам заимствований, отнесенных к~типу 
\textit{двуязычные статьи},
 значительно отличается от данного распределения.
Диаграммы десяти наиболее представительных руб\-рик для данных типов срабатываний 
показаны на рис.~2.




\paragraph*{Анализ ложно-от\-ри\-ца\-тель\-ных срабатываний.}
Для анализа лож\-но-от\-ри\-ца\-тель\-ных срабатываний представленного алгоритма была 
проанализирована полнота нахождения двуязычных документов. Оценка полноты была 
проведена с~помощью метаданных, полученных из {\sf eLibrary.ru}. Анализ срабатываний 
алгоритма показал, что только~85$\%$ документов были найдены алгоритмом 
корректно.  Заметим, что представленная оценка полноты является грубой, так как 
учитывает только полные переводы текстов.





Основная причина лож\-но-от\-ри\-ца\-тель\-ных срабатываний~--- низкое качество машинного 
перевода. Другой проб\-ле\-мой, значительно повлиявшей на качество нахождения 
двуязычных статей, является используемый алгоритм поиска кандидатов, позволяющий 
находить только близкие по структуре заимствования. Кроме того, значительная 
часть проанализированных документов имела некорректную кодировку, что также 
повлияло на полноту поиска документов.

\vspace*{-12pt}

\paragraph*{Анализ лож\-но-по\-ло\-жи\-тель\-ных срабатываний.}
Для анализа лож\-но-по\-ло\-жи\-тель\-ных срабатываний были проанализированы вручную 90 
документов, отнесенных к~типу \textit{ошибочные срабатывания}.
\mbox{Основная} проблема лож\-но-по\-ло\-жи\-тель\-ных срабатываний со\-сто\-яла в~некорректном 
векторном представлении предложений, содержащих именованные сущ\-ности, не 
встре\-ча\-емые в~обуча\-ющей \mbox{выборке}, а~также содержащих слова, незнакомые модели 
машинного перевода. Также было замечено, что алгоритм сравнения документов час\-то 
находил общие фразы вида <<Работа посвящена сле\-ду\-ющей проб\-ле\-ме$\ldots$>> и~т.\,п. 
Несмотря на корректность данных срабатываний, общие фразы представленного вида 
встречаются в~большом числе документов и~потому не долж\-ны рассматриваться как 
переводные заимствования.
Общий процент документов с~ложно-положительными срабатываниями составил 7\%.

\section{Заключение}

В работе предложен алгоритм обнаружения переводных заимствований. Предложена 
декомпозиция алгоритма обнаружения переводных заимствований, позволяющая 
проводить эффективный поиск заимствований на больших текстовых коллекциях. 
Проведен анализ предложенного метода обнаружения заимствований, а также 
составной функции ошибки, используемой для оптимизации модели глубокого 
обучения.  Для анализа качества представленного алгоритма были проведены 
эксперименты на синтетических данных для пары языков рус\-ский--анг\-лий\-ский. 
Качество алгоритма было также продемонстрировано на коллекции русскоязычных 
документов, входящих в~РИНЦ.
В~дальнейшем планируется развитие предложенного алгоритма: использование модели 
векторного представления предложений для задачи поиска кандидатов и~улучшение 
качества отображения, ставящего в~соответствие фразе вектор.\\

\bigskip

Авторы выражают свою благодарность Г.\,О.~Еременко, ООО <<Научная электронная 
библиотека>>, за предоставленные материалы.

{\small\frenchspacing
{%\baselineskip=10.8pt
%\addcontentsline{toc}{section}{References}
\begin{thebibliography}{99}

\bibitem{plag_cheh}
\Au{Никитов А.\,В., Орчаков~О.\,А., Чехович~Ю.\,В.} Плагиат в~работах 
студентов и~аспирантов: проблема и~методы противодействия~// Университетское 
управление: практика и~анализ, 2012. Т.~5. С.~61--68.

\bibitem{hist} %2
\Au{Khritankov A., Botov~P., Surovenko~N., Tsarkov~S., Viuchnov~D., 
Chekhovich~Y.} Discovering text reuse in large collections of documents: A~study 
of theses in history sciences~//  Artificial Intelligence and Natural Language 
\& Information Extraction, Social Media and Web Search FRUCT Conference.~--- 
IEEE, 2015. P.~26--32.


\bibitem{shingles1}
\Au{Зеленков И.\,В., Сегалович~И.\,В.} Сравнительный анализ методов 
определения нечетких дубликатов для Web-до\-ку\-мен\-тов~// Электронные библиотеки: 
перспективные методы и~технологии, электронные коллекции: Тр. 9-й 
Всеросс. научн. конф. RCDL.~--- Пе\-ре\-славль-За\-лес\-ский: 
Университет г.~Переславля, 2007. С.~166--174.



\bibitem{clkga} %4
\Au{Franco-Salvador~M., Gupta~P., Rosso~P.} Cross-language plagiarism 
detection using a multilingual semantic network~//  European Conference on 
Information Retrieval~/
Eds. P.~Serdyukov, P.~Braslavski, S.\,O.~Kuznetsov, \textit{et al}.~---
Lecture notes in computer science ser.~--- Berlin--Heidelberg: Springer,  2013. Vol.~7814. P.~710--713.

\bibitem{clfreshonto}
\Au{Franco-Salvador M., Gupta~P., Rosso~P.,  Banchs~R.} Cross-language 
plagiarism detection over continuous-space-and knowledge graph-based 
representations of language~// Knowl.-Based Syst., 2016. Vol.~111. P.~87--99.


\bibitem{framework1} %6
\Au{Grman J.,  Ravas~R.} Improved implementation for finding text 
similarities in large collections of data~// Notebook papers of CLEF 2011 Labs and Workshops~/ 
Eds. V.~Petras, P.~Forner, P.\,D.~Clough.~--- Amsterdam, The Netherlands, 
2011. Vol.~1177. 6~p. {\sf http://ceur-ws.org/Vol-1177/CLEF2011wn-PAN-GrmanEt2011.pdf}.

\bibitem{framework2} %7
\Au{Grozea C.,  Popescu~M.} The encoplot similarity measure for automatic 
detection of plagiarism~// Notebook papers of CLEF 2011 Labs and Workshops~/ Eds. V.~Petras, P.~Forner, P.\,D.~Clough.~--- Amsterdam, The Netherlands, 
2011. Vol.~1177. {\sf http://ceur-ws.org/Vol-1177/CLEF2011wn-PAN-GrozeaEt2011.pdf}. 

\bibitem{mono} %8
\Au{Muhr M., Kern~R., Zechner~M., Granitzer~M.} External and intrinsic 
plagiarism detection using a cross-lingual retrieval and segmentation system~// 
Notebook papers of CLEF 2010 Labs and Workshops~/
Eds. M.~Braschler, D.~Harman, E.~Pianta.~---
Padua, Italy, 2010. Vol.~1176.
{\sf http://ceur-ws.org/Vol-1176/CLEF2010wn-PAN-MuhrEt2010.pdf}.

\bibitem{fruct}
\Au{Bakhteev O., Kuznetsova~R., Romanov~A., Khritankov~A.} A~monolingual 
approach to detection of text reuse in Russian--English collection~// Artificial 
Intelligence and Natural Language \& Information Extraction, Social Media and 
Web Search FRUCT Conference.~--- IEEE, 2015. P.~3--10.

\bibitem{moses} %10
\Au{Koehn P., Hoang Hien, Birch~A., \textit{et al.}} Moses: Open source toolkit for statistical machine 
translation~//  45th Annual Meeting of the Association for 
Computational Linguistics Companion Volume Proceedings of the Demo and Poster 
Sessions Proceedings.~--- ACL, 2007. P.~177--180.


\bibitem{Socher1}
\Au{Tai K., Socher~R., Manning~C.}  Improved semantic representations from 
tree-structured long short-term memory networks~// 53rd 
Annual Meeting of the Association for Computational Linguistics and the 7th 
 Joint Conference (International) on Natural Language Processing Proceedings.~--- ACL, 2015. 
Vol.~1. P.~1556--1566.


\bibitem{wieting}
\Au{Wieting J.,  Bansal~M., Gimpel~K.,  Livescu~K.} Towards universal 
paraphrastic sentence embeddings~// arXiv.org, 2015. arXiv:1511.08198 [cs.CL].



\bibitem{Iyyer}
\Au{Iyyer M., Manjunatha~V.,  Boyd-Graber~J. Daume~H.} Deep unordered 
composition rivals syntactic methods for text classification~// 
53rd Annual Meeting of the Association for Computational Linguistics and the 
7th  Joint Conference (International) on Natural Language Processing Proceedings.~--- ACL, 
2015. Vol.~1. P.~1681--1691.

\bibitem{vbta} %14
\Au{Kuznetsova~R., Bakhteev~O., Ogaltsov~A.} Variational learning across 
domains with triplet information~// 3rd Workshop on Bayesian Deep Learning.~--- Montreal, Canada. 
{\sf http://bayesiandeeplearning.org/2018/papers/65.pdf}.


\bibitem{ann}
\Au{Wang J.,  Shen~H.,  Song~J., Ji~J.} Hashing for similarity search: 
A~survey~// arXiv.org, 2014. 29~p. \mbox{arXiv}:1408.2927 [cs.DS].


\bibitem{ae}
\Au{Alain G., Bengio~Y.} What regularized auto-encoders learn from the data-generating 
distribution~// J.~Mach. Learn. Res., 2014. 
Vol.~15. No.\,1. P.~3563--3593.


\bibitem{sphere_code}
\Au{Jenssen M., Joos~F., Perkins~W.} On kissing numbers and spherical codes 
in high dimensions~// Adv.  Math., 2018. Vol.~335. P.~307--321.


\bibitem{cybenko}
\Au{Cybenko G.} Approximation by superpositions of a~sigmoidal function~// 
Math. Control Signal., 1989. Vol.~2. No.\,4. P.~303--314.

\bibitem{dataset}
Синтетическая выборка для задачи обнаружения переводных заимствований. 
{\sf https://tiny.cc/cl\_ru\_en}.


\bibitem{ft}
\Au{Bojanowski P., Grave~E., Joulin~A., Mikolov~T.} Enriching word vectors 
with subword information~// Transactions Association for Computational 
Linguistics, 2017.  Vol.~5. P.~135--146.


\bibitem{gru}
\Au{Chung J.,  Gulcehre~C.,  Cho~K.,  Bengio~Y.} Empirical evaluation of 
gated recurrent neural networks on sequence modeling~// arXiv.org, 2014. 9~p.
arXiv:1412.3555 [cs.NE].

\bibitem{opus} %22
\Au{Tiedemann~J.} News from OPUS~--- a~collection of multilingual parallel 
corpora with tools and interfaces~// Advances in natural language 
processing.~--- Amsterdam/Philadelphia: John Benjamins, 2009. Vol.~5. P.~237--248.
\end{thebibliography}

}
}

\end{multicols}

\vspace*{-3pt}

\hfill{\small\textit{Поступила в~редакцию 19.03.2020}}

\vspace*{8pt}

%\pagebreak

%\newpage

%\vspace*{-28pt}

\hrule

\vspace*{2pt}

\hrule

%\vspace*{-2pt}

\def\tit{METHODS OF CROSS-LINGUAL TEXT REUSE DETECTION IN~LARGE TEXTUAL COLLECTIONS}

\def\titkol{Methods of cross-lingual text reuse detection in~large textual collections}

\def\aut{R.\,V.~Kuznetsova$^1$, O.\,Yu.~Bakhteev$^{1,2}$, and~Yu.\,V.~Chekhovich$^3$}

\def\autkol{R.\,V.~Kuznetsova, O.\,Yu.~Bakhteev, and~Yu.\,V.~Chekhovich}

\titel{\tit}{\aut}{\autkol}{\titkol}

\vspace*{-11pt}


\noindent
$^1$Moscow Institute of Physics and Technology, 9~Institutskiy Per., 
Dolgoprudny, Moscow Region 141700, Russian\linebreak
$\hphantom{^1}$Federation

\noindent
$^2$Antiplagiat Co., 42-1 Bolshoy Blvd., Moscow 121205, Russian Federation

\noindent
$^3$A.\,A.~Dorodnicyn Computing Center, Federal Research Center ``Computer Science and Control''
 of the Russian\linebreak
 $\hphantom{^1}$Academy of Sciences, 40~Vavilov Str., Moscow 119333, Russian Federation
 
 
\def\leftfootline{\small{\textbf{\thepage}
\hfill INFORMATIKA I EE PRIMENENIYA~--- INFORMATICS AND
APPLICATIONS\ \ \ 2021\ \ \ volume~15\ \ \ issue\ 1}
}%
\def\rightfootline{\small{INFORMATIKA I EE PRIMENENIYA~---
INFORMATICS AND APPLICATIONS\ \ \ 2021\ \ \ volume~15\ \ \ issue\ 1
\hfill \textbf{\thepage}}}

\vspace*{3pt}




\Abste{The paper investigates the cross-lingual text reuse detection problem. 
The paper proposes a~monolingual approach to this problem: to translate 
the suspicious document into the language of the collection for the further monolingual analysis. 
One of the major requirements for the proposed method is robustness to the machine translation ambiguity. 
The further document analysis is divided into two steps. 
At the first step, the authors retrieve documents-candidates which are likely to be the source 
of the text reuse. For the robustness, the authors propose to retrieve the documents 
using word clusters that are constructed using distributional semantics. 
At the second\linebreak\vspace*{-12pt}}

\Abstend{step, the authors compare the suspicious document with candidates 
using sentence embeddings that are obtained by deep learning neural networks.
 The experiment was conducted for the ``English--Russian'' language pair both on the synthetic
  data and on the articles included in the Russian Science Citation Index.}
  
  \KWE{natural language processing; machine translation; deep learning; cross-lingual text 
  reuse detection; distributional semantics}




\DOI{10.14357/19922264210105}

%\vspace*{-15pt}

\Ack
\noindent
This research was supported by RFBR (project 18-07-01441) and Foundation for Assitance to Small Innovative
Enterprises in Science and Technology (project 44116).

\vspace*{12pt}

  \begin{multicols}{2}

\renewcommand{\bibname}{\protect\rmfamily References}
%\renewcommand{\bibname}{\large\protect\rm References}

{\small\frenchspacing
 {%\baselineskip=10.8pt
 \addcontentsline{toc}{section}{References}
 \begin{thebibliography}{99}
\bibitem{1-ce}
\Aue{Nikitov, A.\,V., O.\,A.~Orchakov, and Y.\,V.~Chekhovich.}
 2012. Plagiat v~rabotakh studentov i~aspirantov: problema i~metody protivodeystviya 
 [Plagiarism in papers of students and graduate students: The problem and methods of counteraction].
 \textit{Universitetskoe upravlenie: praktika i~analiz}
  [University Management: Practice and Analysis] 5:61--68.
\bibitem{2-ce}
\Aue{Khritankov, A.\,S., P.\,V.~Botov, N.\,S.~Surovenko, S.\,V.~Tsarkov, D.\,V.~Viuchnov, and 
Y.\,V.~Chekhovich.}
 2015. Discovering text reuse in large collections of documents: 
 A~study of theses in history sciences. 
 \textit{Artificial Intelligence and Natural Language and Information Extraction, 
 Social Media and Web Search FRUCT Conference Proceedings}. IEEE. 26--32.
\bibitem{3-ce}
\Aue{Zelenkov, I.\,V., and I.\,V.~Segalovich.} 
2007. Sravnitel'nyy analiz metodov opredeleniya nechetkikh dublikatov dlya 
Web-dokumentov [Comparative analysis of methods for determining fuzzy duplicates for Web-documents]. 
\textit{Tr. 9-y Vseross. nauchn. konf. ``Elektronnye biblioteki: perspektivnye metody i~tekhnologii, 
elektronnye kollektsii''} [9th All-Russian Scientific Conference ``Digital libraries: 
Advanced Methods and Technologies, Electronic Collections'' Proceedings]. Pereslavl-Zalessky:
Pereslavl-Zalessky University. 166--174.
\bibitem{4-ce}
\Aue{Franco-Salvador, M., P.~Gupta, and P.~Rosso.}
 2013. Cross-language plagiarism detection using a~multilingual semantic network. 
 \textit{European Conference on Information Retrieval}. 
 Eds. P.~Serdyukov, P.~Braslavski, S.\,O.~Kuznetsov, \textit{et al}. Lecture notes in computer science ser. 
 Berlin--Heidelberg: Springer. 7814:710--713.
\bibitem{5-ce}
\Aue{Franco-Salvador, M., P.~Gupta., P.~Rosso, and R.\,E.~Banchs.}
 2016. Cross-language plagiarism detection over continuous-space-and knowledge graph-based 
 representations of language. \textit{Knowl.-Based Syst.} 111:87--99.
\bibitem{6-ce}
\Aue{Grman, J., and R.~Ravas.}
 2011. Improved implementation for finding text similarities in large collections of data.
 \textit{Notebook papers of CLEF 2011 Labs and Workshops}.  Eds. V.~Petras, P.~Forner, and P.\,D.~Clough. 1177. 6~p.
 {\sf http://ceur-ws.org/Vol-1177/CLEF2011wn-PAN-GrmanEt2011.pdf} (accessed January~18, 2021).
\bibitem{7-ce}
\Aue{Grozea, C., and M.~Popescu.} 2011. 
The encoplot similarity measure for automatic detection of plagiarism. 
\textit{Notebook papers of CLEF 2011 Labs and Workshops}.  Eds. V.~Petras, P.~Forner, and P.\,D.~Clough.
Amsterdam, The Netherlands. 1177. Available at: 
{\sf http://ceur-ws.org/\linebreak  Vol-1177/CLEF2011wn-PAN-GrozeaEt2011.pdf} (accessed January~18, 2021).
\bibitem{8-ce}
\Aue{Muhr, M., R.~Kern, M.~Zechner, and M.~Granitzer.}
 2010. External and intrinsic plagiarism detection using 
 a~cross-lingual retrieval and segmentation system. 
 \textit{Notebook paper of CLEF 2010 Labs and Workshops}.
Eds. M.~Braschler, D.~Harman, and  E.~Pianta. Padua, Italy. 1176. 
 Available at: {\sf http://ceur-ws.org/Vol-1176/CLEF2010wn-PAN-MuhrEt2010.pdf} (accessed January~18, 2021).
\bibitem{9-ce}
\Aue{Bakhteev, O., R.~Kuznetsova, A.~Romanov, and A.~Khritankov.}
 2015. A~monolingual approach to detection of text reuse in Russian--English collection. 
 \textit{Artificial Intelligence and Natural Language and Information Extraction, 
 Social Media and Web Search FRUCT Conference Proceedings}. IEEE. 3--10.
\bibitem{10-ce}
\Aue{Koehn, P.,  Hien Hoang, A.~Birch, \textit{et al.}} 2007. Moses: 
Open source toolkit for statistical machine translation. 
\textit{45th Annual Meeting of the Association for Computational Linguistics 
Companion Volume Proceedings of the Demo and Poster Sessions Proceedings}. ACL. 177--180.
\bibitem{11-ce}
\Aue{Tai, K.\,S., R.~Socher, and C.\,D.~Manning.}
 2015. Improved semantic representations from tree-structured long short-term memory networks. 
 \textit{53rd Annual Meeting of the Association for Computational Linguistics and the 
 7th  Joint Conference (International) on Natural Language Processing Proceedings}. ACL. 1:1556--1566.
\bibitem{12-ce}
\Aue{Wieting, J., M.~Bansal, K.~Gimpel, and K.~Livescu.} 2015.
 Towards universal paraphrastic sentence embeddings. 19~p. 
 Available at: {\sf https://arxiv.org/abs/1511.08198} (accessed January~18, 2021).
\bibitem{13-ce}
\Aue{Iyyer, M., V.~Manjunatha, J.~Boyd-Graber, and H.~Daum$\acute{\mbox{e}}$.} 
2015. Deep unordered composition rivals syntactic methods for text classification. 
\textit{53rd Annual Meeting of the Association for Computational Linguistics and the
 7th  Joint Conference (International) on Natural Language Processing Proceedings}. ACL. 1:1681--1691.
\bibitem{14-ce}
\Aue{Kuznetsova, R., O.~Bakhteev, and A.~Ogaltsov.}
 2018. Variational learning across domains with triplet information. 
 \textit{3rd Workshop on Bayesian Deep Learning Proceedings}. 
 Available at: {\sf http://bayesiandeeplearning.org/2018/\linebreak papers/65.pdf} (accessed January~18, 2021).
\bibitem{15-ce}
\Aue{Wang, J., H.\,T.~Shen, J.~Song, and J.~Ji.}
 2014. Hashing for similarity search: A~survey. 29~p. Available at: 
 {\sf https:// arxiv.org/abs/1408.2927} (accessed January~18, 2021).
\bibitem{16-ce}
\Aue{Alain, G., and Y.~Bengio.} 
2014. What regularized auto-encoders learn from the data-generating distribution. 
\textit{J.~Mach. Learn. Res.} 15(1):3563--3593.
\bibitem{17-ce}
\Aue{Jenssen, M., F.~Joos, and W.~Perkins.} 2018. 
On kissing numbers and spherical codes in high dimensions. \textit{Adv. Math.} 335:307--321.
\bibitem{18-ce}
\Aue{Cybenko, G.}
 1989. Approximation by superpositions of a~sigmoidal function. 
 \textit{Math. Control Signal.} 2(4):303--314.
\bibitem{19-ce}
Sinteticheskaya vyborka dlya zadachi obnaruzheniya perevodnykh zaimstvovaniy
 [Synthetic dataset for the cross-lingual text reuse detection problem]. 
 Available at: {\sf https://tiny.cc/cl\_ru\_en} (accessed January~18, 2021).
\bibitem{20-ce}
\Aue{Bojanowski, P., E.~Grave, A.~Joulin, and T.~Mikolov.}
 2017. Enriching word vectors with subword information. 
 \textit{Transactions Association for Computational Linguistics} 5:135--146.
\bibitem{21-ce}
\Aue{Chung, J., C.~Gulcehre, K.~Cho, and Y.~Bengio.}
 2014. Empirical evaluation of gated recurrent neural networks on sequence modeling. 9~p. 
 Available at: {\sf https://\linebreak arxiv.org/abs/1412.3555} (accessed January~18, 2021).
\bibitem{22-ce}
\Aue{Tiedemann, J.}
 2009. News from OPUS~-- a~collection of multilingual parallel corpora with tools and interfaces. 
 \textit{Advances in natural language 
processing}. Amsterdam/Philadelphia: John Benjamins. 5:237--248.
 \end{thebibliography}

 }
 }

\end{multicols}

\vspace*{-3pt}

  \hfill{\small\textit{Received March~19, 2020}}


%\pagebreak

%\vspace*{-8pt}     

\Contr

\noindent
\textbf{Kuznetsova Rita V.} (b.\ 1990)~--- 
PhD student, Moscow Institute of Physics and Technology, 
9~Institutskiy Per., Dolgoprudny, Moscow Region 141701, Russian Federation; 
\mbox{rita.kuznetsova@phystech.edu}

\vspace*{3pt}

\noindent
\textbf{Bakhteev Oleg Yu.} (b.\ 1993)~--- assistant professor, Moscow Institute of Physics and Technology, 
9~Institutskiy Per., Dolgoprudny, Moscow Region 141701, Russian Federation; Head of Research Department, 
Antiplagiat Co., 42-1~Bolshoy Blvd., Moscow 121205, Russian Federation;
\mbox{bakhteev@ap-team.ru}

\vspace*{3pt}

\noindent
\textbf{Chekhovich Yury V.} (b.\ 1976)~--- 
Candidate of Science (PhD) in physics and mathematics, 
Head of Department, A.\,A.~Dorodnicyn Computing Center, 
Federal Research Center ``Computer Science and Control'' of the Russian Academy of Sciences, 
40~Vavilov Str., Moscow 119333, Russian Federation; \mbox{chehovich@ap-team.ru}

\label{end\stat}

\renewcommand{\bibname}{\protect\rm Литература}    %5
\def\stat{strijov}

\def\tit{ОПРЕДЕЛЕНИЕ РЕЛЕВАНТНОСТИ ПАРАМЕТРОВ НЕЙРОСЕТИ$^*$}

\def\titkol{Определение релевантности параметров нейросети}

\def\aut{А.\,В.~Грабовой$^1$, О.\,Ю.~Бахтеев$^2$, В.\,В.~Стрижов$^3$}

\def\autkol{А.\,В.~Грабовой, О.\,Ю.~Бахтеев, В.\,В.~Стрижов}

\titel{\tit}{\aut}{\autkol}{\titkol}

\index{Грабовой А.\,В.}
\index{Бахтеев О.\,Ю.}
\index{Стрижов В.\,В.}
\index{Grabovoy A.\,V.}
\index{Bakhteev O.\,Yu.}
\index{Strijov V.\,V.}


{\renewcommand{\thefootnote}{\fnsymbol{footnote}} \footnotetext[1]
{Работа выполнена 
при поддержке РФФИ (проект 19-07-0875) и~Правительства РФ (соглашение 
05.Y09.21.0018). Настоящая статья содержит результаты проекта <<Статистические 
методы машинного обучения>>, выполняемого в~рамках реализации Программы Центра 
компетенций Национальной технологической инициативы <<Центр хранения и~анализа 
больших данных>>, поддерживаемого Министерством науки и~высшего образования 
Российской Федерации по Договору МГУ им.\ М.\,В.~Ломоносова  с~Фондом поддержки 
проектов Национальной технологической инициативы от 11.12.2018 №\,13/1251/2018.}}


\renewcommand{\thefootnote}{\arabic{footnote}}
\footnotetext[1]{Московский физико-технический институт, 
\mbox{grabovoy.av@phystech.edu}}
\footnotetext[2]{Московский физико-технический 
институт, \mbox{bakhteev@phystech.edu}}
\footnotetext[3]{Вычислительный центр им.\ А.\,А.~Дородницына Федерального 
исследовательского центра <<Информатика и~управление>> Российской академии наук; 
Московский фи\-зи\-ко-тех\-ни\-че\-ский институт, \mbox{strijov@ccas.ru}}


%\vspace*{-2pt}


\Abst{Работа посвящена оптимизации структуры нейронной сети. 
Предполагается, что чис\-ло па\-ра\-мет\-ров нейросети можно существенно снизить без 
значимой потери качества и~значимого повышения дис\-пер\-сии функции ошиб\-ки. 
Предлагается метод прореживания па\-ра\-мет\-ров нейронной сети, основанный на 
автоматическом определении релевантности па\-ра\-мет\-ров. Для определения 
релевантности па\-ра\-мет\-ров предлагается проанализировать ковариационную мат\-ри\-цу 
апостериорного распределения параметров и~удалить из нейросети 
мультикоррелирующие па\-ра\-мет\-ры. Для определения мультикорреляции используется 
метод Белсли. Для анализа качества представленного алгоритма проводятся 
эксперименты на выборке Boston Housing, а также на синтетических данных.}
    
\KW{нейронные сети; оптимизация гиперпараметров; метод 
Белсли; релевантность па\-ра\-мет\-ров; прореживание нейронной сети}

\DOI{10.14357/19922264190209}
  
%\vspace*{4pt}


\vskip 10pt plus 9pt minus 6pt

\thispagestyle{headings}

\begin{multicols}{2}

\label{st\stat}


\section{Введение}

Решается задача выбора оптимальной структуры нейронной сети. В~силу высокой 
вычислительной сложности время оптимизации нейронных сетей может занимать до 
нескольких дней~\cite{sutskever2014}. Поэтому по\-стро\-ение и~выбор оптимальной 
структуры нейронной сети так\-же является вычислительно слож\-ной процедурой, 
которая значимо влияет на итоговое качество модели. Использование избыточно 
слож\-ных моделей с~избыточным чис\-лом неинформативных параметров служит 
препятствием для использования глубоких сетей на мобильных устройствах в~режиме 
реального времени.

Существует ряд подходов к~построению оптимальной сети. 
В~работах~\cite{maclarin2015, luketina2015} предлагается ис\-пользовать модель 
градиентного спуска для оптимизации сети. В~\cite{molchanov2017} используются 
байесовские методы~\cite{neal1995} оптимизации па\-ра\-мет\-ров нейронных сетей.  Еще 
один метод поиска оптимальной структуры за\-клю\-ча\-ет\-ся в~прореживании избыточно 
слож\-ной модели~\cite{cun1990, louizos2017, graves2011}. В~работе~\cite{cun1990} 
предлагается\linebreak
 удалять наименее релевантные па\-ра\-мет\-ры на основе значений первой 
и~второй производных функции \mbox{ошибки}.

Данная работа посвящена прореживанию структуры сети. Предлагается удалять 
наименее релевантные па\-ра\-мет\-ры модели. Под ре\-ле\-вант\-ностью~\cite{cun1990} 
подразумевается то, насколько сильно параметр влияет на функцию ошиб\-ки. Малая 
ре\-ле\-вант\-ность указывает на то, что удаление этого па\-ра\-мет\-ра не влечет значимого 
изменения функции ошиб\-ки. Метод предлагает по\-стро\-ение исходной избыточно сложной 
нейросети с~большим числом избыточных па\-ра\-мет\-ров. Для определения ре\-ле\-вант\-ности 
па\-ра\-мет\-ров предлагается оптимизировать параметры и~гиперпараметры в~единой 
процедуре. Для удаления па\-ра\-мет\-ров предлагается использовать метод 
Белсли~\cite{neychev2016} .

Проверка и~анализ метода проводятся на выборке Boston Housing~\cite{Boston}, 
Wine~\cite{Wine} и~синтетических данных. Результат сравнивается с~моделью, 
полученной при помощи базовых алгоритмов.

\vspace*{-9pt}

\section{Постановка задачи}

Задана выборка
\begin{equation*}
\mathfrak{D} = \{\mathbf{x}_i,y_i\},\enskip  i =1,\ldots,N\,, 
%\label{e2.1-str}
\end{equation*}
где~$\mathbf{x}_i \in \mathbb{R}^{m}$, $y_i \hm\in \{1, \ldots, Y\}$,~$Y$~--- число 
классов.
Рассмотрим модель~$f(\mathbf{x}, \mathbf{w}): \mathbb{R}^m \times \mathbb{R}^n 
\to \{1,\ldots,Y\}$, где~$\mathbf{w}\hm \in \mathbb{R}^n$~--- пространство 
параметров модели,
\begin{equation*}
f(\mathbf{x}, \mathbf{w}) = \mathrm{softmax}
\left( f_1\left( f_2\left( \cdots \left(f_l(\mathbf{x}, 
\mathbf{w})\right)\right)\right)\right)\,, 
%\label{e2.2-str}
\end{equation*}
где~$f_k(\mathbf{x}, \mathbf{w}) \hm=  
\mathrm{tanh}(\mathbf{w}^\mathsf{T}\mathbf{x})$,
 $k \hm\in \{1,\ldots ,l\}$; $l$~--- чис\-ло слоев нейронной 
сети.
Параметр~$w_j$ модели~$f$\linebreak\vspace*{-12pt}

\pagebreak

\noindent
  называется активным, если~$w_j \not = 0$. Множество 
индексов активных па\-ра\-мет\-ров обозначим~$\mathcal{A} \hm\subset \mathcal{J} \hm= 
\{1,\ldots ,n\}$.
Задано пространство па\-ра\-мет\-ров модели
\begin{equation*}
\mathbb{W_\mathcal{A}} = \{ \textbf{w} \in \mathbb{R}^n~|~w_j\not=0,\enskip j \in 
\mathcal{A}  \}. %\label{e2.3-str}
\end{equation*}


Для модели~$f$ с~множеством индексов активных па\-ра\-мет\-ров~$\mathcal{A}$ 
и~соответствующего ей вектора па\-ра\-мет\-ров~$\mathbf{w} \hm\in \mathbb{W_\mathcal{A}}$  
определим логарифмическую функцию правдоподобия выборки:
\begin{equation}
\mathcal{L}_\mathfrak{D}(\mathfrak{D}, \mathcal{A}, \mathbf{w}) = \log 
p(\mathfrak{D}|\mathcal{A}, \mathbf{w}), \label{e2.4-str}
\end{equation}
где~$p(\mathfrak{D}|\mathcal{A},\textbf{w})$~--- апостериорная вероятность 
выборки~$\mathfrak{D}$ при заданных~$\mathbf{w}, \mathcal{A}$.
Оптимальные значения~$\mathbf{w},\mathcal{A}$ находятся из 
минимизации~$-\mathcal{L}_{\mathcal{A}}(\mathfrak{D},\mathcal{A})$~--- логарифма правдоподобия 
модели:
\begin{multline}
\mathcal{L}_{\mathcal{A}}(\mathfrak{D},\mathcal{A}) =\log 
p(\mathfrak{D}|\mathcal{A}) = {}\\
{}=\log  \int\limits_{{\mathbf{w}\in\mathbb{W_\mathcal{J}}}}
p(\mathfrak{D} | \mathbf{w}) p(\mathbf{w} | \mathcal{A})\, d \mathbf{w}, 
\label{e2.5-str}
\end{multline}
где~$p(\mathbf{w}|\mathcal{A})$~---  априорная вероятность век\-то\-ра па\-ра\-мет\-ров 
в~пространстве~$\mathbb{W_\mathcal{J}}$.

Так как вычисление интеграла~(\ref{e2.5-str}) является вы\-чис\-ли\-тель\-но слож\-ной задачей, 
рас\-смот\-рим вариационный подход~\cite{bishop2006} для решения этой задачи. Пусть 
задано распределение:
\begin{equation*}
q(\mathbf{w})\sim \mathcal{N}\left(\mathbf{m}, \mathbf{A}^{-1}_{\mathrm{ps}}\right). 
%\label{e2.6-str}
\end{equation*}
Здесь~$\mathbf{m}, \mathbf{A}^{-1}_{\mathrm{ps}}$~--- вектор средних и~матрица 
ковариации, аппроксимирующее неизвестное апостериорное 
распределение~$p(\mathbf{w}|\mathfrak{D},\mathcal{A})$, полученное при априорном 
предположении
\begin{equation*}
p(\mathbf{w} | \mathcal{A})\sim \mathcal{N}
\left(\boldsymbol{\mu},\mathbf{A}^{-1}_{\mathrm{pr}}\right), 
%\label{e2.7-str}
\end{equation*}
где~$\boldsymbol{\mu},\mathbf{A}^{-1}_{\mathrm{pr}}$~--- вектор средних и~матрица 
ковариации априорного распределения.

Приблизим интеграл~(\ref{e2.5-str}) методом, предложенным в~\cite{bishop2006}:
\begin{multline*}
\mathcal{L}_{\mathcal{A}}(\mathfrak{D},\mathcal{A}) = \log 
p(\mathfrak{D}|\mathcal{A}) = {}\\
{} =\int\limits_{\mathbf{w}\in\mathbb{W_\mathcal{J}}} q(\mathbf{w}) \log 
\fr{p(\mathfrak{D}, \mathbf{w}|\mathcal{A})}{q(\mathbf{w})}\,d \mathbf{w} - {}\\
{}-
\int\limits_{\mathbf{w}\in\mathbb{W_\mathcal{J}}}  q(\mathbf{w}) \log 
\fr{p(\mathbf{w}|\mathfrak{D},\mathcal{A})}{q(\mathbf{w})}\,d \mathbf{w} \approx {}\\
{}\approx \int\limits_{\mathbf{w}\in\mathbb{W_\mathcal{J}}} q(\mathbf{w}) \log 
\fr{p(\mathfrak{D}, \mathbf{w}|\mathcal{A})}{q(\mathbf{w})}\,d \mathbf{w} = {}\\[-20pt]
\end{multline*}

\noindent
\begin{multline}
{}= \int\limits_{\mathbf{w}\in\mathbb{W_\mathcal{J}}} q(\mathbf{w}) \log 
\fr{p(\mathbf{w}| \mathcal{A})}{q(\mathbf{w})}\,d \mathbf{w} + {}\\
{}+
\int\limits_{\mathbf{w}\in\mathbb{W_\mathcal{J}}} q(\mathbf{w}) \log 
p(\mathfrak{D}|\mathcal{A}, \mathbf{w})\,d \mathbf{w}={}\\
{}=\mathcal{L}_\mathbf{w}(\mathfrak{D}, \mathcal{A}, 
\mathbf{w})+\mathcal{L}_{E}(\mathfrak{D},\mathcal{A})\,.
\label{e2.8-str}
\end{multline}

Первое слагаемое формулы~(\ref{e2.8-str})~--- это слож\-ность модели. Оно определяется 
расстоянием Куль\-ба\-ка--Лейб\-лера:
\begin{equation*}
\mathcal{L}_\mathbf{w}(\mathfrak{D}, \mathcal{A}, \mathbf{w}) = -
D_{\mathrm{KL}}\left(q(\textbf{w})||p(\textbf{w}|\mathcal{A})\right). 
%\label{e2.9-str}
\end{equation*}
Второе слагаемое формулы~(\ref{e2.8-str}) пред\-став\-ля\-ет собой математическое
ожидание правдоподобия 
выборки~$\mathcal{L}_\mathfrak{D}(\mathfrak{D},\mathcal{A}, \mathbf{w})$. 
В~данной работе оно является функцией ошибки:
\begin{equation*}
\mathcal{L}_{E}(\mathfrak{D},\mathcal{A}) = \mathsf{E}_{\mathbf{w}\sim  q}
\mathcal{L}_\mathfrak{D}(\mathbf{y}, \mathfrak{D}, \mathcal{A}, \mathbf{w}). 
%\label{e2.10-str}
\end{equation*}

Требуется найти параметры, достав\-ля\-ющие минимум суммарному функционалу 
потерь $\mathcal{L}_\mathcal{A}(\mathfrak{D},\mathcal{A},\mathbf{w})$ из~(\ref{e2.8-str}):
\begin{multline}
\hat{\mathbf{w}} =\! \argmin_{\mathcal{A}\subset\mathcal{J}, \mathbf{w} \in 
\mathbb{W_\mathcal{A}}} \!-\mathcal{L}_\mathcal{A}(\mathfrak{D}, \mathcal{A}, 
\mathbf{w}) = {}\\
\!\!{}=\!\!\argmin_{\mathcal{A}\subset\mathcal{J}, \mathbf{w} \in 
\mathbb{W_\mathcal{A}}} \!\!
D_{\mathrm{KL}}\left(q(\textbf{w})||p(\textbf{w}|\mathcal{A})\right) - 
\mathcal{L}_\mathfrak{D}(\mathfrak{D}, \mathcal{A}, \mathbf{w}). 
\label{e2.11-str}
\end{multline}



\section{Базовые методы прореживания нейросети}

\subsection{Случайное удаление}

Метод случайного удаления заключается в~том, что случайным образом удаляется 
некоторый параметр~$w_\xi$ из множества активных па\-ра\-мет\-ров сети.  Индекс 
па\-ра\-мет\-ра~$\xi$ из равномерного распределения~--- случайная величина, 
предположительно до\-став\-ля\-ющая оптимум в~(\ref{e2.11-str}):
\begin{equation*}
\xi \sim \mathcal{U}(\mathcal{A}). 
%\label{e3.1.1-str}
\end{equation*}

\subsection{Оптимальное прореживание}

Метод оптимального прореживания~\cite{cun1990} использует вторую производную 
целевой функции~(\ref{e2.4-str}) по па\-ра\-мет\-рам для выявления нерелевантных 
па\-ра\-мет\-ров. 
Рас\-смот\-рим функцию потерь~$\mathcal{L}$~(\ref{e2.4-str}), разложенную в~ряд Тейлора 
в~некоторой окрест\-ности вектора па\-ра\-мет\-ров~$\mathbf{w}$:
\begin{equation}
\delta \mathcal{L} = \!\!\sum\limits_{j\in \mathcal{A}} \!g_j\delta w_j + 
\fr{1}{2}\!\sum\limits_{i,j\in \mathcal{A}} \! h_{ij}\delta w_i\delta w_j + 
O\left(||\delta\mathbf{w}||^3\right)\!,
\!\! \label{e3.2.1-str}
\end{equation}
где~$\delta w_j~$~--- компоненты вектора~$\delta\mathbf{w}$; $g_j$~--- 
компоненты вектора градиента~$\nabla \mathcal{L}$; $h_{ij}$~--- компоненты 
гесcиана~$\mathbf{H}$:

\noindent
\begin{equation*}
g_j = \fr{\partial \mathcal{L}}{\partial w_j}\,, \qquad h_{ij} = 
\fr{\partial^2\mathcal{L}}{\partial w_i \partial w_j}\,. 
%\label{e3.2.2-str}
\end{equation*}

Задача является вычислительно слож\-ной в~силу высокой раз\-мер\-ности мат\-ри\-цы 
\textbf{H}. Введем предположение~\cite{cun1990} о~том, что удаление нескольких 
па\-ра\-мет\-ров приводит к~такому же изменению функции потерь~$\mathcal{L}$, как 
и~суммарное изменение при индивидуальном удалении:

\vspace*{2pt}

\noindent
\begin{equation*}
\delta \mathcal{L} = \sum\limits_{j\in \mathcal{A}} \delta \mathcal{L}_j, 
%\label{e3.2.3-str}
\end{equation*}

\vspace*{-2pt}

\noindent
где~$\mathcal{A}$~--- множество активных параметров; $\delta\mathcal{L}_j$~--- 
изменение функции потерь при удалении одного па\-ра\-мет\-ра~$\mathbf{w}_j$.

В силу данного предположения будем рас\-смат\-ри\-вать только диагональные элементы 
мат\-ри\-цы~\textbf{H}. После введенного предположения 
выражение~(\ref{e3.2.1-str}) принимает 
вид:

\vspace*{-1pt}

\noindent
\begin{equation*}
\delta \mathcal{L} = \fr{1}{2} \sum\limits_{j\in \mathcal{A}} h_{jj}\delta w_j^2. 
%\label{e3.2.4-str}
\end{equation*}

\vspace*{-2pt}

Получаем следующую задачу оптимизации:
\begin{equation*}
\xi = \argmin\limits_{j\in \mathcal{A}} h_{jj}\fr{w_j^2}{2}\,, 
%\label{e3.2.5-str}
\end{equation*}
где~$\xi$~--- индекс наименее релевантного, уда\-ля\-емо\-го па\-ра\-мет\-ра, 
предположительно до\-став\-ля\-юще\-го оптимум в~(\ref{e2.11-str}).

\vspace*{-4pt}

\subsection{Удаление неинформативных параметров с~помощью вариационного вывода}

\vspace*{-2pt}

Для удаления параметров в~работе~\cite{graves2011} предлагается удалить 
па\-ра\-мет\-ры, которые имеют максимальное отношение 
плот\-ности~$p(\mathbf{w}|\mathcal{A})$ априорной вероятности в~нуле к~плотности 
априорной ве\-ро\-ят\-ности в~математическом ожидании~$\mu_j$ па\-ра\-мет\-ра~$w_j$.

Для гауссовского распределения с~диагональной матрицей ковариации получаем:
\begin{equation*}
p_j(\mathbf{w}|\mathcal{A})(w) = \fr{1}{\sqrt[]{2\sigma_j^2}}\exp\left({-
\fr{(w-\mu_j)^2}{2\sigma_j^2}}\right), %\label{e3.3.1-str}
\end{equation*}
где $w$~--- значение носителя распределенного параметра.
Разделим плотность вероятности в~нуле к~плот\-ности в~математическом ожидании
\begin{equation*}
\fr{p_j(\mathbf{w}|\mathcal{A})(0)}{p_j(\mathbf{w}|\mathcal{A})(\mu_j)}= 
\exp\left({-\fr{\mu_j^2}{2\sigma_j^2}}\right) 
%\label{e3.3.2-str}
\end{equation*}
и поставим следующую задачу оптимизации:
\begin{equation*}
\xi = \argmin\limits_{j\in \mathcal{A}} \left\vert \fr{\mu_j}{\sigma_j}\right\vert\,, 
%\label{e3.3.3-str}
\end{equation*}

\vspace*{-8pt}

\columnbreak

\noindent
где~$\xi$~--- индекс наименее релевантного, удаляемого па\-ра\-метра.

\vspace*{-8pt}

\section{Предлагаемый метод определения релевантности параметров нейросети}

\vspace*{-2pt}

Предлагается метод, основанный на модификации метода Белсли. Пусть~$\mathbf{w}$~--- 
вектор параметров, доставляющий минимум функционалу потерь 
$\mathcal{L}_\mathcal{A}$ из~(\ref{e2.8-str}) на  множестве $\mathbb{W_\mathcal{A}}$, 
а~$\mathbf{A}_{\mathrm{ps}}$~--- соответствующая ему ковариационная матрица.

Выполним сингулярное разложение мат\-рицы

\vspace*{3pt}

\noindent
\begin{equation*}
\mathbf{A}_{\mathrm{ps}} = \mathbf{U}{\bf\Lambda}\mathbf{V}^{\mathsf{T}}. 
%\label{e4.1-str}
\end{equation*}

\vspace*{-2pt}

\noindent
Индекс обусловленности~$\eta_{j}$ определим как отношение максимального 
элемента к~$j$-му элементу мат\-ри\-цы~$\mathbf{\Lambda}$. Для
 нахождения муль\-ти\-кор\-ре\-ли\-ру\-ющих 
признаков требуется найти индекс~$\xi$ вида

\vspace*{2pt}

\noindent
\begin{equation*}
\xi = \argmax\limits_{j\in \mathcal{A}}{\eta_j}\,. 
%\label{e4.2-str}
\end{equation*}

\vspace*{-2pt}


Дисперсионный долевой коэффициент~$q_{ij}$ определим как вклад $j$-го признака 
в~дис\-пер\-сию $i$-го элемента вектора па\-ра\-мет\-ра~$\mathbf{w}$:

\vspace*{3pt}

\noindent
\begin{equation*}
q_{ij} = \fr{u^2_{ij}/\lambda_{jj}}{\sum\nolimits^n_{j=1}{u^2_{ij}/\lambda_{jj}}}\,. 
%\label{e4.3-str}
\end{equation*}

\vspace*{-2pt}

Большие значения дисперсионных долей указывают на наличие зависимости между 
параметрами. Находим долевые коэффициенты, которые вносят максимальный вклад 
в~дис\-пер\-сию па\-ра\-мет\-ра~$w_\xi$:

\vspace*{2pt}

\noindent
\begin{equation*}
\zeta = \argmax\limits_{j\in \mathcal{A}} q_{\xi j}\,. 
%\label{e4.4-str}
\end{equation*}

\vspace*{-2pt}

\noindent
Параметр с~индексом~$\zeta$ определим как наименее релевантный параметр 
нейросети.
%Для удаления нескольких зависимых параметров за раз, предлагается удалить 
%параметры с~номерами $p \in \mathcal{A}: q_{\xi i} > \lambda  \in  
%\mathbb{R}_+$.




Проиллюстрируем принцип работы метода Белсли на примере. Рас\-смот\-рим данные, 
по\-рож\-ден\-ные сле\-ду\-ющим образом:

\vspace*{2pt}

\noindent
$$
\mathbf{w} = \begin{bmatrix}
\sin(x)\\
\cos (x)\\
2+\cos(x)\\
2+\sin(x)\\
\cos(x) + \sin(x)\\
x
\end{bmatrix}\,,
$$

\vspace*{-2pt}


\noindent
с матрицей ковариации, пред\-став\-лен\-ной на рис.~\ref{CovBel},\,\textit{а}, где $x \hm\in 
[0{,}0; 0{,}02; \ldots; 20{,}0]$.

В табл.~\ref{CovBelTable} приведены индексы обуслов\-лен\-ности и~соответствующие им 
дис\-пер\-си\-он\-ные доли, ко-\linebreak\vspace*{-12pt}

\pagebreak

\end{multicols}

\begin{figure*} %fig1
  \vspace*{1pt}
    \begin{center}  
  \mbox{%
 \epsfxsize=161.599mm 
 \epsfbox{str-1.eps}
 }
 \end{center}
\vspace*{-11pt}
\Caption{Иллюстрация метода Белсли: (\textit{a})~мат\-ри\-ца ковариации; 
(\textit{б})~дис\-пер\-си\-он\-ные доли}
\label{CovBel}
\vspace*{-12pt}
\end{figure*}

\begin{table*}\small
\begin{center}
\Caption{Илюстрация метода Белсли
\label{CovBelTable}}
\vspace*{2ex}

\begin{tabular}{|c|l|l|l|l|l|l|}
\hline
$\eta$ & \multicolumn{1}{c|}{$q_1$}& 
\multicolumn{1}{c|}{$q_2$}& \multicolumn{1}{c|}{$q_3$}& 
\multicolumn{1}{c|}{$q_4$}& \multicolumn{1}{c|}{$q_5$}& \multicolumn{1}{c|}{$q_6$}\\
\hline
&&&&&&\\[-9pt]
$1{,}0$ &  $2\cdot 10^{-17}$ &  $4\cdot 10^{-17}$ &  $1\cdot 10^{-16}$ &  $2\cdot 
10^{-17}$ &  $6\cdot 10^{-17}$&  $3\cdot 10^{-4}$ \\
%\hline
$1{,}5$ &  $5\cdot 10^{-17}$ &  $9\cdot 10^{-17}$ &  $2\cdot 10^{-16}$ &  $5\cdot 
10^{-17}$ &  $3\cdot 10^{-20}$ &  $3\cdot 10^{-2}$ \\
%\hline
$3{,}3$ &  $9\cdot 10^{-18}$ &  $1\cdot 10^{-17}$ &  $2\cdot 10^{-17}$ &  $9\cdot 
10^{-18}$ &  $2\cdot 10^{-19}$ &  $9\cdot 10^{-1}$ \\
%\hline
$2\cdot 10^{15}$ &  $1\cdot 10^{-2}$ &  $1\cdot 10^{-1}$ &  $8\cdot 10^{-1}$ &  
$2\cdot 10^{-3}$ &  $9\cdot 10^{-2}$ &  $1\cdot 10^{17}$ \\
%\hline
$8\cdot 10^{15}$ &  $6\cdot 10^{-2}$ &  $8\cdot 10^{-1}$ &  $9\cdot 10^{-2}$ &  
$8\cdot 10^{-2}$ &  $9\cdot 10^{-1}$ & $ 2\cdot 10^{17} $\\
%\hline
$1\cdot 10^{16}$ &  $\bf9\cdot 10^{-1}$ &  $1\cdot 10^{-2}$& $ 4\cdot 10^{-2}$&  
$\bf9\cdot 10^{-1}$ &  $1\cdot 10^{-3}$ & $ 5\cdot 10^{-21}$ \\
\hline
\end{tabular}
\end{center}
\vspace*{-6pt}
\end{table*}

\begin{multicols}{2}



\noindent
торые так\-же изоб\-ра\-же\-ны на рис.~\ref{CovBel},\,\textit{б}. Согласно 
этим данным максимальный индекс обуслов\-лен\-ности $\eta_6 \hm= 1{,}2\cdot 
10^{16}$. Ему соответствуют максимальные дис\-пер\-си\-он\-ные доли признаков 
с~индексами~1 и~4, которые, как видно из по\-стро\-ения выборки, коррелируют между 
собой.

\vspace*{-6pt}


\section{Вычислительный эксперимент}

\vspace*{-2pt}

Для анализа свойств предложенного алгоритма и~сравнения его с~существующими был 
проведен вычислительный эксперимент, в~котором па\-ра\-мет\-ры нейросети удалялись 
методами, описанными в~подразд.~3.1---3.3, и~методом Белсли.

В качестве данных использовались три выборки,
представленные в~табл.~2. Выборки Wine~\cite{Wine} 
и~Boston~Housing~\cite{Boston}~--- это реальные данные. Синтетические данные 
сгенерированы таким образом, чтобы па\-ра\-мет\-ры сети
 были муль\-тикор\-ре\-ли\-руемы\-ми.
Генерация
 данных состояла из двух этапов.\linebreak\vspace*{-12pt}
 
%\begin{table*}
{\small %tabl2
\begin{center}
{{\tablename~2}\ \ \small{Описание выборок}}
\vspace*{2ex}

\tabcolsep=3pt
\begin{tabular}{|c|c|c|c|}
\hline
    Выборка &\tabcolsep=0pt\begin{tabular}{c}Тип\\ задачи\end{tabular}& 
    \tabcolsep=0pt\begin{tabular}{c}Размер\\ выборки\end{tabular}& 
    \tabcolsep=0pt\begin{tabular}{c}Число\\ признаков\end{tabular}\\
    \hline
    \multicolumn{1}{|l|}{Wine}
    &
    \multicolumn{1}{l|}{Классификация}
     & \hphantom{\,99}178 & 13\\
    %\hline
    \multicolumn{1}{|l|}{Boston Housing}
    &
    \multicolumn{1}{l|}{Регрессия}
    & \hphantom{\,99}506 & 13\\
    %\hline
    \multicolumn{1}{|l|}{Synthetic data}
    &
    \multicolumn{1}{l|}{Регрессия}
    & 10\,000 & 100\hphantom{9}\\
\hline
\end{tabular}
\end{center}
}
%\end{table*}

\vspace*{9pt}


\noindent
 На первом этапе генерировался вектор па\-ра\-мет\-ров~$\mathbf{w}_{\mathrm{synthetic}}$:
 
 \noindent
\begin{equation*}
\mathbf{w}_{\mathrm{synthetic}}  \sim \mathcal{N}\left(\mathbf{m}_{\mathrm{synthetic}}, 
\mathbf{A}_{\mathrm{synthetic}}\right), 
%\label{e5.1-str}
\end{equation*}
где

\vspace*{-9pt}

\noindent
\begin{align*}
\mathbf{m}_{\mathrm{synthetic}} &= 
\begin{bmatrix}
1{,}0\\
0{,}0025\\[-3pt]
\vdots\\[-2pt]
0{,}0025
\end{bmatrix}\,;
\\
\mathbf{A}_{\mathrm{synthetic}} &= \begin{bmatrix}
1{,}0& 10^{-3}& \cdots& 10^{-3}& 10^{-3}\\
10^{-3}& 1{,}0& \cdots& 0{,}95& 0{,}95\\[-2pt]
\cdots&\cdots&\cdots&\cdots&\cdots\\[-2pt]
10^{-3}& 0{,}95& \cdots& 0{,}95& 1{,}0
\end{bmatrix}\,.
\end{align*}

На втором этапе генерировалась выборка $\mathfrak{D}_{\mathrm{synthetic}}$:

\vspace*{-2pt}

\noindent
\begin{multline*}
\mathfrak{D}_{\mathrm{synthetic}} = \left\{(\mathbf{x}_i,y_i)| \mathbf{x}_i \sim  
\mathcal{N}(\mathbf{1}, \mathbf{I}),\right.\\
\left. y_i = x_{i0},\ i = 1, \ldots , 10\,000\right\}. 
%\label{e5.2-str}
\end{multline*}
В приведенном выше векторе параметров $\mathbf{w}_{\mathrm{synthetic}}$\linebreak для 
выборки $\mathfrak{D}_{\mathrm{synthetic}}$ наиболее релевантным является первый 
параметр, а~все остальные\linebreak параметры~--- нерелевантные. Матрица ковариации была 
вы\-брана таким образом, чтобы все нерелевантные па\-ра\-мет\-ры были зависимы и~метод 
Белсли был максимально эффективен.







Для алгоритмов тренировочная и~тестовая выборки составили~80\% и~20\% 
соответственно. Критерием качества прореживания служит доля па\-ра\-мет\-ров 
нейросети, удаление которых не влечет\linebreak значимой потери качества прогноза. Так\-же 
критерием качества служит устой\-чи\-вость нейросети к~зашумленности данных.

Качеством прогноза~$R_{\mathrm{cl}}$ модели для задачи классификации выступает 
точность прогноза модели:

\vspace*{4pt}

\noindent
\begin{equation*}
R_{\mathrm{cl}} = \fr{\sum\nolimits_{(\textbf{x},y)\in \mathfrak{D}} [f(\textbf{x}, 
\mathbf{w}) = y]}{\left|\mathfrak{D}\right|}. 
%\label{e5.3-str}
\end{equation*}

Качеством прогноза~$R_{\mathrm{rg}} $ модели для задачи регрессии является 
сред\-не\-квад\-ра\-тич\-ное отклонение результата модели от точ\-ного:
\begin{equation*}
R_{\mathrm{rg}} = \fr{\sum\nolimits_{(\mathbf{x},y)\in \mathfrak{D}} 
\left(f(\mathbf{x}, \mathbf{w}) - 
y\right)^2}{\left|\mathfrak{D}\right|}\,.
%\label{e5.4-str}
\end{equation*}

\vspace*{-5pt}

\paragraph*{Выборка Wine.} Рассмотрим нейронную сеть с~13~нейронами на входе, 
13~нейронами в~скрытом слое и~3~нейронами на выходе.


  

На рис.~2,\,\textit{а} показано, как меняется точ\-ность прогноза $R_{\mathrm{cl}}$ 
при удалении па\-ра\-мет\-ров указанными методами. Из графика видно, что метод 
оптимального прореживания, вариационный метод и~метод Белсли поз\-во\-ля\-ют удалить 
$\approx80\%$ па\-ра\-мет\-ров и~качество всех этих методов падает при удалении 
$\approx90\%$ па\-ра\-мет\-ров нейросети.

На рис.~3 представлены по\-верх\-ности изменения уровня шума ответов 
нейросети при изменении доли удаленных па\-ра\-мет\-ров и~уровня шума вход\-ных 
данных для раз\-ных методов прореживания. Из графиков видно, что при удалении 
па\-ра\-мет\-ров нейросети методом Белсли шум меньше, чем при удалении па\-ра\-мет\-ров 
другими методами. На это указывает то, что по\-верх\-ность, которая соответствует 
методу Белсли, ниже других поверхностей.



  



\vspace*{-9pt}

\paragraph*{Выборка Boston Housing.} Рассмотрим нейронную сеть с~13~нейронами на 
входе, 39~нейронами в~скрытом слое и~одним нейроном на выходе.


{ \begin{center}  %fig2
 \vspace*{-6pt}
  \mbox{%
 \epsfxsize=78mm %79.149mm 
 \epsfbox{str-2.eps}
 }


\end{center}

\vspace*{-9pt}


\noindent
{{\figurename~2}\ \ \small{Качество прогноза при удалении 
па\-ра\-мет\-ров на выборках Wine~(\textit{а}),
Boston~(\textit{б}) и~синтетической~(\textit{в}):
\textit{1}~--- произвольное удаление; \textit{2}~--- оптимальное прореживание;
\textit{3}~--- вариационный метод; \textit{4}~--- метод Белсли}}
}

\vspace*{9pt}

\addtocounter{figure}{1}


На рис.~2,\,\textit{б} показано, как меняется сред\-не\-квад\-ра\-тич\-ное отклонение 
прогноза $R_{\mathrm{rg}}$ от точного ответа  при удалении па\-ра\-мет\-ров указанными 
методами.
%
 График показывает, что метод Белсли является
более эффективным, чем 
другие методы, так как
поз\-во\-ляет удалить больше па\-ра\-мет\-ров нейросети без потери 
качества.

\pagebreak

\end{multicols}

\setcounter{figure}{2}
\begin{figure*} %fig3
  \vspace*{1pt}
  \begin{minipage}[t]{80mm}
    \begin{center}  
  \mbox{%
 \epsfxsize=77.956mm 
 \epsfbox{str-3.eps}
 }
 \end{center}
\vspace*{-6pt}
\Caption{Влияние шума в~начальных данных на шум выхода 
нейросети на выборке 
Wine: (\textit{а})~произвольное удаление па\-ра\-мет\-ров; 
(\textit{б})~оптимальное прореживание; (\textit{в})~вариационный метод.
Серый цвет~--- метод Белсли}
\end{minipage}
%\label{BostonNoise}
%\end{figure*}
\hfill
%\begin{figure*} %fig4
\vspace*{1pt}
  \begin{minipage}[t]{80mm}
    \begin{center}  
  \mbox{%
 \epsfxsize=77.956mm 
 \epsfbox{str-4.eps}
 }
 \end{center}
\vspace*{-6pt}
\Caption{Влияние шума в~начальных данных на шум выхода нейросети на выборке 
Boston: (\textit{а})~произвольное удаление па\-ра\-мет\-ров; 
(\textit{б})~оптимальное прореживание; (\textit{в})~вариационный метод.
Серый цвет~--- метод Белсли}
\label{BostonNoise}
\end{minipage}
\vspace*{12pt}
\end{figure*}


\begin{multicols}{2}


На рис.~4 представлены поверхности изменения уровня шума ответов 
нейросети при изменении доли удаленных параметров и~уровня шума входных 
данных для разных методов прореживания. График показывает, что уровень шума всех 
методов одинаковый, так как поверхности всех методов находятся на одном уровне.

\vspace*{-9pt}

\paragraph*{Синтетические данные.} Рассмотрим нейронную сеть со~100~нейронами на 
входе и~одним нейроном на выходе.



На рис.~2,\,\textit{в} показано, как меняется сред\-не\-квад\-ра\-тич\-ное отклонение 
прогноза от $R_{\mathrm{rg}}$ точного ответа при удалении параметров указанными 
методами. График показывает, что удаление параметров
методом Белсли является 
более эффективным, чем другие методы прореживания, так как качество прогноза 
нейросети улучшается при удалении шумовых параметров.

На рис.~5 представлены поверхности изменения уровня шума ответов 
нейросети при изменении\linebreak\vspace*{-12pt}

{ \begin{center}  %fig5
 \vspace*{-6pt}
  \mbox{%
 \epsfxsize=79mm 
 \epsfbox{str-5.eps}
 }


\end{center}


\noindent
{{\figurename~5}\ \ \small{Влияние шума в~начальных данных на шум выхода нейросети на 
синтетической выборке: (\textit{а})~произвольное удаление па\-ра\-мет\-ров; 
(\textit{б})~оптимальное 
прореживание; (\textit{в})~вариационный метод.
Серый цвет~--- метод Белсли}}
}

\vspace*{9pt}

\addtocounter{figure}{1}


\noindent
 доли удаленных параметров и~уровня шума входных 
данных для разных методов прореживания.\linebreak
 Из графиков видно, что при удалении 
параметров нейросети методом Белсли шум меньше, чем при удалении параметров 
другими методами, так как поверхность, которая соответствует методу Белсли, ниже 
других поверхностей.

\section{Заключение}

В работе рассматривалась задача прореживания моделей нейросетей. Рассматривались 
метод оптимального прореживания и~метод, основанный на вариационном подходе. Был 
предложен алгоритм прореживания, основанный на методе Белсли, для удаления 
зависимых параметров модели. В~ходе эксперимента было показано, что нейросети, 
прореженные методом Белсли, более устойчивы к~шуму на входных данных. Качество 
прогноза нейросетей после прореживания методом Белсли не хуже качества прогноза 
нейросетей, прореженных другими методами.




 {\small\frenchspacing
 {%\baselineskip=10.8pt
 \addcontentsline{toc}{section}{References}
 \begin{thebibliography}{99}

   \bibitem{sutskever2014}
   \Au{Sutskever I., Vinyals~O., Le~Q.} Sequence to sequence learning 
with neural networks~// Adv. Neur. Inf., 2014. 
Vol.~2. P.~3104--3112.
    
    \bibitem{maclarin2015}
    \Au{Maclaurin D., Duvenaud~D., Adams~R.} Gradient-based hyperparameter 
optimization through reversible learning~//  32th 
 Conference (International) on Machine Learning Proceedings.~---
 Lille, 2015. Vol.~37. P.~2113--2122.
        
    \bibitem{luketina2015}
\Au{Luketina J., Berglund~M., Raiko~T., Greff~K.} Scalable gradient-based 
tuning of continuous regularization hyperparameters~//  
33th  Conference (International) 
on Machine Learning Proceedings.~--- New York, NY, USA, 2016. Vol.~48. P.~2952--2960.

    \bibitem{molchanov2017}
    \Au{Molchanov D., Ashukha~A., Vetrov~D.} Variational dropout 
sparsifies deep neural networks~// 34th Conference 
(International) on Machine Learning Proceedings.~--- Sydney, 2017. Vol.~70. P.~2498--2507.

    \bibitem{neal1995}
    \Au{Neal A., Radford~M.} Bayesian learning for neural network.~--- 
    Toronto, ON, Canada, 1995.  Ph.D. Thesis. 195~p.
    
    \bibitem{cun1990} %6
   \Au{LeCun Y., Denker~J., Solla~S.} Optimal brain damage~// Adv. 
Neur. Inf., 1989. Vol.~2. P.~598--605.

 \bibitem{graves2011} %7
    \Au{Graves A.} Practical variational inference for neural networks~// 
Adv. Neur. Inf., 2011. Vol.~24. P.~2348--2356.
    
    \bibitem{louizos2017} %8
    \Au{Louizos C., Ullrich~K., Welling~M.} Bayesian compression for deep 
learning~// Adv. Neur. Inf., 2017. Vol.~30. 
P.~3288--3298.
    
   

    \bibitem{neychev2016}
    \textit{Neychev R., Katrutsa~A., Strijov~V.} Robust selection of 
multicollinear features in forecasting~// Factory Laboratory, 2016. Vol.~82. 
No.\,2. P.~68--74.
    
    \bibitem{Boston}
    \Au{Harrison D.,  Rubinfeld~D.} Hedonic prices and the demand for 
clean air~//
J.~Environ. Econ. Manag., 1978. Vol.~5. P.~81--102. 
{\sf https://www.cs.toronto.edu/$\sim$delve/ data/boston/bostonDetail.html.}
    
    \bibitem{Wine}
    \Au{Aeberhard S.} Wine Data Set, 1991. 
{\sf http://archive.ics.\linebreak uci.edu/ml/datasets/Wine.}

    \bibitem{bishop2006}
    \Au{Bishop C.} Pattern recognition and machine learning.~--- Berlin: 
Springer, 2006. 758~p.
 \end{thebibliography}

 }
 }

\end{multicols}

\vspace*{-9pt}

\hfill{\small\textit{Поступила в~редакцию 31.10.18}}

%\vspace*{8pt}

%\pagebreak

\newpage

\vspace*{-29pt}

%\hrule

%\vspace*{2pt}

%\hrule

%\vspace*{-2pt}

\def\tit{ESTIMATION OF~THE~RELEVANCE OF~THE~NEURAL NETWORK PARAMETERS}


\def\titkol{Estimation of~the~relevance of~the~neural network parameters}

\def\aut{A.\,V.~Grabovoy$^1$, O.\,Yu.~Bakhteev$^1$, and~V.\,V.~Strijov$^{1,2}$}

\def\autkol{A.\,V.~Grabovoy, O.\,Yu.~Bakhteev, and~V.\,V.~Strijov}

\titel{\tit}{\aut}{\autkol}{\titkol}

\vspace*{-11pt}


\noindent
$^1$Moscow Institute of Physics and Technology,
9~Institutskiy Per., Dolgoprudny, Moscow Region 141700, Russian\linebreak
$\hphantom{^1}$Federation

\noindent
$^2$A.\,A.\,Dorodnicyn Computing Center, Federal Research Center 
``Computer Science and Control'' of the Russian\linebreak
$\hphantom{^1}$Academy of Sciences,
40~Vavilov Str., Moscow 119333, Russian Federation 

\def\leftfootline{\small{\textbf{\thepage}
\hfill INFORMATIKA I EE PRIMENENIYA~--- INFORMATICS AND
APPLICATIONS\ \ \ 2019\ \ \ volume~13\ \ \ issue\ 2}
}%
 \def\rightfootline{\small{INFORMATIKA I EE PRIMENENIYA~---
INFORMATICS AND APPLICATIONS\ \ \ 2019\ \ \ volume~13\ \ \ issue\ 2
\hfill \textbf{\thepage}}}

\vspace*{6pt}    
    


\Abste{The paper investigates a method for optimizing the structure of 
a~neural network. It is assumed that the number of neural network parameters can be 
reduced without significant loss of quality and without significant increase in 
the variance of the loss function. The paper proposes a~method for automatic 
estimation of the relevance of parameters to prune a~neural network. This method 
analyzes the covariance matrix of the posteriori distribution of the model 
parameters and removes the least relevant and multicorrelate parameters. It uses the 
Belsly method to search for multicorrelation in the neural network. The proposed 
method was tested on the Boston Housing data set, the Wine data set, and synthetic 
data.}

\KWE{neural network; hyperparameters optimization; Belsly method; relevance 
of parameters; neural network pruning}



\DOI{10.14357/19922264190209}

%\vspace*{-14pt}

\Ack
\noindent
This research was supported by the Russian Foundation for Basic
Research, project 19-07-0875, 
and by Government of the Russian Federation, agreement 05.Y09.21.0018. 
This paper contains results of the project ``Statistical methods of 
machine learning,'' which is carried out within the framework of the Program 
``Center of Big Data Storage and Analysis'' of the National Technology 
Initiative Competence Center supported by the Ministry of Science 
and Higher Education of the Russian Federation according to the agreement 
between M.\,V.~Lomonosov Moscow State University and the Foundation 
of Project Support of the National Technology Initiative from 11.12.2018, 
No.\,13/1251/2018.


%\vspace*{6pt}

  \begin{multicols}{2}

\renewcommand{\bibname}{\protect\rmfamily References}
%\renewcommand{\bibname}{\large\protect\rm References}

{\small\frenchspacing
 {%\baselineskip=10.8pt
 \addcontentsline{toc}{section}{References}
 \begin{thebibliography}{99}
    \bibitem{sutskever2014-str}
\Aue{Sutskever,~I., O.~Vinyals, and Q.~Le.} 2014. 
Sequence to sequence learning with neural networks. 
\textit{Adv. Neur. Inf.} 2:3104--3112.

    \bibitem{maclarin2015-str}
    \Aue{Maclaurin, D., D.~Duvenaud, and R.~Adams.}
     2015. Gradient-based hyperparameter optimization through reversible learning. 
     \textit{32th  Conference (International) on Machine Learning Proceedings}. 
     Lille. 37:2113--2122.
    
    \bibitem{luketina2015-str}
    \Aue{Luketina, J., M.~Berglund, T.~Raiko, and K.~Greff.} 2016. 
    Scalable gradient-based tuning of continuous regularization hyperparameters. 
    \textit{33th Conference (International)
    on Machine Learning Proceedings}. New~York, NY. 48:2952--2960.

    \bibitem{molchanov2017-str}
    \Aue{Molchanov, D., A.~Ashukha, and D.~Vetrov.} 
    2017. Variational dropout sparsifies deep neural networks. 
    \textit{34th Conference 
    (International) on Machine Learning Proceedings}. Sydney. 70:2498--2507.
    
    \bibitem{neal1995-str}
    \Aue{Neal, A., and M.~Radford.} 1995. Bayesian learning for neural networks. 
    Toronto, ON: University of Toronto. Ph.D. Thesis. 195~p. 
    
    \bibitem{cun1990-str}
    \Aue{LeCun, Y., J. Denker, and S.~Solla.} 1989. Optimal brain damage. 
    \textit{Adv. Neur. Inf.} 2:598--605.
    
    
    
    \bibitem{graves2011-str}
    \Aue{Graves, A.}  2011. Practical variational inference for neural networks.
     \textit{Adv. Neur. Inf.} 24:2348--2356.
     
     \bibitem{louizos2017-str}
    \Aue{Louizos, C., K.~Ullrich, and M.~Welling.} 
    2017. Bayesian compression for deep learning. \textit{Adv. 
    Neur. Inf}. 30:3288--3298. 

    \bibitem{neychev2016-str}
    \Aue{Neychev, R., A.~Katrutsa, and V.~Strijov.} 
    2016. Robust selection of multicollinear features in forecasting. 
    \textit{Factory Laboratory} 82(3):68--74.
    
    \bibitem{Boston-str}
\Aue{Harrison, D., and D.~Rubinfeld.}  1978.
 Hedonic prices and the demand for clean air.
 \textit{J.~Environ. Econ. Manag.}
 5:81--102. Available at:
 {\sf https://www.cs.toronto.edu/$\sim$delve/ data/boston/bostonDetail.html}
 (accessed June~4, 2019).
    
    \bibitem{Wine-str}
    \Aue{Aeberhard, S.} 1991.  Wine Data Set. Available at:
    {\sf http://archive.ics.uci.edu/ml/datasets/Wine}
    (accessed June~4, 2019).
    
    \bibitem{bishop2006-str}
    \Aue{Bishop, C.} 2006. \textit{Pattern recognition and machine learning}. 
    Berlin: Springer. 758~p.
\end{thebibliography}

 }
 }

\end{multicols}

\vspace*{-6pt}

\hfill{\small\textit{Received October 31, 2018}}

%\pagebreak

%\vspace*{-18pt}    


\Contr

\noindent
\textbf{Grabovoy Andrey V.} (b.\ 1997)~--- 
student, Moscow Institute of Physics and Technology, 
9~Institutskiy Per., Dolgoprudny, Moscow Region 141700, Russian Federation; 
\mbox{grabovoy.av@phystech.edu}

\vspace*{6pt}

\noindent
\textbf{Bakhteev Oleg Yu.} (b.\ 1991)~--- 
graduate student, Moscow Institute of Physics and Technology, 
9~Institutskiy Per., Dolgoprudny, Moscow Region 141700, Russian Federation; 
\mbox{bakhteev@phystech.edu} 

\vspace*{6pt}

\noindent
\textbf{Strijov Vadim V.} (b.\ 1967)~---
Doctor of Science in physics and mathematics, leading scientist, 
Dorodnicyn Computing Center, Federal Research Center ``Computer Science and Control'' 
of the Russian Academy of Sciences, 40~Vavilov Str., Moscow 119333, Russian Federation; 
professor, Moscow Institute of Physics and Technology, 
9~Institutskiy Per., Dolgoprudny, Moscow Region 141701, Russian Federation; 
\mbox{strijov@ccas.ru}
\label{end\stat}

\renewcommand{\bibname}{\protect\rm Литература}        %6
\def\stat{flerov}

\def\tit{АВТОМАТИЗИРОВАННАЯ СИСТЕМА ВЕСОВОГО 
ПРОЕКТИРОВАНИЯ САМОЛЕТОВ}

\def\titkol{Автоматизированная система весового 
проектирования самолетов}

\def\aut{Л.\,Л.~Вышинский$^1$, Ю.\,А.~Флеров$^2$, Н.\,И.~Широков$^1$}

\def\autkol{Л.\,Л.~Вышинский, Ю.\,А.~Флеров, Н.\,И.~Широков}

\titel{\tit}{\aut}{\autkol}{\titkol}

\index{Вышинский Л.\,Л.}
\index{Флеров Ю.\,А.}
\index{Широков Н.\,И.}
\index{Vyshinsky L.\,L.}
\index{Flerov Yu.\,A.}
\index{Shirokov N.\,I.}




%{\renewcommand{\thefootnote}{\fnsymbol{footnote}} \footnotetext[1]
%{Работа выполнена при финансовой поддержке РФФИ (проект 17-01-00816).}}


\renewcommand{\thefootnote}{\arabic{footnote}}
\footnotetext[1]{Вычислительный центр им.\ А.\,А.~Дородницына Федерального исследовательского 
центра <<Информатика и~управ\-ле\-ние>> Российской академии наук, 
\mbox{Wysh@ccas.ru}}
\footnotetext[2]{Вычислительный центр им.\ А.\,А.~Дородницына Федерального исследовательского 
центра <<Информатика и~управ\-ле\-ние>> Российской академии наук, 
fler@ccas.ru}
%\footnotetext[3]{Вычислительный центр им.\ А.\,А.~Дородницына Федерального исследовательского 
%центра <<Информатика и~управ\-ле\-ние>> Российской академии наук, 
%\mbox{Wysh@ccas.ru}}

%\vspace*{-6pt}


 
  \Abst{Статья посвящена вопросам автоматизации задач весового проектирования 
самолетов. Весовые и~мас\-со\-во-инер\-ци\-он\-ные параметры являются одними из основных 
величин, влияющих на эксплуатационные характеристики самолетов. Информационной 
основой системы служит весовая модель самолета. Описывается структура весовой 
модели и~даны характеристики отдельным ее компонентам. Показана программная 
реализация системы, которая выполнена в~рамках архитектуры кли\-ент--сер\-вер. 
Автоматизированная система весового проектирования (АСВП)
реализована с~использованием 
про\-грам\-мно-ин\-стру\-мен\-таль\-но\-го комплекса <<Генератор проектов>> (технология ГП), 
который был разработан в~Вычислительном центре Российской академии наук. Создание 
ин\-фор\-ма\-ци\-он\-но-вы\-чис\-ли\-тель\-ных сис\-тем в~рамках технологии ГП базируется на так 
называемом <<проектном подходе>>, когда по формальному описанию системы автоматически 
генерируются тексты программного кода для клиентских и~серверных компонент системы.}
   
  \KW{математическое моделирование; автоматизация проектирования; самолет; весовое 
проектирование; весовая модель; дерево конструкции; генератор проектов; генерация 
программного кода; архитектура кли\-ент--сер\-вер}

  \DOI{10.14357/19922264180103} 
  
\vspace*{12pt}


\vskip 10pt plus 9pt minus 6pt

\thispagestyle{headings}

\begin{multicols}{2}

\label{st\stat}
   
\section{Введение}

  Развитие и~повсеместное использование информационных технологий за 
последние несколько десятилетий существенно изменили традиционный 
процесс проектирования и~создания различных инженерных систем, 
сооружений, машин. Во многих проектных организациях давно отказались от 
ко\-гда-то привычных инструментов конструктора~--- кульмана 
и~логарифмической линейки. 
%
Сейчас первые эскизы новых проектов 
появляются чаще не на бумаге, как было всегда, а~на экране монитора. Этому 
способствует широкий спектр имеющихся систем автоматизированного 
проектирования. В~российских авиационных конструкторских бюро, например, уже давно 
применяются такие CAD (computer aided design)
сис\-те\-мы, как NX (Unigraphics), CATIA и~др. 
%
Эти развитые системы геометрического трех\-мер\-но\-го (3D) мо\-де\-ли\-ро\-ва\-ния позволяют 
создавать сложные по\-верх\-ности, конструировать любые детали, осуществлять 
сборку узлов, агрегатов и~сложнейших изделий. Однако построение 
геометрических моделей изделий является финальной стадией проектирования, 
за которой следует их реализация <<в~металле>>. Построению электронных 
геометрических макетов предшествует и~сопутствует решение множества 
расчетных задач, а~также задач анализа и~оптимизации в~разных областях инженерных 
знаний. В~авиастроении это аэродинамика, динамика полета, прочность, 
системы управления, двигателестроение и~пр. Все эти задачи 
требуют разработки разноплановых математических моделей и~специальных 
вычислительных программ. 
  
  Одной из важнейших технических характеристик самолета является его вес. 
При решении подавляющего большинства проектных и~конструкторских задач 
весовые параметры в~том или ином виде участвуют в~расчетах. Необходимость 
проведения весовых расчетов возникает на самых ранних шагах 
проектирования и~сопровождает все дальнейшие стадии разработки 
и~эксплуатации. 

В~процессе создания и~эксплуатации самолетов постоянно 
контролируются вес и~другие мас\-со\-во-инер\-ци\-он\-ные характеристики (МИХ)
всех размещаемых на борту систем, агрегатов, узлов и~деталей. Количество 
агрегатов, узлов и~деталей современных самолетов исчисляется 
десятками тысяч, поэтому в~авиастроении весовые расчеты, весовой анализ, 
весовой контроль выливаются в~сложную инженерную проблему и~выделяются 
в~целое направление инженерной деятельности, которое принято называть 
весовым проектированием~[1].
  
  Данная статья посвящена вопросам автоматизации задач весового 
проектирования самолетов. В~разные годы Вычислительным центром РАН\linebreak был 
разработан и~внедрен в~эксплуатацию ряд \mbox{программ}, решающих отдельные 
задачи весовых рас\-че\-тов летательных аппаратов (ЛА)~[2--4]. В~настоящей статье 
представлено описание интегрированной АСВП, предназначенной для использования на всех 
этапах жизненного цикла изделий. Она разработана как интерактивная 
многопользовательская информационная система кли\-ент-сер\-вер\-ной 
архитектуры с~централизованной базой данных. Информационным ядром 
и~основой АСВП является единая струк\-тур\-но-па\-ра\-мет\-ри\-че\-ская весовая модель 
самолета, описание которой дает довольно полное представление о~задачах, 
решаемых с~помощью АСВП.

\section{Структурно-параметрическая весовая модель самолета}

  Самолет является сложным техническим объ\-ектом, состоящим из множества 
различных \mbox{ком\-понентов}, функционально и~конструктивно связанных между 
собой. Под струк\-тур\-но-па\-ра\-мет\-ри\-че\-ской весовой моделью самолета 
здесь понимается база данных, которая содержит всю необходи\-мую 
информацию для проведения комплекса расчетов 
МИХ и~мас\-со\-во-цент\-ро\-воч\-ных данных (МЦД) 
самолета. Весовая модель состоит из нескольких структур, ориентированных на 
определенные группы параметров и~задач весового проектирования. Ниже 
перечислены основные структуры весовой модели, реализованные в~системе 
АСВП:
  \begin{itemize}
\item дерево конструкции самолета;
\item иерархия систем координат, связанных с~самолетом и~его агрегатами;
\item геометрические структуры весовой модели самолета;
\item каталог целевой нагрузки, размещаемой во внут\-рен\-них отсеках и~на 
подвесках;
\item реестр допустимых вариантов загрузки само\-лета;
\item таблицы тарировочных характеристик топливных баков;
\item таблицы характеристик выработки топлива.
\end{itemize}


  \subsection{Дерево конструкции самолета}

  Дерево конструкции самолета является центральной структурой весовой 
модели, которая отражает членение изделия на его составные части~--- 
системы, агрегаты, узлы, детали. В~базе данных весовой модели эта структура 
представлена в~виде многоуровневого корневого дерева $W \hm= (U, V)$, где 
вершинам $U \hm= \{U_i\}$ соответствуют различные\linebreak
 элементы конструкции. 
Ориентированные дуги дере\-ва, идущие из~$U_i$ в~$U_j$, означают вхождение 
конструкции~$U_j$ в~конструкцию~$U_i$ в~качестве ее составной части. 
Терминальными или висячими вершинами дерева конструкции будем называть 
вершины, у которых нет ни одной подчиненной конструкции.
  
  Многолетний опыт самолетостроения выработал устоявшиеся 
конструктивные схемы самолетов различного назначения. Существуют 
отраслевые стандарты и~классификаторы, которые вводят определения 
основных элементов конструкции самолетов. На рис.~1 показан пример 
представления в~АСВП нескольких верхних уровней дерева конструкции 
самолета. 


    

  Существующие классификаторы отражают лишь самые общие принципы 
построения конструкции самолетов. Разумеется, каждый новый проект 
самолета имеет свои конструктивные особенности, которые отражаются на 
структуре весовой модели. Дерево конструкции строится постепенно, сверху 
вниз, в~течение всего процесса проектирования самолета. 

 { \begin{center}  %fig1
 \vspace*{9pt}
\mbox{%
 \epsfxsize=77.216mm 
 \epsfbox{fle-1.eps}
 }

\vspace*{6pt}


\noindent
{{\figurename~1}\ \ \small{Дерево конструкции самолета}}
\end{center}
}

\addtocounter{figure}{1}
  
  Понятие <<конструкции>> в~данном контексте используется и~для 
обозначения любой вершины графа, и~для всего поддерева подчиненных 
конструкций этой вершине. Каждая конструкция дерева имеет уникальное 
в~рамках весовой модели обозначение, которое может быть шифром, кодом, 
идентификатором или чертежным номером конструкции. Разумеется, для более 
полного и~наглядного вербального представления конструкции  
в~струк\-тур\-но-па\-ра\-мет\-ри\-че\-ской модели можно задать ее текстовое 
описание.
  
  \textbf{Масса конструкции.} Основную содержательную и~необходимую 
информацию весовой модели содержит список значений  
МИХ, соответствующих каждой 
вершине дерева конструкций. Центральным параметром является масса. 
  
  На разных стадиях создания самолета, когда неизвестно точное значение 
массы, прибегают к~различным оценкам.  
В~струк\-тур\-но-па\-ра\-мет\-ри\-че\-ской весовой модели фиксируются 
перечисленные ниже оценки массы, которые соответствуют разным этапам 
проектирования:
  \begin{description}
\item[\,]  $M_{\mathrm{теор}}$~--- теоретическая масса~--- оценка массы, вычисленная на 
основании некоторой математической модели конструкции; 
  
\item[\,]  $M_{\mathrm{лим}}$~--- лимитная масса конструкции, уста\-нав\-ли\-ва\-емая на 
основании теоретических оценок и~используемая для весового контроля 
в~процессе детальной разработки конструкции;
  
\item[\,]  $M_{\mathrm{черт}}$~--- чертежная масса конструкции, рассчитанная по чертежу или по 
электронной гео\-мет\-ри\-че\-ской модели конструкции;
  
\item[\,]  $M_{\mathrm{креп}}$~--- масса крепежа конструкции~--- дополнение к~чертежной массе, 
которое учитывает мелкие детали конструкции, предназначенные для 
соединения подчиненных деталей (заклепки, болты, гайки, шайбы и~т.\,п.). 
Введение такой дополнительной массы позволяет избавить дерево конструкции 
от десятков и~сотен тысяч вершин, которые несут относительно небольшую 
нагрузку в~весовых характеристиках, но тем не менее их учет необходим при 
контроле веса. Масса крепежа распределяется по подчиненным конструкциям;  
\item[\,]  $M_{\mathrm{факт}}$~--- фактическая масса изготовленной 
и~взвешенной конструкции. 
Фактическая масса может задаваться не только для изготавливаемых 
конструкций ЛА, но и~для готовых по\-став\-ля\-емых 
изделий при их установке на борту.
\end{description}
  
  Порядок задания оценок массы диктуется логикой развития проекта. 
В~дереве конструкции все оценки массы, кроме $M_{\mathrm{лим}}$ и~$M_{\mathrm{креп}}$, 
суммируются по подчиненным вершинам снизу вверх. Однако если для 
некоторых терминальных значений одна или несколько оценок не определены, 
например некоторые детали конструкции не изготовлены и, стало быть, 
$M_{\mathrm{факт}}$ не определена, то и~для всех вышестоящих конструкций эти оценки не 
определены. При задании $M_{\mathrm{лим}}$ это правило может не соблюдаться. 
  
  На основании оценок массы определяется то расчетное значение массы 
конструкции, которое используется во всех расчетах на текущей стадии 
проекта: 
  $M$~--- текущая масса конструкции. Значение текущей массы \textit{для 
нетерминальных} конструкций определяется суммированием по подчиненным 
конструкциям. \textit{Для терминальных} вершин дерева конструкций 
применяется процедура определения текущей массы по первому известному 
значению из следующего списка в~указанном порядке: $M_{\mathrm{факт}}$, 
$M_{\mathrm{черт}}$\;+\;$M_{\mathrm{креп}}$, $M_{\mathrm{теор}}$, $M_{\mathrm{лим}}$.
  
  \textbf{Геометрия масс конструкции.} Кроме собственно массы в~весовой 
модели задаются или вычисляются значения характеристик, которые принято 
называть характеристиками геометрии масс: 
  \begin{description}
  \item[\,] $X$, $Y$ и $Z$~--- положение центра масс конструкции; 
  \item[\,] $L_x$, $L_y$ и $L_z$~--- габаритные параметры конструкции;
  \item[\,] $I_x$, $I_y$ и $I_z$~--- полные плоскостные моменты инерции;
  \item[\,]  $I_{xy}$, $I_{xz}$ и $I_{yz}$~--- полные центробежные моменты 
инерции;
  \item[\,] $I^c_x$, $I^c_y$ и  $I^c_z$~--- собственные плоскостные моменты 
инерции:
  \begin{align*}
  I^c_x &= I_x - M X^2\,;\\ 
  I^c_y &= I_y - M Y^2\,;\\ 
  I^c_z &= I_z - M Z^2\,;
 \end{align*}
  \item[\,] $I^c_{xy}$, $I^c_{xz}$ и~$I^c_{yz}$~--- собственные центробежные 
моменты инерции:
 \begin{align*}
  I^c_{xy} &= I_{xy}- M X Y\,;\\
   I^c_{xz} &= I_{xz}- M X Z\,;\\
   I^c_{yz} &= I_{yz}- M Y Z\,;
\end{align*}
  \item[\,] $J_x$, $J_y$ и $J_z$~--- собственные осевые моменты инерции 
конструкции:
  \begin{align*}
  J_x &= I^c_y + I^c_z\,;\\ 
  J_y &= I^c_x + I^c_z\,;\\
   J_z &= I^c_y + I^c_x\,;
  \end{align*}
  \item[\,] СК~--- система координат конструкции, в~которой задаются 
характеристики геометрии масс.
  \end{description}
  
  \begin{figure*} %fig2
  \vspace*{1pt}
 \begin{center}
 \mbox{%
 \epsfxsize=162mm 
 \epsfbox{fle-2.eps}
 }
 \end{center}
\vspace*{-9pt}
  \Caption{Основные параметры конструкций весовой модели самолета}
  \end{figure*}
  
  Каждая конструкция привязывается к~одной из систем координат, которые 
описаны в~весовой модели. В~весовой модели изделия для удобства описания 
различных агрегатов может быть описано несколько систем координат. Все 
описанные сис\-те\-мы координат организованы в~иерархическую структуру. 
Считается предописанной глобальная самолетная система координат, в~которой 
могут быть заданы или вычислены координаты всех объектов весовой модели. 
Каждая система координат в~весовой модели задается уникальным именем, 
положением начала координат относительно вышестоящей системы координат 
и~тремя углами поворота относительно вышестоящей. 

Параметр, 
обозначенный как СК,~--- это имя одной из сис\-тем координат весовой модели. 
Если СК не задано, то считается, что характеристики гео\-мет\-рии масс заданы 
в~глобальной системе координат. Каж\-дая сис\-те\-ма координат весовой модели 
содержит матрицу преобразования координат из самолетной (глобальной) 
системы координат в~данную и~обратно. Эта матрица для каждой системы 
координат есть произведение локальных матриц преобразований 
в~соответствии с~положением данной системы в~иерархии систем координат. 
Любое изменение параметров систем координат требует пе\-ре\-вы\-чис\-ле\-ния 
матриц преобразования как измененной сис\-те\-мы, так и~всех подчиненных. На 
рис.~2 показана панель параметрического пред\-став\-ле\-ния конструкций весовой 
модели.
  
  Так же как и~масса, центры тяжести и~моменты инерции вычисляются снизу
вверх от терминальных конструкций к~вышестоящим. При этом осуществляется 
пересчет характеристик по заданной иерархии систем координат от 
нижестоящих к~вышестоящим и~к~самолетной системе координат. Расчет 
МИХ терминальных конструкций 
осуществляется на основании гео\-мет\-ри\-че\-ских моделей. Геометрические модели 
на этапе рабочего проекта строятся в~системах гео\-мет\-ри\-че\-ско\-го 
моделирования. В~процессе их построения автоматически вычисляются 
объемы, массы, положение центра тяжести и~другие характеристики гео\-мет\-рии 
масс. Рассчитанная в~системах гео\-мет\-ри\-че\-ско\-го моделирования масса 
с~по\-мощью соответствующих интерфейсных средств может быть загружена как 
$M_{\mathrm{черт}}$ в~весовую модель. (Раньше документация была представлена в~виде 
чертежей на бумажных носителях и~$M_{\mathrm{черт}}$ вручную вычислялась по этим 
чертежам.) Однако на более ранних этапах проектирования, когда еще не 
проработана гео\-мет\-рия многих элементов конструкции, весовые расчеты 
проводятся на основании эскизов и~наборов гео\-мет\-ри\-че\-ских и~конструктивных 
параметров агрегатов изделия. Для этого в~весовой модели должны быть 
предусмотрены средства параметрического представления гео\-мет\-рии 
конструкций. Геометрическое пред\-став\-ле\-ние конструкций 
в~автоматизированной системе весового проектирования выполняет 
и~немаловажную функцию визуализации конструкций, их компоновки, 
размещения нагрузки и~т.\,д. В~АСВП реализовано несколько форм 
представления гео\-мет\-ри\-че\-ской информации, ориентированных не только на 
расчет МИХ, но и~на визуализацию выполняемых расчетов. Это чертежи 
геометрических проекций изделия, это таб\-лич\-ное задание типовых 
геометрических конструкций, это каркасное представление трехмерных 
геометрических моделей и, наконец, задание объемных конструкций 
триангуляционной (фасеточной) поверхностью. Последний вид представления 
является наиболее перспективным для точного вычисления МИХ. В~АСВП он 
применяется для расчета тарировочных характеристик топливных баков, о~чем 
будет сказано ниже.
  
  \textbf{Классификационные признаки конструкции.} В~весовой модели 
кроме числовых параметров опре\-делен ряд классификационных признаков 
конструкций, по которым проводится весовой анализ.\linebreak
 Таки\-ми маркерами могут 
быть подразделения, ответст\-вен\-ные за разработку конструкции, поставщики 
или изготовители готовых изделий, принадлежность конструкции 
к~определенным функциональным системам, конструкционные материалы 
и~пр.
  
  \textbf{Функциональные подсистемы изделия.} Конст\-рук\-тив\-ное членение 
самолета может не совпадать с~его функциональной структурой. Отдельные\linebreak 
элементы функциональных подсистем самолета удобнее описывать в~составе 
конструкции ка\-ко\-го-ни\-будь агрегата планера. Например, некоторая деталь 
может конструктивно входить в~состав крыла, а принадлежать 
к~функциональной подсистеме гидравлики или электрооборудования. Для того 
чтобы иметь возможность выполнять весовые расчеты, учитывая разные 
подходы к~классификации конструкции самолета, в~АСВП отдельно от дерева 
конструкции ведется реестр подсистем, для которых может быть проведен 
специальный расчет весовых параметров. В~этом реестре ведется полный 
перечень конструкций весовой модели, входящих в~подсистемы реестра, 
независимо от того, в~какой ветви дерева конструкции они находятся. Любая 
конструкция может быть включена только в~одну из подсистем реестра. 
В~зависимости от режима расчетов МИХ
конструкций, входящих в~под\-сис\-те\-му, будут учтены либо в~со\-ста\-ве 
вышестоящих агрегатов дерева конструкции, либо отдельно в~под\-сис\-теме. 
{\looseness=1

}
  
  \textbf{Распределенные характеристики изделия.} Задача вычисления 
распределенных характеристик изделия является родственной задачей 
вычисления характеристик геометрии масс. Основное отличие состоит в~том, 
что в~данной задаче рассчитываются не интегральные характеристики 
распределения материала, а сама функция распределения массы по объему 
конструкции. Такие функции рассчитываются по заданному геометрическому 
разбиению конструкции на пространственные отсеки. Функции распределения 
массы по объему конструкции в~процессе проектирования используются при 
построении динамически подобных моделей для проведения некоторых видов 
испытаний и~продувок, а~также для выполнения прочностных расчетов. 
  
  Каждый отсек разбиения для расчета распределенных характеристик 
представляет собой вы\-пук\-лый многогранник, ограниченный конечным набором 
плоскостей. Задача построения распределенных весовых характеристик состоит 
в~вычислении для каждого отсека массы и~положения центра тяжести той части 
конструкции самолета, которая геометрически расположена внутри этого 
отсека. Эта задача решается путем нахождения геометрического пересечения 
каждой терминальной конструкции с~каждым отсеком разбиения, и~если такое 
пересечение не пусто, то вычисление массы и~центра тяжести той части 
конструкции, которая попадает в~отсек. Некоторые конструкции могут быть 
объявлены сосредоточенными массами. Использование сосредоточенных масс 
позволяет исключить все подчиненные конструкции из распределения по 
отсекам и~рассматривать их отдельно для задания сосредоточенных нагрузок. 
Список сосредоточенных масс с~уникальными именами представляет собой 
отдельную структуру весовой модели. Каждая сосредоточенная масса содержит 
список ссылок на конструкции весовой модели. Любая конструкция может 
быть включена не более чем в~одну сосредоточенную массу.
  
  \textbf{Весовые сводки.} Одной из основных задач \mbox{АСВП} является 
построение так называемых весовых сводок. Весовые сводки являются 
документами, сопровождающими построение весовой модели самолета 
в~процессе его создания. В АСВП реализовано несколько форм весовых 
сводок, которые с~разных сторон отражают дерево конструкции самолета или 
отдельных ветвей этого дерева. Назначение этих сводок и~форма представления 
зависят от ре\-ша\-емых задач. Весовые данные в~сводках могут быть 
представлены либо в~табличном виде, либо в~виде иерархии конструкций. 
Могут содержать информацию в~детализированном или в~укрупненном виде. 
Отдельные виды весовых сводок пред\-став\-ля\-ют распределенные 
характеристики по отсекам. Весовые сводки предназначены для решения задач 
весового контроля и~весового анализа. 
  
  Весовой контроль при проектировании самолетов представляет собой  
ор\-га\-ни\-за\-ци\-он\-но-тех\-ническую сис\-те\-му, нацеленную на создание 
конструк\-ции минимального веса. Для эффективного \mbox{весового} контроля 
необходима оперативная информация о текущей массе изделия и~любой его 
части. Весовая информация для весового контроля в~АСВП представляется 
в~виде оперативных весовых сводок по отдельным подразделениям 
предприятия. В~таких весовых сводках содержится информация о текущей, 
теоретической, лимитной,\linebreak чертежной и~фактической массах конструкций, 
разрабатываемых данным подразделением. Могут также выпускаться 
оперативные сводки по группе подразделений или по всему проекту. Сводки 
весового контроля предназначены для использования руководителями проекта.
  
  Весовой анализ также связан с~выпуском определенного вида весовых 
сводок. Для решения задач весового анализа в~АСВП осуществляется 
сортировка и~выборки конструкций по определенному классификационному 
признаку. Например, могут быть рассчитаны массы силового и~несилового 
набора конструкции, массы продольного и~поперечного набора, массы 
конструкций определенного материала, массы готовых изделий или изделий 
конкретного поставщика и~т.\,д. Весовой контроль и~анализ позволяют 
выявить резервы конструкции, узкие места, тренды в~изменении веса 
кон\-ст\-рук\-ции.
{\looseness=1

}
  
  \subsection{Постоянные и~переменные структуры весовой модели 
самолета}
  
  Дерево конструкции весовой модели готового изделия не является 
статической структурой. Конфигурация самолета зависит от конкретных 
условий его применения. Мас\-со\-во-инер\-ци\-он\-ные характеристики при 
взлете и~посадке отличаются от тех же характеристик в~полете, когда убраны 
стойки шасси. Конфигурация меняется и~в~полете у~самолетов с~изменяемым 
углом стреловидности или с~измененяемым вектором тяги. Текущая 
конфигурация является одним из параметров весовой модели и~параметров 
весовых расчетов. По самому смыс\-лу создания самолета как транспортного 
средства предполагается, что кроме собственно конструкции, которая 
обеспечивает выполнение основных задач, на его  
МИХ существенным образом влияет 
перевозимая нагрузка. Перевозимая нагрузка есть переменная часть структуры 
дерева конструкции. Самолетные весовые классификаторы выделяют 
постоянную часть массы изделия и~переменную, состоящую из снаряжения, 
топлива и~целевой нагрузки:
  \begin{multline*}
{M} = M_{\mathrm{пустого}} + 
M_{\mathrm{снаряжения}} + {}\\
{}+M_{\mathrm{топлива}} + 
M_{\mathrm{целевой\_нагрузки}}\,.
  \end{multline*}
  
  Все переменные и~постоянные компоненты самолета составляют единое 
целое, и~расчет мас\-со\-во-инер\-ци\-он\-ных и~центровочных характеристик 
допусти\-мых конфигураций является одной из главных задач проектирования 
самолетов любого назначения. Переменные структуры в~весовой модели могут 
задаваться альтернативными конструкциями, когда у некоторой вершины 
дерева есть несколько вариантов поддеревьев и~когда любой из вариантов, но 
только один из них, может быть активирован в~конкретный момент времени. 
Существует своя специфика задания переменных структур весовой модели для 
разных содержательных задач. 
  
  \textbf{Пустой самолет}~--- это постоянная часть конструкции самолета, 
которая не меняется в~процессе эксплуатации готового изделия. Компонентами 
пустого самолета являются конструкция планера самолета, силовая установка 
и~ее системы, другие самолетные системы, обеспечивающие управление 
самолетом, а~также специальные системы бортового оборудования, 
предназначенные для решения целевых задач самолета. В~процессе 
проектирования и~при эксплуатации самолетов рассматриваются различные 
варианты отдельных конструкций планера, а~чаще~--- различные варианты 
по\-став\-ля\-емых готовых изделий. В~связи с~этим в~весовой модели АСВП 
рассматриваются возможные комбинации вариантов пустого самолета, 
вариантов снаряжения и~полезной нагрузки. 

\begin{figure*} %fig3
\vspace*{1pt}
 \begin{center}
 \mbox{%
 \epsfxsize=162mm 
 \epsfbox{fle-3.eps}
 }
 \end{center}
\vspace*{-9pt}
\Caption{Тарировочная таблица топливного бака}
\end{figure*}
  
  \textbf{Снаряжение самолета} устанавливается на борту в~процессе 
предполетной подготовки. Снаряжение самолета принято разделять на 
основное и~дополнительное. Основное снаряжение включает несколько 
позиций. Это экипаж и~системы жизнеобеспечения экипажа, системы 
жизнеобеспечения пассажиров, заправляемые компоненты и~расходуемые 
материалы, несливаемый остаток топлива и~другие возможные компоненты. 
Использование различных вариантов экипажа и~другого снаряжения самолета 
связано с~различным характером выполняемых задач. Как правило, существует 
несколько типовых вариантов комплектации экипажа 
и~элементов снаряжения. Весовая модель должна содержать перечень 
альтернативных вариантов снаряжения и~их характеристик. Естественно, что 
этот перечень может модифицироваться. К~дополнительному снаряжению 
относят временное оборудование и~средства, связанные с~установкой на борту 
и~закреплением на подвесках перевозимых грузов. Временно устанавливаемое 
оборудование, как правило, связано со спецификой полетных заданий. Это 
может быть специальная измерительная аппаратура или оборудование, которое 
необходимо проверить в~условиях реального полета. Перечень такого 
оборудования и~его характеристики в~весовой модели должны быть 
пред\-став\-ле\-ны в~специальном реестре, или в~каталоге. Для установки 
оборудования, размещения любой коммерческой нагрузки и~вооружения в~конструкции самолета
должны быть  предусмотрены специальные места 
размещения и~узлы крепления. Точки размещения оборудования и~любых 
элементов целевой нагрузки задаются своими координатами и~установочными 
углами закрепления. 

\begin{figure*} %fig4
  \vspace*{1pt}
 \begin{center}
 \mbox{%
 \epsfxsize=162mm 
 \epsfbox{fle-4.eps}
 }
 \end{center}
\vspace*{-11pt}
\Caption{Варианты размещения целевой нагрузки самолета на подвесках}
\end{figure*}
  
  \textbf{Топливо}~--- величина переменная и~на земле, при подготовке 
самолета к~вылету, и~в~воздухе, при выработке топлива, и, если это 
предусмотрено, при дозаправке в~воздухе. Одной из самых сложных и~важных 
задач построения весовой модели является отражение изменяющихся в~полете  
МИХ топлива, находящегося 
в~топливных баках. Топливные баки современных ЛА
могут иметь довольно сложные геометрические формы. В~процессе выработки 
топлива все характеристики располагаемого запаса топлива меняются. 
Необходимо отслеживать эти изменения в~процессе произвольных допустимых 
эволюций траектории полета. Функции изменения МИХ в~зависимости от 
объема оставшегося топлива задаются тарировочными характеристиками баков. 
Для расчета тарировочных характеристик топливных баков при произвольных 
углах атаки, углах тангажа и~крена в~весовой модели наиболее удобно 
триангуляционное задание баков. В~тарировочной таблице вычисляется масса 
оставшегося топлива в~зависимости от уровня поверхности жидкости 
в~топливном баке. На рис.~3 приведен пример расчета тарировочной таблицы 
крыльевого топливного бака.



  Если МИХ топлива в~конкретном баке по 
мере его выработки определяются тарировочной характеристикой, то 
зависимость МИХ оставшегося топлива определяется последовательностью, 
в~которой осуществляется выработка из разных баков. Топливная система 
самолета состоит из нескольких баков~--- как внутренних, так и~размещенных 
на подвесках, а~также из системы трубопроводов, перекачивающих насосов и~управляющей автоматики. Основой управления расходом топлива является 
программа, определяющая порядок расходования топлива из разных баков. 
Переключение перекачки топлива между разными баками осуществляется для 
обеспечения центровки самолета в~заданных границах. Одним из критериев при 
разработке алгоритмов перекачки является число переключений и~обеспечение 
бесперебойной подачи топлива при любых допустимых параметрах траектории 
полета. Массово-инерционные характеристики топлива в~процессе тарировки 
баков задаются их разбиением плоскопараллельными сечениями на тонкие 
слои. Для каждого слоя указывается масса, координаты центра тяжести 
и~плоскостные моменты инерции. Программа выработки топлива пред\-став\-ля\-ет 
собой последовательность выработки слоев из разных баков в~соответствии 
с~диаграммой переключений. В~весовой модели может быть задано несколько 
вариантов программ расходования топлива. Разумеется, в~процессе выполнения 
полетного задания программа расходования топлива фиксирована. 
Предварительный расчет характеристик для разных вариантов порядка 
выработки топлива необходим для выбора наилучшего, удовле\-тво\-ря\-юще\-го 
всем ограничениям.
  
  \textbf{Целевая нагрузка} зависит от назначения самолета и~от конкретного 
полетного задания. Для пасса\-жирских самолетов целевая нагрузка~--- это 
пассажи\-ры с~багажом, для транспортных са\-мо\-летов~--- это коммерческие 
грузы, для военных~--- подвесное или размещаемое в~специальных \mbox{отсеках} 
вооружение. В~полете возможен сброс и~десантирование целевой нагрузки. 
Комплектация и~установка целевой нагрузки представляет собой довольно 
сложный процесс. Выбор состава грузов и~их размещение могут проходить 
в~несколько этапов. Сложность выбора обусловлена большим количеством 
типов перевозимой нагрузки, наличием большого числа вспомогательных 
специальных устройств закрепления грузов как во внутренних отсеках 
самолета, так и~на внешних подвесках. На рис.~4 приведена панель 
формирования различных расчетных вариантов целевой нагрузки самолета. 
Визуализация этого процесса существенно облегчает решение различных задач 
анализа допустимой нагрузки как на этапе проектирования самолета, так и~при 
эксплуатации во время подготовки полетных заданий.
  
  \begin{figure*} %fig5
\vspace*{1pt}
 \begin{center}
 \mbox{%
 \epsfxsize=162mm 
 \epsfbox{fle-5.eps}
 }
 \end{center}
\vspace*{-9pt}
\Caption{Область допустимых центровок}
\end{figure*}

  Для удобства выбора и~проведения расчетов множества вариантов загрузки 
самолета в~рамках весовой модели реализованы каталоги нагрузки~--- 
специального оборудования, коммерческой нагрузки, вооружения. В~этих 
каталогах ведутся клас\-си\-фи\-ка\-то\-ры, позволяющие в~громадных переч\-нях 
находить нужные позиции и~их характеристики. Кроме  
МИХ размещаемой нагрузки в~каталогах 
даются ссылки на их геометрические модели, задаются габариты, другие 
геометрические па\-ра\-мет\-ры. Эти данные нужны для визуализации размещения 
и~компоновки, для вычисления их МИХ. 
Как правило, существуют довольно жесткие ограничения на 
размещение нагрузки на борту, а~также на внешних узлах крепления. Эти 
ограничения должны указываться в~каталоге и~учитываться в~процессе 
формирования вариантов загрузки самолета. 
  
  Ограничения, которые проверяются при анализе различных вариантов 
снаряжения самолета, программы выработки топлива и~допустимых вариантов 
целевой нагрузки, задают область допустимых центровок самолета. 
  
  \textbf{Область допустимых центровок} является неотъемлемой частью 
весовой модели и~служит одной из основных весовых характеристик самолета, 
особенно важной и~контролируемой в~процессе его эксплуатации. На рис.~5 
проиллюстрированы ограничения, образующие область допустимых центровок, 
и~приведен график изменения центровки самолета при выработке топлива. 



  По оси абсцисс на этом графике откладывается центровка самолета, которая 
определяется как положение центра тяжести самолета на средней 
аэро\-ди\-на\-ми\-че\-ской хорде в~процентах от ее длины. По оси ординат 
откладывается текущая масса самолета с~учетом массы снаряжения, массы 
целевой нагрузки и~текущего запаса топлива. Точки излома на графиках 
центровки соответствуют моментам переключения подачи топлива с~одного 
бака на\linebreak другой, которые определяются программой выработки топлива или 
моментами сброса целевой нагрузки. Двойной график изменения центровки 
соответствует двум полетным конфигурациям~--- с~убранными 
и~выпущенными стойками шасси. Ограничения, которые обеспечивают 
устой\-чи\-вость и~управ\-ля\-емость полета, задаются предельными значениями 
центровки. Предельно передняя и~предельно задняя центровки на графике 
показаны вертикаль\-ными штриховыми линиями. Горизонтальные линии задают 
ограничения на взлетную и~посадочную массы. Ограничения максимальной 
взлетной и~посадочной массы при определенных условиях могут нарушаться, 
но эти нарушения допускаются в~исключительных условиях и~сказываются на 
ресурсных характеристиках самолета.\linebreak Превышение \textbf{предельных} 
значений взлетной и~посадочной массы не допускается. Наклонные штриховые 
линии на графике задают ограничения, связанные с~максимально допустимыми 
нагрузками на переднюю и~главную опоры шасси.  

\begin{figure*} %fig6
\vspace*{1pt}
 \begin{center}
 \mbox{%
 \epsfxsize=165mm 
 \epsfbox{fle-6.eps}
 }
 \end{center}
\vspace*{-9pt}
\Caption{Архитектура программной реализации исполнительных модулей АСВП}
\end{figure*} 

%\vspace*{-12pt}

\section{Программная реализация автоматизированной системы весового
проектирования}

  Представленная здесь струк\-тур\-но-па\-ра\-мет\-ри\-че\-ская весовая модель 
самолета позволяет решать широкий круг задач весового проектирования. 
Весовая модель составляет информационную основу,\linebreak на базе которой могут 
быть построены различные вычислительные программы и~пользовательские 
модули. Рассматриваемая в~данной работе АСВП построена по 
кли\-ент-сер\-вер\-ной архитектуре, где весовая модель служит единым хранилищем 
информации, базой данных системы. Непосредственно с~информацией, 
хранящейся в~этой базе данных, взаимодействуют различные вычислительные, 
расчетные программы~--- серверы, которые кроме расчетных функций 
обеспечивают информационную связь клиентских модулей с~весовой моделью 
самолета. Непосредственными пользователями клиентских модулей являются 
конструкторы и~проектировщики, решающие различные задачи весового 
проектирования.  Построена АСВП как многопользовательская интерактивная 
система. На рис.~6 представлена архитектура АСВП, ее основные программные 
и~информационные компоненты.




  Ниже перечислены основные функции программных модулей АСВП:
 \begin{description}
 \item[\,] 
Сервер ПУСТОЙ ЛА\;+\;Модуль расчета МИХ пус\-то\-го самолета:
\begin{itemize}
\item создание и~модификация дерева конструкции пустого самолета;
\item расчет МИХ пустого изделия, всех его сис\-тем, узлов, агрегатов и~деталей 
на любых уровнях дерева конструкции;
\item весовой анализ и~контроль текущего состояния проекта, выполнения 
лимитных ограничений по весу, осуществление выборок весовой информации 
по различным признакам~--- сис\-те\-мам, агрегатам, типу конструкции 
(си\-ло\-вая/не\-си\-ло\-вая),  материалу конструкции, подразделениям и~т.\,д.;
\item расчет распределения массы самолета по различным разбиениям на 
отсеки; эта информация используется для построения динамически подобных 
моделей и~при прочностных расчетах;
\item расчет МИХ при различных вариантах полетной конфигурации при 
убранных и~выпущенных стойках шасси, при отклонениях консолей крыла для 
самолетов с~из\-ме\-ня\-емой геометрией, при отклонении органов управления.
\end{itemize}
\begin{figure*} %fig7
\vspace*{1pt}
 \begin{center}
 \mbox{%
 \epsfxsize=155.86mm 
 \epsfbox{fle-7.eps}
 }
 \end{center}
\vspace*{-1pt}
\Caption{Проектный подход~--- технология ГП}
\vspace*{6pt}
\end{figure*}
 \item[\,]
Сервер НАГРУЗКА ЛА\;+\;Модуль расчета МИХ самолета с~переменной 
массой:
\begin{itemize}
\item создание и~модификация реестра допустимых вариантов нагрузки 
самолета;
\item расчеты МИХ снаряженного и~загруженного самолета для разных 
вариантов компоновки и~размещения на борту полезной нагрузки;
\item расчет изменения МИХ самолета в~полете при выработке топлива, 
дозаправке в~воздухе, сбросе нагрузки;
\item расчет МИХ самолета в~виде табличных зависимостей для различных 
вариантов снаряжения и~размещения нагрузки;
\item расчет МИХ самолета в~виде графических зависимостей от массы 
самолета и/или от массы топлива;
\item проверка выполнения установленных эксплуатационных ограничений по 
центровке, взлетной и~посадочной массе, нагрузке на опоры шасси для 
различных вариантов снаряжения и~размещения нагрузки; сигнализация 
в~случае нарушения ограничений, а~также для различных вариантов программ 
выработки топлива.
\end{itemize}

\pagebreak

 \item[\,]
Сервер КАТАЛОГ\;+\;Модуль ведения каталога элементов нагрузки:\\[-9pt]
\begin{itemize}
\item создание и~модификация каталога элементов целевой нагрузки самолета;\\[-9pt]
\item создание и~модификация базы данных вариантов размещения 
и~закрепления элементов нагрузки каталога на борту самолета или на подвесках;\\[-9pt]
\item создание и~модификация базы данных вспомогательных элементов 
конструкции установки элементов нагрузки.\\[-9pt]
\end{itemize}
 \item[\,]
Сервер ТОПЛИВО\;+\;Модуль расчета порядка выработки топлива:\\[-9pt]
\begin{itemize}
\item создание и~модификация базы данных различных вариантов программы 
выработки топлива;\\[-9pt]
\item расчет МИХ и~МЦД для различных вариантов переключения выработки 
топлива из внутренних, закладных и~подвесных баков;\\[-9pt]
\item расчет МИХ и~МЦД при различных программах заливки и~дозаправки 
топлива во внутренние, закладные и~подвесные баки.\\[-9pt]
\end{itemize}
 \item[\,]
Сервер БАКИ\;+\;Модуль расчета тарировки топливных баков:\\[-9pt]
\begin{itemize}
\item создание и~модификация базы данных гео\-мет\-рии топливных баков;\\[-9pt]
\item расчет тарировочных характеристик топливных баков при различных 
углах тангажа и~крена.\\[-9pt]
  \end{itemize}
  \end{description}
  
  Программная реализация АСВП велась с~использованием инструментального комплекса 
<<Генератор проектов>> (технология ГП)~\cite{5-fl}. Технология ГП 
обеспечивает возможность разработки приклад\-ных систем многоуровневой  
кли\-ент-сер\-вер\-ной архитектуры с~использованием реляционных и~сетевых 
баз данных со сложным пользовательским и~межпрограммным интерфейсом. 
Создание ин\-фор\-ма\-ци\-он\-но-вы\-чис\-ли\-тель\-ных сис\-тем в~рамках 
технологии ГП базируется на так называемом <<проектном подходе>>. Под 
проектом здесь понимается пакет документов (файлов), содержащий описание 
структуры проекта, описание логической структуры баз данных, спецификации 
пользовательского интерфейса, перечень команд и~сценарии работы 
пользователей, описание функций и~процедур обработки пользовательских 
запросов. Исходное описание проекта подается на вход <<Генератора 
проекта>>, который строит в~памяти модель проекта, осуществляет ее анализ 
на предмет корректности и~целостности, а затем на основании этой модели 
генерирует тексты программного кода для клиентских и~серверных компонент 
системы, а~так\-же ге\-нерирует утилиты, необходимые для сборки, инсталляции 
и~сопровождения системы. 

На рис.~7 показана общая архитектура 
программной конструкции, связанной с~применением технологии ГП.
  


  В приведенной цепочке разработчик прикладной информационной системы 
имеет дело только с~первым ее звеном~--- проектом системы. При этом он 
избавлен от необходимости иметь дело с~системным программным окружением 
вычислительной среды, в~которой должна функционировать разрабатываемая 
прикладная система. Все связи прикладных информационных процессов 
с~конкретной системной вычислительной средой привносит 
в~результирующую рабочую программу <<Генератор проектов>> на стадии 
анализа и~генерации итогового программного кода. Естественно, что при этом 
объем описания проекта оказывается существенно короче программного кода, 
который создается автоматически. Экономия трудозатрат разработчика 
оказывается существенной. В~частности, объем описания проекта АСВП на 
порядок меньше, чем объем сгенерированного программного кода. Даже если 
предположить, что написанный вручную программный код благодаря 
искусству программистов будет весьма экономным, то все равно трудоемкость 
разработки прикладных систем будет в~разы меньше. 

Но главное даже не 
в~числе строк программ, а~прежде всего в~экономии интеллектуальных затрат 
разработчиков прикладных систем и,~в~итоге, автоматически созданные 
программы более надежны и~свободны от нечаянных ошибок и~опечаток.\linebreak 
И~кроме того, разрабатываемые в~рамках технологии ГП прикладные системы 
обеспечивают-\linebreak\vspace*{-12pt}

\pagebreak

\noindent
ся эффективными средствами сопровождения, т.\,е.\linebreak достаточно 
простой процедурой внесения ис\-прав\-ле\-ний и~развития программ в~процессе их 
эксплу\-а\-тации. 

Прикладные программные комплексы в~рамках технологии ГП 
разрабатываются как автономные системы и~не требуют для своей работы 
специальной среды и~дорогостоящих программных продуктов (кроме 
использующихся систем управления базами данных
(СУБД) и~общесистемного обеспечения). Разрабатываемые 
в~рамках технологии ГП прикладные системы допускают масштабирование 
и~портирование на различные вычислительные платформы и~СУБД.
  
  \bigskip
  
  Как уже говорилось, система АСВП разрабатывалась в~течение ряда лет, 
многие ее компоненты и~версии были апробированы и~использовались 
в~реальном проектировании. 
%
Авторы выражают благодарность 
С.\,И.~Скобелеву, М.\,К.~Курьянскому, Д.\,Ю.~Стрель\-цу, П.\,В.~Плунскому 
и~К.\,Н.~Ерасову за плодотворные обсуждения проблем весового проектирования 
самолетов, за постановку многих задач и~за апробацию разработанных 
программ.

%\vspace*{-12pt}

{\small\frenchspacing
 {%\baselineskip=10.8pt
 \addcontentsline{toc}{section}{References}
 \begin{thebibliography}{9}
\bibitem{1-fl}
\Au{Шейнин В.\,М., Козловский~В.\,И.} Весовое проектирование и~эффективность 
пассажирских самолетов.~--- М.: Машиностроение, 1977.   Т.~1. 343~с.

%\columnbreak

\bibitem{2-fl}
\Au{Скобелев С.\,И., Широков~Н.\,И.} Весовой анализ и~контроль в~САПР ЛА~// Задачи 
и~методы автоматизированного проектирования.~--- М.: ВЦ РАН, 1991. С.~92--100.
\bibitem{3-fl}
\Au{Широков Н.\,И.} Автоматизированная система весовых расчетов в~САПР ЛА~// 
Автоматизация проектирования инженерных и~финансовых информационных систем 
средствами Генератора проектов~/ Отв. ред. Ю.\,А.~Флеров.~--- М.: ВЦ РАН, 
2010. С.~55--66.

\vspace*{6pt}

\bibitem{4-fl}
\Au{Вышинский Л.\,Л., Широков~Н.\,И.} Система автоматизации расчетов 
массово-инерционных характеристик ЛА с~переменной массой~// Развитие и~применение 
инструментального комплекса Генератор проектов~/ Отв. ред. Ю.\,А.~Флеров.~--- 
М.: ВЦ РАН, 2014. С.~20--31.
{\looseness=1

}

\vspace*{6pt}

\bibitem{5-fl}
\Au{Вышинский Л.\,Л., Гринев~И.\,Л., Флеров~Ю.\,А., Широков~А.\,Н., Широков~Н.\,И.} 
Генератор проектов~--- инструментальный комплекс для разработки  
<<кли\-ент-сер\-вер\-ных>> сис\-тем~// Информационные технологии и~вычислительные 
системы, 2003. №\,1-2. С.~6--25.
 \end{thebibliography}

 }
 }

\end{multicols}

\vspace*{-6pt}

\hfill{\small\textit{Поступила в~редакцию 24.05.17}}

\vspace*{8pt}

%\newpage

%\vspace*{-24pt}

\hrule

\vspace*{2pt}

\hrule

%\vspace*{8pt}


\def\tit{COMPUTER-AIDED SYSTEM OF~AIRCRAFT WEIGHT DESIGN}

\def\titkol{Computer-aided system of~aircraft weight design}

\def\aut{L.\,L.~Vyshinsky, Yu.\,A.~Flerov, and~N.\,I.~Shirokov}

\def\autkol{L.\,L.~Vyshinsky, Yu.\,A.~Flerov, and~N.\,I.~Shirokov}

\titel{\tit}{\aut}{\autkol}{\titkol}

\vspace*{-9pt}


\noindent
A.\,A.~Dorodnicyn Computing Centre, Federal Research Center ``Computer Science and 
Control'' of the Russian Academy of Sciences,  40~Vavilov Str., Moscow 119333, Russian 
Federation 



\def\leftfootline{\small{\textbf{\thepage}
\hfill INFORMATIKA I EE PRIMENENIYA~--- INFORMATICS AND
APPLICATIONS\ \ \ 2018\ \ \ volume~12\ \ \ issue\ 1}
}%
 \def\rightfootline{\small{INFORMATIKA I EE PRIMENENIYA~---
INFORMATICS AND APPLICATIONS\ \ \ 2018\ \ \ volume~12\ \ \ issue\ 1
\hfill \textbf{\thepage}}}

\vspace*{3pt}
   

\Abste{The article is devoted to the problems of computer-aided weight design of 
aircraft. Weight and mass-inertial parameters are one of the basic values that affect 
the performance characteristics of aircraft. The informational basis of the system is 
the weight model of the aircraft. The paper describes the structure of the weight 
model and its individual components. The program implementation of the system, 
which is executed within the framework of the client-server architecture, is shown. 
The automated system of weight design is implemented using the software tool 
complex ``Project Generator'' (GP technology), which was developed at the 
Computing Centre of the Russian Academy of Sciences. The creation of information 
and computing systems within the framework of the GP technology is based on the 
so-called ``project approach,'' when the formal description of the system 
automatically generates code for the client and server components of the system.}

\KWE{math modeling; design automation; aircraft; weight design; weighting model; 
design tree; project generator; code generation; client-server architecture}

  \DOI{10.14357/19922264180103} 

%\vspace*{-12pt}

%\Ack
%\noindent




%\vspace*{8pt}

  \begin{multicols}{2}

\renewcommand{\bibname}{\protect\rmfamily References}
%\renewcommand{\bibname}{\large\protect\rm References}

{\small\frenchspacing
 {%\baselineskip=10.8pt
 \addcontentsline{toc}{section}{References}
 \begin{thebibliography}{9} 
 
 %\vspace*{-6pt}
 
 \bibitem{1-fl-1}
\Au{Sheynin, V.\,M., and V.\,I.~Kozlovskiy}. 1977. \textit{Vesovoe 
proektirovanie i~effektivnost' passazhirskikh samoletov} [Weight design and 
efficiency of passenger aircraft]. Moscow: Mechanical Engineering. Vol.~1. 343~p.
\bibitem{2-fl-1}
\Aue{Skobelev, S.\,I., and N.\,I.~Shirokov.} 1991. Vesovoy analiz i~kontrol' v~SAPR 
LA [Weight analysis and control in CAD of aircraft]. \textit{Zadachi i~metody 
avtomatizirovannogo proektirovaniya} [Tasks and methods of computer-aided 
design]. Moscow: Computing Centre of the USSR Academy of Sciences.  
92--100.
\bibitem{3-fl-1}
\Aue{Shirokov, N.\,I.} 2010. Avtomatizirovannaya sistema vesovykh raschetov 
v~SAPR LA [Automated system weight calculations in CAD].  
\textit{Avtomatizatsiya proektirovaniya inzhenernykh i~finansovykh 
informatsionnykh system sredsvami Generatora proektov} [Computer 
aided  design of engineering and financial information systems by the means of the 
Project Generator]. Moscow: Computing Centre of RAS. 
55--66.
\bibitem{4-fl-1}
\Aue{Vyshinskiy, L.\,L., and N.\,I.~Shirokov.} 2014. Sistema avtomatizatsii 
raschetov massovo-inertsionnykh kharakteristik LA s~peremennoy massoy [CAD 
system of calculation  aircraft mass-inertial characteristics with variable mass].  
\textit{Razvitie i~primenenie instrumental'nogo kompleksa Generator proektov} 
[The development and application of a tool set Project Generator]. 
Moscow: Computing Centre of RAS. 20--31.
{\looseness=1

}

\bibitem{5-fl-1}
\Aue{Vyshinskiy, L.\,L., I.\,L.~Grinev, Yu.\,A.~Flerov, A.\,N.~Shirokov, and 
N.\,I.~Shirokov.} 2003. Generator proektov~--- instrumental'nyy kompleks dlya 
razrabotki ``klient--servernykh'' sistem [The project generator~--- tool complex for 
development of ``client--server'' systems]. 
\textit{Informatsionnye tekhnologii i~vychislitel'nye sistemy} [Information 
Technologies and Computer Systems] 1-2:6--25.

\end{thebibliography}

 }
 }

\end{multicols}

\vspace*{-6pt}

\hfill{\small\textit{Received May 24, 2017}}

%\vspace*{-10pt}

\Contr

\noindent
\textbf{Vyshinsky Leonid L.} (b.\ 1941)~--- Candidate of Sciences (PhD) in physics and 
mathematics, Head of Laboratory, A.\,A.~Dorodnicyn Computing 
Centre, Federal Research Center ``Computer Science and Control'' of the Russian 
Academy of Sciences, 40~Vavilov Str., Moscow 119333, Russian Federation; 
\mbox{Wysh@ccas.ru} 

\vspace*{3pt}

\noindent
\textbf{Flerov Yuri A.} (b.\ 1942)~--- Corresponding Member of the Russian 
Academy of Science, Doctor of Science in physics and mathematics, professor, 
Deputy Director, A.\,A.~Dorodnicyn Computing Centre, Federal Research Center 
``Computer Science and Control'' of the Russian Academy of Sciences, 40~Vavilov 
Str., Moscow 119333, Russian Federation; \mbox{fler@ccas.ru}

\vspace*{3pt}

\noindent
\textbf{Shirokov Nikolai I.} (b.\ 1963)~--- Candidate of Sciences (PhD) in physics and 
mathematics, senior scientist, A.\,A.~Dorodnicyn Computing Centre, Federal 
Research Center ``Computer Science and Control'' of the Russian Academy of 
Sciences, 40~Vavilov Str., Moscow 119333, Russian Federation; 
\mbox{Wysh@ccas.ru} 



\label{end\stat}


\renewcommand{\bibname}{\protect\rm Литература}        %7
\def\stat{agalarov}

\def\tit{ОБ ОДНОПОРОГОВОМ УПРАВЛЕНИИ ОЧЕРЕДЬЮ В~СИСТЕМЕ МАССОВОГО 
ОБСЛУЖИВАНИЯ С~НЕТЕРПЕЛИВЫМИ ЗАЯВКАМИ}

\def\titkol{Об однопороговом управлении очередью в~системе массового 
обслуживания с~нетерпеливыми заявками}

\def\aut{Я.\,М.~Агаларов$^1$}

\def\autkol{Я.\,М.~Агаларов}

\titel{\tit}{\aut}{\autkol}{\titkol}

\index{Агаларов Я.\,М.}
\index{Agalarov Ya.\,M.}


%{\renewcommand{\thefootnote}{\fnsymbol{footnote}} \footnotetext[1]
%{Работа выполнялась с~использованием инфраструктуры Центра коллективного пользования <<Высокопроизводительные вы\-чис\-ле\-ния и~большие данные>> 
%(ЦКП <<Информатика>>) ФИЦ ИУ РАН.}}


\renewcommand{\thefootnote}{\arabic{footnote}}
\footnotetext[1]{Федеральный исследовательский центр <<Информатика и~управление>> Российской академии наук, 
\mbox{agglar@yandex.ru}}


\vspace*{-6pt}

  
  
  \Abst{Изложены результаты теоретического исследования управ\-ля\-емой системы 
массового обслуживания (СМО) типа $M/M/s$ с~нетерпеливыми заявками и~однопороговым 
управлением очередью. Ставится задача оптимизации однопорогового управления очередью, 
суть которой заключается в~вычислении для длины очереди некоторого порогового значения, 
максимизирующего заданную целевую функцию. В~исследуемой сис\-те\-ме заявка покидает 
систему необслуженной, если время ожидания в~очереди (или время обслуживания на приборе) 
превышает некоторый интервал времени случайной длины, распределенной по показательному 
закону с~заданным параметром. В~качестве показателя эф\-фек\-тив\-ности управ\-ле\-ния очередью 
(целевой функции) используется стоимостная функция, учитывающая потери в~единицу 
времени из-за технического обслуживания сис\-те\-мы, отклонения заявок на входе сис\-те\-мы, 
ухода заявок до завершения обслуживания. Предложены метод решения задачи максимизации 
стоимостной целевой функции на множестве однопороговых управ\-ле\-ний очередью 
и~алгоритм гарантированного вычисления оптимального порога. }
  
  
  \KW{система массового обслуживания; нетерпеливые заявки; управ\-ле\-ние очередью}
  
\DOI{10.14357/19922264240206}{JZHAKU}
  
\vspace*{6pt}


\vskip 10pt plus 9pt minus 6pt

\thispagestyle{headings}

\begin{multicols}{2}

\label{st\stat}

  
\section{Введение}

\vspace*{-6pt}


  Настоящая работа служит продолжением исследований, посвященных 
проблеме оптимизации порогового управления очередью в~СМО с~учетом 
стоимостных потерь из-за отклонения и~задержек заявок, а также затрат на 
техническое обслуживание системы. Суть порогового управления очередью 
заключается в~том, что для длины очереди задается одно или несколько 
пороговых значений, по достижении каждого из которых принимается 
соответствующее решение по сбросу нагрузки из очереди с~\mbox{целью} 
повышения эффективности работы сис\-те\-мы~[1]. 
  
  Ниже будем рассматривать оптимизационную задачу управления очередью для 
простейшей СМО, у~которой ограничено время 
пребывания заявки в~очереди или на приборе. Заявка покидает сис\-те\-му 
необслуженной, если время ожидания в~очереди или на приборе превышает 
некоторую случайную\linebreak величину с~заданным средним значением. В~прос\-тей\-шей 
модели системы такого типа предполагают, что заявки покидают очередь через 
случайные интервалы времени, распределенные по \mbox{показательному} закону, т.\,е.\ 
возникает поток уходящих из очереди с~постоянной интенсивностью заявок. 
Таким образом, каж\-дая заявка, на\-хо\-дя\-ща\-яся в~очереди или на приборе, может 
покинуть систему, не дождавшись обслуживания, через случайный интервал 
времени, распределенный по показательному закону. Заявки в~этом случае 
называют <<нетерпеливыми>>, а~СМО~--- сис\-те\-мой с~<<нетерпеливыми>> 
заявками. Такая СМО имеет четыре потока, влия\-ющих на со\-сто\-яние сис\-те\-мы: 
входной поток заявок, поток обслуженных заявок, поток заявок, по\-ки\-да\-ющих 
очередь, не дождавшись начала обслуживания, и~поток уходящих с~приборов 
заявок, не дождавшихся завершения обслуживания. Так как поток уходящих 
заявок пуассоновский, то процесс, протекающий в~сис\-те\-ме под влиянием такого 
потока, будет марковским. 
  
  С увеличением порога длины очереди, с~одной стороны, увеличивается поток 
заявок~--- потенциальных плательщиков за обслуживание, с~другой~--- 
увеличиваются потери сис\-те\-мы (из-за увеличения задержек заявок, ухода 
<<нетерпеливых>> заявок, затрат сис\-те\-мы на хранение и~обслуживание 
заявок). Возникает задача поиска значения порога длины очереди, 
максимизирующего доход сис\-темы.
  
  Результаты теоретических и~экспериментальных исследований по 
рассматриваемой в~данной \mbox{статье} проб\-ле\-ме, изложенные в~ранее 
опубликованных работах, получены для задачи оптимизации порогового 
управления очередью в~одноканальных и~многоканальных СМО с~терпеливыми 
заявками (заявки не покидают сис\-те\-му до завершения обслуживания) (см., 
например,~[2--8]). При исследовании СМО с~управ\-ля\-емы\-ми очередями 
методами математического моделирования, как правило, требуется 
предварительно решить подзадачу расчета  
ве\-ро\-ят\-ност\-но-вре\-мен\-н$\acute{\mbox{ы}}$х характеристик (показателей) исследуемой  
сис\-те\-мы~[9--13] и~использовать данную расчетную модель для разработки 
и~исследования алгоритма управ\-ле\-ния очередью (очередями), что приводит 
к~математической модели с~более сложными функциональными 
взаимозависимостями параметров системы по сравнению с~расчетной моделью. 
В~научных пуб\-ли\-ка\-ци\-ях, посвященных исследованию сис\-тем 
с~нетерпеливыми заявками, отсутствуют результаты по оптимизации  
управ\-ле\-ния очередью, в~основном в~них рас\-смот\-ре\-ны задачи по расчету  
ве\-ро\-ят\-ност\-но-вре\-мен\-н$\acute{\mbox{ы}}$х характеристик и~оптимизации структуры таких 
сис\-тем~\cite{9-ag, 10-ag}.
  
  Ниже приводим метод и~результаты теоретического исследования 
однопорогового управления очередью системы $M/M/s$ с~<<нетерпеливыми>> 
заявками при стоимостном критерии оптимальности. 
  
\section{Постановка задачи }

  Рассматривается многоканальная СМО $M/M/s$ с~управляемой очередью, 
в~которой заявки, не дожидаясь завершения обслуживания, могут покинуть 
систему по истечении некоторого времени пребывания в~очереди или на приборе. 
Предполагается, что время, через которое заявка покидает сис\-те\-му,~--- 
показательно распределенная случайная величина, при этом параметр 
распределения равен~$\alpha_i$ ($\alpha_1\leq \alpha_2\leq\cdots$), если заявка  
\mbox{$i$-я} в~очереди, а~если на приборе, то параметр равен~$\beta$. 
Поступившая извне заявка допускается в~систему, если длина очереди в~системе 
меньше, чем $h\hm\geq 0$~--- некоторая заданная величина (пороговое значение), 
иначе отклоняется и~теряется. Допущенная в~систему заявка занимает любой из 
свободных приборов, если такой есть, иначе становится в~конец очереди. Будем 
считать, что заявки обслуживаются в~порядке поступления. Отметим, что 
поведение такой сис\-те\-мы описывается цепью Маркова, в~которой 
состоянием считается число заявок в~системе~\cite{9-ag}.
  
   Введем обозначения:
  \begin{description}
  \item[\,]  $\lambda$~--- интенсивность входного потока;
  \item[\,]  $\mu$~--- интенсивность обслуживания заявки на приборе;
  \item[\,]  $s$~--- число приборов в~системе;
  \item[\,]  $h+s$~--- объем накопителя;
   \item[\,]  $C_0$~--- плата заявки, принятой в~накопитель сис\-темы; 
  \item[\,]  $C_1$~--- стоимость потерь из-за отклонения заявки на входе системы;
  \item[\,]  $C_2$~--- стоимость потерь из-за ухода $i$-й заявки, находящейся в~очереди;
  \item[\,]  $C_3$~--- стоимость потерь из-за ухода с~прибора заявки, не 
дождавшейся завершения обслуживания;
  \item[\,]  $\pi_i^{(h)}$~--- стационарная вероятность состояния $i$ сис\-те\-мы 
при пороговом значении~$h$;
%  \item[\,]  $\overline{L}^{(h)} =s- \sum\nolimits^S_{i=1} (s-i)\pi_i^{(h)}$~--- среднее 
%значение длины очереди при пороге~$h$;
  \item[\,]  $\overline{S}^{(h)} =s-\sum\nolimits^S_{i=1} (s-i)\pi_i^{(h)}$~--- среднее 
чис\-ло занятых приборов;
  \item[\,]  $Q(h)$~--- доход сис\-те\-мы в~единицу времени при пороговом 
значении~$h$.
  \end{description}
  
  В качестве целевой функции задачи оптимизации порогового управления 
рассматривается предельный средний доход системы в~единицу времени, 
вычисляемый по формуле:
  \begin{multline}
  Q(h)=\lambda C_0\left( 1-\pi^{(h)}_{s+h}\right) -\lambda C_1\pi^{(h)}_{s+h}-{}\\
  {}- d\left( \overline{L}^{(h)}\right) -\beta C_3 \overline{S}^{(h)}.
  \label{e1-ag}
  \end{multline}
  Здесь $\lambda C_0(1\hm- \pi^{(h)}_{s+h})$~--- средняя суммарная плата 
заявок, принимаемых в~накопитель в~единицу времени; $\lambda 
C_1\pi^{(h)}_{s+h}$~--- средние потери в~единицу времени  
из-за отклонения заявок; $ d( \overline{L}^{(h)})$~--- средние потери 
в~единицу времени из-за ухода заявок из очереди:
$$
 d\left(  \overline{L}^{(h)}\right)= C_2\sum\limits^h_{i=1} \sum\limits^i_{j=1} 
\alpha_j \pi_{i+S}^{(h)};
$$
 $\beta C_3\overline{S}^{(h)}$~--- средние потери в~единицу 
времени из-за ухода с~приборов заявок, не дождавшихся завершения 
обслуживания.
  
  Ставится задача оптимизации порогового значения длины очереди, т. \,е.\ 
математическая задача вида
  \begin{equation}
  h^*= \argmax\limits_{0\leq h} Q(h)\,.
  \label{e2-ag}
  \end{equation}

\section{Метод решения и~результаты}

  Для стационарных вероятностей состояний описанной в~предыдущем разделе 
СМО справедливы равенства~\cite{9-ag}:
  \begin{equation}
  \left.
  \begin{array}{rl}
  \pi_l^{(h)} &=\fr{\rho^l}{l!(1+\gamma)^i}\,\pi_0^{(h)}\ \mbox{при}\ l\in \overline{1,s}\,,
  \\[6pt]
  \pi^{(h)}_{s+l} &=\fr{\rho^s}{s!(1+\gamma)^3} \prod\limits^l_{j=1} 
\fr{\rho}{s(1+\gamma) + j\theta_j}\,\pi_0^{(h)}\\[6pt]
 &\hspace*{27mm}\mbox{при}\ l\in \overline{1, h}\,,
 \end{array}
 \right\}
   \label{e3-ag}
  \end{equation}
  где
  
  \vspace*{-6pt}
  
  \noindent
\begin{multline*}
  \pi_0^{(h)} =\left[ 
  \vphantom{\prod\limits^l_{j=1}}
  1+\sum\limits_{m=1}^s \fr{\rho^m}{m!(1+\gamma)^m} + {}\right.\\
\left.   {}+
\fr{\rho^s}{s!(1+\gamma)^s} \sum\limits^h_{l=1} \prod\limits^l_{j=1} 
\fr{\rho}{s(1+\gamma)+j\theta_j}\right]^{-1};
\end{multline*}
  $$
  \rho=\fr{\lambda}{\mu}\,;\quad \gamma= \fr{\beta}{\mu}\,;\quad \theta_j= \fr{\alpha_j}{\mu}\,.
  $$
  
  Покажем, что для стационарных вероятностей справедливы соотношения
  \begin{align}
  \pi_l^{(h+1)} &= \left(1-\pi^{(h+1)}_{s+h+1}\right)\pi_l^{(h)},\ l=\overline{0, s+h}\,;
  \label{e4-ag}\\
  \pi^{(h+1)}_{s+h+1} &={}\notag\\
  &\hspace*{-10mm}{}= \left( 1-\pi_{s+h+1}^{(h+1)} \right) \pi^{(h)}_{s+h}  
\fr{\rho}{s(1+\gamma)+(h+1)\theta_{h+1}}\,.
  \label{e5-ag}
  \end{align}
  
  Из~(\ref{e3-ag}) при $l\hm=\overline{1,  h}$ следует 
  \begin{multline*}
  \pi^{(h)}_{s+l} -\pi_{s+l}^{(h+1)} ={}\\
  {}=\fr{\rho^s}{s!(1+\gamma)^s} 
\prod\limits^l_{j=1} \fr{\rho}{s(1+\gamma)+j\theta_j} \left( \pi_0^{(h)} -
\pi_0^{(h+1)}\right) ={}\\
  {}= \left( \fr{\rho^s}{s!(1+\gamma)^s}\right)^2 \prod\limits^l_{j=1} 
\fr{\rho}{s(1+\gamma)+j\theta_j}\times{}\\
{}\times  \prod\limits^{h+1}_{j=1} 
\fr{\rho}{s(1+\gamma)+j\theta_j}\,\pi_0^{(h)} \pi_0^{(h+1)}= \pi^{(h)}_{s+l} \pi_{s+h+1}^{(h+1)}\,.
  \end{multline*}
  
  Точно так же, использовав~(\ref{e3-ag}) при $l\hm= \overline{0, s}$, получаем 
равенство 
$$
\pi_l^{(h)} - \pi_l^{(h+1)} = \pi_l^{(h)} \pi^{(h+1)}_{s+h+1}.
$$ 
Следовательно, равенства~(\ref{e4-ag}) справедливы. Аналогично, 
использовав~(\ref{e3-ag}) и~(\ref{e4-ag}), находим
  \begin{multline*}
  \pi^{(h+1)}_{s+h+1} =\fr{\rho^s}{s!(1+\gamma)^2} \prod\limits_{j=1}^{h+1} 
\fr{\rho}{s(1+\gamma)+j\theta_j}\,\pi_0^{(h+1)}={}\\
  {}= 
  \fr{\rho^s}{s!(1+\gamma)^s}\,
  \fr{\rho}{s(1+\gamma)+(h+1)\theta_{h+1}} \times{}\\
  {}\times \prod\limits^h_{j=1} \fr{\rho} 
{s(1+\gamma)+j\theta_j} \left( 1-\pi^{(h+1)}_{s+h+1}\right) \pi_0^{(h)},
  \end{multline*}
откуда следует~(\ref{e5-ag}).

  Покажем, что имеет место равенство
  \begin{equation}
  Q(h)-Q(h+1) =\pi_{s+h+1}^{(h+1)}\left[ Q(h)-G(h)\right],
  \label{e6-ag}
  \end{equation}
  
\vspace*{-12pt}
  
  \columnbreak 


\noindent
где



\noindent
\begin{multline}
G(h)=\left( C_0+C_1\right) (\mu+\beta)s +\left( C_0+C_1\right) \alpha_{h+1}(h+1) -
{}\\
{}- \sum\limits_{j=1}^{h+1} \alpha_j C_2 - C_1\lambda -C_3\beta s\,.
\label{e7-ag}
\end{multline}

\vspace*{-6pt}
  
  Использовав~(\ref{e1-ag})--(\ref{e7-ag}), получим:
  
  \vspace*{-6pt}
  
  \noindent
  \begin{multline*}
  Q(h)-Q(h+1) =\lambda C_0\left( 1-\pi^{(h)}_{s+h}\right) -{}\\
  {}- \lambda C_1 
\pi^{(h)}_{s+h} -d\left( \overline{L}^{(h)}\right) -\beta C_3 \overline{S}^{(h)}-{}\\
  {}-
  \lambda C_0\left( 1- \pi^{(h+1)}_{s+h+1}\right) +\lambda C_1 
\pi_{s+h+1}^{(h+1)} +d\left( \overline{L}^{(h+1)}\right) +{}\\
{}+\beta C_3  \overline{S}^{(h+1)}=
  -\lambda C_0\pi^{(h)}_{s+h} -\lambda C_1 \pi^{(h)}_{s+h} -d\left( 
\overline{L}^{(h)}\right) -{}\\
{}- \beta C_3 \overline{S}^{(h)} +\lambda  C_0\pi_{s+h+1}^{(h+1)}+\lambda C_1 \pi^{(h+1)}_{s+h+1}+{}\\
{}+\left( 1-\pi_{s+h+1}^{(h+1)}\right) d\left( \overline{L}^{(h)}\right)+
 C_2\sum\limits_{j=1}^{h+1} \alpha_j \pi_{s+h+1}^{(h+1)} 
+{}\\
{}+\beta s C_3 \pi_{s+h+1}^{(h+1)} +
\beta C_3\left( 1-\pi_{s+h+1}^{(h+1)}\right) \overline{S}^{(h)} ={}\\
{}= -\lambda C_0 
\pi^{(h)}_{s+h} -\lambda C_1\pi^{(h)}_{s+h} +\lambda C_0 \pi^{(h+1)}_{s+h+1} 
+{}\\
  {}+ \lambda C_1\pi_{s+h+1}^{(h+1)} +C_2\sum\limits_{j=1}^{h+1} \alpha_j 
\pi^{(h+1)}_{s+h+1}+\beta s C_3 \pi_{s+h+1}^{(h+1)} -{}\\
{}- \pi_{s+h+1}^{(h+1)} d\left( \overline{L}^{(h)}\right)-
 \beta C_3\pi^{(h+1)}_{s+h+1} \overline{S}^{(h)} ={}\\
 {}=\pi_{s+h+1}^{(h+1)} \Bigg[ 
 %\vphantom{\fr{C_0+C_1}{\pi^{(h+1)}_{s+h+1}}}
\lambda C_0+\lambda C_1 +C_2\sum\limits_{j=1}^{h+1} \alpha_j -d\left( 
\overline{L}^{(h)}\right) - {}\\
  {}- C_3\beta \overline{S}^{(h)} +\beta s C_3 -\lambda 
\fr{C_0+C_1}{\pi^{(h+1)}_{s+h+1}}\,\pi^{(h)}_{s+h}
 %\vphantom{\fr{C_0+C_1}{\pi^{(h+1)}_{s+h+1}}}
 \Bigg]={}\\
  {}=  \pi^{(h+1)}_{s+h+1} \left[
 \vphantom{\fr{C_0+C_1}{\pi^{(h+1)}_{s+h+1}}}
 \lambda C_0 \left( 1-\pi^{(h)}_{s+h}\right) -
\lambda C_1\pi^{(h)}_{s+h} +\lambda C_1 +{}\right.\\
{}+\sum\limits^{h+1}_{j=1} \alpha_j C_2 -
d\left( \overline{L}^{(h)}\right)- C_3\beta\overline{S}^{(h)} +\lambda C_1\pi^{(h)}_{s+h} +{}\\
\left.{}+\lambda 
C_0\pi^{(h)}_{s+h} +\beta s C_3 -
\lambda\fr{C_0+C_1}{\pi^{(h+1)}_{s+h+1}}\,\pi^{(h)}_{s+h}\right]={}\\
  {}= \pi^{(h+1)}_{s+h+1} \left[ 
  Q(h) +\lambda C_1+ \sum\limits_{j=1}^{h+1} 
\alpha_j C_2 +\beta s C_3- {}\right.\\[-2pt]
\left.  {}-  (C_0+C_1)\lambda \fr{s(1+\gamma)+(h+1)\theta}{\rho} 
\vphantom{\sum\limits_{j=1}^{h+1}}
\right]={}\\[-2pt]
{}=  \pi^{(h+1)}_{s+h+1} \Bigg[ Q(h)+\lambda C_1+ C_2 \sum\limits_{j=1}^{h+1} 
\alpha_j +\beta s C_3 -{}\\[-2pt]
{}-(C_0+C_1) [ s(\mu+\beta)+\alpha_{h+1} (h+1)]\Bigg].
  \end{multline*} 
  
\begin{figure*}[b] %fig1
\vspace*{1pt}
\begin{minipage}[t]{80mm}
      \begin{center}
     \mbox{%
\epsfxsize=79mm 
\epsfbox{aga-1.eps}
}
\end{center}
\vspace*{-9pt}
\Caption{Зависимости функций $Q$~(\textit{1}) и~$G$~(\textit{2}) от порогового значения~$h$: $h^*$~--- 
оптимальное пороговое значение длины очереди}
\end{minipage}
%\end{figure*}
\hfill
%\begin{figure*} %fig2
\vspace*{1pt}
\begin{minipage}[t]{80mm}
      \begin{center}
     \mbox{%
\epsfxsize=79mm 
\epsfbox{aga-2.eps}
}
\end{center}
\vspace*{-9pt}
\Caption{Зависимости функций $Q$~(\textit{1}) и~$G$~(\textit{2}) от порогового значения~$h$}
\end{minipage}
\end{figure*}
  
 \vspace*{-6pt}
  
Значит, равенство~(\ref{e6-ag}) имеет место. 

\pagebreak

Так как верно равенство 
$$
\lambda \left(1 - \pi^{(h)}_{s+h}\right) \hm= \sum\limits^h_{i=1} \sum\limits^i_{j=1} 
\alpha_j \pi_i^{(h)} +(\beta+\mu) \overline{S}^{(h)}\,,
$$
то равенства для $Q(h)$ и~$G(h)$ в~(\ref{e1-ag}) и~(\ref{e7-ag}) эквивалентны 
равенствам
\begin{align*}
Q(h) &= \left[ \left( C_0+C_1\right)(\beta+\mu) -\beta C_3\right]\overline{S}^{(h)} 
+{}\\
&{}+\left( C_0 +C_1-C_2\right) \sum\limits^h_{i=1} \sum\limits^i_{j=1} \alpha_j 
\pi_i^{(h)} -C_1\lambda\,;\\
G(h) &= \left[ \left( C_0+C_1\right)(\beta+\mu) -\beta C_3\right]s +{}\\
&{}+\left( 
C_0+C_1\right) \alpha_{h+1} (h+1) -C_2\sum\limits_{j=1}^{h+1} \alpha_j -C_1\lambda\,.
\end{align*}
При $h=0$ последние равенства примут вид:
\begin{align*}
Q(0) &= \left[ \left( C_0+C_1\right((\beta+\mu)-\beta C_3\right] \overline{S}^{(0)} -
C_1\lambda\,;\\
G(0)&= \left[ \left( C_0+C_1\right) (\beta+\mu) -\beta C_3\right] s+{}\\
&\hspace*{18mm}{}+\left( C_0+C_1- C_2\right) \alpha_1 -C_1\lambda\,.
\end{align*}
  
  
  Далее всюду будем предполагать, что при условии $C_0\hm+ C_1\hm- C_2 
\hm<0$ выполняется и~условие $(C_2/(C_0\hm+C_1)-1) 
\alpha_{i+1}/(\alpha_{i+1}\hm-\alpha_i)\hm\geq i$ для всех $i\hm\geq 1$. Обратим 
внимание, что функция $G(h)$ возрастает по~$h$ при $C_0\hm+C_1\hm- C_2\hm 
>0$ и~не возрастает, когда $C_0\hm+C_1\hm- C_2\hm\leq 0$ и~$\alpha_i$ такие, 
что условие $(C_2/(C_0\hm+C_1)-1) \alpha_{i+1}/(\alpha_{i+1}\hm-\alpha_i)\hm\geq 
i$ для всех $i\hm\geq 1$.
  
  Воспользуемся теоремой~1 из работы~\cite{14-ag}. Нетрудно заметить (см.\ 
равенство~(\ref{e6-ag})), что функция $Q(h)$ при $C_0\hm+C_1\hm\leq C_2$ и~функция $-Q(h)$ при $C_0\hm+C_1\hm> C_2$ удовлетворяют условиям 
теоремы~1 из~\cite{14-ag}. Тогда из указанной теоремы непосредственно следует 
справедливость следующего утверж\-де\-ния. 
  
  \smallskip
  
  \noindent
  \textbf{Утверждение.} \textit{При выполнении предположения, введенного 
выше относительно параметров $C_0$, $C_1$, $C_2$ и~$\alpha_i$, $i\hm\geq 1$, 
решение задачи~$(2)$ обладает следующими свойствами}: 
  \begin{enumerate}[(1)]
\item \textit{если $C_0+C_1\hm\leq C_2$, то $Q(h)$~--- унимодальная функция 
(так как $G(h)$ не возрастает по~$h$ и~$G(0)\hm\leq Q(0)$, и~если $[(C_0\hm+ 
C_1)(\beta\hm+ \mu) \hm- \beta C_3] (s\hm- \overline{S}^{(0)} )\hm\leq 
(C_0\hm+C_1\hm-C_2)\alpha_1$, то $h^*\hm=0$, иначе существует $0\hm< h^*\hm< \infty$};
\item \textit{если $C_0+C_1\hm >C_2$ и~$[(C_0\hm+ C_1)(\beta\hm+ \mu) \hm- \beta 
C_3] (s\hm- \overline{S}^{(0)} ) \hm+ (C_0\hm+ C_1\hm- C_2)\alpha_1\hm>0$, то 
$h^*\hm=\infty$ и~при этом $Q(h)$ монотонно возрастает по~$h$ $($так как 
$G(h)$ возрастает по~$h$ и~$G(0)\hm >Q(0))$};
\item \textit{если $C_0+C_1\hm> C_2$ и~$[(C_0\hm+ C_1)(\beta\hm+ \mu) \hm- \beta 
C_3] (s\hm- \overline{S}^{(0)} ) \hm+ (C_0\hm+ C_1\hm- C_2)\alpha_1\hm\leq0$, то 
функция $-Q(h)$ унимодальная (так как удовлетворяет условиям теоремы~$1$ 
из}~\cite{14-ag}) \textit{и~при этом}
$$
h^*=\begin{cases}
0\,, & \mbox{\textit{если}\ \ } Q(\infty) \leq Q(0);\\
\infty\,, & \mbox{\textit{если}\ \ } Q(\infty) > Q(0)
\end{cases}
$$
\textit{$($так как $G(h)$ возрастает по~$h$ 
и}~$G(0)\hm\leq Q(0)$$)$. 
\end{enumerate}

\smallskip

 На рис.~1 и~2 проиллюстрировано поведение функций $Q(h)$ и~$G(h)$ для двух 
наборов значений па\-ра\-мет\-ров рассматриваемой СМО: 
\begin{enumerate}[(1)]
\item рис.~1: $\lambda=8$; $\mu=2$; $\alpha_i=0{,}5$; $i\hm= \overline{1, h}$; 
$\beta\hm= 0{,}25$; $C_0\hm= 20$; $C_1\hm= 5$; $C_2\hm= 40$; $C_3\hm=10$; 
\item рис.~2:
$\lambda=8$; $\mu=1$; $\alpha_i=0{,}25$; $i\hm= \overline{1, h}$; $\beta\hm= 0{,}125$; 
$C_0\hm= 20$; $C_1\hm= 3$; $C_2\hm= 10$; $C_3\hm=15$. 
\end{enumerate}

Заметим, что в~случае, изображенном на рис.~1, целевая функция достигает 
максимума при пороговом значении $h^*\hm=18$, что согласуется 
с~утверж\-де\-ни\-ем пункта~1, а~в~случае рис.~2 оптимальное значение порога 
$h^*\hm=\infty$, что соответствует пункту~2. 



\section{Заключение}

  Практическим результатом проведенных выше исследований стал 
сле\-ду\-ющий прос\-той алгоритм оптимизации однопорогового управ\-ле\-ния 
оче\-редью для рассмотренной выше модели СМО при выполнении условий 
утверж\-де\-ния относительно па\-ра\-мет\-ров~$C_0$, $C_1$, $C_2$ и~$\alpha_i$, 
$i\hm\geq 1$.
  \begin{enumerate}[1.]
\item Если выполняется условие $C_0\hm+ C_1\hm\leq C_2$, то
\begin{enumerate}[(1)]
\item  положить $h\hm=0$;
\item  до тех пор пока выполняется неравенство $Q(h+1)\hm> Q(h)$, полагать 
$h\hm= h+1$;
\item  положить $h^*=h$.
\end{enumerate}
\item Если выполняются неравенства $C_0\hm+C_1\hm >C_2$ и~$[(C_0\hm+ 
C_1)(\beta\hm+ \mu) \hm- \beta C_3] (s\hm- \overline{S}^{(0)} ) \hm+ (C_0\hm+ 
C_1\hm- C_2)\alpha_1\hm>0$, то положить $h^*\hm= \infty$. 
\item Если $C_0\hm+C_1\hm> C_2$ и~$[(C_0\hm+ C_1)(\beta\hm+ \mu) \hm- \beta 
C_3] (s\hm- \overline{S}^{(0)} ) \hm+ (C_0\hm+ C_1\hm- C_2)\alpha_1\hm\leq 0$, то 
положить 
$$
h^*= \begin{cases}
0\,, &\mbox{если\ } Q(\infty)\leq Q(0);\\
\infty &\mbox{иначе.}
\end{cases}
$$
\end{enumerate}
  
  Обратим внимание, что при выполнении условий третьего пункта алгоритма 
справедливы неравенства $Q(0)\hm\leq 0$ и~$G(0)\hm\leq Q(0)$ и~на отрезке 
$[0,h^0]$, где~$h^0$~--- максимальное значение, такое что $Q(h^0)\hm\geq 
G(h^0)$, функция~$Q(h)$ не возрастает, а~при $h\hm\in [h^0,\infty)$ возрастает 
(так как в~случае пункта~3 функция $-Q(h)$ унимодальная). Следовательно, если 
$C_0\hm+C_1\hm\geq C_2$ и~выполняется условие $Q(0)\hm\leq Q(1)$, то 
$h^*\hm=\infty$.

{\small\frenchspacing
 {\baselineskip=11.5pt
 %\addcontentsline{toc}{section}{References}
 \begin{thebibliography}{99}
\bibitem{1-ag}
\Au{Floyd S., Jacobson~V.} Random early detection gateways for congestion avoidance~// 
IEEE ACM T. Network., 1993. Vol.~1. P. 397--413. doi: 10.1109/90.251892.

\bibitem{2-ag}
\Au{Коновалов М.\,Г.} Об одной задаче оптимального управ\-ле\-ния нагрузкой на сервер~// 
Информатика и~её применения, 2013. Т.~7. Вып.~4. С.~34--43. doi: 10.14357/19922264130404. EDN: 
RRROXB.


\bibitem{4-ag} %3
\Au{Konovalov M.\,G., Razumchik~R.\,V.} Comparison of two active queue management schemes 
through the $M/D/1/N$ queue~// Информатика и~её применения, 2018. Т.~12. Вып.~4. С.~9--15. doi: 
10.14357/19922264180402. EDN: VOGJOZ.

\bibitem{3-ag} %4
\Au{Агаларов Я.\,М.} Оптимизация объема буферной памяти узла коммутации при схеме 
полного разделения памяти~// Информатика и~её применения, 2018. Т.~12. Вып.~4. С.~25--32.
doi: 10.14357/19922264180404. EDN: YQHHGP.


\bibitem{5-ag}
\Au{Агаларов Я.\,М., Ушаков~В.\,Г.} Об унимодальности функции дохода системы массового 
обслуживания типа $G/M/s$ с~управ\-ля\-емой очередью~// Информатика и~её применения, 
2019. Т.~13. Вып.~1. С.~55--61. doi: 10.14357/19922264190108. EDN: HYAODW.
\bibitem{6-ag}
\Au{Коновалов М.\,Г., Разумчик~Р.\,В.} Комплексное управ\-ле\-ние в~одном классе систем 
с~параллельным обслуживанием~// Информатика и~её применения, 2019. Т.~13. Вып.~4.  
С.~54--59. doi: 10.14357/19922264190409. EDN: REESRH.
\bibitem{7-ag}
\Au{Агаларов Я.\,М.} Об оптимизации работы резервного прибора в~многолинейной системе 
массового обслуживания~// Информатика и~её применения, 2023. Т.~17. Вып.~1. С.~89--95.  doi: 
10.14357/19922264230112. EDN: FCYDUT.
\bibitem{8-ag}
\Au{Агаларов Я.\, М.} Оптимизация схемы распределения буферной памяти узла пакетной 
коммутации~// Информатика и~её применения, 2023. Т.~17. Вып.~3. С.~39--48. doi: 
10.14357/19922264230306. EDN: \mbox{ХQLXCKV}.

\bibitem{9-ag}
\Au{Кирпичников Ф.\,П., Флакс~Д.\,Б., Галямова~К.\,Н.} Средняя длина очереди в~сис\-те\-ме 
массового обслуживания с~ограниченным средним временем пребывания заявки в~сис\-те\-ме~// 
Вестник Технологического университета, 2017. Т.~20. №\,2. С.~81--84. EDN: 
XVFSTN.

\bibitem{10-ag}
\Au{Савинов Ю.\,Г., Табакова~Е.\,Д., Сафиуллов~И.\,Д.} Оптимизация в~СМО с~нетерпеливыми 
заявками~// Ученые записки УлГУ. Сер. Математика и~информационные технологии, 2019. 
№\,1. С.~92--98. EDN: OWOZYR.

\bibitem{11-ag}
\Au{Мейханаджян Л.\,А., Разумчик~Р.\,В.} Система массового обслуживания 
$\mathrm{Geo}/G/1/\infty$ с~инверсионным порядком обслуживания и~ресамплингом в~дискретном 
времени~// Информатика и~её применения, 2019. Т.~13. Вып.~4. С.~60--67. doi: 
10.14357/19922264190410. EDN: LNIHGC.

\bibitem{12-ag}
\Au{Милованова Т.\,А., Разумчик~Р.\,В.} Однолинейная система массового обслуживания 
с~инверсионным порядком обслуживания с~вероятностным приоритетом, групповым 
пуассоновским потоком и~фоновыми заявками~// Информатика и~её применения, 2020. Т.~14. 
Вып.~3. С.~26--34. doi: 10.14357/19922264200304. EDN: NOMSAM.
\bibitem{13-ag}
\Au{Берговин А.\,К., Ушаков~В.\,Г.} Исследование сис\-тем обслуживания со смешанными 
приоритетами~// Информатика и~её применения, 2023. Т.~17. Вып.~2. С.~57--61. doi: 
10.14357/19922264230208. EDN: \mbox{JULPWS}.



\bibitem{14-ag}
\Au{Агаларов Я.\,М.} Признак унимодальности целочисленной функции одной переменной~// 
Обозрение прикладной и~промышленной математики, 2019. Т.~26. Вып.~1. С.~65--66.


\end{thebibliography}

 }
 }

\end{multicols}

\vspace*{-6pt}

\hfill{\small\textit{Поступила в~редакцию 23.02.24}}

%\vspace*{10pt}

%\pagebreak

\newpage

\vspace*{-28pt}

%\hrule

%\vspace*{2pt}

%\hrule



\def\tit{ON SINGLE-THRESHOLD QUEUE MANAGEMENT\\ IN~A~QUEUING SYSTEM 
WITH~IMPATIENT CUSTOMERS}


\def\titkol{On single-threshold queue management in~a~queuing system 
with~impatient customers}


\def\aut{Ya.\,M.~Agalarov}

\def\autkol{Ya.\,M.~Agalarov}

\titel{\tit}{\aut}{\autkol}{\titkol}

\vspace*{-8pt}


\noindent
Federal Research Center ``Computer Science and Control'' of the Russian Academy of 
Sciences, 44-2~Vavilov Str., Moscow 119333, Russian Federation

\def\leftfootline{\small{\textbf{\thepage}
\hfill INFORMATIKA I EE PRIMENENIYA~--- INFORMATICS AND
APPLICATIONS\ \ \ 2024\ \ \ volume~18\ \ \ issue\ 2}
}%
 \def\rightfootline{\small{INFORMATIKA I EE PRIMENENIYA~---
INFORMATICS AND APPLICATIONS\ \ \ 2024\ \ \ volume~18\ \ \ issue\ 2
\hfill \textbf{\thepage}}}

\vspace*{4pt}




\Abste{The results of a theoretical study of a~managed queuing system of $M/M/k$ type 
with impatient customers and single-threshold queue management are presented. The task of optimizing single-threshold 
queue management is set, the essence of which is to calculate for the queue length a~certain threshold value that 
maximizes a given objective function. In the system under study, a~customer leaves the system unattended if 
the waiting time in the queue (or the service time on the device) exceeds a~certain time interval of random 
length distributed according to an exponential law with a~given parameter. A~cost function is used as an 
indicator of the effectiveness of queue management (objective function) which takes into account the losses per 
unit of time due to system technical maintenance, rejection of customers at the entrance of the system, and 
leaving of customers until the end of the service. A~method for solving the problem of maximizing the cost 
objective function on a~set of single-threshold queue controls and an algorithm for guaranteed calculation of the 
optimal threshold are proposed.}

\KWE{queuing system; impatient customers; queue management}

\DOI{10.14357/19922264240206}{JZHAKU}

%\vspace*{-12pt}

%\Ack

%\vspace*{-3pt}


 %    \noindent
  


  \begin{multicols}{2}

\renewcommand{\bibname}{\protect\rmfamily References}
%\renewcommand{\bibname}{\large\protect\rm References}

{\small\frenchspacing
 {%\baselineskip=10.8pt
 \addcontentsline{toc}{section}{References}
 \begin{thebibliography}{99} 
\bibitem{1-ag-1}
\Aue{Floyd, S., and V.~Jacobson.} 1993. Random early detection gateways for congestion avoidance. 
\textit{IEEE ACM T. Network.} 1:397--413. doi: 10.1109/90.251892.
\bibitem{2-ag-1}
\Aue{Konovalov, M.\,G.} 2013. Ob odnoy zadache optimal'nogo upravleniya nagruzkoy na server [About one 
task of overload control]. \textit{Informatika i~ee Primeneniya~--- Inform. Appl.} 7(4):34--43. doi: 
10.14357/19922264130404. EDN: RRROXB.

\bibitem{4-ag-1}
\Aue{Konovalov, M., and R.~Razumchik}. 2018. Comparison of two active queue management schemes 
through the $M/D/1/N$ queue. \textit{Informatika i~ee Primeneniya~--- Inform. Appl.} 12(4):9--15. doi: 
10.14357/19922264180402. EDN: VOGJOZ.

\bibitem{3-ag-1}
\Aue{Agalarov, Ya.\,M.}  2018. Optimizatsiya ob''ema bufernoy pamyati uzla kommutatsii pri skheme polnogo 
razdeleniya pamyati [Optimization of buffer memory size of switching node in mode of full memory sharing]. 
\textit{Informatika i~ee Primeneniya~--- Inform. Appl.} 12(4):25--32. doi: 10.14357/19922264180404. EDN: 
YQHHGP.

\bibitem{5-ag-1}
\Aue{Agalarov, Ya.\,M., and V.\,G.~Ushakov.} 2019. Ob unimodal'nosti funktsii dokhoda sistemy massovogo 
obsluzhivaniya tipa $G/M/s$ s~upravlyaemoy ochered'yu [On the unimodality of the income function of a~type 
$G/M/s$ queueing system with controlled queue]. \textit{Informatika i~ee Primeneniya~--- Inform. Appl.} 
13(1):55--61. doi: 10.14357/19922264190108. EDN: HYAODW.
\bibitem{6-ag-1}
\Aue{Konovalov, M.\,G., and R.\,V.~Razumchik.} 2019. Komp\-leks\-noe upravlenie v~odnom klasse sistem 
s~parallel'nym obsluzhivaniem [Mixed policies for online job allocation in one class of systems with parallel 
service]. \textit{Informatika i~ee Primeneniya~--- Inform. Appl.} 13(4):54--59. doi: 10.14357/19922264190409. 
EDN: REESRH.
\bibitem{7-ag-1}
\Aue{Agalarov, Ya.\,M.} 2023. Ob optimizatsii raboty rezervnogo pribora v~mnogolineynoy sisteme 
massovogo obsluzhivaniya [Optimization of a~queue-length dependent additional server in the multiserver 
queue]. \textit{Informatika i~ee Primeneniya~--- Inform. Appl.} 17(1):89--95. doi: 10.14357/19922264230112. 
EDN: FCYDUT.
\bibitem{8-ag-1}
\Aue{Agalarov, Ya.\,M.} 2023. Optimizatsiya skhemy raspredeleniya bufernoy pamyati uzla paketnoy 
kommutatsii [Optimization of the buffer memory allocation scheme of the packet switching node]. 
\textit{Informatika i~ee Primeneniya~--- Inform. Appl.}  17(3):39--48. doi: 10.14357/19922264230306. EDN: 
QLXCKV.
\bibitem{9-ag-1}
\Aue{Kirpichnikov, F.\,P., D.\,B.~Flaks, and K.\,N.~Galyamova.} 2017. Srednyaya dlina ocheredi v~sisteme 
massovogo obsluzhivaniya s~ogranichennym srednim vremenem prebyvaniya zayavki v~sisteme [The average 
queue length in a~queuing system with a~limited average time for the request to stay in the system]. 
\textit{Vestnik Tekhnologicheskogo universiteta} [Bulletin of  Technological University] 
20(2):81--84. EDN: XVFSTN.
\bibitem{10-ag-1}
\Aue{Savinov, Yu.\,G., E.\,D.~Tabakova, and I.\,D.~Safiullov.} 2019. Optimizatsiya v~SMO s~neterpelivymi 
zayavkami [Optimization in the queuing system with impatient customers]. \textit{Uchenyye zapiski UlGU. Ser. 
Matematika i~informatsionnye tekhnologii} [Scientific Notes of UlSU. Ser. Mathematics and Information Technology] 1:92--98. EDN: \mbox{OWOZYR}.

\bibitem{11-ag-1}
\Aue{Meykhanadzhyan, L.\,A., and R.\,V.~Razumchik.} 2019. Sistema massovogo obsluzhivaniya 
$\mathrm{Geo}/G/1/\infty$ s~inversionnym poryadkom obsluzhivaniya i~resamplingom v~diskretnom vremeni 
[Discrete-time $\mathrm{Geo}/G/1/\infty$ LIFO queue with resampling policy]. \textit{Informatika i~ee Primeneniya~--- Inform. 
Appl.} 13(4):60--67. doi: 10.14357/ 19922264190410. EDN: LNIHGC.
\bibitem{12-ag-1}
\Aue{Milovanova, T.\,A., and R.\,V.~Razumchik.} 2020. Od\-no\-li\-ney\-naya sistema massovogo obsluzhivaniya 
s~in\-ver\-si\-on\-nym poryadkom obsluzhivaniya s~veroyatnostnym prioritetom, gruppovym puassonovskim 
\mbox{potokom} i~fonovymi zayavkami [A~single-server queueing system with \mbox{LIFO} service, probabilistic priority, 
batch Poisson arrivals, and background customers]. \textit{Informatika i~ee Primeneniya~--- Inform. Appl.} 
14(3):26--34. doi: 10.14357/ 19922264200304. EDN: NOMSAM.
\bibitem{13-ag-1}
\Aue{Bergovin, A.\,K., and V.\,G.~Ushakov.} 2023. Issledovanie sistem obsluzhivaniya so smeshannymi 
prioritetami [Analysis of the queueing systems with mixed priorities]. \textit{Informatika i~ee Primeneniya~--- 
Inform. Appl.} 17(2):57--61. doi: 10.14357/19922264230208. EDN: \mbox{JULPWS}.
{\looseness=1

}
\bibitem{14-ag-1}
\Aue{Agalarov, Ya.\,M.} 2019. Priznak unimodal'nosti tselochislennoy funktsii odnoy peremennoy [A~sign of 
unimodality of an integer function of one variable]. \textit{Obozrenie prikladnoy i~promyshlennoy matematiki} 
[Surveys Applied and Industrial Mathematics] 26(1):65--66.

\end{thebibliography}

 }
 }

\end{multicols}

\vspace*{-6pt}

\hfill{\small\textit{Received February 23, 2024}} 

\vspace*{-18pt}


\Contrl

\vspace*{-3pt}

\noindent
\textbf{Agalarov Yaver M.} (b.\ 1952)~--- Candidate of Science (PhD) in technology, associate professor, 
leading scientist, Federal Research Center ``Computer Science and Control'' of the Russian Academy of 
Sciences, 44-2~Vavilov Str., Moscow 119333, Russian Federation; \mbox{agglar@yandex.ru}




\label{end\stat}

\renewcommand{\bibname}{\protect\rm Литература}      %8
\newcommand{\al}{a_\lambda}
\newcommand{\bll}{b_\lambda}
\newcommand{\am}{a_\mu}
\newcommand{\bmm}{b_\mu}
\newcommand{\fl}{f_\lambda}
\newcommand{\fm}{f_\mu}
%\newcommand{\Ik}{\mathbb{1}}

\def\stat{kudr}

\def\tit{БАЙЕСОВСКИЕ МОДЕЛИ МАССОВОГО  ОБСЛУЖИВАНИЯ И~НАДЕЖНОСТИ:
АПРИОРНЫЕ РАСПРЕДЕЛЕНИЯ С~КОМПАКТНЫМ НОСИТЕЛЕМ$^*$}

\def\titkol{Байесовские модели массового  обслуживания и~надежности:
априорные распределения с~компактным носителем}

\def\aut{А.\,А.~Кудрявцев$^1$}

\def\autkol{А.\,А.~Кудрявцев}

\titel{\tit}{\aut}{\autkol}{\titkol}

{\renewcommand{\thefootnote}{\fnsymbol{footnote}} \footnotetext[1]
{Исследование выполнено при поддержке Российского научного фонда (проект 14-11-00397).}}


\renewcommand{\thefootnote}{\arabic{footnote}}
\footnotetext[1]{Московский государственный университет им.~М.\,В.~Ломоносова, 
факультет вычислительной математики и~кибернетики; 
Институт проб\-лем информатики Федерального исследовательского центра 
<<Информатика и~управ\-ле\-ние>> Российской академии наук, nubigena@mail.ru}

\vspace*{-12pt}

\Abst{Данная работа является очередной в~серии статей, посвященных изучению 
байесовских моделей массового обслуживания и~надежности. В~работе приводятся 
соотношения для функции распределения и~плотности частного~$\rho$ независимых 
случайных величин, имеющих априорные распределения с~компактным носителем, 
которые интерпретируются как параметр, <<препятствующий>> функционированию 
системы, и~параметр, <<способствующий>> функционированию. Описание жизненного цикла 
многих реальных систем осуществляется в~терминах~$\rho$, например в~теории 
массового обслуживания~$\rho$ называется параметром загрузки системы и~входит 
во многие формулы, описывающие разнообразные характеристики. Рассматриваются 
частные случаи априорных распределений с~компактным носителем, для которых плотности 
имеют полиномиальный или ку\-соч\-но-по\-ли\-но\-ми\-аль\-ный вид.}


\KW{байесовский подход; системы массового обслуживания; надежность; смешанные
распределения; распределения с~компактным носителем}

\DOI{10.14357/19922264160106} %

\vspace*{-4pt}

\vskip 12pt plus 9pt minus 6pt

\thispagestyle{headings}

\begin{multicols}{2}

\label{st\stat}

\section{Введение}

\vspace*{-2pt}

Во многих областях исследования математических моделей функционирования 
реальных систем аналитические результаты, характеризующие жизненный цикл 
рассматриваемых объектов, так или иначе зависят от параметров, <<способ\-ст\-ву\-ющих>> 
функционированию системы и~<<препятствующих>> функционированию. Так, в~моделях 
массового об-\linebreak служивания к~параметрам, <<способствующим>> функционированию, можно 
отнести интенсивность обслуживания запросов, а~к~па\-ра\-мет\-рам, <<препятствующим>> 
функционированию,~--- интенсивность входящего потока требований. 

Аналогично 
в~теории надежности па\-ра\-метр <<эффективности>> средства, исправляющего ошибки 
в~системе, <<способствует>> функционированию, а~па\-ра\-метр <<дефективности>>~--- 
<<препятствует>>. Очевидно, что итоговые результаты работы системы зависят не 
столько от значений самих па\-ра\-мет\-ров, влияющих на функционирование, сколько от 
их отношения. В~общем случае такое отношение можно назвать 
{\it <<коэффициентом баланса системы>>}.

Хорошо известно, что одним из основных показателей при изучении моделей 
массового обслуживания $M|M|1$ является коэффициент загрузки системы~$\rho$, 
равный отношению параметра входящего потока~$\lambda$ к~параметру обслуживания~$\mu$. 
От значения~$\rho$ зависит наличие стационарного режима у~рассматриваемой системы; 
величина~$\rho$ входит во многие формулы, описывающие характеристики разнообразных 
систем массового обслуживания. По своей сути коэффициент загрузки является коэффициентом 
баланса, характеризующим систему: функционирование системы тем эффективнее, чем ближе 
к~нулю значения~$\rho$. При достаточно больших значениях~$\rho$ ($\rho\hm\ge1$) 
система работает столь неэффективно, что, например, среднее число заявок 
в~системе $M|M|1|\infty$ считается равным бесконеч\-ности.

В рекуррентных моделях роста надежности удобно рассматривать коэффициент 
баланса $\rho\hm=\lambda/\mu$, где~$\lambda$~--- параметра <<эффективности>> 
средства, исправляющего ошибки в~системе, а $\mu$~--- параметр 
<<дефективности>>. При этом, в~отличие от теории массового обслуживания, 
система тем надежнее, чем больше значение~$\rho$.

Байесовский подход к~задачам массового обслуживания и~надежности предполагает 
рандомизацию параметров~$\lambda$ и~$\mu$. 

Подробное описание предпосылок для 
исследования, особенностей и~биб\-ли\-о\-графии байесовских моделей в~теории массового 
обслуживания и~надежности можно найти в~книге~\cite{KuSh2015}. 
В~основе всех результатов для байесовских моделей из~\cite{KuSh2015} лежит 
вероятностное распределение коэффициента баланса~$\rho$. Принципиальное отличие 
байесовских постановок задач в~указанных теориях заключается в~том, что 
в~задачах теории надежности коэффициенты <<эффективности>> и~<<дефективности>> 
соответственно имеют смысл рандомизированных вероятностей исправления ошибки 
в~системе и~внесения новой ошибки, а следовательно, соответствующие вероятностные 
распределения должны быть подмножествами единичного отрезка. 
В~теории массового обслуживания для систем $M|M|1$ параметры входящего потока 
и~обслуживания должны быть положительными числами, поэтому соответствующие 
распределения имеют лишь нижнее ограничение в~нуле.

Далее приводятся результаты для распределения величины $\rho\hm=\lambda/\mu$ 
в~случае, когда носителями распределений~$\lambda$ и~$\mu$ являются отрезки 
на положительной полупрямой. При применении изложенных ниже результатов 
к~надежностным постановкам необходимо ограничивать правые концы носителей 
распределений единицей.

\vspace*{-9pt}

\section{Основные результаты}

Пусть $\lambda$ и~$\mu$~--- независимые абсолютно непрерывные случайные величины, 
причем  ${\sf P}(\lambda\hm\in[\al,\bll])\hm=1$, $0\hm<\al\hm<\bll$, 
и~не существует множества $S\hm\subset[\al,\bll]$ положительной меры Лебега такого, что 
${\sf P}(\lambda\hm\in S)\hm=0$, а для случайной величины~$\mu$ выполнены аналогичные 
требования с~параметрами~$\am$ и~$\bmm$. Плотности случайных величин~$\lambda$ 
и~$\mu$ обозначим через $\fl(x)$ и~$\fm(x)$ соответственно. Во 
всех последующих выкладках будем предполагать, что $x\hm>0$.

Найдем функцию распределения $F_\rho(x)$ и~плотность $f_\rho(x)$ случайной 
величины~$\rho\hm=\lambda/\mu$. Имеем

\noindent
\begin{multline*}
F_\rho(x)=\il{-\infty}{+\infty}{\sf P}(\lambda<xy)\, d{\sf P}(\mu<y)={}\\[-1pt]
{}=\il{\am}{\bmm} \left[\il{\al}{xy}\fl(u)\, du\cdot
\Ik\left(\al\le xy\le \bll\right)+{}\right.\\[-1pt]
\left.{}+\Ik\left(xy>\bll\right)
\vphantom{\il{\am}{\bmm}}
\right] \fm(y)\, dy\,.
\end{multline*}
Рассмотрим всевозможные комбинации расположения точек~$\am$, $\bmm$, 
$\al/x$ и~$\bll/x$ на прямой. При этом существенную роль играет взаимное 
расположение точек~$\al/\am$ и~$\bll/\bmm$. Так, при $\am\hm<\al/x\hm<\bmm\hm<\bll/x$ 
и~$\al/\am\hm<\bll/\bmm$

\noindent
$$
F_\rho(x)=\il{\al/x}{\bmm}\il{\al}{xy}\fl(u)\fm(y)\, dudy\,.
$$
Аналогично рассмотрев остальные случаи, убеждаемся в~справедливости следующего 
утверждения.

\smallskip

\noindent
\textbf{Теорема 1.} \textit{Пусть независимые абсолютно непрерывные случайные 
величины~$\lambda$ и~$\mu$ имеют соответственно носители 
распределений $[\al,\bll]$ и~$[\am,\bmm]$, $0\hm<\al\hm<\bll$, $0\hm<\am\hm<\bmm$, 
и~плот\-ности $\fl(x)$ и~$\fm(x)$. Тогда случайная величина 
$\rho\hm=\lambda/\mu$ имеет функцию распределения}:
\begin{multline*}
F_\rho(x)=\Ik\left(\fr{\al}{\bmm}<x\le 
\min\left\{\fr{\al}{\am},\fr{\bll}{\bmm}\right\}\right)\times{}\\
{}\times \il{\al/x}{\bmm}
\il{\al}{xy}\fl(u)\fm(y)\, dudy+\Ik\left(\fr{\al}{\am}<x\le 
\fr{\bll}{\bmm}\right)\times{}\\
{}\times \il{\am}{\bmm}\il{\al}{xy}\fl(u)\fm(y)\, dudy+
\Ik\left(\fr{\bll}{\bmm}<x\le \fr{\al}{\am}\right)\times{}\\
{}\times \left[\ \il{\al/x}{\bll/x}\il{\al}{xy}\fl(u)\fm(y)\, dudy +
\il{\bll/x}{\bmm}\fm(x)\, dy\right]+{}\\
{}+\Ik\left(\max\left\{\fr{\al}{\am},\fr{\bll}{\bmm}\right\}<x\le\fr{\bll}{\am}\right)\times{}\\
{}\times
\left[\ \il{\am}{\bll/x}\il{\al}{xy}\fl(u)\fm(y)\, dudy+\il{\bll/x}{\bmm}\fm(y)\, dy\right]+{}\\
{}+\Ik\left(x>\fr{\bll}{\am}\right).
\end{multline*}


%\smallskip

Для нахождения плотности случайной величины~$\rho$ достаточно воспользоваться 
соотношением
$$
f_\rho(x)=\il{-\infty}{+\infty}{y}\fl(xy)\fm(y)\, dy
$$
и рассуждениями, приведенными выше, для функции распределения~$F_\rho(x)$.

\smallskip

\noindent
\textbf{Теорема 2.}\ \textit{Пусть независимые абсолютно непрерывные случайные 
величины~$\lambda$ и~$\mu$ имеют соответственно носители распределений 
$[\al,\bll]$ и~$[\am,\bmm]$, $0\hm<\al\hm<\bll$, $0\hm<\am\hm<\bmm$, 
и~плот\-ности $\fl(x)$ и~$\fm(x)$. Тогда случайная величина 
$\rho\hm=\lambda/\mu$ имеет плотность распределения}

\noindent
\begin{multline*}
f_\rho(x)=\Ik\left(\fr{\al}{\bmm}<x\le\min\left\{\frac{\al}{\am},\fr{\bll}{\bmm}\right\}\right)\times{}\\
{}\times\il{\al/x}{\bmm}
{y}\fl(xy)\fm(y)\, dy+{}
\end{multline*}

\noindent
\begin{multline*}
{}+\Ik\left(\fr{\al}{\am}<x\le\fr{\bll}{\bmm}\right)\il{\am}{\bmm}{y}\fl(xy)\fm(y)\, dy+{}\\
{}+\Ik\left(\fr{\bll}{\bmm}<x\le\fr{\al}{\am}\right)\il{\al/x}{\bll/x}{y}\fl(xy)\fm(y)\, dy +{}\\
{}+\Ik\left(\max\left\{\fr{\al}{\am},\fr{\bll}{\bmm}\right\}<x\le
\fr{\bll}{\am}\right)\!\!\il{\am}{\bll/x}\!\!{y}\fl(xy)\fm(y)\, dy.\hspace*{-5.39891pt}
\end{multline*}


\noindent
\textbf{Замечание~1.}\ Применительно к~байесовским моделям массового 
обслуживания $M|M|1$ теоремы~1 и~2 позволяют не только находить распределение 
коэффициента загрузки~$\rho$, но и~распределения таких характеристик, 
как вероятность <<непотери>> вызова 
$\pi\hm=1/(1\hm+\rho)$ в~модели $M|M|1|0$ или среднее чис\-ло требований 
в~очереди $N\hm=\rho/(1\hm-\rho)$ в~модели $M|M|1|\infty$ и~др. Приведенные 
теоремы также упрощают поиск средней надежности системы, поскольку 
для рекуррентных моделей байесовской тео\-рии надежности априорные 
распределения всегда ограничены и~являются подмножествами единичного отрезка.


\smallskip

\noindent
\textbf{Замечание~2.}\ В~формулировке теоремы~1 вместо плотности случайной 
величины~$\lambda$ можно использовать ее функцию распределения, учитывая
$$
\il{\al}{xy}\fl(u)\,du=F_\lambda(xy)\,,\enskip y\in\left[\fr{\al}{x},\fr{\bll}{x}\right]\,.
$$




\noindent
\textbf{Замечание~3.}\ 
В~формулировках теорем~1 и~2 один из индикаторов 
$\Ik\left({\al}/{\am}<x\le{\bll}/{\bmm}\right)$ 
и~$\Ik\left({\bll}/{\bmm}<x\le{\al}/{\am}\right)$ всегда равен нулю 
в~зависимости от взаимного расположения точек~$\al/\am$ и~$\bll/\bmm$.

\smallskip


Отдельный интерес в~классе распределений с~компактным носителем 
(для некоторой случайной величины~$\xi$) представляют распределения, 
плотности которых могут быть представлены в~виде полинома:
\begin{equation}\label{Density_Polynomial}
f_\xi(x)=\sum\limits_{i=0}^{n_\xi}c_{\xi,i}\, x^i\cdot\Ik(x\in[a_\xi,b_\xi])\,.
\end{equation}
К таким распределениям, в~частности, относятся равномерное (при $n_\xi\hm=0$) 
и~параболическое (при $n_\xi\hm=2$) распределения.

Пусть случайные величины~$\lambda$ и~$\mu$ удовлетворяют условиям теоремы~2, 
а их плотности~--- соотношению~(\ref{Density_Polynomial}) 
с~соответствующими параметрами. Для некоторых~$a$ и~$b$, одновременно 
принадлежащих отрезкам $[\am,\bmm]$ и~$[\al/x,\bll/x]$, определим

\noindent
\begin{multline}
I(a,b,x)=\il{a}{b}{y}\fl(xy)\fm(y)\, dy={}\\
{}=
\il{a}{b}\sum\limits_{i=0}^{n_\lambda}\sum\limits_{j=0}^{n_\mu}
c_{\lambda,i}c_{\mu,j}\, x^{i}y^{i+j+1}\, dy={}\\
{}=\sum\limits_{i=0}^{n_\lambda}\sum\limits_{j=0}^{n_\mu}c_{\lambda,i}c_{\mu,j}
\fr{b^{i+j+2}-a^{i+j+2}}{i+j+2}\, x^i.
\label{Integral_General}
\end{multline}

Теорема~2 дает возможность сформулировать следующее утверждение для 
распределений с~плотностями, имеющими полиномиальный вид.

\smallskip

\noindent
\textbf{Следствие~1.}\ Пусть распределения случайных величин~$\lambda$ и~$\mu$ 
удовлетворяют условиям теоремы~2 и~соотношению~(\ref{Density_Polynomial}) 
с~соответствующими пара\-мет\-рами. Тогда случайная величина $\rho\hm=\lambda/\mu$ 
имеет плотность:
\begin{multline}
f_\rho(x)={}\\
{}=\Ik\left(\fr{\al}{\bmm}<x\le\min\left\{\fr{\al}{\am},\fr{\bll}{\bmm}\right\}
\right)I\left(\fr{\al}{x},\bmm,x\right)+{}\\
{}+\Ik\left(\fr{\al}{\am}<x\le\fr{\bll}{\bmm}\right)I\left(\am,\bmm,x\right)+{}\\
{}+
\Ik\left(\fr{\bll}{\bmm}<x\le\fr{\al}{\am}\right)I\left(
\fr{\al}{x},\fr{\bll}{x},x\right)+{}\\
\!{}+\Ik\left(\max\left\{\fr{\al}{\am},\fr{\bll}{\bmm}\right\}<x\le\fr{\bll}{\am}\right)
I\left(\am,\fr{\bll}{x},x\right),\!\!
\label{Density_rho_Polynomial_case}
\end{multline}
где величины $I(a,b,x)$ определены соотношением~(\ref{Integral_General}).

\smallskip


Следствие~1 дает возможность вычислить моменты
$$
{\sf E}\rho^k=\int x^k f_\rho(x)\,dx\,, \enskip k=1,2,\ldots\,,
$$
для случая, когда $f_\rho(x)$ имеет вид~(\ref{Density_rho_Polynomial_case}). 
Будем полагать для определенности, что $c_{\mu,j}\hm=0$ при $j\hm>n_\mu$. 
Введем обозначение:
\begin{multline}
J(d)=\sum\limits_{i=0}^{n_\lambda}\sum\limits_{j=0}^{n_\mu}
\fr{c_{\lambda,i}c_{\mu,j}d^{i+j+2}}{i+j+2}\il{d/\bmm}{d/\am}x^{k-j-2}\, dx={}\\
{}=
\sum\limits_{i=0}^{n_\lambda}\fr{c_{\lambda,i}c_{\mu,k-1}d^{k+i+1}}{k+i+1}\,
\ln\fr{\bmm}{\am}+{}\\
{}+\sum\limits_{i=0}^{n_\lambda}\sum\limits_{j=0}^{n_\mu}\Ik(j\neq k-1)\times{}\\
{}\times
\fr{c_{\lambda,i}c_{\mu,j}(\bmm^{k-j-1}-\am^{k-j-1})d^{k+i+1}}
{(i+j+2)(k-j-1)\am^{k-j-1}\bmm^{k-j-1}}.
\label{Integral_for_moment}
\end{multline}
Непосредственно из следствия~1 вытекает сле\-ду\-ющее утверждение.

\smallskip

\noindent
\textbf{Следствие~2.}\ Пусть распределения случайных величин~$\lambda$ и~$\mu$ 
удовлетворяют условиям теоремы~2 и~соотношению~(\ref{Density_Polynomial}) 
с~соответствующими пара\-мет\-рами. Тогда моменты $k$-го порядка ($k\hm=1,2,\ldots$) 
случайной величины $\rho\hm=\lambda/\mu$ имеют вид:
\begin{multline*}
{\sf E}\rho^k={}\\
{}=\sum\limits_{i=0}^{n_\lambda}\sum\limits_{j=0}^{n_\mu}
c_{\lambda,i}c_{\mu,j}\left(\bll^{k+i+1}-\al^{k+i+1}\right)
\left(\am^{k-j-1}-{}\right.\\
\left.{}-\bmm^{k-j-1}\right)\!\Bigg/ \!
\left((i+j+2)(k+i+1)\am^{k-j-1}\bmm^{k-j-1}\right)+{}\\
{}+J(\bll)-J(\al)\,,
\end{multline*}
где величины $J(d)$ определены соотношением~(\ref{Integral_for_moment}).


\smallskip

\noindent
\textbf{Замечание~4.}\ Следствия~1 и~2 из теоремы~2 обобщают некоторые 
полученные ранее результаты (см., например,~\cite{KuSh2015}).

\smallskip


\noindent
\textbf{Замечание~5.}\ На величину моментов ${\sf E}\rho^k$ не 
влияет взаимное расположение точек~$\al/\am$ и~$\bll/\bmm$.

\smallskip


Изложенный выше метод дает возможность находить характеристики 
распределения частного случайных величин, плотности которых имеют 
полиномиальный вид не на всем носителе, а на некоторых его подмножествах. 
Рассмотрим следующее разбиение отрезка $[a_\xi,b_\xi]$:
$$
0<a_\xi=a_\xi^{(1)}<b_\xi^{(1)}=a_\xi^{(2)}<\cdots<b_\xi^{(l_\xi)}=b_\xi\,.
$$
Пусть плотность некоторой случайной величины~$\xi$, носителем распределения 
которой является отрезок $[a_\xi,b_\xi]$, имеет ку\-соч\-но-по\-ли\-но\-ми\-аль\-ный вид:
\begin{equation}
f_\xi(x)=\sum\limits_{l=1}^{l_\xi}\sum\limits_{i=0}^{n_\xi}c_{\xi,i,l}\, 
x^i\cdot\Ik\left(x\in\left[a_\xi^{(l)},b_\xi^{(l)}\right]\right).
\label{Density_Piecewise_Polynomial}
\end{equation}
Частными случаями таких распределений являются распределение 
Симпсона (при $l_\xi\hm=2$ и~$n_\xi\hm=1$) и~трапецеидальное распределение 
(при $l_\xi\hm=3$ и~$n_\xi\hm=1$).

\smallskip

\noindent
\textbf{Замечание~6.} Для распределений случайных величин~$\lambda$ и~$\mu$, 
удовлетворяющих условиям теоремы~2 и~соотношению~(\ref{Density_Piecewise_Polynomial}) 
с~соответствующими параметрами, при каждом конкретном наборе параметров~$n_\lambda$, 
$n_\mu$, $l_\lambda$ и~$l_\mu$ несложно сформулировать утверждения, 
аналогичные следствиям из теоремы~2, в~которых величины $I(a,b,x)$ будут 
определяться соотношением:
\begin{multline*}
\!\!\!I(a,b,x)=\il{a}{b}\sum\limits_{l=1}^{l_\lambda}\sum\limits_{i=0}^{n_\lambda}
\sum\limits_{m=1}^{l_\mu}\sum\limits_{j=0}^{n_\mu}c_{\lambda,i,l}
c_{\mu,j,m}\, x^iy^{i+j+1}\times{}\ \\
{}\times\Ik\left(y\in\left[a_\lambda^{(l)}/x,b_\lambda^{(l)}/x\right]\right)
\Ik\left(y\in\left[a_\mu^{(m)},b_\mu^{(m)}
\right]\right)\, dy.
\end{multline*}

{\small\frenchspacing
 {%\baselineskip=10.8pt
 \addcontentsline{toc}{section}{References}
 \begin{thebibliography}{9}

\bibitem{KuSh2015}
\Au{Кудрявцев А.\,А., Шоргин С.\,Я.}
Байесовские модели в~тео\-рии массового обслуживания и~надежности.~--- 
М.: ФИЦ ИУ РАН, 2015. 76~с.

\end{thebibliography}

 }
 }

\end{multicols}

\vspace*{-3pt}

\hfill{\small\textit{Поступила в~редакцию 17.01.16}}

\vspace*{8pt}

%\newpage

%\vspace*{-24pt}

\hrule

\vspace*{2pt}

\hrule

\vspace*{8pt}



\def\tit{BAYESIAN QUEUEING AND~RELIABILITY MODELS:\\
\textit{A~PRIORI} DISTRIBUTIONS WITH~COMPACT SUPPORT}

\def\titkol{Bayesian queueing and reliability models: \textit{A~priori} distributions with compact support}

\def\aut{A.\,A.~Kudryavtsev$^{1,2}$}

\def\autkol{A.\,A.~Kudryavtsev}

\titel{\tit}{\aut}{\autkol}{\titkol}

\vspace*{-9pt}

\noindent
$^1$Department of Mathematical Statistics, Faculty of Computational Mathematics 
and Cybernetics,\linebreak
$\hphantom{^1}$M.\,V.~Lomonosov Moscow State University, 1-52~Leninskiye Gory, GSP-1, 
Moscow 119991, Russian\linebreak
$\hphantom{^1}$Federation

\noindent
$^2$Institute of Informatics Problems, Federal Research Center 
``Computer Science and Control'' of the Russian\linebreak
 $\hphantom{^1}$Academy of Sciences, 
44-2~Vavilov Str., Moscow 119333, Russian Federation

\def\leftfootline{\small{\textbf{\thepage}
\hfill INFORMATIKA I EE PRIMENENIYA~--- INFORMATICS AND
APPLICATIONS\ \ \ 2016\ \ \ volume~10\ \ \ issue\ 1}
}%
 \def\rightfootline{\small{INFORMATIKA I EE PRIMENENIYA~---
INFORMATICS AND APPLICATIONS\ \ \ 2016\ \ \ volume~10\ \ \ issue\ 1
\hfill \textbf{\thepage}}}

\vspace*{3pt}

\Abste{This work is the latest in a series of articles devoted to the 
study of Bayesian queueing and reliability models. The paper presents 
relations for the distribution function and the density of the quotient~$\rho$ of 
independent random variables with \textit{a priori} distributions with compact support, 
which are interpreted as a~parameter  ``obstructing'' the functioning of the 
system and a~parameter ``conducing'' to the functioning of the system. Description 
of the life cycle of many real systems is carried out in terms of~$\rho$; for example, 
in the queueing theory, parameter~$\rho$ is called\linebreak\vspace*{-12pt}}

\Abstend{the ``system load factor'' 
and is a~part of many formulas that describe various characteristics. The paper 
considers particular cases of \textit{a~priori} distributions with compact 
support for which densities have polynomial or piecewise polynomial form.}

\KWE{Bayesian approach; mass service theory; reliability theory; mixed distributions; 
distributions with compact support}

\DOI{10.14357/19922264160106} 

\Ack
\noindent
This work was financially supported by the Russian Science Foundation 
(grant No.\,14-11-00397).



%\vspace*{3pt}

  \begin{multicols}{2}

\renewcommand{\bibname}{\protect\rmfamily References}
%\renewcommand{\bibname}{\large\protect\rm References}

{\small\frenchspacing
 {%\baselineskip=10.8pt
 \addcontentsline{toc}{section}{References}
 \begin{thebibliography}{9}

\bibitem{1-kudr}
\Aue{Kudryavtsev, A.\,A., and S.\,Ya.~Shorgin}. 
2015. \textit{Bayesovskie modeli v~teorii massovogo obsluzhivaniya i~nadezhnosti} 
[Bayesian models in mass service and reliability theories]. 
Moscow: Federal Research Center ``Computer Science and Control'' of the Russian
Academy of Sciences, 2015.\linebreak 76~p.

\end{thebibliography}

 }
 }

\end{multicols}

\vspace*{-3pt}

\hfill{\small\textit{Received January 17, 2016}}

\Contrl

\noindent
\textbf{Kudryavtsev Alexey A.} (b.\ 1978)~---
Candidate of Sciences (PhD) in physics and mathematics, associate professor, 
Department of Mathematical Statistics, Faculty of Computational Mathematics 
and Cybernetics, M.\,V.~Lomonosov Moscow State University, 1-52~Leninskiye Gory, 
GSP-1, Moscow 119991, Russian Federation; Institute of Informatics Problems, 
Federal Research Center ``Computer Science and Control'' 
of the Russian Academy of Sciences, 44-2~Vavilov Str., Moscow 119333, 
Russian Federation; nubigena@mail.ru
\label{end\stat}


\renewcommand{\bibname}{\protect\rm Литература}         %9
%\DeclareMathOperator{\mathrm{tr}\,}{tr}
\def\stat{lipatyev}

\def\tit{НЕАСИМПТОТИЧЕСКИЙ АНАЛИЗ СТАТИСТИКИ БАРТЛЕТТА--НАНДА--ПИЛАЯ 
ДЛЯ~ДАННЫХ\\ БОЛЬШОЙ РАЗМЕРНОСТИ}

\def\titkol{Неасимптотический анализ статистики Бартлетта--Нанда--Пилая 
для~данных большой размерности}

\def\aut{А.\,А.~Липатьев$^1$}

\def\autkol{А.\,А.~Липатьев}

\titel{\tit}{\aut}{\autkol}{\titkol}

\index{Липатьев А.\,А.}
\index{Lipatiev A.\,A.}

%{\renewcommand{\thefootnote}{\fnsymbol{footnote}} \footnotetext[1]
%{Работа выполнена при частичной финансовой поддержке РФФИ
%(проекты 18-07-00692, 19-07-00739 и~20-07-00804).}}

\renewcommand{\thefootnote}{\arabic{footnote}}
\footnotetext[1]{Московский государственный университет имени М.\,В.~Ломоносова, 
факультет вычислительной математики и~кибернетики,
кафедра математической статистики, \mbox{allipatev@cs.msu.ru}}


\vspace*{-10pt}


\Abst{Представлены вычислимые оценки скорости сходимости нормированной статистики Барт\-лет\-та--Нан\-да--Пи\-лая 
к~стандартному нормальному распределению при условии, что размерность данных возрастает 
пропорционально объему выборки. Приведенный результат позволяет корректно вычислять 
p-зна\-че\-ния в~прикладных задачах многомерного анализа данных. Задачи в~постановке, когда 
число анализируемых признаков сравнимо с~объемом выборки, все чаще возникают в~об\-ласти 
обработки сигналов. Доказательство базируется существенным образом на нормальности распределения 
элементов рассматриваемых матриц с~распределением Уишарта. Для случайных величин, представляющих 
собой матричные следы произведения и~квадратов мат\-риц с~нормированным распределением Уишарта, 
находятся удобные оценки сверху для $1-F$, где $F$~--- функция распределения соответствующего 
следа мат\-ри\-цы. Применяя свойства обратных мат\-риц и~неотрицательно определенных мат\-риц, статистика 
Барт\-лет\-та--Нан\-да--Пи\-лая ограничивается сверху комбинацией из упомянутых выше следов матриц.}

\KW{точность приближений; многомерный дисперсионный анализ; вычислимые оценки; 
статистика Барт\-лет\-та--Нан\-да--Пи\-лая; данные большой размерности}

\DOI{10.14357/19922264210110}


\vspace*{-2pt}

\vskip 10pt plus 9pt minus 6pt

\thispagestyle{headings}

\begin{multicols}{2}

\label{st\stat}




\section{Введение}
\label{sec:intro}

В большом числе прикладных задач исследователи анализируют многомерные данные, 
в~которых количество~$p$ признаков сравнимо с~числом~$n$\linebreak наблюдений. Для анализа 
данных фиксированной размерности существует множество статистических процедур, 
уже ставших классическими.\linebreak
Однако час\-то нет возможности использовать традиционную статистическую процедуру, 
лишь устремив в~ней чис\-ло признаков к~бесконечности,
так как при этом изменяется предельное распределение статистики критерия  (см.~[1,  
разд.~6.3.4]).

Цель данной работы~--- нахождение вычислимых оценок точности аппроксимации 
статистики Барт\-лет\-та--Нан\-да--Пи\-лая (Bartlett--Nanda--Pillai test) нормальным 
распределением в~модели многомерного дисперсионного анализа (MANOVA~--- multivariate analysis of variance) 
для данных 
большой размерности,
когда отношение числа \mbox{признаков} к~чис\-лу  наблюдений~$p/n$ стремится к~некоторой 
константе из интервала $(0, 1)$.

Результаты, касающиеся распределений статистик, возникающих в~модели MANOVA при 
условии, что нулевая гипотеза верна, оказываются полезны в~области обработки 
сигналов. \mbox{Например}, в~\cite{lit:Johnstone_RoyStat} показано, каким образом 
результаты из MANOVA и~обработки сигналов могут сводиться к~спектру определенной 
матрицы~$E^{-1}H$. В~\mbox{статье}~\cite{lit:Akbari_AppliedLH} приводится пример 
применения статистики Лоу\-ли--Хо\-тел\-лин\-га, родственной статистике 
Барт\-лет\-та--Нан\-да--Пи\-лая, в~контексте обработки данных радаров с~синтезированной апертурой.

В разд.~\ref{sec:results} сформулирован основной результат работы~--- 
теорема~1.
Теорема~2 является   вспомогательной, но при этом 
представляет самостоятельный интерес.
В~разд.~\ref{sec:proofs} даны доказательства основных теорем, которые 
опираются на леммы из разд.~\ref{sec:lemmas}.

%%%%%%%%%%%%%%%%%
\section{Постановка задачи и~основной результат}
\label{sec:results}


В рамках многомерного дисперсионного анализа исследуется следующая многомерная 
линейная модель:
$
X\hm=Q\mathbb{B}\hm+\mathcal{E},
$
где $X$~--- случайная матрица наблюдений размера $N \times p$; $Q$~--- неслучайная 
матрица плана эксперимента размера $N \times k$;
$\mathbb{B}$~---  неслучайная матрица $k \times p$ регрессионных коэффициентов;
$\mathcal{E}$~--- матрица ошибок  $N \times p$ с~распределением $N_{N\times 
p}\left(O,I_{N}\otimes\Sigma\right)$.

Рассмотрим следующую линейную гипотезу:
$
H_{0} : C\mathbb{B}\hm=O,
$
где $C$~--- известная матрица размера $q \times k$ ранга~$q$.  Статистики 
критериев, инвариантные относительно некоторой группы аффинных преобразований, 
оказываются функциями от ненулевых собственных значений матрицы $S_{h}S_{e}^{-1}$, где
\begin{equation}
\left.
\begin{array}{rl}
S_{h}&=\hat{\mathbb{B}}^{\mathrm{T}}C^{\mathrm{T}}\left(C\left(Q^{\mathrm{T}}Q\right) ^{-1}C^{\mathrm{T}}\right)^{-
1}C\hat{\mathbb{B}} \,;\\[6pt]
 S_{e}&=\left(X-Q\hat{\mathbb{B}}\right)^{\mathrm{T}}\left(X-
Q\hat{\mathbb{B}}\right)
\end{array}
\right\}
\label{S_h_and_S_e}
\end{equation}
при $\hat{\mathbb{B}}=\left(Q^{\mathrm{T}}Q\right)^{-1}Q^{\mathrm{T}}X$ (см.~[4, гл.~8]). 
Одной из наиболее известных инвариантных статистик является 
статистика Барт\-лет\-та--Нан\-да--Пи\-лая:
$
V_{\mathrm{BNP}}\hm=\left(n\hm+q\right)\mathrm{tr}\, S_{h}\left(S_{h}\hm+S_{e}\right)^{-1}.
$
В~дальнейшем предполагаем, что гипотеза~$H_0$ верна.

В~\cite{lit:MuirLargeSamp} рассмотрен случай большого объема выборки, т.\,е.\ 
выполнено \textit{условие}~\textbf{А1}:
$$
{\bf A1}: p\mbox{ и~}q\mbox{ фиксированы},\ n\rightarrow\infty ,
$$
и получены неасимптотические  оценки точности аппроксимации функции 
распределения статистики Барт\-лет\-та--Нан\-да--Пи\-лая:
\begin{multline*}
    \mathbf{P}\{V_{\mathrm{BNP}}<x\}=G_{a}\left(x\right)+\fr{3a}{4n}\{G_{a}\left(x\right)-{}\\
{}-2G_{a+2}\left(x\right)+G_{a+4}\left(x\right)\}+O\left(n^{-2}\right),
\end{multline*}
где $a=pq$; $G_{a}$~--- функция $\chi^{2}$-рас\-пре\-де\-ле\-ния
с~$a$~степенями свободы. В~\cite{lit:LU01} для остаточного члена найде\-на 
оценка сверху.

В~\cite{lit:WFU} рассмотрен случай большой размерности данных, т.\,е.\ выполнено 
\textit{условие} \textbf{А2}:
\begin{multline*}
{\bf A2}: q\mbox{ фиксировано},\enskip p\rightarrow\infty,\enskip n\rightarrow\infty,\\ 
\fr{p}{n}\rightarrow c\in\left(0;1\right),
\end{multline*}
и получено следующее приближение:
\begin{multline*}
\mathbf{P}\left(\fr{1}{\sigma}T_{\mathrm{BNP}}<z\right)={}\\
{}=\Phi(z)-
\phi(z)\left[\fr{1}{\sqrt{p}}\left\{\fr{1}{\sigma}b_{1}+\fr{1}{\sigma^{3}}
b_{3}\,H_{2}\left(z\right)\right\} + \right.\\
\left.{}+\fr{1}{p}\left\{\fr{1}{\sigma^{2}}b_{2}\,H_{1}\left(z\right) + 
\fr{1}{\sigma^{4}}b_{4}\,H_{3}\left(z\right) + 
\fr{1}{\sigma^{6}}b_{6}\,H_{5}\left(z\right)\right\}\right]
+{}\\
{}+O\left(\fr{1}{p\sqrt{p}}\right),
\end{multline*}
где   
$T_{\mathrm{BNP}}=\sqrt{p}\left(1+m^{-1}p\right)\left\{p^{-1}V_{\mathrm{BNP}}-q\right\}$;
$\Phi (z)$ и~$\phi (z)$~--- соответственно функция распределения 
и~плотность распределения стандартного нормального закона;
$m\hm=n\hm-p\hm+q$; $r \hm= p/m$;
$\sigma\hm=\sqrt{2q(1+r)}$; $b_{i}\hm=b_{i}(r,q)$ суть некоторые функции от~$r$ 
и~$q$;
$H_{i}(z)$~--- полиномы Эрмита.
При этом результат имел именно асимптотический вид, верхние оценки остаточного 
члена не находились.

Основной результат данной работы~--- две тео\-ре\-мы,
дающие оценку точности аппроксимации распределения статистики Барт\-лет\-та--Нан\-да--Пи\-лая
нормальным распределением для данных большой размерности, т.\,е.\ при выполнении 
условия~\textbf{A2}:


\smallskip

\noindent
\textbf{Теорема~1.}
\textit{При всех $m\hm>M\hm=M\left(r,q\right)$ справедливо неравенство}
    $$
     \sup\limits_{z}{\left|\mathbf{P}\left(\fr{T_{\mathrm{BNP}}}{\sqrt{2q\left(1+r\right)}}<z\right)-
\Phi\left(z\!\right)\right|}\leqslant
\fr{K_{2}\left(r, q\right)\ln m}{\sqrt{m}}\,,
    $$
   \textit{где $K_{2}\left(r, q\right)$~--- вычислимая функция от~$r$ и~$q$}.

\smallskip

Отметим, что результат теоремы~1 на логарифмический 
множитель уступает результату из~\cite{lit:WFU}, но превосходит последний в~том, 
что для ошибки погрешности дается вычислимая оценка сверху. При этом само 
доказательство является новым.

\smallskip

\noindent
\textbf{Теорема~2.}
\textit{Пусть матрицы $U$ и~$V$ суть нормированные варианты матриц~$B$ и~$W$}:
    \begin{equation}
    \label{U and V}
    U=  \fr{B-pI_{q}}{\sqrt{p}}\,;\quad V= \fr{W-mI_{q}}{\sqrt{m}}\,,
    \end{equation}
     \textit{где $B$ и~$W$ независимы и~имеют распределения Уишарта $W_{q}\left(p, 
I_{q}\right)$ и~$W_{q}\left(m, I_{q}\right)$ с~$m\hm=n\hm-p\hm +q$ соответственно.
    Если $\mathrm{tr}{\left(\sqrt{r}\,U\hm+V\right)^{2}}\hm<\left(r\hm+1\right)m$, то выполнено 
следующее неравенство}:

\noindent
    \begin{multline}
    \label{Theor2}
     \left|\sqrt{m}\left(r+1\right)\left(\left(r+1\right)\mathrm{tr}\,{B\left(B+W\right)^{-1}}-
     rq\right)-{}\right.\\
\left.     {}-\left(\sqrt{r}\,\mathrm{tr}\,{U}-r\mathrm{tr}\,{V}\right)
\vphantom{\sqrt{}W^{-1}}
\right|\leqslant{}\\
{}\leqslant
\fr{ \left(r+1\right)\sqrt{r}\left(\left| \mathrm{tr}\,{UV}\right|+
\sqrt{r}\,\mathrm{tr}\,U^{2}\right)}
{ \left(r+1\right)\sqrt{m}-
{\mathrm{tr}{\left(\sqrt{r}\,U+V\right)^{2}}}/{\sqrt{m}}}+{}\\
{}+
     \left(   rq\left(r+1\right)+\fr{\left(\sqrt{r}\,\mathrm{tr}\,{U}-r\mathrm{tr}\,{V}\right)}
     {\sqrt{m}}\right)\times{}\\
     {}\times 
     \fr{\mathrm{tr}{\left(\sqrt{r}\,U+V\right)^{2}}}
     { \left(r+1\right)\sqrt{m}-
{\mathrm{tr}{\left(\sqrt{r}\,U+V\right)^{2}}}/{\sqrt{m}}}.
    \end{multline}



Заметим, что   вероятность события, противоположного событию 
$\mathrm{tr}{\left(\sqrt{r}\,U\hm+V\right)^{2}}\hm<\left(r\hm+1\right)m$, фигурирующему 
в~теореме~2, имеет порядок $O\left( {1}/{\sqrt{m}}\right)$, 
как это станет ясно из результатов разд.~\ref{sec:lemmas}.

\section{Вспомогательные утверждения}
\label{sec:lemmas}


В этой части приведены вспомогательные утверж\-де\-ния, используемые 
в~доказательствах тео\-рем~1 и~2.

Введем дополнительные случайные величины:
\begin{equation}
\left.
\begin{array}{rlrl}
Z_1 &= \mathrm{tr}\,{UV};&\hspace{1cm} Z_3 &= \mathrm{tr}\,{U}-\sqrt{r}\,\mathrm{tr}\,{V};\\[6pt]
Z_2 &= \mathrm{tr}\,{V^{2}}; &\hspace{1cm} Z_4 &= \mathrm{tr}\,{U^{2}},
\end{array}
\right\}
\label{Z_i}
\end{equation}
где случайные матрицы~$U$ и~$V$ определены в~\eqref{U and V}.

\pagebreak


Положим
\begin{equation*}
%\label{def_B}
    B = B(q, r, m) = 4\,\left(q^2 + \sqrt{r}\right)\left(\sqrt{\ln m} + \sqrt{\ln p}\right)^2.
\end{equation*}

Определим также для $i \hm= 1, 2, 3, 4$ и~натуральных~$m$ случайные события $A_{i, m}$ 
как 
$$
A_{i, m} = \left\{\omega : |Z_i(\omega)|\leqslant B \right\}.
$$

Положим
\begin{multline*}
%\label{Z}
Z =
   %\left( 
   \vphantom{\fr{\left|\mathrm{tr}\,{U}-\sqrt{r}\,\mathrm{tr}\,{V}\right|}{\sqrt{m}}}
  \fr{\left(r+1\right)\sqrt{r}\left(\left|\mathrm{tr}\,{UV}\right|+\sqrt{r}\,\mathrm{tr}\,{U^{2}}\right)}
  {\left(r+1\right)\sqrt{m}-{S_{Z}}/{\sqrt{m}}}+{}\\
  {}+
\fr{
          rq\left(r+1\right)+\sqrt{r/m}\,{\left|\mathrm{tr}\,{U}-\sqrt{r}\,\mathrm{tr}\,{V}\right|}
}{\left(r+1\right)\sqrt{m}-{S_{Z}}/{\sqrt{m}}}\,
S_{Z}\,,
\end{multline*}
где $S_{Z}=\left(r\mathrm{tr}\,{U^{2}}+2\sqrt{r}\left|\mathrm{tr}\,{UV}\right|+\mathrm{tr}\,{V^{2}}\right).$

Ясно, что существует натуральное $M_1 \hm= M_1(r, q, c)$ такое, что при всех $m\hm\geq 
M_1$ и~$\omega \hm\in \mathop{\cap}\nolimits_{i=1}^4 A_{i, m}$ выполняется
\begin{multline}
\label{ineq_Z}
Z(\omega) \leqslant
\left(
(r+1)\sqrt{r}\left(1+\sqrt{r}\right)B+
    \left(
    \vphantom{\fr{B}{\sqrt{m}}}
    rq(r+1)+{}\right.\right.\\
\left.\left.    {}+\sqrt{r}\,\fr{B}{\sqrt{m}}\right)B\left(1+\sqrt{r}\right)^{2}\right)\Bigg/
\left( 
\vphantom{\fr{B\left(1+\sqrt{r}\right)^{2}}{\sqrt{m}}}
(r+1)\sqrt{m}-{}\right.\\
\left.{}-\fr{B\left(1+\sqrt{r}\right)^{2}}{\sqrt{m}}\right)
\leqslant{}\\
{}\leqslant 
16\,\fr{\left(1+\sqrt{r}\right)^{2}\,\left(2\sqrt{r}+r\,(rq+q+1)\right)
\left(q^2+\sqrt{r}\right)}{r+1}\times{}\\
{}\times \fr{\left(\ln m + \ln\sqrt{r}\right)}{\sqrt{m}}\,.
\end{multline}

Оценим вероятности $\mathbf{P}(A_{i, m}^c)$ для $i \hm= 1, 2, 3, 4$. Согласно 
леммам~1 и~2 из~\cite{lit:LU02} справедливо сле\-ду\-ющее неравенство:
\begin{multline}
    \label{L-1, f-la}
\mathbf{P}\left(|Z_1| > B\right) + \mathbf{P}\left(Z_2 > B\right) +
 \mathbf{P}\left(|Z_3| > B\right) + {}\\
 {}+
\mathbf{P}\left(Z_4 > B\right) \leqslant 25{,}8\,q^2\,\fr{1+1/\sqrt{r}}{\sqrt{m}}\,.
\end{multline}

\noindent
\textbf{Лемма~1.}
\textit{Пусть случайные величины $T$, $Y$ и~$Z$ определены на одном вероятностном 
пространстве $\left(\Omega,\mathbf{A},\mathbf{P}\right)$,
при этом распределение~$Y$ является абсолютно непрерывным с~ограниченной 
плотностью $f_{Y}\left(z\right)$.
Предположим, что для некоторого события $A\hm\in \mathbf{A}$ при всех $\omega\hm\in A$ 
выполнено следующее соотношение}:
$$
\left|T(\omega)-Y(\omega)\right|\leqslant Z(\omega) \leqslant a
$$
\textit{с некоторой положительной постоянной~$a$.
Тогда
справедливо неравенство}:
\begin{multline}
\label{uniform}
\sup\limits_{x}\left|\mathbf{P}(T<x)-\mathbf{P}(Y<x)\right| \leqslant{}\\
{}\leqslant \mathbf{P}(A^c) + 
a\sup\limits_{x}{f_{Y}\left(x\right)}.
\end{multline}  

\noindent
Д\,о\,к\,а\,з\,а\,т\,е\,л\,ь\,с\,т\,в\,о\ \ леммы~1. См.\ лемму~3 в~\cite{lit:LU02}.~\hfill $\Box$

В следующих двух леммах приводятся два известных результата о скорости 
сходимости в~центральной предельной теореме для независимых одинаково 
распределенных случайных величин.
Первый из результатов относится к~случайным величинам без ограничений на тип 
распределения.
Второй результат относится к~случайной величине с~распределением $\chi^{2}$, 
рассматриваемой как сумма независимых одинаково распределенных 
случайных величин с~известным распределением. Согласно~\cite{lit:Shevtsova1}, справедлива 
следующая лемма.

\noindent
\textbf{Лемма~2.}
\textit{Пусть случайные величины $\xi_{1},\,\xi_{2},\ldots$ независимы и~одинаково 
распределены, выполнено $\mathbf{D}\xi_{1}\hm=\sigma^{2}\hm>0$ и~cуществует 
$\mathbf{E}\left|\xi_{1}\right|^{3}\hm<\infty$. Тогда  для нормированной суммы}
$
T_{n}\hm=
{(S_{n}-\mathbf{E}{S_{N}})}/{\sqrt{\mathbf{D}\,{S_{N}}}}
$
\textit{выполнено неравенство}:
$$
\sup\limits_{x}{\left|F_{T_{n}}\left(x\right)-\Phi\left(x\right)\right|}\leqslant 
0{,}4748\fr{\mathbf{E}\left|\xi_{1}-
\mathbf{E}\xi_{1}\right|^{3}}{\sigma^{3}\sqrt{n}}\,.
$$


Случайная величина с~функцией распределения $G_{p}(x)$, имеющая   $\chi^{2}$-рас\-пре\-де\-ле\-ние 
с~$p$~степенями свободы, может быть представлена в~виде суммы~$p$~независимых 
одинаково распределенных случайных величин с~$\chi^{2}$-рас\-пре\-де\-ле\-ни\-ем с~одной степенью свободы.
Этот факт позволяет дать более точные оценки точ\-ности аппроксимации нормальным 
распределением, чем те, которые можно получить в~общем случае с~помощью
неравенства Бер\-ри--Ес\-се\-ена, а~именно: имеет место следующий результат (см.\ лемму~2 
в~\cite{lit:Kavaguchi} при $\lambda\hm=0{,}5$).

\smallskip

\noindent
\textbf{Лемма~3.}
\textit{Для всех $\lambda\hm\in\left(0; \sqrt{3}-1\right)$ и~целых $p\hm>1$ выполнено}
    $$
    \sup\limits_{x}{\left|G_{p}\left(p+x\sqrt{2p}\right)-
\Phi\left(x\right)\right|}\leqslant \fr{6{,}22}{\sqrt{p}}\,.
    $$

\smallskip

\noindent
\textbf{Лемма~4.}
\textit{Для любых случайных величин $X$ и~$Y$ и~любого действительного числа $a > 
0$ справедливы неравенства}:
    \begin{equation*}
    %\label{split_formula_1}
    \mathbf{P}(|X+Y|\geq 2a) \leqslant \mathbf{P}(|X|\geq a)  + 
\mathbf{P}(|Y|\geq a)\,;
    \end{equation*}
          \begin{equation*}
  %  \label{split_formula_2}
    \mathbf{P}(|X\cdot Y|\geq a^2) \leqslant \mathbf{P}(|X|\geq a)  + 
\mathbf{P}(|Y|\geq a).
    \end{equation*}

\noindent
Д\,о\,к\,а\,з\,а\,т\,е\,л\,ь\,с\,т\,в\,о\ \ леммы~4 очевидным образом вытекает из рассуждений 
от противного.~\hfill $\Box$

\smallskip

\noindent
\textbf{Лемма~5.}
\textit{Если случайные величины~$X_1, \dots , X_k$ независимы и~таковы, что   $|\mathbf{P}(X_j 
\hm\leqslant x) \hm- \Phi (x)| \hm\leqslant D_j$ при всех~$x$ и~$j \hm= 1, \dots , k$
 с~некоторыми постоянными   $D_1, \dots , D_k$, то}
  \begin{equation*}
  \left| \mathbf{P}\left(\sum\limits_{j=1}^{k}c_j\,X_j \leqslant x\right) - \Phi (x)\right| \leqslant 
\sum\limits_{j=1}^{k} D_j,
  \end{equation*}
\textit{где $c_1,\dots , c_k $ суть произвольные постоянные, для которых}   $c_1^2 + 
\dots + c_k^2 \hm= 1$.

\noindent
Д\,о\,к\,а\,з\,а\,т\,е\,л\,ь\,с\,т\,в\,о\ \ леммы~5 см., например, в~теореме~3.1 
в~~\cite{lit:Letters_2006}. \hfill $\Box$



\section{Доказательства теорем~1 и~2}
\label{sec:proofs}


Начнем с~доказательства теоремы~2, поскольку неравенство~\eqref{Theor2} 
является ключевым в~доказательстве 
теоремы~1.

\noindent
Д\,о\,к\,а\,з\,а\,т\,е\,л\,ь\,с\,т\,в\,о\ \ теоремы~2.
Воспользовавшись матричным равенством
\begin{equation*}
%\label{Matr Geom}
\left(I+A\right)^{-1}-\left(I-A\right)=A^{2}\left(I+A\right)^{-1},
\end{equation*}
из определения \eqref{U and V} получаем
\begin{multline*}
\left(B+W\right)^{-1}=\left(\sqrt{p}\,U+pI_{q}+\sqrt{m}\,V+mI_{q}\right)^{-1}={}\\
{}=\fr{1}{p+m}\left(
\vphantom{\fr{\left(\sqrt{p}\,U+\sqrt{m}\,V\right)^{2}}
{p+m}}
I_{q}-
\fr{1}{p+m}\left(\sqrt{p}\,U+\sqrt{m}\,V\right)+{}\right.\\
\left.{}+\fr{\left(\sqrt{p}\,U+\sqrt{m}\,V\right)^{2}}
{p+m}\left(B+W\right)^{-1}\right),
\end{multline*}
далее
\begin{multline*}
%\label{T2_1}
\sqrt{m}\left(r+1\right)\left(\left(r+1\right)B\left(B+W\right)^{-1}-
rI_{q}\right)-{}\\
{}-\left(\sqrt{r}\,U-rV\right)=\fr{\sqrt{r}}{\sqrt{m}}U\left(\sqrt{r}\,U+V\right)+{}\\
{}+
        \fr{1}{\sqrt{m}}\,B\left(\sqrt{r}\,U+V\right)^{2}\left(B+W\right)^{-1}.
\end{multline*}
Отсюда для следов этих матриц имеем следующее неравенство:
\begin{multline}
\label{trBW}
\left|\sqrt{m}\left(r+1\right)\left(\left(r+1\right)\mathrm{tr}\,{B\left(B+W\right)^{-1}}-
rq\right)-{}\right.\\
\left.{}-\left(\sqrt{r}\,\mathrm{tr}\,{U}-r\mathrm{tr}\,{V}\right)
\vphantom{\left(W\right)^{-1}}
\right|
\leqslant{}\\
{}\leqslant \fr{1}{\sqrt{m}}\sqrt{r}\left(\left|\mathrm{tr}\,{UV}\right|+\sqrt{r}\,\mathrm{tr}\,{U^{2}}\right)+{}\\
{}+
\fr{1}{\sqrt{m}}\left|\mathrm{tr}\,{\left[B\left(\sqrt{r}\,U+V\right)^{2}\left(B+W\right)^{-1}\right]}
\right|\leqslant{}\\
{}\leqslant
\fr{1}{\sqrt{m}}\sqrt{r}\left(\left|\mathrm{tr}\,{UV}\right|+\sqrt{r}\,\mathrm{tr}\,{U^{2}}\right)+{}\\
{}+
\fr{1}{\sqrt{m}}\,\mathrm{tr}\,{\left(\sqrt{r}\,U+V\right)^{2}}\mathrm{tr}\,{B\left(B+W\right)^{-1}}.
\end{multline}
Для получения предпоследнего неравенства использованы симметричность 
и~неотрицательная определенность обеих случайных матриц 
$\left(\sqrt{r}\,U+V\right)^{2}$ и~$B\left(B+W\right)^{-1}$, поскольку для 
сим\-мет\-рич\-ных неотрицательно определенных матриц~$X$ и~$Y$ выполнено   
соотношение (см.~\cite{lit:TraceIneq})
$\mathrm{tr}\,{XY}\hm\leqslant\mathrm{tr}\,{X}\mathrm{tr}\,{Y}.$

Видно, что случайная величина $\mathrm{tr}\,{B\left(B+W\right)^{-1}}$ фигурирует в~крайней 
левой и~крайней правой час\-тях неравенства~\eqref{trBW}.
Преобразуя полученное неравенство, получаем, что при 
$\mathrm{tr}\,{\left(\sqrt{r}\,U+\hm V\right)^{2}}\hm<\left(r\hm+1\right)m$ выполнено~\eqref{Theor2}. 
Тем самым доказательство теоремы~2 завершено.
\hfill $\Box$

\smallskip

Переходим к~доказательству теоремы~1.

\smallskip

\noindent
Д\,о\,к\,а\,з\,а\,т\,е\,л\,ь\,с\,т\,в\,о\ \ теоремы~1.
Используя лемму~1 из~\cite{lit:WFU}, перейдем к~представлению статистики 
Барт\-лет\-та--Нан\-да--Пи\-лая
\begin{multline*}
T_{\mathrm{BNP}}={}\\
{}=\sqrt{p}\left(\!1+\fr{p}{m}\!\right)\!\left\{\!\left(1+\fr{m}{p}\right)\mathrm{tr}\!\left[S_{h}
\left(S_{h}+S_{e}\right)^{-1}\right]-q\right\}\hspace*{-4.9pt}
\end{multline*}
в терминах матриц~$B$ и~$W$ размера $q \times q$ вместо матриц~$S_{h}$ и~$S_{e}$ 
размера $p \times p$, где $S_{h}$ и~$S_{e}$ определены в~\eqref{S_h_and_S_e}, 
а~матрицы~$B$ и~$W$ независимы и~имеют распределения Уишарта $W_{q}\left(p, 
I_{q}\right)$ и~$W_{q}\left(m, I_{q}\right)$ с~$m\hm=n\hm-p\hm+q$ соответственно.
При этом будем пользоваться сле\-ду\-ющим соотношением (см.~\cite{lit:WFU}):
$$
\mathrm{tr}\,{S_{h}\left(S_{h}+S_{e}\right)^{-1}}=
\mathrm{tr}\,{B\left(B+W\right)^{-1}}.
$$

Согласно~\eqref{Theor2} для $Z_1$, $Z_2$, $Z_3$ и~$Z_4$ (см.\ определение 
в~\eqref{Z_i}) при $rZ_{4}\hm+2\sqrt{r}\left|Z_{1}\right|\hm+Z_{2} \hm< m$ 
имеем:
\begin{multline*}
%\label{proof_T-1-1}
|\sqrt{r}\,T_{\mathrm{BNP}} - \left(\sqrt{r}\,\mathrm{tr}\,{U}-r\mathrm{tr}\,{V}\right)|={}\\
{}= \left|\sqrt{m}\left(r+1\right)\left(\left(r+1\right)\mathrm{tr}\,{B\left(B+W\right)^{-1}}
-rq\right)-{}\right.\\
\left.{}-\left(\sqrt{r}\,\mathrm{tr}\,{U}-r\mathrm{tr}\,{V}\right)
\vphantom{\left(W\right)^{-1}}
\right|\leqslant{}\\
{}\leqslant
  \left( 
   \vphantom{\fr{\sqrt{r}}{\sqrt{v}}}
   \left(r+1\right)\sqrt{r}\left(\left|Z_{1}\right|+\sqrt{r}\,Z_{4}\right)+{}\right.
 \\
 {}+     \left(
\vphantom{\fr{\sqrt{r}\left|Z_{3}\right|}{\sqrt{m}}}
    rq\left(r+1\right)+\fr{\sqrt{r}\left|Z_{3}\right|}{\sqrt{m}}\right)\left(rZ_{4}+
2\sqrt{r}\left|Z_{1}\right|+{}\right.\\
\left.\left.{}+Z_{2}\right)
\vphantom{\fr{\sqrt{r}}{\sqrt{v}}}
\right)\Bigg/
    \left( \left(r+1\right)\sqrt{m}-
\fr{rZ_{4}+2\sqrt{r}\left|Z_{1}\right|+Z_{2}}{\sqrt{m}}\right).\hspace*{-6.32483pt}
\end{multline*}
Следовательно, в~силу~\eqref{ineq_Z} и~\eqref{uniform} при всех $m\hm\geq M_1$ 
получаем
\begin{multline}
\label{interm}
\mathop{\sup}\limits_{z}\left|\mathbf{P}\left (\fr{T_{\mathrm{BNP}}}
{\sqrt{2q\left(1+r\right)}}<z\right)-{}\right.\\
\left.{}-
\mathbf{P}\left(\fr{\mathrm{tr}\,{U}-\sqrt{r}\,\mathrm{tr}\,{V}}{\sqrt{2q\left(1+r\right)}} < z 
\right)\right| \leqslant{}\\
{}\leqslant \sum\limits_{i=1}^{4} \mathbf{P}\left(|Z_i|> B\right) +  K_{4}(r, q)\,
\fr{\ln m}{\sqrt{m}}\, \sup\limits_{x}{f(x)},
\end{multline}
где $f(x)$ есть плотность случайной величины ${(\mathrm{tr}\,{U}\hm-
\sqrt{r}\,\mathrm{tr}\,{V})}/{\sqrt{2q\left(1\hm+r\right)}}$;
$K_{4}\left(r, q\right)$~--- 
некоторая вычислимая функция от~$r$ и~$q$.

Отметим, что, поскольку матрицы~$B$ и~$W$ независимы, матрицы $U$ и~$V$ также 
независимы между собой. Известно (см., например, гл.~2 в~\cite{lit:FUS01}), что 
$\mathrm{tr}\, B$ и~$\mathrm{tr}\, W$ имеют $\chi^{2}$-рас\-пре\-де\-ле\-ния с~$pq$ и~$mq$ степенями свободы 
соответственно. Известно также, что плотность $\chi^{2}$-распределения с~$k\hm\geq 3$ степенями свободы ограничена сверху величиной
$1/(2\sqrt{\pi\left(k-2\right)}).$
Поэтому для плотности $f(x)$ справедлива равномерная оценка
\begin{multline}
\label{density}
f(x) \leqslant{}\\
\hspace*{-1.5mm}{}\leqslant \min{\left( \fr{\sqrt{p}}{ \sqrt{ \left(pq-
2\right)}},\,\fr{\sqrt{m}}{\sqrt{r\left(mq-2\right)}}\right)} 
\fr{\sqrt{q(1+r)}}{\sqrt{2\pi}}\,.\!\!\!
\end{multline}
Объединяя утверждения лемм~3 и~5 и~соотношения~\eqref{U and V}, \eqref{L-1, f-la}, \eqref{interm} 
и~\eqref{density}, получаем утверждение теоремы~1.\hfill $\Box$%\\[2ex]


%\vspace*{-8pt}

{\small\frenchspacing
{%\baselineskip=10.8pt
%\addcontentsline{toc}{section}{References}
\begin{thebibliography}{99}

%\vspace*{-2pt}
\bibitem{lit:FUS01}
\Au{Fujikoshi Y., Ulyanov~V.\,V., Shimizu~R.}  Multivariate statistics: High-dimensional 
and large-sample approximations.~--- Hoboken, NJ, USA: John Wiley \& Sons, 2010. 512~p.

\bibitem{lit:Johnstone_RoyStat}
\Au{Johnstone I.\,M., Nadler~B.} Roy's largest root test under rank-one 
alternatives~// Biometrika, 2017. Vol.~104. No.\,1. P.~181--193.

\bibitem{lit:Akbari_AppliedLH}
\Au{Akbari V., Anfinsen S.\,N., Doulgeris~A.\,P., Eltoft~T., Moser~G., Serpico~S.\,B.} 
Polarimetric SAR change detection with the complex Hotelling--Lawley 
trace statistic~// IEEE T.~Geosci. Remote,  2016. Vol.~54. Iss.~7. 
P.~3953--3966.

\bibitem{lit:And}
\Au{Anderson T.\,W.}  An introduction to multivariate analysis.~--- 3rd ed.~--- Hoboken, NJ,
USA: John Wiley \& Sons,  2003. 742~p.

\bibitem{lit:MuirLargeSamp}
\Au{Muirhead R.\,J.}  Asymptotic distributions of some multivariate tests~// 
Ann. Math. Stat., 1970. Vol.~41. No.\,3. P.~1002--1010.

\bibitem{lit:LU01}
\Au{Липатьев А.\,А., Улья\-нов~В.\,В.} Вы\-чис\-ли\-мые оценки точ\-ности при\-бли\-же\-ний 
для статистики Барт\-лет\-та--Нан\-да--Пил\-лай~//
Математические труды, 2016. Т.~19. №\,2. С.~109--118.

\bibitem{lit:WFU}
\Au{Wakaki~H., Fujikoshi~Y., Ulyanov~V.\,V.}  Asymptotic expansions of the 
distributions of
MANOVA test statistics when the dimension is large~// Hiroshima Math.~J., 2014. 
Vol.~44. No.\,3. P.~247--259.

\bibitem{lit:LU02}
\Au{Липатьев А.\,А., Ульянов~В.\,В.} Неасимптотический анализ статистики 
Лоу\-ли--Хо\-тел\-лин\-га для данных большой раз\-мер\-ности~//
Записки научных семинаров \mbox{ПОМИ}, 2019. Т.~486. С.~178--189.

\bibitem{lit:Shevtsova1}
\Au{Shevtsova I.\,G.}  On the absolute constants in the Berry--Esseen type 
inequalities for identically distributed summands~//
arXiv.org, 2011. arXiv:1111.6554 [math.PR]. 7~p.

\bibitem{lit:Kavaguchi}
\Au{Кавагучи Ю., Ульянов~В.\,В., Фуджикоши~Я.}  Приближения для статистик,
описывающих геометрические свойства данных большой размерности, с~оценками 
ошибок~//
Информатика и~её применения, 2010. Т.~4. Вып.~1. С.~22--27.

\bibitem{lit:Letters_2006}
\Au{Ulyanov V.\,V., Wakaki~H., Fujikoshi~Y.}  Berry--Esseen bound for high 
dimensional asymptotic approximation of Wilks' Lambda distribution~//
Stat. Probabil. Lett., 2006. Vol.~76. No.\,12. P.~1191--1200.

\bibitem{lit:TraceIneq}
\Au{Coope I.\,D.}  On matrix trace inequalities and related topics for 
products of Hermitian matrices~//
J.~Math. Anal. Appl., 1949. Vol.~188. No.\,3. P.~999--1001.
\end{thebibliography}

}
}

\end{multicols}

\vspace*{-3pt}

\hfill{\small\textit{Поступила в~редакцию 07.01.2020}}

\vspace*{8pt}

%\pagebreak

%\newpage

%\vspace*{-28pt}

\hrule

\vspace*{2pt}

\hrule

%\vspace*{-2pt}

\def\tit{NONASYMPTOTIC ANALYSIS OF~BARTLETT--NANDA--PILLAI STATISTIC FOR~HIGH-DIMENSIONAL DATA}

\def\titkol{Nonasymptotic analysis of~Bartlett--Nanda--Pillai statistic for~high-dimensional data}

\def\aut{A.\,A.~Lipatiev}

\def\autkol{A.\,A.~Lipatiev}

\titel{\tit}{\aut}{\autkol}{\titkol}

\vspace*{-11pt}


\noindent
Department of Mathematical Statistics, Faculty of Computational 
Mathematics and Cybernetics, M.\,V.~Lomonosov Moscow State University, 
1-52~Leninskiye Gory, GSP-1, Moscow 119991, Russian Federation


\def\leftfootline{\small{\textbf{\thepage}
\hfill INFORMATIKA I EE PRIMENENIYA~--- INFORMATICS AND
APPLICATIONS\ \ \ 2021\ \ \ volume~15\ \ \ issue\ 1}
}%
\def\rightfootline{\small{INFORMATIKA I EE PRIMENENIYA~---
INFORMATICS AND APPLICATIONS\ \ \ 2021\ \ \ volume~15\ \ \ issue\ 1
\hfill \textbf{\thepage}}}

\vspace*{3pt}




\Abste{The author gets the computable error bounds for normal 
approximation of Bartlett--Nanda--Pillai statistic when dimensionality grows proportionally
 to the sample size. This result enables one to get more precise calculations of the p-values 
 in applications of multivariate analysis. In practice, more and more often, analysts encounter 
 situations when the number of factors is large and comparable with the sample size. 
 The examples include signal processing. The proof is essentially based on the normality 
 of the distribution of the elements of the matrices under consideration with the Wishart 
 distribution. For random variables that are the matrix traces of the product and squares 
 of matrices with the normalized Wishart distribution, convenient upper bounds for $1-F$ are 
 found\linebreak\vspace*{-12pt}}
 
 \Abstend{where $F$ is the distribution function of the corresponding matrix trace. Applying 
 the properties of inverse matrices and positive semidefinite matrices, the Bartlett--Nanda--Pillai 
statistic is bounded from above by a combination of the above-mentioned matrix traces.}

\KWE{computable estimates; accuracy of approximation; MANOVA; computable error bounds; 
Bartlett--Nanda--Pillai statistic; high-dimensional data}

\DOI{10.14357/19922264210110}

%\vspace*{-15pt}

%\Ack
%\noindent


\vspace*{12pt}

  \begin{multicols}{2}

\renewcommand{\bibname}{\protect\rmfamily References}
%\renewcommand{\bibname}{\large\protect\rm References}

{\small\frenchspacing
 {%\baselineskip=10.8pt
 \addcontentsline{toc}{section}{References}
 \begin{thebibliography}{99}
\bibitem{1-lip-1}
\Aue{Fujikoshi, Y., V.\,V.~Ulyanov, and R.~Shimizu.} 2010. 
\textit{Multivariate statistics: High-dimensional and large-sample approximations}.
 Hoboken, NJ: John Wiley \& Sons. 512~p.
\bibitem{2-lip-1}
\Aue{Johnstone, I.\,M., and B.~Nadler.} 2017. Roy's largest root test under rank-one 
alternatives. \textit{Biometrika} 104(1):181--193.
\bibitem{3-lip-1}
\Aue{Akbari, V., S.\,N.~Anfinsen, A.\,P.~Doulgeris, T.~Eltoft, G.~Moser, and S.\,B.~Serpico.}
 2016. Polarimetric SAR change detection with the complex Hotelling--Lawley trace statistic. 
 \textit{IEEE~T. Geosci. Remote} 54(7):3953--3966.
\bibitem{4-lip-1}
\Aue{Anderson, T.\,W.} 2003. \textit{An introduction to multivariate analysis}. 
3rd ed. Hoboken, NJ: John Wiley \& Sons. 742~p.
\bibitem{5-lip-1}
\Aue{Muirhead, R.\,J.} 1970. Asymptotic distributions of some multivariate tests. 
\textit{Ann. Math. Stat.} 41(3):1002--1010.
\bibitem{6-lip-1}
\Aue{Lipatiev, A.\,A., and V.\,V.~Ulyanov.} 2017. On computable estimates for accuracy of approximation
 for the Bartlett--Nanda--Pillai statistic. \textit{Siberian Adv. Math.} 27(3):153--159.
\bibitem{7-lip-1}
\Aue{Wakaki, H., Y.~Fujikoshi, and V.\,V.~Ulyanov.} 2014. Asymptotic expansions of the distributions 
of MANOVA test statistics when the dimension is large. \textit{Hiroshima Math.~J.} 44(3):247--259.
\bibitem{8-lip-1}
\Aue{Lipatiev, A.\,A., and V.\,V.~Ulyanov.}
 2019. Neasimptoticheskiy analiz statistiki Louli--Khotellinga dlya dannykh bol'shoy razmernosti 
 [Nonasymptotic analysis of Lawley--Hotelling statistic for high dimensional data]. 
 \textit{Zapiski nauchnykh seminarov POMI} [POMI Notes of Scientific Seminars] 486:178--189.
\bibitem{9-lip-1}
\Aue{Shevtsova, I.\,G.} 2011. On the absolute constants in the Berry--Esseen type inequalities 
for identically distributed summands. arXiv:1111.6554 [math.PR]. Available at: 
{\sf https://arxiv.org/pdf/1111.6554} (accessed December~16, 2020).
\bibitem{10-lip-1}
\Aue{Kawaguchi, Yu., V.\,V.~Ulyanov, and Ya.~Fujikoshi.}
 2010. Priblizheniya dlya statistik, opisyvayushchikh geo\-met\-ri\-che\-skie svoystva dannykh 
 bol'shoy razmernosti, s~otsenkami oshibok [Asymptotic distributions of basic statistics in 
 geometric representation for high-dimensional data and their error bounds]. 
 \textit{Informatika i~ee Primeneniya~--- Inform. Appl.} 4(1):22--27.
\bibitem{11-lip-1}
\Aue{Ulyanov, V.\,V., H.~Wakaki, and Y.~Fujikoshi.} 2006. 
Berry--Esseen bound for high dimensional asymptotic approximation of Wilks' Lambda distribution. 
\textit{Stat. Probabil. Lett.} 76(12):1191--1200.
\bibitem{12-lip-1}
\Aue{Coope, I.\,D.} 1949. On matrix trace inequalities and related topics for products of Hermitian matrices. 
\textit{J.~Math. Anal. Appl.} 188(3):999--1001.
 \end{thebibliography}

 }
 }

\end{multicols}

\vspace*{-3pt}

  \hfill{\small\textit{Received January~7, 2020}}


%\pagebreak

%\vspace*{-8pt}

\Contrl

\noindent
\textbf{Lipatiev Alexander A.} (b.\ 1988)~--- 
PhD student,  Faculty of Computational Mathematics and Cybernetics,
 M.\,V.~Lomonosov Moscow State University, 1-52~Leninskiye Gory, GSP-1, Moscow 119991, 
 Russian Federation; \mbox{allipatev@cs.msu.ru}


\label{end\stat}

\renewcommand{\bibname}{\protect\rm Литература}     %10
\include{stup-bruhov}  %11
\def\stat{gon-zats}

\def\tit{ПРЕДСТАВЛЕНИЕ НОВЫХ ЛЕКСИКОГРАФИЧЕСКИХ ЗНАНИЙ В~ДИНАМИЧЕСКИХ 
КЛАССИФИКАЦИОННЫХ СИСТЕМАХ$^*$}

\def\titkol{Представление новых лексикографических знаний в~динамических 
классификационных системах}

\def\aut{А.\,А.~Гончаров$^1$, И.\,М.~Зацман$^2$, М.\,Г.~Кружков$^3$}

\def\autkol{А.\,А.~Гончаров, И.\,М.~Зацман, М.\,Г.~Кружков}

\titel{\tit}{\aut}{\autkol}{\titkol}

\index{Гончаров А.\,А.}
\index{Зацман И.\,М.}
\index{Кружков М.\,Г.}
\index{Goncharov A.\,A.}
\index{Zatsman I.\,M.}
\index{Kruzhkov M.\,G.}

{\renewcommand{\thefootnote}{\fnsymbol{footnote}} \footnotetext[1]
{Работа выполнена в~Институте проблем информатики ФИЦ ИУ РАН при поддержке РФФИ (проект  
20-012-00166).}}

\renewcommand{\thefootnote}{\arabic{footnote}}
\footnotetext[1]{Институт проблем информатики Федерального исследовательского центра <<Информатика и~управление>> 
Российской академии наук, \mbox{a.gonch48@gmail.com}}
\footnotetext[2]{Институт проблем информатики Федерального исследовательского центра <<Информатика и~управление>> 
Российской академии наук, \mbox{izatsman@yandex.ru}}
\footnotetext[3]{Институт проблем информатики Федерального исследовательского центра <<Информатика и~управление>> 
Российской академии наук, \mbox{magnit75@yandex.ru}}


%\vspace*{-12pt}


     
     \Abst{Характерная особенность динамических классификационных систем (ДКС) состоит 
     в~том, что в~процессе применения этих систем в~них в~любой момент времени могут 
добавляться новые рубрики и/или изменяться дефиниции существующих рубрик, включая 
перераспределение смыслового содержания между ними. С~одной стороны, эта 
особенность ДКС дает возможность оперативно 
отражать в~них новое знание и~сразу начинать его использовать, например в~процессе 
лингвистического аннотирования. С~другой стороны, если некоторая рубрика 
использовалась при аннотировании, а~затем была изменена, то аннотации с~этой 
рубрикой, сформированные до внесения изменений, в~ряде случаев должны быть 
реклассифицированы. Статья преследует двоякую цель, которая состоит, во-пер\-вых, 
в~сопоставлении подходов к~классификации сущностей на основе (1)~ДКС 
и~(2)~онтологий, изменяемых во времени, а~во-вто\-рых, 
в~описании специфики представления новых лексикографических знаний в~ДКС.}
     
     \KW{динамическая классификационная система; версионные онтологии; 
лингвистическое аннотирование; реклассификация аннотаций}

\DOI{10.14357/19922264210112}


\vskip 10pt plus 9pt minus 6pt

\thispagestyle{headings}

\begin{multicols}{2}

\label{st\stat}
     
\section{Введение}

Характерная особенность ДКС 
состоит в~том, что в~процессе применения в~них могут добавляться новые 
рубрики и/или изменяться дефиниции существующих рубрик, включая 
перераспределение смыслового содержания между ними. В~отличие от 
версионных классификационных систем, для которых установлен период их 
обновления (например, рубрики Международной патентной классификации 
могут меняться не чаще, чем раз в~квартал~[1]), добавления и~изменения 
в~ДКС в~случае необходимости могут быть сделаны в~любой момент 
времени. С~одной стороны, эта особенность ДКС позволяет оперативно 
отражать в~них новое знание и~сразу начинать его использовать, например 
в~процессе лингвистического аннотирования~[2]. С~другой стороны, если 
некоторая рубрика использовалась при аннотировании, а~затем была 
изменена, то аннотации с~этой рубрикой, сформированные до внесения 
изменений, в~ряде случаев должны быть реклассифицированы~[3].

Цель статьи состоит в~сопоставлении подходов к~классификации сущностей 
на основе (1)~ДКС и~(2)~онтологий, изменяемых во времени, а также 
в~описании специфики представления новых лексикографических знаний 
в~ДКС. В~качестве примера ДКС в~статье рассматривается фасетная 
классификация (ФК) надкорпусной базы данных (НБД)~[4--6]. В~проекте по 
гранту №\,20-012-00166\linebreak НБД используется для аннотирования таких 
сущностей, как употребления немецких модальных глаго\-лов (НМГ) 
в~параллельных текстах~[7--9], в~процессе которого могут 
(1)~обнаруживаться новые значения НМГ, (2)~добавляться новые рубрики 
для этих значений в~момент их обнаружения~[10, 11] и~(3)~изменяться 
дефиниции рубрик ФК~[3].

\vspace*{-14pt}

\section{Онтологии, изменяемые во~времени}

\vspace*{-4pt}

Методы аннотирования сущностей с~использованием, с~одной стороны, ДКС 
и,~с~другой стороны, онтологий, изменяемых во времени, во многом схожи. 
При аннотировании сущностей \mbox{с~по\-мощью} ДКС им присваиваются рубрики, в~том числе 
до\-бав\-лен\-ные или измененные непосредственно в~процессе 
формирования ка\-кой-ли\-бо аннотации. При классификации сущностей 
с~по\-мощью онтологии для них устанавливаются атрибуты, пред\-став\-ля\-ющие 
собой ссылки на концепты онтологии. С~течением времени могут меняться 
структура и~наполнение как ДКС, так и~онтологии. Это может быть связано 
с~изменениями (1)~самой предметной об\-ласти, (2)~экспертных знаний об этой 
области и/или (3)~стандартов ее описания~[12].

Кроме этого, при проведении исследований научные коллективы могут 
пользоваться теми классификационными системами и~онтологиями для \mbox{своей} 
предметной области, которые находятся в~откры\-том доступе, а~затем 
модифицировать их в~зависимости от целей и~задач выполняемых проектов. 
В~силу перечисленных сходств работы, связанные с~эволюцией 
и~версионностью онтологий, представляют значительный интерес для 
разработки средств актуализации ДКС.

В~[12] описывается подход к~управлению версионностью онтологий 
и~предлагается система нумерации версий онтологий, позволяющая 
определять, обладают ли версии свойством обратной \mbox{совместимости}, или 
эквивалентности (с~точ\-ностью до синтаксических различий). Обратная 
совместимость важна потому, что в~случае ее сохранения снимается 
необходимость в~реклассификации ранее сформированных аннотаций (об 
этой проблеме применительно к~ДКС на примере ФК НБД см.~[3]). 
Сохранять историю изменений понятий онтологии предлагается либо 
в~отдельной онтологии, либо в~выделенных для этого классах исходной 
онтологии. В~\cite{13-gon} представлены инструменты, позволяющие 
выделять и~визуализировать структурные различия между версиями одной 
и~той же онтологии.

В~\cite{14-gon} показано, как с~помощью онтологий могут фиксироваться 
временн$\acute{\mbox{ы}}$е данные, знания, правила и~отношения, 
и~рассмотрены подходы к~описанию изменений в~онтологиях. Темпоральная 
дескриптивная логика (Temporal Description Logics) служит основой 
дополнения онтологии логическими функциями для работы со временем, 
поз\-во\-ля\-ющи\-ми выражать и~обрабатывать такие концепты,\linebreak как <<постоянно 
в~прошлом>> или <<в~некоторый момент в~будущем>>. Для регистрации 
фактов, относящихся к~определенному периоду времени, используются 
подходы, позволяющие обходить \mbox{ограничения} большинства онтологий, 
опирающихся на язык Web Ontology Language (OWL): реификация 
(Reification), четырехмерные (4D-fluent) онтологии и~переход к~n-ар\-ным 
отношениям (N-ary relations). Поскольку по умолчанию отношения 
в~онтологиях OWL являются бинарными, тогда как для фиксации 
временн$\acute{\mbox{о}}$го интервала события необходим по крайней мере 
один дополнительный аргумент, все перечисленные подходы 
характеризуются тем, что для описания временн$\acute{\mbox{ы}}$х фактов 
в~онтологиях каждый раз создается новая сущность.\linebreak На основе 
вышеназванных подходов разработана структура SOWL (spatio-temporal OWL), пред\-на\-значенная 
для включения %\linebreak 
про\-стран\-ст\-вен\-но-вре\-мен\-н$\acute{\mbox{о}}$й информации 
в~онтологии OWL, а~также\linebreak инструмент CHRONOS, упро\-ща\-ющий создание 
и~редактирование данных об \mbox{изменениях} в~онтологиях, ис\-поль\-зу\-ющих эту 
структуру. Кроме того, для извлечения информации из SOWL-он\-то\-ло\-гий 
создан язык запросов, описанный в~[15].

В~[16] дается обзор процессов и~алгоритмов, связанных с~развитием 
онтологий (ontology evolution). Комплекс таких процессов описан как единый 
и~непрерывный цикл, который можно разделить на несколько этапов: 

\begin{enumerate}[(1)]
\item выявление потребности в~изменениях; 
\item формулировка предлагаемых 
изменений; 
\item оценка адекватности предлагаемых изменений; 
\item оценка 
последствий реализации изменений; 
\item внедрение изменений.
\end{enumerate}

 Каждый этап 
рассматривается отдельно с~учетом опыта, полученного в~ходе других 
исследований.

Следует отметить ряд особенностей подхода, предлагаемого в~настоящей 
работе, в~сравнении с~\cite{12-gon, 13-gon, 14-gon, 15-gon, 16-gon}.  
Во-пер\-вых, он нацелен не на описание любых изменений онтологий во 
времени, а~на фиксирование дополнений, изменений порядка и~смыслового 
содержания рубрик самой ДКС. Таким образом, он ближе всего к~работам, 
где описываются подходы к~управлению различными версиями онтологий 
и~развитию онтологий; работы же, посвященные описанию темпоральных 
сущностей в~онтологиях, как правило, имеют иную целевую направленность.

\begin{figure*}[b] %fig1
\vspace*{1pt}
\begin{center}
\mbox{%
\epsfxsize=88mm
\epsfbox{gon-1.eps}
}
\end{center}

\vspace*{3pt}

\noindent
{\small %\begin{center}
{Таблицы для хранения истории изменений рубрик ФК в~НБД.}
%\end{center} %\newline
%
%\vspace*{-6pt}
%
%\noindent
Поля таблицы \textbf{PropHistory}: %\protect\newline
%\begin{itemize}
%\item 
\textbf{PropId} --- уникальный идентификатор рубрики (соответствующей значению НМГ);
\textbf{Code}~--- краткое обозначение рубрики (для значений НМГ обычно имеет следующий вид: 
модальный глагол, дефис, номер значения, например, <<sollen-01>>);
%\item 
\textbf{Name}~--- дефиниция рубрики;
\textbf{isCurrent}~--- признак актуальности данного состояния рубрики (1~--- актуально, 0~--- не 
актуально);
%\item
\textbf{OperationId}~--- идентификатор операции, в~результате которой рубрика приняла вид, 
соответствующий данной строке таблицы;
%\item
\textbf{OperationAttr}~--- атрибут, который присваивается рубрике в~рамках операции с~Id, 
указанным в~поле OperationId (см.\ об атрибутах и~их значениях в~операциях в~табл.~3).
%\end{itemize}
Поля таблицы \textbf{PropOperation}:
%\begin{itemize}
%\item 
\textbf{Id}~--- идентификатор операции (обеспечивает связь между операциями и~рубриками, 
которые они затрагивают);
%\item 
\textbf{Operation}~--- задает тип операции (допустимые значения: CREATE, DELETE, MERGE, 
SPLIT, REORDER, REDISTR, REVISE);
%\item 
\textbf{UserId}~--- идентификатор пользователя, осуществившего операцию;
%\item 
\textbf{TimeDate}~--- дата и~время выполнения операции
%\end{itemize}
}
\end{figure*}

Во-вто\-рых, хотя структура онтологии обычно сложнее, чем структура ДКС, 
это отличие не имеет принципиального значения для задач  
классификации~--- описываемую здесь ФК в~качестве примера ДКС можно 
рассматривать как потенциальную составляющую онтологии знаний по 
лексикографии немецкого языка. Узкая направленность ФК позволяет 
уделить больше внимания непосредственно теме исследования в~рамках 
упомянутого проекта, что немаловажно, поскольку выделение значений НМГ 
является нетривиальной задачей, и~помимо таких операций, как слияние или 
разделение значений, порой требуется перенести часть компонентов 
смыслового содержания из одного значения в~другое, изменить нумерацию 
значений в~соответствии с~последовательностью их описания в~словарных 
статьях, предназначенных для включения в~словарь~\cite{17-gon}, и~т.\,д.

В-третьих, описываемая ФК вложена в~НБД~--- объекты аннотирования 
и~сама ФК физически располагаются в~одной и~той же базе данных, 
благодаря чему разработчикам не нужно беспокоиться о~том, что  
ка\-кие-ли\-бо внешние данные окажутся не\-со\-вмес\-ти\-мы\-ми с~той или иной 
версией ФК. Однако при внесении изменений в~ФК важно в~случае 
необходимости сразу же вносить соответствующие изменения в~аннотации, 
сформированные ранее. Если возможно, это делается автоматически, 
в~противном случае система должна помечать аннотации, затронутые 
изменениями, чтобы их реклассифицировали эксперты~\cite{3-gon}.

\vspace*{-9pt}

\section{Динамическая классификационная система}

\vspace*{-2pt}

В отличие от упомянутых выше решений по поддержке версионности 
онтологий в~предлагаемом подходе и~его реализации в~проекте не 
предусмот\-ре\-на нумерация версий или создание новой версии ФК после 
вносимых изменений. Вместо этого для каждой рубрики ФК в~НБД 
фиксируются все ее изменения и~для них проставляются временн$\acute{\mbox{ы}}$е штампы, 
а~все старые версии сохраняются. Поэтому можно проследить историю 
эволюции каждой рубрики, включая ее взаимодействие с~другими 
руб\-ри\-ка\-ми, а~также восстановить состояние ФК на любой момент времени.

В структуре НБД за сохранение изменений отвечают две связанные между 
собой таблицы: в~первой (PropHistory) хранятся все~--- текущие 
и~устаревшие~--- состояния рубрик ФК (которые в~проекте соответствуют 
значениям НМГ), а~во второй (\mbox{PropOperation})~--- все операции изменения, 
которые применялись к~этим рубрикам. Структура таблиц и~связь между 
ними показаны на рисунке.



Различаются следующие виды операций. 
\begin{enumerate}[1.]
\item CREATE~--- создание новой рубрики.
\item REORDER~--- изменение кода рубрики (соответствует номеру 
значения НМГ по словарю~\cite{17-gon}).
\item REVISE~--- изменение дефиниции рубрики (не затрагивающее 
никакие другие рубрики).
\item MERGE~--- слияние дефиниций двух рубрик, в~результате которого 
одна из рубрик удаляется.
\item DELETE~--- удаление рубрики.
\item SPLIT~--- разделение дефиниции одной рубрики на две, в~результате 
чего создается новая рубрика, а~дефиниция исходной рубрики 
перераспределяется между старой и~новой рубриками.
\item REDISTR (от англ.\ \textit{redistribute})~--- изменение дефиниций 
двух рубрик, предусматривающее перенос части компонентов смыслового 
содержания из одной рубрики в~другую.
\end{enumerate}

\begin{table*}\small %tabl1
\begin{center}
\Caption{Пример выделения компонентов смыслового содержания дефиниции рубрики}
\vspace*{2ex}

\begin{tabular}{|c|c|c|p{52mm}|p{64mm}|}
\hline
Рубрика&Id&Код&\multicolumn{1}{c|}{\tabcolsep=0pt\begin{tabular}{c}
Дефиниция рубрики\\ без структурного выделения\\ компонентов 
смыслового\\ содержания\end{tabular}} &
\multicolumn{1}{c|}{\tabcolsep=0pt\begin{tabular}{c}
Дефиниция рубрики,\\ в~которой выделены компоненты\\ ее 
смыслового содержания\end{tabular}}\\
\hline
X&482&sollen-01&Обязанность что-л. делать по чье\-\mbox{му-л.}\ указанию, по закону, по 
правилам и~т.\,п.: должен. Моральный запрет (под отрицанием): \mbox{нельзя}&$a$.~Обязанность 
что-л.\ делать по чьему-л.\ указанию, по закону, по правилам и~т.\,п.: должен\newline
$b$.~Моральный запрет (под отрицанием): \mbox{нельзя}\\
\hline
\end{tabular}
\end{center}
\end{table*}


В то время как для понимания операций~\mbox{1--6} достаточно приведенных 
определений, операция REDISTR заслуживает более детального 
рассмотрения. Для этого понадобятся условные обозначения, введенные 
в~\cite{3-gon}, а~именно:
\begin{itemize}
\item X, Y,\ \ldots~--- рубрики ФК, обозначающие смыс\-ло\-вые значения 
НМГ;
\item def$_{\mathrm{X}}$, def$_{\mathrm{Y}}$,\ \ldots~---  дефиниции 
рубрик;
\item $\mathbf{S}_{\mathrm{def}_{\mathrm{X}}}$, 
$\mathbf{S}_{\mathrm{def}_{\mathrm{Y}}}$, \ldots ~--- смысловое 
содержание дефиниций рубрик.
\end{itemize}
Кроме того, для обозначения сущностей, которые были каким-то образом 
изменены в~результате выполнения операции, используется индекс <<ch>> 
(от англ.\ \textit{changed}).

Операция REDISTR выполняется только в~том случае, если 
def$_{\mathrm{X}}$ такова, что 
в~$\mathbf{S}_{\mathrm{def}_{\mathrm{X}}}$~--- смысловом содержании 
дефиниции рубрики (соответствующей значению НМГ)~--- выделены 
компоненты (соответствующие подзначениям внутри значения 
НМГ\footnote{С~точки зрения смыслового содержания в~дефиниции рубрики могут быть 
выделены и~более мелкие части, соответствующие частям подзначений. Данная ситуация не 
рассматривается в~рамках настоящей статьи.}). В табл.~1 приводится пример  
руб\-ри\-ки~X, в~дефиниции которой структурно выделены компоненты ее 
смыслового содержания, соответствующие подзначениям,~--- $a$ и~$b$.




Если представить $\mathbf{S}_{\mathrm{def_X}}$, компонентами которого 
являются~$a$ и~$b$, в~виде множества 
$\mathbf{S}_{\mathrm{def_X}}\{a,b\}$, то изменение def$_{\mathrm{X}}$ 
может быть таким, что набор элементов этого множества: (1)~сократится: 
$\mathbf{S}_{\mathrm{def_X}}^1\hm=\{a\}$; (2)~увеличится: 
$\mathbf{S}_{\mathrm{def_X}}^2\hm=\{a, b, c, \ldots\}$.



Более того, возможна ситуация, когда изменение набора элементов затронет 
не только def$_{\mathrm{X}}$, а~одновременно def$_{\mathrm{X}}$ 
и~def$_{\mathrm{Y}}$. Рубрики до внесения изменений обозначим как~X 
и~Y, а~после их внесения~--- как X$_{\mathrm{ch}}$ и~Y$_{\mathrm{ch}}$. 
Если изменение руб\-рик~X и~Y таково, что  $\mathbf{S}_{\mathrm{def_X}} 
\cap \mathbf{S}_{\mathrm{def_Y}}^{\mathrm{ch}}\not= \emptyset$ 
(смысловое содержание def$_{\mathrm{X}}$ и~смысловое содержание 
измененной def$_{\mathrm{Y}}$ имеют один или более общих компонентов) 
и/или $\mathbf{S}_{\mathrm{def_Y}} \cap 
\mathbf{S}_{\mathrm{def_X}}^{\mathrm{ch}}\not= \emptyset$ (смысловое 
содержание def$_{\mathrm{Y}}$ и~смысловое содержание измененной 
def$_{\mathrm{X}}$ имеют один или более общих компонентов), то оно может 
быть описано с~по\-мощью операции \mbox{REDISTR\,(X, Y)}. Экспертная 
реклассификация потребуется для тех аннотаций, которые до выполнения 
операции REDISTR содержали: (1)~руб\-ри\-ку~X, смысловое содержание 
которой в~результате выполнения операции REDISTR сужается;  
(2)~руб\-ри\-ку~X или~Y, если перенос компонентов смыслового содержания 
дефиниций осуществляется как из~X в~Y, так и~из~Y в~X.


Ниже приводится пример выполнения операции REDISTR для рубрик 
с~постоянными номерами (id) 482 и~484, которые позволяют отслеживать 
историю изменений рубрики (табл.~2). В~данном примере: X~---  
руб\-ри\-ка~482 до выполнения операции, причем 
$\mathbf{S}_{\mathrm{def_X}}\hm= \{a, b\}$; Y~--- руб\-ри\-ка~484 до 
выполнения операции, причем $\mathbf{S}_{\mathrm{def_Y}}\hm= 
\{m,n,o\}$; X$_{\mathrm{ch}}$~--- руб\-ри\-ка~482 после выполнения операции, 
причем $\mathbf{S}_{\mathrm{def_X}}^{\mathrm{ch}}\hm= \{a, b, n, o\}$; 
Y$_{\mathrm{ch}}$~--- руб\-ри\-ка~484 после выполнения операции, причем 
$\mathbf{S}_{\mathrm{def_Y}}^{\mathrm{ch}}\hm=\{m\}$.


Поскольку $\mathbf{S}_{\mathrm{def_Y}}\cap 
\mathbf{S}_{\mathrm{def_X}}^{\mathrm{ch}}\hm=\{n, o\}$ 
и~$\mathbf{S}_{\mathrm{def_X}}\cap 
\mathbf{S}_{\mathrm{def_Y}}^{\mathrm{ch}}\hm=\emptyset$, смысловое 
содержание руб\-ри\-ки~482 после внесения изменения расширяется, 
а~смысловое содержание руб\-ри\-ки~484~--- сужается. Следовательно,\linebreak 
аннотации, которые до выполнения операции \mbox{REDISTR} содержали 
 руб\-ри\-ку~482, после ее выполнения не требуют экспертной 
реклассификации и~автоматически обозначаются кодом <<sollen-01>>, тогда 
как аннотации, которые до выполнения операции REDISTR содержали  
руб\-ри\-ку~484, после ее выполнения требуют экспертной реклассификации 
и~поэтому автоматически помечаются тегом <<\mbox{TBR-R}>> (от англ. \textit{To 
Be Reclassified because of Redistribution}).

\begin{table*}\small %tabl2
\begin{center}
\Caption{Исходные данные и~результат выполнения операции REDISTR}
\vspace*{2ex}

\begin{tabular}{|c|c|c|p{120mm}|}
\hline
Рубрика&Id&Код&Дефиниции двух рубрик, в~которых структурно выделены 
компоненты их смыслового содержания, до и~после выполнения операции 
\mbox{REDISTR}\\
\hline
X&482&sollen-01&$a$.~Обязанность что-л.\ делать по чьему-л.\ указанию, по закону, по 
правилам и~т.\,п.: должен\\
&&&$b$.~Моральный запрет (под отрицанием): нельзя\\
\hline
Y&484&sollen-03&$m$.~Желательность по мнению говорящего (в~формах praet conj 
и~pqp conj): следовало (бы), нужно было (бы), должно было (бы)\\
&&&$n$. Совет, рекомендация (только в~формах praet conj)\\
&&&$o$. Нежелательность (под отрицанием): не следовало (бы), нельзя\\
\hline
X$_{\mathrm{ch}}$&482&sollen-01&$a$. Обязанность что-л.\ делать по чьему-л.\ 
указанию, по закону, по правилам и~т.\,п.: должен\\
&&&$b$.~Моральный запрет (под отрицанием): нельзя\\
&&&$n$.~Совет, рекомендация (только в~формах praet conj)\\
&&&$o$.~Нежелательность (под отрицанием): не следовало (бы), нельзя\\
\hline
Y$_{\mathrm{ch}}$&484&sollen-03&$m$.~Желательность по мнению говорящего (в 
формах praet conj и~pqp conj): следовало (бы), нужно было (бы), должно было (бы)\\
\hline
\multicolumn{4}{p{162.7mm}}{\footnotesize\hspace*{3mm}\textbf{Примечания.}
Расшифровка используемых в~таблице сокращений:
\begin{itemize}
\addtolength{\itemsep}{-4pt}
\item praet conj~--- форма прошедшего времени (лат.\ \textit{praeteritum}) сослагательного 
наклонения (лат.\ \textit{conjunctivus});
\item pqp conj~--- форма предпрошедшего времени (лат.\ \textit{plusquamperfectum}) 
сослагательного наклонения (лат.\ \textit{conjunctivus}).
\end{itemize}
Примеры употребления глагола \textit{sollen} по словарю~\cite{17-gon}, иллюстрирующие каждый 
из компонентов смыслового содержания дефиниций (формы глагола \textit{sollen} выделены 
полужирным шрифтом):
\begin{itemize}
\addtolength{\itemsep}{-4pt}
\item[$a$.] ich \textbf{soll} heute noch in die Stadt fahren~--- я должен сегодня еще поехать 
в~город;
\item[$b$.] du \textbf{sollst} nicht t$\ddot{\mbox{o}}$ten!~--- не убий! 
(\textit{библейская заповедь});
\item[$m$.] das \textbf{sollte} sie doch wissen~--- это она (вообще-то) должна была (бы) 
знать;
\item[$n$.] Sie \textbf{sollten} mit dem Rauchen aufh$\ddot{\mbox{o}}$ren~--- вам следует 
бросить курить;
\item[$o$.] das \textbf{sollte} man nie tun~--- этого не следует делать.
\end{itemize}
}
\end{tabular}
\end{center}
\vspace*{-23pt}
%\end{table*}
%\begin{table*}\small %tabl3  %\multicolumn{1}{|c|}{\raisebox{-6pt}[0pt][0pt]{
\begin{center}
\Caption{Соответствия между операциями, атрибутами и~рубриками}
\vspace*{2ex}

\tabcolsep=4.2pt
\begin{tabular}{|l|c|p{105mm}|}
\hline
\multicolumn{1}{|c|}{Операция}&Атрибут&\multicolumn{1}{c|}{Рубрика}\\
\hline
{\raisebox{-6pt}[0pt][0pt]{\tabcolsep=0pt\begin{tabular}{l}CREATE (создается новая\\ 
рубрика X)\end{tabular}}}&A&Создаваемая рубрика X\\
\cline{2-3}
&C&Рубрики, которые надо перенумеровать после выполнения операции, чтобы 
освободить в~нумерации нужную позицию для X\\
\hline
{\raisebox{-6pt}[0pt][0pt]{\tabcolsep=0pt\begin{tabular}{l}REORDER (код рубрики~X\\ изменяется)\end{tabular}}}&A&Рубрика X, код которой изменяется\\
\cline{2-3}
&C&Рубрики, которые надо перенумеровать после выполнения операции, чтобы 
освободить в~нумерации нужную позицию для X\\
\hline
REVISE (изменяется def$_{\mathrm{X}}$)&A&Рубрика X, дефиниция которой изменяется\\
\hline
{\raisebox{-6pt}[0pt][0pt]{\tabcolsep=0pt\begin{tabular}{l}MERGE (объединяются\\ def$_{\mathrm{X}}$ 
и~def$_{\mathrm{Y}}$)\end{tabular}}}&А&Рубрика X, причем 
def$_{\mathrm{X}}$ поглощает def$_{\mathrm{Y}}$, а X остается в~базе данных\\
\cline{2-3}
&B&Рубрика Y, причем def$_{\mathrm{Y}}$ включается в~def$_{\mathrm{X}}$, а~Y 
удаляется из базы данных\\
\cline{2-3}
&C&Рубрики, которые напрямую не затрагиваются операцией MERGE, но которые надо 
перенумеровать после выполнения операции, чтобы заполнить пробелы в~нумерации, 
образовавшиеся из-за удаления Y\\
\hline
{\raisebox{-6pt}[0pt][0pt]{\tabcolsep=0pt\begin{tabular}{l}DELETE (рубрика~X удаля-\\ ется)\end{tabular}}}&A&Удаляемая рубрика X\\
\cline{2-3}
&C&Рубрики, которые надо перенумеровать после выполнения операции, чтобы 
заполнить пробелы в~нумерации, образовавшиеся из-за удаления X\\
\hline
{\raisebox{-6pt}[0pt][0pt]{\tabcolsep=0pt\begin{tabular}{l}SPLIT (def$_{\mathrm{X}}$ делится на две\\
 части~--- def$_{\mathrm{X}}1$ 
и~def$_{\mathrm{X}}2$)\end{tabular}}}&A&Рубрика~X, дефиниция которой делится на две части, причем 
def$_{\mathrm{X}}1$ становится новой дефиницией X\\
\cline{2-3}
&B&Рубрика~Y, которая создается в~базе данных, причем def$_{\mathrm{X}}2$ 
становится дефиницией~Y\\
\hline
{\raisebox{-24pt}[0pt][0pt]{\tabcolsep=0pt\begin{tabular}{l}REDISTR (def$_{\mathrm{X}}$ и~def$_{\mathrm{Y}}$ изме-\\
няются так, что происходит\\ 
перераспределение компо-\\ нентов смыс\-ло\-во\-го содер-\\ жания 
между~$\mathbf{S}_{\mathrm{def_X}}$ 
и~~$\mathbf{S}_{\mathrm{def_Y}}$)\end{tabular}}}&A&Рубрика X, если 
$\mathbf{S}_{\mathrm{def_X}}$ расширяется за счет переноса компонентов 
смыс\-ло\-во\-го содержания из~$\mathbf{S}_{\mathrm{def_Y}}$, причем ни один 
компонент~$\mathbf{S}_{\mathrm{def_X}}$ не переносится 
в~$\mathbf{S}_{\mathrm{def_Y}}$\\
\cline{2-3}
&B&Рубрика Y, если ~$\mathbf{S}_{\mathrm{def_Y}}$ сужается за счет переноса 
компонентов смыс\-ло\-во\-го содержания в~~$\mathbf{S}_{\mathrm{def_X}}$, причем 
ни один компонент~$\mathbf{S}_{\mathrm{def_X}}$ не переносится 
в~~$\mathbf{S}_{\mathrm{def_Y}}$\\
\cline{2-3}
&AB&Рубрики~X и~Y, если одновременно осуществляется перенос компонентов 
смыс\-ло\-во\-го содержания из~$\mathbf{S}_{\mathrm{def_X}}$ 
в~~$\mathbf{S}_{\mathrm{def_Y}}$ и~из~$\mathbf{S}_{\mathrm{def_Y}}$ 
в~~$\mathbf{S}_{\mathrm{def_X}}$\\
\hline
\end{tabular}
\end{center}
\end{table*}

Может показаться, что введение операции \mbox{REDISTR} не оправданно, так как 
рассмотренное перераспределение компонентов значений между~482 и~484 
можно описать последовательностью операций SPLIT\,(MERGE\,(X, Y)). 
Однако при таком подходе объем реклассификации может существенно 
возрасти. Объем реклассификации будет тот же, лишь если верно 
одновременно и~$\mathbf{S}_{\mathrm{def_X}}\hm\cap 
\mathbf{S}_{\mathrm{def_Y}}^{\mathrm{ch}}\hm\not= \emptyset$, 
и~$\mathbf{S}_{\mathrm{def_Y}} \cap 
\mathbf{S}_{\mathrm{def_X}}^{\mathrm{ch}}\not= \emptyset$, т.\,е.\ 
осуществляется перенос компонентов смыслового содержания одновременно 
и~из $\mathbf{S}_{\mathrm{def_X}}$ в~$\mathbf{S}_{\mathrm{def_Y}}$, 
и~из~$\mathbf{S}_{\mathrm{def_Y}}$ в~$\mathbf{S}_{\mathrm{def_X}}$. Если 
же верно или $\mathbf{S}_{\mathrm{def_X}} \cap 
\mathbf{S}_{\mathrm{def_Y}}^{\mathrm{ch}}\not= \emptyset$, 
а~$\mathbf{S}_{\mathrm{def_Y}} \cap 
\mathbf{S}_{\mathrm{def_X}}^{\mathrm{ch}} = \emptyset$, или 
$\mathbf{S}_{\mathrm{def_Y}} \cap 
\mathbf{S}_{\mathrm{def_X}}^{\mathrm{ch}}\not= \emptyset$, 
а~$\mathbf{S}_{\mathrm{def_X}} \cap 
\mathbf{S}_{\mathrm{def_Y}}^{\mathrm{ch}}=\emptyset$ (т.\,е.\ смысловое 
содержание дефиниции одной рубрики расширяется, а~другой~--- сужается), 
то в~таком случае реклассификация нужна только для аннотаций, 
содержавших рубрику с~дефиницией, смысловое
содержание которой 
в~результате изменения сужается (в~примере из табл.~2 это аннотации, 
содержавшие руб\-ри\-ку~Y). Таким образом, введение операции \mbox{REDISTR} 
оправданно, так как позволяет сократить объем реклассификации.

Рассмотренный пример (см.\ табл.~2) показывает, что операция REDISTR  
по-раз\-но\-му влияет на руб\-ри\-ки~482 и~484: смысловое содержание 
дефиниции руб\-ри\-ки~482 расширяется, а~руб\-ри\-ки~484~--- сужается. 
Для того чтобы на основе таб\-лиц НБД с~историей изменения руб\-рик иметь 
возможность определять, как именно операция повлияла на некоторую 
руб\-ри\-ку, каждой из руб\-рик, затронутых операцией, присваивается атрибут. 
В~этом примере руб\-ри\-ке~482 будет присвоен атрибут~A,  
а~руб\-ри\-ке~484~--- B. Все возможные атрибуты и~их смыс\-ло\-вое 
содержание для каждой из семи операций приведены и~расшифрованы 
в~табл.~3.

\vspace*{-3pt}

\section{Заключение}

Предлагаемый подход к~ведению ДКС дает возможность сохранять всю 
информацию об изменениях рубрик ФК, фиксировать виды изменений, время 
их внесения и~данные о~пользователе, который их внес. Он позволяет 
отслеживать совершенные изменения в~хронологическом порядке, а~также, 
при необходимости, восстанавливать состояние ДКС на любой момент 
времени в~прошлом.

\vspace*{-3pt}

{\small\frenchspacing
{%\baselineskip=10.8pt
%\addcontentsline{toc}{section}{References}
\begin{thebibliography}{99}

%\vspace*{-2pt}
\bibitem{1-gon}
\Au{Зацман И.\,М., Косарик~В.\,В., Курчавова~О.\,А.} Задачи представления личностных 
и~коллективных концептов в~цифровой среде~// Информатика и~её применения, 2008. 
Т.~2. Вып.~3. С.~54--69.
\bibitem{2-gon}
Handbook of linguistic annotation~/ Eds. N.~Ide, J.~Pustejovsky.~--- Dordrecht, The 
Netherlands: Springer Science\;+\;Business Media, 2017. 1468~p.
\bibitem{3-gon}
\Au{Гончаров А.\,А., Зацман~И.\,М., Кружков~М.\,Г.} Эволюция классификаций 
в~надкорпусных базах данных~// Информатика и~её применения, 2020. Т.~14. Вып.~4. 
С.~108--116.
\bibitem{4-gon}
\Au{Зацман И.\,М., Инькова~О.\,Ю., Кружков~М.\,Г., Попкова~Н.\,А.} Представление 
кроссязыковых знаний о~коннекторах в~надкорпусных базах данных~// Информатика 
и~её применения, 2016. Т.~10. Вып.~1. С.~106--118.
\bibitem{5-gon}
\Au{Зализняк А., Зацман~И.\,М., Инькова~О.\,Ю.} Надкорпусная база данных коннекторов: 
построение системы терминов~// Информатика и~её применения, 2017. Т.~11. Вып.~1. 
С.~100--108.
\bibitem{6-gon}
\Au{Зацман И.\,М., Кружков~М.\,Г.} Надкорпусная база данных коннекторов: развитие 
системы терминов проектирования~// Системы и~средства информатики, 2018. Т.~28. 
№\,4. С.~156--167.
\bibitem{7-gon}
\Au{Добровольский Д.\,О., Зализняк~Анна~А.} Немецкие конструкции с~модальными 
глаголами и~их русские соответствия: проект надкорпусной базы данных~//\linebreak 
Компьютерная лингвистика и~интеллектуальные технологии: По мат-лам Междунар. 
конф. <<Диалог>>.~--- М.: РГГУ, 2018. С.~172--184.
\bibitem{8-gon}
\Au{Добровольский Д.\,О.} Немецкие модальные глаголы в~параллельном корпусе и~задачи 
двуязычной лексикографии~// Германские языки: текст, корпус, перевод.~--- М.: Институт 
языкознания РАН, 2020. С.~103--116.
\bibitem{9-gon}
\Au{Добровольский Д.\,О., Зализняк~Анна~А.} Русские конструкции с~потенциально 
модальным значением по данным параллельных корпусов~// Труды Института русского 
языка им.\ В.\,В.~Виноградова, 2020. №\,3. С.~35--49.
\bibitem{10-gon}
\Au{Zatsman I.} Finding and filling lacunas in linguistic typologies~// 15th  Forum 
(International) on Knowledge Asset Dynamics Proceedings.~--- Matera: Institute of 
Knowledge Asset Management, 2020. P.~780--793.
\bibitem{11-gon}
\Au{Zatsman I.} Three-dimensional encoding of emerging meanings in AI-systems~// 21st 
European Conference on Knowledge Management Proceedings.~--- Reading: Academic 
Publishing International Ltd., 2020. P.~878--887.
\bibitem{12-gon}
\Au{Klein M., Fensel~D., De~A.} Ontology versioning on the Semantic Web~// 1st  Conference 
(International)  on Semantic Web Working Proceedings.~--- Stanford, CA, USA: Stanford 
University, 2001. P.~75--91.
\bibitem{13-gon}
\Au{Noy N., Kunnatur~S., Klein~M., Musen~M.} Tracking changes during ontology evolution~// 
International Semantic Web Conference~/ Eds. S.\,A.~McIlraith, D.~Plexousakis, F.~van Harmelen.~--- 
Lecture notes in computer science ser.~--- Springer, 2004. Vol.~3298. P.~259--273.
\bibitem{14-gon}
\Au{Preventis A., Petrakis~E.\,G.\,M., Batsakis~S.} CHRONOS Ed: A~tool for handling 
temporal ontologies in prot$\acute{\mbox{e}}$g$\acute{\mbox{e}}$~// Int. J.~Artif. 
Intell.~T., 2014. Vol.~23. No.\,4. P.~1460018-1--1460018-26. doi: 
10.1142/S0218213014600185.
\bibitem{15-gon}
\Au{Stravoskoufos K., Petrakis~E., Mainas~N., Batsakis~S., Samoladas~V.} SOWL QL: 
Querying spatio-temporal ontologies in OWL~// J.~Data Semantics, 2016. Vol.~5. No.\,4. 
P.~249--269.
\bibitem{16-gon}
\Au{Zablith F., Antoniou~G., D'Aquin~M., Flouris~G., Kondylakis~H., Motta~E., 
Plexousakis~D., Sabou~M.} Ontology evolution: A~process-centric survey~// Knowl. 
Eng. Rev., 2015. Vol.~30. No.\,1. P.~45--75.
\bibitem{17-gon}
Немецко-русский словарь актуальной лексики~/ Под ред. Д.\,О.~Добровольского.~--- М.: 
Лексрус, 2021 (в~печати).
\end{thebibliography}

}
}

\end{multicols}

\vspace*{-6pt}

\hfill{\small\textit{Поступила в~редакцию 12.01.2021}}

%\vspace*{8pt}

%\pagebreak

\newpage

\vspace*{-28pt}

%\hrule

%\vspace*{2pt}

%\hrule

%\vspace*{-2pt}

\def\tit{REPRESENTATION OF~NEW LEXICOGRAPHICAL KNOWLEDGE IN~DYNAMIC CLASSIFICATION 
SYSTEMS}

\def\titkol{Representation of new lexicographical knowledge in~dynamic classification 
systems}

\def\aut{A.\,A.~Goncharov, I.\,M.~Zatsman, and~M.\,G.~Kruzhkov}

\def\autkol{A.\,A.~Goncharov, I.\,M.~Zatsman, and~M.\,G.~Kruzhkov}

\titel{\tit}{\aut}{\autkol}{\titkol}

\vspace*{-11pt}


\noindent
Institute of Informatics Problems, Federal Research Center ``Computer Science and
Control'' of the Russian Academy of Sciences, 44-2~Vavilov Str., Moscow 119333,
Russian Federation

\def\leftfootline{\small{\textbf{\thepage}
\hfill INFORMATIKA I EE PRIMENENIYA~--- INFORMATICS AND
APPLICATIONS\ \ \ 2021\ \ \ volume~15\ \ \ issue\ 1}
}%
\def\rightfootline{\small{INFORMATIKA I EE PRIMENENIYA~---
INFORMATICS AND APPLICATIONS\ \ \ 2021\ \ \ volume~15\ \ \ issue\ 1
\hfill \textbf{\thepage}}}

\vspace*{3pt}


\Abste{The distinctive feature of dynamic classification systems is that new categories may be introduced 
in the course of their use or definitions of existing categories may be modified, including cases of 
rearranging semantic content between categories. On one hand, this feature of dynamic classification 
systems provides a possibility to integrate new knowledge on-the-fly and to start using it immediately for 
linguistic annotation. On the other hand, if a category is changed, then, in some cases, the annotations it 
has been previously applied to will have to be reclassified. This paper has a twofold purpose, which is, 
first, to compare approaches to classification of entities based on ($i$)~dynamic classification systems and 
($ii$)~ontologies that change over time; and then, second, to describe how new lexicographical knowledge is 
represented in dynamic classification systems.}

\KWE{dynamic classification system; ontology versioning; linguistic annotation; reclassification of 
annotations}




\DOI{10.14357/19922264210112}

\vspace*{-15pt}

\Ack
\noindent
The study has been conducted at the Institute of Informatics Problems, Federal Research Center 
``Computer Science and Control'' of the Russian Academy of Sciences (FRC CSC RAS) with financial 
support of the Russian Foundation for Basic Research (grant No.\,20-012-00166).
%\vspace*{6pt}

  \begin{multicols}{2}

\renewcommand{\bibname}{\protect\rmfamily References}
%\renewcommand{\bibname}{\large\protect\rm References}

{\small\frenchspacing
 {%\baselineskip=10.8pt
 \addcontentsline{toc}{section}{References}
 \begin{thebibliography}{99}
\bibitem{1-gon-1}
\Aue{Zatsman, I.\,M., V.\,V.~Kosarik, and O.\,A.~Kurchavova.} 2008. Zadachi predstavleniya 
lichnostnykh i~kollektivnykh kontseptov v~tsifrovoy srede [Representation of individual and collective 
concepts in digital medium]. \textit{Informatika i~ee Primeneniya~--- Inform. Appl.} 2(3):54--69.
\bibitem{2-gon-1}
Ide, N., and J.~Pustejovsky, eds. 2017. \textit{Handbook of linguistic annotation}. Dordrecht, The 
Netherlands: Springer Science\;+\;Business Media. 1468~p.
\bibitem{3-gon-1}
\Aue{Goncharov, A.\,A., I.\,M.~Zatsman, and M.\,G.~Kruzhkov.} 2020. Evolyutsiya klassifikatsiy 
v~nadkorpusnykh ba\-zakh dannykh [Evolution of classifications in supracorpora databases]. 
\textit{Informatika i~ee Primeneniya~--- Inform. Appl.} 14(4):108--116.
\bibitem{4-gon-1}
\Aue{Zatsman, I.\,M., O.\,Yu.~Inkova, M.\,G.~Kruzhkov, and N.\,A.~Popkova.}
 2016. Predstavlenie kross-yazykovykh znaniy o~konnektorakh v~nadkorpusnykh 
 bazakh dannykh [Representation of cross-lingual 
knowledge about connectors in suprocorpora databases]. 
\textit{Informatika i~ee Primeneniya~--- Inform. Appl.} 10(1):106--118.
\bibitem{5-gon-1}
\Aue{Zaliznyak, A.\,A., I.\,M.~Zatsman, and O.\,Yu.~In'kova.} 2017. Nadkorpusnaya basa dannykh 
konnektorov: postroenie sistemy terminov [Supracorpora database of connectives: Term system 
development]. \textit{Informatika i~ee Primeneniya~--- Inform. Appl.} 11(1):100--108.
\bibitem{6-gon-1}
\Aue{Zatsman, I.\,M., and M.\,G.~Kruzhkov.} 2018. Nadkorpusnaya baza dannykh konnektorov: razvitie 
sistemy terminov proektirovaniya [Supracorpora database of connectives: Design-oriented evolution of 
the term system]. \textit{Sistemy i~Sredstva Informatiki~--- Systems and Means of Informatics} 
28(4):156--167.
\bibitem{7-gon-1}
\Aue{Dobrovol'skiy, D.\,O., and Anna A.~Zaliznyak.} 2018. Ne\-mets\-kie konstruktsii s~modal'nymi 
glagolami i~ikh russkie sootvetstviya: proekt nadkorpusnoy bazy dannykh [German constructions with 
modal verbs and their Russian correlates: A~supracorpora database project]. \textit{Komp'yuternaya 
lingvistika i~intellektual'nyye tekhnologii: po mat-lam Mezhdunar. konf. ``Dialog'}' [Computational 
Linguistics and Intellectual Technologies. Papers from the Annual Conference (International) 
``Dialogue'']. Moscow: RSHI. 17(24):172--184. 
\bibitem{8-gon-1}
\Aue{Dobrovol'skiy, D.\,O.} 2020. Nemetskie modal'nye glagoly v~parallel'nom korpuse i~zadachi 
dvuyazychnoy leksikografii [German modal verbs in a parallel corpus and bilingual lexicography tasks]. 
\textit{Germanskie yazyki: tekst, korpus, perevod} [German languages: Text, corpus, translation].  Moscow:
Institute of Linguistics RAS. 103--116.
\bibitem{9-gon-1}
\Aue{Dobrovol'skiy, D.\,O., and Anna A.~Zaliznyak.} 2020. Russkie konstruktsii s~potentsial'no 
modal'nym znacheniem po dannym parallel'nykh korpusov [Russian constructions with potentially modal 
meanings: An analysis based on parallel corpus data]. \textit{Trudy Instituta russkogo yazyka im.\ 
V.\,V.~Vinogradova} [V.\,V.~Vinogradov Russian Language Institute Proceedings]. 35--49.
\bibitem{10-gon-1}
\Aue{Zatsman, I.} 2020. Finding and filling lacunas in linguistic typologies. \textit{15th Forum 
(International) on Knowledge Asset Dynamics Proceedings}. Matera: Institute of Knowledge Asset 
Management. 780--793.
\bibitem{11-gon-1}
\Aue{Zatsman, I.} 2020. Three-dimensional encoding of emerging meanings in AI-systems. \textit{21st 
European Conference on Knowledge Management Proceedings}. Reading: Academic Publishing 
International Ltd. 878--887.
\bibitem{12-gon-1}
\Aue{Klein, M., D.~Fensel, and A.~De.} 2001. Ontology versioning on the semantic web. \textit{1st 
Conference (International) on Semantic Web Working Proceedings}. Stanford, CA: Stanford University. 
75--91.
\bibitem{13-gon-1}
\Aue{Noy, N., S.~Kunnatur, M.~Klein, and M.~Musen.} 2004. Tracking changes during ontology 
evolution. \textit{International Semantic Web Conference}. Eds. S.\,A.~McIlraith, D.~Plexousakis, and 
F.~van Harmelen. Lecture notes in computer science ser. Springer. 3298:259--273.
\bibitem{14-gon-1}
\Aue{Preventis, A., E.\,G.\,M.~Petrakis, and S.~Batsakis.} 2014. CHRONOS Ed: A~tool for handling 
temporal ontologies in prot$\acute{\mbox{e}}$g$\acute{\mbox{e}}$. \textit{Int. J.~Artif. 
Intell.~T.} 23(4):1460018. 26~p.
doi: 
10.1142/S0218213014600185.
\bibitem{15-gon-1}
\Aue{Stravoskoufos, K., E.~Petrakis, N.~Mainas, S.~Batsakis, and V.~Samoladas.} 2016. SOWL QL: 
Querying spatio-temporal ontologies in OWL. \textit{J.~Data Semantics} 5(4):249--269.
\bibitem{16-gon-1}
\Aue{Zablith F., G.~Antoniou, M.~D'Aquin, G.~Flouris, H.~Kondylakis, E.~Motta, D.~Plexousakis, and 
M.~Sabou.} 2015. Ontology evolution: A~process-centric survey. \textit{Knowl. Eng. 
Rev.} 30(1):45--75.
\bibitem{17-gon-1}
Dobrovol'skiy, D.O., ed. 2021 (in press). \textit{Nemetsko-russkiy slovar' aktual'noy leksiki} 
[German--Russian dictionary of actual vocabulary]. Moscow: Leksrus.
\end{thebibliography}

 }
 }

\end{multicols}

\vspace*{-3pt}

  \hfill{\small\textit{Received January~12, 2021}}


%\pagebreak

%\vspace*{-8pt}

\Contr

\noindent
\textbf{Goncharov Alexander A.} (b.\ 1994)~--- junior scientist, Institute of Informatics Problems, 
Federal Research Center ``Computer Science and Control'' of the Russian Academy of Sciences,  
44-2~Vavilov Str., Moscow 119333, Russian Federation; \mbox{a.gonch48@gmail.com}

\vspace*{3pt}

\noindent
\textbf{Zatsman Igor M.} (b.\ 1952)~--- Doctor of Science in technology, Head of Department, Institute 
of Informatics Problems, Federal Research Center ``Computer Science and Control'' of the Russian 
Academy of Sciences, 44-2~Vavilov Str., Moscow 119333, Russian Federation; 
\mbox{izatsman@yandex.ru}

\vspace*{3pt}

\noindent
\textbf{Kruzhkov Mikhail G.} (b.\ 1975)~--- senior scientist, Institute of Informatics Problems, Federal 
Research Center ``Computer Science and Control'' of the Russian Academy of Sciences, 44-2~Vavilov 
Str., Moscow 119333, Russian Federation; \mbox{magnit75@yandex.ru}

\label{end\stat}

\renewcommand{\bibname}{\protect\rm Литература}   %12
\def\stat{zatsman}

\def\tit{ПРОЦЕССЫ ЦЕЛЕНАПРАВЛЕННОЙ ГЕНЕРАЦИИ И РАЗВИТИЯ КРОСС-ЯЗЫКОВЫХ 
ЭКСПЕРТНЫХ ЗНАНИЙ: СЕМИОТИЧЕСКИЕ~ОСНОВАНИЯ~МОДЕЛИРОВАНИЯ$^*$}

\def\titkol{Процессы целенаправленной генерации и развития кросс-языковых 
экспертных знаний: семиотические основания} % моделирования}

\def\aut{И.\,М.~Зацман$^1$}

\def\autkol{И.\,М.~Зацман}

\titel{\tit}{\aut}{\autkol}{\titkol}

{\renewcommand{\thefootnote}{\fnsymbol{footnote}} \footnotetext[1]
{Работа выполнена при поддержке РФФИ 
(проекты 14-07-00785, 13-06-00403) и РГНФ (проект 15-04-00507).}


\renewcommand{\thefootnote}{\arabic{footnote}}
\footnotetext[1]{Институт проблем информатики Федерального исследовательского
центра <<Информатика и~управление>> Российской академии наук,
iz\_ipi@a170.ipi.ac.ru}

 
   \Abst{Представлены результаты разработки семиотических оснований для создания 
моделей процессов целенаправленной генерации и~развития новых экспертных знаний 
и~разработки технологий, обеспечивающих эти процессы. Необходимость разработки таких 
технологий проявляется наиболее наглядно в~ситуациях, когда имеющиеся системы 
экспертных знаний не удовлетворяют новым социально или технологически значимым 
целям, отражающим новые или изменившиеся потребности общества. В~статье речь идет 
не о~хорошо известных в~области искусственного интеллекта методах и~моделях 
представления знаний, процессах управления формами их представления, а~о~разработке 
новых моделей процессов целенаправленной генерации знаний, отражающих динамику их 
формирования. Рассматриваемый подход к~моделированию этих процессов и~разработке 
обеспечивающих их технологий ориентирован на те прикладные области, где знания 
генерируются экспертами в~процессе анализа текстов или других объектов интерпретации, 
которые могут изменяться во времени, с~последующим пред\-став\-ле\-ни\-ем экспертных знаний 
в~надкорпусных базах данных (НБД). Отличительная черта предлагаемого подхода 
к~моделированию заключается в~явном описании отношений между новыми экспертными 
знаниями и~теми объектами интерпретации, на основе анализа которых были 
сгенерированы элементы новых знаний. Другая отличительная черта заключается в~явном 
описании изменяемых во времени элементов знаний, соответствующих объектам 
интерпретации. Реализуемость такого подхода демонстрируется на примере 
экспериментальной информационной технологии, которая поддерживает 
целенаправленную генерацию экспертами кросс-язы\-ко\-вых знаний о~переводах глагольных 
конструкций русского языка на французский. Эти кросс-язы\-ко\-вые знания формируются 
в~процессе анализа параллельных текстов на русском и~французском языках, пары 
выровненных предложений которых являются объектами интерпретации.}
   
   \KW{кросс-языковые экспертные знания; компьютерное моделирование; генерация 
знаний; объекты интерпретации; семиотические основания; модели процессов генерации 
знаний; надкорпусные базы данных}

\DOI{10.14357/19922264150311 }


\vspace*{-6pt}

\vskip 12pt plus 9pt minus 6pt

\thispagestyle{headings}

\begin{multicols}{2}

\label{st\stat}

\section{Введение}

      Модели процессов генерации и развития новых знаний (далее~--- модели генерации) 
стали активно разрабатываться в~последнем десятилетии прошлого века. Наиболее 
известную модель генерации, названную автором спиральной, предложил Икуджиро 
Нонака~[1, 2]. В~процессе ее построения Нонака рассматривал личностные знания человека 
и~коллективные (согласованные) знания группы людей, которые были разделены на 
выражаемые (explicit) и~невыражаемые знания (tacit). Таким образом, спиральная 
модель генерации включает в~рассмотрение следующие четыре понятия и~соответствующие 
им четыре множества знаний (рис.~1):
      \begin{enumerate}[(1)]
\item личностные невыражаемые знания (individual tacit knowledge);\\[-15pt]
\item коллективные невыражаемые знания (group tacit knowledge);\\[-15pt]
\item личностные выражаемые знания (individual explicit knowledge);\\[-15pt]
\item коллективные выражаемые знания (group explicit knowledge).
\end{enumerate}

\begin{figure*} %fig1
       \vspace*{1pt}
 \begin{center}
 \mbox{%
 \epsfxsize=87.663mm 
 \epsfbox{zac-1.eps}
 }
 \end{center}
 \vspace*{-12pt}
\Caption{Спиральная модель генерации знаний Икуджиро Нонака~\cite[с.~69]{3-zat}}
\vspace*{-4pt}
\end{figure*}


      Наряду с~этими четырьмя понятиями были определены следующие четыре вида 
процессов:
      \begin{enumerate}[(1)]
\item социализация личностных невыражаемых знаний;\\[-15pt]
\item экстернализация коллективных невыража\-емых знаний;\\[-15pt]
\item синтез личностных выража\-емых знаний;
\item интернализация личностных выражаемых знаний.
\end{enumerate}



      Используя эти четыре процесса, Нонака ввел метафорическое понятие спирали 
генерации знаний, каждый виток которой включает следующую последовательность:  
со\-циа\-ли\-за\-ция\;$\to$\;экстер\-на\-ли-\linebreak за\-ция\;$\to$\;син\-тез\;$\to$\;ин\-тер\-на\-ли\-за\-ция\;$\to$\;со\-циа\-ли\-за\-ция 
(как начало следующего витка спирали). 
Было показано на примерах, что эта спираль может служить качественной моделью 
итерационного процесса генерации новых знаний во время <<мозгового штурма>>.
      
      Обобщение и существенное развитие спиральной модели генерации знаний было 
предложено в~работах Йошитеру Накамори и~Анджея Вежбицкого в~рамках создаваемой 
ими научной дисциплины, которую они называют <<Наука о знаниях>>~[3--8]. В~этих 
работах знания разделены на личностные знания человека, коллективные 
и~конвенциональные знания. Это деление они называют социальным аспектом, или 
измерением, так как в~результате обобщения было определено три уровня социализации 
знаний (от первого личностного уровня до третьего конвенционального). С~учетом деления 
на выражаемые и~невыражаемые знания в~результате обобщения были определены еще два 
новых множества знаний, которых нет в~спиральной модели: конвенциональные 
невыражаемые и~конвенциональные выражаемые знания.
      
      Вежбицкий и Накамори определили систему\linebreak
       отношений между множествами 
знаний. Свою модель, включающую шесть множеств знаний и~сис\-те\-му отношений между 
ними, они в~совокупности назвали \textit{креативным пространством} (далее в~\mbox{статье} 
термины <<креативное пространство>> и~<<модель Веж\-биц\-ко\-го--На\-ка\-мо\-ри>> 
будут использоваться как синонимы). Кроме шести множеств знаний ими были также 
определены множества эмоций (личностные, коллективные и конвенциональные), которые 
в статье не рассматриваются. 

В~предлагаемых далее в~статье моделях не используются три 
множества невыражаемых знаний человека, которые по определению непосредст\-венно не 
поддаются экспликации. Однако в~процессе компьютерного моделирования эти три 
множества могут использоваться опосредованно.\linebreak Например, в~технологии генерации новых 
кросс-язы\-ко\-вых знаний, рас\-смат\-ри\-ва\-емой далее в~статье, анализируются объекты 
интерпретации, которые сформированы в~процессе перевода текстов на русском языке 
и~являются результатом применения переводчиком одновременно как конвенциональных 
знаний, так и его личностных невыражаемых знаний\footnote{В статье по смысловому 
содержанию разделяются понятия невыражаемых (tacit) и подразумеваемых (implicit) знаний. Невыражаемые 
и незакрепленные в~знаковой форме знания используются переводчиком часто неявно и могут быть известны 
только ему, т.\,е.\ такие знания могут являться личностными. Подразумеваемые знания косвенно выражаются 
в знаковой форме. Например, фраза <<фирма закрыла отдел разработки прикладных программ>> 
подразумевает, что раньше в~этой фирме существовал отдел прикладного программирования. Такие знания 
могут быть коллективными или конвенциональными.}. Этот анализ является примером извле\-чения и 
опосредованного использования послед-\linebreak них.

\begin{figure*} %fig2
       \vspace*{1pt}
 \begin{center}
 \mbox{%
 \epsfxsize=160.234mm
 \epsfbox{zac-2.eps}
 }
 \end{center}
 \vspace*{-9pt}
\Caption{Система терминов для описания объектов трех сред предметной области информатики и 
интерфейсов между ними~\cite{10-zat, 12-zat} (\textit{денотат по определению является компонентом, 
отношением или свойством объекта интерпретации}$^1$)}
\end{figure*}


      
      
      Отметим, что в~модели Веж\-биц\-ко\-го--На\-ка\-мо\-ри и спиральной модели, 
которые относятся к~категории качественных, нет явно определенной оси времени. Это не 
дает возможности фиксировать моменты времени генерации каждого нового структурного 
элемента экспертных знаний. Кроме того, в~этих моделях не рассматриваются объекты 
интерпретации, служащие источниками новых знаний.

\pagebreak

\renewcommand{\thefootnote}{\arabic{footnote}}
\footnotetext[1]{Отличие денотата от объекта интерпретации будет описано далее 
в~примере генерации  кросс-язы\-ко\-вых знаний.}


      Основная цель статьи заключается в~описании разработанных семиотических 
оснований для создания количественных моделей процессов целенаправленной генерации 
и~развития экспертных знаний, а также для разработки обеспечивающих их технологий и~баз 
данных. С~использованием этих оснований в~статье предлагается развитие модели 
Веж\-биц\-ко\-го--На\-ка\-мо\-ри в~следующих двух на\-прав\-ле\-ниях:
      \begin{enumerate}[(1)]
\item вводится ось времени, на которой фиксируют\-ся дискретные моменты времени, 
в~которые по\-рож\-да\-ют\-ся новые элементы вы\-ра\-жа\-емых экспертных знаний, а~также 
фиксируются моменты времени их изменения;
\item в~явном виде специфицируется связь каждого нового элемента знаний с~тем 
объектом интерпретации, в~результате анализа которого формируется этот элемент 
и~дается формализованное его описание.
\end{enumerate}




\section{Семиотические основания}

      В процессе развития модели Веж\-биц\-ко\-го--На\-ка\-мо\-ри использовалась ранее 
разработанная система терминов~\cite{9-zat, 10-zat}. Она включает, в~част\-ности,\linebreak
термины и~дефиниции для таких понятий, как\linebreak
 кон\-цеп\-ты знаний человека (личностные, коллективные, 
конвенциональные), семантическая информация, сен\-сор\-но вос\-при\-ни\-ма\-емые данные, 
циф\-ро\-вая информация, циф\-ро\-вые данные, циф\-ро\-вые коды нескольких категорий и~ряд 
других терминов, а~также задает их распределение по трем\linebreak средам предметной области 
информатики (ментальной, со\-ци\-аль\-но-ком\-му\-ни\-ка\-ци\-он\-ной и цифровой) 
и~описание отношений между этими терминами. Первая отличительная черта этой системы\linebreak 
терминов заключается в~том, что значения терминов <<знания>>, <<семантическая 
информация>>, <<сен\-сор\-но вос\-при\-ни\-ма\-емые данные>>, <<цифровая инфор\-мация>> 
и~<<цифровые данные>> четко разграничены и они по определению не пересекаются по их 
смысловому содержанию (рис.~2). Вторая ее отличительная черта состоит в~том, что она 
является <<масштабируемой>> по числу сред, так как определен принцип добавления 
новых сред в~предметную область информатики как ин\-фор\-ма\-ци\-он\-но-ком\-пьютерной науки, 
а~также соответствующего расширения системы терминов за счет именования объектов 
каждой новой среды и их интерфейсов с~объектами уже существующих сред. Этот принцип 
получил название аксиомы герметичности сред предметной области 
информатики~\cite{12-zat, 11-zat}.
      
      Основная идея развития модели Веж\-биц\-ко\-го--На\-ка\-мо\-ри заключается 
      в~установлении связи каж\-до\-го нового концепта с~объектом интерпретации. При этом анализ 
объектов интерпретации выполняется в~соответствии с~явно определенными целями 
формирования новых знаний, так как статья посвящена именно целенаправленным 
процессам формирования новых выражаемых знаний на основе извлечения и экспликации 
невыражаемых.





      Развитие модели Веж\-биц\-ко\-го--На\-ка\-мо\-ри было выполнено в~несколько 
этапов. Сначала на первом этапе в~модель были добавлены следующие шесть множеств:
      \begin{enumerate}[(1)]
\item формы представления личностных концептов (структурированные тексты, 
изображения или другие виды семантической информации);
\item формы представления коллективных концептов;
\item формы представления конвенциональных концептов;
\item цифровые коды личностных концептов и~форм их представления;
\item цифровые коды коллективных концептов и~форм их представления;
\item цифровые коды конвенциональных концептов и~форм их представления.
\end{enumerate}

      В последних трех множествах разделяются коды концептов и~их имен (как частного случая
      форм представления концептов), например 
коды для пред\-став\-ле\-ния смыслового содержания слов и~последовательностей их литер 
в~цифровой среде относятся к~разным категориям (см.\ рис.~2).
      
      Затем для количественного описания динамики социализации, определяемой 
процессами согласования личностных концептов в~группе экспертов, была введена ось 
с~числами от нуля до бесконечности. Единица на оси социализации обозначает\linebreak
 личностный 
концепт, $N\hm>1$~--- коллективный,\linebreak который согласован группой из~$N$~экспертов, 
а~бесконечность~--- конвенциональный концепт. Необходимость в~нуле на этой оси 
возникла в~процессе проведения эксперимента для обозначения тех <<бывших>> 
личностных концептов, от которых со временем отказались их авторы. Ось социализации 
позволяет кодировать степень согласованности между экспертами результатов анализа 
динамически изменяемых объектов интерпретации и изменение степени согласованности во 
времени.
      
      Таким образом, на первом этапе развития модели Веж\-биц\-ко\-го--На\-ка\-мо\-ри 
были определены шесть новых множеств знаний и введена ось социализации. Эта ось 
служит для обозначения уровня социализации не только выражаемых знаний, но также 
форм их представления и их цифровых кодов. Отметим, что система отношений между 
множествами знаний, определенная Вежбицким и Накамори, не охватывает шесть новых 
множеств знаний. Поэтому далее потребуется доопределить или построить новую систему 
отношений.
      
      На втором этапе была добавлена ось времени, на которой фиксируются моменты 
порождения новых личностных, коллективных и конвенциональных концептов, а~также 
моменты времени их изменений. Следствием этого этапа является то, что появляется 
возможность фиксировать на оси времени не только моменты порождения и изменения 
концептов, но также форм их представления и~их цифровых кодов.
      
      Третий этап заключается в~добавлении множества объектов интерпретации, каждый 
из которых имеет уникальный цифровой код. Следствием этого этапа является то, что 
появляется потенциальная возможность фиксировать на оси времени не только моменты 
порождения и изменения концептов, форм их представления и их цифровых кодов, но 
также динамику изменения соответствующих им объектов интерпретации.
      
      Результаты трех перечисленных этапов развития модели 
Веж\-биц\-ко\-го--На\-ка\-мо\-ри позволяют фиксировать на оси времени:
      \begin{itemize}
\item моменты изменения экспертами объектов интерпретации, являющихся 
источниками новых знаний;\\[-14pt]
\item моменты порождения новых концептов в~процессе интерпретации объектов 
и интроспекции результатов интерпретации;\\[-14pt]
\item моменты изменения формируемых концептов, форм их представления (их 
имен) и их цифровых кодов.
\end{itemize}

      Определим новую систему отношений между объектами интерпретации, концептами 
как эле\-мен\-тами множеств знаний, их именами и~цифро\-вы\-ми кодами, являющимися их 
идентифика\-торами. Она необходима потому, что система отношений в~модели  
Веж\-биц\-ко\-го--На\-ка\-мо\-ри не охватывает шесть новых множеств знаний. Для 
описания отношений между объектами интерпретации, концептами и~именами предлагается 
использовать треугольник Фреге~[13--15], что представляет собой 
\textit{первое семиотическое основание} моделирования процессов генерации знаний. 
Семиотический треугольник Фреге по определению связывает сам объект интерпретации 
(точнее, некоторый денотат, определяемый в~процессе анализа объекта интерпретации; как 
правило, это его компонент, отношение или свойство), понимание денотата (его смысловое 
содержание), т.\,е.\ его концепт, а также некоторое имя как текстовую, или невербальную, 
форму обозначения денотата и~его концепта. Для циф\-ро\-вого кодирования денотатов, 
концептов и~имен предлагается использовать циф\-ро\-вой семиотический треугольник, что 
представляет собой \textit{второе семиотическое основание} моделирования процессов 
генерации знаний. Его определение дано в~работе~\cite{16-zat}, в~которой рис.~6 
иллюстрирует взаимосвязи этих двух треугольников.
   
   Основная идея цифрового семиотического треугольника заключается в~том, что для 
каждой из трех вершин треугольника Фреге используется своя категория цифровых кодов, 
в~том числе и~для концептов. Предлагаемое введение отдельной кодировки для концептов 
дает возможность строить взаимно однозначные отношения между концептами 
и~циф\-ро\-вы\-ми кодами, являющимися их идентификаторами. Построение таких отношений 
является основой компьютерного моделирования процессов генерации знаний. Важно 
отметить, что циф\-ро\-вой семиотический треугольник дает возможность строить взаимно 
однозначные отношения, но остав\-ля\-ет открытым вопрос о~конкретных методах 
приписывания цифровых кодов концептам. В~ряде задач компьютерного моделирования 
процессов генерации знаний существует свобода выбора метода назначения кодов. Однако 
в~задачах оценивания релевантности новых знаний явно определенным целям их генерации 
этот метод может быть во многом обусловлен заданными целями генерации. Примеры 
таких целей рассматриваются далее в~статье.
      
      Перечислим кратко основные результаты, которые были получены в~процессе 
развития модели Веж\-биц\-ко\-го--На\-ка\-мо\-ри. С~помощью цифр~1 и~2 в~списке 
отмечены положения, взятые из модели Веж\-биц\-ко\-го--На\-ка\-мо\-ри~(1) и~полученные 
в результате развития этой модели~(2):
      \begin{itemize}
\item три множества выражаемых знаний (личностные, коллективные и~
конвенциональные)~(1), которые состоят из концептов соответству\-ющих 
категорий~(2);
\item три множества имен (форм представления концептов)~(2);
\item множество объектов интерпретации~(2);
\item три множества цифровых кодов концептов~(2) (личностные, коллективные 
и конвенциональные);
\item три множества цифровых кодов форм пред\-став\-ле\-ния концептов~(2);
\item множество цифровых кодов, построенное на основе уникальных 
идентификаторов объектов интерпретации~(2);
\item ось времени для отражения динамики процессов генерации знаний~(2);
\item ось социализации выражаемых экспертных знаний (1 и~2, так как в~модели 
Веж\-биц\-ко\-го--На\-ка\-мо\-ри нет детализации коллективных знаний в~
зависимости от числа экспертов в~группе, а в~результате развития этой модели 
добавлена их детализация);
\item система отношений между объектами интерпретации, денотатами, 
концептами вы\-ра\-жа\-емых знаний, именами и~их цифровыми кодами (1 и~2, так 
как система отношений задана в~модели Веж\-биц\-ко\-го--На\-ка\-мо\-ри только 
между знаниями с~учетом трех уровней их социализации; в~результате развития 
модели система отношений дополнена треугольником Фреге и~цифровым 
семиотическим треугольником).
\end{itemize}

      Для описания еще одного, третьего, семиотического основания необходимо 
вернуться к~определению классического треугольника Фреге. В~семиотике он определяется 
как треугольник с~тремя вершинами (денотат, концепт как идеальная вершина 
треугольника, имя как форма представления концепта), находящимися в~отношениях 
устойчивой связи, опосредованной сознанием, представляет собой устойчивое единство, 
которое посредством сенсорно воспринимаемой формы \textit{конвенционально} 
репрезентирует концепт и~денотат. Строго говоря, в~новой системе отношений 
классический треугольник Фреге применим только для случая конвенциональной 
репрезентации концепта сенсорно воспринимаемой формой. Следовательно, дополнительно 
необходим некоторый способ для построения личностного и~коллективного семиотических 
треугольников Фреге, аналогичных классическому треугольнику Фреге и~применимых в~
случае генерации личностных и~коллективных концептов. При этом необходимо учитывать 
то обстоятельство, что личностные и~коллективные концепты понимают только их авторы, 
так как для них отсутствует конвенциональная репрезентация их формой. Следовательно, 
для экспертов, участвующих в~процессе формирования новых знаний, но не являющихся 
авторами генерируемых новых концептов, они будут недоступны, если их авторы не 
эксплицируют свою личностную или коллективную репрезентацию в~форме, доступной 
другим экспертам тем или иным способом.
      
      Отметим, что в~процессе личностной репрезентации кроме объекта интерпретации, 
денотата, концепта и~имени <<задействовано>> персональное авторское сознание. 
В~приведенном определении классического треугольника Фреге говорится об устойчивой 
связи, \textit{опосредованной сознанием}, но ничего не говорится о том, как и~где (в какой 
среде или средах) эта связь закреплена материально. Остановимся кратко на истории этого 
вопроса, чтобы затем предложить способ экспликации личностной репрезентации концепта 
в форме, доступной другим экспертам.

\begin{figure*} %fig3
       \vspace*{1pt}
 \begin{center}
 \mbox{%
 \epsfxsize=138.786mm
 \epsfbox{zac-3.eps}
 }
 \end{center}
 \vspace*{-9pt}
\Caption{Семиотический нейротетраэдр и~нейроквадрат~\cite{12-zat}}
\end{figure*}
      
      В 1988~г.\ была сформирована рабочая группа <<FRamework of Information System 
COncepts~--- FRISCO>> в~рамках Международной федерации по обработке информации 
(International Federation for Information Processing~--- IFIP). Основной целью этой группы 
было создание системы определений для базовых терминов, которую затем можно было бы 
предложить использовать как терминологическую основу разработки и~описания 
информационных систем. Итоги ее работы опубликованы в~виде отчета 
      в~1998~г.~\cite{17-zat}. В~результате работы группы FRISCO, в~частности, было 
определено понятие семиотического тетраэдра с~вершинами <<объект, концепт, имя 
объекта и~интерпретатор>>. Отметим, что в~определении этого понятия есть субъект, 
который \textit{интерпретирует} объект, генерирует концепт и~имя объекта, а также 
\textit{устанавливает} связь между ними~\cite{18-zat}. Идея 
      субъ\-ек\-та-ин\-тер\-пре\-та\-то\-ра или интерпретанта\footnote{Интерпретатор~--- это 
только человек, который анализирует предмет, генерирует концепт и~его имя, а интерпретант~--- это не 
обязательно человек. В~общей семиотике функция абстрактного интерпретанта заключается в~интерпретации 
предметов, генерации концептов и~присвоении им имен~\cite[с.~15, 16]{19-zat}.}, который \textit{является 
носителем} этой связи, зародилась еще раньше в~общей семиотике и~в~рамках этой 
научной дисциплины оказалась весьма продуктивной~\cite{19-zat, 20-zat}.
      
      Однако с~точки зрения разработки и~описания информационных систем и~технологий 
включение субъектов в~дефиниции терминов вместо объектов обладает рядом недостатков. 
Основной из них заключа\-ется в~том, что в~задачах когнитивной информатики 
и~нейроинформатики, например при разработке нейрокоммуникаторов, необходимо 
различать объекты \textit{ментальной}, \textit{нейрофизиологической}, 
со\-ци\-аль\-но-ком\-му\-ни\-ка\-ци\-он\-ной и~цифровой сред. В~традиционных моделях 
с~субъек\-та\-ми-ин\-тер\-пре\-та\-то\-ра\-ми и~интерпретантами это различие между объектами трудно 
провести, так как в~них ментальная среда и~нейрофизиологическая среда (далее~--- 
нейросреда), а~также их объекты, как правило, не различаются. Поэтому 
      в~работе~\cite{12-zat} эти две среды было предложено рассматривать в~информатике 
раздельно, а систему терминов дополнить следующими понятиями (рис.~3):
      \begin{itemize}
\item <<нейроинформация>>, в~частности фик\-си\-ру\-ющая постоянные или 
временные связи между объектом интерпретации, концептом (личностным, 
коллективным или конвенциональным) и~именем объекта;
\item <<нейросемиотический тетраэдр>> с~вершинами объект, концепт, имя 
и~нейроинформация;
\item <<нейроквадрат>> как четверка кодов следующих категорий:
\begin{description}
\item[\,] К1~--- для концептов ментальных знаний человека;
\item[\,] К2~--- для слов как имен объектов и~других знаковых форм для 
обозначения ментальных знаний;
\item[\,] К3~--- для кодирования предметов матери\-альной сферы (в~общем 
случае~--- для кодирования любых объектов, в~результате семантиче\-ской 
интерпретации которых человеком определяются денотаты и~генерируются 
концепты);
\item[\,] К4~--- для кодирования в~цифровой среде нейроинформации,  
с~по\-мощью которой фиксируются связи между объектом интерпретации, 
концептом и~именем;
\end{description}
\item <<нейрокод>> как аналог компьютерных таблиц кодировки символов для 
нейроинформации.
\end{itemize}

      Определение нейросемиотического тетраэдра по своему содержанию во многом 
совпадает с~терми\-ном <<психосемиотический тетраэдр>> Ф.\,Е.~Ва\-силюка с~вершинами 
<<предмет, личностный\linebreak концепт, имя предмета и~чувственная ткань (которая
свя\-зы\-ва\-ет 
воедино первые три вершины)>>~\cite{21-zat}. Различие этих двух тетраэдров заключается 
в том, что первый описывает любые кон\-цеп\-ты (личностный, коллективный 
и~конвенциональный), а~второй был определен Ф.\,Е.~Василюком только для личностных 
концептов. В~нейросемиотическом тет\-ра\-эд\-ре три из четырех его вершин являются 
сущностями трех разных сред: ментальной, со\-ци\-аль\-но-ком\-му\-ни\-ка\-ци\-он\-ной и~нейросреды, к~
которой принадлежит нейроинформация. Таким образом, в~определении 
нейросемиотического тетраэдра использовался альтернативный подход по сравнению 
с~традиционной идеей субъ\-ек\-та-ин\-тер\-пре\-та\-то\-ра или интерпретанта.
      
      Сопоставление рис.~3 и~рис.~6 из работы~\cite{16-zat}, в~которой даны определения 
понятий <<формокод>> и~<<семокод>>, наглядно иллюстрирует то, что нейроквадрат 
является обобщением цифрового се-\linebreak миотического треугольника. Итак, нейротетраэдр 
и~нейро\-квадрат являются \textit{третьим семиотическим основанием} моделирования 
процессов генерации знаний.



\section{Семиотические модели}

      В этом разделе три семиотических основания, рассмотренные в~предыдущем 
разделе, будут использованы в~процессе построения двух моделей генерации знаний: 
модели фиксированного состояния и~нестационарной модели. Они являются результатом 
обобщения двух ранее разработанных моделей, основанных на треугольнике Фреге 
и~циф\-ро\-вом семиотическом треугольнике~\cite{16-zat, 22-zat}.
      
      Кроме трех семиотических оснований исходными данными для обобщения является 
следующее описание этих двух моделей, разработанных для итерационных процессов 
генерации знаний экспертами, которые используют информационную систему для 
фиксации эволюции объектов интерпретации, определения денотатов, сгенерированных 
ими концептов, описания которых эксперты формируют в~процессе интроспекции, имен 
концептов и~денотатов. По определению из работы~\cite{16-zat} первая модель фиксирует 
\textit{состояние процесса генерации в~момент времени}, соответствующий окончанию 
некоторой итерации этого процесса, и~состоит из:
      \begin{itemize}
\item трех сред предметной области информатики: ментальной,  
со\-ци\-аль\-но-ком\-му\-ни\-ка\-ци\-он\-ной и~цифровой;
\item треугольника Фреге, включающего объект интерпретации, концепт, 
сгенерированный или измененный на этой итерации некоторым экспертом, и~имя;
\item цифрового семиотического треугольника, включающего коды объекта 
интерпретации, концепта и~имени, сгенерированные в~этот же момент времени 
информационной системой, обеспечивающей работу экспертов.
\end{itemize}

      Согласно первой модели три вершины треугольника Фреге кодируются тремя 
цифровыми кодами разных категорий, которые генерируются в~конце каждой итерации:
      \begin{itemize}
\item семантическим кодом концепта (К1);
\item информационным кодом его имени, если оно создано экспертом 
(в~противном случае этот код равен нулю) (К2);
\item объектным кодом объекта интерпретации (К3).
\end{itemize}

      Таким образом, после завершения каждой итерации в~информационной системе по 
некоторому заданному алгоритму генерируются три цифровых\linebreak кода разных категорий. Если 
на некоторой итерации принял участие один эксперт, то создается\linebreak только одна запись 
с~результатами личностной семантической интерпретации рассматриваемого объекта и~три 
кода, а если несколько экспертов, то для каж\-до\-го из них~--- одна запись и~три кода.
      
      В процессе целенаправленной генерации знаний существует отдельный вид 
итераций для согласования личностных концептов и~имен, которые могут меняться 
в~пределах итераций, но не между ними. На этих итерациях эксперты ставят своей целью 
согласовать между собой свои личностные интерпретации и~сформировать коллективные 
концепты и~имена. Если это удается сделать, то в~информационной системе создается еще 
одна запись с~результатами коллективной семантической интерпретации и~еще три кода 
с~указанием идентификаторов всех экспертов, которые приняли учас\-тие в~процессе 
согласования и~выработали единую позицию. В~общем случае генерация и~согласование 
концептов могут быть совмещены в~рамках одной комплексной итерации. Отметим, что 
между двумя любыми итерациями объекты интерпретации могут меняться экспертами, но 
не в~пределах итераций. Регистрация в~информационной системе изменений объектов 
интерпретации между итерациями, описаний новых концептов и~результатов их 
согласования дает возможность восстановить ретроспективно все этапы процесса генерации 
знаний.
      
      Вторая модель предназначена для описания динамики процесса 
      генерации~\cite{22-zat}. Концепты, име-\linebreak на и~объекты интерпретации могут 
изменяться\linebreak в~широком диапазоне в~процессе согласования экспертами их личностных 
концептов и~имен. По определению вторая модель описывает динамику процесса генерации 
концептов одним экспертом и~состоит из:
      \begin{itemize}
\item трех сред предметной области информатики: ментальной,  
со\-ци\-аль\-но-ком\-му\-ни\-ка\-ци\-он\-ной и~цифровой;
\item треугольников Фреге, построенных экспертом в~моменты времени 
окончания итераций~$t_i$, $i \hm= 1, 2,\ldots$;
\item цифровых семиотических треугольников, построенных информационной 
системой в~эти же моменты времени~$t_i$.
\end{itemize}

      На основе этой модели динамики процесса генерации концептов 
      в~работе~\cite{22-zat} было дано определение пространства Фреге как 4-мерного 
множества точек для трех кодов разных категорий $\{t_i$, семантический код ($t_i$), 
информационный код ($t_i$), объектный код ($t_i$) при $i \hm= 1, 2,\ldots\}$, 
сгенерированных информационной системой в~процессе работы одного эксперта. 
Аналогичную модель и~соответствующее пространство Фреге можно определить для случая 
генерации знаний коллективом экспертов, пример которого рассматривается 
      в~работе~\cite{23-zat}.
      
      Пространство Фреге имеет три оси координат цифровых кодов: семантическую, 
информационную и~объектную, а также четвертую~--- ось времени, содержащую 
дискретный набор точек начала и~окончания итераций генерации концептов,\linebreak их 
согласования или комплексных итераций. Пространст\-во Фреге дает возможность 
представить графически динамику процесса, используя последовательности значений 
семантических, информационных и~объектных кодов, сгенерированные информационной 
системой в~дискретные моменты времени окончания итераций.
      
      Отметим, что вид дискретных траекторий точек будет определяться алгоритмами 
назначения семантических, информационных и~объектных кодов. Можно ли задать 
некоторую метрику в~пространстве Фреге? В~настоящее время этот вопрос остается 
открытым. Пока этот вопрос находится в~стадии изучения, термин <<пространство Фреге>> 
определен только как семиотическое понятие, но не математическое. При этом если в~
информационной системе фиксируется содержательная эволюция объектов интерпретации, 
сгенерированных концептов, описания которых эксперты формируют в~процессе 
интроспекции, и~имен, то пространство Фреге служит для количественного описания 
процесса генерации концептов.
      
      Прежде чем приступить к~построению модели фиксированного состояния 
      и~нестационар-\linebreak ной модели на основе обобщения двух ранее 
      раз\-работанных моделей, отметим, 
что в~последних использова\-лись только три среды предметной об\-ласти информати\-ки 
(ментальная, со\-ци\-аль\-но-ком\-му\-ни\-ка\-ци\-он\-ная и~цифровая среды) и~первые два 
семиотических основания для моделирования процессов генерации знаний (треугольник 
Фреге\linebreak и~цифровой семиотический треугольник). Пере\-чис\-лим исходные данные построения 
модели фиксированного состояния и~нестационарной модели на основе обобщения двух 
ранее разработанных моделей:
      \begin{itemize}
\item определение семиотического нейротетраэдра и~нейроквадрата, которые 
служат третьим семиотическим основанием;
\item первая исходная модель, которая фиксирует состояние процесса генерации 
концептов в~момент времени, соответствующий окончанию некоторой итерации 
этого процесса;
\item вторая исходная модель, которая описывает динамику процесса генерации 
концептов.
\end{itemize}

      Для обобщения перечисленных двух моделей рассмотрим четыре среды предметной 
области информатики как ин\-фор\-ма\-ци\-он\-но-компью\-тер\-ной науки: ментальную, 
со\-ци\-аль\-но-ком\-му\-ни\-ка\-ци\-он\-ную, нейро- и~цифровую среды~\cite{12-zat}. 
С~формальной 
точки зрения это обобщение представляет собой замену треугольника Фреге на 
семиотический тет\-раэдр, а цифрового семиотического треугольника~--- на нейроквадрат кодов 
(см.\ рис.~3). Сделав такую замену, получаем два следующих обобщения.

      \begin{figure*}[b] %fig4
\vspace*{9pt}
      \begin{center}
      {\small
      \begin{tabular}{|c|p{65mm}|p{65mm}|}
      \hline
\multicolumn{1}{|c|}{Номер пары}&
\multicolumn{1}{c|}{Оригинальный текст}&
\multicolumn{1}{c|}{Перевод}\\
\hline
\hphantom{9}9&Цвет лица у Ильи Ильича не был ни румяный, ни смуглый, ни положительно бледный, а 
безразличный или казался таким, может быть, потому, что Обломов как-то обрюзг не по 
летам: от недостатка ли движения или воздуха, а может быть, того и~другого.&Le teint d'Ilia 
Ilitch n'$\acute{\mbox{e}}$tait ni rose, ni h$\hat{\mbox{a}}$l$\acute{\mbox{e}}$, 
ni carr$\acute{\mbox{e}}$ment p$\hat{\mbox{a}}$le, mais indiff$\acute{\mbox{e}}$rent ou, du 
moins, il le paraissait. Peut-$\hat{\mbox{e}}$tre parce que la chair d'Oblomov 
$\acute{\mbox{e}}$tait pr$\acute{\mbox{e}}$matur$\acute{\mbox{e}}$ment flasque: faute 
d'exercice ou manque d'air, peut-$\hat{\mbox{e}}$tre l'un et l'autre.\\
\hline
18&Халат имел в~глазах Обломова тьму неоцененных достоинств: он мягок, гибок; тело не 
чувствует его на себе; он, как послушный раб, покоряется самомалейшему движению 
тела.&Aux yeux d'Oblomov cette robe de chambre avait une foule de qualit$\acute{\mbox{e}}$s 
inappr$\acute{\mbox{e}}$ciables: elle $\acute{\mbox{e}}$tait douce, souple, ne pesait pas sur le 
corps; telle une esclave docile, elle se pliait au moindre mouvement.\\
\hline
21&Лежанье у Ильи Ильича не было ни необходимостью, \textbf{как} у~больного или как 
\textbf{у~человека, который хочет спать}, ни случайностью, как у~того, кто устал, ни 
наслаждением, как у лентяя: это было его нормальным состоянием.&La position 
allong$\acute{\mbox{e}}$e n'$\acute{\mbox{e}}$tait pour Ilia Ilitch ni 
n$\acute{\mbox{e}}$cessaire, \textbf{comme} pour un malade ou \textbf{pour un homme qui 
veut dormir}, ni accidentelle, comme pour une personne fatigu$\acute{\mbox{e}}$e, ni 
voluptueuse comme chez le fain$\acute{\mbox{e}}$ant; c'$\acute{\mbox{e}}$tait son 
$\acute{\mbox{e}}$tat normal.\\
\hline
\end{tabular}
}
\end{center}

\vspace*{-3pt}

\Caption{Три предложения параллельных текстов на русском языке и~их переводы ({полужирным 
шрифтом выделен контекст, используемый далее на рис.}~6)}
\end{figure*}
      
      Для случая четырех сред модель фиксированного состояния процесса генерации 
знаний со\-сто\-ит~из:
      \begin{itemize}
\item ментальной, со\-ци\-аль\-но-ком\-му\-ни\-ка\-ци\-он\-ной, цифровой сред и~
нейросреды;
\item семиотического тетраэдра, включающего объект интерпретации, концепт, 
сгенерированный или измененный на этой итерации некоторым экспертом, имя 
объекта интерпретации, которое одновременно является и~именем концепта 
в~этот момент времени, а также нейроинформацию о связях между объектом, 
концептом и~их именем;
\item нейроквадрата, включающего цифровые коды объекта, концепта, имени 
и~связывающей их нейроинформации, сгенерированные в~этот же момент времени 
информационной системой, обеспечивающей работу экспертов.
\end{itemize}

      Нестационарная модель динамики процесса генерации знаний одним экспертом 
состоит из:
      \begin{itemize}
\item тех же самых четырех сред предметной области информатики;
\item семиотических тетраэдров, построенных экспертом в~моменты времени 
окончания итераций~$t_i$, $i \hm= 1, 2,\ldots$;
\item нейроквадратов, построенных в~эти же моменты времени~$t_i$.
\end{itemize}

      На основе этой модели динамики процесса аналогично можно определить 
обобщенное пространство Фреге как 5-мер\-ное множество точек для четырех кодов разных 
категорий $\{t_i$, семантический код ($t_i$), информационный код ($t_i$), объектный код 
($t_i$), код нейроинформации ($t_i$) при $i \hm= 1, 2,\ldots\}$, сгенерированных 
информационной системой на $i$-й итерации работы эксперта.
      


      Главное содержание приведенного формального обобщения состоит в~замене 
треугольника Фреге на семиотический тетраэдр, основное отличие которого от 
треугольника Фреге заключается в~наличии нейроинформации. По определению она 
фиксирует связи между объектом, концептом и~именем. Однако остается открытым вопрос 
о практических способах получения нейроинформации об этих связях в~процессе решения 
прикладных задач. Раньше, когда такие задачи решались с~использованием моделей, 
которые охватывали объекты только трех сред, объекты интерпретации были доступны 
экспертам для изменений и~анализа, так как они представляли собой изменяемые во 
времени:
      \begin{itemize}
\item компьютерные программы и~данные, используемые для вычисления 
значений новых индикаторов~\cite{9-zat, 23-zat};
\item фрагменты параллельных текстов на русском и~французском языке (рис.~4), 
в результате контрастивного анализа которых определялись денотаты и~
формировались их кросс-язы\-ко\-вые концепты в~процессе анализа 
параллельных фрагментов~\cite{24-zat, 25-zat}.
      \end{itemize}
      
      Новые концепты знаний, принадлежащие ментальной среде, описывались 
экспертами в~результате субъективной интроспекции с~последующим присвоением имен 
сформированным ими концептам. Иначе говоря, в~моделях, которые охватывали объекты 
трех сред, эксперты сами анализировали объекты интерпретации, описывали концепты 
и~давали имена. После расширения числа сред до четырех в~обобщенных моделях появляется 
нейроинформация о связях между объектом интерпретации, концептом и~именем, которая 
экспертам недоступна. Поэтому и~возникает вопрос о~способах получения 
нейроинформации в~процессе решения практических задач.
      
      Сегодня есть возможность отобразить в~компьютерной форме уровень активности 
разных участ\-ков мозга экспертов в~режиме реального времени, используя метод 
функциональной маг\-нит-\linebreak но-ре\-зо\-нанс\-ной томографии (functional Magnetic Resonance 
Imaging~--- fMRI)~\cite{26-zat}. Этот метод позволяет использовать объективные 
индикаторы уровня активности, наблюдая количественные измерения мозговой 
деятельности, одновременно фиксируя и~описывая концепты как результаты личностного 
анализа экспертами объектов интерпретации в~процессе субъективной интроспекции.
      
      Но и~здесь возникают вполне закономерные вопросы. Можно ли использовать 
объективные индикаторы уровня активности в~процессе генерации новых концептов для 
описания связей между объектом интерпретации, концептом и~именем, а также для 
сопоставления с~результатами личностного анализа? Можно ли их использовать для 
описания процесса согласования новых концептов между экспертами?
      
      В настоящее время действительно есть возможность наблюдать одновременно и~
количественные\linebreak данные измерения мозговой деятельности, и~результаты субъективного 
мышления в~процессе субъективной интроспекции, но из первых сегодня трудно получить 
именно ту нейроинформацию, которая соответствует личностным или согласованным 
концептам экспертов, чтобы провести ее сопоставление с~результатами личностного 
субъективного анализа. Кроме того, есть гипотеза и~подтверждающие ее 
экспериментальные данные, что у экспертов часть нейроинформации, соответствующей 
устоявшимся конвенциональным концептам знаний, носит структурный характер, скорее 
всего, на уровне связей между нейронами долговременной памяти, что не фиксируется 
fMRI и~другими современными методами. Стремительное развитие когнитивной 
нейронауки и~ее инструментальных средств позволяет надеяться, что в~будущем станет 
возможным соотнести структурную нейроинформацию и~количественные нейроданные 
измерений мозговой деятельности с~устоявшимися и~новыми концептами экспертных 
знаний~[26--29].

      
      Однако в~настоящее время при разработке информационных технологий и~решении 
практических задач, когда недоступна \textit{объективная нейроинформация} о~связях 
между объектом интерпретации, концептом и~именем, предлагается по-преж\-не\-му 
использовать результаты \textit{субъективной интроспекции}. Иначе говоря, в~процессе 
итерационной генерации новых знаний эксперты в~информационной системе должны 
описывать не только свои кон\-цеп\-ты и~присваивать им имена, но также устанавливать 
и~фиксировать их связи с~объектами интерпретации, определенными ими денотатами 
и~присвоенными именами.
      
      Предлагаемый подход позволяет уже сегодня использовать обобщенные модели при 
разработке информационных технологий, поддерживающих генерацию новых знаний, 
и~решении практических задач. Кроме того, использование цифровой среды информационной 
системы как носителя этих связей обеспечит доступ всех экспертов к~описаниям 
личностных и~коллективных концептов. Другими словами, анализировать и~обсуждать 
описания таких концептов смогут все эксперты, а не только их авторы, если в~процессе 
итерационной генерации новых знаний эксперты в~информационной системе описывают 
свои концепты, устанавливают и~фиксируют их связи с~объектами интерпретации, 
определенными ими денотатами и~присвоенными именами.


\section{Технология, обеспечивающая генерацию знаний}


      Модели фиксированного состояния и~динамики процесса генерации знаний были 
использованы при разработке информационной технологии,\linebreak
 обеспечивающей 
целенаправленную генерацию и~развитие кросс-язы\-ко\-вых знаний коллективом 
экспертов. Необходимость разработки подобной технологии проявляется наиболее 
наглядно в~ситуации, когда необходимо повысить качество машинного перевода и~для этого 
требуется существенное развитие контрастивных грамматик на основе формирования 
новых кросс-язы\-ко\-вых знаний. При этом направления развития контрастивных 
грамматик должны определяться явно эксплицированными целями, достижение которых 
и~должно непосредственно способствовать повышению качества машинного перевода. 

\begin{figure*}[b] %fig5
\vspace*{1pt}
 \begin{center}
 \mbox{%
 \epsfxsize=165.062mm
 \epsfbox{zac-5.eps}
 }
 \end{center}
 \vspace*{-9pt}
\Caption{Основные этапы технологии (нейросреда не показана;  
со\-ци\-аль\-но-ком\-му\-ни\-ка\-ци\-он\-ная среда для краткости обозначена как 
<<Информационная среда>>)}
\end{figure*}

%\begin{multicols}{2}

      
      При таком подходе кроме моделей состояния и~динамики процесса генерации 
знаний необходимо использовать некоторый способ описания целей. Как было уже 
отмечено, рассматрива\-емый подход к~моделированию процесса генерации знаний при 
явном описании целей и~разработке обеспечивающей технологии ориентирован на те 
прикладные области, где генерируемые экспертные знания являются результатом анализа 
объектов\linebreak интерпретации. В~рассматриваемом примере це\-ле\-на\-прав\-лен\-но\-го формирования 
кросс-язы\-ко\-вых знаний объектами интерпретации являются предложения параллельных 
текстов на рус\-ском и~французском языках, а~денотатами~--- пары тех параллельных 
фрагментов, которые выделяются\linebreak экспертами согласно рассматриваемому ими 
направлению развития контрастивной грамматики (выделенные полужирным шрифтом на 
рис.~4 параллельные фрагменты в~паре №\,21 станут далее одним из объектов анализа).

%\pagebreak

%\end{multicols}




      
      
      Отличительная черта предлагаемого подхода к~моделированию заключается в~явном 
описании
 отношений между новыми экспертными знаниями, объектами интерпретации 
и~денотатами, на основе анализа которых могут быть сгенерированы элементы новых знаний (т.\,е.\ не 
каждый анализируемый объект и~определенный в~процессе анализа денотат всегда 
порождают новый концепт). Реализуемость такого подхода была продемонстрирована 
в~процессе выполнения контрастивных исследований, включающих задачи целенаправленной 
генерации кросс-язы\-ко\-вых знаний:
      \begin{itemize}
\item о переводах глагольных конструкций русского языка на французский;
\item о возможных вариантах перевода лингвоспецифичных слов русского языка 
на французский.
\end{itemize}

      При проведении этих контрастивных исследований кросс-язы\-ко\-вые знания 
формировались экспертами в~процессе анализа параллельных текстов на русском 
и~французском языках с~использованием НДБ~\cite{30-zat, 31-zat}. 
Отметим, что переводной текст является результатом применения переводчиком как 
конвенциональных знаний (в~этом случае анализ соответствующих параллельных текстов 
не приводит к~генерации новых концептов), так и~его невыражаемых знаний, что может 
привести к~генерации новых концептов. Невыражаемые знания могут использоваться 
переводчиками неявно, при этом быть новыми и~неописанными в~контрастивных 
грамматиках в~явной (эксплицитной) форме. В~приведенных далее примерах 
рас\-смат\-ри\-ва\-ют\-ся оригинальные тексты на русском языке, при переводе которых на 
французский язык невыражаемые знания использовались переводчиками, что и~нашло свое 
отражение в~результатах перевода. Поэтому результаты сопоставления оригинальных 
текстов на русском языке и~их переводов могут помочь сформировать и~описать новые 
знания.

     
      
      Разработанная технология~\cite{24-zat, 32-zat, 33-zat}, обеспечивающая генерацию и~
целенаправленное формирование кросс-язы\-ко\-вых знаний, основана на методике, 
созданной Анной А.~Зализняк~[32--34], и~включает следующие 
основные этапы (рис.~5):
      \begin{itemize}
\item из корпуса параллельных текстов отбираются пары предложений как 
объекты интерпретации, содержащие исследуемые языковые объекты (см.\ пару 
выделенных фрагментов на рис.~4, которая является примером 
денотата)\footnote{На рис.~4 приведена пара предложений №\,21 с~глаголом настоящего 
времени русского языка, который в~приведенном далее примере станет исследуемым языковым 
объектом. На момент проведения эксперимента, описанного в~статье, в~рус\-ско-фран\-цуз\-ском 
подкорпусе Национального корпуса русского языка было около 4~тыс.\ пар с~глаголами 
настоящего времени (сейчас их около 20~тыс.).};
\item в~каждой из отобранных пар предложений эксперты анализируют перевод 
исследуемого языково\-го объекта на французский язык и~определяют его 
функционально эквивалентный фрагмент (ФЭФ), в~терминологии 
Д.\,О.~Добровольского;
\item языковой объект текста оригинала сопоставляется с~его ФЭФ согласно 
заданному направлению развития контрастивной грамматики\footnote[2]{Например, 
целью развития контрастивной грамматики может быть формирование расширенного списка 
вариантов перевода глагольных конструкций, включая низкочастотные варианты, 
в том числе такие 
варианты, которые могут отсутствовать в~суще\-ст\-ву\-ющих описаниях контрастивной 
грамматики~\cite{35-zat, 36-zat}, но которые могут использоваться переводчиками. Тогда их 
можно извлекать в~процессе сопоставления текстов оригинала и~перевода.};
\item результат сопоставления описывается экспертами в~формализованном виде 
с использованием методики Анны А.~Зализняк;
\item если вариант перевода исследуемого языкового объекта из текста оригинала 
уже включен в~существующие контрастивные грамматики (это случай пары 
выделенных фрагментов на рис.~4), то формализованное описание 
анализируемого экземпляра переводного соответствия вводится в~НБД 
без дополнения (развития) контрастивной грамматики;
\item если вариант перевода исследуемого языкового объекта из текста оригинала 
экспертами считается новым, то формализованное описание этого экземпляра 
переводного соответствия вводится в~НБД, а в~типологию включается новый вид 
соответствия исследуемого языкового объекта и~его ФЭФ;
\item одновременно НБД генерирует четыре цифровых идентификатора (см.\ рис.~3) 
для обозначения:
\begin{itemize}
\item объекта интерпретации с~выделенным в~нем денотатом, который 
представляет собой пару параллельных текстовых фрагментов;
\item концепта денотата как исследуемого языкового объекта или явления, 
который эксперты описывают в~формализованном виде;
\item имени концепта, которое одновременно является и~именем денотата;
\item связей между объектом интерпретации, включающим денотат, 
концептом и~именем.
\end{itemize}
\end{itemize}

 \begin{figure*} %fig6
      \begin{center}
      {\small
      \begin{tabular}{|p{40mm}| p{30mm}| p{40mm}| p{30mm}|}
      \hline
\multicolumn{1}{|c|}{\tabcolsep=0pt\begin{tabular}{c}Контекст\\ глагольной\\ конструкции\end{tabular}}&
\multicolumn{1}{c|}{\tabcolsep=0pt\begin{tabular}{c}Вид глагольной\\ конструкции\\ и~грамматические\\ 
признаки ее контекста\end{tabular}}&\multicolumn{1}{c|}{Контекст ФЭФ}&
\multicolumn{1}{c|}{\tabcolsep=0pt\begin{tabular}{c}Вид  конструкции ФЭФ\\
и~грамматические\\ признаки его контекста\end{tabular}}\\
\hline
как $[$\ldots$]$ у человека, который \textbf{хочет спать}, 
&\multicolumn{1}{l|}{\tabcolsep=0pt\begin{tabular}{l}\textbf{НастВ}\\
$\langle$~SubInf-IPF~$\rangle$\\
$\langle$~SubAttr~$\rangle$\end{tabular}}&
comme $[$\ldots$]$ pour un homme qui \textbf{veut dormir}, &
\multicolumn{1}{l|}{\tabcolsep=0pt\begin{tabular}{l}\textbf{Present}\\ $\langle$~SubInf~$\rangle$\\ 
$\langle$~SubAttr~$\rangle$\end{tabular}}\\
\hline
 \end{tabular}}
\end{center}
\Caption{Формализованное описание глагольной конструкции и~ее ФЭФ с~известным типологическим видом 
соответствия (НастВ, pr$\acute{\mbox{e}}$sent) ({контекст извлечен из пары №\,$21$ на рис.}~4)}
\end{figure*}


      
      Реализуемость разработанной технологии была проверена в~процессе проведения 
эксперимента по анализу переводов глагольных конструкций русского языка на 
французский~\cite{25-zat}. Целью анализа было развитие типологии видов соответствия 
исследуемых глагольных конструкций и~их ФЭФ. Из рус\-ско-фран\-цуз\-ско\-го 
подкорпуса Национального корпуса русского языка (НКРЯ) были отобраны\linebreak 
около~4000~пар предложений, содержащих глагольные конструкции настоящего времени 
русского языка и~их переводы на французский. Согласно работам~\cite{35-zat, 36-zat} они 
могут быть переведены с~помощью следующих~9~грамматических конструкций: 
pr$\acute{\mbox{e}}$sent, imparfait, infinitif, pass$\acute{\mbox{e}}$ 
compos$\acute{\mbox{e}}$, futur simple, subjonctif pr$\acute{\mbox{e}}$sent, 
g$\acute{\mbox{e}}$rondif, futur imm$\acute{\mbox{e}}$diat и~imp$\acute{\mbox{e}}$ratif.
      
      Таким образом, до начала эксперимента типология видов включала девять записей 
для русского настоящего времени (НастВ). Первая запись имела вид (НастВ, pr$\acute{\mbox{e}}$sent), 
вторая~--- (НастВ, imparfait) и~т.\,д.\ до (НастВ, imp$\acute{\mbox{e}}$ratif). Во время эксперимента эксперты 
сравнивали оригинальный и~переведенный текст в~отобранных парах предложений, выделяя 
глагольную конструкцию НастВ в~тексте оригинала и~ее ФЭФ в~тексте перевода, которые 
в~совокупности являются денотатом.
      
      Если вариант перевода глагольной конструкции НастВ уже был изначально включен 
в типологию (видов соответствия), то формализованное описание этой конструкции и~ее 
ФЭФ (рис.~6) до\-бав\-ля\-ют\-ся в~НБД без изменения типологии видов. Если вариант 
перевода глагольной конструкции НастВ эксперты считают новым, то формализованное 
описание этой конструкции и~ее ФЭФ добавляются в~НБД, а~в~типологию включается 
новый вид (см.\ рис.~5). Формализованное описание создается экспертами на основе 
смыслового содержания соответствия глагольной конструкции и~ее ФЭФ. Это смысловое 
содержание вида соответствия и~является тем концептом, который формируется в~процессе 
анализа этого соответствия. Если концепт оказывается новым, то типология дополняется 
его именем в~формате (вид глагольной конструкции, вид ее ФЭФ).

\begin{table*}\small
\begin{center}
\Caption{Четыре новых типологических вида}
\vspace*{2ex}

\begin{tabular}{|c|l|c|c|}
\hline
№ п/п &\multicolumn{1}{|c|}
{\tabcolsep=0pt\begin{tabular}{c}Типологический\\ вид соответствия\end{tabular}}&
\tabcolsep=0pt\begin{tabular}{c}Число\\ экземпляров\\ вида в~НБД\end{tabular}&
\tabcolsep=0pt\begin{tabular}{c}Статус вида\\ 
(\textit{известный} до начала эксперимента\\ или \textit{новый})\end{tabular}\\
\hline
1&(НастВ, pr$\acute{\mbox{e}}$sent)&311\hphantom{99}&\textit{известный}\\
2&(НастВ, imparfait)&53\hphantom{9}&\textit{известный}\\
3&(НастВ, infinitif)&15\hphantom{9}&\textit{известный}\\
4&(НастВ, pass$\acute{\mbox{e}}$ compos$\acute{\mbox{e}}$)&8&\textit{известный}\\
5&(НастВ, conditionnel pr$\acute{\mbox{e}}$sent)&7&\textit{новый}\\
6&(НастВ, futur simple)&5&\textit{известный}\\
7&(НастВ, subjonctif pr$\acute{\mbox{e}}$sent)&4&\textit{известный}\\
8&(НастВ, participe pass$\acute{\mbox{e}}$)&2&\textit{новый}\\
9&(НастВ, g$\acute{\mbox{e}}$rondif)&2&\textit{известный}\\
10\hphantom{9}&(НастВ, subjonctif imparfait)&1&\textit{новый}\\
11\hphantom{9}&(НастВ, plus-que-parfait)&1&\textit{новый}\\
12\hphantom{9}&(НастВ, futur imm$\acute{\mbox{e}}$diat)&0&\textit{известный}\\
13\hphantom{9}&(НастВ, imp$\acute{\mbox{e}}$ratif)&0&\textit{известный}\\
\hline
\multicolumn{2}{|l|}{Всего записей}&409\hphantom{99}&\\
\hline
\end{tabular}
\end{center}
\end{table*}


      
      Данные эксперимента по извлечению и~описанию новых концептов, полученные на 
первом и~втором его этапах, приведены в~табл.~1 и~2 соответственно.
      

\begin{table*}\small
\begin{center}
\Caption{Восемь новых типологических видов}
\vspace*{2ex}

\begin{tabular}{|c|l|c|c|}
\hline
№ п/п&\multicolumn{1}{|c|}
{\tabcolsep=0pt\begin{tabular}{c}Типологический\\ вид соответствия\end{tabular}}&
\tabcolsep=0pt\begin{tabular}{c}Число\\ экземпляров\\ вида в~НБД\end{tabular}&
\tabcolsep=0pt\begin{tabular}{c}Статус вида\\ 
(\textit{известный} до начала эксперимента\\ или \textit{новый})\end{tabular}\\
\hline
1&(НастВ, pr$\acute{\mbox{e}}$sent)&1587\hphantom{99}&\textit{известный}\\
2&(НастВ, imparfait)&328\hphantom{9}&\textit{известный}\\
3&(НастВ, infinitif)&71&\textit{известный}\\
4&(НастВ, pass$\acute{\mbox{e}}$ compos$\acute{\mbox{e}}$)&30&\textit{известный}\\
5&(НастВ, conditionnel pr$\acute{\mbox{e}}$sent)&23&\textit{новый}\\
6&(НастВ, participe pass$\acute{\mbox{e}}$)&22&\textit{новый}\\
7&(НастВ, subjonctif pr$\acute{\mbox{e}}$sent)&19&\textit{известный}\\
8&(НастВ, futur simple)&19&\textit{известный}\\
9&(НастВ, participe pr$\acute{\mbox{e}}$sent)&19&\textit{новый}\\
10\hphantom{9}&(НастВ, g$\acute{\mbox{e}}$rondif)&15&\textit{известный}\\
11\hphantom{9}&(НастВ, futur imm$\acute{\mbox{e}}$diat)&10&\textit{известный}\\
12\hphantom{9}&(НастВ, pass$\acute{\mbox{e}}$ simple)&10&\textit{новый}\\
13\hphantom{9}&(НастВ, plus-que-parfait)&\hphantom{9}8&\textit{новый}\\
14\hphantom{9}&(НастВ, subjonctif imparfait)&\hphantom{9}6&\textit{новый}\\
15\hphantom{9}&(НастВ, imp$\acute{\mbox{e}}$ratif)&\hphantom{9}5&\textit{известный}\\
16\hphantom{9}&(НастВ, infinitif pass$\acute{\mbox{e}}$)&\hphantom{9}3&\textit{новый}\\
17\hphantom{9}&(НастВ, pass$\acute{\mbox{e}}$ imm$\acute{\mbox{e}}$diat)&\hphantom{9}1&\textit{новый}\\
\hline
\multicolumn{2}{|l|}{Всего записей}&2176\hphantom{99}&\\
\hline
\end{tabular}
\end{center}
\end{table*}

      
      Таблица~1 содержит результаты первого этапа анализа 409~пар предложений 
из~4000, т.\,е.\ приблизительно 10\% от общего их числа. На этом этапе эксперты выявили и~
описали четыре новых типологических вида перевода русского настоящего времени, 
которым присвоили следующие имена: (НастВ, conditionnel pr$\acute{\mbox{e}}$sent), 
(НастВ, participe pass$\acute{\mbox{e}}$), (НастВ, subjonctif imparfait) и~(НастВ, 
      plus-que-parfait). В~то же самое время они не нашли примеры вариантов перевода с~
глагольными конструкциями futur imm$\acute{\mbox{e}}$diat и~imp$\acute{\mbox{e}}$ratif.
      
      Таким образом, первый вариант описания цели развития этой типологии видов мог 
бы состоять в~том, чтобы найти примеры для всех изначально известных девяти 
типологических видов и~описать найденные новые виды (как минимум найти и~описать 
один новый вид). В~этом случае обработка 409~пар недостаточна, так как для достижения 
такой цели эксперты должны продолжать искать примеры французских переводов с~
глагольными конструкциями futur imm$\acute{\mbox{e}}$diat и~imp$\acute{\mbox{e}}$ratif 
(см.\ табл.~1).
      
      Таблица~2 содержит результаты второго этапа анализа 2176~пар предложений 
из~4000, т.\,е.\ приблизительно 54\% от общего их числа. Эксперты выявили и~описали еще 
четыре новых типологических вида перевода русского настоящего времени: (НастВ, 
participe pr$\acute{\mbox{e}}$sent), (НастВ, pass$\acute{\mbox{e}}$ simple), (НастВ, infinitif 
pass$\acute{\mbox{e}}$) и~(НастВ, pass$\acute{\mbox{e}}$ imm$\acute{\mbox{e}}$diat) 
(см.\ табл.~2).
      

      
      Одновременно они нашли французские переводы с~глагольными конструкциями 
futur imm$\acute{\mbox{e}}$diat и~imp$\acute{\mbox{e}}$ratif. В~итоге проведенного 
эксперимента все восемь новых типологических видов были добавлены экспертами 
к~девяти уже имеющимся в~типологии. Дополненная типология включала 17~видов после 
обработки 2176~пар предложений из~4000 (см.\ табл.~2).
      
      Второй вариант описания цели развития этой типологии видов мог бы состоять 
      в~том, чтобы \mbox{найти} примеры для всех изначально известных девяти видов, 
      описать 
найденные новые типологические виды (как минимум найти и~описать один новый вид) при 
условии, что эксперты должны обработать не менее чем 55\% от всех пар предложений 
с~глагольной конструкцией НастВ, имеющихся в~рус\-ско-фран\-цуз\-ском подкорпусе НКРЯ, 
т.\,е.\ 55\% от~4000 на момент проведения эксперимента. В~этом случае анализ 2176~пар 
был недостаточен и~эксперты должны были бы продолжить свою работу, пока не будет 
обработано 2200~пар.
      
      Кроме развития типологии видов соответствия глагольной конструкции и~ее ФЭФ 
разработанная технология в~настоящее время используется для формирования списков 
возможных вариантов перевода лингвоспецифичных слов русского языка на французский 
язык~\cite{37-zat}, а также для формирования методологии контрастивного корпусного 
исследования категории безличности в~русском языке. Таким образом, компьютерная 
поддержка процессов генерации и~целенаправленного развития экспертами  
кросс-язы\-ко\-вых знаний была опробована в~процессе развития контрастивной  
рус\-ско-фран\-цуз\-ской грамматики для глагольных конструкций и~продолжает 
применяться для контрастивных исследований лингвоспецифичных слов русского языка.

\section{Заключение}

      Двуязычные параллельные корпуса, в~которых каждому тексту на русском языке 
соответствует один или несколько его переводов на другой язык, являются потенциальным 
и неисчерпаемым источником генерации новых, но трудно извлека\-емых кросс-язы\-ко\-вых 
знаний. Являясь уникальным и~постоянно пополняемым, он может быть использован для 
существенного повышения качества машинного перевода, актуализации моно- 
и~двуязычных грамматик, а также для обновления широкого спектра образовательных курсов 
по лингвистике, теории и~практике перевода.
      
      Однако функциональность традиционных электронных корпусов не обеспечивает 
извлечения тех невыражаемых знаний переводчиков, которые применялись ими в~процессе 
перевода. Эти знания могут быть личностными или коллективными и~передаваться 
в~процессе их социализации (см.\ рис.~1), например в~процессе демонстрации образцов 
перевода в~процессе обучения, но при этом они могут продолжать оставаться 
невыражаемыми и~неэксплицированными. Наблюдается парадокс: с~одной\linebreak стороны, 
в~электронных корпусах есть образцы переводов, полученные с~применением невыра-\linebreak жа\-емых 
знаний переводчиков; с~другой стороны, традиционные параллельные корпуса не могут 
поддержать процессы извлечения и~экспликации этих знаний. Поэтому понадобилось 
существенное дополнение функциональности традиционных корпусов за счет разработки 
новой информационной технологии целенаправленной генерации знаний.
      
      Разработка этой технологии была связана с~развитием семиотических оснований 
информатики как ин\-фор\-ма\-ци\-он\-но-компью\-тер\-ной науки и~созданием новых 
моделей целенаправленной генерации и~развития новых знаний с~использованием четырех 
сред ее предметной области. К~ментальной, со\-ци\-аль\-но-ком\-му\-ни\-ка\-ци\-он\-ной и~цифровой 
средам была добавлена нейросреда. Ее добавление стало основой для определения 
нейросемиотического тет\-ра\-эдра, что представляет собой качественно новое развитие 
понятия семиотического тет\-ра\-эдра, предложенного группой FRISCO в~конце прошлого 
века.
      
      Суть этого развития в~том, что предложено разделять объекты ментальной среды 
      и~нейросреды в~предметной области информатики. На практике такое деление уже 
используется в~процессе разработки ряда когнитивных и~нейрокоммуникационных 
технологий. Следовательно, это должно найти свое отражение и~в теоретических 
основаниях информатики, а также в~образовательных курсах по ее изучению в~системе 
среднего и~высшего профессионального образования. Отметим, что разделение объектов 
ментальной среды и~нейросреды существенно увеличивает спектр интерфейсов, которые 
являются новыми объектами исследований в~информатике~\cite{12-zat}.
      
      Адаптация разработанной технологии для проведения контрастивных исследований 
повлекла за собой необходимость в~новой категории инфор\-мационных лингвистических 
ресурсов, получивших название НБД, методы формирования которых 
разработаны М.\,Г.~Кружковым~\cite{30-zat, 31-zat, 33-zat, 34-zat}. С~прикладной точки 
зрения реализация в~этих базах данных моделей и~технологии целена\-прав\-лен\-ной генерации 
и развития новых знаний дала возможность существенно дополнить функциональность 
электронных корпусов текстов и~тем самым обеспечить извлечение тех невыражаемых 
и~труд\-нодоступных знаний переводчиков, которые применялись ими, являясь 
неэксплицированными и~новыми в~контрастивной лингвистике.
      
{\small\frenchspacing
 {%\baselineskip=10.8pt
 \addcontentsline{toc}{section}{References}
 \begin{thebibliography}{99}
\bibitem{1-zat}
\Au{Nonaka I.} The knowledge-creating company~// Harvard Bus. Rev., 1991. 
Vol.~69. No.\,6. P.~96--104.
\bibitem{2-zat}
\Au{Nonaka I.} A dynamic theory of organizational knowledge creation~// Organ. 
Sci., 1994. Vol.~5. No.\,1. P.~14--37.
\bibitem{3-zat}
\Au{Wierzbicki A.\,P., Nakamori~Y.} Basic dimensions of creative space // Creative space: 
Models of creative processes for knowledge civilization age~/ Eds. A.\,P.~Wierzbicki, 
Y.~Nakamori.~--- Berlin--Heidelberg: Springer Verlag, 2006. P.~59--90.
\bibitem{4-zat}
\Au{Wierzbicki A.\,P., Nakamori Y.} Knowledge sciences: Some new developments~// 
Zeitschrift f$\ddot{\mbox{u}}$r Betriebswirtschaft, 2007. Vol.~77. No.\,3. P.~271--295.
\bibitem{5-zat}
\Au{Wierzbicki A.\,P., Nakamori Y.} The importance of multimedia principle and emergence 
principle for the knowledge civilisation age~// J.~Syst. Sci. Syst. Eng., 2008. 
Vol.~17. No.\,3. P.~297--318.
\bibitem{6-zat}
\Au{Nakamori Y.} Methodology for knowledge synthesis~// 
Cutting-edge research topics on  multiple criteria decision making~/ 
Eds.\ Y.~Shi, S.~Wang, Y.~Peng, J.~Li, Y.~Zeng.~---
Communications in computer and information science ser.~--- 
Berlin: Springer, 2009. Vol.~35. P.~311--317.
\bibitem{7-zat}
Knowledge science~--- modeling the knowledge creation process~/
Ed. Y.~Nakamori.~--- London\,--\,New York: CRC Press, 2011. 
177~p.
\bibitem{8-zat}
\Au{Nakamori Y.} Knowledge and systems science~--- enabling systemic knowledge 
synthesis.~--- London\,--\,New York: CRC Press, 2013. 234~p.
\bibitem{9-zat}
\Au{Zatsman I., Buntman P.} Theoretical framework and denotatum-based models of 
knowledge creation for monitoring and evaluating R\&D program implementation~// 
Int. J.~Softw. Sci. Comput. Intell., 2013. Vol.~5. No.\,1. 
P.~15--31.
\bibitem{10-zat}
\Au{Зацман И.\,М.} Построение системы терминов 
ин\-фор\-ма\-ци\-он\-но-компью\-тер\-ной науки: 
проб\-лем\-но-ори\-ен\-ти\-ро\-ван\-ный подход~// Теория и~практика общественной 
научной информации.~--- М.: \mbox{ИНИОН} РАН, 2013. Вып.~21. С.~120--159.

\bibitem{12-zat} %11
\Au{Зацман И.\,М.} Таблица интерфейсов информатики как  
ин\-фор\-ма\-ци\-он\-но-компью\-тер\-ной науки~// 
На\-уч.-тех\-нич. информация. Сер.~1: Организация и~методика информационной 
работы, 2014. №\,11. С.~1--15.

\bibitem{11-zat} %12
\Au{Зацман И.\,М.} Ин\-фор\-ма\-ци\-он\-но-компью\-тер\-ная наука: технологические 
предпосылки становления~// Информационные технологии, 2014. №\,3. С.~3--12.

\bibitem{13-zat}
\Au{Успенский В.\,А.} К~публикации статьи Г.~Фреге <<Смысл и~денотат>>~// 
Семиотика и~информатика, 1997. Вып.~35. 
С.~351--352.
\bibitem{14-zat}
\Au{Фреге Г.} Смысл и~денотат~// Семиотика и~информатика, 1997. Вып.~35. С.~352--379.
\bibitem{15-zat}
\Au{Фреге Г.} Понятие и~вещь~// Семиотика и~информатика, 1997. Вып.~35. С.~380--396.
\bibitem{16-zat}
\Au{Зацман И.\,М.} Семиотическая модель взаимосвязей концептов, информационных 
объектов и~компьютерных кодов~// Информатика и~её применения, 2009. Т.~3. Вып.~2. 
С.~65--81.
\bibitem{17-zat}
A~framework of information system concepts (Web edition): The FRISCO Report.~--- IFIP, 
1998. {\sf http://www.mathematik.uni-marburg.de/$\sim$hesse/\linebreak papers/fri-full.pdf}.
\bibitem{18-zat}
\Au{Hesse W., Verrijn-Stuart A.\,A.} Towards a~theory of information systems: The FRISCO 
approach~// Information modelling and knowledge bases~XII~/ Eds. H.~Kangassalo, 
H.~Jaakkola, E.~Kawaguchi.~--- Amsterdam: IOS Press, 2001. P.~81--91.
\bibitem{19-zat}
\Au{Eco U.} A~theory of semiotics.~--- Bloomington: Indiana University Press, 1976. 356~p.
\bibitem{20-zat}
\Au{Пирс Ч.} Логические основания теории знаков~/
Пер. с~англ.~---  СПб.: Алетейя, 2000. 352~с.

\bibitem{21-zat}
\Au{Василюк Ф.\,Е.} Структура образа~// Вопросы психологии, 1993. №\,5. С.~5--19.
\bibitem{22-zat}
\Au{Зацман И.\,М.} Нестационарная семиотическая модель компьютерного кодирования 
концептов, информационных объектов и~денотатов~// Информатика и~её применения, 
2009. Т.~3. Вып.~4. С.~87--101.
\bibitem{23-zat}
\Au{Зацман И.\,М., Бунтман П.\,С.} Проектирование индикаторов мониторинга в~сфере 
науки: теоретические основания и~модели~// Онтология проектирования, 2014. №\,3(13). 
С.~32--51.
\bibitem{24-zat}
\Au{Zatsman I., Buntman N., Kruzhkov M., Nuriev~V., Zalizniak Anna~A.} Conceptual 
framework for development of computer technology supporting cross-linguistic knowledge 
discovery~// 15th European Conference on Knowledge Management Proceedings.~---  
Reading: Academic Publishing International Ltd., 2014. Vol.~3. P.~1063--1071.
\bibitem{25-zat}
\Au{Zatsman I., Buntman N.} Outlining goals for discovering new knowledge and 
computerised tracing of emerging meanings discovery~// 16th European Conference on 
Knowledge Management Proceedings.~--- Reading: Academic Publishing International 
Ltd., 2015. P.~851--860.
\bibitem{26-zat}
\Au{Баарс~Б., Гейдж~Н.} Мозг, познание, разум: введение в~когнитивные 
нейронауки~/ Пер.\ с~англ.~--- М.: БИНОМ. Лаборатория знаний, 2014. Ч.~1. 544~с.; Ч.~2. 464~с.
(\Au{Baars~B., Gage~N.} Cognition, brain, and consciousness: Introduction to cognitive 
neuroscience.~--- Burlington, MA, USA: Academic Press/Elsevier, 2010. 677~p.)
\bibitem{27-zat}
\Au{Секерина И.\,А.} Метод вызванных потенциалов мозга в~экспериментальной 
психолингвистике~// Вопросы языкознания, 2006. №\,3. С.~22--45.
\bibitem{28-zat}
\Au{De Charms R.\,C.} Applications of real-time fMRI~// Nat. Rev. Neurosci., 
2008. Vol.~9. No.\,9. P.~720--729.
\bibitem{29-zat}
\Au{Kumaran D., Summereld J.\,J., Hassabis~D., Maguire~E.\,A.} Tracking the emergence of 
conceptual knowledge during human decision-making~// Neuron, 2009. Vol.~63. No.\,6. 
P.~889--901.
\bibitem{30-zat}
\Au{Зализняк А.\,А., Зацман И.\,М., Инькова~О.\,Ю., Кружков~М.\,Г.} Надкорпусные 
базы данных как лингвистический ресурс~// Корпусная лингвистика-2015: Тр. 7-й 
Междунар. конф.~--- СПб.: СПбГУ, 2015. С.~211--218.
\bibitem{31-zat}
\Au{Кружков М.\,Г.} Информационные ресурсы контрастивных лингвистических 
исследований: электронные корпуса текстов~// Системы и~средства информатики, 2015. 
Т.~25. Вып.~2. С.~140--159.
\bibitem{32-zat}
\Au{Loiseau~S., Sitchinava~D.\,V., Zalizniak~Anna~A., Zatsman~I.\,M.} Information 
technologies for creating the database of equivalent verbal forms in the Russian-French 
multivariant parallel corpus~// Информатика и~её применения, 2013. Т.~7. №\,2. 
С.~100--109.
\bibitem{33-zat}
\Au{Kruzhkov M.\,G., Buntman N.\,V., Loshchilova~E.\,Ju., Sitchinava~D.\,V., Zalizniak 
Anna~A., Zatsman~I.\,M.} A~database of Russian verbal forms and their French translation 
equivalents~// Компьютерная лингвистика и~интеллектуальные технологии: По мат-лам 
ежегодной Междунар. конф. <<Диалог>>.~--- М.: РГГУ, 2014. Вып.~13(20).
С.~284--297.
\bibitem{34-zat}
\Au{Бунтман Н.\,В., Зализняк Анна~A., Зацман~И.\,M., Кружков~М.\,Г., 
Лощилова~Е.\,Ю., Сичинава~Д.\,В.} Инфор\-ма\-ционные технологии корпусных 
исследований:\linebreak принципы построения кросслингвистических баз данных~// Информатика 
и её применения, 2014. Т.~8. Вып.~2. С.~98--110.
\bibitem{35-zat}
\Au{Гак В.\,Г.} Русский язык в~сопоставлении с~французским.~--- М.: УРСС, 2006. 264~с.
\bibitem{36-zat}
\Au{Kouznetsova I.\,N.} Grammaire contrastive du francais et du russe.~--- Moscow: Nestor 
Academic Publs., 2009. 272~p.
\bibitem{37-zat}
\Au{Зализняк Анна А.} Лингвоспецифичные единицы русского языка в~свете 
контрастивного корпусного анализа~// Компьютерная лингвистика и~интеллектуальные 
технологии: По мат-лам ежегодной Междунар. конф. <<Диалог>>.~--- М.: РГГУ, 2015. 
Вып.~14(21). Т.~1. С.~683--695.

 \end{thebibliography}

 }
 }

\end{multicols}

\vspace*{-3pt}

\hfill{\small\textit{Поступила в~редакцию 22.07.15}}

%\newpage

\vspace*{12pt}

\hrule

\vspace*{2pt}

\hrule

%\vspace*{12pt}

\def\tit{GOAL-ORIENTED PROCESSES OF~CROSS-LINGUAL EXPERT KNOWLEDGE CREATION:
SEMIOTIC FOUNDATIONS FOR~MODELING}

\def\titkol{Goal-oriented processes of cross-lingual expert knowledge creation:
Semiotic foundations for modeling}

\def\aut{I.\,M.~Zatsman}

\def\autkol{I.\,M.~Zatsman}

\titel{\tit}{\aut}{\autkol}{\titkol}

\vspace*{-9pt}


\noindent
Institute of Informatics Problems, Federal Research Center ``Computer Science and Control'' of 
the Russian Academy of Sciences, 44-2 Vavilov Str., Moscow 119333, Russian Federation


\def\leftfootline{\small{\textbf{\thepage}
\hfill INFORMATIKA I EE PRIMENENIYA~--- INFORMATICS AND
APPLICATIONS\ \ \ 2015\ \ \ volume~9\ \ \ issue\ 3}
}%
 \def\rightfootline{\small{INFORMATIKA I EE PRIMENENIYA~---
INFORMATICS AND APPLICATIONS\ \ \ 2015\ \ \ volume~9\ \ \ issue\ 3
\hfill \textbf{\thepage}}}

\vspace*{3pt}

\Abste{The results of development of semiotic foundations for modeling goal-oriented processes 
of cross-lingual expert knowledge creation are described. The technology supporting these 
processes is outlined. The demand for such technologies is obvious in situations where present 
systems of expert knowledge do not answer to new socially or technologically significant 
purposes, corresponding to new or changed requirements of modern society. Instead of centering 
on the well-known artificial intelligence methods and models of information processing for 
knowledge representation, this paper focuses on development of new models of goal-oriented 
processes of expert knowledge creation reflecting dynamics of its formation. The suggested 
approach to modeling these processes and to development of technologies supporting them is 
focused on those applied areas where expert knowledge is elicited from domain experts. The 
experts analyze texts or other interpretation objects which can vary over time and enter\linebreak\vspace*{-12pt}}

\Abstend{the 
results of analysis into supracorpus databases. The distinguishing feature of the semiotic approach 
to modeling is the explicit description of relations between new expert knowledge and those 
interpretation objects, from which parts of new knowledge were generated. Other important 
feature is the explicit description of parts of knowledge corresponding to interpretation objects that 
may vary over time. Feasibility of the approach is demonstrated on
the example of information 
technology, which supports the processes of creation of cross-lingual expert knowledge based on 
French translations of Russian verbal constructions. Cross-lingual knowledge is generated in the 
course of analysis of parallel texts in Russian and French languages.}

\KWE{cross-lingual expert knowledge; computer modeling; knowledge creation; interpretation 
objects; semiotic foundations; models of knowledge creation processes}

\DOI{10.14357/19922264150311}

\Ack
\noindent
The work was supported by the Russian Foundation for Basic Research 
(projects 14-07-00785, 13-06-00403) and the Rusian Foundation for Humanities 
(project 15-04-00507).

%\vspace*{3pt}

  \begin{multicols}{2}

\renewcommand{\bibname}{\protect\rmfamily References}
%\renewcommand{\bibname}{\large\protect\rm References}

{\small\frenchspacing
 {%\baselineskip=10.8pt
 \addcontentsline{toc}{section}{References}
 \begin{thebibliography}{99}
\bibitem{1-zat-1}
\Aue{Nonaka, I.} 1991. The knowledge-creating company. \textit{Harvard Bus. Rev.} 
69(6):96--104.
\bibitem{2-zat-1}
\Aue{Nonaka, I.} 1994. A~dynamic theory of organizational knowledge creation. 
\textit{Organ. Sci.} 5(1):14--37.
\bibitem{3-zat-1}
\Aue{Wierzbicki, A.\,P., and Y.~Nakamori}. 2006. Basic dimensions of creative space. 
\textit{Creative space: Models of creative processes for knowledge civilization age}. 
Eds. A.\,P.~Wierzbicki,  and
Y.~Nakamori. Berlin--Heidelberg: Springer Verlag. 59--90.
\bibitem{4-zat-1}
\Aue{Wierzbicki, A.\,P., and Y.~Nakamori}. 2007. Knowledge sciences: Some new 
developments. \textit{Zeitschrift f$\ddot{\mbox{u}}$r Betriebswirtschaft} 77(3):271--295.
\bibitem{5-zat-1}
\Aue{Wierzbicki, A.\,P., and Y. Nakamori}. 2008. The importance of multimedia principle 
and emergence principle for the knowledge civilisation age. \textit{J.~Syst. Sci. Syst.
Eng.} 17(3):297--318.
\bibitem{6-zat-1}
\Aue{Nakamori, Y.} 2009. Methodology for knowledge synthesis. \textit{Cutting-edge 
research topics on multiple criteria decision making}. 
Eds.\ Y.~Shi, S.~Wang, Y.~Peng, J.~Li, and Y.~Zeng. Communications in computer and 
information science ser. Berlin: Springer. 35:311--317.
\bibitem{7-zat-1}
Nakamori, Y., ed. 2011. \textit{Knowledge science~--- modeling the knowledge creation 
process}. London\,--\,New York: CRC Press. 177~p.
\bibitem{8-zat-1}
\Aue{Nakamori, Y.} 2013. \textit{Knowledge and systems science~--- enabling systemic 
knowledge synthesis}. London\,--\,New York: CRC Press. 234~p.
\bibitem{9-zat-1}
\Aue{Zatsman, I., and P. Buntman}. 2013. Theoretical framework and denotatum-based 
models of knowledge creation for monitoring and evaluating R\&D program implementation. 
\textit{Int. J.~Softw. Sci. Comput. Intell.} 5(1):15--31.
\bibitem{10-zat-1}
\Aue{Zatsman, I.} 2013. Postroenie sistemy terminov informatsionno-komp'yuternoy nauki: 
problemno-orientirovannyy podkhod [Construction of the system of terms of information and 
computer science: A~problem-oriented approach]. \textit{Teoriya i~praktika obshchestvennoy 
nauchnoy informatsii} [Theory and practice of scientific information for social sciences]. 
Moscow: INION RAS. 120--159.

\bibitem{12-zat-1} %11
\Aue{Zatsman, I.} 2014. Tablitsa interfeysov informatiki kak informatsionno-komp'yuternoy 
nauki [A~table of interfaces of informatics as computer and information science]. 
\textit{Nauchno-tekhnicheskaya informatsiya. Ser.~1:\linebreak Organizatsiya i~metodika 
informatsionnoy raboty} [Scientific and Technical Information. Ser.~1: Management and 
methodology of information work] (11):1--15.

\bibitem{11-zat-1} %12
\Aue{Zatsman, I.} 2014. Informatsionno-komp'yuternaya nauka: Tekhnologicheskie 
predposylki stanovleniya [Information and computer science: Technological prerequisites of 
formation]. \textit{Informatsionnye Tekhnologii} [Information Technologies] (3):3--12.

\bibitem{13-zat-1}
\Aue{Uspenskiy, V.\,A.} 1997. K~publikatsii stat'i G.~Frege ``Smysl i~denotat'' [To the 
publication of the paper of G.~Frege ``Sense and reference'']. \textit{Semiotika i Informatika} 
[Semiotics and Informatics] 35:351--352.
\bibitem{14-zat-1}
\Aue{Frege, G.} 1997. Smysl i denotat [Sense and reference]. \textit{Semiotika i~Informatika} 
[Semiotics and Informatics] 35:352--379.
\bibitem{15-zat-1}
\Aue{Frege, G.} 1997. Ponyatie i~veshch' [Concept and thing]. \textit{Semiotika 
i~Informatika} [Semiotics and Informatics] 35:380--396.
\bibitem{16-zat-1}
\Aue{Zatsman, I.} 2009. Semioticheskaya model' vzaimosvyazey kontseptov, 
informatsionnykh ob"ektov i komp'yuternykh kodov [Semiotic model of relationships of 
concepts, information objects, and computer codes]. \textit{Informatika i~ee Primeneniya}~--- 
\textit{Inform. Appl.} 3(2):65--81.
\bibitem{17-zat-1}
FRISCO.  1998. A framework of information system concepts. 
Report. Available at: {\sf 
http://www.mathematik.uni-marburg.de/$\sim$hesse/papers/fri-full.pdf} (accessed July~29, 
2015).
\bibitem{18-zat-1}
\Aue{Hesse, W., and A.\,A.~Verrijn-Stuart}. 2001. {Towards a~theory of information 
systems: The FRISCO approach}. \textit{Information modelling and knowledge bases~XII}. 
Eds. H.~Kangassalo,  H.~Jaakkola, and E.~Kawaguchi.
Amsterdam: IOS Press. 81--91.
\bibitem{19-zat-1}
\Aue{Eco, U.} 1976. \textit{A~theory of semiotics}. Bloomington: Indiana University Press. 
356~p.
\bibitem{20-zat-1}
\Aue{Peirce, Ch.\,S.} 1931--1958. \textit{Collected papers of Charles S.~Peirce}. 
Cambridge: Harvard University Press. 8~vols.
\bibitem{21-zat-1}
\Aue{Vasilyuk, F.\,E.} 1993. Struktura obraza [Structure of image]. \textit{Voprosy 
Psikhologii} [Questions of Psychology] (5):5--19.
\bibitem{22-zat-1}
\Aue{Zatsman, I.} 2009. Nestatsionarnaya semioticheskaya model' komp'yuternogo 
kodirovaniya kontseptov, informatsionnykh ob"ektov i~denotatov [Nonstationary semiotic 
model of computer coding of concepts, information objects and denotata]. \textit{Informatika 
i~ee Primeneniya}~--- \textit{Inform. Appl.} 3(4):87--101.
\bibitem{23-zat-1}
\Aue{Zatsman, I., and P. Buntman}. 2014. Proektirovanie indikatorov monitoringa v~sfere 
nauki: Teoreticheskie osnovaniya i~modeli [Design of indicators for monitoring in science: 
Theoretical foundations and models]. \textit{Ontologiya Proektirovaniya} [Ontology of 
Design] (3):32--51.
\bibitem{24-zat-1}
\Aue{Zatsman, I., N. Buntman, M.~Kruzhkov, V.~Nuriev, and Anna A.~Zalizniak}. 2014. 
Conceptual framework for development of computer technology supporting cross-linguistic 
knowledge discovery. \textit{15th European Conference on Knowledge Management 
Proceedings}. Reading: Academic Publishing International Ltd. 3:1063--1071.
\bibitem{25-zat-1}
\Aue{Zatsman, I., and N. Buntman}. 2015. Outlining goals for discovering new knowledge 
and computerised tracing of emerging meanings discovery. \textit{16th European Conference 
on Knowledge Management Proceedings}. Reading: Academic Publishing International 
Ltd. 851--860.
\bibitem{26-zat-1}
\Aue{Baars, B., and N. Gage}. 2010. \textit{Cognition, brain, and consciousness: Introduction 
to cognitive neuroscience}. Burlington, MA: Academic Press/Elsevier. 677~p.
\bibitem{27-zat-1}
\Aue{Sekerina, I.} 2006. Metod vyzvannykh potentsialov mozga v~eksperimental'noy 
psikholingvistike [Method of evoked potentials of brain in experimental psycholinguistics]. 
\textit{Voprosy Yazykoznaniya} [Topics in the Study of Language] 3:22--45.
\bibitem{28-zat-1}
\Aue{De Charms, R.\,C.} 2008. Applications of real-time fMRI. \textit{Nat. Rev. 
Neurosci.} 9(9):720--729.
\bibitem{29-zat-1}
\Aue{Kumaran, D., J.\,J. Summereld, D.~Hassabis, and E.\,A.~Maguire}. 2009. Tracking the 
emergence of conceptual knowledge during human decision-making. \textit{Neuron} 
63(6):889--901.
\bibitem{30-zat-1}
\Aue{Zalizniak, Anna A., I.~Zatsman, O.~Inkova, and M.~Kruzhkov}. 2015. Nadkorpusnye 
bazy dannykh kak lingvisticheskiy resurs [Supracorpus database as linguistic resource]. 
\textit{Tr. 7-y konf. po Korpusnoy Lingvistike} [7th Conference on Corpus Linguistics 
Proceedings]. St.\ Petersburg. 211--218.
\bibitem{31-zat-1}
\Aue{Kruzhkov, M.} 2015. Informatsionnye resursy kontrastivnykh lingvisticheskikh 
issledovaniy: Elektronnye korpusa tekstov [Information resources for contrastive studies: 
Digital text corpora]. \textit{Sistemy i~Sredstva Informatiki}~--- \textit{Systems and Means of 
Informatics} 25(2):140--159.
\bibitem{32-zat-1}
\Aue{Loiseau, S., D.\,V. Sitchinava, Anna A.~Zalizniak, and I.\,M.~Zatsman}. 2013. 
Information technologies for creating the database of equivalent verbal forms in the 
Russian-French multivariant parallel corpus. \textit{Informatika i~ee Primeneniya}~--- 
\textit{Inform.s Appl.} 7(2):100--109.
\bibitem{33-zat-1}
\Aue{Kruzhkov, M.\,G., N.\,V. Buntman, E.\,Ju.~Loshchilova, D.\,V.~Sitchinava, Anna 
A.~Zalizniak, and I.\,M.~Zatsman}. 2014. A~database of Russian verbal forms and their 
French translation equivalents. \textit{Komp'yuternaya Lingvistika i~Intellektual'nye 
Tekhnologii. Po mat-lam ezhegodnoy Mezhdunar. konf. ``Dialog-2014''} [Computational 
Linguistics and Intellectual Technologies: Conference (International) ``Dialog-2014'' 
Proceedings]. Moscow. 13(20):284--297.
\bibitem{34-zat-1}
\Aue{Buntman, N.\,V., Anna A.~Zaliznyak, I.\,M.~Zatsman, M.\,G.~Kruzhkov, 
E.\,Yu.~Loshchilova, and D.\,V.~Sitchinava}. 2014. Informatsionnye tekhnologii korpusnykh 
issledovaniy: Printsipy postroeniya krosslingvisticheskikh baz dannykh [Information 
technologies for corpus studies: Underpinnings for cross-linguistic database creation]. 
\textit{Informatika i~ee Primeneniya}~--- \textit{Inform. Appl.} 8(2):98--110.
\bibitem{35-zat-1}
\Aue{Gak, V.\,G.} 2006. \textit{Russkiy yazyk v~sopostavlenii s~fran\-tsuz\-skim} [Russian in 
comparison to French]. Moscow: URSS. 264~p.
\bibitem{36-zat-1}
\Aue{Kouznetsova, I.\,N.} 2009. \textit{Grammaire contrastive du francais et du russe}. 
Moscow: Nestor Academic Publs. 272~p.
\bibitem{37-zat-1}
\Aue{Zalizniak, Anna~A.} Lingvospetsifichnye edinitsy russkogo yazyka v~svete 
kontrastivnogo korpusnogo analiza [Russian language-specific words in light of the contrastive 
corpus analysis]. \textit{Komp'yuternaya Lingvistika i~Intellektual'nye Tekhnologii. Po 
mat-lam ezhegodnoy Mezhdunar. konf. ``Dialog-2015''} [Computational Linguistics and 
Intellectual Technologies: Conference (International) ``Dialog-2015'' Proceedings]. Moscow.  
14(21):683--695.
\end{thebibliography}

 }
 }

\end{multicols}

\vspace*{-3pt}

\hfill{\small\textit{Received July 22, 2015}}

\Contrl

\noindent
\textbf{Zatsman Igor M.} (b.\ 1952)~--- 
Doctor of Science in technology, Head of Department, Institute of Informatics Problems, Federal Research Center 
``Computer Science and Control'' of the Russian Academy of Sciences, 44-2 Vavilov Str., Moscow 119333, 
Russian Federation;  iz\_ipi@a170.ipi.ac.ru
\label{end\stat}


\renewcommand{\bibname}{\protect\rm Литература}      %13
\def\stat{hvatova}

\def\tit{МОДЕЛИРОВАНИЕ СТОХАСТИЧЕСКОЙ ДИНАМИКИ ИЗМЕНЕНИЯ СОСТОЯНИЙ УЗЛОВ 
И~ПЕРКОЛЯЦИОННЫХ ПЕРЕХОДОВ В~СОЦИАЛЬНЫХ СЕТЯХ С~УЧЕТОМ 
САМООРГАНИЗАЦИИ И~НАЛИЧИЯ ПАМЯТИ}

\def\titkol{Моделирование стохастической динамики изменения состояний узлов 
и~перколяционных переходов} % в~социальных сетях с~учетом самоорганизации и~наличия памяти}

\def\aut{Д.\,О.~Жуков$^1$, Т.\,Ю.~Хватова$^2$, А.\,Д.~Зальцман$^3$}

\def\autkol{Д.\,О.~Жуков, Т.\,Ю.~Хватова, А.\,Д.~Зальцман}

\titel{\tit}{\aut}{\autkol}{\titkol}

\index{Жуков Д.\,О.}
\index{Хватова Т.\,Ю.}
\index{Зальцман А.\,Д.}
\index{Zhukov D.\,O.}
\index{Khvatova T.\,Yu.}
\index{Zaltcman A.\,D.}

%{\renewcommand{\thefootnote}{\fnsymbol{footnote}} \footnotetext[1]
%{Финансовое обеспечение исследований осуществлялось из средств федерального бюджета на 
%выполнение государственного задания Карельского научного центра Российской академии наук 
%(Институт прикладных математических исследований КарНЦ РАН).}}

\renewcommand{\thefootnote}{\arabic{footnote}}
\footnotetext[1]{МИРЭА~--- Российский технологический университет, zhukov\_do@mirea.ru}
\footnotetext[2]{Санкт-Петербургский политехнический университет Петра Великого, \mbox{khvatova.ty@spbstu.ru}}
\footnotetext[3]{МИРЭА~--- Российский технологический университет, ad.zaltcman@gmail.com}


\vspace*{-6pt}



\Abst{Ообсуждаются вопросы использования подходов теоретической информатики 
и~применение ее приложений для анализа и~моделирования процессов в~социотехнических 
системах (социальных сетях). Разработана стохастическая модель динамики изменения 
состояний (настроений или мнений) пользователей (узлов) и~достижения порога перколяции 
в~социальной сети, имеющей случайные связи между узлами. Модель показывает 
возможность скачкообразных переходов между состояниями (мнений, настроений и~т.\,д.)\ 
узлов в~социальной сетевой структуре в~течение короткого времени без внешнего 
воздействия, что может быть связано с~памятью о предыдущих состояниях 
и~самоорганизацией. При создании модели были рассмотрены схемы вероятностей 
переходов между возможными состояниями узлов с~учетом предыдущих шагов 
(немарковские процессы с~наличием памяти) и~выведено нелинейное дифференциальное 
уравнение второго порядка, которое содержит член, отвечающий за возможность 
самоорганизации, а также сформулирована и~решена граничная задача для определения 
функции плотности вероятности нахождения системы в~определенном состоянии с~течением 
времени. Разработанная модель может быть связана с~полученными ее авторами ранее 
результатами описания процессов в~социальных сетевых структурах с~помощью теории 
перколяции (определение времени достижения пороговых значений доли узлов сети, при 
котором мнения или предпочтения могут беспрепятственно распространяться по сети 
в~целом).}

\KW{стохастическая динамика; состояния узлов социальной сети; самоорганизация систем; 
процессы с~памятью; перколяция в~социальных сетях}

\DOI{10.14357/19922264210114}

\vspace*{-4pt}

\vskip 10pt plus 9pt minus 6pt

\thispagestyle{headings}

\begin{multicols}{2}

\label{st\stat}

\section{Введение}

  Отличительной особенностью динамики явлений в~социотехнических 
и~социальных системах является активное воздействие на них человеческого 
фактора, который, с~одной стороны, вносит неопределенность и~создает 
стохастичность, а~с~другой стороны, создает возможности для 
самоорганизации, позволяет говорить о наличии памяти и~придает динамике 
процессов существенно нелинейный характер.
  
  Для моделирования нелинейной динамики самоорганизующихся социальных 
систем с~памятью можно и~нужно применять методы и~средства теоретической 
информатики и~кибернетики, которая, по определению Роберта Винера, 
является наукой об управлении не только техническими, но и~биологическими 
системами.
  
  Использование методов теоретической информатики, разработанных в~ней 
принципов моделирования и~ее приложений может позволить получить 
качественно новые результаты для описания сложных социальных, 
экономических и~социотехнических систем, а также создать новые методики 
прогнозирования поведения людей в~социальных и~социотехнических системах.

%\vspace*{-6pt}
  
\section{Обзор некоторых моделей описания динамики процессов 
в~социальных сетевых структурах}

%\vspace*{-2pt}

  Многие из существующих теоретических подходов к~описанию социальных 
сетевых систем имеют много общего с~кинетическим описанием физических 
систем и~распространением вирусов в~компьютерных сетях. Однако эти модели 
в~основном рассматривают цепные явления, 
где макроскопическая доля узлов с~определенным состоянием в~сети быстро возникает из некоторого 
микроскопического состояния, захватывающего все новые и~новые узлы.
  
  В более сложных моделях взаимодействие пользователей социальных сетей 
описывается теорией многоагентных систем~[1--3], а также аппаратом теории 
клеточных автоматов~[4, 5]. В этих моделях на основании некоторых правил 
переходов агенты принимают определенные состояния, образуют связанную по 
своим свойствам группу, могут сотрудничать, чтобы решить некую задачу или 
достигнуть определенной цели~\cite{1-hv}, а~временн$\acute{\mbox{а}}$я логика поведения 
агентов может зависеть от динамически меняющихся условий~\cite{2-hv}.
  
  В работе~\cite{4-hv} было изучено влияние структуры сетей (случайные 
структуры, маленькие миры, цикл, колесо, звезда, иерархическая) и~правил 
поведения клеточных автоматов на динамику процессов в~социальных сетях. 
При одинаковых правилах взаимодействия клеток динамика процессов зависит 
от топологии сети (наибольшая скорость наблюдается в~регулярных 
структурах, а наименьшая~--- в~неупорядоченных).
  
  Для описания процессов в~социальных сетях также применяются 
стохастические подходы, учитывающие зависимости изменения состояния 
узлов от времени. В~работе~[6] описана модель смешанного членства 
в~стохастически формирующихся группах, основанная на попарном 
рассмотрении присутствия или отсутствия связей между объектами. Анализ 
вероятностных изменений связей требует специальных предположений, 
например независимости или предположения непостоянства данной связи 
(смешанного членства в~стохастически формирующихся группах). Данная 
модель позволяет описать динамику кластеризации членов по группам 
и~изменение их численности.
  
  Другим направлением анализа и~прогнозирования динамики процессов 
в~сложных социальных системах является использование нестационарных 
временн$\acute{\mbox{ы}}$х рядов. Традиционный подход к~их анализу основан на том, чтобы 
с~помощью применения линейных методов свести их к~стационарным 
(например, авторегрессионные интегрированные модели скользящего 
среднего~--- ARIMA, autoregressive integrated moving average~[7]). Эти модели оперируют не функциями 
распределения, а~непосредственно элементами временн$\acute{\mbox{о}}$го ряда. Ряды, не 
укла\-ды\-ва\-ющи\-еся в~рамки регрессионного анализа, чаще\linebreak всего изучаются 
адаптивными эвристическими методами, в~которых ряды на некоторой длине 
описываются стационарной моделью типа регрессии или авторегрессии, 
а~параметры модели пересчитываются с~учетом новой информации или 
с~учетом сравнения предсказанного значения с~фактом. Недостаток этих 
подходов заключается в~том, что длина участка возможной стационарности не 
является известной величиной. При исследовании стационарных случайных 
процессов, согласно теореме Гливенко~[8] (о~сходимости эмпирической 
вероятности к~теоретическому распределению), чем больше учтено 
наблюдаемых значений, тем точнее будут получены теоретические 
характеристики распределения. Для нестационарных временн$\acute{\mbox{ы}}$х рядов данное 
условие, в~силу их специфики, не может быть выполнено, что затрудняет 
возможности прогнозирования.
  
  Следует отметить, что ни одна из существующих моделей не рассматривает 
самоорганизацию и~возможность наличия памяти. Поэтому можно сделать 
вывод о том, что для прогнозирования динамики процессов в~социотехнических 
системах, имеющих сетевую структуру, требуется продолжение разработки их 
моделей с~учетом структурных свойств, самоорганизации и~наличия памяти.

\vspace*{-3pt}
  
\section{Постановка задач исследования}

   С~позиций структурного подхода социотехнические системы представляют 
собой случайную сеть взаимосвязей и~взаимодействий пользователей, которая 
приводит к~нелинейной динамике изменения состояний узлов. При 
моделировании нелинейных динамических процессов в~социальных сетевых 
структурах необходимо ответить минимум на два важных вопроса.  
Во-пер\-вых, как учесть стохастичность, неопределенность, самоорганизацию 
процессов и~наличие памяти и~как они влияют на наблюдаемые явления.  
Во-вто\-рых, как структура сетей влияет на динамику процессов и~как она 
может быть связана со стохастичностью, неопределенностью, 
самоорганизацией процессов и~наличием памяти. Ответы на эти вопросы могут 
позволить создать эффективные алгоритмы мониторинга состояния социальных 
сетей.
   
  В~теории перколяции (теория вероятностей на графах) изучают решение 
задач узлов и~связей для сетей с~различной структурой. При решении задачи 
связей определяют долю связей, которую нужно разорвать, чтобы сеть 
распалась минимум на две несвязанные части (или, наоборот, долю 
проводящих связей в~сети, когда в~целом между любыми произвольными 
узлами появляется проводимость). В~задаче узлов определяют среднюю долю 
блокированных узлов, при которой сеть распадется на не связанные между 
собой кластеры, внутри которых сохраняются связи (или, наоборот, долю 
проводящих узлов, когда проводимость возникает). Доля блокированных узлов 
(в~задаче узлов) или разорванных связей (в~задаче связей), при которой 
исчезает проводимость (или, наоборот, появляется) между двумя произвольно 
выбранными узлами сети, называется порогом перколяции (протекания)~[9].
  
  Использование понятия долей блокированных узлов или связей эквивалентно 
понятию вероятности нахождения случайно выбранного узла (или связи) 
в~блокированном (разорванном) состоянии. Поэтому величина порога 
перколяции определяет вероятность передачи информации через всю сеть 
в~целом, если задана средняя вероятность блокирования узла или связи.
  
  Величину порога перколяции для случайной сетевой структуры можно 
определить теоретически методами численного моделирования или 
экспериментально при изучении реальных сетей \mbox{найти} с~по\-мощью 
инструментов социального сетевого анализа (SNA~--- social network analysis).
  
  Если для случайной сети социальных связей порог ее перколяции известен, 
то, описав механизмы перехода ее узлов в~блокированное или проводящее 
состояние, можно определить время его достижения, а~следовательно, 
спрогнозировать динамику распространения определенных мнений или 
взглядов.
  
\section{Перколяционные свойства случайных сетевых структур}

  Исследования~\cite{1-hv, 10-hv, 11-hv, 12-hv, 13-hv, 14-hv} показывают, что 
пороги перколяции случайных сетей зависят от среднего числа связей в~расчете 
на один узел (плотности) сети. Для задачи связей имеется линейная 
зависимость: $y\hm= -6{,}581z\hm- 0{,}203$; а~для задачи узлов: $y\hm= 4{,}39 
z\hm- 2{,}41$. Здесь $z\hm=1/x,$ где~$x$~--- плот\-ность связей; $y$~--- натуральный 
логарифм доли разорванных связей (или узлов в~задаче узлов), при которой 
исчезает проводимость всей сети  
в~целом~\cite{11-hv, 12-hv, 13-hv, 14-hv, 15-hv, 16-hv, 17-hv}.
  
  Полученные ранее  
результаты~~\cite{11-hv, 12-hv, 13-hv, 14-hv, 15-hv, 16-hv, 17-hv} 
о~перколяционных свойствах случайных сетей позволяют сделать ряд очень 
важных выводов. Например, о наличии насыщения порога перколяции, о роли 
увеличения плотности связей в~информационном влиянии сети и~ряд других. 
Следует отметить, что динамика изменения состояния узлов сетей 
в~совокупности с~их перколяционными свойствами была рассмотрена 
в~работе~\cite{10-hv}, где исследовалось распространение компьютерных 
вирусов. Однако влияние процессов самоорганизации и~наличия памяти на 
динамику изменения состояния узлов и~достижение порогов перколяции 
с~течением времени исследовано не было.
  
\section{Стохастическая динамика переходов между состояниями 
в~сетях социальных связей и~достижение порога перколяции 
с~учетом памяти и~самоорганизации}

\subsection{Построение разностных вероятностных схем переходов 
между~состояниями} %5.1

  Будем описывать социальную сеть как систему, состояния которой в~любой 
момент времени могут быть заданы параметром, принимающим непрерывные 
случайные значения с~недетерминированным законом распределения. 
Например, это может быть доля пользователей (узлов сети), раз\-де\-ля\-ющих 
и~пропагандирующих определенные взгляды или настроения.
  
  Все множество состояний будем обозначать как~$X$. Состояние, 
наблюдаемое в~момент времени~$t$, можно обозначить как~$x_i$ ($x_i\hm\in  
{X}$).
  
  Введем интервал времени~$\tau_0$, за который возможно изменение 
состояния~$x_i$. В~данном случае любое значение текущего времени 
$t\hm=h\tau_0$, где~$h$~--- номер шага перехода между состояниями 
(процесс перехода между состояниями становится квазинепрерывным 
с~бесконечно малым временным интервалом~$\tau_0$); $h\hm=0, 1, 2, \ldots, 
N$. Текущее состояние~$x_i$ на шаге~$h$ после перехода на шаг $h\hm+1$ 
может \textit{за счет случайно возникающих факторов} увеличиваться на 
некоторую величину~$\varepsilon$ или уменьшаться на величину~$\xi$, т.\,е.\ 
оказаться равным $x_i\hm+\varepsilon$ или $x_i\hm- \xi$ 
соответственно. Введем понятие вероятности нахождения системы в~том или 
ином состоянии: после некоторого числа шагов~$h$ про описываемую систему 
можно сказать, что ${\sf P}(x\hm- \varepsilon, h)$~--- вероятность того, что она 
находится в~состоянии $(x\hm - \varepsilon)$; ${\sf P}(x,h)$~--- вероятность того, что 
она находится в~состоянии~$x$; ${\sf P}(x\hm+\xi, h)$~--- вероятность того, что она 
находится в~состоянии $(x\hm + \xi)$.
  
  После каждого шага состояние~$x_i$ (далее индекс~$i$ для краткости 
опустим) может изменяться на величину~$\varepsilon$ или~$\xi$. Вероятность 
${\sf P}(x, h\hm+1)$ того, что на следующем, $(h\hm+1)$-м, шаге система (или 
процесс) окажется в~состоянии~$x$ описывается уравнением (рис.~1):
  \begin{equation}
  {\sf P}(x, h+1)= {\sf P}(x - \varepsilon, h)+{\sf P}(x+\xi, h) - {\sf P}(x, h)\,.
  \label{e1-hv}
  \end{equation}



  Поясним уравнение~(1) и~представленную на рис.~1 схему. Вероятность 
перехода в~состояние~$x$ на\linebreak\vspace*{-12pt}

{ \begin{center}  %fig1
 \vspace*{-1pt}
    \mbox{%
\epsfxsize=77.502mm
\epsfbox{hva-1.eps}
}

\end{center}

\noindent
{{\figurename~1}\ \ \small{
Схема возможных переходов между состояниями системы (или процесса) на ($h + 1$)-м шаге
}}}


\vspace*{14pt}

\addtocounter{figure}{1}

\noindent
 ($h\hm+1$)-м шаге ${\sf P}(x, h\hm+1)$ определяется 
суммой вероятностей перехода в~это состояние из состояний ($x\hm-
\varepsilon$): ${\sf P}(x-\varepsilon, h)$ и~$(x+\xi)$: ${\sf P}(x+\xi, h)$, в~которых 
находилась система на шаге~$h$ за вычетом вероятности ${\sf P}(x, h)$ перехода 
системы из состояния~$x$ (в~котором она находилась на шаге~$h$) в~любое 
другое состояние на $(h\hm+1)$-м шаге. В~реальности в~социальной сети 
всегда остается память о предыдущих состояниях. Для учета этого определим 
вероятности ${\sf P}(x\hm- \varepsilon,h)$, ${\sf P}(x\hm+\xi, h)$ и~${\sf P}(x, h)$ через 
состояния на $(h\hm-1)$-м шаге. Схемы соответствующих переходов можно 
изобразить аналогично схеме, представленной на рис.~1, и~получить для 
вероятности перехода следующее алгебраическое уравнение:
  \begin{multline*}
  {\sf P}(x,h+2)={}\\
  {}=\{ {\sf P}(x-2\varepsilon, h)+{\sf P}(x-\varepsilon+\xi, h) -{\sf P}(x-\varepsilon, 
h)\}+{}\\
  {}+ \{ {\sf P}(x+\xi-\varepsilon,h)+ {\sf P}(x+\xi,h) -{\sf P}(x+\xi,h)\} -{}\\
  {}-{\sf P}(x-\varepsilon,h)- {\sf P}(x+\xi, h-1)+ {\sf P}(x,h)\,.
 % \label{e2-hv}
\end{multline*}
Далее, учитывая, что $t\hm=h\tau_0$, перейдем от~$h$ к~$t$, а~затем проведем 
соответствующие разложения в~ряд Тейлора и~получим:
\begin{equation}
\fr{d{\sf P}(x,t)}{dt} =a\fr{ d^2{\sf P}(x,t)}{dx^2} -b\fr{ d{\sf P}(x,t)}{dx} -c\fr{ 
d^2{\sf P}(x,t)}{dt^2}\,,
\label{e3-hv}
\end{equation}
где 
$$
a=\fr{\varepsilon^2-\varepsilon\xi+\xi^2}{\tau_0}\,;\enskip b=\fr{\varepsilon -
\xi}{\tau_0}\,;\enskip c-\tau_0\,.
$$
  
  Член уравнения $d{\sf P}(x,t)/dx$ описывает упорядоченный переход либо 
в~состояние, когда оно увеличивается ($\varepsilon\hm > \xi$), либо когда оно 
уменьшается ($\varepsilon \hm <\xi$); член $d^2{\sf P}(x,t)/dx^2$ описывает 
случайное изменение состояния (неопределенность изменения). Член 
$d{\sf P}(x,t)/dt$ определяет скорость общего изменения состояния системы 
с~течением времени; член $d^2{\sf P}(x,t)/dt^2$ описывает процесс, при котором 
состояния сами становятся источниками возникновения других состояний 
(\textit{самоорганизация} и~ускорение упорядоченных и~случайных переходов).
  
  Сравним полученный результат с~су\-щест\-ву\-ющи\-ми моделями анализа 
и~описания поведения нестационарных временн$\acute{\mbox{ы}}$х рядов. В~настоящее время 
в~качестве аппроксимаций выборочных распределений чаще всего 
используются диффузионные уравнения, включая нелинейную 
диффузию~\cite{18-hv}:
  $$
  \fr{\partial \rho(x,t)}{\partial t} =\fr{D(t)\partial^2\rho^{(n-1)/(n+1)}(x,t)}{\partial x^2}\,,
  $$
  где $n$~--- числовой параметр модели; $(n\hm-1)/(n\hm+1)$~--- показатель 
степени функции плотности распределения~$\rho(x,t)$ (это уравнение 
учитывает только случайные переходы); уравнение Лиувилля~\cite{18-hv}: 
  $$
  \fr{\partial \rho(x,t)}{\partial t}=-\fr{\partial \{ U(x,t)\rho(x,t)\}}{\partial x}\,,
  $$
  которое определяет упорядоченный перенос; уравнение  
Фок\-ке\-ра--План\-ка~\cite{19-hv}: 
  $$
  \fr{\partial\rho(x,t)}{\partial t}=\fr{D(t)}{2}\,\fr{\partial^2\rho(x,t)}{\partial 
x^2} - \fr{\partial \{ U(x,t)\rho(x,t)\}}{\partial x}\,,
  $$
  где $U(x,t)$~--- скорость <<сноса>>; $D(t)$~--- коэффициент диффузии (это 
уравнение учитывает не только случайное изменение (член 
$\partial^2\rho(x,t)/\partial x^2$), но и~упорядоченные переходы (член $\partial 
\{ U(x,t)\rho(x,t)\}/\partial x$), или <<снос>>) и~ряд других. Однако ни одна из 
этих моделей не рассматривает самоорганизацию и~память. Разработанная 
авторами модель обобщает другие модели, и~при равенстве нулю некоторых 
коэффициентов в~уравнении~(\ref{e3-hv}) оно переходит в~уравнения 
Лиувилля или  
Фок\-ке\-ра--План\-ка, которые выступают ее частными случаями.
  
\subsection{Формулировка и~решение краевой~задачи}

  Считая функцию~${\sf P}(x,t)$ непрерывной, можно перейти от вероятности 
${\sf P}(x,t)$ (уравнение~(\ref{e3-hv})) к~плотности вероятности $\rho(x,t)\hm= 
=d{\sf P}(x,t)/dx$ и~сформулировать граничную задачу, решение которой и~будет 
описывать процесс перехода между состояниями. Предположим, что 
необходимо, чтобы доля пользователей (узлов) социальной сети, имеющих 
негативное мнение, не превышала определенного значения (т.\,е.\ величина 
доли негативных настроений должна находиться на отрезке от~0 до величины 
порога перколяции~$l$ для данной сети).
  
  \textbf{Первое граничное условие}. Состояние $x\hm=0$ определяет полное 
отсутствие негативных мнений (доля равна~0). Сама вероятность обнаружить 
такое состояние может быть отлична от~0, однако плотность вероятности, 
определяющую поток в~состоянии $x\hm=0$, необходимо положить равной~0 
(состояния системы не могут быть отрицательными): $\rho(x,t)_{x=0}\hm=0$.
  
  \textbf{Второе граничное условие}. Вероятность обнаружить состояние 
с~максимально возможной долей негативно настроенных пользователей 
$x\hm=L\hm=1$ отлична от~0. Однако плотность вероятности, определяющая 
поток в~этом состоянии, необходимо положить равной~0 (величина состояния 
не может быть больше, чем максимально возможная доля): 
$\rho(x,t)_{x=L}\hm=0$.
  
  Поскольку в~момент времени $t\hm=0$ состояние системы уже может быть 
равно некоторому значению~$x_0$, то начальное условие зададим в~виде:
  \begin{multline*}
  \rho(x,t=0)=\delta(x-x_0)= {}\\
  {}=\begin{cases} 
  \displaystyle \int \delta(x-x_0)\,dx=1\,, &x=x_0\,;\\
  0\,, & x\not= x_0\,.
  \end{cases}
  \end{multline*}
  
  Второе начальное условие можно задать в~виде: 
  $$
  \fr{\partial G(x,t)}{\partial t}\Big\vert_{t=0}=0\,,
  $$ 
  так как начальное условие содержит дель\-та-функ\-цию; кроме того, ее 
наличие приводит к~тому, что решение для $\rho(x,t)$ разбивается на две 
области при $x\hm> x_0$ и~при $x\hm\leq x_0$. Используя методы 
операционного исчисления для плотности вероятности $\rho_1(x,t)$ 
и~$\rho_2(x,t)$ обнаружения состояния системы в~одном из значений на отрезке 
от~0 до~$L$, можно получить следующую систему уравнений:
  \begin{multline*}
  \mbox{при } x\geq x_0:\ \rho_1(x,t)= -\fr{2}{L}e^{-t/(2\tau_0)} e^{k(x-x_0)}\times{}\\
  {}\times \sum\limits^\infty_{n=1} \fr{\sin ( \pi 
n x_0/L) \sin (\pi n (L-x)/L)}{\cos(\pi n)}\times{}\\
{}\times \mathrm{ch}\left( 
\fr{t}{\tau_0}\sqrt{\fr{k\varepsilon\xi}{2(\varepsilon-\xi)}- \fr{\pi^2 n^2 
(\varepsilon-\xi)}{2kL^2}}\right)\,;
  \end{multline*}
  
  \vspace*{-12pt}
  
  \noindent
  \begin{multline*}
  \mbox{при } x < x_0:\ \rho_2(x,t)= -\fr{2}{L}e^{-t/(2\tau_0)} e^{k(x-x_0)} \times{}\\
  {}\times \sum\limits^\infty_{n=1} \fr{\sin (\pi 
n (L-x_0)/L)\sin (\pi n x/L)}{\cos (\pi n)}\times{}\\
{}\times \mathrm{ch} 
\left(\fr{t}{\tau_0}\sqrt{\fr{k\varepsilon\xi}{2(\varepsilon-\xi)}- \fr{\pi^2 n^2 
(\varepsilon-\xi)}{2kL^2}}\right)\,,
  \end{multline*}
где 
$$
k=\fr{\varepsilon -\xi}{2(\varepsilon^2 -\varepsilon\xi +\xi^2)}\,.
$$ 
Если вычислить интеграл 
\begin{equation}
{\sf P}(l,t)=\int\limits_0^{x_0} \rho_2(x,t)\,dx+\int\limits^l_{x_0}\rho_1(x,t)\,dx\,,
\label{e4-hv}
\end{equation}
то функция ${\sf P}(l,t)$ будет задавать вероятность того, что состояние системы 
к~моменту времени~$t$ будет находиться на отрезке от~0 до~$l$, т.\,е.\ порог 
перколяции~$l$ не будет достигнут. Соответственно, вероятность~$Q_i(t)$ того, 
что порог перколяции~$l$ окажется к~моменту времени~$t$ достигнутым или 
превзойденным, будет равна:
\begin{equation}
Q(l,t)=1-{\sf P}(l,t)\,.
\label{e5-hv}
\end{equation}

\vspace*{-12pt}

\subsection{Моделирование динамики и~самоорганизации состояний узлов~социальной сети}

  При анализе модели необходимо задать приемлемые величины значений 
порогов перколяции случайной сети. Плотность связей можно определить 
экспериментально, а затем, используя зависимости величины порогов 
перколяции от среднего числа связей, приходящегося на один  
узел~~\cite{11-hv, 12-hv, 13-hv, 14-hv, 15-hv, 16-hv, 17-hv}, рассчитать их 
допустимые величины (см.\ уравнения в~разд.~4).
  
  Для моделирования примем, что начальная доля~$x_0$ негативных мнений 
равна~5\% ($x_0\hm=0{,}05$), величину~$\tau_0$ примем равной одной 
условной единице времени ($\tau_0\hm =1$), $\varepsilon\hm=0{,}02$ (2\%) 
и~$\xi\hm=0{,}01$ (1\%). Результаты моделирования времени достижения 
порога перколяции~(\ref{e5-hv}) с~использованием~(\ref{e4-hv}) при заданном 
выше в~качестве примера наборе параметров модели представлены 
в~графическом виде на рис.~2.  Кривые~\textit{1} и~\textit{2} на рис.~2 показывают, что чем 
ближе значение величины начального со\-сто\-яния сис\-те\-мы~$x_0$ в~момент 
времени $t\hm=0$ к~пороговому значению, тем быстрее возрастает вероятность 
перехода и~тем сильнее вероятность его достижения приближается к~единице. 
Кривая~\textit{4} на рис.~2 показывает, что при большой разности между 
величиной порогового значения и~$x_0$ вероятность его достижения имеет 
осциллирующий характер, при этом она сначала снижается с~течением времени, 
а затем показывает рост, причем чем дальше значение величины~$x_0$ от 
порогового значения, тем сильнее проявляются осцилляции. Кроме того, 
существует и~отличная от нуля вероятность достижения порогового значения 
при $t\hm=0$ (мгновенная реализация).
  

\setcounter{figure}{1}
\begin{figure*} %fig2
\vspace*{1pt}
\begin{center}
\mbox{%
\epsfxsize=163mm
\epsfbox{hva-2.eps}
}
\end{center}
\vspace*{-14pt}
\begin{minipage}[t]{79mm}
\Caption{Графическое представление результатов моделирования преодоления порогов 
перколяции для распространения негативных мнений в~социальной сети: \textit{1}~--- $l\hm=0{,}1$; \textit{2}~--- 0,15;
\textit{3}~--- 0,20; \textit{4}~--- $l\hm=0{,}25$}
\end{minipage}
\hfill
%\end{figure*}
%\begin{figure*} %fig3
%\vspace*{1pt}
%\begin{center}
%\mbox{%
%\epsfxsize=77.502mm
%%\epsfbox{hva-3.eps}
%}
%\end{center}
\vspace*{-14pt}
\begin{minipage}[t]{79mm}
\Caption{Графическое представление результатов моделирования преодоления порогов 
перколяции для распространения негативных мнений в~социальной сети при $\varepsilon\hm=\xi\hm= 0{,}02$:
\textit{1}~--- $l\hm=0{,}1$; \textit{2}~--- 0,15; \textit{3}~--- 0,20; \textit{4}~--- $l\hm=0{,}25$}
\end{minipage}
\vspace*{6pt}
\end{figure*}

  Рост вероятности перехода через пороговое значение имеет ступенчатый 
характер, а~про\-тя\-жен\-ность ступени во времени зависит от того, насколько 
начальная величина со\-сто\-яния сис\-те\-мы~$x_0$
 \mbox{близка} к~пороговому значению. 
Процесс достижения порогового значения имеет протяженное во времени 
плато, величина которого (в~единицах ве\-ро\-ят\-ности) зависит от~$x_0$.
  
  Увеличение значения величин ~$\varepsilon$ и~$\xi$ (при выполнении 
условия $\varepsilon \hm >\xi$) изменяет величину плато (горизонтальный 
участок зависимости ве\-ро\-ят\-ности перехода через пороговое значение до 
\mbox{второго} участка резкого рос\-та) на рис.~2, однако общая за\-ви\-си\-мость 
ве\-ро\-ят\-ности перехода от времени качественно не изменяется.
  
  Ход кривых на рис.~2 показывает возможность роста вероятности 
достижения порогового значения состояния системы практически сразу после 
начала процесса. Вероятность перехода через пороговое значение отлична от 
нуля уже после первого шага и~нелинейно возрастает с~течением времени. Это 
следствие того, что не только величины~$\varepsilon$ и~$\xi$ определяют 
изменение состояния~$x$, но и~сами состояния~$x$ служат источником изменения 
вследствие наличия памяти о предыдущих состояниях и~самоорганизации, за 
которую отвечает в~дифференциальном уравнении член $d^2 {\sf P}(x,t)/dt^2$. 
Арифметический расчет показывает, что число шагов (обозначим его 
как~$q_0$), за которое можно достичь порогового значения~$l$, должно быть 
не меньше чем $q\hm=(l\hm - x_0)/(\varepsilon \hm - \xi)$. Например, для 
пороговых значений состояния системы $l\hm=0{,}1$ и~0,2 при ее начальном 
состоянии $x_0\hm=0{,}05$, $\varepsilon\hm=0{,}02$ и~$\xi\hm=0{,}01$ 
для~$q$ получим соответственно~5 и~15. Результаты (см.\ рис.~2) показывают, 
что это не так, т.\,е.\ происходит самоорганизация.
  
  При равенстве~$\varepsilon$ и~$\xi$ (например, 
$\varepsilon\hm=\xi\hm=0{,}02$)\linebreak характер хода кривых, описывающих 
вероятность достижения пороговых значений, изменяется (рис.~3).  
В~част\-ности, не наблюдается протяженного во времени плато 
с~последующим плавным ростом вероятности достижения пороговых значений 
до единицы, а рост вероятности имеет характер резкого скачка. Это связано 
с~тем, что коэффициент~$b$ в~уравнении~(\ref{e3-hv}) окажется равен нулю 
и~упорядоченные переходы будут невозможны, а член $d^2\rho(x,t)/dt^2$ будет 
ускорять только случайные переходы $d^2\rho(x,t)/dx^2$.

\vspace*{-4pt}

\section{Алгоритм мониторинга состояний социальной сетевой~структуры}

\vspace*{-2pt}

  Разработанная модель позволяет создать практически реализуемый алгоритм 
мониторинга состояния социальной сети.\\[-12pt]
\begin{enumerate}[1.]
\item Определяем с~помощью социологического мониторинга плотность сети 
и~долю узлов~$x_0$ с~определенным мнением или настроением (состояние 
узла) в~данный момент времени $t\hm=0$.\\[-15pt]
\item Спустя одну выбранную условную единицу времени $\tau\hm=1$ 
(например, одна неделя) снова находим долю узлов с~определенным мнением 
или настроением в~данный момент времени~$x_1$ ($t\hm=0\hm +\tau$). 
Находим величину $\varepsilon\hm=x_1 \hm- x_0$, а~величину~$\xi$ считаем 
равной~0. Если $\varepsilon\hm<0$, то считаем $\xi\hm= x_1 \hm- x_0$, 
а~$\varepsilon\hm=0$.\\[-15pt]
\item На основании данных о~среднем числе связей рассчитываем порог 
перколяции данной сети. Используя уравнения~(\ref{e4-hv}) и~(\ref{e5-hv}), по 
определенным в~пп.~1 и~2 значениям параметров $x_0$, $\varepsilon$, 
$\xi$ и~порогу перколяции~$l$ моделируем поведение от условного времени 
вероятности перехода поро-\linebreak\vspace*{-12pt}
\end{enumerate}

\begin{enumerate}
\item[\,]
га перколяции и~определяем допустимый лимит 
времени для изменения ситуации.
\end{enumerate}

\vspace*{-6pt}

\section{Заключение}

  Создана новая модель описания динамики изменения состояния узлов 
и~перколяционных переходов в~социальных сетях с~учетом самоорганизации 
и~наличия памяти, которая вносит \mbox{значительный} вклад в~развитие тео\-рии 
управления сложными системами.
  
  Результаты анализа разработанной модели могут быть связаны 
с~полученными ранее результатами описания процессов в~социальных сетевых 
структурах с~помощью теории перколяции (это необходимо для определения 
времени достижения пороговых значений доли узлов социальной сети, когда 
определенные мнения или предпочтения могут беспрепятственно 
распространяться по сети в~целом).
  
  Полученные результаты существенно отличаются от применяемых 
в~настоящее время моделей для описания нестационарных процессов на 
основе тео\-рии хаоса, диффузионных подходов, уравнений Лиувилля  
и~Фок\-ке\-ра--План\-ка. Все это в~целом представляется абсолютно новым 
и~оригинальным, а также вносит вклад в~развитие теории управления 
сложными системами.
  
  Разработанная модель во взаимосвязи с~методами теории перколяции 
существенно расширяет возможности применения уравнений математической 
физики и~теоретической информатики для моделирования социальных систем.

\vspace*{-6pt}
  
{\small\frenchspacing
{\baselineskip=10.85pt
%\addcontentsline{toc}{section}{References}
\begin{thebibliography}{99}
\bibitem{1-hv}
\Au{Gasser L.} Social conceptions of knowledge and action: DAI foundations and open system 
semantics~// Artif. Intell., 1991. Vol.~47. No.\,1-3. P.~107--138.
\bibitem{2-hv}
\Au{Jennings N.\,R., Faratin~P., Lomuscio~A.\,R., Parsons~S., Sierra~C., Wooldridge~M.} 
Automated negotiation: Prospects, methods and challenges~// Group Decis.
Negot., 2001. Vol.~10. No.\,2. P.~199--215.
\bibitem{3-hv}
\Au{Plikynas D., Raudys~A., Raudys~S.} Agent-based modelling of excitation propagation in 
social media groups~// J.~Exp. Theor. Artif. In., 2015. Vol.~27. No.\,4. 
P.~373--388.

\bibitem{5-hv} %4
\Au{Hay J., Flynn D.} The effect of network structure on individual behavior~// Complex Systems, 
2014. Vol.~23. No.\,4. P.~295--311.

\bibitem{4-hv} %5
\Au{Hay J., Flynn D.} How external environment and internal structure change the behavior of 
discrete systems~// Complex Systems, 2016. Vol.~25. No.\,1. P.~39--49.

\bibitem{6-hv}
\Au{Airoldi~E.\,M., Blei~D.\,M., Fienberg~S.\,E., Xing~E.\,P.} Mixed membership stochastic 
blockmodels~// J.~Mach. Learn. Res, 2008. Vol.~9. P.~1981--2014.
\bibitem{7-hv}
\Au{Бокс Дж., Дженкинс~Г.} Анализ временных рядов. Прогноз и~управление~/ Пер. 
с~англ.~--- М.: Мир, 1974. 553~с. (\Au{Box~G.\,E.\,P., Jenkins~G.\,M.} Time series analysis: 
Forecasting and control.~--- Holden-day, 1970. 553~p.)
\bibitem{8-hv}
\Au{Гнеденко Б.\,В.} Курс теории вероятностей.~--- М.: Физматлит, 1961. 406~с.
\bibitem{9-hv}
\Au{Grimmet G.\,R.} Percolation.~--- New York, NY, USA: Springer-Verlag, 1989. 296~p.

\bibitem{13-hv} %10
\Au{Zhukov D., Lesko S.} Percolation models of information dissemination in social networks~// 
IEEE Conference (International) on Smart City/SocialCom/SustainCom 
Together with DataCom Proceedings.~--- 
IEEE, 2015. P.~213--216.
\bibitem{14-hv} %11
\Au{Khvatova T., Block~M., Zhukov~D., Lesko~S.} Studying the structural topology of the 
knowledge sharing network~// 11th European Conference on Management Leadership and 
Governance Proceedings.~--- Lisbon, Portugal: Academic Conferences and Publishing 
Itnternational Ltd., 2015. P.~20--27.

\bibitem{12-hv} %12
\Au{Khvatova T.\,Yu., Zaltsman~A.\,D., Zhukov~D.\,O.} Information processes in social networks: 
Percolation and stochastic dynamics~// CEUR Workshop Procee., 2017. Vol.~2064. 
P.~277--288.

\bibitem{11-hv} %13
\Au{Zhukov D., Khvatova~T., Lesko~S., Zaltsman~A.} Managing social networks: Applying 
percolation theory methodology to understand individuals' attitudes and moods~// Technol. 
Forecast. Soc., 2018. Vol.~129. P.~297--307.


\bibitem{10-hv} %14
\Au{Лесько С.\,А., Алёшкин~.\,С., Филатов~В.\,В.} Стохастические и~перколяционные 
модели динамики блокировки вычислительных сетей при распространении эпидемий 
эволюционирующих компьютерных вирусов~// Российский технологический~ж., 2019. 
Т.~7. №\,3(29). С.~7--27.

\bibitem{16-hv} %15
\Au{Khvatova T., Block~M., Zhukov~D., Lesko~S.} How to measure trust: The percolation model 
applied to intraorganisational knowledge sharing networks~// J.~Knowl. Manag., 2016. Vol.~20. 
No.\,5. P.~918--935.

\bibitem{15-hv} %16
\Au{Жуков Д.\,О., Хватова~Т.\,Ю., Лесько~С.\,А., Зальцман~А.\,Д.} Влияние плотности 
связей на кластеризацию и~порог перколяции при распространении информации 
в~социальных сетях~// Информатика и~её применения, 2018. Т.~12. Вып.~2. С.~90--97.


\bibitem{17-hv}
\Au{Zhukov D.\,O., Khvatova~T.\,Y., Millar~C., Zaltcman~A.} Modelling the stochastic dynamics 
of transitions between states in social systems incorporating self-organisation and memory~// 
Technol. Forecast. Soc., 2020. Vol.~158. Art. ID: 120134.
\bibitem{18-hv}
\Au{Орлов Ю.\,Н., Федоров~С.\,Л.} Генерация нестационарных траекторий временного ряда 
на основе уравнения Фок\-ке\-ра--План\-ка~// Труды МФТИ, 2016. Т.~8. №\,2(30).  
С.~126--133.
\bibitem{19-hv}
\Au{Fuentes M.} Non-linear diffusion and power law properties of heterogeneous systems: 
Application to financial time series~// Entropy, 2018. Vol.~20. Iss.~9. Art. No.~649.
\end{thebibliography}

}
}

\end{multicols}

\vspace*{-10pt}

\hfill{\small\textit{Поступила в~редакцию 18.06.2019}}

%\vspace*{8pt}

%\pagebreak

\newpage

\vspace*{-28pt}

%\hrule

%\vspace*{2pt}

%\hrule

%\vspace*{-2pt}

\def\tit{MODELING OF~THE~STOCHASTIC DYNAMICS OF~CHANGES IN~NODE STATES AND~PERCOLATION 
TRANSITIONS IN~SOCIAL NETWORKS WITH~SELF-ORGANIZATION AND~MEMORY}

\def\titkol{Modeling of the~stochastic dynamics of~changes in~node states and~percolation 
transitions in~social networks} % with~self-organization and~memory}

\def\aut{D.\,O.~Zhukov$^1$, T.\,Yu.~Khvatova$^2$, and~A.\,D.~Zaltcman$^1$}

\def\autkol{D.\,O.~Zhukov, T.\,Yu.~Khvatova, and~A.\,D.~Zaltcman}

\titel{\tit}{\aut}{\autkol}{\titkol}

\vspace*{-11pt}


\noindent
$^1$Russian Technological University (MIREA), 78~Vernadskogo Ave., Moscow 119454, Russian Federation

\noindent
$^2$Peter the Great St.\ Petersburg Polytechnic University, 29~Polytechnicheskaya Str., St.\ Petersburg 195251, 
Russian\linebreak
$\hphantom{^1}$Federation

\def\leftfootline{\small{\textbf{\thepage}
\hfill INFORMATIKA I EE PRIMENENIYA~--- INFORMATICS AND
APPLICATIONS\ \ \ 2021\ \ \ volume~15\ \ \ issue\ 1}
}%
\def\rightfootline{\small{INFORMATIKA I EE PRIMENENIYA~---
INFORMATICS AND APPLICATIONS\ \ \ 2021\ \ \ volume~15\ \ \ issue\ 1
\hfill \textbf{\thepage}}}

\vspace*{3pt}





\Abste{This paper explores the use of theoretical informatics applied for analyzing and modeling 
the processes in sociotechnical systems (social networks). A~stochastic model of users' (network nodes) dynamic changes of states (opinions 
or moods) and the percolation threshold in a~social network with random connections among nodes was developed. 
This model demonstrates the opportunity for jump-like transitions in states (opinions, moods) of the nodes in a~social 
network over a short period of time without external influence. While developing the model, the probabilistic schemes 
of state-to-state transitions of nodes (users having certain opinions and views) were considered; 
a~second-order  
nonlinear differential equation was derived; the boundary for calculating the probability density function for a~system 
being in a certain state depending on the time interval was formulated. The differential equation of the model contains 
a~member representing the opportunity for self-organization; it also considers the presence of memory. The results of 
analysis of the stochastic model support those previously obtained by the authors when investigating social network 
processes using the percolation theory. This theory was used for defining the time of reaching the threshold values for 
the share of social network nodes when certain opinions or preferences can spread freely within the whole social 
network.}

\KWE{stochastic dynamics; states of social network nodes; system self-organization; processes involving memory; 
percolation in social networks}

\DOI{10.14357/19922264210114}

%\vspace*{-15pt}

%\Ack
%\noindent

%\vspace*{6pt}

  \begin{multicols}{2}

\renewcommand{\bibname}{\protect\rmfamily References}
%\renewcommand{\bibname}{\large\protect\rm References}

{\small\frenchspacing
 {%\baselineskip=10.8pt
 \addcontentsline{toc}{section}{References}
 \begin{thebibliography}{99}

\bibitem{1-hv-1}
\Aue{Gasser, L.} 1991. Social conceptions of knowledge and action: DAI foundations and open 
system semantics. \textit{Artif. Intell.} 47(1-3):107--138.
\bibitem{2-hv-1}
\Aue{Jennings, N.\,R., P.~Faratin, A.\,R.~Lomuscio, S.~Parsons, C.~Sierra, and M.~Wooldridge}. 
2001. Automated negotiation: Prospects, methods and challenges. \textit{Group Decis. 
Negot.} 10(2):199--215.
\bibitem{3-hv-1}
\Aue{Plikynas, D., A.~Raudys, and S.~Raudys.} 2015. Agent-based modelling of excitation 
propagation in social media groups. \textit{J.~Exp. Theor. Artif. In.} 
27(4):373--388.

\bibitem{5-hv-1}
\Aue{Hay, J., and D.~Flynn.} 2014. The effect of network structure on individual behavior. 
\textit{Complex Systems} 23(4):295--311.

\bibitem{4-hv-1}
\Aue{Hay, J., and D.~Flynn.} 2016. How external environment and internal structure change the 
behavior of discrete systems. \textit{Complex Systems} 25(1):39--49.

\bibitem{6-hv-1}
\Aue{Airoldi, E.\,M., D.\,M.~Blei, S.\,E.~Fienberg, and E.\,P.~Xing.} 2008. Mixed membership 
stochastic blockmodels. \textit{J.~Mach. Learn. Res.} 9:1981--2014.
\bibitem{7-hv-1}
\Aue{Box, G.\,E.\,P., and G.\,M.~Jenkins.} 1970. \textit{Time series analysis: Forecasting and 
control.} Holden-day. 553~p.
\bibitem{8-hv-1}
\Aue{Gnedenko, B.\,V.} 1961. \textit{Kurs teorii veroyatnostey} [Probability theory]. Moscow: Fizmatlit. 406~p.
\bibitem{9-hv-1}
\Aue{Grimmet, G.\,R.} 1989. \textit{Percolation}. New York, NY: Springer-Verlag. 296~p.

\bibitem{13-hv-1} %10
\Aue{Lesko, S.\,A., and D.\,O.~Zhukov.} 2015. Percolation models of information dissemination in 
social networks. \textit{IEEE Conference (International) on Smart City/SocialCom/\linebreak SustainCom 
Together with DataCom Proceedings}. IEEE. 213--216.
\bibitem{14-hv-1} %11
\Aue{Block, M., T.~Khvatova, D.~Zhukov, and S.~Lesko.} 2015. Studying the structural topology 
of the knowledge sharing network. \textit{11th European Conference on Management, Leadership 
and Governance Proceedings}. Lisbon, Portugal: Academic Conferences and Publishing 
International Ltd. 20--27.

\bibitem{12-hv-1} %12
\Aue{Khvatova, T.\,Yu., A.\,D.~Zaltcman, and D.\,O.~Zhukov.} 2017. Information processes in 
social networks: Percolation and stochastic dynamics. CEUR Workshop Procee.
2064:277--288.

\bibitem{11-hv-1} %13
\Aue{Zhukov, D., T.~Khvatova, S.~Lesko, and A.~Zaltcman.} 2018. Managing social networks: 
Applying Percolation theory methodology to understand individuals' attitudes and moods. 
\textit{Technol. Forecast. Soc.} 129:297--307.


\bibitem{10-hv-1} %14
\Aue{Lesko, S.\,A., A.\,S.~Alyoshkin, and V.\,V.~Filatov.} 2019. Stokhasticheskie 
i~perkolyatsionnye modeli dinamiki blokirovki vychislitel'nykh setey pri rasprostranenii epidemiy 
evolyutsioniruyushchikh komp'yuternykh virusov [Stochastic and percolating models of blocking 
computer networks dynamics during distribution of epidemics of evolutionary computer viruses]. 
\textit{Rossiyskiy tekh\-no\-lo\-gi\-che\-skiy zh.} [Russian Technological~J.] 7(3):7--27.


\bibitem{16-hv-1} %15
\Aue{Khvatova, T., M.~Block, D.~Zhukov, and S.~Lesko.} 2016. How to measure trust: The 
percolation model applied to intraorganisational knowledge sharing networks. \textit{J.~Knowl. 
Manag.} 20 (5):918--935.

\bibitem{15-hv-1} %16
\Aue{Zhukov, D.\,O., T.\,Yu.~Khvatova, S.\,A.~Les'ko, and A.\,D.~Zal'tsman.} 2018. Vliyanie 
plotnosti svyazey na klasterizatsiyu i~porog perkolyatsii pri rasprostranenii informatsii 
v~sotsial'nykh setyakh [The influence of the connections density on clusterization and percolation 
threshold during information distribution in social networks]. \textit{Informatika i~ee 
Primeneniya~---Inform. Appl.} 12(2):90--97.

\bibitem{17-hv-1}
\Aue{Zhukov, D.\,O., T.\,Y.~Khvatova, C.~Millar, and A.~Zaltcman.} 2020. Modelling the 
stochastic dynamics of transitions between states in social systems incorporating self-organisation 
and memory. \textit{Technol. Forecast. Soc.} 158:120134.
\bibitem{18-hv-1}
\Aue{Orlov, Yu.\,N., and S.\,L.~Fedorov.} 2016. Generatsiya ne\-sta\-tsio\-nar\-nykh traektoriy 
vremennogo ryada na osnove uravneniya Fokkera--Planka [Generating nonstationary trajectories 
of a time series based on Fokker--Plank equation]. \textit{Trudy MFTI} [MIPT Proceedings~J.] 
8(2):126--133.
\bibitem{19-hv-1}
\Aue{Fuentes, M.} 2018. Non-linear diffusion and power law properties of heterogeneous systems: 
Application to financial time series. \textit{Entropy} 20(9):649. 8~p.
\end{thebibliography}

 }
 }

\end{multicols}

\vspace*{-3pt}

  \hfill{\small\textit{Received June~18, 2019}}


%\pagebreak

%\vspace*{-8pt}

\Contr

\noindent
\textbf{Zhukov Dmitry O.} (b.\ 1965)~--- 
Doctor of Science in technology, professor, Head of Department, Russian Technological 
University (MIREA), 78~Vernadskogo Ave., Moscow 119454, Russian Federation; 
\mbox{zhukov\_do@mirea.ru}

\vspace*{6pt}

\noindent
\textbf{Khvatova Tatiana Yu.} (b.\ 1971)~--- Doctor of Science in economics, professor, Peter the 
Great St.\ Petersburg Polytechnic University, 29~Polytechnicheskaya Str., St.\ Petersburg 195251, 
Russian Federation; \mbox{khvatova.ty@spbstu.ru}

\vspace*{6pt}

\noindent
\textbf{Zaltcman Anastasia D.} (b.\ 1989)~--- lecturer, Russian Technological University 
(MIREA), 78~Vernadskogo Ave., Moscow 119454, Russian Federation; 
\mbox{ad.zaltcman@gmail.com}

\label{end\stat}

\renewcommand{\bibname}{\protect\rm Литература}       %14
\def\stat{grinchenko}

\def\tit{О ГЕНЕЗИСЕ ИНФОРМАЦИОННОГО ОБЩЕСТВА:  
ИНФОРМАТИКО-КИБЕРНЕТИЧЕСКОЕ МОДЕЛЬНОЕ ПРЕДСТАВЛЕНИЕ}

\def\titkol{О генезисе информационного общества:  
информатико-кибернетическое модельное представление}

\def\aut{С.\,Н.~Гринченко$^1$}

\def\autkol{С.\,Н.~Гринченко}

\titel{\tit}{\aut}{\autkol}{\titkol}

\index{Гринченко С.\,Н.}
\index{Grinchenko S.\,N.}


%{\renewcommand{\thefootnote}{\fnsymbol{footnote}} \footnotetext[1]
%{Работа выполнена при частичной финансовой 
%поддержке РФФИ (проект 17-07-00577).}}


\renewcommand{\thefootnote}{\arabic{footnote}}
\footnotetext[1]{Институт проблем информатики Федерального исследовательского центра <<Информатика и~управление>> 
Российской академии наук, \mbox{sgrinchenko@ipiran.ru}}

\vspace*{-3.5pt}




  \Abst{Вводится понятие <<генезис информационного общества>>, которое рассматривается 
  с~позиций ин\-фор\-ма\-ти\-ко-ки\-бер\-не\-ти\-че\-ско\-го моделирования (ИКМ)
  процесса развития 
Человечества как са\-мо\-управ\-ля\-ющей\-ся иерар\-хо-се\-те\-вой системы. На этой основе 
получены количественные оценки его типовых про\-стран\-ст\-вен\-но-вре\-мен\-ных характеристик, 
представляющих собой геометрические прогрессии со знаменателем 
<<$e$~в~степени~$e$>> (15,15426$\ldots$), а~также скоординированных с~ними во времени 
и~в~пространстве пси\-хи\-ко-ант\-ро\-по\-ло\-ги\-че\-ских, образовательных  
и~ин\-фор\-ма\-ци\-он\-но-ком\-му\-ни\-ка\-ци\-он\-ных параметров и~возможностей 
включенного в~этот процесс усложняющегося человека и~его сообществ различной 
величины. Это позволило раздвинуть рамки существования информационного общества на 
всю историческую и~даже археологическую эпоху такого развития. Результирующая 
последовательность информационных технологий (ИТ) <<сигнальные  
по\-зы/зву\-ки/дви\-же\-ния\,--\,ми\-ми\-ка/жес\-ты\,--\,речь/язык\,--\,пись\-мен\-ность\,--\,ти\-ра\-жи\-ро\-ва\-ние текстов\,--\,компью\-те\-ры\,--\,те\-ле\-ком\-му\-ни\-ка\-ции\,--\,ин\-фор\-ма\-ци\-он\-ная на\-но\-тех\-но\-ло\-гия\,--\,$\ldots$>> 
позволяет рас\-смат\-ри\-вать генезис 
информационного общества в~широком контексте единой исторической ретроспективы 
и~перспективы.}
  
  \KW{информационное общество; информационные технологии;  
ин\-фор\-ма\-ти\-ко-ки\-бер\-не\-ти\-че\-ская модель; самоуправляющаяся 
 иерар\-хо-се\-те\-вая система Человечества; археологическая эпоха}
 
 \DOI{10.14357/19922264190214}
  
%\vspace*{4pt}


\vskip 10pt plus 9pt minus 6pt

\thispagestyle{headings}

\begin{multicols}{2}

\label{st\stat}
  
  В~литературе, даже энциклопедической, распространена трактовка 
<<информационного общества>> как общества <<современного типа>>, 
в~котором общение людей опирается на компьютерные 
и~телекоммуникационные ИТ\footnote[2]{В~[1] дано следующее определение:
<<\textbf{Информационное общество}, одно из понятий, используемых 
в~социологич.\ теории для обозначения обществ.\ систем <<современного типа>>$\ldots$ 
Важнейшие характеристики~И.\,о.: 1)~лавинообразное распространение информац. 
технологий (прежде всего компьютерных и~телекоммуникационных); 2)~превращение 
информации в~важнейший социальный ресурс, необходимую предпосылку управленч. 
деятельности, развития экономики, образования, сферы услуг, домашнего быта, 
рекреационной сферы и~т.\,д.; по некоторым данным, в~наиболее развитых странах проф. 
деятельность более половины занятых связана исключительно с~производством и~обработкой 
информации; 3)~наделение СМИ статусом <<четвертой ветви власти>>; 4)~расширение 
границ и~<<репертуара>> массовой культуры; 5)~увеличение каналов вертикальной 
и~горизонтальной мобильности; 6)~изменение представлений о~социальном пространстве 
(<<глобализация>> пространства, мгновенная доступность даже периферийных его 
сегментов) и~времени (расширение рамок <<современности>>, когда даже отдаленные 
историч. события воспринимаются как происходящие <<здесь>> и~<<сейчас>>); 
7)~возникновение в~процессе коммуникации особой виртуальной реальности, несводимой 
к~результатам технич. визуализации и~выходящей за пределы воображения и~памяти 
индивида; 8)~превращение информац. технологий в~базу для развития высоких технологий 
(Hi-Tech)>>.}. Такая трактовка этого понятия создает иллюзию 
отстраненности информационного общества от его собственного исторического 
прошлого, когда вышеперечисленных ИТ еще не изобрели, но люди в~составе 
сообществ как-то общались между собой, используя иные ИТ. 

Поскольку от 
этой иллюзии недалеко до недооценки полезности соответствующего 
исторического опыта для современности, попытаюсь развеять ее.
  
Результаты ИКМ процесса развития на Земле 
Человечества как самоуправляющейся ие\-рар\-хо-се\-те\-вой\footnote[3]{<<\textbf{Иерархо-сетевая}>> 
структура~--- иерархическая структура типа <<матрешки>>, но с~существенно большим 
единицы числом вложений на каждом ее иерархическом уровне, которые и~образуют 
соответствующие сетевые структуры.} системы~[2--14] (рис.~1) позволяют раздвинуть рамки 
существования информационного\linebreak общества на всю историческую и~даже археологическую эпоху такого 
развития, что дает возможность выделить ту эволюционную линию этого процесса, которую логично 
определить как \textit{генезис информационного общества}. 


\begin{figure*} %fig1
   \vspace*{1pt}
    \begin{center}  
  \mbox{%
 \epsfxsize=130.287mm 
 \epsfbox{gri-1.eps}
 }
\end{center}
%\vspace*{-9pt}
%\Caption{Схема иерархо-сетевой самоуправляющейся (по алгоритмам случайной поисковой 
%оптимизации целевых критериев энергетического характера с~ограничениями типа 
%равенств и~неравенств) системы лич\-ност\-но-про\-из\-вод\-ст\-вен\-но-со\-ци\-аль\-ной природы 
%(Человечества)~\cite{5-grn}}
\end{figure*}


На рис.~1 используются следующие обозначения:
\begin{itemize}
\item восходящие стрелки (имеющие структуру <<мно\-гие\,--\,к~од\-но\-му>>) 
отражают первую из~5~основных со\-став\-ля\-ющих контура поисковой 
оптимизации~--- \textit{поисковую активность} представителей 
соответствующих ярусов в~иерархии; 
\item нисходящие сплошные (имеющие 
структуру <<один\,--\,ко мно\-гим>>) стрелки отражают вторую 
со\-став\-ля\-ющую~--- \textit{целевые критерии} поисковой оптимизации 
энергетики системы Человечества; 
\item нисходящие пунктирные (<<один\,--\,ко 
многим>>) стрелки отражают третью со\-став\-ля\-ющую~--- 
\textit{оптимизационную системную память}  
лич\-ност\-но-про\-из\-вод\-ст\-вен\-но-со\-ци\-аль\-но\-го (результат 
адаптивных влияний представителей вышележащих иерархических ярусов на 
структуру вложенных в~них нижележащих); 
\item полужирными стрелками 
в~левой части схемы условно показана четвертая со\-став\-ля\-ющая~--- 
\textit{антропогенная ак\-тив\-ность} индивидов и~их групп, трак\-ту\-емая как 
<<трудовая деятельность по созданию со\-от\-вет\-ст\-ву\-юще\-го инструментария 
и~результатов его применения>>; 
\item пунктирными полужирными стрелками 
в~правой части схемы условно показана пятая со\-став\-ля\-ющая~--- 
\textit{антропогенная системная\linebreak память}  
лич\-ност\-но-про\-из\-вод\-ст\-вен\-но-со\-ци\-ального (процессы вовлечения 
результатов антропогенной активности в~структуру со\-от\-вет\-ст\-ву\-ющей  
иерар\-хо-се\-те\-вой под\-сис\-те\-мы Человечества).
\end{itemize}

Рассмотрим этот феномен поэтапно, сведя в~общую таблицу расчетные данные 
о~различных его проявлениях. 
       


\begin{table*}\footnotesize
\begin{center}
\Caption{Свод основных характеристик генезиса информационного общества (как 
проявления развития са\-мо\-управ\-ля\-ющей\-ся и~метаэволюционирующей, т.\,е.\ 
наращивающей чис\-ло своих иерархических уров\-ней/яру\-сов, сис\-те\-мы Человечества) от 
прошлого до модельно прогнозируемого будущего}
\vspace*{2ex}

\tabcolsep=1.5pt
\begin{tabular}{|c|c|l|c|c|c|c|}
\hline
&\tabcolsep=0pt\begin{tabular}{c}Характерный\\ ареал (радиус\\
 круга той же\\ площади); точность\\ антропогенного\\ 
воздействия\,/\\
производственных\\ технологий\end{tabular}&
\tabcolsep=0pt\begin{tabular}{c}Характерные\\ времена\\ старта;\\ кульминации\\ 
развития\\ подсистемы\end{tabular}&
\tabcolsep=0pt\begin{tabular}{c}Уровень\\ развития\\ Homo\\  
(и его пред-\\ шествен-\\ ников)\end{tabular}&
\tabcolsep=0pt\begin{tabular}{c}Носитель системной\\ памяти~---\\ субстрат психики\end{tabular}&
\tabcolsep=0pt\begin{tabular}{c}Лидирующая\\ ИТ\end{tabular}&
\tabcolsep=0pt\begin{tabular}{c}Требуемый уровень\\ образованности Homo;\\
аналогия филогенеза\\ и~онтогенеза:\\ примерный возраст\\ гармонично\\ образовываемого\\ 
(сегодня)\end{tabular}\\
\hline
1&2&\multicolumn{1}{c|}{3}&4&5&6&7\\
\hline
0&$\sim4{,}2$~м&\tabcolsep=0pt\begin{tabular}{c} $\sim428$~млн\\ лет назад;\\
$\sim 140{,}1$~млн\\ лет назад\end{tabular}&
\tabcolsep=0pt\begin{tabular}{c}Цефализация\\ позвоночных\end{tabular}&
\tabcolsep=0pt\begin{tabular}{c}Многоклеточный\\организм в~целом\end{tabular}&
\tabcolsep=0pt\begin{tabular}{c}Формирование\\ головного\\ мозга как основы\\
 для реализации\\ 
будущих ИТ\end{tabular}&\tabcolsep=0pt\begin{tabular}{c} ---\\
$\sim0{,}6$--1,0~год\end{tabular}\\
\hline
1&\tabcolsep=0pt\begin{tabular}{c} $\sim64$~м;\\
$\sim28$~см
\end{tabular}&\tabcolsep=0pt\begin{tabular}{c}$\sim28{,}23$~млн\\ лет назад;\\
$\sim9{,}26$~млн\\ лет назад
\end{tabular}&\tabcolsep=0pt\begin{tabular}{c}Пред-пред-\\
люди\\ Hominoidea\end{tabular}&
\tabcolsep=0pt\begin{tabular}{c}Органы многоклеточного\\ организма (его 
нервной\\ системы в~целом)\end{tabular}&
\tabcolsep=0pt\begin{tabular}{c}Сигнальные позы/\\
движения\\ и~неинтонированные\\ звуки (типа 
рычания,\\ ворчания, писка\\ и~т.\,п.)\end{tabular}&
\tabcolsep=0pt\begin{tabular}{c}Выработка\\ 
(младенцами)\\ сигнальных поз;\\
$\sim1{,}0$--1,6~лет \end{tabular}\\
\hline
2&\tabcolsep=0pt\begin{tabular}{c} $\sim1$~км;\\
$\sim1{,}8$~см\end{tabular}&\tabcolsep=0pt\begin{tabular}{c} $\sim1{,}86$~млн\\ лет 
назад;\\
$\sim612$~тыс.\\ лет назад\end{tabular}&
\tabcolsep=0pt\begin{tabular}{c}Пред-люди\\ Homo ergaster\,/\\
Homo erectus\end{tabular}&
\tabcolsep=0pt\begin{tabular}{c}Ткани 
многоклеточного\\ организма\\ (сетей/ансамблей\\ нейронов и~др.)\end{tabular}&
\tabcolsep=0pt\begin{tabular}{c}Мимика/жесты\\ 
и~интонированные\\ звуки\end{tabular}&
\tabcolsep=0pt\begin{tabular}{c}Овладение (ре-\\ бенком) мимикой/\\ 
жестами,\\
начальное\\ понимание речи; \\ $\sim1{,}6$--2,6~лет \end{tabular}\\
\hline
3&\tabcolsep=0pt\begin{tabular}{c} $\sim15$~км; \\
$\sim1{,}2$~мм
\end{tabular}&
\tabcolsep=0pt\begin{tabular}{c} $\sim123$~тыс.\\ лет назад;\\
$\sim40$~тыс.\\ лет назад\end{tabular}&
\tabcolsep=0pt\begin{tabular}{c}Homo\\ sapiens$^\prime$\end{tabular}&
\tabcolsep=0pt\begin{tabular}{c}Эвкариотические\\ клетки\\ 
многоклеточного\\ организма\\ (отдельные нервные\\ и~глиальные клетки\\ и~др.)\end{tabular}&
\tabcolsep=0pt\begin{tabular}{c}Речь/язык\\ 
(артикулированная\\ устная речь)\end{tabular}&
\tabcolsep=0pt\begin{tabular}{c}Овладение (детьми)\\ 
речью/языком\\ (протообразование); \\ $\sim2{,}6$--4,2~лет \end{tabular}\\
\hline
4&\tabcolsep=0pt\begin{tabular}{c} $\sim222$~км;\\
$\sim 80$~мкм
\end{tabular}&\tabcolsep=0pt\begin{tabular}{c}$\sim8{,}1$~тыс.\\ лет назад;\\
$\sim2{,}7$~тыс.\\ лет назад\end{tabular}&
\tabcolsep=0pt\begin{tabular}{c}Homo\\ sapiens$^{\prime\prime}$\end{tabular}&
\tabcolsep=0pt\begin{tabular}{c}Компартменты\\ 
эвкариотической\\ клетки (отдельные\\ рецепторные,\\ или постсинаптические,\\ зоны нейронов и~т.\,п.)\end{tabular}
&Письменность&\tabcolsep=0pt\begin{tabular}{c}Овладение чтением/ \\ письмом 
(дошкольное\\ образование);\\
$\sim4{,}2$--6,9~лет \end{tabular}\\
\hline
5&\tabcolsep=0pt\begin{tabular}{c}$\sim3370$~км;\\
$\sim5$~мкм
\end{tabular}&\tabcolsep=0pt\begin{tabular}{l}$\sim1446$~г.;\\
$\sim1806$~г.\end{tabular}&
\tabcolsep=0pt\begin{tabular}{c}Homo\\ sapiens$^{\prime\prime\prime}$\end{tabular} &
\tabcolsep=0pt\begin{tabular}{c}Субкомпартменты\\ эвкариотической 
клетки\end{tabular}&
\tabcolsep=0pt\begin{tabular}{c}Тиражирование\\ текстов,\\ или книгопечатание\end{tabular}&
\tabcolsep=0pt\begin{tabular}{c}Начальное\\ образование;\\ 
$\sim6{,}9$--11,1~лет \end{tabular}\\
\hline
6&\tabcolsep=0pt\begin{tabular}{c} $\sim51$~тыс.\ км\\ (общепланетарный);\\
$\sim0{,}35$~мкм\end{tabular}&\tabcolsep=0pt\begin{tabular}{l} $\sim1946$~г.;\\
$\sim 1970$~г.\end{tabular}&\tabcolsep=0pt\begin{tabular}{c}Homo \\
sapiens$^{\prime\prime\prime\prime}$\end{tabular}&
\tabcolsep=0pt\begin{tabular}{c}Ультраструктурные\\ (прокариотические)\\ 
внутриклеточные элементы\\ эвкариотической клетки\\ (типа клеточного ядра,\\ деталей 
эндоплазматической\\ сети и~т.\,п.\ образований)\end{tabular}&
Компьютерная ИТ&\tabcolsep=0pt\begin{tabular}{c}Среднее\\ образование;\\
$\sim11{,}1$--18~лет \end{tabular}\\
\hline
7&\tabcolsep=0pt\begin{tabular}{c} $\sim773$~тыс.\ км\\ (ближний\\ космос);\\
$\sim23$~нм\end{tabular}&\tabcolsep=0pt\begin{tabular}{l} $\sim1979$~г.;\\
$\sim2003$~г.\end{tabular}&\tabcolsep=0pt\begin{tabular}{c}Homo\\
 sapiens$^{\prime\prime\prime\prime\prime}$\end{tabular} 
&
\tabcolsep=0pt\begin{tabular}{c}Макромолекулы/гены\\ (компартменты\\ 
ультраструктурных--\\
прокариотических--\\
внутриклеточных\\ элементов)\end{tabular}&
\tabcolsep=0pt\begin{tabular}{c}Телекоммуника-\\ ционная ИТ\end{tabular}&\tabcolsep=0pt\begin{tabular}{c}Высшее обра-\\
зование\;+\;<<аспи-\\ рантура>>; \\
$\sim18$--29,1~лет \end{tabular}\\
\hline
8&\tabcolsep=0pt\begin{tabular}{c}
$\sim11{,}7$~млн км\\ (промежуточный\\ космос);\\
$\sim1{,}5$~нм\end{tabular}&\tabcolsep=0pt\begin{tabular}{l} $\sim1981$~г.;\\ 
$\sim2341$~г.~(?)\end{tabular}&\tabcolsep=0pt\begin{tabular}{c}Homo\\ 
sapiens$^{\prime\prime\prime\prime\prime\prime}$\end{tabular}&
\tabcolsep=0pt\begin{tabular}{c}Органические молекулы \\
(субкомпартменты\\ ультраструктурных--
\\прокариотических--
\\внутриклеточных \\
элементов)\end{tabular}&
\tabcolsep=0pt\begin{tabular}{c}Нано-ИТ (возможно,\\
 <<наноаппаратно\\ поддерживаемая\\ селективная\\ телепатия>>~\cite{16-grn})\end{tabular}&
 \tabcolsep=0pt\begin{tabular}{c}<<Докторантура>>; \\ 
$\sim29{,}1$--47,1~лет \end{tabular}\\
\hline
9&$\cdots$&\multicolumn{1}{c|}{$\cdots$}&$\cdots$&$\cdots$&$\cdots$&$\cdots$\\
\hline
\end{tabular}
\end{center}
\end{table*}




  Промежутки времени между возникновением новых ие\-рар\-хо-се\-те\-вых 
подсистем Человечества (а~следовательно, и~между стартами новых ИТ) 
подчиняются, согласно ИКМ, простой математической за\-ко\-но\-мер\-ности: 
каж\-дый из них в~$e^e\hm= 15{,}15426$\ldots раз короче 
предыдущего\footnote{Эту геометрическую прогрессию~--- как модель критических 
уровней развития биологических сис\-тем~--- выявили А.\,В.~Жирмунский 
и~В.\,И.~Кузьмин~\cite{17-grn}.} (третий\linebreak
 столбец таблицы). В~свою очередь, этой 
же закономерности подчиняются и~размеры ареалов\linebreak
 (радиусы кругов той же 
площади) устойчивых и~эффективно са\-мо\-управ\-ля\-ющих\-ся сообществ 
человека как базисного элемента сис\-те\-мы Человечества, и~точ\-ности 
доступных услож\-ня\-юще\-му\-ся человеку~--- в~конкретный момент 
исторического времени~--- антропогенных воздействий и/или 
производственных технологий (второй столбец таб\-ли\-цы) (рис.~2).
  
  Эмпирические оценки этих времен и~пространств, сделанные 
и~опуб\-ли\-ко\-ван\-ные палео\-ант\-ро\-по\-ло\-га\-ми, археологами и~историками,~--- 
когда они имеются!~--- не противоречат модельным  
результатам~\cite{14-grn}.
  %
Диапазоны примерного возраста <<образовываемых>>, приведенные 
в~седьмом столб\-це таб\-ли\-цы, рассчитаны, исходя из <<золотого сечения>> 
(соотношения смеж\-ных членов чис\-ло\-во\-го ряда, равного 1,618$\ldots$ при 
увеличении ряда, либо 0,618$\ldots$ при его уменьшении, аде\-кват\-ность 
использования которого при выработке количественных оценок в~самых 
различных областях знания хорошо известна\footnote{Применительно 
к~периодизации истории Человечества в~археологическую эпоху это продемонстрировано 
Ю.\,Л.~Щаповой~\cite{18-grn, 19-grn, 20-grn}, согласование подхода к~такой периодизации на 
основе золотого сечения и~пред\-ла\-га\-емо\-го информатико-ки\-бер\-не\-ти\-че\-ско\-го подхода 
подробно показано в~\cite{10-grn, 12-grn, 13-grn, 14-grn, 15-grn, 21-grn}.}), 
опирающегося на ориентировочную оценку завершения человеком среднего 
образования к~18~годам (на сегодня).


  Базируясь на ИКМ, в~качестве нулевого этапа развития будущего 
информационного общества, как пред\-став\-ля\-ет\-ся, можно рас\-смат\-ри\-вать 
процесс \textit{цефализации} позвоночных, т.\,е.\ возникновения 
и~усложнения у~них головного мозга как основного носителя механизмов 
запоминания и~считывания информации о~результатах их адаптивного 
и~социального поведения, начавшейся около 428~млн лет назад 
с~кульминацией около 140,1~млн лет назад (шестой стол\-бец таб\-ли\-цы) на 
<<территории>> порядка 4,2~м~--- т.\,е.\ в~пределах отдельного 
многоклеточного организма.
  

  
  Далее в~качестве первого этапа такого развития будем рассматривать 
начавшуюся около 28,23~млн лет назад, с~кульминацией около 9,26~млн лет 
назад, на территориях порядка 64~м, ИТ сигнальных поз/дви\-же\-ний 
и~неинтонированных звуков (типа рычания, ворчания, писка и~т.\,п.), 
характерную для стад\-ных/стай\-ных животных, в~том числе  
пред-пред-людей {Hominoidea} (четвертый стол\-бец таб\-ли\-цы), 
способных обеспечивать точность своих воздействий на природу порядка~28~см. 
Субстрат их психики относится к~иерархическому уровню органов 
многоклеточного организма (пятый стол\-бец), а~уровень об\-ра\-зо\-ван\-ности 
соответствует современному младенцу возрастом около~1--1,6~лет (седьмой 
столбец).
  
  Следующий, второй этап развития ИТ~--- ми\-ми\-ки/жес\-тов, начавшийся 
около~1,86~млн лет назад, с~кульминацией около~612~тыс.\ лет назад, на 
территориях порядка~1~км, реализовался далекими\linebreak предками современного 
человека~--- пред-людь\-ми {Homo ergaster/Homo erectus}, способными 
обеспечивать точ\-ность своих воздействий на природу\linebreak порядка~1,8~см, 
с~субстратом психики уров\-ня тканей многоклеточного организма и~уровнем 
обра\-зо\-ван\-ности, соответствующим современному ребенку~1,6--2,6~лет.

\pagebreak

\end{multicols}

\setcounter{figure}{1}
\begin{figure*} %fig2
 \vspace*{1pt}
    \begin{center}  
  \mbox{%
 \epsfxsize=163.101mm 
 \epsfbox{gri-2.eps}
 }
\end{center}
\vspace*{-6pt}
\Caption{Пространственно-временн$\acute{\mbox{ы}}$е характеристики и~тренд ИТ в~процессе генезиса 
информационного общества (по ИКМ, в~двойном логарифмическом масштабе; 
иерархическая слож\-ность~--- число уров\-ней/яру\-сов в~системной иерархии)}
\vspace*{1pt}
\end{figure*}

\begin{multicols}{2}



  
  Все последующие этапы развития ИТ~--- речь/язык, пись\-мен\-ность, 
тиражирование текстов (книгопечатание), компьютеры, телекоммуникации, 
на\-но-ИТ~--- реализовались последовательно усложняющимися формами 
{Homo sapiens}, который при этом образовывал относительно 
устойчивые и~относительно эффективно функционирующие 
и~самоуправляющиеся сообщества на все больших ареалах, одновременно 
повышая точность своих (антропогенных) действий при формировании 
вокруг себя <<второй (рукотворной) природы>>.
  
  Так, третий этап развития ИТ~--- речи/языка, начавшийся около 123~тыс.\ 
лет назад, с~кульминацией (верхнепалеолитической революцией) 
около~40~тыс.\ лет назад, на территориях порядка~15~км, реализовался 
{Homo sapiens}$^\prime$, способными обеспечивать точность своих 
производственных технологий порядка~1,2~мм, с~субстратом психики 
уровня эвкариотических клеток многоклеточного организма и~уровнем 
образованности, соответствующим современному ребенку~2,6--4,2~лет.

\begin{figure*}[b] %fig3
%\vspace*{-4pt}
    \begin{center}  
  \mbox{%
 \epsfxsize=162.821mm 
 \epsfbox{gri-3.eps}
 }
\end{center}
\vspace*{-6pt}
\Caption{Тренд изменения времен запаздывания кульминаций развития под\-сис\-тем  
иерар\-хо-се\-те\-вой сис\-те\-мы Человечества относительно их стартов (по ИКМ, в~двойном 
логарифмическом мас\-штабе)}
\end{figure*}
  
  Четвертый этап развития ИТ~--- письменности, начавшийся 
около~8,1~тыс.\ лет назад, с~кульминацией (городской революцией 
<<осевого времени>>) около 2,7~тыс.\ лет назад, на территориях 
порядка~222~км, реализовался {Homo sapiens}$^{\prime\prime}$, 
способными обеспечивать точность своих производственных технологий 
порядка~80~мкм, с~суб\-стра\-том психики уровня компартментов 
эвкариотических клеток многоклеточного организма и~уровнем 
образованности, соответствующим современному ребенку~4,2--6,9~лет 
(дошкольное образование).
  
  Пятый этап развития ИТ~--- тиражирования\linebreak текс\-тов (книгопечатания), 
начавшийся около 1446~г.\ н.\,э., с~кульминацией (промышленной\linebreak 
революцией) около 1806~г., на территориях порядка~3370~км, реализовался 
{Homo sapiens}$^{\prime\prime\prime}$, способными обеспечивать 
точность своих производственных технологий порядка~5~мкм, с~субстратом 
психики уровня субкомпартментов эвкариотических клеток многоклеточного 
организма и~уровнем об\-ра\-зо\-ван\-ности, соответствующим современному 
ребенку~6,9--11,1~лет (начальное образование).
  
  Шестой этап развития ИТ~--- компьютеров (локальных), начавшийся 
около~1946~г., с~кульминацией (изобретением микропроцессоров) 
около~1970~г., на территориях порядка~51~тыс.\ км (т.\,е.\ 
общепланетарного, или глобального размера), реализовался {Homo 
sapiens}$^{\prime\prime\prime\prime}$, способными обеспечивать точ\-ность 
своих производственных технологий порядка~0,35~мкм, с~субстратом 
психики уровня\linebreak
 ультраструктурных (прокариотических) внутриклеточных 
элементов эвкариотической клетки и~уровнем об\-ра\-зо\-ван\-ности, 
соответствующим современному  
под\-рост\-ку-юно\-ше/де\-вуш\-ке~11,1--18~лет\linebreak (среднее образование).
  
  Седьмой этап развития ИТ~--- телекоммуникаций, начавшийся около 
1979~г., с~кульминацией (пиком ско\-рости распространения на планете 
мобильной телефонии, интернета и~т.\,п.) около\linebreak
 2003~г., в~космическом 
объеме радиусом (шара)\linebreak порядка 773~тыс.\ км (т.\,е.\ в~ближнем космосе), 
реализовался {Homo sapiens}$^{\prime\prime\prime\prime\prime}$, 
способными обеспечивать точ\-ность своих производственных технологий 
порядка~23~нм, с~субстратом психики уровня мак\-ро\-мо\-ле\-кул/ге\-нов 
(компартментов\ ульт\-ра\-струк\-тур\-ных--про\-ка\-рио\-ти\-че\-ских--\linebreak
внут\-ри\-кле\-точ\-ных 
элементов эвкариотической клетки) и~уровнем 
об\-ра\-зо\-ван\-ности, со\-от\-вет\-ст\-ву\-ющим современному молодому  
че\-ло\-ве\-ку~18--29,1~лет (высшее обра\-зо\-ва\-ние\;+\;<<ас\-пи\-ран\-ту\-ра, 
с~защитой диссертации кандидата наук>>).
  
  Восьмой этап развития перспективной нано-ИТ (возможно, <<ИТ 
наноаппаратно поддерживаемой селективной телепатии>>~\cite{16-grn}), 
начавшийся около~1981~г., с~кульминацией (пиком скорости ее 
распространения на планете) около~2341~г.\ (расчетная дата), в~космическом 
объеме радиусом шара порядка~11,7~млн км (т.\,е.\ в~промежуточном 
космосе~\cite{5-grn}), реализовался {Homo 
sapiens}$^{\prime\prime\prime\prime\prime\prime }$, способными обеспечивать 
точность своих производственных технологий порядка~1,5~нм (отсюда 
наименование ИТ), с~субстратом психики уровня органических молекул 
(субкомпартментов ульт\-ра\-струк\-тур\-ных--про\-ка\-риоти\-че\-ских--внут\-ри\-кле\-точ\-ных 
элементов эвкариотической клетки) и~уровнем 
об\-ра\-зо\-ван\-ности,\linebreak соответству\-ющим современному зрелому  
человеку~29,1--47,1~лет (<<докторантура>>).
  
  Важно отметить, что процесс появления всех вышеперечисленных 
подсистем подчиняется кумулятивному принципу: возникновение каждой 
новой подсистемы не отменяет существование предыду\-щей: они все активно 
взаимодействуют между собой, коэволюционируют и~т.\,п., но исторически 
более ранние, естественно, постепенно переходят на второй, третий и~т.\,д.\ 
планы исторической сцены.
  
  Точка сходимости этого ряда находится около\linebreak 1981~г., знаменуя собой 
завершение этапа <<детст\-ва--от\-ро\-че\-ст\-ва--юности>> Человечества как 
целого и~начало этапа его <<зрелости>>~--- до\-сти\-же\-ния его максималь\-ной 
иерархической слож\-ности (чис\-ла уров\-ней/яру\-сов в~сис\-тем\-ной 
иерархии)~\cite{5-grn, 7-grn}.
  
  С позиции прогнозирования генезиса информационного общества на 
будущие времена отмечу, что, согласно ИКМ, тренд изменения времен 
запаздывания кульминаций развития под\-сис\-тем относительно их стартов 
сменился прямо на наших глазах. Если во временн$\acute{\mbox{о}}$м диапазоне с~428~млн 
лет назад и~до 1946~г.\ он со\-стоял в~равномерном (в~логарифмическом 
масштабе) укорочении согласно той же за\-ко\-но\-мер\-ности 
(в~$e\hm=15,15426\ldots$~раз), то в~диапазоне от~1946 по 1979~гг.\ это время 
запаздывания не изменилось, а~начиная с~1979~г.\ начало удлиняться 
(рис.~3). 
  

  
  Таким образом, метаэволюция сис\-те\-мы Человечества завершилась около 
1981~г.\ в~том смыс\-ле, что все воз\-мож\-ные ее ие\-рар\-хо-се\-те\-вые под\-сис\-те\-мы 
\textit{в~потенции} уже созданы. Но их \textit{актуализация}, дальнейшее 
услож\-не\-ние, эволюция и~коэволюция с~ранее возникшими аналогичными 
под\-сис\-те\-ма\-ми будет продолжаться неопределенно длительное время.

\vspace*{-10pt}
  
  \section*{Выводы}
  
  \vspace*{-2pt}
  
  \noindent
  \begin{enumerate}[1.]
\item  Изучение \textit{генезиса информационного общества} во всех его 
последовательных формах~--- от древности до современности и~далее~--- на 
базе\linebreak
 ин\-фор\-ма\-ти\-ко-ки\-бер\-не\-ти\-че\-ско\-го модельного подхода 
и~формализации процесса метаэволю\-ционного развития в~соответствующих 
терминах, позволило получить количественные\linebreak оценки его типовых  
про\-стран\-ст\-вен\-но-вре\-менн$\acute{\mbox{ы}}$х характеристик, 
а~также скоординированных с~ними во времени и~в~пространстве  
психико-ант\-ро\-по\-ло\-ги\-че\-ских, образовательных %\linebreak  
и~ин\-фор\-ма\-ци\-он\-но-ком\-му\-ни\-ка\-ци\-он\-ных параметров 
и~возможностей включенного в~этот процесс усложняющегося человека 
и~его сообществ различной величины.
  \item  Позиционирование ИТ локальных компьютеров и~ИТ 
телекоммуникаций в~качестве неотъемлемых составляющих совокупности\linebreak 
монотонно усложняющихся в~ходе цивилизационного развития~--- 
и~информационного общества!~--- ИТ позволяет 
рассматривать их появление и~функционирование в~широком контексте 
единой исторической ретроспективы и~перспективы, давая возможность 
делать не только теоретические, но и~практические выводы.
  \end{enumerate}
  
{\small\frenchspacing
 {%\baselineskip=10.8pt
 \addcontentsline{toc}{section}{References}
 \begin{thebibliography}{99}
\bibitem{1-grn}
\Au{Мелик-Гайгазян И.\,В.} Информационное общество~// Большая российская 
энциклопедия. Т.~11.~--- М.: Большая Российская энциклопедия, 2008. С.~490.
\bibitem{2-grn}
\Au{Гринченко С.\,Н.} Социальная метаэволюция Человечества как последовательность 
шагов формирования механизмов его системной памяти~// Исследовано в~России: 
Электронный журнал, 2001. Т.~145. С.~1652--1681. {\sf  
https://cyberleninka.ru/article/v/sotsialnaya-metaevolyutsiya-chelovechestva-kak-posledovatelnost-shagov-formirovaniya-mehanizmov-ego-sistemnoy-pamyati}.
\bibitem{3-grn}
\Au{Гринченко С.\,Н.} Системная память живого (как основа его метаэволюции
и~периодической структуры).~--- М.: ИПИ РАН, Мир, 2004. 512~с.
\bibitem{4-grn}
\Au{Grinchenko S.\,N.} Meta-evolution of nature system~--- the framework of history~// Social 
Evolution History, 2006. Vol.~5. No.\,1. P.~42--88.
\bibitem{5-grn}
\Au{Гринченко С.\,Н.} Метаэволюция (сис\-тем неживой, живой  
и~со\-ци\-аль\-но-тех\-но\-ло\-ги\-че\-ской природы).~--- М.: ИПИ РАН, 2007. 456~с.
\bibitem{6-grn}
\Au{Гринченко С.\,Н.} Homo eruditus (человек образованный) как элемент сис\-те\-мы 
Человечества~// Открытое образование, 2009. №\,2. С.~48--55.

\bibitem{10-grn} %7
\Au{Гринченко С.\,Н., Щапова~Ю.\,Л.} История Человечества: модели периодизации~// 
Вестник РАН, 2010. №\,12. С.~1076--1084.

%\bibitem{11-grn}  %8
%\Au{Grinchenko S.\,N., Shchapova~Y.\,L.} Human history periodization models~// Herald of the 
%Russian Academy of Sciences, 2010. Vol.~80. No.\,6. P.~498--506.
\bibitem{7-grn} %9
\Au{Grinchenko S.\,N.} The pre- and post-history of Humankind: What is it?~// Problems of 
contemporary world futurology.~--- Newcastle-upon-Tyne: Cambridge Scholars Publishing, 
2011. P.~341--353.
\bibitem{8-grn} %10
\Au{Гринченко С.\,Н.} Об эволюции психики как иерархической сис\-те\-мы 
(кибернетическое пред\-став\-ле\-ние)~// Историческая психология и~социология истории, 
2012. Т.~5. №\,2. С.~60--76.

\bibitem{12-grn} %11
\Au{Гринченко С.\,Н., Щапова~Ю.\,Л.} Информационные технологии в~истории 
Человечества.~--- М.: Новые технологии, 2013. 32~с. (Приложение к~журналу 
<<Информационные технологии>>, 2013. №\,8.)

\bibitem{9-grn} %12
\Au{Гринченко С.\,Н.} Эволюция темпов жизни людей и~развитие человечества~// Человек, 
2014. №\,5. С.~28--36.



\bibitem{13-grn}
\Au{Grinchenko S.\,N., Shchapova~Y.\,L.} Archaeological epoch as the succession of generations 
of evolutive subject-carrier archaeological sub-epoch~// Philosophy of Nature in Cross-Cultural 
Dimensions: The Result of the International Symposium at the University of Vienna~/ 
Komparative Philosophie und Interdisziplin$\ddot{\mbox{a}}$re Bildung (KoPhil). Band~5.~--- 
Hamburg: Verlag Dr.\ Kova$\Check{\mbox{c}}$, 2017. P.~478--499.
\bibitem{14-grn}
\Au{Щапова Ю.\,Л., Гринченко~С.\,Н.} Введение в~теорию археологической эпохи: 
числовое моделирование и~логарифмические шкалы про\-стран\-ст\-вен\-но-вре\-мен\-ных 
координат.~--- М.: Истфак МГУ, ФИЦ ИУ РАН, 2017. 236~с.
\bibitem{15-grn}
\Au{Grinchenko S.\,N., Shchapova~Yu.\,L.} Communications: Model representations about 
historical retrospective and possible perspective~// Communications Media 
Design Electronic~J., 2018. Vol.~3. No.\,2. P.~65--78.
\bibitem{16-grn}
\Au{Гринченко С.\,Н.} Послесловие~// Мат-лы доклада на Совместном научном семинаре 
ИПИ РАН и~\mbox{ИНИОН} РАН <<Методологические проблемы наук об информации>>.~---
М., 2012. С.~5--8. {\sf 
http://legacy.\linebreak inion.ru/files/File/MPNI\_9\_13\_12\_12\_posl.pdf}.
\bibitem{17-grn}
\Au{Жирмунский А.\,В., Кузьмин~В.\,И.} Критические уровни в~процессах развития 
биологических систем.~--- М.: Наука, 1982. 179~с.
\bibitem{18-grn}
\Au{Щапова Ю.\,Л.} Хронология и~периодизации древнейшей истории как числовая 
последовательность (ряд Фибоначчи)~// Информационный бюллетень Ассоциации 
<<История и~компьютер>>, 2000. №\,25.
\bibitem{19-grn}
\Au{Щапова Ю.\,Л.} Археологическая эпоха: хронология, периодизация, теория,  
модель.~--- М.: КомКнига, 2005. 192~с.
\bibitem{20-grn}
\Au{Щапова Ю.\,Л.} Материальное производство в~археологическую эпоху.~--- СПб.: 
Алетейя, 2011. 244~с.
\bibitem{21-grn}
\Au{Гринченко С.\,Н., Щапова~Ю.\,Л.} Пространство и~время в~археологии. Часть~3. 
О~метрике базисной пространственной структуры человечества в~археологическую 
эпоху~// Пространство и~время, 2014. №\,1(15). С.~78--89.
 \end{thebibliography}

 }
 }

\end{multicols}

\vspace*{-8pt}

\hfill{\small\textit{Поступила в~редакцию 17.10.18}}

\vspace*{6pt}

%\pagebreak

%\newpage

%\vspace*{-29pt}

\hrule

\vspace*{2pt}

\hrule

%\vspace*{-2pt}

\def\tit{ON THE GENESIS OF~THE~INFORMATION SOCIETY: INFORMATICS-CYBERNETIC 
MODEL REPRESENTATION}


\def\titkol{On the genesis of~the~information society: Informatics-cybernetic 
model representation}

\def\aut{S.\,N.~Grinchenko}

\def\autkol{S.\,N.~Grinchenko}

\titel{\tit}{\aut}{\autkol}{\titkol}

\vspace*{-11pt}


\noindent
Institute of Informatics Problems of the Federal Research Center ``Informatics and Control'' of 
the Russian Academy of Sciences, 44-2~Vavilov Str., Moscow 119333, Russian Federation

\def\leftfootline{\small{\textbf{\thepage}
\hfill INFORMATIKA I EE PRIMENENIYA~--- INFORMATICS AND
APPLICATIONS\ \ \ 2019\ \ \ volume~13\ \ \ issue\ 2}
}%
 \def\rightfootline{\small{INFORMATIKA I EE PRIMENENIYA~---
INFORMATICS AND APPLICATIONS\ \ \ 2019\ \ \ volume~13\ \ \ issue\ 2
\hfill \textbf{\thepage}}}

\vspace*{6pt}


  
  \Abste{The concept of the information society genesis is introduced, which is 
viewed from the standpoint of informatics-cybernetic modeling of the development 
of Humankind as a self-controlling hierarchical-networking system. On this basis, 
the author obtained quantitative assessments of its typical spatial-temporal 
characteristics, representing geometric progressions with the denominator ``$e$ to the 
degree~$e$'' (15.15426$\ldots$), as well as coordinated with them in time and space 
of the psychoanthropological, educational, and informational communication 
parameters and possibilities of a person who becomes complicated in this process 
and his communities of various sizes. This allowed us to push the framework of 
the information society for the entire historical and even archaeological epoch of 
such development. The resulting sequence of information technologies ``signal 
poses\,/\,sounds/movements\,--\,mimics/gestures\,--\,speech/language\,--\,writing\,--\,replicating 
texts\,--\,computers\,--\,telecommunications\,--\,information 
nanotechnology\,--\,$\ldots$'' allows us to consider the information society genesis 
in the broad context of a unified historical retrospective and perspective.}
  
  \KWE{information society; information technologies; informatics-cybernetic 
model; self-controlling hierarchical-networking system of Humankind; 
archaeological epoch}
  

\DOI{10.14357/19922264190214}

%\vspace*{-14pt}

%\Ack
%\noindent



%\vspace*{6pt}

  \begin{multicols}{2}

\renewcommand{\bibname}{\protect\rmfamily References}
%\renewcommand{\bibname}{\large\protect\rm References}

{\small\frenchspacing
 {%\baselineskip=10.8pt
 \addcontentsline{toc}{section}{References}
 \begin{thebibliography}{99}

\bibitem{1-grn-1}
\Aue{Melik-Gaygazyan, I.\,V.} 2008. Informatsionnoe ob\-shche\-st\-vo [Information 
society]. \textit{Bol'shaya rossiyskaya entsiklopediya} [Great Russian 
Encyclopedia].  Moscow: Great Russian 
Encyclopedia Publs. 11:490.
\bibitem{2-grn-1}
\Aue{Grinchenko, S.\,N.} 2001. Sotsial'naya me\-ta\-evo\-lyu\-tsiya Chelovechestva kak 
posledovatel'nost' shagov for\-mi\-ro\-va\-niya mekhanizmov ego sistemnoy pamyati 
[Social meta-evolution of Mankind as a~sequence of steps for the formation of the 
mechanisms of its system memory]. \textit{Elektronnyy zhurnal <<Issledovano 
v~Rossii>>} [Electronical J. ``Invstigated in Russia'']. 145:1652--1681. Avalable 
at: {\sf  
https://cyberleninka.ru/article/v/sotsialnaya-metaevolyutsiya-chelovechestva-kak-posledovatelnost-shagov-formirovaniya-mehanizmov-ego-sistemnoy-pamyati} (accessed 
October~5, 2018).
\bibitem{3-grn-1}
\Aue{Grinchenko, S.\,N.} 2004. \textit{Sistemnaya pamyat' zhivogo (kak osnova 
ego metaevolyutsii i~periodicheskoy struktury)} [System memory of the life (as the 
basis of its meta-evolution and periodic structure)]. Moscow: IPIRAN, MIR. 
512~p.
\bibitem{4-grn-1}
\Aue{Grinchenko, S.\,N.} 2006. Meta-evolution of nature system~--- the 
framework of history. \textit{Social Evolution History} 5(1):42--88.
\bibitem{5-grn-1}
\Aue{Grinchenko, S.\,N.} 2007. \textit{Metaevolyutsiya (sistem nezhivoy, zhivoy 
i~sotsial'no-tekhnologicheskoy prirody)} [Meta-evolution (of inanimate, animate, 
and socio-technological nature systems)]. Moscow: IPIRAN. 456~p. 
\bibitem{6-grn-1}
\Aue{Grinchenko, S.\,N.} 2009. Homo eruditus (chelovek obrazovannyy) kak 
element sistemy Chelovechestva [Homo eruditus (educated human) as an element 
of the Humakind's system]. \textit{Otkrytoe obrazovanie} [Open Education]  
2:48--55.

\bibitem{10-grn-1} %7
\Aue{Grinchenko, S.\,N., and Yu.\,I.~Shchapova.} 2010. 
Human history periodization models. \textit{Her. Russ. Acad. Sci.} 80(6):498--506.
%\bibitem{11-grn-1} %8
%\Aue{Grinchenko, S.\,N., and Y.\,I.~Shchapova.}  2010. Human history 
%periodization models. \textit{Herald of the Russian Academy of Sciences} 
%80(6):498--506.

\bibitem{7-grn-1} %9
\Aue{Grinchenko, S.\,N.} 2011.The pre- and post-history of Humankind: What is 
it?  \textit{Problems of contemporary world futurology}. 
 Newcastle-upon-Tyne: Cambridge Scholars 
Publishing.  341--353.
\bibitem{8-grn-1} %10
\Aue{Grinchenko, S.\,N.} 2012. Ob evolyutsii psikhiki kak ie\-rar\-khi\-che\-skoy 
sistemy (kiberneticheskoe predstavlenie) [On the evolution of mind as 
a~hierarchical system (a~cybernetic approach)]. \textit{Istoricheskaya 
psikhologiya i~sotsiologiya istorii} [Historical Psychology \& Sociology of 
History] 6(2):\linebreak 60--77.


\bibitem{12-grn-1} %11
\Aue{Grinchenko, S.\,N., and Y.\,I.~Shchapova.} 2013. \textit{In\-for\-ma\-tsi\-on\-nye 
tekhnologii v~istorii Chelovechestva} [Information technology in the history of 
Humankind]. Moscow: Novye tekhnologii. 32~p. (Prilozhenie k zhurnalu 
<<\textit{Informatsionnye tekhnologii}>> [Supplement to J.~Information Technology] 8.

\bibitem{9-grn-1} %12
\Aue{Grinchenko, S.\,N.} 2014. Evolyutsiya tempov zhizni lyudey i~razvitie 
chelovechestva [The evolution of the pace of human life and human development]. 
\textit{Human Being} 5:28--36.

\bibitem{13-grn-1}
\Aue{Grinchenko, S.\,N., and Y.\,I.~Shchapova.} 2017. Archaeological epoch as 
the succession of generations of evolutive subject-carrier archaeological  
sub-epoch. \textit{Philosophy of Nature in Cross-Cultural Dimensions: The Result of 
the International Symposium at the University of Vienna}~/ Komparative 
Philosophie und Interdisziplin$\ddot{\mbox{a}}$re Bildung (KoPhil), Band~5. 
Hamburg: Verlag Dr.\ Kova$\Check{\mbox{c}}$.  478--499.
\bibitem{14-grn-1}
\Aue{Shchapova, Y.\,L., and S.\,N.~Grinchenko.} 2017. \textit{Vvedenie 
v~teoriyu arkheologicheskoy epokhi: chislovoe modelirovanie i~logarifmicheskie 
shkaly prostranstvenno-vremennykh koordinat} [Introduction to the theory of the 
archaeological epoch: Numerical modeling and logarithmic scales of space--time 
coordinates]. Moscow: Faculty 
of History MSU, FRC CSC RAS]. 236~p. 

\vspace*{1pt}

\bibitem{15-grn-1}
\Aue{Grinchenko, S.\,N., and Y.\,I.~Shchapova}. 2018.  Communications: Model 
representations about historical retrospective and possible perspective. 
\textit{Communications Media Design Electronic~J.}  3(2):65--78. 
Available at: {\sf https://elibrary.ru/item.asp?id=36272286} (accessed October~5, 
2018).

\vspace*{1pt}

\bibitem{16-grn-1}
\Aue{Grinchenko, S.\,N.} 2012. Posleslovie [Afterword]. \textit{Mat-ly doklada 
na Sovmestnom nauchnom seminare IPI \mbox{INION} RAN ``Metodologicheskie 
problemy nauk ob informatsii''}  [Report materials at the Joint Scientific 
Seminar of the Institute of Informatics Problems of the Russian Academy of 
Sciences and the Institute of Scientific Information on Social Sciences of the 
Russian Academy of Sciences ``Methodological problems of information 
sciences''].  Moscow. 5--8.  Available at: {\sf 
http://legacy. inion.ru/files/File/MPNI\_9\_13\_12\_12\_posl.pdf} (accessed 
October~5, 2018).

\vspace*{1pt}

\bibitem{17-grn-1}
\Aue{Zhirmunskiy, A.\,V., and V.\,I.~Kuz'min.} 1982. \textit{Kriticheskie urovni 
v~protsessakh razvitiya biologicheskikh sistem} [Critical levels in the development 
of biological systems]. Moscow: Nauka. 179~p.

\vspace*{1pt}

\bibitem{18-grn-1}
\Aue{Shchapova, Y.\,L.} 2000. Khronologiya i~periodizatsii drev\-ney\-shey istorii 
kak chislovaya posledovatel'nost' (ryad Fibonachchi) [Chronology and 
periodization of ancient history as a numerical sequence (Fibonacci's series)]. 
\textit{Informatsionnyy byulleten' Assotsiatsii ``Istoriya i~komp'yuter''} 
[Newsletter of the Association ``History and Computer'']  25.

\vspace*{1pt}

\bibitem{19-grn-1}
\Aue{Shchapova, Y.\,L.} 2005. \textit{Arkheologicheskaya epokha: khro\-no\-lo\-giya, 
periodizatsiya, teoriya, model'} [Archaeological epoch: Chronology, periodization, 
theory, model]. Moscow: KomKniga, 192~p.

\vspace*{1pt}

\bibitem{20-grn-1}
\Aue{Shchapova, Y.\,L.} 2011. \textit{Material'noe proizvodstvo 
v~arkheologicheskuyu epokhu} [Material production in the archaeological epoch]. 
St.\ Petersburg: Aleteyya. 244~p.

\vspace*{1pt}

\bibitem{21-grn-1}
\Aue{Grinchenko, S.\,N., and Yu.\,I.~Shchapova.} 2014. Prostranstvo i~vremya 
v~arheologii. Chast'~3. O~metrike bazisnoy prostranstvennoy struktury 
chelovechestva v~arkheologicheskuyu epokhu [Space and time in archeology. 
Part~3. About the metric of Humankind basic spatial structure  in  
archaeological epoch]. \textit{Space and Time}  
1(15):\linebreak 78--89.
\end{thebibliography}

 }
 }

\end{multicols}

\vspace*{-6pt}

\hfill{\small\textit{Received October 17, 2018}}

%\pagebreak

%\vspace*{-18pt}


  
  \Contrl
  
  \noindent
   \textbf{Grinchenko Sergey N.} (b.\ 1946)~--- Doctor of Science in technology, professor, principal 
scientist, Institute of Informatics Problems, Federal Research Center ``Computer Science and 
Control'' of the Russian Academy of Sciences, 44-2~Vavilov Str., Moscow 119333, Russian 
Federation; \mbox{sgrinchenko@ipiran.ru}
\label{end\stat}

\renewcommand{\bibname}{\protect\rm Литература}     %15 

\def\stat{dorofeeva}

\def\tit{О ТОЧНОСТИ НОРМАЛЬНОЙ АППРОКСИМАЦИИ
ПРИ~ОТСУТСТВИИ НОРМАЛЬНОЙ СХОДИМОСТИ$^*$}

\def\titkol{О точности нормальной аппроксимации
при~отсутствии нормальной сходимости}

\def\aut{В.\,Ю.~Королев$^1$,  А.\,В.~Дорофеева$^2$}

\def\autkol{В.\,Ю.~Королев,  А.\,В.~Дорофеева}

\titel{\tit}{\aut}{\autkol}{\titkol}

\index{Королев В.\,Ю.}
\index{Дорофеева А.\,В.}
\index{Korolev V.\,Yu.}
\index{Dorofeeva A.\,V.}

{\renewcommand{\thefootnote}{\fnsymbol{footnote}} \footnotetext[1]
{Работа выполнена при поддержке РФФИ (проект 18-07-01405).
  }}

\renewcommand{\thefootnote}{\arabic{footnote}}
\footnotetext[1]{Факультет вычислительной математики и~кибернетики Московского государственного университета имени
 М.\,В.~Ломоносова; Институт проб\-лем информатики Федерального исследовательского цент\-ра 
 <<Информатика и~управ\-ле\-ние>> Российской академии наук, \mbox{vkorolev@cs.msu.ru}}
\footnotetext[2]{Факультет вычислительной математики и~кибернетики Московского государственного 
 университета имени М.\,В.~Ломоносова, \mbox{alex.dorofeyeva@gmail.com}}

%\vspace*{-12pt}

  

\Abst{При решении прикладных задач в~самых разных областях принято использовать 
нормальное распределение в~качестве модели статистических закономерностей в~наблюдаемых 
данных с~аддитивной структурой. В~качестве критерия степени адекватности такой 
модели можно использовать оценки ско\-рости сходимости в~центральной предельной теореме (ЦПТ) 
тео\-рии вероятностей, устанавливающей, что при определенных условиях (например, 
при условии Линдеберга) суммарное воздействие большого числа случайных факторов проявляется в~виде 
случайной величины с~нормальным распределением. 
Классические оценки скорости сходимости в~ЦПТ 
типа неравенства Бер\-ри--Эс\-се\-ена доказаны при условии конечности третьих моментов слагаемых.
 Известны также оценки скорости сходимости при существовании моментов порядка $2\hm+\delta$ с~$0<\delta\hm<1$. 
 Если существуют моменты лишь второго порядка, то сходимость в~ЦПТ может быть как угодно медленной. 
 Если же у слагаемых моменты второго порядка не существуют, то сходимость распределений сумм 
 независимых случайных величин к~нормальному закону не имеет места. 
 Условия, гарантирующие 
 справедливость ЦПТ, практически невозможно достоверно проверить 
 при ограниченном объеме наблюдаемой выборки. Поэтому вопрос о том, какой может быть 
 реальная точ\-ность нормальной аппроксимации, когда она теоретически не применима, 
 но используется в~практических вычислениях, представляет большой интерес. Более того, в~некоторых 
 ситуациях при имитационном моделировании, когда распределения слагаемых принадлежат об\-ласти 
 притяжения устойчивого закона с~характеристическим показателем, меньшим двух, при увеличении числа 
 слагаемых сначала наблюдается убывание расстояния между распределением нормированной суммы и~нормальным 
 законом и~лишь при довольно большом числе слагаемых это расстояние начинает увеличиваться. 
 В~данной заметке предпринята попытка дать ответ на сформулированный выше вопрос и~привести 
 некоторые теоретические объяснения указанному эффекту.}

\KW{центральная предельная теорема; точ\-ность нормальной аппроксимации; тяжелые хвос\-ты; 
равномерное расстояние}

\DOI{10.14357/19922264210116}

\vspace*{3pt}

\vskip 10pt plus 9pt minus 6pt

\thispagestyle{headings}

\begin{multicols}{2}

\label{st\stat}

\section{Введение}


При решении прикладных задач в~самых разных областях принято использовать нормальное 
распределение в~качестве модели статистических закономерностей в~наблюдаемых данных с~аддитивной структурой. 
В~качестве критерия степени адекватности такой модели можно использовать оценки ско\-рости 
сходимости в~ЦПТ тео\-рии вероятностей, устанавливающей, 
что при определенных условиях (например, при условии Линдеберга) суммарное воздействие 
большого числа случайных факторов проявляется в~виде случайной величины с~нормальным распределением. 

Классические оценки скорости сходимости в~ЦПТ типа неравенства Бер\-ри--Эс\-се\-ена доказаны при 
условии конечности третьих моментов сла\-га\-емых. 

Известны также оценки ско\-рости сходимости при 
существовании моментов порядка $2\hm+\delta$ с~$0\hm<\delta\hm<1$ (см.\ подробный обзор в~[1]).
 Если существуют моменты лишь второго порядка, то сходимость в~ЦПТ может быть как угодно медленной~[2, 3]. 
 Если же у слагаемых моменты второго порядка не существуют, то сходимость распределений 
 сумм независимых случайных величин к~нормальному закону не имеет места. 
 
 Условия, 
 гарантирующие справедливость ЦПТ, практически 
 невозможно достоверно проверить при ограниченном объеме наблюдаемой выборки. 
 В~част\-ности, гистограмма, построенная по выборке из имеющего очень тяжелые хвосты 
 распределения Коши (у~которого отсутствует даже математическое ожидание), при умеренном 
 объеме выборки может быть визуально практически неотличимой от нормального распределения. 
 Поэтому вопрос о том, какой может быть реальная точность нормальной аппроксимации, когда 
 она теоретически не применима, но используется в~практических вычислениях, представляет 
 большой интерес. Более того, в~некоторых ситуациях при имитационном моделировании, 
 когда распределения слагаемых принадлежат области притяжения устойчивого закона с~характеристическим 
 показателем, меньшим двух, при увеличении числа слагаемых сначала наблюдается убывание 
 расстояния между распределением нормированной суммы и~нормальным законом и~лишь при довольно 
 большом числе слагаемых это расстояние начинает увеличиваться. 
 
 В~данной заметке предпринята 
 попытка дать ответ на сформулированный выше вопрос и~привести некоторые теоретические объяснения 
 указанному эффекту.


\section{Обозначения и~вспомогательные результаты}


Пусть $n\hm\in\mathbb{N}$, $\xi_1,\ldots,\xi_n$~--- независимые необязательно одинаково 
распределенные случайные величины, заданные на вероятностном пространстве 
$(\Omega,\mathfrak{A},{\sf P})$. Обозначим $F_j(x)\hm={\sf P}(\xi_j<x)$, $x\hm\in\mathbb{R}$, 
$j\hm\in\mathbb{N}$. Без существенного ограничения общ\-ности для удобства будем считать, 
что все функции распределения~$F_j(x)$ непрерывны.

Обозначим $S_n=\xi_1+\cdots+\xi_n$. Индикатор множества (события)~$A$ обозначим $\mathbb{I}(A)$. 
Пусть $u\hm>0$. Очевидно, 
$$
\xi_j=\xi_j\mathbb{I}\left(|\xi_j|\le u\right)+\xi_j\mathbb{I}\left(|\xi_j|> u\right)\,.
$$
 Тогда
\begin{multline*}
S_n=\sum\limits_{j=1}^n \xi_j\mathbb{I}(|\xi_j|\le u)+\sum\limits_{j=1}^n \xi_j\mathbb{I}(|\xi_j|> u)\equiv{}\\
{}\equiv
S_n^{(\le u)}+S_n^{(> u)}.
\end{multline*}
Если условиться считать равенство единице индикатора $\mathbb{I}(|\xi_j|\hm\le u)$ <<успехом>>, 
а~противоположное событие~--- <<неудачей>>, то число~$N_n(u)$ ненулевых слагаемых в~сумме 
$S_n^{(\le u)}$ будет случайной величиной, имеющей пуас\-сон-би\-но\-ми\-аль\-ное распределение 
с~па\-ра\-мет\-ра\-ми~$n$ и~$p_j\hm=p_j(u)\hm={\sf P}(|\xi_j|\hm\le u)\hm=F_j(u)\hm-F_j(-u)$, 
$j\hm=1,\ldots,n$. Заметим, что при неограниченном увеличении~$u$ параметры~$p_j$ стремятся к~единице.

\smallskip

\noindent
\textbf{Лемма~1.}\ \textit{Пусть $A,B\hm\in\mathfrak{A}$. Тогда} 
$$
{\sf P}(A\bigcap B)\ge{\sf P}(A) -{\sf P}\left(\overline{B}\right).
$$

%\smallskip

\noindent
Д\,о\,к\,а\,з\,а\,т\,е\,л\,ь\,с\,т\,в\,о\  элементарно.

\smallskip

Равномерное расстояние между функциями распределения~$F_{\xi}$ и~$F_{\eta}$ случайных величин~$\xi$ 
и~$\eta$ будем обозначать $\rho(F_{\xi},\,F_{\eta})$: 
$$
\rho(F_{\xi},\,F_{\eta})= \sup\limits_x\left\vert F_{\xi}(x)-F_{\eta}(x)\right\vert\,.
$$

 Нормальную функцию распределения со средним $a\hm\in\mathbb{R}$ и~дисперсией 
$\sigma^2\hm>0$ обозначим~$\Phi_{a,\sigma}$:
\begin{multline*}
\Phi_{a,\sigma}(x)=\fr{1}{\sigma\sqrt{2\pi}}
\int\limits_{-\infty}^{x}\exp\left\{-\fr{(z-a)^2}{2\sigma^2}\right\}dz={}\\
{}=
\Phi_{0,1}\left(\fr{x-a}{\sigma}\right)=
\Phi_{0,\sigma}(x-a)\,,\enskip x\in\mathbb{R}\,.
\end{multline*}

\smallskip

\noindent
\textbf{Лемма~2.}\ \textit{Для любых $a\hm\in\mathbb{R}$, $\sigma\hm>0$, $b\hm\in\mathbb{R}$}
$$
\rho\left(\Phi_{a+b,\sigma},\,\Phi_{a,\sigma}\right)= 
2\Phi_{0,\sigma}\left(\fr{|b|}{2}\right)-1\,.
$$

\smallskip

\noindent
Д\,о\,к\,а\,з\,а\,т\,е\,л\,ь\,с\,т\,в\,о\,.\ \
 Заметим, что если $H(x)$ и~$G(x)$~--- дифференцируемые функции распределения, то $\rho(H,G)$ 
 реализуется (т.\,е.\ точная верхняя грань $\sup_x|H(x)\hm-G(x)|$ по~$x$ достигается) в~одной из точек~$x$, 
 где $F'(x)\hm=G'(x)$. Действительно,
\begin{multline*}
\rho(H,G)=\sup\limits_x|H(x)-G(x)|={}\\
{}=\max\left\{\max\limits_x\left[H(x)-G(x)\right], 
\max\limits_x\left[G(x)-H(x)\right]\right\}
\end{multline*}
и экстремум каждого из выражений в~фигурных скобках достигается в~такой точке, где производная 
соответствующего выражения равна нулю, что равносильно равенству производных функций 
распределения~$H$ и~$G$, т.\,е.\ равенству соответствующих плотностей. 
В~рассматриваемом случае нормальных плотностей последнее условие эквивалентно тому, что
\begin{multline*}
\fr{1}{\sigma\sqrt{2\pi}}\exp\left\{-\fr{1}{2}\left(\fr{x-a-b}{\sigma}\right)^2\right\}={}\\
{}=
\fr{1}{\sigma\sqrt{2\pi}}\exp\left\{-\fr{1}{2}\left(\fr{x-a}{\sigma}\right)^2\right\},
\end{multline*}
или $\big(x-(a+b)\big)^2\hm=(x-a)^2$. Решая данное уравнение, получаем $x\hm-a\hm={b}/{2}$, 
откуда с~учетом соотношения $\Phi_{0,\sigma}(-|b|)\hm=1-\Phi_{0,\sigma}(|b|)$ вытекает требуемое утверждение.

\smallskip

Используя формулу Лагранжа, из леммы~2 легко получить известное неравенство:
$$
\rho\left(\Phi_{a+b,\sigma},\Phi_{a,\sigma}\right)\le \fr{|b|}{\sigma\sqrt{2\pi}}
$$
(см., например, неравенство~(3.4) в~книге~[4]).

\smallskip

\noindent
\textbf{Лемма~3.}\ \textit{Пусть $n\hm\in\mathbb{N}$, $\xi_1,\ldots,\xi_n$~--- случайные величины, 
$a_1,\ldots,a_n$~--- положительные числа, такие что
$a_1+\cdots+a_n\hm=1$. Тогда для любого $x\hm>0$
$$
{\sf P}\left(\left|\sum\limits_{j=1}^n\xi_j\right|\ge x\right)\le\sum\limits_{j=1}^n{\sf P}\left(|\xi_j|\ge a_jx\right).
$$
Если дополнительно случайные величины $\xi_1,\ldots,\xi_n$ одинаково распределены, то}
$$
{\sf P}\left(\left|\sum\limits_{j=1}^n\xi_j\right|\ge x\right)\le n{\sf P}\left(\left\vert \xi_1\right\vert \ge 
\fr{x}{n}\right).
$$


\smallskip

\noindent
Д\,о\,к\,а\,з\,а\,т\,е\,л\,ь\,с\,т\,в\,о\,.\ \ Заметим, что
$$
{\sf P}\left(\left|\sum\limits_{j=1}^n\xi_j\right|\ge x\right)\le{\sf P}\left(\sum\limits_{j=1}^n|\xi_j|
\ge x\right).
$$
Из геометрических соображений вытекает, что
\begin{multline*}
\left\{\omega:\, \sum\limits_{j=1}^n|\xi_j(\omega)|\ge x\right\}\subseteq
\left\{\omega:\,\left\vert \xi_1(\omega)\right\vert \ge a_1x\right\}\bigcup{}\\
{}\bigcup
\left\{\omega:\, \sum\limits_{j=2}^n \left\vert \xi_j(\omega)\right\vert \ge 
\left(1-a_1\right)x\right\}\subseteq
{}\\
{}
\subseteq\left\{\omega:\,\left\vert \xi_1(\omega)\right\vert \ge a_1x\right\}\bigcup
\left\{\omega:\,\left\vert \xi_2(\omega)\right\vert \ge a_2x\right\}\bigcup{}\\
{}\bigcup
\left\{\omega:\, \sum\limits_{j=3}^n\left\vert \xi_j(\omega)\right\vert
\ge \left(1-a_1-a_2\right)x\right\}\subseteq\cdots
\\
\cdots\subseteq\bigcup\limits_{j=1}^n\left\{\omega:\,\left\vert \xi_j(\omega)\right\vert
\ge a_jx\right\}.
\end{multline*}
Поэтому
\begin{multline*}
{\sf P}\left(\left\vert \sum\nolimits_{j=1}^n\xi_j\right\vert \ge x\right)
\le{\sf P}\left(\sum\limits_{j=1}^n\left\vert \xi_j\right\vert \ge x\right)\le
{}\\
{}
\le{\sf P}\!\!\left(\bigcup\limits_{j=1}^n\left\{\omega:\,\left\vert \xi_j(\omega)\right\vert
\ge a_jx\right\}\!\right)\!
\le\!\sum\limits_{j=1}^n{\sf P}\left(\left\vert \xi_j\right\vert \ge a_jx\right).\hspace*{-0.93279pt}
\end{multline*}
Лемма доказана.


\section{Основные результаты}


Рассмотрим оценку равномерного расстояния между распределением суммы $S_n\hm=S_n^{(\le u)}
\hm+S_n^{(>u)}$ и~нормальным законом с~соответствующими математическим ожиданием 
$a\hm\in\mathbb{R}$ и~дисперсией $\sigma\hm>0$, конкретный выбор которых прокомментируем \mbox{ниже}.

\smallskip

\noindent
\textbf{Теорема~1.} \textit{ Пусть $\epsilon\hm>0$, $u\hm>0$~--- произвольны. 
Тогда для любых $a\hm\in\mathbb{R}$, $\sigma\hm>0$}
\begin{multline}
\rho\left(F_{S_n},\Phi_{a,\,\sigma}\right)\le
\rho\left(F_{S_n^{(\le u)}},\Phi_{a,\,\sigma}\right)+{}\\
{}+
\sum\limits_{j=1}^n\left[F_j(-u)+1-F_j(u)\right].
\label{e1-dor}
\end{multline}

\smallskip

\noindent
Д\,о\,к\,а\,з\,а\,т\,е\,л\,ь\,с\,т\,в\,о\,.\ \
 Пусть $\epsilon\hm>0$ произвольно. На основании леммы~1 имеем:
\begin{multline}
{\sf P}\left(S_n<x\right)={\sf P}\left(S_n<x;\,\left\vert S_n^{(>u)}\right\vert
\le\epsilon\right)+{}\\
{}+{\sf P}\left(S_n<x;\,\left\vert S_n^{(>u)}\right\vert >\epsilon\right)\ge{}
\\
{}\ge{\sf P}\left(S_n^{(\le u)}<x-S_n^{(>u)};\,|S_n^{(>u)}|\le\epsilon\right)\ge{}\\
{}\ge
{\sf P}\left(S_n^{(\le u)}<x-\epsilon;\,\left\vert S_n^{(>u)}\right\vert \le\epsilon\right)\ge
{}\\
{}\ge{\sf P}\left(S_n^{(\le u)}<x-\epsilon\right)-{\sf P}\left(
\left\vert S_n^{(>u)}\right\vert \ge\epsilon\right).
\label{e2-dor}
\end{multline}
С другой стороны, очевидно:
\begin{multline}
{\sf P}\left(S_n<x\right)={\sf P}\!\left(S_n^{(\le u)}<x-S_n^{(>u)};\,\left\vert S_n^{(>u)}\right\vert
\le\epsilon\right)+{}\\
{}+
{\sf P}\left(S_n<x;\,\left\vert S_n^{(>u)}\right\vert >\epsilon\right)\le
{}\\
{}
\le{\sf P}\left(S_n^{(\le u)}<x+\epsilon;\,\left\vert S_n^{(>u)}\right\vert \le\epsilon\right)+{}\\
{}+
{\sf P}(S_n<x;\,\left\vert S_n^{(>u)}\right\vert >\epsilon)\le
{}\\
{}\le{\sf P}\left(S_n^{(\le u)}<x+\epsilon\right)+{\sf P}\left(\left\vert S_n^{(>u)}\right\vert
>\epsilon\right).
\label{e3-dor}
\end{multline}
Легко видеть, что
\begin{multline}
\hspace*{-9.96pt}\left\vert {\sf P}\left(S_n<x\right)-\Phi_{a,\sigma}(x)\right\vert\!=\!
\max\left\{{\sf P}\left(S_n<x\right)-\Phi_{a,\sigma}(x),\right.\\
\left.\Phi_{a,\sigma}(x)-{\sf P}\left(S_n<x\right)\right\}.
\label{e4-dor}
\end{multline}
Используя~(\ref{e3-dor}) и~лемму~2, получим:
\begin{multline}
{\sf P}\left(S_n<x\right)-\Phi_{a,\sigma}(x)\le 
{\sf P}\left(\left\vert S_n^{(>u)}\right\vert>\epsilon\right)+{}\\
{}+
\left[{\sf P}\left(S_n^{(\le u)}<x+\epsilon\right)-\Phi_{a,\sigma}(x+\epsilon)\right]+{}
\\
{}+\left[\Phi_{a,\sigma}(x+\epsilon)-\Phi_{a,\sigma}(x)\right]\le
{\sf P}\left(\left\vert S_n^{(>u)}\right\vert >\epsilon\right)+{}\\
{}+
\rho\left(F_{S_n^{(\le u)}},\Phi_{a,\,\sigma}\right)+
\left[2\Phi_{0,\sigma}\left({\fr{\epsilon}{2}}\right)-1\right].
\label{e5-dor}
\end{multline}
Используя~(\ref{e2-dor}) и~лемму~2, получим
\begin{multline}
\Phi_{a,\sigma}(x)-{\sf P}\left(S_n<x\right)\le 
\Phi_{a,\sigma}(x)-{}\\
{}-{\sf P}\left(S_n^{(\le u)}<x-\epsilon\right)+{\sf P}\left(\left\vert S_n^{(>u)}\right\vert
>\epsilon\right)={}
\\
{}=\left[\Phi_{a,\sigma}(x)-\Phi_{a,\sigma}(x-\epsilon)\right]-
\left[{\sf P}\left(S_n^{(\le u)}<x-\epsilon\right)-{}\right.\\
\left.{}-\Phi_{a,\sigma}(x-\epsilon)
\vphantom{\left(S_n^{(\le u)}<x-\epsilon\right)}
\right]+ 
{\sf P}\left(\left\vert S_n^{(>u)}\right\vert>\epsilon\right) \le
{}\\
{}\le\left[2\Phi_{0,\sigma}\left({\fr{\epsilon}{2}}\right)-1\right]+
\rho(F_{S_n^{(\le u)}},\Phi_{a,\sigma})+{}\\
{}+{\sf P}\left(\left\vert S_n^{(>u)}\right\vert>\epsilon\right).
\label{e6-dor}
\end{multline}
Подставив оценки~(\ref{e5-dor}) и~(\ref{e6-dor}) в~(\ref{e4-dor}), получим
\begin{multline}
\rho(F_{S_n},\Phi_{a,\sigma})\le\rho\left(F_{S_n^{(\le u)}},\Phi_{a,\sigma}\right)+{}\\
{}+
\left[2\Phi_{0,\sigma}\left({\fr{\epsilon}{2}}\right)-1\right]+
{\sf P}\left(\left\vert S_n^{(>u)}\right\vert >\epsilon\right).
\label{e7-dor}
\end{multline}
Рассмотрим третье слагаемое в~правой части~(\ref{e7-dor}). На основании леммы~3 
по формуле полной ве\-ро\-ят\-ности, принимая во внимание тот факт, что 
${\epsilon}/{n}\hm>0$ и~$|\xi_j(\omega)|\mathbb{I}(|\xi_j(\omega)|>u)\hm=0$ для тех~$\omega$, 
для которых $|\xi_j(\omega)|\hm\le u$, $j\hm=1,\ldots,n$, имеем:
\begin{multline}
{\sf P}\left(\left\vert S_n^{(>u)}\right\vert > \epsilon\right)=
{\sf P}\left(\left\vert \sum\limits_{j=1}^n\xi_j\mathbb{I}
\left(\left\vert \xi_j\right\vert >u\right)\right\vert>\epsilon\right)\le{}\\
{}\le
\sum\limits_{j=1}^n{\sf P}\left(
\left\vert \xi_j\right\vert \mathbb{I}\left(\left\vert \xi_j\right\vert >u\right)>
{\fr{\epsilon}{n}}\right)=
{}\\
{}=\!\sum\limits_{j=1}^n\!\left[ {\sf P}\left(\!|\xi_j|\mathbb{I}(|\xi_j|>u)>
{\fr{\epsilon}{n}}\!\left\vert\!
\vphantom{\fr{\epsilon}{n}} \right.
|\xi_j|>u\!\right)\!{\sf P}(|\xi_j|>u)+{}\right.
\\
\left.{}+{\sf P}\!\left(\left\vert \xi_j\right\vert
\mathbb{I}\left(\left\vert \xi_j\right\vert >u\right)>{\fr{\epsilon}{n}}\!\left\vert\!
\vphantom{\fr{\epsilon}{n}}\right.\,\left\vert \xi_j\right\vert 
\le u\right)
{\sf P}\left(\left\vert \xi_j\right\vert \le u\right)\right]={}
\\
{}=\sum\limits_{j=1}^n\left[1-p_j(u)\right]
{\sf P}\!\left(\!|\xi_j|\mathbb{I}\left(\left\vert \xi_j\right\vert>u\right)>
\left.{\fr{\epsilon}{n}}
\right\vert
\left\vert \xi_j\right\vert >u\right)\le{}\\
{}\le
\sum\limits_{j=1}^n\left[F_j(-u)+1-F_j(u)\right].
\label{e8-dor}
\end{multline}
Подставив~(\ref{e8-dor}) в~(\ref{e7-dor}) и~устремив~$\epsilon$ к~нулю, получим требуемое. Теорема доказана.

\smallskip


На практике в~качестве параметров~$a$ и~$\sigma$ можно брать, например,
\begin{multline*}
a=a(u)={\sf E}S_n^{(\le u)}={\sf E}\sum\limits_{j=1}^n \xi_j\mathbb{I}\left(\left\vert \xi_j\right\vert 
\le u\right)={}\\
{}= \sum\limits_{j=1}^n {\sf E}\left[\xi_j\mathbb{I}\left(\left\vert \xi_j\right\vert
\le u\right)\right]=\sum\limits_{j=1}^np_j(u)\int\limits_{-u}^{u}x\,dF_j(x);
\end{multline*}

%\vspace*{-12pt}

\noindent
\begin{multline*}
\sigma^2=\sigma^2(u)={\sf D}S_n^{(\le u)}={\sf D}\sum\limits_{j=1}^n 
\xi_j\mathbb{I}\left(\left\vert \xi_j\right\vert \le u\right)={}\\
{}=
\sum\limits_{j=1}^n {\sf D}\left[\xi_j\mathbb{I}\left(\left\vert \xi_j\right\vert \le u\right)\right]=
{}\\
{}=\sum\limits_{j=1}^n\left[p_j(u)\!\int\limits_{-u}^{u}\!x^2\,dF_j(x)-
\left(\! p_j(u)\!\int\limits_{-u}^{u}\!x\,dF_j(x)\!
\right)^{\!\!2\,}\!\right].
\end{multline*}

При фиксированном $u\hm>0$ первое слагаемое в~правой части~(\ref{e1-dor}) 
убывает с~увеличением~$n$, тогда как второе возрастает. При этом существует $n_0\hm\ge1$ такое, что при
 $1\hm\le n\hm\le n_0$ вся правая часть~(\ref{e1-dor}) убывает, а~при $n\hm\ge n_0$ воз\-рас\-та\-ет. 
 В~случае одинаково распределенных слагаемых второе слагаемое воз\-рас\-та\-ет как~$kn$, где $k\hm>0$. 
 При этом за счет выбора очень большого~$u$ можно добиться произвольной малости коэффициента~$k$ 
 и,~как следствие, очень медленного роста второго слагаемого. Поэтому параметр~$n_0$ может 
 принимать довольно большие значения.

Задачу определения указанного~$n_0$ конкретизируем для частного случая. Предположим, что
\begin{equation}
\rho\left(F_{S_n^{(\le u)}},\Phi_{a,\sigma}\right)\le C n^{-\gamma}, \label{e9-dor}
\end{equation}
где $\gamma>0$, а коэффициент $C\hm=C(u)$ определяется свойствами распределений слагаемых 
в~сумме $S_n^{(\le u)}$. Некоторые критерии справедливости~(\ref{e9-dor}) 
приведены, например, в~[5]. Предположим также, что
\begin{equation}
\sum\limits_{j=1}^n\left[F_j(-u)+1-F_j(u)\right]\le kn\,,
\label{e10-dor}\end{equation}
где $k=k(u)\hm>0$. Несложно убедиться, что минимум функции
$$
g(x)=\fr{C}{x^{\gamma}}+kx
$$
достигается в~точке
$$
x_0=\left(\fr{C\gamma}{k}\right)^{{1}/({1+\gamma})},
$$
причем
\begin{equation}
\min\limits_{x\ge0}g(x)=g(x_0)=
(\gamma+1)\left(\fr{Ck^{\gamma}}{\gamma^{\gamma}}\right)^{{1}/({\gamma+1})}.
\label{e11-dor}
\end{equation}
Таким образом, в~качестве~$n_0$ выступает либо $[x_0]$, либо $[x_0]+1$. 
При этом правую часть~(\ref{e11-dor}) можно рассматривать как приближенное значение наилучшей 
верхней оценки точности нормальной аппроксимации при справедливости условий~(\ref{e9-dor}) и~(\ref{e10-dor}).

{\small\frenchspacing
{%\baselineskip=10.8pt
%\addcontentsline{toc}{section}{References}
\begin{thebibliography}{9}
\bibitem{Shevtsova2016}
\Au{Шевцова И.\,Г.} Точность нормальной аппроксимации: 
методы оценивания и~новые результаты.~--- М.: Аргамак-Медиа, 2016. 380~с.

\bibitem{Mackevicius1983}
\Au{Мацкявичюс В.\,К.} О~нижней оценке скорости сходимости в~центральной предельной теореме~// 
Теория вероятностей и~ее применения,
1983. Т.~28. Вып.~3. С.~565--569.

\bibitem{KorolevDorofeeva2017}
\Au{Korolev V.\,Yu., Dorofeeva~A.\,V.} Bounds of the accuracy of 
the normal approximation to the distributions of random sums under relaxed moment conditions~// 
Lith. Math.~J., 2017. Vol.~57. No.\,1. P.~38--58.

\bibitem{Petrov1972}
\Au{Петров В.\,В.} Суммы независимых случайных величин.~--- М.: Наука, 1972. 414~с.

%\bibitem{Klebanov1999}
%{\it L. B. Klebanov, S. T. Rachev, G. J. Szekely.} Pre-limit theorems and their applications // Acta Applicandae Mathematicae, 1999. Vol 58. P. %159--174.

\bibitem{Ibragimov1966}
\Au{Ибрагимов И.\,А.} О~точ\-ности аппроксимации функций распределения сумм 
независимых случайных величин нормальным распределением~// Теория вероятностей и~ее применения, 1966. 
Т.~11. Вып.~4. С.~632--655.
\end{thebibliography}

}
}

\end{multicols}

\vspace*{-3pt}

\hfill{\small\textit{Поступила в~редакцию 13.10.2020}}

\vspace*{8pt}

%\pagebreak

%\newpage

%\vspace*{-28pt}

\hrule

\vspace*{2pt}

\hrule

%\vspace*{-2pt}

\def\tit{ON THE ACCURACY OF~THE~NORMAL APPROXIMATION UNDER~THE~VIOLATION OF~THE~NORMAL CONVERGENCE}

\def\titkol{On the accuracy of~the~normal approximation under~the~violation of~the~normal convergence}

\def\aut{V.\,Yu.~Korolev$^{1,2}$ and~A.\,V.~Dorofeeva$^1$}

\def\autkol{V.\,Yu.~Korolev and~A.\,V.~Dorofeeva}

\titel{\tit}{\aut}{\autkol}{\titkol}

\vspace*{-11pt}


\noindent
$^1$Faculty of Computational Mathematics and Cybernetics, 
M.\,V.~Lomonosov Moscow State University, GSP-1,\linebreak
$\hphantom{^1}$Leninskie Gory, Moscow 119991, Russian Federation

\noindent
$^2$Institute of Informatics Problems, Federal Research Center ``Computer Sciences and Control'' 
of the Russian\linebreak
$\hphantom{^1}$Academy of Sciences; 44-2~Vavilov Str., Moscow 119133, Russian Federation

\def\leftfootline{\small{\textbf{\thepage}
\hfill INFORMATIKA I EE PRIMENENIYA~--- INFORMATICS AND
APPLICATIONS\ \ \ 2021\ \ \ volume~15\ \ \ issue\ 1}
}%
\def\rightfootline{\small{INFORMATIKA I EE PRIMENENIYA~---
INFORMATICS AND APPLICATIONS\ \ \ 2021\ \ \ volume~15\ \ \ issue\ 1
\hfill \textbf{\thepage}}}

\vspace*{6pt}





\Abste{When solving applied problems in various fields, it is conventional 
to use the normal approximation to the distribution of data with additive structure. 
As a~criterion of the adequacy of such a~model, it is possible to use bounds for the convergence 
rate in the central limit theorem (CLT) of the probability theory stating that under certain 
conditions (say, under the Lindeberg condition), the total effect of very many random factors acts as 
a~random variable with the normal distribution. The classical bounds for the convergence 
rate in the CLT such as the Berry--Esseen inequality are proved under the condition 
that the third moments of the summands exist. Also, bounds are known that require the
 existence of the moments of orders $2+\delta$ with $0< \delta <1$. 
 If only the moments of the second order exist, then the convergence in the CLT can 
 be arbitrarily slow. But if the moments of the summands of the second order do 
 not exist, then the convergence of the distributions of sums of independent 
 random variables to the normal law does not take place. It is practically impossible 
 to reliably check the conditions of the central limit theorem with the limited size 
 of the available sample. Therefore, the question of what is the real accuracy 
 of the normal approximation if it is theoretically impossible is of great interest. 
 Moreover, in some situations, in computer simulation of sums of 
 random variables whose distributions belong to the domain of attraction of the stable 
 distribution with the characteristic exponent less than two, as the number of summands grows, 
 first, the distance between the distribution of the normalized sum and the normal law 
 decreases and starts to increase only when the number of summands becomes sufficiently large. 
 In this paper, an attempt is undertaken to give some theoretical explanation of this 
 effect and to give an answer to the question posed above.}

       
\KWE{central limit theorem; accuracy of normal approximation; heavy tails; uniform distance}



\DOI{10.14357/19922264210116}

%\vspace*{-15pt}

\Ack
\noindent
Research supported by the Russian Foundation for Basic Research, project 18-07-01405.

\vspace*{4pt}

  \begin{multicols}{2}

\renewcommand{\bibname}{\protect\rmfamily References}
%\renewcommand{\bibname}{\large\protect\rm References}

{\small\frenchspacing
 {%\baselineskip=10.8pt
 \addcontentsline{toc}{section}{References}
 \begin{thebibliography}{9}
\bibitem{1-dor}
\Aue{Shevtsova, I.\,G.} 2016. \textit{Tochnost' normal'noy approksimatsii: metody otsenivaniya 
i~novye resul'taty} [The accuracy of the normal approximation: 
Methods of estimation and new results]. Moscow: Argamak-Media. 380~p.
{\looseness=1

}

\columnbreak


\bibitem{2-dor}
\Aue{Mackevicius, V.\,K.} 1984. A~lower bound for the convergence rate in the central limit theorem. 
\textit{Theor. Probab. Appl.} 28(3):596--601.

\vspace*{-2pt}

\bibitem{3-dor}
\Aue{Korolev, V.\,Yu., and A.\,V.~Dorofeeva.}
 2017. Bounds of the accuracy of the normal approximation to the distributions of random 
 sums under relaxed moment conditions. \textit{Lith. Math.~J.} 57(1):38--58.
\bibitem{4-dor}
\Aue{Petrov, V.\,V.} 1972. \textit{Summy nezavisimykh sluchaynykh velichin} 
[Sums of independent random variables]. Moscow: Nauka. 416~p.
\bibitem{5-dor}
\Aue{Ibragimov, I.\,A.} 1966. On the accuracy of Gaussian approximation to the distribution functions 
of sums of independent variables. \textit{Theor. Probab. Appl.} 11(4):559--579.
 \end{thebibliography}

 }
 }

\end{multicols}

\vspace*{-3pt}

  \hfill{\small\textit{Received October~13, 2020}}


%\pagebreak

%\vspace*{-8pt}     

\Contr

\noindent
\textbf{Korolev Victor Yu.} (b.\ 1954)~--- 
Doctor of Science in physics and mathematics, professor, Head of Department, Faculty of 
Computational Mathematics and Cybernetics, M.\,V.~Lomonosov Moscow State University, 
GSP-1, Leninskie Gory, Moscow 119991, Russian Federation; leading scientist, 
Federal Research Center ``Computer Science and Control'' 
of the Russian Academy of Sciences, 44-2~Vavilov Str.,Moscow 119333, Russian Federation; 
\mbox{vkorolev@cs.msu.ru}

\vspace*{6pt}

\noindent
\textbf{Dorofeeva Alexandra V.} (b.\ 1991)~--- 
graduate PhD student, Faculty of Computational Mathematics and Cybernetics, 
M.\,V.~Lomonosov Moscow State University, GSP-1, Leninskie Gory, Moscow 119991, 
Russian Federation; \mbox{alex.dorofeyeva@gmail.com}

\label{end\stat}

\renewcommand{\bibname}{\protect\rm Литература}  %16














%%%%%%%%%%%%%%%%%%%%%%%%%%%%%%%%%%%%%%%%

%\def\stat{rez}
{%\hrule\par
%\vskip 7pt % 7pt
\raggedleft\Large \bf%\baselineskip=3.2ex
Р\,Е\,Ц\,Е\,Н\,З\,И\,И \vskip 17pt
    \hrule
    \par
\vskip 6pt plus 6pt minus 3pt }

%\thispagestyle{headings} %с верхним колонтитулом
%\thispagestyle{myheadings} %с нижним колонтитулом, но в верхнем РЕЦЕНЗИИ

\def\tit{НОВАЯ КНИГА И.\,Н.~СИНИЦЫНА, А.\,С.~ШАЛАМОВА <<ЛЕКЦИИ ПО ТЕОРИИ 
ИНТЕГРИРОВАННОЙ ЛОГИСТИЧЕСКОЙ ПОДДЕРЖКИ>> (М.: ТОРУС ПРЕСС, 2012. 624~с.)}

%1
\def\aut{Д.ф.-м.н., профессор С.\,Я.~Шоргин}

\def\auf{\ }

\def\leftkol{\ % РЕЦЕНЗИИ
}

\def\rightkol{ \ } 

%\def\leftkol{\ } % ENGLISH ABSTRACTS}

%\def\rightkol{\ } %ENGLISH ABSTRACTS}

%\def\leftkol{РЕЦЕНЗИИ}

%\def\rightkol{РЕЦЕНЗИИ}

\titele{\tit}{\aut}{\auf}{\leftkol}{\rightkol}
\vspace*{-18pt}


     \label{st\stat}

     \begin{multicols}{2}
     {\small
     {\baselineskip=10.1pt
     

      В книге представлено системное изложение теоретических основ одного из новейших 
направлений в \mbox{об\-ласти} экономики послепродажного обслуживания изделий наукоемкой 
продукции (ИНП) длительного пользования~--- интегрированной логистической поддержки
(ИЛП). 
{\looseness=1

}

Приведены также результаты новых работ, выполненных в Институте проблем информатики 
Российской академии наук в рамках научного направления <<Информационные технологии и 
анализ сложных сис\-тем>>.
 {%\looseness=1

}
     
      Излагаемые в книге научные подходы позво\-ляют карди\-наль\-но реформировать 
существующие системы производства и эксплуатации ИНП путем создания и внед\-ре\-ния 
методов рационального и оптимального управ\-ле\-ния процессами расходования 
вре\-мен\-н$\acute{\mbox{ы}}$х, 
мате\-ри\-аль\-ных, трудовых и других ресурсов на всех стадиях жизненного цикла изделий (ЖЦИ) по 
критериям экономической целесообразности и эф\-фек\-тив\-ности.
  {\looseness=1

}
    
      В книге приведен краткий обзор причин возник\-новения и
      развития CALS-методологии как основы 
современных международных стандартов по созданию и функционированию глобальных 
ин\-фор\-ма\-ци\-он\-но-ком\-му\-ни\-ка\-ци\-он\-ных систем, ее ключевых возможностей и эффективности 
результатов ее использования. 
Авторы %\linebreak 
предлагают ряд научных обоснований для разработки 
единой теории проектирования и управления систем ИЛП для полноценного использования 
преимуществ %\linebreak
 суще\-ст\-ву\-ющей методологии, определяют \mbox{общую} структурную схему 
комплексной системы <<ИНП-СППО>> и необходимость разработки для ее описания 
гибридных стохастических моделей.
{%\looseness=1

}

%\columnbreak
      
      Книга состоит из пяти частей, где последовательно излагается материал по каждой из 
следующих тем: <<Интегрированная логистическая поддержка>>, <<Теория гибридных 
стохастических систем и компьютерная поддержка исследований и разработок>>, <<Основы 
математического моделирования, анализа и синтеза систем послепродажного обслуживания>>, 
<<Определение и анализ показателей экспортного потенциала ИНП при проектировании>>, 
<<Задачи управления поддержкой послепродажного обслуживания>>, а также 
<<Моделирование инвестиционных процессов ИЛП в условиях неравновесных финансовых 
рынков>>. 
   
      В конце каждой главы приведены выводы и даны вопросы и задания для 
самоконтроля. В~приложениях содержатся основные определения по программам работ по 
анализу ИЛП, логистическим базам данных и компьютерным решениям, эквивалентной статистической 
линеаризации нелинейных преобразований ИЛП, справочный материал, а также развернутые 
уравнения для вероятностных характеристик.


      \def\leftkol{РЕЦЕНЗИИ}

\def\rightkol{РЕЦЕНЗИИ} 

      
      Книга заинтересует широкий круг специалистов и может быть использована научными 
проектными организациями в сфере промышленного производства ИНП. Большое количество 
иллюстраций, примеров и вопросов, обращенных к читателю, позволяет использовать книгу 
также в качестве учебного пособия для студентов и аспирантов машиностроительных, 
транспортных и~других специальностей, а также для самостоятельного изучения. 
{%\looseness=-1

}

Книга 
представляет несомненный интерес для специалистов и студентов в области прикладной 
математики и информатики.
    

}

}
\end{multicols}

%\newpage

\def\stat{authorsrus}
{%\hrule\par
%\vskip 7pt % 7pt
\raggedleft\Large \bf%\baselineskip=3.2ex
О\,Б\ \ А\,В\,Т\,О\,Р\,А\,Х \vskip 17pt
    \hrule
    \par
\vskip 21pt plus 8pt minus 6pt }


\def\tit{\ }

\def\aut{\ }

\def\auf{\ }

\def\leftkol{ОБ АВТОРАХ}

\def\rightkol{\ }

\titele{\tit}{\aut}{\auf}{\leftkol}{\rightkol}
\addcontentsline{toc}{subsection}{\textrm\textbf ОБ АВТОРАХ}
\label{st\stat}



\vspace*{-38pt}

\begin{multicols}{2}

\noindent
\textbf{Агаларов Явер Мирзабекович} (р.\ 1952)~--- 
кандидат технических наук, доцент, ведущий научный сотрудник 
Института проб\-лем информатики Федерального исследовательского центра 
<<Информатика и~управ\-ле\-ние>> Российской академии наук

\vspace*{3pt}

\noindent
\textbf{Битюков Юрий Иванович} (р.\ 1972)~---
доктор технических наук, доцент Московского авиационного института 
(национального исследовательского университета) 

\vspace*{3pt}

\noindent
\textbf{Буянов Михаил Владимирович} (р.\ 1994)~--- 
аспирант Московского авиационного института (национального исследовательского 
университета)

\vspace*{3pt}

\noindent
\textbf{Вихрова Ольга Геннадиевна} (р.\ 1990)~---
 аспирант Российского университета дружбы народов
 
 \vspace*{3pt}
 

\noindent
\textbf{Гайдамака Юлия Васильевна} (р.\ 1971)~--- 
кандидат фи\-зи\-ко-ма\-те\-ма\-ти\-че\-ских наук, доцент Российского университета 
дружбы народов; старший научный сотрудник Института проб\-лем информатики 
Федерального исследовательского центра <<Информатика и~управ\-ле\-ние>> 
Российской академии наук 

\vspace*{3pt}

\noindent
\textbf{Горшенин Андрей Константинович} (р.\ 1986)~--- 
кандидат фи\-зи\-ко-ма\-те\-ма\-ти\-че\-ских наук, доцент, 
ведущий научный сотрудник Института проб\-лем информатики Федерального 
исследовательского\linebreak
 центра <<Информатика и~управ\-ле\-ние>> Российской академии наук;
 старший научный сотрудник
 Института океанологии им.\ П.\,П.~Ширшова Российской академии наук

\vspace*{3pt}


\noindent
\textbf{Гребешков Александр Юрьевич} (р.\ 1967)~--- 
кандидат технических наук, старший научный сотрудник Поволжского 
государственного университета телекоммуникаций и информатики

\vspace*{3pt}

\noindent
\textbf{Грушо Александр Александрович} (р.\ 1946)~--- доктор 
фи\-зи\-ко-ма\-те\-ма\-ти\-че\-ских наук, профессор, заведующий лабораторией 
Института проб\-лем информатики Федерального исследовательского центра 
<<Информатика и~управ\-ле\-ние>> Российской академии наук 

\vspace*{3pt}

\noindent
\textbf{Забежайло Михаил Иванович} (р.\ 1956)~--- 
кандидат фи\-зи\-ко-ма\-те\-ма\-ти\-че\-ских наук, доцент, заведующий лабораторией 
Института проб\-лем информатики Федерального исследовательского центра 
<<Информатика и~управ\-ле\-ние>> Российской академии наук 

%\vspace*{3pt}
\columnbreak

\noindent
\textbf{Зарипова Эльвира Ринатовна} (р.\ 1979)~--- 
кандидат фи\-зи\-ко-ма\-те\-ма\-ти\-че\-ских наук, доцент Российского университета 
дружбы народов

\vspace*{3pt}

\noindent
\textbf{Иванов Сергей Валерьевич} (р.\ 1989)~--- 
кандидат фи\-зи\-ко-ма\-те\-ма\-ти\-че\-ских наук, доцент Московского 
авиационного института (национального исследовательского университета)

\vspace*{3pt}

\noindent
\textbf{Кибзун Андрей Иванович}  (р.\ 1951)~--- 
доктор фи\-зи\-ко-ма\-те\-ма\-ти\-че\-ских наук, профессор, 
заведующий кафедрой Московского авиационного института 
(национального исследовательского университета)

\vspace*{3pt}

\noindent
\textbf{Королев Виктор Юрьевич} (р.\ 1954)~--- доктор 
фи\-зи\-ко-ма\-те\-ма\-ти\-че\-ских наук, профессор, 
заведующий кафедрой математической статистики факультета вычислительной 
математики и~кибернетики МГУ им.\ М.\,В.~Ломоносова; 
ведущий научный сотрудник Института проб\-лем информатики 
Федерального исследовательского центра <<Информатика и~управ\-ле\-ние>> 
Российской академии наук; профессор Университета Дианьзи города Ханчжоу (Китай)

\vspace*{3pt}


\noindent
\textbf{Кружков Михаил Григорьевич} (р.\ 1975)~--- 
старший научный сотрудник Института проб\-лем 
информатики Федерального исследовательского центра 
<<Информатика и~управ\-ле\-ние>> Российской академии наук

\vspace*{3pt}

\noindent
\textbf{Кудрявцев Алексей Андреевич} (p.\ 1978)~--- кандидат 
фи\-зи\-ко-ма\-те\-ма\-ти\-че\-ских наук, 
доцент кафедры математической статистики факультета вычислительной математики 
и~кибернетики Московского государственного университета им.\ М.\,В.~Ломоносова

\vspace*{3pt}

\noindent
\textbf{Лисовская Екатерина Юрьевна} (р.\ 1992)~--- 
аспирант Национального исследовательского 
Томского государственного университета 

\vspace*{3pt}

\noindent
\textbf{Малашенко Юрий Евгеньевич} (р.\ 1946)~---
доктор фи\-зи\-ко-ма\-те\-ма\-ти\-че\-ских наук, заведующий сектором 
Вычислительного центра им.\ А.\,А.~Дородницына Федерального исследовательского центра 
<<Информатика и~управ\-ле\-ние>> Российской академии \mbox{наук}

\vspace*{3pt}


\noindent
\textbf{Моисеева Светлана Петровна} (р.\ 1971)~--- 
доктор фи\-зи\-ко-ма\-те\-ма\-ти\-че\-ских наук, доцент; 
профессор Национального исследовательского Томского государственного 
университета  

%\vspace*{3pt}
\pagebreak

\noindent
\textbf{Мокров Евгений Владимирович} (р.\ 1988)~--- 
аспирант Российского университета дружбы народов 

\vspace*{3pt}

\noindent
\textbf{Назарова Ирина Александровна} (р.\ 1966)~---
 кандидат фи\-зи\-ко-ма\-те\-ма\-ти\-че\-ских наук, научный сотрудник 
 Вычислительного центра им.\ А.\,А.~Дородницына Федерального исследовательского центра 
 <<Информатика и~управ\-ле\-ние>> Российской академии наук

\vspace*{3pt}

\noindent
\textbf{Наумов Андрей Викторович} (р.\ 1966)~--- 
доктор фи\-зи\-ко-ма\-те\-ма\-ти\-че\-ских наук, доцент, 
профессор\linebreak Московского авиационного института (национального исследовательского 
университета)

\vspace*{3pt}

\noindent
\textbf{Наумов Валерий Арсентьевич} (р.\ 1950)~--- 
кандидат фи\-зи\-ко-ма\-те\-ма\-ти\-че\-ских наук, 
научный руководитель Исследовательского института инноваций, 
г.~Хельсинки, Финляндия

\vspace*{3pt}

\noindent
\textbf{Новикова Наталья Михайловна} (р.\ 1953)~--- 
доктор фи\-зи\-ко-ма\-те\-ма\-ти\-че\-ских наук, профессор, ведущий научный сотрудник 
Вычислительного центра им.\ А.\,А.~Дородницына Федерального исследовательского центра 
<<Информатика и~управ\-ле\-ние>> Российской академии наук

\vspace*{3pt}

\noindent
\textbf{Пагано Микеле} (р.\ 1968)~---
PhD по информационным технологиям, профессор Университета 
г.\ Пиза (Италия) 

\vspace*{3pt}

\noindent
\textbf{Платонов Евгений Николаевич} (р.\ 1976)~---  
кандидат фи\-зи\-ко-ма\-те\-ма\-ти\-че\-ских наук, 
доцент Московского авиационного института (национального исследовательского 
университета)

\vspace*{3pt}

\noindent
\textbf{Потатуева Виктория Владимировна} (р.\ 1993)~---  
студентка магистратуры Национального исследовательского 
Томского государственного университета

\vspace*{3pt}


\noindent
\textbf{Разумчик Ростислав Валерьевич} (р.\ 1984)~--- 
кандидат фи\-зи\-ко-ма\-те\-ма\-ти\-че\-ских наук, 
ведущий научный сотрудник Института проб\-лем 
информатики Федерального исследовательского центра <<Информатика и~управ\-ле\-ние>>
Российской академии наук;  доцент Российского университета дружбы народов

\vspace*{3pt}

\noindent
\textbf{Самуйлов Константин Евгеньевич} (р.\ 1955)~---
доктор технических наук, профессор, заведующий ка\-фед\-рой Российского 
университета дружбы наро-\linebreak дов, директор Института прикладной математики\linebreak 
и~телекоммуникаций Российского университета дружбы народов; 
старший научный сотрудник Института проб\-лем информатики Федерального 
исследовательского центра <<Информатика и~управ\-ле\-ние>> 
Российской академии наук

\vspace*{3pt}

\noindent
\textbf{Смирнов Дмитрий Владимирович} (р.\ 1984)~--- 
биз\-нес-парт\-нер по информационным технологиям Департамента безопасности ПАО 
<<Сбербанк России>>

\vspace*{3pt}

\noindent
\textbf{Тимонина Елена Евгеньевна} (р.\ 1952)~--- 
доктор технических наук, профессор, ведущий научный\linebreak сотрудник 
Института проб\-лем информатики Федерального исследовательского центра 
<<Информатика и~управ\-ле\-ние>> Российской академии наук 

\vspace*{3pt}

\noindent
\textbf{Титова Анастасия Игоревна} (p.\ 1995)~--- 
студентка кафедры математической статистики факультета вычисли\-тельной математики 
и~кибернетики Московского государственного университета им.\ М.\,В.~Ломоносова

\vspace*{3pt}

\noindent
\textbf{Шоргин Всеволод Сергеевич} (р.\ 1978)~---
кандидат технических наук, старший научный сотрудник Института проб\-лем 
информатики Федерального исследовательского центра <<Информатика и~управ\-ле\-ние>> 
Российской академии наук

\vspace*{3pt}

\noindent
\textbf{Шоргин Сергей Яковлевич} (р.\ 1952)~--- 
доктор фи\-зи\-ко-ма\-те\-ма\-ти\-че\-ских наук, профессор, заместитель директора 
Федерального исследовательского цент\-ра <<Информатика и~управ\-ле\-ние>> 
Российской академии наук (ФИЦ ИУ РАН); главный научный сотрудник Института проб\-лем 
информатики ФИЦ ИУ РАН
 



 \label{end\stat}

%\def\leftfootline{\small{\textbf{\thepage}
%\hfill ИНФОРМАТИКА И ЕЁ ПРИМЕНЕНИЯ\ \ \ том~11\ \ \ выпуск~4\ \ \ 2017}
%}%
% \def\rightfootline{\small{ИНФОРМАТИКА И ЕЁ ПРИМЕНЕНИЯ\ \ \ том~11\ \ \ выпуск~4\ \ \ 2017
%\hfill \textbf{\thepage}}}


%\thispagestyle{myheadings}



\end{multicols}

\newpage  

%\def\stat{cont}
{%\hrule\par
%\vskip 7pt % 7pt
\raggedleft\Large \bf%\baselineskip=3.2ex
А\,В\,Т\,О\,Р\,С\,К\,И\,Й\ \ У\,К\,А\,З\,А\,Т\,Е\,Л\,Ь\ \ З\,А\ \ 2\,0\,0\,7 г. \vskip 17pt
    \hrule
    \par
\vskip 21pt plus 6pt minus 3pt }

\label{st\stat}

\def\tit{\ }

\def\aut{\ }
\def\auf{\ }

\def\leftkol{\ } % ENGLISH ABSTRACTS}

\def\rightkol{\ } %ENGLISH ABSTRACTS}

\titele{\tit}{\aut}{\auf}{\leftkol}{\rightkol}


\contentsline {chapter}{\ }{Выпуск \quad Стр.} 
\contentsline {section}{\textbf{Батракова Д.\,А., Королев В.\,Ю., Шоргин С.\,Я.}\ \ Новый метод вероятностно-ста\-ти\-сти\-че\-ско\-го анализа информационных потоков в\nobreakspace {}телекоммуникационных сетях}{\qquad 1 \qquad 40} 
\contentsline {section}{\textbf{Борисов А.\,В.}\ \ Байесовское оценивание в системах наблюдения с\nobreakspace {}марковскими скачкообразными процессами: игровой подход}{\qquad 2 \qquad 65}
\contentsline {section}{\textbf{Босов А.\,В., Иванов А.\,В.}\ \ Программная инфраструктура информационного Web-пор\-тала}{\qquad 2 \qquad 50}
\contentsline {section}{\textbf{Захаров В.\,Н., Калиниченко Л.\,А., Соколов И.\,А., Ступников С.\,А.}\ \ Конструирование канонических информационных моделей для интегрированных информационных систем}{\qquad 2 \qquad 15}
\contentsline {section}{\textbf{Захаров В.\,Н., Козмидиади В.\,А.}\ \ Средства обеспечения отказоустойчивости при\-ло\-жений}{\qquad 1 \qquad 14} 
\contentsline {section}{\textbf{Иванов А.\,В.}\ \ см. Босов А.\,В.\hfill\hfill\hfill\hfill\hfill\hfill\hfill\hfill\hfill\hfill\hfill\hfill\hfill\hfill\hfill\hfill\hfill\hfill\hfill\hfill\hfill\hfill\hfill\hfill\hfill\hfill\hfill\hfill\hfill\hfill\hfill\hfill\hfill\hfill\hfill}{\ }
\contentsline {section}{\textbf{Ильин В.\,Д., Соколов И.\,А.}\ \ Символьная модель системы знаний информатики в\nobreakspace {}че\-ло\-ве\-ко-автоматной среде}{\qquad 1 \qquad 66} 
\contentsline {section}{\textbf{Калиниченко Л.\,А.}\ \ см. Захаров В.\,Н.\hfill\hfill\hfill\hfill\hfill\hfill\hfill\hfill\hfill\hfill\hfill\hfill\hfill\hfill\hfill\hfill\hfill\hfill\hfill\hfill\hfill\hfill\hfill\hfill\hfill\hfill\hfill\hfill\hfill\hfill\hfill\hfill\hfill\hfill\hfill}{\ }
\contentsline {section}{\textbf{Козеренко Е.\,Б.}\ \ Лингвистическое моделирование для систем машинного перевода и обработки знаний}{\qquad 1 \qquad 54} 
\contentsline {section}{\textbf{Козмидиади В.\,А.}\ \ см. Захаров В.\,Н.\hfill\hfill\hfill\hfill\hfill\hfill\hfill\hfill\hfill\hfill\hfill\hfill\hfill\hfill\hfill\hfill\hfill\hfill\hfill\hfill\hfill\hfill\hfill\hfill\hfill\hfill\hfill\hfill\hfill\hfill\hfill\hfill\hfill\hfill\hfill }{\ } 
\contentsline {section}{\textbf{Королев В.\,Ю.}\ \ см. Батракова Д.\,А.\hfill\hfill\hfill\hfill\hfill\hfill\hfill\hfill\hfill\hfill\hfill\hfill\hfill\hfill\hfill\hfill\hfill\hfill\hfill\hfill\hfill\hfill\hfill\hfill\hfill\hfill\hfill\hfill\hfill\hfill\hfill\hfill\hfill\hfill\hfill}{\ } 
\contentsline {section}{\textbf{Кудрявцев А.\,А., Шоргин С.\,Я.}\ \ Байесовский подход к\nobreakspace {}анализу систем массового обслуживания и\nobreakspace {}показателей надежности}{\qquad 2 \qquad 76}
\contentsline {section}{\textbf{Печинкин А.\,В., Соколов И.\,А., Чаплыгин В.\,В.}\ \ Многолинейная система массового обслуживания с конечным накопителем и ненадежными приборами}{\qquad 1 \qquad 27} 
\contentsline {section}{\textbf{Печинкин А.\,В., Соколов И.\,А., Чаплыгин В.\,В.}\ \ Стационарные характеристики многолинейной\nobreakspace {}системы массового обслуживания с\nobreakspace {}одновременными отказами приборов}{\qquad 2 \qquad 39}
\contentsline {section}{\textbf{Синицын И.\,Н.}\ \ Корреляционные методы построения аналитических информационных моделей флуктуаций полюса Земли по априорным данным}{\qquad 2 \qquad \hphantom{9}2}
\contentsline {section}{\textbf{Синицын И.\,Н.}\ \ Развитие теории фильтров Пугачева для оперативной обработки информации в стохастических системах}{{\qquad 1 \qquad \hphantom{9}3}} 
\contentsline {section}{\textbf{Соколов И.\,А.}\ \ см. Захаров В.\,Н.\hfill\hfill\hfill\hfill\hfill\hfill\hfill\hfill\hfill\hfill\hfill\hfill\hfill\hfill\hfill\hfill\hfill\hfill\hfill\hfill\hfill\hfill\hfill\hfill\hfill\hfill\hfill\hfill\hfill\hfill\hfill\hfill\hfill\hfill\hfill}{\ }
\contentsline {section}{\textbf{Соколов И.\,А.}\ \ см. Ильин В.\,Д.\hfill\hfill\hfill\hfill\hfill\hfill\hfill\hfill\hfill\hfill\hfill\hfill\hfill\hfill\hfill\hfill\hfill\hfill\hfill\hfill\hfill\hfill\hfill\hfill\hfill\hfill\hfill\hfill\hfill\hfill\hfill\hfill\hfill\hfill\hfill}{\ } 
\contentsline {section}{\textbf{Соколов И.\,А.}\ \ см. Печинкин А.\,В.\hfill\hfill\hfill\hfill\hfill\hfill\hfill\hfill\hfill\hfill\hfill\hfill\hfill\hfill\hfill\hfill\hfill\hfill\hfill\hfill\hfill\hfill\hfill\hfill\hfill\hfill\hfill\hfill\hfill\hfill\hfill\hfill\hfill\hfill\hfill}{\ } 
\contentsline {section}{\textbf{Соколов И.\,А.}\ \ см. Печинкин А.\,В.\hfill\hfill\hfill\hfill\hfill\hfill\hfill\hfill\hfill\hfill\hfill\hfill\hfill\hfill\hfill\hfill\hfill\hfill\hfill\hfill\hfill\hfill\hfill\hfill\hfill\hfill\hfill\hfill\hfill\hfill\hfill\hfill\hfill\hfill\hfill}{\ }
\contentsline {section}{\textbf{Ступников С.\,А.}\ \ см. Захаров В.\,Н.\hfill\hfill\hfill\hfill\hfill\hfill\hfill\hfill\hfill\hfill\hfill\hfill\hfill\hfill\hfill\hfill\hfill\hfill\hfill\hfill\hfill\hfill\hfill\hfill\hfill\hfill\hfill\hfill\hfill\hfill\hfill\hfill\hfill\hfill\hfill}{\ }
\contentsline {section}{\textbf{Чаплыгин В.\,В.}\ \ см. Печинкин А.\,В.\hfill\hfill\hfill\hfill\hfill\hfill\hfill\hfill\hfill\hfill\hfill\hfill\hfill\hfill\hfill\hfill\hfill\hfill\hfill\hfill\hfill\hfill\hfill\hfill\hfill\hfill\hfill\hfill\hfill\hfill\hfill\hfill\hfill\hfill\hfill}{\ } 
\contentsline {section}{\textbf{Чаплыгин В.\,В.}\ \ см. Печинкин А.\,В.\hfill\hfill\hfill\hfill\hfill\hfill\hfill\hfill\hfill\hfill\hfill\hfill\hfill\hfill\hfill\hfill\hfill\hfill\hfill\hfill\hfill\hfill\hfill\hfill\hfill\hfill\hfill\hfill\hfill\hfill\hfill\hfill\hfill\hfill\hfill}{\ }
\contentsline {section}{\textbf{Шоргин С.\,Я.}\ \ см. Батракова Д.\,А.\hfill\hfill\hfill\hfill\hfill\hfill\hfill\hfill\hfill\hfill\hfill\hfill\hfill\hfill\hfill\hfill\hfill\hfill\hfill\hfill\hfill\hfill\hfill\hfill\hfill\hfill\hfill\hfill\hfill\hfill\hfill\hfill\hfill\hfill\hfill}{\ } 
\contentsline {section}{\textbf{Шоргин С.\,Я.}\ \ см. Кудрявцев А.\,А.\hfill\hfill\hfill\hfill\hfill\hfill\hfill\hfill\hfill\hfill\hfill\hfill\hfill\hfill\hfill\hfill\hfill\hfill\hfill\hfill\hfill\hfill\hfill\hfill\hfill\hfill\hfill\hfill\hfill\hfill\hfill\hfill\hfill\hfill\hfill}{\ }
%\thispagestyle{myheadings}
\def\leftfootline{\small{\textbf{\thepage}
\hfill ИНФОРМАТИКА И ЕЁ ПРИМЕНЕНИЯ\ \ \ том~1\ \ \ выпуск~2\ \ \ 2007}
}%
 \def\rightfootline{\small{ИНФОРМАТИКА И ЕЁ ПРИМЕНЕНИЯ\ \ \ том~1\ \ \ выпуск~2\ \ \ 2007
 \hfill \textbf{\thepage}}}
 \label{end\stat} 
                     
%\def\stat{cont-e}
{%\hrule\par
%\vskip 7pt % 7pt
\raggedleft\Large \bf%\baselineskip=3.2ex
2\,0\,0\,7\ \ A\,U\,T\,H\,O\,R\ \ I\,N\,D\,E\,X \vskip 17pt
    \hrule
    \par
\vskip 21pt plus 6pt minus 3pt }

\label{st\stat}

\def\tit{\ }

\def\aut{\ }
\def\auf{\ }

\def\leftkol{\ } % ENGLISH ABSTRACTS}

\def\rightkol{\ } %ENGLISH ABSTRACTS}

\titele{\tit}{\aut}{\auf}{\leftkol}{\rightkol}


\contentsline {chapter}{\ }{Issue \quad Page} 
\contentsline {subsection}{\textbf{Batrakova D.\,A., Korolev V.\,Yu., Shorgin S.\,Ya.}\ \ A New Method for the Probabilistic and Statistical Analysis of Information Flows in Telecommunication Networks}{\qquad 1 \qquad 40} 
\contentsline {subsection}{\textbf{Borisov A.\,V.}\ \ Bayesian Estimation in\nobreakspace {}Observation Systems with\nobreakspace {}Markov Jump Processes: Game-Theoretic Approach}{\qquad 2 \qquad 65} 
\contentsline {subsection}{\textbf{Bosov A.\,V., Ivanov A.\,V.}\ \ Linguistic Simulation for Machine Translation and Knowledge Management Systems}{\qquad 2 \qquad 50} 
\contentsline {subsection}{\textbf{Chaplygin V.\,V.} see Pechinkin A.\,V.\hfill\hfill\hfill\hfill\hfill\hfill\hfill\hfill\hfill\hfill\hfill\hfill\hfill\hfill\hfill\hfill\hfill\hfill\hfill\hfill\hfill\hfill\hfill\hfill\hfill\hfill\hfill\hfill\hfill\hfill\hfill\hfill\hfill\hfill\hfill}{\ }
\contentsline {subsection}{\textbf{Chaplygin V.\,V.} see Pechinkin A.\,V.\hfill\hfill\hfill\hfill\hfill\hfill\hfill\hfill\hfill\hfill\hfill\hfill\hfill\hfill\hfill\hfill\hfill\hfill\hfill\hfill\hfill\hfill\hfill\hfill\hfill\hfill\hfill\hfill\hfill\hfill\hfill\hfill\hfill\hfill\hfill}{\ }
\contentsline {subsection}{\textbf{Ilyin V.\,D., Sokolov I.\,A.}\ \ The Symbol Model of Informatics Knowledge System in Human-Automaton Environment}{\qquad 1 \qquad 66} 
\contentsline {subsection}{\textbf{Ivanov A.\,V.} see Bosov A.\,V.\hfill\hfill\hfill\hfill\hfill\hfill\hfill\hfill\hfill\hfill\hfill\hfill\hfill\hfill\hfill\hfill\hfill\hfill\hfill\hfill\hfill\hfill\hfill\hfill\hfill\hfill\hfill\hfill\hfill\hfill\hfill\hfill\hfill\hfill\hfill}{\ }
\contentsline {subsection}{\textbf{Kalinichenko L.\,A.} see Zakharov V.\,N.\hfill\hfill\hfill\hfill\hfill\hfill\hfill\hfill\hfill\hfill\hfill\hfill\hfill\hfill\hfill\hfill\hfill\hfill\hfill\hfill\hfill\hfill\hfill\hfill\hfill\hfill\hfill\hfill\hfill\hfill\hfill\hfill\hfill\hfill\hfill}{\ }
\contentsline {subsection}{\textbf{Korolev V.\,Yu.} see Batrakova D.\,A.\hfill\hfill\hfill\hfill\hfill\hfill\hfill\hfill\hfill\hfill\hfill\hfill\hfill\hfill\hfill\hfill\hfill\hfill\hfill\hfill\hfill\hfill\hfill\hfill\hfill\hfill\hfill\hfill\hfill\hfill\hfill\hfill\hfill\hfill\hfill}{\ }
\contentsline {subsection}{\textbf{Kozerenko E.\,B.}\ \ Linguistic Simulation for Machine Translation and Knowledge Management Systems}{\qquad 1 \qquad 54} 
\contentsline {subsection}{\textbf{Kozmidiady V.\,A.} see Zakharov V.\,N.\hfill\hfill\hfill\hfill\hfill\hfill\hfill\hfill\hfill\hfill\hfill\hfill\hfill\hfill\hfill\hfill\hfill\hfill\hfill\hfill\hfill\hfill\hfill\hfill\hfill\hfill\hfill\hfill\hfill\hfill\hfill\hfill\hfill\hfill\hfill}{\ }
\contentsline {subsection}{\textbf{Kudryavtsev A.\,A., Shorgin S.\,Ya.}\ \ Bayesian Approach to Queueing Systems and Reliability Characteristics}{\qquad 2 \qquad 76} 
\contentsline {subsection}{\textbf{Pechinkin A.\,V., Sokolov I.\,A., Chaplygin V.\,V.}\ \ Multichannel Queuing System with Finite Buffer and Unreliable Servers}{\qquad 1 \qquad 27} 
\contentsline {subsection}{\textbf{Pechinkin A.\,V., Sokolov I.\,A., Chaplygin V.\,V.}\ \ Stationary Characteristics of a Multichannel Queueing System with\nobreakspace {}Simultaneous Refusals of Servers}{\qquad 2 \qquad 39} 
\contentsline {subsection}{\textbf{Shorgin S.\,Ya.} see Batrakova D.\,A.\hfill\hfill\hfill\hfill\hfill\hfill\hfill\hfill\hfill\hfill\hfill\hfill\hfill\hfill\hfill\hfill\hfill\hfill\hfill\hfill\hfill\hfill\hfill\hfill\hfill\hfill\hfill\hfill\hfill\hfill\hfill\hfill\hfill\hfill\hfill}{\ }
\contentsline {subsection}{\textbf{Shorgin S.\,Ya.} see Kudryavtsev A.\,A.\hfill\hfill\hfill\hfill\hfill\hfill\hfill\hfill\hfill\hfill\hfill\hfill\hfill\hfill\hfill\hfill\hfill\hfill\hfill\hfill\hfill\hfill\hfill\hfill\hfill\hfill\hfill\hfill\hfill\hfill\hfill\hfill\hfill\hfill\hfill}{\ }
\contentsline {subsection}{\textbf{Sinitsyn I.\,N.}\ \ Correlational Methods for Analytical Informational Models of the Earth Pole Fluctuations Design Based on a priori Data}{\qquad 2 \qquad \hphantom{9}2}
\contentsline {subsection}{\textbf{Sinitsyn I.\,N.}\ \ Development of Pugachev Filtering for Stochastic Systems}{\qquad 1 \qquad \hphantom{9}3}
\contentsline {subsection}{\textbf{Sokolov I.\,A.} see Ilyin V.\,D.\hfill\hfill\hfill\hfill\hfill\hfill\hfill\hfill\hfill\hfill\hfill\hfill\hfill\hfill\hfill\hfill\hfill\hfill\hfill\hfill\hfill\hfill\hfill\hfill\hfill\hfill\hfill\hfill\hfill\hfill\hfill\hfill\hfill\hfill\hfill}{\ }
\contentsline {subsection}{\textbf{Sokolov I.\,A.} see Pechinkin A.\,V.\hfill\hfill\hfill\hfill\hfill\hfill\hfill\hfill\hfill\hfill\hfill\hfill\hfill\hfill\hfill\hfill\hfill\hfill\hfill\hfill\hfill\hfill\hfill\hfill\hfill\hfill\hfill\hfill\hfill\hfill\hfill\hfill\hfill\hfill\hfill}{\ }
\contentsline {subsection}{\textbf{Sokolov I.\,A.} see Pechinkin A.\,V.\hfill\hfill\hfill\hfill\hfill\hfill\hfill\hfill\hfill\hfill\hfill\hfill\hfill\hfill\hfill\hfill\hfill\hfill\hfill\hfill\hfill\hfill\hfill\hfill\hfill\hfill\hfill\hfill\hfill\hfill\hfill\hfill\hfill\hfill\hfill}{\ }
\contentsline {subsection}{\textbf{Sokolov I.\,A.} see Zakharov V.\,N.\hfill\hfill\hfill\hfill\hfill\hfill\hfill\hfill\hfill\hfill\hfill\hfill\hfill\hfill\hfill\hfill\hfill\hfill\hfill\hfill\hfill\hfill\hfill\hfill\hfill\hfill\hfill\hfill\hfill\hfill\hfill\hfill\hfill\hfill\hfill}{\ }
\contentsline {subsection}{\textbf{Stupnikov S.\,A.} see Zakharov V.\,N.\hfill\hfill\hfill\hfill\hfill\hfill\hfill\hfill\hfill\hfill\hfill\hfill\hfill\hfill\hfill\hfill\hfill\hfill\hfill\hfill\hfill\hfill\hfill\hfill\hfill\hfill\hfill\hfill\hfill\hfill\hfill\hfill\hfill\hfill\hfill}{\ }
\contentsline {subsection}{\textbf{Zakharov V.\,N., Kalinichenko L.\,A., Sokolov I.\,A., Stupnikov S.\,A.}\ \ Development of Canonical Information Models for Integrated Information Systems}{\qquad 2 \qquad 15} 
\contentsline {subsection}{\textbf{Zakharov V.\,N., Kozmidiady V.\,A.}\ \ Means Providing Applications Fault Tolerance}{\qquad 1 \qquad 14} 
\def\leftfootline{\small{\textbf{\thepage}
\hfill ИНФОРМАТИКА И ЕЁ ПРИМЕНЕНИЯ\ \ \ том~1\ \ \ выпуск~2\ \ \ 2007}
}%
 \def\rightfootline{\small{ИНФОРМАТИКА И ЕЁ ПРИМЕНЕНИЯ\ \ \ том~1\ \ \ выпуск~2\ \ \ 2007
 \hfill \textbf{\thepage}}}
 \label{end\stat} 


%\end{document}

%
\def\stat{rekl}
%\label{preobr}

%\def\tit{АКАДЕМИК ПУГАЧЁВ  ВЛАДИМИР СЕМЁНОВИЧ\\
%25.03.1911--25.03.1998}


%   \vspace*{-48pt}
%   \begin{center}\LARGE
%Академик Пугачёв  Владимир Семёнович\\ (25.03.1911--25.03.1998)
%   \end{center}

   %\vspace*{2.5mm}

   \begin{center}

{\prgsh\LARGE
ЮБИЛЕИ}

\end{center}
%\hrule

\vspace*{6pt}


   \vspace*{8mm}

   \thispagestyle{empty}


%\def\stat{emel}


\section*{К 70-летию заместителя директора ИПИ РАН,\\ члена редколлегии журнала
<<Информатика и её применения>>\\ доктора технических наук В.\,И.~Будзко}

\vspace*{18pt}




          \begin{multicols}{2}

%            \label{st\stat}

\begin{center}
\vspace*{1pt}
\mbox{%
\epsfxsize=78mm
\epsfbox{bud-1.eps}
}
\end{center}

\vspace*{12pt}

      14 августа 2014~г.\ исполнилось 70~лет за\-мес\-ти\-те\-лю директора ИПИ РАН по
научной работе доктору технических наук Владимиру Игоревичу Будзко.

      Владимир Игоревич Будзко родился в г.~Москве. Высшее образование получил на факультете
элект\-рон\-но-вы\-чис\-ли\-тель\-ных устройств в Московском
ин\-же\-нер\-но-фи\-зи\-че\-ском институте
(МИФИ), который он окончил в 1968~г., после чего был на\-прав\-лен для прохождения
службы в одну из войс\-ко\-вых частей, где прошел путь от инженера до первого заместителя
командира войсковой части.

      С приходом В.\,И.~Будзко в ИПИ РАН (2001~г.)\ в институте
сформировалось новое научное на\-прав\-ле\-ние теоретических исследований~--- <<Постро\-ение
ин\-фор\-ма\-ци\-он\-но-те\-ле\-ком\-му\-ни\-ка\-ци\-он\-ных\linebreak сис\-тем
высокой до\-ступ\-ности>>. В~рамках этого
направления выполнен широкий круг фундаментальных исследований по поиску подходов и
определению принципов построения средств обеспечения доступности, конфиденциальности
и целостности современных крупномасштабных
ин\-фор\-ма\-ци\-он\-но-те\-ле\-ком\-му\-ни\-ка\-ци\-он\-ных
сис\-тем (ИТС). Разработаны основные сис\-тем\-но-тех\-ни\-че\-ские принципы и базовые
архитектурные решения построения перспективных для условий России ИТС с
централизованной обработкой и хранением информации, сочетающих в себе свойства
высокой доступности, отказо- и катастрофоустойчивости, информационной защищенности.
Определены принципы, методы и математические основы рационального построения и
оптимизации средств восстановления функционирования центров обработки данных (ЦОД)
после возникновения отказов и катастроф, передачи и хранения данных, обеспечения
информационной безопасности при достижении минимальной совокупной стоимости
владения такими системами. Результаты нашли практическое воплощение при реализации
проектов в интересах ряда отечественных государственных и негосударственных
организаций, таких как Банк России (БР), Внешторгбанк, ОАО <<ГМК <<Норильский Никель>>,
<<Газпром>>, Минэкономразвития России, Правительство Москвы, а также ряд силовых
ведомств.

      Под руководством В.\,И.~Будзко начиная с 2001~г.\ выполнен комплекс
      на\-уч\-но-ис\-сле\-до\-ва\-тель\-ских и
      опыт\-но-кон\-ст\-рук\-тор\-ских работ (свыше 100~проектов),
направленных на развитие электронной информационной технологии БР.
Разработаны концепции развития ИТС БР сначала до 2008~г., а затем до 2013~г., которые
были приняты в качестве основы проведения технической политики. За реализацию проекта
<<Катастрофоустойчивая тер\-ри\-то\-ри\-аль\-но-рас\-пре\-де\-лен\-ная
      ин\-фор\-ма\-ци\-он\-но-те\-ле\-ком\-му\-ни\-ка\-ци\-он\-ная сис\-те\-ма централизованной
обработки банковской информации>> В.\,И.~Будзко удостоен Премии Правительства РФ в
области науки и техники за 2010~г.

      В.\,И.~Будзко возглавлял и возглавляет работы по ряду других прикладных проектов,
связанных с созданием, совершенствованием и развитием крупномасштабных ИТС.

      В.\,И.~Будзко~--- генерал-майор, доктор технических наук, член-кор\-рес\-пон\-дент
Академии криптографии РФ, известный ученый в области информатики и применения
информационных технологий при построении территориально распределенных ИТС
различного назначения. Является автором свыше 250~научных работ, опубликованных в
на\-уч\-но-тех\-ни\-че\-ских и специальных изданиях.

    \thispagestyle{empty}

      В.\,И.~Будзко уделяет большое внимание подготовке научных кадров. Под его
руководством защищено 6~диссертаций на соискание ученой степени кандидата
технических наук. Свыше 30~лет он читает лекции в ИКСИ Академии ФСБ, профессор
кафедры НИЯУ МИФИ. Является членом двух диссертационных советов, главным
редактором журнала <<Системы высокой доступности>> и членом редколлегии журнала
<<Информатика и её применения>>.

      \bigskip

      Редакционный совет и Редакционная коллегия журнала <<Информатика и её
применения>> сердечно поздравляют Владимира Игоревича Будзко с 70-ле\-ти\-ем и желают
крепкого здоровья и новых научных достижений.

\end{multicols}

%%Информатика и её применения
%Том 14 Выпуск 1-4 Год 2020

\def\stat{cont}
{%\hrule\par
%\vskip 7pt % 7pt
\raggedleft\Large \bf%\baselineskip=3.2ex
А\,В\,Т\,О\,Р\,С\,К\,И\,Й\ \ У\,К\,А\,З\,А\,Т\,Е\,Л\,Ь\ \ З\,А\ \ 2\,0\,2\,0 г. \vskip 17pt
 \hrule
 \par
\vskip 21pt plus 6pt minus 3pt }

\label{st\stat}

\def\tit{\ }

\def\aut{\ }
\def\auf{\ }

\def\leftkol{\ } % ENGLISH ABSTRACTS}

\def\rightkol{\ } %АВТОРСКИЙ УКАЗАТЕЛЬ ЗА 2020 г.} %ENGLISH ABSTRACTS}

\titele{\tit}{\aut}{\auf}{\leftkol}{\rightkol}
\addcontentsline{toc}{subsection}{\textrm\textbf Авторский указатель за 2020 г.}

\vspace*{-24pt}

\noindent
{\tabcolsep=3pt
\begin{tabular}{p{397pt}cc}
&\textbf{Вып.} & \textbf{Стр.}\\[6pt]
\Avtors{Абгарян~К.\,К., Гаврилов~Е.\,С.} Интеграционная платформа для многомасштабного моде-\linebreak
\\[-12pt]
\hspace*{23pt}лирования нейроморфных систем&2&104--110\\
\Avtors{Абгарян~К.\,К., Колбин~И.\,С.} Применение многомасштабного подхода и методов анализа\linebreak
\\[-12pt]
\hspace*{23pt}данных для моделирования теплопроводности в слоистых структурах&4&91--99\\
\Avtors{Агаларов~Я.\,М.} Оптимизация емкости основного накопителя в системе массового\linebreak
\\[-12pt]
\hspace*{23pt}обслуживания типа $G/M/1/K$ с дополнительным накопителем&2&72--79\\
\Avtors{Агасандян~Г.\,А.} Вычислительные аспекты применения CC-VaR на совокупности рынков&3&62--70\\
\Avtors{Агеев~К.\,А., Сопин~Э.\,С., Яркина~Н.\,В., Самуйлов~К.\,Е., Шоргин~С.\,Я.} Анализ механизмов\linebreak
\\[-12pt]
\hspace*{23pt}нарезки сети с учетом гарантий для различных типов трафика&3&\hphantom{1}94--100\\
\Avtors{Адамова~К.\,А.} см.\ Шнурков~П.\,В.&&\\
\Avtors{Базилевский~М.\,П.} Многофакторные модели полносвязной линейной регрессии без\linebreak
\\[-12pt]
\hspace*{23pt}ограничений на соотношения дисперсий ошибок переменных&2&92--97\\
\Avtors{Бахтеев~О.\,Ю.} см.\ Грабовой~А.\,В.&&\\
\Avtors{Беленков~В.\,Г.} см.\ Будзко~В.\,И.&&\\
\Avtors{Бетелин~В.\,Б., Кушниренко~А.\,Г., Леонов~А.\,Г.} Основные понятия программирования\linebreak
\\[-12pt]
\hspace*{23pt}в изложении для дошкольников&3&55--61\\
\Avtors{Бетелин~В.\,Б., Кушниренко~А.\,Г., Семенов~А.\,Л., Сопрунов~С.\,Ф.} О цифровой грамотности\linebreak
\\[-12pt]
\hspace*{23pt}и средах ее формирования&4&100--107\\
\Avtors{Борисов~А.\,В.} Численные схемы фильтрации марковских скачкообразных процессов по\linebreak
\\[-12pt]
\hspace*{23pt}дискретизованным наблюдениям II: случай аддитивных шумов&1&17--23\\
\Avtors{Борисов~А.\,В.} Численные схемы фильтрации марковских скачкообразных процессов по\linebreak
\\[-12pt]
\hspace*{23pt}дискретизованным наблюдениям III: случай мультипликативных шумов&2&10--18\\
\Avtors{Босов~А.\,В.} Управление выходом стохастической дифференциальной системы по квад-\linebreak
\\[-12pt]
\hspace*{23pt}ратичному критерию. V. Случай неполной информации о состоянии&2&19--25\\
\Avtors{Босов~А.\,В., Мартюшова~Я.\,Г., Наумов~А.\,В., Сапунова~А.\,П.} Байесовский подход к~по\-стро\-ению индивидуальной траектории пользователя в~системе дистанционного\linebreak
\\[-12pt]
\hspace*{23pt}обучения&3&86--93\\
\Avtors{Босов~А.\,В., Стефанович~А.\,И.} Управление выходом стохастической дифференциальной\linebreak
\\[-12pt]
\hspace*{23pt}системы по квадратичному критерию. IV. Альтернативное численное решение&1&24--30\\
\Avtors{Брюхов~Д.\,О., Ступников~С.\,А., Ковалёв~Д.\,Ю., Шанин~И.\,А.} Нейрофизиология как\linebreak
\\[-12pt]
\hspace*{23pt}предметная область для решения задач с интенсивным использованием данных&1&40--47\\
\Avtors{Будзко~В.\,И., Ядринцев~В.\,В., Соченков~И.\,В., Королёв~В.\,И., Беленков~В.\,Г.} Об одном подходе
 к формированию в условиях высокой неопределенности марке-\linebreak
\\[-12pt]
\hspace*{23pt}ров конфиденциальности в системах интенсивного использования данных&4&69--76\\
\Avtors{Вайсер~К.\,О.} см.\ Потанин~М.\,С.&&\\
\Avtors{Вохминцев~А.\,В., Мельников~А.\,В., Пачганов~C.\,А.} Метод навигации и составления карты в трехмерном пространстве на основе комбинированного решения вариационной\linebreak
\\[-12pt]
\hspace*{23pt}подзадачи точка--точка ICP для аффинных преобразований&1&101--112\\
\Avtors{Гаврилов~Е.\,С.} см.\ Абгарян~К.\,К.&&\\
\Avtors{Гайдамака~Ю.\,В.} см.\  Москалева~Ф.\,А.&&\\
\Avtors{Голембиовский~Д.\,Ю.} см.\ Данилишин~А.\,Р.&&\\
\Avtors{Голембиовский~Д.\,Ю.} см.\ Данилишин~А.\,Р.&&\\
\Avtors{Гончаров~А.\,А., Зацман~И.\,М., Кружков~М.\,Г.} Эволюция классификаций в надкорпусных\linebreak
\\[-12pt]
\hspace*{23pt}базах данных&4&108--116\\
\Avtors{Гончаров~А.\,В., Стрижов~В.\,В.} Выравнивание декартовых произведений упорядоченных\linebreak
\\[-12pt]
\hspace*{23pt}множеств&1&31--39\\
\end{tabular}
}

\pagebreak

\def\leftkol{АВТОРСКИЙ УКАЗАТЕЛЬ ЗА 2020 г.} % ENGLISH ABSTRACTS}

\def\rightkol{АВТОРСКИЙ УКАЗАТЕЛЬ ЗА 2020 г.} %ENGLISH ABSTRACTS}

%\thispagestyle{myheadings}
\def\leftfootline{\small{\textbf{\thepage}
\hfill ИНФОРМАТИКА И ЕЁ ПРИМЕНЕНИЯ\ \ \ том~14\ \ \ выпуск~4\ \ \ 2020}
}%
 \def\rightfootline{\small{ИНФОРМАТИКА И ЕЁ ПРИМЕНЕНИЯ\ \ \ том~14\ \ \ выпуск~4\ \ \ 2020
 \hfill \textbf{\thepage}}}


\noindent
{\tabcolsep=3pt
\begin{tabular}{p{394pt}cc}
&\textbf{Вып.} & \textbf{Стр.}\\[3pt]
\Avtors{Горшенин~А.\,К., Королев~В.\,Ю.} Аппроксимация распределений размеров частиц лунного\linebreak
\\[-12pt]
\hspace*{23pt}реголита на основе метода статистической симуляции выборок&2&50--57\\
\Avtors{Горшенин~А.\,К., Королев~В.\,Ю., Щербинина~А.\,А.} Статистическое оценивание распределений случайных коэффициентов стохастического дифференциального уравнения\linebreak
\\[-12pt]
\hspace*{23pt}Ланжевена&3&\hphantom{1}3--12\\
\Avtors{Горшенин~А.\,К., Кузьмин~В.\,Ю.} Анализ конфигураций LSTM-сетей для построения\linebreak
\\[-12pt]
\hspace*{23pt}среднесрочных векторных прогнозов&1&10--16\\
\Avtors{Грабовой~А.\,В., Бахтеев~О.\,Ю., Стрижов~В.\,В.} Введение отношения порядка на множестве\linebreak
\\[-12pt]
\hspace*{23pt}параметров аппроксимирующих моделей&2&58--65\\
\Avtors{Грушо~А.\,А., Забежайло~М.\,И., Смирнов~Д.\,В., Тимонина~Е.\,Е.} О вероятностных оценках\linebreak
\\[-12pt]
\hspace*{23pt}достоверности эмпирических выводов&4&3--8\\
\Avtors{Грушо~А.\,А., Забежайло~М.\,И., Смирнов~Д.\,В., Тимонина~Е.\,Е., Шоргин~С.\,Я.} Методы\linebreak
\\[-12pt]
\hspace*{23pt}математической статистики в задаче поиска инсайдера&3&71--75\\
\Avtors{Грушо~А.\,А., Забежайло~М.\,И., Тимонина~Е.\,Е.} О каузальной репрезентативности обуча-\linebreak
\\[-12pt]
\hspace*{23pt}ющих выборок прецедентов в задачах диагностического типа&1&80--86\\
\Avtors{Грушо~А.\,А., Тимонина~Е.\,Е., Грушо~Н.\,А., Терехина~И.\,Ю.} Выявление аномалий с по-\linebreak
\\[-12pt]
\hspace*{23pt}мощью метаданных&3&76--80\\
\Avtors{Грушо~А.\,А.} см.\ Грушо~Н.\,А.&&\\
\Avtors{Грушо~Н.\,А., Грушо~А.\,А., Забежайло~М.\,И., Тимонина~Е.\,Е.} Методы нахождения причин\linebreak
\\[-12pt]
\hspace*{23pt}сбоев в информационных технологиях  с помощью метаданных&2&33--39\\
\Avtors{Грушо~Н.\,А.} см.\ Грушо~А.\,А.&&\\
\Avtors{Данилишин~А.\,Р., Голембиовский~Д.\,Ю.} Оценка стоимости опционов на основе моделей\linebreak
\\[-12pt]
\hspace*{23pt}ARIMA--GARCH с ошибками, распределенными по закону $S_u$ Джонсона&4&83--90\\
\Avtors{Данилишин~А.\,Р., Голембиовский~Д.\,Ю.} Риск-нейтральная динамика для модели ARIMA-\linebreak
\\[-12pt]
\hspace*{23pt}GARCH с ошибками, распределенными по закону $S_U$ Джонсона&1&48--55\\
\Avtors{Диментов~А.\,В.} см.\ Краснов~Ф.\,В.&&\\
\Avtors{Донской~В.\,И.} Извлечение оптимизационных моделей из данных&3&109--118\\
\Avtors{Дубнов~Ю.\,А.} см.\ Попков~Ю.\,С.&&\\
\Avtors{Дулин~С.\,К., Дулина~Н.\,Г., Ермаков~П.\,В.} Информационный синтез документов&1&128--135\\
\Avtors{Дулина~Н.\,Г.} см.\ Дулин~С.\,К.&&\\
\Avtors{Дьяченко~Ю.\,Г.} см.\ Соколов~И.\,А.&&\\
\Avtors{Ермаков~П.\,В.} см.\ Дулин~С.\,К.&&\\
\Avtors{Ефросинин~Д.\,В.} см.\ Харин~П.\,А.&&\\
\Avtors{Жолобов~В.\,А.} см.\ Потанин~М.\,С.&&\\
\Avtors{Забежайло~М.\,И.} см.\ Грушо~А.\,А.&&\\
\Avtors{Забежайло~М.\,И.} см.\ Грушо~А.\,А.&&\\
\Avtors{Забежайло~М.\,И.} см.\ Грушо~А.\,А.&&\\
\Avtors{Забежайло~М.\,И.} см.\ Грушо~Н.\,А.&&\\
\Avtors{Захаров В. Н.} см.\ Френкель С. Л.&&\\
\Avtors{Зацман~И.\,М.} Проблемно-ориентированная верификация полноты темпоральных\linebreak
\\[-12pt]
\hspace*{23pt}онтологий и заполнение понятийных лакун&3&119--128\\
\Avtors{Зацман~И.\,М.} см.\ Гончаров~А.\,А.&&\\
\Avtors{Зацман~И.\,М.} см.\ Нуриев~В.\,А.&&\\
\Avtors{Зейфман~А.\,И.} см.\ Сатин~Я.\,А.&&\\
\Avtors{Кириков~И.\,А.} см.\ Румовская~С.\,Б.&&\\
\Avtors{Кирилюк~И.\,Л., Сенько~О.\,В.} Выбор моделей оптимальной сложности методами Монте-Карло (на примере моделей производственных функций регионов Российской\linebreak
\\[-12pt]
\hspace*{23pt}Федерации)&2&111--118\\
\Avtors{Ковалёв~Д.\,Ю.} см.\ Брюхов~Д.\,О.&&\\
\Avtors{Козеренко~Е.\,Б., Михеев~М.\,Ю., Сомин~Н.\,В., Эрлих~Л.\,И., Кузнецов~К.\,И.} Аналити\-че\-ская
текс\-тология в системах интеллектуальной обработки неструктурированных\linebreak
\\[-12pt]
\hspace*{23pt}данных&1&113--120\\
\Avtors{Колбин~И.\,С.} см.\ Абгарян~К.\,К.&&\\
\end{tabular}
}

\pagebreak

\def\leftkol{АВТОРСКИЙ УКАЗАТЕЛЬ ЗА 2020 г.} % ENGLISH ABSTRACTS}

\def\rightkol{АВТОРСКИЙ УКАЗАТЕЛЬ ЗА 2020 г.} %ENGLISH ABSTRACTS}

%\thispagestyle{myheadings}
\def\leftfootline{\small{\textbf{\thepage}
\hfill ИНФОРМАТИКА И ЕЁ ПРИМЕНЕНИЯ\ \ \ том~14\ \ \ выпуск~4\ \ \ 2020}
}%
 \def\rightfootline{\small{ИНФОРМАТИКА И ЕЁ ПРИМЕНЕНИЯ\ \ \ том~14\ \ \ выпуск~4\ \ \ 2020
 \hfill \textbf{\thepage}}}


\noindent
{\tabcolsep=3pt
\begin{tabular}{p{394pt}cc}
&\textbf{Вып.} & \textbf{Стр.}\\[3pt]
\Avtors{Королев~В.\,Ю.} О распределении отношения суммы элементов выборки, превосходящих\linebreak
\\[-12pt]
\hspace*{23pt}некоторый порог, к сумме всех элементов выборки.~I&3&35--43\\
\Avtors{Королев~В.\,Ю.} О распределении отношения суммы элементов выборки, превосходящих\linebreak
\\[-12pt]
\hspace*{23pt}некоторый порог, к сумме всех элементов выборки.~II&4&33--36\\
\Avtors{Королев~В.\,Ю.} см.\ Горшенин~А.\,К&&\\
\Avtors{Королев~В.\,Ю.} см.\ Горшенин~А.\,К.&&\\
\Avtors{Королёв~В.\,И.} см.\ Будзко~В.\,И.&&\\
\Avtors{Костина~А.\,А., Мирин~А.\,Ю., Молдовян~Д.\,Н., Фахрутдинов~Р.\,Ш.} Метод задания конечных некоммутативных ассоциативных алгебр произвольной четной размерности\linebreak
\\[-12pt]
\hspace*{23pt}для построения постквантовых криптосхем&1&\hphantom{1}94--100\\
\Avtors{Кочеткова~И.\,А.} см.\ Харин~П.\,А.&&\\
\Avtors{Краснов~Ф.\,В., Диментов~А.\,В., Шварцман~М.\,Е.} Использование тематических моделей\linebreak
\\[-12pt]
\hspace*{23pt}для парного сравнения  коллекций научных статей&3&129--135\\
\Avtors{Кривенко~М.\,П.} Последовательный анализ серий данных на основе многомерных ре-\linebreak
\\[-12pt]
\hspace*{23pt}фе\-рен\-с\-ных регионов&2&86--91\\
\Avtors{Кружков~М.\,Г.} см.\ Гончаров~А.\,А.&&\\
\Avtors{Кудрявцев~А.\,А., Шестаков~О.\,В.} Метод логарифмических моментов для оценивания\linebreak
\\[-12pt]
\hspace*{23pt}параметров гамма-экспоненциального распределения&3&49--54\\
\Avtors{Кузнецов~К.\,И.} см.\ Козеренко~Е.\,Б.&&\\
\Avtors{Кузьмин~В.\,Ю.} см.\ Горшенин~А.\,К.&&\\
\Avtors{Кушниренко~А.\,Г.} см.\ Бетелин~В.\,Б.&&\\
\Avtors{Кушниренко~А.\,Г.} см.\ Бетелин~В.\,Б.&&\\
\Avtors{Леонов~А.\,Г.} см.\ Бетелин~В.\,Б.&&\\
\Avtors{Макеева~Е.\,Д.} см.\ Харин~П.\,А.&&\\
\Avtors{Малашенко~Ю.\,Е., Назарова~И.\,А.} Аппроксимация множества достижимых потоков\linebreak
\\[-12pt]
\hspace*{23pt}многопользовательской сети&3&81--85\\
\Avtors{Мартюшова~Я.\,Г.} см.\ Босов~А.\,В.&&\\
\Avtors{Матюшенко~С.\,И., Разумчик~Р.\,В.} Стационарные характеристики системы Geo$/G/1/\infty $\linebreak
\\[-12pt]
\hspace*{23pt}с неординарным входящим потоком, управляющим размером очереди&4&25--32\\
\Avtors{Мейханаджян~Л.\,А., Разумчик~Р.\,В.} Стационарные характеристики системы $M/G/2/\infty$ с одним частным случаем дисциплины инверсионного порядка обслуживания\linebreak
\\[-12pt]
\hspace*{23pt}с обобщенным  вероятностным приоритетом&2&66--71\\
\Avtors{Мельников~А.\,В.} см.\ Вохминцев~А.\,В.&&\\
\Avtors{Мельников~С.\,Ю., Самуйлов~К.\,Е.} Статистические свойства двоичных неавтономных\linebreak
\\[-12pt]
\hspace*{23pt}регистров сдвига  с внутренним суммированием&2&80--85\\
\Avtors{Милованова~Т.\,А., Разумчик~Р.\,В.} Однолинейная система массового обслуживания с инверсионным порядком обслуживания с вероятностным приоритетом, групповым\linebreak
\\[-12pt]
\hspace*{23pt}пуассоновским потоком и фоновыми заявками&3&26--34\\
\Avtors{Мирин~А.\,Ю.} см.\ Костина~А.\,А.&&\\
\Avtors{Михеев~М.\,Ю.} см.\ Козеренко~Е.\,Б.&&\\
\Avtors{Молдовян~Д.\,Н.} см.\ Костина~А.\,А.&&\\
\Avtors{Москалева~Ф.\,А., Гайдамака~Ю.\,В., Шоргин~В.\,С.} Влияние параметров изоляции на\linebreak
\\[-12pt]
\hspace*{23pt}разделение ресурсов при нарезке сети&4&\hphantom{1}9--16\\
\Avtors{Назарова~И.\,А.} см.\ Малашенко~Ю.\,Е.&&\\
\Avtors{Наумов~А.\,В.} см.\ Босов~А.\,В.&&\\
\Avtors{Наумов~В.\,А., Самуйлов~К.\,Е.} О марковских и рациональных потоках случайных со-\linebreak
\\[-12pt]
\hspace*{23pt}бытий.~I&3&13--19\\
\Avtors{Наумов~В.\,А., Самуйлов~К.\,Е.} О марковских и рациональных потоках случайных со-\linebreak
\\[-12pt]
\hspace*{23pt}бытий.~II&4&37--46\\
\Avtors{Новиков~Д.\,А.} см.\ Шнурков~П.\,В.&&\\
\Avtors{Нуриев~В.\,А., Зацман~И.\,М.} Редуцирование спектра моделей перевода в надкорпусных\linebreak
\\[-12pt]
\hspace*{23pt}базах данных&2&119--126\\
\Avtors{Пачганов~C.\,А.} см.\ Вохминцев~А.\,В.&&\\
\end{tabular}
}

\pagebreak

\def\leftkol{АВТОРСКИЙ УКАЗАТЕЛЬ ЗА 2020 г.} % ENGLISH ABSTRACTS}

\def\rightkol{АВТОРСКИЙ УКАЗАТЕЛЬ ЗА 2020 г.} %ENGLISH ABSTRACTS}

%\thispagestyle{myheadings}
\def\leftfootline{\small{\textbf{\thepage}
\hfill ИНФОРМАТИКА И ЕЁ ПРИМЕНЕНИЯ\ \ \ том~14\ \ \ выпуск~4\ \ \ 2020}
}%
 \def\rightfootline{\small{ИНФОРМАТИКА И ЕЁ ПРИМЕНЕНИЯ\ \ \ том~14\ \ \ выпуск~4\ \ \ 2020
 \hfill \textbf{\thepage}}}


\noindent
{\tabcolsep=3pt
\begin{tabular}{p{394pt}cc}
&\textbf{Вып.} & \textbf{Стр.}\\[3pt]
\Avtors{Попков~А.\,Ю.} см.\ Попков~Ю.\,С.&&\\
\Avtors{Попков~Ю.\,С., Попков~А.\,Ю., Дубнов~Ю.\,А.} Методы детерминированных и рандомизи-\linebreak
\\[-12pt]
\hspace*{23pt}рованных энтропийных проекций для редукции размерности матрицы данных&4&47--54\\
\Avtors{Попов~Г.\,А., Симаворян~С.\,Ж., Симонян~А.\,Р., Улитина~Е.\,И.} Моделирование процесса мониторинга систем информационной безопасности на основе систем массового\linebreak
\\[-12pt]
\hspace*{23pt}обслуживания&1&71--79\\
\Avtors{Попов~М.\,В., Посыпкин~М.\,А.} Аппроксимация множества решений систем нелинейных\linebreak
\\[-12pt]
\hspace*{23pt}неравенств с использованием графических ускорителей&3&20--25\\
\Avtors{Посыпкин~М.\,А.} см.\ Попов~М.\,В.&&\\
\Avtors{Потанин~М.\,С., Вайсер~К.\,О., Жолобов~В.\,А., Стрижов~В.\,В.} Оптимизация структуры\linebreak
\\[-12pt]
\hspace*{23pt}сетей глубокого обучения&4&55--62\\
\Avtors{Разумчик~Р.\,В.} см.\ Матюшенко~С.\,И.&&\\
\Avtors{Разумчик~Р.\,В.} см.\ Мейханаджян~Л.\,А.&&\\
\Avtors{Разумчик~Р.\,В.} см.\ Милованова~Т.\,А.&&\\
\Avtors{Рождественский~Ю.\,В.} см.\ Соколов~И.\,А.&&\\
\Avtors{Румовская~С.\,Б., Кириков~И.\,А.} Метод визуального представления конфликтов в гибрид-\linebreak
\\[-12pt]
\hspace*{23pt}ных интеллектуальных многоагентных системах&4&77--82\\
\Avtors{Самуйлов~К.\,Е.} см.\ Агеев~К.\,А.&&\\
\Avtors{Самуйлов~К.\,Е.} см.\ Мельников~С.\,Ю.&&\\
\Avtors{Самуйлов~К.\,Е.} см.\ Наумов~В.\,А.&&\\
\Avtors{Самуйлов~К.\,Е.} см.\ Наумов~В.\,А.&&\\
\Avtors{Сапунова~А.\,П.} см.\ Босов~А.\,В.&&\\
\Avtors{Сатин~Я.\,А., Зейфман~А.\,И., Шилова~Г.\,Н.} О подходах к построению предельных режимов\linebreak
\\[-12pt]
\hspace*{23pt}для некоторых моделей массового обслуживания&2&3--9\\
\Avtors{Севастьянов~Л.\,А., Щетинин~Е.\,Ю.} О методах повышения точности многоклассовой\linebreak
\\[-12pt]
\hspace*{23pt}классификации на несбалансированных данных&1&63--70\\
\Avtors{Семенов~А.\,Л.} см.\ Бетелин~В.\,Б.&&\\
\Avtors{Сенько~О.\,В.} см.\ Кирилюк~И.\,Л.&&\\
\Avtors{Серебрянский~С.\,М., Тырсин~А.\,Н.} Повышение точности решения обратных задач за\linebreak
\\[-12pt]
\hspace*{23pt}счет уточнения граничных условий&1&56--62\\
\Avtors{Симаворян~С.\,Ж.} см.\ Попов~Г.\,А.&&\\
\Avtors{Симонян~А.\,Р.} см.\ Попов~Г.\,А.&&\\
\Avtors{Смирнов~Д.\,В.} см.\ Грушо~А.\,А.&&\\
\Avtors{Смирнов~Д.\,В.} см.\ Грушо~А.\,А.&&\\
\Avtors{Соколов~И.\,А., Степченков~Ю.\,А., Дьяченко~Ю.\,Г., Рождественский~Ю.\,В.} Повышение\linebreak
\\[-12pt]
\hspace*{23pt}сбоеустойчивости самосинхронных схем&4&63--68\\
\Avtors{Сомин~Н.\,В.} см.\ Козеренко~Е.\,Б.&&\\
\Avtors{Сопин~Э.\,С.} см.\ Агеев~К.\,А.&&\\
\Avtors{Сопрунов~С.\,Ф.} см.\ Бетелин~В.\,Б.&&\\
\Avtors{Соченков~И.\,В.} см.\ Будзко~В.\,И.&&\\
\Avtors{Степченков~Ю.\,А.} см.\ Соколов~И.\,А.&&\\
\Avtors{Стефанович~А.\,И.} см.\ Босов~А.\,В.&&\\
\Avtors{Стрижов~В.\,В.} см.\ Гончаров~А.\,В.&&\\
\Avtors{Стрижов~В.\,В.} см.\ Грабовой~А.\,В.&&\\
\Avtors{Стрижов~В.\,В.} см.\ Потанин~М.\,С.&&\\
\Avtors{Ступников~С.\,А.} см.\ Брюхов~Д.\,О.&&\\
\Avtors{Терехина~И.\,Ю.} см.\ Грушо~А.\,А.&&\\
\Avtors{Тимонина~Е.\,Е.} см.\  Грушо~А.\,А.&&\\
\Avtors{Тимонина~Е.\,Е.} см.\ Грушо~А.\,А.&&\\
\Avtors{Тимонина~Е.\,Е.} см.\ Грушо~А.\,А.&&\\
\Avtors{Тимонина~Е.\,Е.} см.\ Грушо~А.\,А.&&\\
\Avtors{Тимонина~Е.\,Е.} см.\ Грушо~Н.\,А.&&\\
\Avtors{Тырсин~А.\,Н.} см.\ Серебрянский~С.\,М.&&\\
\Avtors{Улитина~Е.\,И.} см.\ Попов~Г.\,А.&&\\
\end{tabular}
}

\pagebreak

\def\leftkol{АВТОРСКИЙ УКАЗАТЕЛЬ ЗА 2020 г.} % ENGLISH ABSTRACTS}

\def\rightkol{АВТОРСКИЙ УКАЗАТЕЛЬ ЗА 2020 г.} %ENGLISH ABSTRACTS}

%\thispagestyle{myheadings}
\def\leftfootline{\small{\textbf{\thepage}
\hfill ИНФОРМАТИКА И ЕЁ ПРИМЕНЕНИЯ\ \ \ том~14\ \ \ выпуск~4\ \ \ 2020}
}%
 \def\rightfootline{\small{ИНФОРМАТИКА И ЕЁ ПРИМЕНЕНИЯ\ \ \ том~14\ \ \ выпуск~4\ \ \ 2020
 \hfill \textbf{\thepage}}}


\noindent
{\tabcolsep=3pt
\begin{tabular}{p{394pt}cc}
&\textbf{Вып.} & \textbf{Стр.}\\[3pt]
\Avtors{Фахрутдинов~Р.\,Ш.} см.\ Костина~А.\,А.&&\\
\Avtors{Френкель С. Л., Захаров В. Н.} Совместная оценка предсказуемости данных и качества\linebreak
\\[-12pt]
\hspace*{23pt}предикторов&2&40--49\\
\Avtors{Харин~П.\,А., Макеева~Е.\,Д., Кочеткова~И.\,А., Ефросинин~Д.\,В., Шоргин~С.\,Я.} 
Система массового обслуживания с орбитами для анализа совместного обслуживания трафика 
с малыми задержками URLLC и~широкополосного доступа eMBB в~беспроводных\linebreak
\\[-12pt]
\hspace*{23pt}сетях пятого поколения&4&17--24\\
\Avtors{Хусаинов~А.\,А.} Производительность ограниченного конвейера&1&87--93\\
\Avtors{Шанин~И.\,А.} см.\ Брюхов~Д.\,О.&&\\
\Avtors{Шварцман~М.\,Е.} см.\ Краснов~Ф.\,В.&&\\
\Avtors{Шестаков~О.\,В.} Асимптотика оценки среднеквадратичного риска в задаче обращения\linebreak
\\[-12pt]
\hspace*{23pt}преобразования Радона по проекциям, регистрируемым на случайной сетке&2&26--32\\
\Avtors{Шестаков~О.\,В.} Асимптотическая регулярность вейвлет-методов обращения линейных однородных операторов по наблюдениям, регистрируемым в случайные моменты\linebreak
\\[-12pt]
\hspace*{23pt}времени&1&3--9\\
\Avtors{Шестаков~О.\,В.} О статистических свойствах оценки риска в задаче обращения преобра-\linebreak
\\[-12pt]
\hspace*{23pt}зования Радона при случайном объеме проекционных данных&3&44--48\\
\Avtors{Шестаков~О.\,В.} см.\ Кудрявцев~А.\,А.&&\\
\Avtors{Шилова~Г.\,Н.} см.\ Сатин~Я.\,А.&&\\
\Avtors{Шихиев~Ф.\,Ш.} см.\ Шихиев~Ш.\,Б.&&\\
\Avtors{Шихиев~Ш.\,Б., Шихиев~Ф.\,Ш.} Инкапсуляция семантических представлений в элементы\linebreak
\\[-12pt]
\hspace*{23pt}грамматики&1&121--127\\
\Avtors{Шнурков~П.\,В., Адамова~К.\,А.} Решение задачи безусловного экстремума для дробно-\linebreak
\\[-12pt]
\hspace*{23pt}линейного интегрального функционала, зависящего от параметра&2&\hphantom{1}98--103\\
\Avtors{Шнурков~П.\,В., Новиков~Д.\,А.} О концепции стохастической модели с управлением в~моменты выхода процесса на границу заданного подмножества множества\linebreak
\\[-12pt]
\hspace*{23pt}состояний&3&101--108\\
\Avtors{Шоргин~В.\,С.} см.\ Москалева~Ф.\,А.&&\\
\Avtors{Шоргин~С.\,Я.} см.\ Агеев~К.\,А.&&\\
\Avtors{Шоргин~С.\,Я.} см.\ Грушо~А.\,А.&&\\
\Avtors{Шоргин~С.\,Я.} см.\ Харин~П.\,А.&&\\
\Avtors{Щербинина~А.\,А.} см.\ Горшенин~А.\,К.&&\\
\Avtors{Щетинин~Е.\,Ю.} см.\ Севастьянов~Л.\,А.&&\\
\Avtors{Эрлих~Л.\,И.} см.\ Козеренко~Е.\,Б.&&\\
\Avtors{Ядринцев~В.\,В.} см.\ Будзко~В.\,И.&&\\
\Avtors{Яркина~Н.\,В.} см.\ Агеев~К.\,А.&&\\
\end{tabular}
}

%\thispagestyle{myheadings}
\def\leftfootline{\small{\textbf{\thepage}
\hfill ИНФОРМАТИКА И ЕЁ ПРИМЕНЕНИЯ\ \ \ том~14\ \ \ выпуск~4\ \ \ 2020}
}%
 \def\rightfootline{\small{ИНФОРМАТИКА И ЕЁ ПРИМЕНЕНИЯ\ \ \ том~14\ \ \ выпуск~4\ \ \ 2020
 \hfill \textbf{\thepage}}}

 \label{end\stat}

\newpage

\def\stat{cont-e}
{%\hrule\par
%\vskip 7pt % 7pt
\raggedleft\Large \bf%\baselineskip=3.2ex
2\,0\,2\,0\ \ A\,U\,T\,H\,O\,R\ \ I\,N\,D\,E\,X \vskip 17pt
 \hrule
 \par
\vskip 21pt plus 6pt minus 3pt }

\label{st\stat}

\def\tit{\ }

\def\aut{\ }
\def\auf{\ }

\def\leftkol{\ } %2020 AUTHOR INDEX} % ENGLISH ABSTRACTS}

\def\rightkol{\ } %2020 AUTHOR INDEX} %ENGLISH ABSTRACTS}

\titele{\tit}{\aut}{\auf}{\leftkol}{\rightkol}
\addcontentsline{toc}{subsection}{\textrm\textbf 2020 Author Index}

\def\leftfootline{\small{\textbf{\thepage}
\hfill INFORMATIKA I EE PRIMENENIYA~--- INFORMATICS AND APPLICATIONS\ \ \ 2020\
\ \ volume~14\ \ \ issue\ 4}
}%
 \def\rightfootline{\small{INFORMATIKA I EE PRIMENENIYA~--- INFORMATICS AND APPLICATIONS\ \ \ 2020\ \ \ volume~14\ \ \ issue\ 4
\hfill \textbf{\thepage}}}

\vspace*{-24pt}

\noindent
{\tabcolsep=3pt
\begin{tabular}{p{395.89pt}cc}
&\textbf{Issue} & \textbf{Page}\\[6pt]
\Avtors{Abgaryan~K.\,K. and Gavrilov~E.\,S.} Integration platform for multiscale modeling of neuromorphic\linebreak
\\[-12pt]
\hspace*{23pt}systems&2&104--110\\
\Avtors{Abgaryan~K.\,K. and Kolbin~I.\,S.} Application of multiscale approach and data sciences for\linebreak
\\[-12pt]
\hspace*{23pt}modeling thermal conductivity in layered structures&4&91--99\\
\Avtors{Adamova~K.\,A.} see Shnurkov~~P.\,V.&&\\
\Avtors{Agalarov~Ya.\,M.} Optimization of the capacity of the main storage in $G/M/1/K$ queueing system\linebreak
\\[-12pt]
\hspace*{23pt}with an additional storage device&2&72--79\\
\Avtors{Agasandyan~G.\,A.} Computational aspects of optimization on CC-VaR in a complex of markets&3&62--70\\
\Avtors{Ageev~K.\,A., Sopin~E.\,S., Yarkina~N.\,V., Samouylov~K.\,E., and Shorgin~S.\,Ya.} Analysis of the\linebreak
\\[-12pt]
\hspace*{23pt}network slicing mechanisms with guaranteed allocated resources for various traffic types&3&\hphantom{1}94--100\\
\Avtors{Bakhteev~O.\,Yu.} see Grabovoy~A.\,V.&&\\
\Avtors{Bazilevskiy~M.\,P.} Multifactor fully connected linear regression models without constraints to the\linebreak
\\[-12pt]
\hspace*{23pt}ratios of variables errors variances&2&92--97\\
\Avtors{Belenkov~V.\,G.} see Budzko~V.\,I.&&\\
\Avtors{Betelin~V.\,B., Kushnirenko~A.\,G., and Leonov~A.\,G.} Basic concepts of programming expounded\linebreak
\\[-12pt]
\hspace*{23pt}for preschoolers&3&55--61\\
\Avtors{Betelin~V.\,B., Kushnirenko~A.\,G., Semenov~A.\,L., and Soprunov~S.\,F.} About digital literacy and\linebreak
\\[-12pt]
\hspace*{23pt}environments for its development&4&100--107\\
\Avtors{Borisov~A.\,V.} Numerical schemes of Markov jump process filtering given discretized observa-\linebreak
\\[-12pt]
\hspace*{23pt}tions~II: Additive noise case&1&17--23\\
\Avtors{Borisov~A.\,V.} Numerical schemes of Markov jump process filtering given discretized observa-\linebreak
\\[-12pt]
\hspace*{23pt}tions III: Multiplicative noises case&2&10--18\\
\Avtors{Bosov~A.\,V.} Stochastic differential system output control by the quadratic criterion. V. Case of\linebreak
\\[-12pt]
\hspace*{23pt}incomplete state information&2&19--28\\
\Avtors{Bosov~A.\,V., Martyushova~Ya.\,G., Naumov~A.\,V., and Sapunova~A.\,P.} Bayesian approach to the\linebreak
\\[-12pt]
\hspace*{23pt}construction of an individual user trajectory in the system of distance learning&3&86--93\\
\Avtors{Bosov~A.\,V. and Stefanovich~A.\,I.} Stochastic differential system output control by the quadratic\linebreak
\\[-12pt]
\hspace*{23pt}criterion. IV. Alternative numerical decision&1&24--30\\
\Avtors{Briukhov~D.\,O., Stupnikov~S.\,A., Kovalev~D.\,Yu., and Shanin~I.\,A.} Neurophysiology as a subject\linebreak
\\[-12pt]
\hspace*{23pt}domain for~data intensive problem solving&1&40--47\\
\Avtors{Budzko~V.\,I., Yadrintsev~V.\,V., Sochenkov~I.\,V., Korolev~V.\,I., and Belenkov~V.\,G.} Extraction of confidentiality markers from texts under conditions of high uncertainty in systems with\linebreak
\\[-12pt]
\hspace*{23pt}data intensive usage&4&69--76\\
\Avtors{Danilishin~A.\,R. and Golembiovsky~D.\,Yu.} Estimating the fair value of options based on\linebreak
\\[-12pt]
\hspace*{23pt}ARIMA--GARCH models with errors distributed according to the Johnson's $S_u$ law&4&83--90\\
\Avtors{Danilishin~A.\,R. and Golembiovsky~D.\,Yu.} Risk-neutral dynamics for the ARIMA-GARCH\linebreak
\\[-12pt]
\hspace*{23pt}random process with errors distributed according to the Johnson's $S_u$ law&1&48--55\\
\Avtors{Diachenko~Yu.\,G.} see Sokolov~I.\,A.&&\\
\Avtors{Dimentov~A.\,V.} see Krasnov~F.\,V.&&\\
\Avtors{Donskoy~V.\,I.} Optimization models extraction from data&3&109--118\\
\Avtors{Dubnov~Y.\,A.} see Popkov~Y.\,S.&&\\
\Avtors{Dulin~S.\,K., Dulina~N.\,G., and Ermakov~P.\,V.} Information fusion of documents&1&128--135\\
\Avtors{Dulina~N.\,G.} see Dulin~S.\,K.&&\\
\Avtors{Efrosinin~D.\,V.} see Kharin~P.\,A.&&\\
\Avtors{Ehrlich~L.\,I.} see Kozerenko~E.\,B.&&\\
\Avtors{Ermakov~P.\,V.} see Dulin~S.\,K.&&\\
\end{tabular}
}
\pagebreak

\def\leftfootline{\small{\textbf{\thepage}
\hfill INFORMATIKA I EE PRIMENENIYA~--- INFORMATICS AND APPLICATIONS\ \ \ 2020\
\ \ volume~14\ \ \ issue\ 4}
}%
 \def\rightfootline{\small{INFORMATIKA I EE PRIMENENIYA~---
INFORMATICS AND APPLICATIONS\ \ \ 2020\ \ \ volume~14\ \ \ issue\ 4
\hfill \textbf{\thepage}}}

\def\leftkol{2020 AUTHOR INDEX} % ENGLISH ABSTRACTS}

\def\rightkol{2020 AUTHOR INDEX} %ENGLISH ABSTRACTS}


\noindent
{\tabcolsep=3pt
\begin{tabular}{p{395.48108pt}cc}
&\textbf{Issue} & \textbf{Page}\\[6pt]
\Avtors{Fahrutdinov~R.\,Sh.} see Kostina~A.\,A.&&\\
\Avtors{Frenkel~S.\,L. and Zakharov~V.\,N.} Joint assessment of data predictability and quality pre-\linebreak
\\[-12pt]
\hspace*{23pt}dictors&2&40--49\\
\Avtors{Gaidamaka~Yu.\,V.} see Moskaleva~F.\,A.&&\\
\Avtors{Gavrilov~E.\,S.} see Abgaryan~K.\,K.&&\\
\Avtors{Golembiovsky~D.\,Yu.} see Danilishin~A.\,R.&&\\
\Avtors{Golembiovsky~D.\,Yu.} see Danilishin~A.\,R.&&\\
\Avtors{Goncharov~A.\,V. and Strijov~V.\,V.} Alignment of ordered set Cartesian product&1&31--39\\
\Avtors{Goncharov~A.\,A., Zatsman~I.\,M., and Kruzhkov~M.\,G.} Evolution of classifications in supracorpora\linebreak
\\[-12pt]
\hspace*{23pt}databases&4&108--116\\
\Avtors{Gorshenin~A.\,K. and Korolev~V.\,Yu.} Approximation of particle size distributions of lunar regolith\linebreak
\\[-12pt]
\hspace*{23pt}based on the resampling&2&50--57\\
\Avtors{Gorshenin~A.\,K., Korolev~V.\,Yu., and Shcherbinina~A.\,A.} Statistical estimation of distributions\linebreak
\\[-12pt]
\hspace*{23pt}of random coefficients in the Langevin stochastic differential equation&3&\hphantom{1}3--12\\
\Avtors{Gorshenin~A.\,K. and Kuzmin~V.\,Yu.} Analysis of configurations of LSTM networks for medium-\linebreak
\\[-12pt]
\hspace*{23pt}term vector forecasting&1&10--16\\
\Avtors{Grabovoy~A.\,V., Bakhteev~O.\,Yu., and Strijov~V.\,V.} Ordering the set of neural network parameters&2&58--65\\
\Avtors{Grusho~A.\,A., Timonina~E.\,E., Grusho~N.\,A., and Teryokhina~I.\,Yu.} Identifying anomalies using\linebreak
\\[-12pt]
\hspace*{23pt}metadata&3&76--80\\
\Avtors{Grusho~A.\,A., Zabezhailo~M.\,I., Smirnov~D.\,V., and Timonina~E.\,E.} On probabilistic estimates of\linebreak
\\[-12pt]
\hspace*{23pt}the validity of empirical conclusions&4&3--8\\
\Avtors{Grusho~A.\,A., Zabezhailo~M.\,I., and Timonina~E.\,E.} On causal representativeness of training\linebreak
\\[-12pt]
\hspace*{23pt}samples of precedents in diagnostic type tasks&1&80--86\\
\Avtors{Grusho~A.\,A.} see Grusho~N.\,A.&&\\
\Avtors{Grusho~N.\,A., Grusho~A.\,A., Zabezhailo~M.\,I., and Timonina~E.\,E.} Methods of finding the causes\linebreak
\\[-12pt]
\hspace*{23pt}of information technology failures by means of metadata&2&33--39\\
\Avtors{Grusho~N.\,A., Zabezhailo~M.\,I., Smirnov~D.\,V., Timonina~E.\,E., and Shorgin~S.\,Ya.} Mathematical\linebreak
\\[-12pt]
\hspace*{23pt}statistics in the task of identifying hostile insiders&3&71--75\\
\Avtors{Grusho~N.\,A.} see Grusho~A.\,A.&&\\
\Avtors{Kharin~P.\,A., Makeeva~E.\,D., Kochetkova~I.\,A., Efrosinin~D.\,V., and Shorgin~S.\,Ya.} Retrial\linebreak
\\[-12pt]
\hspace*{23pt}queuing model for analyzing joint URLLC and eMBB transmission in 5G networks&4&17--24\\
\Avtors{Khusainov~A.\,A.} Performance of the bounded pipeline&1&87--93\\
\Avtors{Kirikov~I.\,A.} see Rumovskaya~S.\,B.&&\\
\Avtors{Kirilyuk~I.\,L. and Sen'ko~O.\,V.} Selection of optimal complexity models by methods of nonparametric statistics (on the example of production function model of regions of the Russian\linebreak
\\[-12pt]
\hspace*{23pt}Federation)&2&111--118\\
\Avtors{Kochetkova~I.\,A.} see Kharin~P.\,A.&&\\
\Avtors{Kolbin~I.\,S.} see Abgaryan~K.\,K.&&\\
\Avtors{Korolev~V.\,I.} see Budzko~V.\,I.&&\\
\Avtors{Korolev~V.\,Yu.} On the distribution of the ratio of the sum of sample elements exceeding\linebreak
\\[-12pt]
\hspace*{23pt}a threshold to the total sum of sample elements.~I&3&35--43\\
\Avtors{Korolev~V.\,Yu.} On the distribution of the ratio of the sum of sample elements exceeding\linebreak
\\[-12pt]
\hspace*{23pt}a threshold to the total sum of sample elements.~II&4&33--36\\
\Avtors{Korolev~V.\,Yu.} see Gorshenin~A.\,K.&&\\
\Avtors{Korolev~V.\,Yu.} see Gorshenin~A.\,K.&&\\
\Avtors{Kostina~A.\,A., Mirin~A.\,Yu., Moldovyan~D.\,N., and Fahrutdinov~R.\,Sh.} Method for defining finite noncommutative associative algebras of arbitrary even dimension for development of the\linebreak
\\[-12pt]
\hspace*{23pt}postquantum cryptoschemes&1&\hphantom{1}94--100\\
\Avtors{Kovalev~D.\,Yu.} see Briukhov~D.\,O.&&\\
\Avtors{Kozerenko~E.\,B., Mikheev~M.\,Y., Somin~N.\,V., Ehrlich~L.\,I., and Kuznetsov~K.\,I.} Analytical\linebreak
\\[-12pt]
\hspace*{23pt}textology in intelligent processing systems for unstructured data&1&113--120\\
\Avtors{Krasnov~F.\,V., Dimentov~A.\,V., and Shvartsman~M.\,E.} Using topic models for pairwise comparison\linebreak
\\[-12pt]
\hspace*{23pt}of collections of scientific papers&3&129--135\\
\end{tabular}
}
\pagebreak

\def\leftfootline{\small{\textbf{\thepage}
\hfill INFORMATIKA I EE PRIMENENIYA~--- INFORMATICS AND APPLICATIONS\ \ \ 2020\
\ \ volume~14\ \ \ issue\ 4}
}%
 \def\rightfootline{\small{INFORMATIKA I EE PRIMENENIYA~---
INFORMATICS AND APPLICATIONS\ \ \ 2020\ \ \ volume~14\ \ \ issue\ 4
\hfill \textbf{\thepage}}}

\def\leftkol{2020 AUTHOR INDEX} % ENGLISH ABSTRACTS}

\def\rightkol{2020 AUTHOR INDEX} %ENGLISH ABSTRACTS}


\noindent
{\tabcolsep=3pt
\begin{tabular}{p{395.48108pt}cc}
&\textbf{Issue} & \textbf{Page}\\[6pt]
\Avtors{Krivenko~M.\,P.} Sequential analysis of serial measurements based on multivariate reference\linebreak
\\[-12pt]
\hspace*{23pt}regions&2&86--91\\
\Avtors{Kruzhkov~M.\,G.} see Goncharov~A.\,A.&&\\
\Avtors{Kudryavtsev~A.\,A. and Shestakov~O.\,V.} Method of logarithmic moments for estimating the\linebreak
\\[-12pt]
\hspace*{23pt}gamma-exponential distribution parameters&3&49--54\\
\Avtors{Kushnirenko~A.\,G.} see Betelin~V.\,B.&&\\
\Avtors{Kushnirenko~A.\,G.} see Betelin~V.\,B.&&\\
\Avtors{Kuzmin~V.\,Yu.} see Gorshenin~A.\,K.&&\\
\Avtors{Kuznetsov~K.\,I.} see Kozerenko~E.\,B.&&\\
\Avtors{Leonov~A.\,G.} see Betelin~V.\,B.&&\\
\Avtors{Makeeva~E.\,D.} see Kharin~P.\,A.&&\\
\Avtors{Malashenko~Yu.\,E. and Nazarova~I.\,A.} Approximation of the multiuser network feasible\linebreak
\\[-12pt]
\hspace*{23pt}flows set&3&81--85\\
\Avtors{Martyushova~Ya.\,G.} see Bosov~A.\,V.&&\\
\Avtors{Matyushenko~S.\,I. and Razumchik~R.\,V.} Stationary characteristics of discrete-time Geo$/G/1/\infty$\linebreak
\\[-12pt]
\hspace*{23pt}queue with batch arrivals and one queue skipping policy&4&25--32\\
\Avtors{Melnikov~A.\,V.} see Vokhmintcev~A.\,V.&&\\
\Avtors{Melnikov~S.\,Yu. and Samouylov~K.\,E.} Statistical properties of binary nonautonomous shift\linebreak
\\[-12pt]
\hspace*{23pt}registers with internal xor&2&80--85\\
\Avtors{Meykhanadzhyan~L.\,A. and Razumchik~R.\,V.} Stationary characteristics of $M/G/2/\infty$ queue\linebreak
\\[-12pt]
\hspace*{23pt}with identical servers, LIFO service, and resampling policy&2&66--71\\
\Avtors{Mikheev~M.\,Y.} see Kozerenko~E.\,B.&&\\
\Avtors{Milovanova~T.\,A. and Razumchik~R.\,V.} A single-server queueing system with LIFO service,\linebreak
\\[-12pt]
\hspace*{23pt}probabilistic priority, batch Poisson arrivals, and background customers&3&26--34\\
\Avtors{Mirin~A.\,Yu.} see Kostina~A.\,A.&&\\
\Avtors{Moldovyan~D.\,N.} see Kostina~A.\,A.&&\\
\Avtors{Moskaleva~F.\,A., Gaidamaka~Yu.\,V., and Shorgin~V.\,S.} Impact of the isolation parameters on\linebreak
\\[-12pt]
\hspace*{23pt}resource allocation in the network slicing model&4&\hphantom{1}9--16\\
\Avtors{Naumov~A.\,V.} see Bosov~A.\,V.&&\\
\Avtors{Naumov~V.\,A. and Samouylov~К.\,Е.} On Markovian and rational arrival processes.~I&3&13--19\\
\Avtors{Naumov~V.\,A. and Samouylov~K.\,E.} On Markovian and rational arrival processes.~II&4&37--46\\
\Avtors{Nazarova~I.\,A.} see Malashenko~Yu.\,E.&&\\
\Avtors{Novikov~D.\,A.} see Shnurkov~P.\,V.&&\\
\Avtors{Nuriev~V.\,A. and Zatsman~I.\,M.} Reducing the spectrum of translation models in supracorpora\linebreak
\\[-12pt]
\hspace*{23pt}databases&2&119--126\\
\Avtors{Pachganov~S.\,A.} see Vokhmintcev~A.\,V.&&\\
\Avtors{Popkov~A.\,Y.} see Popkov~Y.\,S.&&\\
\Avtors{Popkov~Y.\,S., Popkov~A.\,Y., and Dubnov~Y.\,A.} Deterministic and randomized methods of entropy\linebreak
\\[-12pt]
\hspace*{23pt}projection for dimensionality reduction problems&4&47--54\\
\Avtors{Popov~G.\,A., Simavoryan~S.\,Zh., Simonyan~A.\,R., and Ulitina~E.\,I.} Modeling of monitoring of\linebreak
\\[-12pt]
\hspace*{23pt}information security process on the basis of queuing systems&1&71--79\\
\Avtors{Popov~M.\,V. and Posypkin~M.\,A.} Approximation of the set of solutions of systems of nonlinear\linebreak
\\[-12pt]
\hspace*{23pt}inequalities using graphic accelerators&3&20--25\\
\Avtors{Posypkin~M.\,A.} see Popov~M.\,V.&&\\
\Avtors{Potanin~M.\,S., Vayser~K.\,O., Zholobov~V.\,A., and Strijov~V.\,V.} Deep learning neural network\linebreak
\\[-12pt]
\hspace*{23pt}structure optimization&4&55--62\\
\Avtors{Razumchik~R.\,V.} see Matyushenko~S.\,I.&&\\
\Avtors{Razumchik~R.\,V.} see Meykhanadzhyan~L.\,A.&&\\
\Avtors{Razumchik~R.\,V.} see Milovanova~T.\,A.&&\\
\Avtors{Rogdestvenski~Yu.\,V.} see Sokolov~I.\,A.&&\\
\Avtors{Rumovskaya~S.\,B. and Kirikov~I.\,A.} Conflict visual representation method in hybrid intelligent\linebreak
\\[-12pt]
\hspace*{23pt}multiagent systems&4&77--82\\
\Avtors{Samouylov~K.\,E.} see Ageev~K.\,A.&&\\
\end{tabular}
}
\pagebreak

\def\leftfootline{\small{\textbf{\thepage}
\hfill INFORMATIKA I EE PRIMENENIYA~--- INFORMATICS AND APPLICATIONS\ \ \ 2020\
\ \ volume~14\ \ \ issue\ 4}
}%
 \def\rightfootline{\small{INFORMATIKA I EE PRIMENENIYA~---
INFORMATICS AND APPLICATIONS\ \ \ 2020\ \ \ volume~14\ \ \ issue\ 4
\hfill \textbf{\thepage}}}

\def\leftkol{2020 AUTHOR INDEX} % ENGLISH ABSTRACTS}

\def\rightkol{2020 AUTHOR INDEX} %ENGLISH ABSTRACTS}


\noindent
{\tabcolsep=3pt
\begin{tabular}{p{395.48108pt}cc}
&\textbf{Issue} & \textbf{Page}\\[6pt]
\Avtors{Samouylov~K.\,E.} see Melnikov~S.\,Yu.&&\\
\Avtors{Samouylov~K.\,E.} see Naumov~V.\,A.&&\\
\Avtors{Samouylov~K.\,Е.} see Naumov~V.\,A.&&\\
\Avtors{Sapunova~A.\,P.} see Bosov~A.\,V.&&\\
\Avtors{Satin~Ya.\,A., Zeifman~A.\,I., and Shilova~G.\,N.} On approaches to constructing limiting regimes\linebreak
\\[-12pt]
\hspace*{23pt}for some queuing models&2&3--9\\
\Avtors{Semenov~A.\,L.} see Betelin~V.\,B.&&\\
\Avtors{Sen'ko~O.\,V.} see Kirilyuk~I.\,L.&&\\
\Avtors{Serebryanskii~S.\,M. and Tyrsin~A.\,N.} Improvement of the accuracy of solution of tasks for the\linebreak
\\[-12pt]
\hspace*{23pt}account of the construction of boundary conditions&1&56--62\\
\Avtors{Sevastianov~L.\,A. and Shchetinin~E.\,Yu.} On methods for improving the accuracy of multiclass\linebreak
\\[-12pt]
\hspace*{23pt}classification on imbalanced data&1&63--70\\
\Avtors{Shanin~I.\,A.} see Briukhov~D.\,O.&&\\
\Avtors{Shcherbinina~A.\,A.} see Gorshenin~A.\,K.&&\\
\Avtors{Shchetinin~E.\,Yu.} see Sevastianov~L.\,A.&&\\
\Avtors{Shestakov~O.\,V.} Asymptotic regularity of the wavelet methods of inverting linear homogeneous\linebreak
\\[-12pt]
\hspace*{23pt}operators from observations recorded at random times&1&3--9\\
\Avtors{Shestakov~O.\,V.} Asymptotics of the mean-square risk estimate in the problem of inverting the\linebreak
\\[-12pt]
\hspace*{23pt}Radon transform from projections registered on a random grid&2&29--32\\
\Avtors{Shestakov~O.\,V.} On the statistical properties of risk estimate in the problem of inverting the\linebreak
\\[-12pt]
\hspace*{23pt}Radon transform with a random volume of projection data&3&44--48\\
\Avtors{Shestakov~O.\,V.} see Kudryavtsev~A.\,A.&&\\
\Avtors{Shihiev~F.\,Sh.} see Shihiev~Sh.\,B.&&\\
\Avtors{Shihiev~Sh.\,B. and Shihiev~F.\,Sh.} Incapsulation of semantic representations into elements of\linebreak
\\[-12pt]
\hspace*{23pt}a grammar&1&121--127\\
\Avtors{Shilova~G.\,N.} see Satin~Ya.\,A.&&\\
\Avtors{Shnurkov~~P.\,V. and Adamova~K.\,A.} Solution of the unconditional extremal problem for a~linear-\linebreak
\\[-12pt]
\hspace*{23pt}fractional integral functional dependent on the parameter&2&\hphantom{1}98--103\\
\Avtors{Shnurkov~P.\,V. and Novikov~D.\,A.} On the concept of a stochastic model with control at the\linebreak
\\[-12pt]
\hspace*{23pt}moments of the process at the border of a presented subset of multiple states&3&101--108\\
\Avtors{Shorgin~S.\,Ya.} see Ageev~K.\,A.&&\\
\Avtors{Shorgin~S.\,Ya.} see Grusho~N.\,A.&&\\
\Avtors{Shorgin~S.\,Ya.} see Kharin~P.\,A.&&\\
\Avtors{Shorgin~V.\,S.} see Moskaleva~F.\,A.&&\\
\Avtors{Shvartsman~M.\,E.} see Krasnov~F.\,V.&&\\
\Avtors{Simavoryan~S.\,Zh.} see Popov~G.\,A.&&\\
\Avtors{Simonyan~A.\,R.} see Popov~G.\,A.&&\\
\Avtors{Smirnov~D.\,V.} see Grusho~A.\,A.&&\\
\Avtors{Smirnov~D.\,V.} see Grusho~N.\,A.&&\\
\Avtors{Sochenkov~I.\,V.} see Budzko~V.\,I.&&\\
\Avtors{Sokolov~I.\,A., Stepchenkov~Yu.\,A., Diachenko~Yu.\,G., and Rogdestvenski~Yu.\,V.} Improvement of\linebreak
\\[-12pt]
\hspace*{23pt}self-timed circuit soft error tolerance&4&63--68\\
\Avtors{Somin~N.\,V.} see Kozerenko~E.\,B.&&\\
\Avtors{Sopin~E.\,S.} see Ageev~K.\,A.&&\\
\Avtors{Soprunov~S.\,F.} see Betelin~V.\,B.&&\\
\Avtors{Stefanovich~A.\,I.} see Bosov~A.\,V.&&\\
\Avtors{Stepchenkov~Yu.\,A.} see Sokolov~I.\,A.&&\\
\Avtors{Strijov~V.\,V.} see Goncharov~A.\,V.&&\\
\Avtors{Strijov~V.\,V.} see Grabovoy~A.\,V.&&\\
\Avtors{Strijov~V.\,V.} see Potanin~M.\,S.&&\\
\Avtors{Stupnikov~S.\,A.} see Briukhov~D.\,O.&&\\
\Avtors{Teryokhina~I.\,Yu.} see Grusho~A.\,A.&&\\
\Avtors{Timonina~E.\,E.} see Grusho~A.\,A.&&\\
\end{tabular}
}
\pagebreak

\def\leftfootline{\small{\textbf{\thepage}
\hfill INFORMATIKA I EE PRIMENENIYA~--- INFORMATICS AND APPLICATIONS\ \ \ 2020\
\ \ volume~14\ \ \ issue\ 4}
}%
 \def\rightfootline{\small{INFORMATIKA I EE PRIMENENIYA~---
INFORMATICS AND APPLICATIONS\ \ \ 2020\ \ \ volume~14\ \ \ issue\ 4
\hfill \textbf{\thepage}}}

\def\leftkol{2020 AUTHOR INDEX} % ENGLISH ABSTRACTS}

\def\rightkol{2020 AUTHOR INDEX} %ENGLISH ABSTRACTS}


\noindent
{\tabcolsep=3pt
\begin{tabular}{p{395.48108pt}cc}
&\textbf{Issue} & \textbf{Page}\\[6pt]
\Avtors{Timonina~E.\,E.} see Grusho~A.\,A.&&\\
\Avtors{Timonina~E.\,E.} see Grusho~A.\,A.&&\\
\Avtors{Timonina~E.\,E.} see Grusho~N.\,A.&&\\
\Avtors{Timonina~E.\,E.} see Grusho~N.\,A.&&\\
\Avtors{Tyrsin~A.\,N.} see Serebryanskii~S.\,M.&&\\
\Avtors{Ulitina~E.\,I.} see Popov~G.\,A.&&\\
\Avtors{Vayser~K.\,O.} see Potanin~M.\,S.&&\\
\Avtors{Vokhmintcev~A.\,V., Melnikov~A.\,V., and Pachganov~S.\,A.} Simultaneous localization and mapping method in  three-dimensional space based on the combined solution of the  point--point\linebreak
\\[-12pt]
\hspace*{23pt}variation problem ICP for an affine transformation&1&101--112\\
\Avtors{Yadrintsev~V.\,V.} see Budzko~V.\,I.&&\\
\Avtors{Yarkina~N.\,V.} see Ageev~K.\,A.&&\\
\Avtors{Zabezhailo~M.\,I.} see Grusho~A.\,A.&&\\
\Avtors{Zabezhailo~M.\,I.} see Grusho~A.\,A.&&\\
\Avtors{Zabezhailo~M.\,I.} see Grusho~N.\,A.&&\\
\Avtors{Zabezhailo~M.\,I.} see Grusho~N.\,A.&&\\
\Avtors{Zakharov~V.\,N.} see Frenkel~S.\,L.&&\\
\Avtors{Zatsman~I.\,M.} Problem-oriented verifying the completeness  of~temporal ontologies and\linebreak
\\[-12pt]
\hspace*{23pt}filling~conceptual lacunas&3&119--128\\
\Avtors{Zatsman~I.\,M.} see Goncharov~A.\,A.&&\\
\Avtors{Zatsman~I.\,M.} see Nuriev~V.\,A.&&\\
\Avtors{Zeifman~A.\,I.} see Satin~Ya.\,A.&&\\
\Avtors{Zholobov~V.\,A.} see Potanin~M.\,S.&&\\
\end{tabular}
}

%\thispagestyle{myheadings}
\def\leftfootline{\small{\textbf{\thepage}
\hfill INFORMATIKA I EE PRIMENENIYA~--- INFORMATICS AND APPLICATIONS\ \ \ 2020\
\ \ volume~14\ \ \ issue\ 4}
}%
 \def\rightfootline{\small{INFORMATIKA I EE PRIMENENIYA~---
INFORMATICS AND APPLICATIONS\ \ \ 2020\ \ \ volume~14\ \ \ issue\ 4
\hfill \textbf{\thepage}}}

 \label{end\stat}

\newpage


%\linebreak
%\\[-12pt]
%\hspace*{23pt}

%   \vspace*{-48pt}

\begin{center}
\vspace*{6pt}
\mbox{%
%\epsfxsize=50mm %56.519mm  
%\epsfbox{smu-1.eps} 

\epsfxsize=50mm %46.402 mm
\epsfbox{nec-rb.eps}
}
%\end{center}

\vspace*{9pt} %Академик


%   \begin{center}
\fbox{\large\textbf{Рустем Бадриевич Сейфуль-Мулюков}}\\[6pt]
\textbf{\large 1928--2020}
   \end{center}


   %\vspace*{2.5mm}

   \vspace*{5mm}

   \thispagestyle{empty}

%\

%\vspace*{-12pt}

  
      Редакция журнала <<Информатика и~её применения>> с глубоким 
      прискорбием сообщают, что 17~марта 2020~г.\ на 93-м~году жизни 
      скончался заведующий редакцией журнала, главный научный сотрудник Федерального исследовательского центра <<Информатика и~управление>> Российской академии наук
      Рустем Бадриевич Сейфуль-Мулюков.
           
     Всю свою жизнь Рустем Бадриевич посвятил служению науке. Закончив в~1956~г.\ аспирантуру Московского ордена Трудового Красного знамени Нефтяного института им.\ академика
     И.\,М.~Губкина, он прошел путь от заведующего отделом Института геологии зарубежных стран Министерства геологии СССР до заместителя директора ВИНИТИ
     АН СССР, доктора гео\-ло\-го-ми\-не\-ра\-ло\-ги\-че\-ских наук, профессора.
     
     С марта 2002~г.\ Рустем Бадриевич успешно применял свои знания и~организационный талант в ИПИ
     РАН (в~дальнейшем~--- ФИЦ ИУ РАН), в~котором руководил лабораторией и~отделом, занимающимися вопросами технологий информационной технической деятельности. 
Р.\,Б.~Сейфуль-Мулюков, являясь автором значительного количества научных трудов и~монографий по геологии, информационным технологиям и~теоретической информатике, осуществлял организацию издания монографий ИПИ РАН и~ФИЦ ИУ РАН, библиографий научных сотрудников Центра.
     
     Р.\,Б.~Сейфуль-Мулюков являлся заведующим редакцией журналов <<Информатика и~её применения>> и~<<Системы и~средства информатики>>, членом редколлегии журнала <<Системы и~средства информатики>>. Он вложил огромный вклад в становление и~развитие этих журналов, организацию их регистрации, функционирования, редактуры и~издания. Включение этих журналов в ряд отечественных и~зарубежных информационных баз и~систем цитирования во многом является его личной заслугой.
     
     На всех занимаемых должностях Рустем Бадриевич отличался высоким профессионализмом, преданностью делу и~вниманием к коллегам.
     
     \thispagestyle{empty}
     
     Рустема Бадриевича отличали доброта, отзывчивость, неиссякаемый
      оптимизм, простота и~сердечность.
     
     Коллеги Рустема Бадриевича запомнят его как многогранного в~своих увлечениях человека, живописца,
     эрудита и~энциклопедиста, интересующегося историей, литературой и~искусством.
     
     Выражаем глубокое
     соболезнование семье, родственникам, друзьям и~коллегам по работе в~связи с~тяжелой невосполнимой утратой.
     Светлый образ Рустема Бадриевича навсегда сохранится в~нашей памяти.
     

      

%\def\stat{cont}
{%\hrule\par
%\vskip 7pt % 7pt
\raggedleft\Large \bf%\baselineskip=3.2ex
А\,В\,Т\,О\,Р\,С\,К\,И\,Й\ \ У\,К\,А\,З\,А\,Т\,Е\,Л\,Ь\ \ З\,А\ \ 2\,0\,1\,0 г. \vskip 17pt
    \hrule
    \par
\vskip 21pt plus 6pt minus 3pt }

\label{st\stat}

\def\tit{\ }

\def\aut{\ }
\def\auf{\ }

\def\leftkol{\ } % ENGLISH ABSTRACTS}

\def\rightkol{\ } %АВТОРСКИЙ УКАЗАТЕЛЬ ЗА 2010 г.} %ENGLISH ABSTRACTS}

\titele{\tit}{\aut}{\auf}{\leftkol}{\rightkol}

\vspace*{-12pt}

{\tabcolsep=3pt
\begin{tabular}{p{388pt}rr}
&\textbf{Выпуск} & \textbf{Стр.}\\[6pt]
\hangindent=23pt\noindent\textbf{Арутюнян~А.\,Р.} Моделирование влияния деформаций отпечатков пальцев на 
точность\linebreak
\vspace*{-12pt}\\
\hspace*{23pt}дактилоскопической идентификации$\dotfill$&1&51\\
\hangindent=23pt\noindent\textbf{Архипов~О.\,П., Зыкова~З.\,П.} Интеграция гетерогенной информации о цветных 
пикселях\linebreak
\vspace*{-12pt}\\
\hspace*{23pt}и их цветовосприятии$\dotfill$&4&15\\
\hangindent=23pt\noindent\textbf{Баранов~С.\,И., Френкель~С.\,Л., Захаров~В.\,Н.} Полуформальная верификация 
цифрового устройства с конвейером, основанная на использовании алгоритмических машин\linebreak
\vspace*{-12pt}\\
\hspace*{23pt}состояния$\dotfill$&4&49\\
\textbf{Бекетова~И.\,В.} см.~Каратеев~С.\,Л.&&\\
\textbf{Белоусов~В.\,В.} см.~Синицын~И.\,Н.&&\\
\hangindent=23pt\noindent\textbf{Бенинг~В.\,Е., Королев~Р.\,А.} О предельном поведении мощностей критериев в 
случае\linebreak
\vspace*{-12pt}\\
\hspace*{23pt}распределения Лапласа$\dotfill$&2&63\\
\hangindent=23pt\noindent\textbf{Бенинг~В.\,Е., Сипина~А.\,В.} Асимптотическое разложение для мощности 
критерия,\linebreak
\vspace*{-12pt}\\
\hspace*{23pt}основанного на выборочной медиане, в случае распределения Лапласа$\dotfill$&1&18\\
\textbf{Бондаренко~А.\,В.} см.~Каратеев~С.\,Л.&&\\
\hangindent=23pt\noindent\textbf{Бородина~А.\,В., Морозов~Е.\,В.} Об оценивании асимптотики вероятности 
большого\linebreak
\vspace*{-12pt}\\
\hspace*{23pt}уклонения стационарной регенеративной очереди с одним прибором$\dotfill$&3&29\\
\hangindent=23pt\noindent\textbf{Бунтман~Н.\,В., Минель~Ж.-Л., Ле~Пезан~Д., Зацман~И.\,М.} Типология и 
компьютерное\linebreak
\vspace*{-12pt}\\
\hspace*{23pt}моделирование трудностей перевода$\dotfill$&3&77\\
\textbf{Визильтер~Ю.\,В.} см.~Каратеев~С.\,Л.&&\\
\hangindent=23pt\noindent\textbf{Гавриленко~С.\,В.} Оценки скорости сходимости распределений случайных сумм с 
безгранично делимыми индексами к нормальному закону$\dotfill$&4&81\\
\hangindent=23pt\noindent\textbf{Григорьева~М.\,Е., Шевцова~И.\,Г.} Уточнение неравенства 
Каца--Берри--Эссеена$\dotfill$&2&75\\
\hangindent=23pt\noindent\textbf{Грушо~А.\,А., Грушо~Н.\,А., Тимонина~Е.\,Е.} Поиск конфликтов в политиках 
безопасности: модель случайных графов$\dotfill$&3&38\\
\textbf{Грушо~Н.\,А.} см.~Грушо~А.\,А.&&\\
\hangindent=23pt\noindent\textbf{Гудков~В.\,Ю.} Математические модели изображения отпечатка пальца на основе 
описания линий$\dotfill$&1&58\\
\textbf{Гуртов~А.\,В.} см.~Лукьяненко~А.\,С.&&\\
\textbf{Желтов~С.\,Ю.} см.~Каратеев~С.\,Л.&&\\
\hangindent=23pt\noindent\textbf{Захаров~А.\,А., Серебряков~В.\,А.} Система управления электронной библиотекой 
LibMeta$\dotfill$&4&2\\
\textbf{Захаров~В.\,Н.} см.~Баранов~С.\,И.&&\\
\textbf{Захарова~Т.\,В.} см.~Матвеева~С.\,С.&&\\
\hangindent=23pt\noindent\textbf{Зацаринный~А.\,А., Чупраков~К.\,Г.} Некоторые аспекты выбора технологии для 
постро-\linebreak
\vspace*{-12pt}\\
\hspace*{23pt}ения систем отображения информации ситуационного центра$\dotfill$&3&59\\
\textbf{Зацман~И.\,М.} см.~Бунтман~Н.\,В.&&\\
\hangindent=23pt\noindent\textbf{Зейфман~А.\,И., Коротышева~А.\,В., Сатин~Я.\,А., Шоргин~С.\,Я.} Об 
устойчивости нестаци-\linebreak
\vspace*{-12pt}\\
\hspace*{23pt}онарных систем обслуживания с катастрофами$\dotfill$&3&9\\
\textbf{Зыкова~З.\,П.} см.~Архипов~О.\,П.&&\\
\hangindent=23pt\noindent\textbf{Илюшин~Г.\,Я., Соколов~И.\,А.} Организация управляемого доступа пользователей 
к\linebreak
\vspace*{-12pt}\\
\hspace*{23pt}разнородным ведомственным информационным ресурсам$\dotfill$&1&24\\
\hangindent=23pt\noindent\textbf{Кавагучи~Ю., Ульянов~В.\,В., Фуджикоши~Я.} Приближения для статистик, 
описывающих\linebreak
\vspace*{-12pt}\\
\hspace*{23pt}геометрические свойства данных большой размерности, с оценками 
ошибок$\dotfill$&1&12\\
\hangindent=23pt\noindent\textbf{Каратеев~С.\,Л., Бекетова~И.\,В., Ососков~М.\,В., Князь~В.\,А., 
Визильтер~Ю.\,В., Бондаренко~А.\,В., Желтов~С.\,Ю.} Автоматизированный контроль 
качества цифровых\linebreak
\vspace*{-12pt}\\
\hspace*{23pt}изображений для персональных документов$\dotfill$&1&65\\
\end{tabular}
}

\pagebreak

\def\leftkol{АВТОРСКИЙ УКАЗАТЕЛЬ ЗА 2010 г.} % ENGLISH ABSTRACTS}

\def\rightkol{АВТОРСКИЙ УКАЗАТЕЛЬ ЗА 2010 г.} %ENGLISH ABSTRACTS}

{\tabcolsep=3pt
\begin{tabular}{p{388pt}rr}
&\textbf{Выпуск} & \textbf{Стр.}\\[3pt]
\hangindent=23pt\noindent\textbf{Козеренко~Е.\,Б.} Лингвистические фильтры в статистических моделях машинного\linebreak
\vspace*{-12pt}\\
\hspace*{23pt}перевода$\dotfill$&2&83\\
\hangindent=23pt\noindent\textbf{Козеренко~Е.\,Б., Кузнецов~И.\,П.} Когнитивно-лингвистические представления в 
систе-\linebreak
\vspace*{-12pt}\\
\hspace*{23pt}мах обработки текстов$\dotfill$&3&69\\
\textbf{Князь~В.\,А.} см.~Каратеев~С.\,Л.&&\\
\hangindent=23pt\noindent\textbf{Колесников~А.\,В., Солдатов~С.\,А.} Алгоритм координации для гибридной 
интеллектуальной системы решения сложной задачи оперативно-производственного\linebreak
\vspace*{-12pt}\\
\hspace*{23pt}планирования$\dotfill$&4&61\\
\hangindent=23pt\noindent\textbf{Коновалов~М.\,Г.} О планировании потоков в системах вычислительных 
ресурсов$\dotfill$&2&3\\
\textbf{Конушин~А.\,С.} см.~Конушин~В.\,С.&&\\
\hangindent=23pt\noindent\textbf{Конушин~В.\,С., Кривовязь~Г.\,Р., Конушин~А.\,С.} Алгоритм распознавания людей 
в видео-\linebreak
\vspace*{-12pt}\\
\hspace*{23pt}последовательности по одежде$\dotfill$&1&74\\
\textbf{Корепанов~Э.\, Р.} см.~Синицын~И.\,Н.&&\\
\textbf{Королев~В.\,Ю.} см.~Соколов~И.\,А.&&\\
\textbf{Королев~Р.\,А.} см.~Бенинг~В.\,Е.&&\\
\textbf{Коротышева~А.\,В.} см.~Зейфман~А.\,И.&&\\
\hangindent=23pt\noindent\textbf{Кривенко~М.\,П.} Непараметрическое оценивание элементов байесовского 
клас\-си-\linebreak
\vspace*{-12pt}\\
\hspace*{23pt}фикатора$\dotfill$&2&13\\
\textbf{Кривовязь~Г.\,Р.} см.~Конушин~В.\,С.&&\\
\textbf{Крылов~А.\,С.} см.~Павельева~Е.\,А.&&\\
\hangindent=23pt\noindent\textbf{Крылов~В.\,А.} Моделирование и классификация многоканальных дистанционных\linebreak
\vspace*{-12pt}\\
\hspace*{23pt}изображений с использованием копул$\dotfill$&4&34\\
\hangindent=23pt\noindent\textbf{Крючин~О.\,В.} Разработка параллельных эвристических алгоритмов подбора 
весовых\linebreak
\vspace*{-12pt}\\
\hspace*{23pt}коэффициентов искусственной нейтронной сети$\dotfill$&2&53\\
\hangindent=23pt\noindent\textbf{Кудрявцев~А.\,А., Шоргин~С.\,Я.} Байесовские модели массового обслуживания и 
надеж-\linebreak
\vspace*{-12pt}\\
\hspace*{23pt}ности: характеристики среднего числа заявок в системе $M\vert M \vert 1\vert 
\infty$$\dotfill$&3&16\\
\hangindent=23pt\noindent\textbf{Кузнецов~А.\,А.} Связь между временными и структурно-топологическими 
характери-\linebreak
\vspace*{-12pt}\\
\hspace*{23pt}стиками диаграмм ритма сердца здоровых людей$\dotfill$&4&39\\
\textbf{Кузнецов~И.\,П.} см.~Козеренко~Е.\,Б.&&\\
\textbf{Ле~Пезан~Д.} см.~Бунтман~Н.\,В.&&\\
\hangindent=23pt\noindent\textbf{Лукьяненко~А.\,С., Морозов~Е.\,В., Гуртов~А.\,В.} Анализ сетевого протокола с общей 
функ-\linebreak
\vspace*{-12pt}\\
\hspace*{23pt}цией расширения окна передачи сообщения при конфликтах$\dotfill$&2&46\\
\hangindent=23pt\noindent\textbf{Лямин~О.\,О.} О предельном поведении мощностей критериев в случае обобщенного\linebreak
\vspace*{-12pt}\\
\hspace*{23pt}распределения Лапласа$\dotfill$&3&47\\
\hangindent=23pt\noindent\textbf{Маркин~А.\,В., Шестаков~О.\,В.} Асимптотики оценки риска при пороговой 
обработке\linebreak
\vspace*{-12pt}\\
\hspace*{23pt}вейвлет-вейглет коэффициентов в задаче томографии$\dotfill$&2&36\\
\hangindent=23pt\noindent\textbf{Матвеева~С.\,С., Захарова~Т.\,В.} Сети массового обслуживания с наименьшей 
длиной\linebreak
\vspace*{-12pt}\\
\hspace*{23pt}очереди$\dotfill$&3&22\\
\hangindent=23pt\noindent\textbf{Матюшенко~С.\,И.} Стационарные характеристики двухканальной системы 
обслужива-\linebreak
\vspace*{-12pt}\\
\hspace*{23pt}ния с переупорядочиванием заявок и распределениями фазового типа$\dotfill$&4&68\\
\textbf{Минель~Ж.-Л.} см.~Бунтман~Н.\,В.&&\\
\textbf{Морозов~Е.\,В.} см.~Бородина~А.\,В.&&\\
\textbf{Морозов~Е.\,В.} см.~Лукьяненко~А.\,С.&&\\
\textbf{Ососков~М.\,В.} см.~Каратеев~С.\,Л.&&\\
\hangindent=23pt\noindent\textbf{Павельева~Е.\,А., Крылов~А.\,С.} Поиск и анализ ключевых точек радужной 
оболочки\linebreak
\vspace*{-12pt}\\
\hspace*{23pt}глаза методом преобразования Эрмита$\dotfill$&1&79\\
\textbf{Печинкин~А.\,В.} см.~Френкель~С.\,Л.,&&\\
\hangindent=23pt\noindent\textbf{Протасов~В.\,И.} Составление субъективного портрета с использованием 
эволюционно-\linebreak
\vspace*{-12pt}\\
\hspace*{23pt}го морфинга и квалиметрия метода$\dotfill$&1&83\\
\hangindent=23pt\noindent\textbf{Рудаков~К.\,В., Торшин~И.\,Ю.} Вопросы разрешимости задачи распознавания 
вторичной\linebreak
\vspace*{-12pt}\\
\hspace*{23pt}структуры белка$\dotfill$&2&25\\
\textbf{Сатин~Я.\,А.} см.~Зейфман~А.\,И.&&\\
\hangindent=23pt\noindent\textbf{Сейфуль-Мулюков~Р.\,Б.} Нефть как носитель информации о своем 
происхождении,\linebreak
\vspace*{-12pt}\\
\hspace*{23pt}структуре и эволюции$\dotfill$&1&41\\
\end{tabular}
}

{\tabcolsep=3pt
\begin{tabular}{p{388pt}rr}
&\textbf{Выпуск} & \textbf{Стр.}\\[6pt]
\textbf{Семендяев~Н.\,Н.} см.~Синицын~И.\,Н.&&\\
\textbf{Серебряков~В.\,А.} см.~Захаров~А.\,А.&&\\
\textbf{Синицын~В.\,И.} см.~Синицын~И.\,Н.&&\\
\hangindent=23pt\noindent\textbf{Синицын~И.\,Н., Синицын~В.\,И., Корепанов~Э.\, Р., Белоусов~В.\,В., 
Семендяев~Н.\,Н.} Оперативное построение информационных моделей движения полюса 
Земли\linebreak
\vspace*{-12pt}\\
\hspace*{23pt}методами линейных и линеаризованных фильтров$\dotfill$&1&2\\
\textbf{Сипина~А.\,В.} см.~Бенинг~В.\,Е.&&\\
\hangindent=23pt\noindent\textbf{Соколов~И.\,А.} О работах заслуженного деятеля науки Российской Федерации 
И.\,Н.~Синицына в области информационных технологий и автоматизации (к 70-летию\linebreak
\vspace*{-12pt}\\
\hspace*{23pt}со дня рождения)$\dotfill$&3&84\\
\textbf{Соколов~И.\,А.} см.~Илюшин~Г.\,Я.&&\\
\hangindent=23pt\noindent\textbf{Соколов~И.\,А., Королев~В.\,Ю.} Предисловие$\dotfill$&2&2\\
\textbf{Солдатов~С.\,А.} см.~Колесников~А.\,В.&&\\
\hangindent=23pt\noindent\textbf{Степанов~С.\,Ю.} Использование координатного метода фрагментации 
коммутаторной\linebreak
\vspace*{-12pt}\\
\hspace*{23pt}нейронной сети для сокращения трафика$\dotfill$&2&57\\
\textbf{Тимонина~Е.\,Е.} см.~Грушо~А.\,А.&&\\
\textbf{Торшин~И.\,Ю.} см.~Рудаков~К.\,В.&&\\
\textbf{Ульянов~В.\,В.} см.~Кавагучи~Ю.&&\\
\textbf{Фазекаш~И.} см.~Чупрунов~А.\,Н.&&\\
\textbf{Френкель~С.\,Л.} см.~Баранов~С.\,И.&&\\
\hangindent=23pt\noindent\textbf{Френкель~С.\,Л., Печинкин~А.\,В.} Оценка времени самовосстановления в 
цифровых\linebreak
\vspace*{-12pt}\\
\hspace*{23pt}системах после сбоев, вызываемых переходными помехами$\dotfill$&3&2\\
\textbf{Фуджикоши~Я.} см.~Кавагучи~Ю.&&\\
\hangindent=23pt\noindent\textbf{Цискаридзе~А.\,К.} Математическая модель и метод восстановления позы человека 
по\linebreak
\vspace*{-12pt}\\
\hspace*{23pt}стереопаре силуэтных изображений$\dotfill$&4&27\\
\hangindent=23pt\noindent\textbf{Чупраков~К.\,Г.} К вопросу о размещении коллективных средств отображения в 
ситуа-\linebreak
\vspace*{-12pt}\\
\hspace*{23pt}ционном зале с заданными параметрами$\dotfill$&4&89\\
\textbf{Чупраков~К.\,Г.} см.~Зацаринный~А.\,А.&&\\
\hangindent=23pt\noindent\textbf{Чупрунов~А.\,Н., Фазекаш~И.} Законы повторного логарифма для числа 
безошибочных\linebreak
\vspace*{-12pt}\\
\hspace*{23pt}блоков при помехоустойчивом кодировании$\dotfill$&3&42\\
\textbf{Шевцова~И.\,Г.} см.~Григорьева~М.\,Е.&&\\
\hangindent=23pt\noindent\textbf{Шестаков~О.\,В.} Аппроксимация распределения оценки риска пороговой 
обработки вейвлет-коэффициентов нормальным распределением при использовании 
выбо-\linebreak
\vspace*{-12pt}\\
\hspace*{23pt}рочной дисперсии$\dotfill$&4&73\\
\textbf{Шестаков~О.\,В.} см.~Маркин~А.\,В.&&\\
\textbf{Шоргин~С.\,Я.} см.~Зейфман~А.\,И.&&\\
\textbf{Шоргин~С.\,Я.} см.~Кудрявцев~А.\,А.&&\\
\end{tabular}
}

%\thispagestyle{myheadings}
\def\leftfootline{\small{\textbf{\thepage}
\hfill ИНФОРМАТИКА И ЕЁ ПРИМЕНЕНИЯ\ \ \ том~4\ \ \ выпуск~4\ \ \ 2010}
}%
 \def\rightfootline{\small{ИНФОРМАТИКА И ЕЁ ПРИМЕНЕНИЯ\ \ \ том~4\ \ \ выпуск~4\ \ \ 2010
 \hfill \textbf{\thepage}}}
 \label{end\stat}
%
%Том 10 Выпуск 1-4 Год 2016

\def\stat{cont-e}
{%\hrule\par
%\vskip 7pt % 7pt
\raggedleft\Large \bf%\baselineskip=3.2ex
2\,0\,1\,6\ \ A\,U\,T\,H\,O\,R\ \ I\,N\,D\,E\,X \vskip 17pt
 \hrule
 \par
\vskip 21pt plus 6pt minus 3pt }

\label{st\stat}

\def\tit{\ }

\def\aut{\ }
\def\auf{\ }

\def\leftkol{\ } %2016 AUTHOR INDEX} % ENGLISH ABSTRACTS}

\def\rightkol{\ } %2016 AUTHOR INDEX} %ENGLISH ABSTRACTS}

\titele{\tit}{\aut}{\auf}{\leftkol}{\rightkol}

\def\leftfootline{\small{\textbf{\thepage}
\hfill INFORMATIKA I EE PRIMENENIYA~--- INFORMATICS AND APPLICATIONS\ \ \ 2016\
\ \ volume~10\ \ \ issue\ 4}
}%
 \def\rightfootline{\small{INFORMATIKA I EE PRIMENENIYA~--- INFORMATICS AND APPLICATIONS\ \ \ 2016\ \ \ volume~10\ \ \ issue\ 4
\hfill \textbf{\thepage}}}

\vspace*{-12pt}
\vspace*{-18pt}

{\tabcolsep=2.8pt
\begin{tabular}{p{382pt}cc}
&\textbf{Issue} & \textbf{Page}\\[6pt]
\Avtors{Agalarov~M.\,Ya.} see~Agalarov~Ya.\,M.&&\\
\Avtors{Agalarov~Ya.\,M., Agalarov~M.\,Ya., and
Shorgin~V.\,S.} About the optimal threshold of queue\linebreak
\\[-12pt]
\hspace*{23pt}length in a~particular problem of profit maximization
in the $M/G/1$ queuing system&2&70--79\\
\Avtors{Alexeyevsky~D.\,A.} BioNLP ontology extraction from 
a~restricted language corpus with\linebreak
\\[-12pt]
\hspace*{23pt}context-free grammars&1&119--128\\
\Avtors{Andreev~S.\,D.} see~Gaidamaka~Yu.\,V.&&\\
\Avtors{Andreev~S.\,D.} see~Ometov~A.\,Ya.&&\\
\Avtors{Arkhipov~O.\,P., Arkhipov~P.\,O., and Sidorkin~I.\,I.} The
option to create a~local coordinate\linebreak
\\[-12pt]
\hspace*{23pt}system for synchronization of selected images&3&91--97\\
\Avtors{Arkhipov~P.\,O.} see~Arkhipov~O.\,P.&&\\
\Avtors{Belousov~V.\,V.} see~Shnurkov~P.\,V.&&\\
\Avtors{Belousov~V.\,V.} see~Shnurkov~P.\,V.&&\\
\Avtors{Bening~V.\,E.} Calculation of~the~asymptotic deficiency
of~some statistical procedures based\linebreak
\\[-12pt]
\hspace*{23pt}on~samples with~random sizes&4&34--45\\
\Avtors{Borisov~A.\,V., Bosov~A.\,V., and Miller~G.\,B.} Modeling and
monitoring of VoIP connection&2&\hphantom{1}2--13\\
\Avtors{Bosov~A.\,V.} see~Borisov~A.\,V.&&\\
\Avtors{Briukhov~D.\,O.} see~Stupnikov~S.\,A.&&\\
\Avtors{Callaos~N.\,K.\ and Seyful-Mulyukov~R.\,B.} Complexity and
its information content&1&129--139\\
\Avtors{Chertok~A.\,V., Kadaner~A.\,I., Khazeeva~G.\,T., and
Sokolov~I.\,A.} Regime switching detection\linebreak
\\[-12pt]
\hspace*{23pt}for~the~Levy driven
Ornstein--Uhlenbeck process using CUSUM methods&4&46--56\\
\Avtors{Chichagov~V.\,V.} Asymptotic expansions of mean absolute
error of uniformly minimum variance unbiased and maximum likelihood
estimators on the one-parameter exponential\linebreak
\\[-12pt]
\hspace*{23pt}family model of lattice distributions&3&66--76\\
\Avtors{Danishevsky~V.\,I.} see~Kolesnikov A.\,V.&&\\
\Avtors{Fazliev~A.\,Z.} see~Kalinichenko~L.\,A.&&\\
\Avtors{Fedoseev~A.\,A.} What is behind the concept of ``knowledge in
small packages''&3&105--110\\
\Avtors{Gaidamaka~Yu.\,V., Andreev~S.\,D., Sopin~E.\,S.,
Samouylov~K.\,E., and Shorgin~S.\,Ya.} Interference analysis
of~the~device-to-device communications model with~regard to~a~signal\linebreak
\\[-12pt]
\hspace*{23pt}propagation environment&4&\hphantom{1}2--10\\
\Avtors{Gasilov~A.\,V.} see~Yakovlev~O.\,A.&&\\
\Avtors{Goncharov~A.\,V.\ and Strijov~V.\,V.} Metric time series
classification using weighted dynamic\linebreak
\\[-12pt]
\hspace*{23pt}warping relative to centroids of classes&2&36--47\\
\Avtors{Gordov~E.\,P.} see~Kalinichenko~L.\,A.&&\\
\Avtors{Gorshenin~A.\,K.} Concept of online service for stochastic
modeling of real processes&1&72--81\\
\Avtors{Gorshenin~A.\,K.} see~Shnurkov~P.\,V.&&\\
\Avtors{Gorshenin~A.\,K.} see~Shnurkov~P.\,V.&&\\
\Avtors{Grusho~A.\,A., Grusho~N.\,A., Zabezhailo~M.\,I., and
Timonina~E.\,E.} Integration of statistical and\linebreak
\\[-12pt]
\hspace*{23pt}deterministic methods for
analysis of information security&3&2--8\\
\Avtors{Grusho~A.\,A., Zabezhailo~M.\,I., and Zatsarinny~A.\,A.} On
the advanced procedure to reduce\linebreak
\\[-12pt]
\hspace*{23pt}calculation of Galois closures&4&\hphantom{1}96--104\\
\Avtors{Grusho~N.\,A.} see~Grusho~A.\,A.&&\\
\Avtors{Havanskov~V.\,A.} see~Minin~V.\,A.&&\\
\Avtors{Inkova~O.\,Yu.} see~Zatsman~I.\,M.&&\\
\Avtors{Isachenko~R.\,V.\ and Strijov~V.\,V.} Metric learning in
multiclass time series classification\linebreak
\\[-12pt]
\hspace*{23pt}problem&2&48--57\\
\end{tabular}
}
\pagebreak

\def\leftfootline{\small{\textbf{\thepage}
\hfill INFORMATIKA I EE PRIMENENIYA~--- INFORMATICS AND APPLICATIONS\ \ \ 2016\
\ \ volume~10\ \ \ issue\ 4}
}%
 \def\rightfootline{\small{INFORMATIKA I EE PRIMENENIYA~---
INFORMATICS AND APPLICATIONS\ \ \ 2016\ \ \ volume~10\ \ \ issue\ 4
\hfill \textbf{\thepage}}}

\def\leftkol{2016 AUTHOR INDEX} % ENGLISH ABSTRACTS}

\def\rightkol{2016 AUTHOR INDEX} %ENGLISH ABSTRACTS}


{\tabcolsep=2.83pt
\begin{tabular}{p{382pt}cc}
&\textbf{Issue} & \textbf{Page}\\[6pt]
\Avtors{Kadaner~A.\,I.} see~Chertok~A.\,V.&&\\[.255pt]
\Avtors{Kalinichenko~L.\,A., Volnova~A.\,A., Gordov~E.\,P.,
Kiselyova~N.\,N., Kovaleva~D.\,A., Malkov~O.\,Yu., Okladnikov~I.\,G.,
Podkolodnyy~N.\,L., Pozanenko~A.\,S., Ponomareva~N.\,V.,
Stupnikov~S.\,A.,} \textbf{and Fazliev~A.\,Z.} Data access challenges for data
intensive\linebreak
\\[-12pt]
\hspace*{23pt}research in Russia&1& 2--22\\[.255pt]
\Avtors{Karasikov~M.\,E.\ and Strijov~V.\,V.} Feature-based
time-series classification&4&121--131\\[.255pt]
\Avtors{Khazeeva~G.\,T.} see~Chertok~A.\,V.&&\\[.255pt]
\Avtors{Khokhlov~Yu.\,S.} Multivariate fractional Levy motion and its
applications&2&\hphantom{1}98--106\\[.255pt]
\Avtors{Kirikov~I.\,A., Kolesnikov~A.\,V., Listopad~S.\,V., and
Rumovskaya~S.\,B.} Fine-grained hybrid\linebreak
\\[-12pt]
\hspace*{23pt}intelligent systems. Part 2:
Bidirectional hybridization&1&\hphantom{1}96--105\\[.255pt]
\Avtors{Kirikov~I.\,A., Kolesnikov~A.\,V., Listopad~S.\,V., and
Rumovskaya~S.\,B.} ``Virtual council''~---\linebreak
\\[-12pt]
\hspace*{23pt}source environment
supporting complex diagnostic decision making&3&81--90\\[.255pt]
\Avtors{Kiselyova~N.\,N.} see~Kalinichenko~L.\,A.&&\\[.255pt]
\Avtors{Kolesnikov A.\,V., Listopad~S.\,V., Rumovskaya~S.\,B., and
Danishevsky~V.\,I.} Informal axiomatic\linebreak
\\[-12pt]
\hspace*{23pt}theory of~the~role visual models&4&114--120\\[.255pt]
\Avtors{Kolesnikov~A.\,V.} see~Kirikov~I.\,A.&&\\[.255pt]
\Avtors{Kolesnikov~A.\,V.} see~Kirikov~I.\,A.&&\\[.255pt]
\Avtors{Kolin~K.\,K.} Humanitarian aspects of information
security&3&111--121\\[.255pt]
\Avtors{Konovalov~M.\,G.\ and Razumchik~R.\,V.} Dispatching
to~two parallel nonobservable queues using\linebreak
\\[-12pt]
\hspace*{23pt}only static
information&4&57--67\\[.255pt]
\Avtors{Korchagin~A.\,Yu.} see~Korolev~V.\,Yu.&&\\[.255pt]
\Avtors{Korchagin~A.\,Yu.} see~Korolev~V.\,Yu.&&\\[.255pt]
\Avtors{Korepanov~E.\,R.} see~Sinitsyn~I.\,N.&&\\[.255pt]
\Avtors{Korepanov~E.\,R.} see~Sinitsyn~I.\,N.&&\\[.255pt]
\Avtors{Korolev~V.\,Yu., Korchagin~A.\,Yu., and Zeifman~A.\,I.} The
Poisson theorem for Bernoulli trials\linebreak
\\[-12pt]
\hspace*{23pt}with~a~random probability
of~success and~a~discrete analog of~the~Weibull distribution&4&11--20\\[.255pt]
\Avtors{Korolev~V.\,Yu., Zeifman~A.\,I., and Korchagin~A.\,Yu.}
Asymmetric Linnik distributions as~limit\linebreak
\\[-12pt]
\hspace*{23pt}laws for~random sums
of~independent random variables with~finite variances&4&21--33\\[.255pt]
\Avtors{Koucheryavy~E.\,A.} see~Ometov~A.\,Ya.&&\\[.255pt]
\Avtors{Kovaleva~D.\,A.} see~Kalinichenko~L.\,A.&&\\[.255pt]
\Avtors{Kovalyov~S.\,P.} Metaprogramming to increase
manufacturability of large-scale software-\linebreak
\\[-12pt]
\hspace*{23pt}intensive systems&1&56--66\\[.255pt]
\Avtors{Krivenko~M.\,P.} Significance tests of feature selection for
classification&3&32--40\\[.255pt]
\Avtors{Kruzhkov~M.\,G.} see~Zalizniak~Anna~A.&&\\[.255pt]
\Avtors{Kruzhkov~M.\,G.} see~Zatsman~I.\,M.&&\\[.255pt]
\Avtors{Kudryavtsev~A.\,A.} Bayesian queueing and reliability models:
\textit{A~priori} distributions with\linebreak
\\[-12pt]
\hspace*{23pt}compact support&1&67--71\\[.255pt]
\Avtors{Kudryavtsev~A.\,A.} Characteristics dependent on the balance
coefficient in Bayesian models\linebreak
\\[-12pt]
\hspace*{23pt}with compact support of \textit{a priori}
distributions&3&77--80\\[.255pt]
\Avtors{Kudryavtsev~A.\,A.\ and Palionnaia~S.\,I.} Bayesian recurrent
model of reliability growth:\linebreak
\\[-12pt]
\hspace*{23pt}Parabolic distribution of parameters&2&80--83\\[.255pt]
\Avtors{Kudryavtsev~A.\,A.\ and Titova~A.\,I.} Bayesian queuing
and~reliability models: Degenerate-\linebreak
\\[-12pt]
\hspace*{23pt}Weibull case&4&68--71\\[.255pt]
\Avtors{Leontyev~N.\,D.\ and Ushakov~V.\,G.} Analysis of a queueing
system with autoregressive arrivals\linebreak
\\[-12pt]
\hspace*{23pt}and nonpreemptive priority&3&15--22\\[.255pt]
\Avtors{Listopad~S.\,V.} see~Kirikov~I.\,A.&&\\[.255pt]
\Avtors{Listopad~S.\,V.} see~Kirikov~I.\,A.&&\\[.255pt]
\Avtors{Listopad~S.\,V.} see~Kolesnikov A.\,V.&&\\[.255pt]
\Avtors{Malkov~O.\,Yu.} see~Kalinichenko~L.\,A.&&\\[.255pt]
\Avtors{Markov~A.\,S., Monakhov~M.\,M., and
Ulyanov~V.\,V.} Generalized Cornish--Fisher expansions\linebreak
\\[-12pt]
\hspace*{23pt}for distributions of statistics based on samples
of random size&2&84--91\\[.255pt]
\Avtors{Melnikov~A.\,K.\ and Ronzhin~A.\,F.} Generalized statistical
method of~text analysis based\linebreak
\\[-12pt]
\hspace*{23pt}on~calculation of~probability distributions
of~statistical values&4&89--95\\
\end{tabular}
}
\pagebreak

\def\leftfootline{\small{\textbf{\thepage}
\hfill INFORMATIKA I EE PRIMENENIYA~--- INFORMATICS AND APPLICATIONS\ \ \ 2016\
\ \ volume~10\ \ \ issue\ 4}
}%
 \def\rightfootline{\small{INFORMATIKA I EE PRIMENENIYA~---
INFORMATICS AND APPLICATIONS\ \ \ 2016\ \ \ volume~10\ \ \ issue\ 4
\hfill \textbf{\thepage}}}

\def\leftkol{2016 AUTHOR INDEX} % ENGLISH ABSTRACTS}

\def\rightkol{2016 AUTHOR INDEX} %ENGLISH ABSTRACTS}


{\tabcolsep=3pt
\begin{tabular}{p{381pt}cc}
&\textbf{Issue} & \textbf{Page}\\[6pt]
\Avtors{Meykhanadzhyan~L.\,A.} Stationary characteristics of the finite
capacity queueing system with\linebreak
\\[-12pt]
\hspace*{23pt}inverse service order and generalized
probabilistic priority&2&123--131\\[.23pt]
\Avtors{Miller~G.\,B.} see~Borisov~A.\,V.&&\\[.23pt]
\Avtors{Minin~V.\,A., Zatsman~I.\,M., Havanskov~V.\,A., and
Shubnikov~S.\,K.} Intensity of citation of scientific publications in
inventions on information and computer technologies patented\linebreak
\\[-12pt]
\hspace*{23pt}in Russia by domestic and foreign applicants&2&107--122\\[.23pt]
\Avtors{Monakhov~M.\,M.} see~Markov~A.\,S.&&\\[.23pt]
\Avtors{Naumov~V.\,A.\ and Samouylov~K.\,E.} On relationship
between queuing systems with resources\linebreak
\\[-12pt]
\hspace*{23pt}and Erlang networks&3&\hphantom{1}9--14\\[.23pt]
\Avtors{Okladnikov~I.\,G.} see~Kalinichenko~L.\,A.&&\\[.23pt]
\Avtors{Ometov~A.\,Ya., Andreev~S.\,D., Turlikov~A.\,M., and
Koucheryavy~E.\,A.} Performance analysis of\linebreak
\\[-12pt]
\hspace*{23pt}a wireless data
aggregation system with contention for contemporary sensor
networks&3&23--31\\[.23pt]
\Avtors{Palionnaia~S.\,I.} see~Kudryavtsev~A.\,A.&&\\[.23pt]
\Avtors{Podkolodnyy~N.\,L.} see~Kalinichenko~L.\,A.&&\\[.23pt]
\Avtors{Ponomareva~N.\,V.} see~Kalinichenko~L.\,A.&&\\[.23pt]
\Avtors{Popkova~N.\,A.} see~Zatsman~I.\,M.&&\\[.23pt]
\Avtors{Pozanenko~A.\,S.} see~Kalinichenko~L.\,A.&&\\[.23pt]
\Avtors{Razumchik~R.\,V.} see~Konovalov~M.\,G.&&\\[.23pt]
\Avtors{Ronzhin~A.\,F.} see~Melnikov~A.\,K.&&\\[.23pt]
\Avtors{Rumovskaya~S.\,B.} see~Kirikov~I.\,A.&&\\[.23pt]
\Avtors{Rumovskaya~S.\,B.} see~Kirikov~I.\,A.&&\\[.23pt]
\Avtors{Rumovskaya~S.\,B.} see~Kolesnikov A.\,V.&&\\[.23pt]
\Avtors{Samouylov~K.\,E.} see~Gaidamaka~Yu.\,V.&&\\[.23pt]
\Avtors{Samouylov~K.\,E.} see~Naumov~V.\,A.&&\\[.23pt]
\Avtors{Serebryanskii~S.\,M.} see~Tyrsin~A.\,N.&&\\[.23pt]
\Avtors{Seyful-Mulyukov~R.\,B.} see~Callaos~N.\,K.&&\\[.23pt]
\Avtors{Shestakov~O.\,V.} Statistical properties of the denoising method
based on the stabilized hard\linebreak
\\[-12pt]
\hspace*{23pt}thresholding&2&65--69\\[.23pt]
\Avtors{Shestakov~O.\,V.} The strong law of large numbers for the risk
estimate in the problem of\linebreak
\\[-12pt]
\hspace*{23pt}tomographic image reconstruction from
projections with a correlated noise&3&41--45\\[.23pt]
\Avtors{Shestakov~O.\,V.} see~Zakharova~T.\,V.&&\\[.23pt]
\Avtors{Shnurkov~P.\,V., Gorshenin~A.\,K., and Belousov~V.\,V.}
Analytical solution of~the~optimal control\linebreak
\\[-12pt]
\hspace*{23pt}task of~a~semi-Markov
process with~finite set of~states&4&72--88\\[.23pt]
\Avtors{Shnurkov~P.\,V., Zasypko~V.\,V., Belousov~V.\,V., and
Gorshenin~A.\,K.} Development of the algorithm of numerical solution
of the optimal investment control problem\linebreak
\\[-12pt]
\hspace*{23pt}in the closed dynamical model of three-sector economy&1&82--95\\[.23pt]
\Avtors{Shorgin~S.\,Ya.} see~Gaidamaka~Yu.\,V.&&\\[.23pt]
\Avtors{Shorgin~V.\,S.} see~Agalarov~Ya.\,M.&&\\[.23pt]
\Avtors{Shubnikov~S.\,K.} see~Minin~V.\,A.&&\\[.23pt]
\Avtors{Sidorkin~I.\,I.} see~Arkhipov~O.\,P.&&\\[.23pt]
\Avtors{Sinitsyn~I.\,N.} Analytical modeling of processes in stochastic
systems with complex fractional\linebreak
\\[-12pt]
\hspace*{23pt}order Bessel nonlinearities&3&55--65\\[.23pt]
\Avtors{Sinitsyn~I.\,N.} Orthogonal supoptimal filters for nonlinear
stochastic systems on manifolds&1&34--44\\[.23pt]
\Avtors{Sinitsyn~I.\,N.\ and Korepanov~E.\,R.} Normal Pugachev
conditionally-optimal filters and extra-\linebreak
\\[-12pt]
\hspace*{23pt}polators for state linear stochastic systems&2&14--23\\[.23pt]
\Avtors{Sinitsyn~I.\,N.\ and Sinitsyn~V.\,I.} Analytical modeling of
distributions in stochastic systems on\linebreak
\\[-12pt]
\hspace*{23pt}manifolds based on ellipsoidal approximation&1&45--55\\[.23pt]
\Avtors{Sinitsyn~I.\,N., Sinitsyn~V.\,I., and
Korepanov~E.\,R.} Ellipsoidal suboptimal filters for nonlinear\linebreak
\\[-12pt]
\hspace*{23pt}stochastic systems on manifolds&2&24--35\\[.23pt]
\Avtors{Sinitsyn~V.\,I.} see~Sinitsyn~I.\,N.&&\\[.23pt]
\Avtors{Sinitsyn~V.\,I.} see~Sinitsyn~I.\,N.&&\\[.23pt]
\Avtors{Skvortsov~N.\,A.} see~Stupnikov~S.\,A.&&\\[.23pt]
\Avtors{Sokolov~I.\,A.} see~Chertok~A.\,V.&&\\
\end{tabular}
}
\pagebreak

\def\leftfootline{\small{\textbf{\thepage}
\hfill INFORMATIKA I EE PRIMENENIYA~--- INFORMATICS AND APPLICATIONS\ \ \ 2016\
\ \ volume~10\ \ \ issue\ 4}
}%
 \def\rightfootline{\small{INFORMATIKA I EE PRIMENENIYA~---
INFORMATICS AND APPLICATIONS\ \ \ 2016\ \ \ volume~10\ \ \ issue\ 4
\hfill \textbf{\thepage}}}

\def\leftkol{2016 AUTHOR INDEX} % ENGLISH ABSTRACTS}

\def\rightkol{2016 AUTHOR INDEX} %ENGLISH ABSTRACTS}


{\tabcolsep=3pt
\begin{tabular}{p{382pt}cc}
&\textbf{Issue} & \textbf{Page}\\[6pt]
\Avtors{Sopin~E.\,S.} see~Gaidamaka~Yu.\,V.&&\\
\Avtors{Strijov~V.\,V.} see~Goncharov~A.\,V.&&\\
\Avtors{Strijov~V.\,V.} see~Isachenko~R.\,V.&&\\
\Avtors{Strijov~V.\,V.} see~Karasikov~M.\,E.&&\\
\Avtors{Stupnikov~S.\,A., Briukhov~D.\,O., and Skvortsov~N.\,A.}
Co-lending systemic risk analysis over\linebreak
\\[-12pt]
\hspace*{23pt}heterogeneous data collections&1&23--33\\
\Avtors{Stupnikov~S.\,A.} see~Kalinichenko~L.\,A.&&\\
\Avtors{Suchkov~A.\,P.} see~Zatsarinny~A.\,A.&&\\
\Avtors{Timonina~E.\,E.} see~Grusho~A.\,A.&&\\
\Avtors{Titova~A.\,I.} see~Kudryavtsev~A.\,A.&&\\
\Avtors{Turlikov~A.\,M.} see~Ometov~A.\,Ya.&&\\
\Avtors{Tyrsin~A.\,N.\ and Serebryanskii~S.\,M.} Recognition of
dependences on the basis of inverse\linebreak
\\[-12pt]
\hspace*{23pt}mapping&2&58--64\\
\Avtors{Ulyanov~V.\,V.} see~Markov~A.\,S.&&\\
\Avtors{Ushakov~V.\,G.} Queueing system with working vacations and
hyperexponential input stream&2&92--97\\
\Avtors{Ushakov~V.\,G.} see~Leontyev~N.\,D.&&\\
\Avtors{Volnova~A.\,A.} see~Kalinichenko~L.\,A.&&\\
\Avtors{Yakovlev~O.\,A.\ and Gasilov~A.\,V.} Speeded-up stereo
matching using geodesic support weights&3&\hphantom{1}98--104\\
\Avtors{Zabezhailo~M.\,I.} see~Grusho~A.\,A.&&\\
\Avtors{Zabezhailo~M.\,I.} see~Grusho~A.\,A.&&\\
\Avtors{Zakharova~T.\,V.\ and Shestakov~O.\,V.} Precision analysis of
wavelet processing of aerodynamic\linebreak
\\[-12pt]
\hspace*{23pt}flow patterns&3&46--54\\
\Avtors{Zalizniak~Anna~A.\ and Kruzhkov~M.\,G.} Database
of~Russian impersonal verbal constructions&4&132--141\\
\Avtors{Zasypko~V.\,V.} see~Shnurkov~P.\,V.&&\\
\Avtors{Zatsarinny~A.\,A.\ and Suchkov~A.\,P.} Systems engineering
approaches to~the~establishment of\linebreak
\\[-12pt]
\hspace*{23pt}a~system for~decision support based
on~situational analysis&4&105--113\\
\Avtors{Zatsarinny~A.\,A.} see~Grusho~A.\,A.&&\\
\Avtors{Zatsman~I.\,M., Inkova~O.\,Yu., Kruzhkov~M.\,G., and
Popkova~N.\,A.} Representation of cross-\linebreak
\\[-12pt]
\hspace*{23pt}lingual knowledge about
connectors in supracorpora databases&1&106--118\\
\Avtors{Zatsman~I.\,M.} see~Minin~V.\,A.&&\\
\Avtors{Zeifman~A.\,I.} see~Korolev~V.\,Yu.&&\\
\Avtors{Zeifman~A.\,I.} see~Korolev~V.\,Yu.&&\\
\end{tabular}
}

%\thispagestyle{myheadings}
\def\leftfootline{\small{\textbf{\thepage}
\hfill INFORMATIKA I EE PRIMENENIYA~--- INFORMATICS AND APPLICATIONS\ \ \ 2016\
\ \ volume~10\ \ \ issue\ 4}
}%
 \def\rightfootline{\small{INFORMATIKA I EE PRIMENENIYA~---
INFORMATICS AND APPLICATIONS\ \ \ 2016\ \ \ volume~10\ \ \ issue\ 4
\hfill \textbf{\thepage}}}

 \label{end\stat}

\newpage

%\documentclass[10pt]{book}
\usepackage[utf8]{inputenc}

\usepackage{latexsym,amssymb,amsfonts,amsmath,indentfirst,shapepar,%fleqn,%
picinpar,shadow,floatflt,enumerate,multicol,ipi}

\input{epsf}

%\nofiles

\usepackage{acad}
\usepackage{courier}
\usepackage{decor}
\usepackage{newton}
\usepackage{pragmatica}
\usepackage{zapfchan}
\usepackage{petrotex}
\usepackage{bm}                     % полужирные греческие буквы
\usepackage{upgreek}                % прямые греческие буквы
%\usepackage{verbatim}

\renewcommand{\bottomfraction}{0.99}
\renewcommand{\topfraction}{0.99}
\renewcommand{\textfraction}{0.01}

%NEW COMMANDS
\renewcommand{\r}{{\rm I\hspace{-0.7mm}\rm R}}

\newcommand{\il}[2]{\int\limits_{#1}^{#2}}%интеграл с пределами #1 и #2

%\pagestyle{myheadings}

\setlength{\textwidth}{167mm}      % 122mm
\setlength{\textheight}{658pt}
%\setlength{\textheight}{635.6pt}
\setlength{\columnsep}{4.5mm}

\setcounter{secnumdepth}{4}

%\addtolength{\headheight}{2pt}
%\addtolength{\headsep}{-2mm}

%\addtolength{\topmargin}{-20mm}  % for printing


\hoffset=-30mm  % From Yap
%\hoffset=-20mm  % From Acrobat

%\voffset=0mm % From Yap
%\voffset=-15mm   % From Acrobat

\addtolength{\evensidemargin}{-9.5mm} % for printing
\addtolength{\oddsidemargin}{9.5mm}  % for printing

\renewcommand\labelitemi{$\bullet$}

\begin{document}
\Rus

\nwt
%\ptb

%\vspace*{-12pt}

\begin{center}

{\prgsh\LARGE
ОБЪЯВЛЕНИЯ О КОНФЕРЕНЦИЯХ}

\end{center}
%\hrule

\vspace*{6pt}
%\begin{center}
%\mbox{%
%\epsfxsize=167mm
%\epsfbox{recl-dia-b.eps}
%}
%\end{center}
\begin{flushright}
{\prg http://www.tvp.ru/conferen/20091001\_1.htm}
\end{flushright}

%\vspace*{6pt}

\begin{center}\prg
\Large
X Всероссийский симпозиум\\ по прикладной и промышленной математике\\
(осенняя открытая сессия)

\end{center}

\begin{center}\prg
%Очередная 15-я международная конференция <<Диалог>> состоится c
1--8 октября 2009~г., Сочи--Дагомыс
\end{center}


%Всероссийские симпозиумы по прикладной и промышленной математике проводятся ежегодно с 2000~года. 

\smallskip

{\centering Симпозиум проводится по следующим направлениям:

\begin{multicols}{2}
\begin{itemize}
\item Безопасность компьютерных систем\item 
Геометрическая нелинейная оптика\item 
Инженерно-технологическая математика\item 
Информационные технологии и задачи связи\item 
Квантовые вычисления\item 
Математические методы биологических и экологических систем\item 
Математические модели в жидких кристаллах\item 
Математические методы в педагогических исследованиях\item 
Математические модели в теории оболочек\item 
Математическое моделирование процессов рассеяния примесей в турбулентной атмосфере\item 
Математическое моделирование свойств материалов и конструкций\item 
Математическое образование\item  
Медицина\item  
Метод конечных элементов\item 
Механика жидкости и газа\item  
Механика природных процессов\item  
Механика разрушения\item 
Модели горения и взрыва\item  
Нанотехнологии: математические модели\item 
Науки о Земле, геология, геофизика\item 
 Неклассические задачи для уравнений математической физики
  \item Нелинейное моделирование и управление\item 
  Обработка данных, анализ и обработка изображений\item 
  Прикладная вероятность и статистика\item 
  Прикладная геометрия\item 
  Обработка и распознавание образов\item 
  Прикладная дискретная математика\item 
  Обработка и защита информации\item 
  Системы поддержки принятия решений для регионального управления\item 
  Социология\item 
  Психология\item 
  Специальные функции и ортогональные многочлены\item 
  Супер-, нейро-, биокомпьютеры\item
  Эволюционные и мембранные вычисления\item
  Теория управления и системные исследования\item
Процессы принятия решений\item 
Тепло- и массоперенос\item 
Физика океана и атмосферы\item 
Фракталы и масштабный эффект\item 
Экономика, страховая и финансовая матема\-тика\item 
Энергетика и передача энергии\item
Юриспруденция\item 
Криминалистика
\end{itemize}
\end{multicols}
}

\smallskip

{\centering В программу симпозиума входят также:
\begin{itemize}
\item учредительное собрание общероссийского <<Общества прикладной и промышленной математики>> (ОППМ);\\[-13pt] 
\item минисимпозиумы по предложенным позднее темам; \\[-13pt]
\item круглые столы по продвижению современных фундаментальных математических методов в различные сферы науки и технологий, по междисциплинарному сотрудничеству; \\[-13pt]
\item выставка-продажа научных изданий, демонстрация программного обеспечения.
\end{itemize}

{\large Организаторы симпозиума: }
%\noindent
%\begin{tabular}{p{7mm}p{420pt}}
%\hspace*{7mm}
\begin{itemize}
\item Управление по науке и образованию г.~Сочи
\item Сочинский государственный университет туризма и курортного дела
\item Академия криптографии Российской Федерации
\item Институт проблем информатики Российской академии наук
\item Редакции журналов <<Информатика и её применения>>, <<Обозрение прикладной и промышленной математики>> (ОПиПМ), <<Прикладная информатика>>, <<Теория вероятностей и ее применения>> (Научное издательство <<ТВП>>) 
\item Экономический факультет Санкт-Петербургского государственного университета (ЭФ СПбГУ)
\end{itemize}
%\end{tabular}

%\vspace*{6pt}
\bigskip

{\large Организационный комитет:}
\vspace*{6pt}

Академик Ю.\,В.~Прохоров (председатель)
\vspace*{6pt}

\tabcolsep=15pt
\begin{tabular}{p{60mm}p{60mm}}
академик В.\,А.~Бабешко\newline
академик А.\,А.~Боровков\newline 
академик С.\,С.~Григорян\newline 
академик И.\,А.~Ибрагимов\newline 
академик В.\,И.~Колесников\newline 
академик А.\,Б.~Куржанский\newline 
академик В.\,П.~Маслов\newline
академик И.\,А.~Соколов\newline
академик В.\,П.~Шорин\newline 
член-корр.\ РАН А.\,Б.~Жижченко\newline
член-корр.\ РАН С.\,В.~Кисляков\newline 
член-корр.\ РАН В.\,В.~Русанов\newline 
член-корр.\ РАН Б.\,А.~Севастьянов 
&
член-корр.\ РАН В.\,А.~Сойфер\newline
член-корр.\ РАН А.\,Н.~Ширяев\newline
И.\,П.~Бойко\newline 
А.\,А.~Емельянов\newline
А.\,М.~Зубков\newline 
В.\,Ф.~Колчин\newline
 А.\,С.~Максимов\newline
О.\,Н.~Медведева\newline 
Г.\,М.~Романова\newline 
В.\,В.~Сапожников\newline 
А.\,Р.~Симонян\newline 
В.\,И.~Хохлов\newline
С.\,Я.~Шоргин
\end{tabular}


}
\end{document}


%\noindent
%\begin{tabular}{p{7mm}p{420pt}}
%\hspace*{7mm}
%&\begin{itemize}
%\item Филологический факультет МГУ
%\item Институт лингвистики РГГУ
%\item Институт проблем информатики РАН
%\item Институт проблем передачи информации РАН
%\item Российский НИИ искусственного интеллекта
%\item Яндекс (Москва)
%\end{itemize}
%\end{tabular}

%Конференция проводится по следующим направлениям, сочетающим теоретические
%исследования и приложения:
%\vspace*{-6pt}

%\noindent
%\begin{tabular}{p{7mm}p{420pt}}
%\hspace*{7mm}&
%\begin{itemize}
%\item Лингвистическая семантика и семантический анализ
%\item Формальные модели языка и их применение
%\item Теоретическая и компьютерная лексикография
%\item Создание и применение компьютерных лексических ресурсов
%\item Корпусная лингвистика. Создание, применение, оценка корпусов
%\item Интернет как лингвистический ресурс. Лингвистические технологии в
%интернете
%\item Извлечение знаний из текстов
%\item Модели общения. Коммуникация, диалог и речевой акт
%\item Анализ и синтез речи
%\item Компьютерный анализ документов: реферирование, классификация,
%поиск
%\item Машинный перевод
%\item Вопросно-ответные системы
%\end{itemize}
%\end{tabular}

%Сайт конференции: {\prg http://www.dialog-21.ru/ }

%\newpage

\begin{center}

{\prgsh\LARGE
ОБЪЯВЛЕНИЯ О КОНФЕРЕНЦИЯХ}

\end{center}
%\hrule

\vspace*{12pt}

%\hrule
\begin{center}
\mbox{%
\epsfxsize=167mm
\epsfbox{recl-RL-b.eps}
}
\end{center}
\begin{flushright}
{\prg http://rcdl2009.krc.karelia.ru/}
\end{flushright}

%\vspace*{6pt}

{\begin{center}\prg
{\Large
XI Всероссийская научная конференция RCDL 2009\\
Электронные библиотеки: перспективные методы и технологии,\\ электронные
коллекции}
\end{center}}

{\begin{center}\prg
17--21 сентября 2009 г.,
Петрозаводск, Россия
\end{center}}

\vspace*{12pt}

Электронные библиотеки~--- область исследований и разработок, направленных 
на развитие теории и практики обработки, распространения, хранения, поиска 
и анализа цифровых данных различной природы. 

Основная цель серии конференций RCDL заключается в том, чтобы способствовать 
формированию сообщества специалистов России, ведущих исследования и разработки 
в области электронных библиотек. Конференция также способствует изучению 
зарубежного опыта, развитию международного сотрудничества в области электронных 
библиотек. 

Значительное внимание в тематике RCDL уделяется практическим проектам, в 
которых решаются сложные задачи. RCDL придает большое значение исследованиям 
в области создания крупномасштабных электронных библиотек (Very Large Digital 
Libraries~--- VLDL), включая использование сервисных архитектур, архитектур, 
основанных на грид, и обеспечение их качества, развитие техники 
интероперабельности и устойчивости VLDL, а также разработке организационных 
моделей крупных электронных библиотек. Особый интерес представляет применение 
современных научных подходов в контексте высоких нагрузок: сотни тысяч 
пользователей, десятки гигабайт данных, терабайты трафика.


%RCDL'2009~---  одиннадцатая конференция в серии Всероссийских научных
%конференций <<Электронные библиотеки: перспективные методы и технологии,
%электронные коллекции>>.
За 10 лет проведения RCDL в работе конференции приняло участие несколько
сотен ученых из ведущих российских и зарубежных научных центров Австрии,
Германии, Греции, Италии, Новой Зеландии, США, Украины и других стран.

Традиционно совместно с RCDL проводятся Всероссийские научные семинары по оценке
методов текстового поиска РОМИП. В 2009 году планируется совмещение с RCDL
Семинара РОМИП и\linebreak
 Третьей Российской летней школы по информационному поиску RuSSIR'2009,
во время которой ведущие российские и зарубежные ученые прочитают
обзорные лекции по актуальным проблемам развития поиска цифровых данных для
решения фундаментальных и прикладных задач.
\vspace*{6pt}

Организаторы конференции RCDL'2009:
\vspace*{-6pt}

\noindent
\begin{tabular}{p{7mm}p{420pt}}
\hspace*{7mm}&
\begin{itemize}
\item Российская академия наук
\item Российский фонд фундаментальных исследований
\item Карельский научный центр РАН
\item Институт прикладных математических исследований
\item Петрозаводский государственный университет
\item Институт проблем информатики РАН
\item Московская секция АСМ SIGMOD
\end{itemize}
\end{tabular}

%Сайт конференции: {\prg http://rcdl2009.krc.karelia.ru/}

\end{document}

%\include{rekl-1}

%\end{document}

%\include{nekrolog-rb}


%\end{document}

%\include{IPPM-25}

\vspace*{-60pt} %{ %small
{ %\baselineskip=9.1pt
\section*{Правила подготовки рукописей  для публикации в журнале
<<Информатика и её применения>>}

\thispagestyle{empty}

\noindent
\begin{enumerate}[1.]
\item В журнале печатаются статьи, содержащие результаты, ранее не опубликованные и 
не предназначенные к одновременной публикации в других изданиях. 
 
Публикация не должна нарушать закон об авторских правах. 
 
Направляя рукопись в редакцию, авторы сохраняют все права собственников данной 
рукописи и при этом передают учредителям и редколлегии неисключительные права на 
издание статьи на русском языке (или на языке статьи, если он отличен от русского) и на 
ее распространение в России и за рубежом. Авторы должны представить в редакцию 
письмо в следующей форме: 


{\bfseries\textit{Соглашение о передаче права на публикацию:}}

<<\textit{Мы, нижеподписавшиеся, авторы рукописи} <<\ldots>>, 
\textit{передаем учредителям и редколлегии журнала <<Информатика и её 
применения>> неисключительное право опубликовать данную рукопись 
статьи на русском языке как в печатной, так и в электронной версиях 
журнала. Мы подтверждаем, что данная публикация не нарушает 
авторского права других лиц или организаций. }
 
\textit{Подписи авторов: (ф. и. о., дата, адрес)}>>.  
 
Это соглашение может быть представлено в бумажном виде или в виде 
отсканированной копии (с подписями авторов).  
 
Редколлегия вправе запросить у авторов экспертное заключение о 
возможности публикации пред\-став\-лен\-ной статьи в открытой печати. 

\item К статье прилагаются данные автора (авторов) (см.\ п.~8). При наличии нескольких 
авторов указывается фамилия автора, ответственного за переписку с редакцией. 

\item Редакция журнала осуществляет экспертизу присланных статей в соответствии с 
принятой в журнале процедурой рецензирования.

Возвращение рукописи на доработку не означает ее принятия к печати.  

Доработанный вариант с ответом на замечания рецензента необходимо прислать в 
редакцию. 

\item Решение редколлегии о публикации статьи или ее отклонении сообщается авторам.  
Редколлегия может также направить авторам текст рецензии на их статью. Дискуссия по 
поводу отклоненных статей не ведется. 

\item Редактура статей высылается авторам для просмотра. Замечания к редактуре должны 
быть присланы авторами в кратчайшие сроки. 

\item Рукопись предоставляется в электронном виде в форматах MS WORD (.doc или 
.docx) или \LaTeX\ (.tex), дополнительно~--- в формате .pdf, на дискете, лазерном диске 
или электронной почтой. Предоставление бумажной рукописи необязательно.

\item При подготовке рукописи в MS Word рекомендуется использовать следующие 
настройки.

Параметры страницы:  
формат~--- А4; ориентация~--- книжная; поля (см): внутри~--- 2,5, снаружи~--- 1,5, 
сверху~--- 2, снизу~--- 2, от края до нижнего колонтитула~--- 1,3.  

Основной текст: стиль~--- <<Обычный>>, шрифт ~--- Times New Roman, размер~--- 
14~пунктов, абзацный отступ~--- 0,5~см, 1,5~интервала, выравнивание~--- по ширине.  
 
Рекомендуемый объем рукописи~--- не свыше 20 страниц указанного формата.  

Сокращения слов, помимо стандартных, не допускаются. Допускается минимальное 
количество аббревиатур. 

Все страницы рукописи нумеруются. 

Шаблоны примеров оформления, представлены в Интернете: 

{\sf http://www.ipiran.ru/journal/template.doc}.

\item Статья должна содержать следующую информацию на {\bfseries\textit{русском и 
английском языках:}} 
\begin{itemize}
\item название статьи; 
\item Ф.И.О.\ авторов, на английском можно только имя и фамилию;
\item место работы, с указанием города и страны и электронного адреса каждого 
автора; 

\pagebreak  

\thispagestyle{empty}


\vspace*{-36pt}


\item сведения об авторах, в соответствии с форматом, образцы которого 
представлены на страницах: 

{\sf http://www.ipiran.ru/journal/issues/2013\_07\_01/authors.asp}  и

 {\sf 
http://www.ipiran.ru/journal/issues/2013\_07\_01\_eng/authors.asp};

\item аннотация (не менее 100 слов на каждом из языков). Аннотация~--- это 
краткое резюме работы, которое может публиковаться отдельно. Она является 
основным источником информации в информационных системах и базах данных; 
Английская аннотация должна быть оригинальной, может не быть дословным 
переводом русского текста и должна быть написана хорошим английским языком.
\item ключевые слова, желательно из принятых в мировой научно-технической 
литературе тематических тезаурусов. Предложения не могут быть ключевыми 
словами.
\end{itemize}

\item Литература. По включенным в список литературы работам на русском языке 
информация в списке представляется как в кириллице, так и с использованием латинской 
транслитерации, а по работам, написанным латиницей,~--- на языке оригинала. 

Ссылки на литературу в тексте статьи нумеруются (в квадратных скобках) и 
располагаются в списке литературы в порядке упоминания.  

В списке литературы не должно быть позиций, на которые нет ссылки в тексте статьи.  

\item Присланные в редакцию материалы авторам не возвращаются. 

\item При отправке файлов по электронной почте просим придерживаться следующих 
правил: 
\begin{itemize}
\item указывать в поле subject (тема) название журнала и фамилию автора; 
\item использовать attach (присоединение); 
\item в состав электронной версии статьи должны входить: файл, содержащий 
текст статьи, и файл(ы), содержащий(е) иллюстрации. 
\end{itemize}
\item  Журнал <<Информатика и её применения>> является некоммерческим изданием. 
Плата за публикацию не взимается, гонорар авторам не выплачивается. 
\end{enumerate}


\thispagestyle{empty}

\noindent
\textbf{Адрес редакции:} Москва 119333, ул.~Вавилова, д.~44, корп.~2, ИПИ РАН

\noindent
\hphantom{\textbf{Адрес редакции: }}Тел.: +7 (499) 135-86-92\ \ Факс:  +7 (495) 930-45-05  

\noindent
\hphantom{\textbf{Адрес редакции: }}e-mail:   {\sf rust@ipiran.ru} (Сейфуль-Мулюков Рустем Бадриевич)
}
%}


%\vfill
%\begin{center}


%Технический редактор Л. Кокушкина\\
%%Выпускающий редактор Т. Торжкова\\
%Художественный редактор М. Седакова\\
%Сдано в набор 03.04.13. Подписано в печать 14.06.13. Формат 60 х 84 / 8\\
%Бумага офсетная. Печать цифровая. Усл.-печ.\ л.\ ??,?. Уч.-изд.\ л.\ ??,?. Тираж 100 экз.\\
%\ \\
%Заказ №\,\\
%\ \\
%Издательство <<ТОРУС ПРЕСС>>, Москва 121614, ул. Крылатская, 29-1-43\\
%torus@torus-press.ru; http://www.torus-press.ru\\
%\ \\
%Отпечатано в Академиздатцентре <<Наука>> РАН с готовых файлов\\
%Москва 121099, Шубинский пер., д.~6\\
%\end{center}
\vspace*{-60pt} %{ %small
{ %\baselineskip=9.1pt
\section*{Requirements for manuscripts submitted to Journal ``Informatics and~Applications''}

\thispagestyle{empty}



\noindent
\begin{enumerate}[1.]
\item The Journal publishes original articles which have not been published before and are not 
intended for publication in other editions. An article submitted to the Journal must not violate 
the Copyright law. Sending the manuscript to the Editorial Board, the authors retain all rights 
of the owners of the manuscript and transfer the nonexclusive rights to publish the article in Russian 
(or the language of the article, if not Russian) and its distribution in Russia and abroad to the Founders 
and the Editorial Board. Authors should submit a letter to the Editorial Board in the following form:


{\bfseries\textit{Agreement on the transfer of rights to publish:}}

<<\textit{We, the undersigned authors of the manuscript ``\ldots,'' pass to the 
Founder and the Editorial Board of the Journal ``Informatics and Applications'' 
the nonexclusive right to publish the manuscript of the article in Russian (or English) 
 in both print and electronic versions of the Journal. We affirm that 
this publication does not violate the Copyright of other persons or organizations. }
 
\textit{Author(s) signature(s): (name(s), address(es), date)}>>.  
 
This agreement should be submitted in paper form or in the form of a scanned copy (signed by the authors). 

The Editorial Board has the right to request from the authors an official expert conclusion 
that the submitted article does not have secret data prohibited for publication. 


\item A submitted article should be attached with \textbf{the data on the author(s)} (see p.~ 8). 
If there are several authors, the contact person should be indicated who is responsible 
for correspondence with the Editorial Board. 

\item The Editorial Board of the Journal examines the article according to the established reviewing procedure. 
If authors receive their article for correction after reviewing it does not mean that the article is approved 
to be published. The corrected article should be sent to the Editorial Board for the subsequent review and approval.

\item The decision on the article publication or its rejection is communicated to the authors. The Editorial Board may 
also send the reviews on the submitted articles to the authors. Any discussion upon the rejected articles is not possible.

\item The edited articles will be sent to the authors for proofread. The comments of the authors to the edited 
text of the article should be sent to the Editorial Board as soon as possible. 

\item The manuscript of the article should be presented electronically in the MS WORD (.doc or .docx) 
or \LaTeX\ (.tex) formats and, additionally, in the .pdf format. All documents may be sent by e-mail or 
on a CD or diskette. A~hard copy submission is not necessary.

\item The recommended typesetting instructions for manuscript.  

Pages parameters: format A4, portrait orientation, document margins (cm): 
left~--- 2.5, right~--- 1.5, above~--- 2.0, below~--- 2.0, footer~1.3. 

Text: font~--- Times New Roman, font size~--- 14, paragraph~--- 0.5, line spacing~--- 1.5, justified alignment. 
 
The recommended manuscript size: no more than 20 pages of the specified format. 

Word abbreviations are not allowed except the standard ones. 

Abbreviations should be minimal. All pages of the manuscript should be numbered.

The templates for the manuscript typesetting are presented on site: 


{\sf http://www.ipiran.ru/journal/template.doc}.

\item Articles should enclose data both in \textbf{Russian and English}: 
\begin{itemize}
\item title;
\item  author(s) name(s) and surname(s); 
\item affiliation~--- organization, its address with ZIP code, city, country, and e-mail address; 
\item   data on authors according to the format (see site):

{\sf http://www.ipiran.ru/journal/issues/2013\_07\_01/authors.asp}  and

{\sf 
http://www.ipiran.ru/journal/issues/2013\_07\_01\_eng/authors.asp};

\item abstract (not less than 100~words) both in Russian and in English. Abstract is a short 
summary of the article that can be published separately from the article. The abstract is 
the main source of information on the article and it could be included in leading information 
systems and data bases. The abstract in English has to be an original text and should not be 
an exact translation of the Russian one. Good English is required;

\pagebreak

\thispagestyle{empty}

\vspace*{-36pt}


\item indexing is performed on the basis of key words. The use of key words from the internationally 
accepted thematic Thesauri is recommended. 

Important! Key words must not be sentences.
\end{itemize}

\item References. Russian references have to be presented both in Cyrillic and Latin transliteration. 

References in Latin transcript are presented in original language.

References in the text are numbered according to the order of the appearance 
and the number is placed in square bracket. References absent in the text should not 
be included into the list of references.
 

\item Manuscripts and additional materials are not returned to authors by the Editorial Board. 

\item Submissions of files by e-mail must include: 
\begin{itemize}
\item  the journal title and author(s) name(s) in the ``Subject'' field; 
\item  an article and additional materials have to be attached using the ``attach'' function;
\item  an electronic version of the article should contain the file with the text and (a) 
separate file(s) with figures. 
\end{itemize}
\item  ``Informatics and Applications'' Journal is not a profit publication. There are no charges 
for the authors as well as there are no royalties.
\end{enumerate}

\noindent
\textbf{Editorial Board address:}\  119333, IPIRAN, Vavilova St., 44, block 2, Moscow, Russia

\noindent
\hphantom{\textbf{Editorial Board address:}\ }Ph.: +7(499) 135 8692,\ \  Fax: +7 (495) 930 4505

\noindent
\hphantom{\textbf{Editorial Board address:}\ }e-mail: {\sf rust@ipiran.ru} 
(To Prof.\ Rustem Seyfoul-Mulyukov)


\vfill
\begin{center}


Технический редактор Л. Кокушкина\\
%Выпускающий редактор Т. Торжкова\\
Художественный редактор М. Седакова\\
Сдано в набор 01.07.13. Подписано в печать 19.09.13. Формат 60 х 84 / 8\\
Бумага офсетная. Печать цифровая. Усл.-печ.\ л.\ 17,25. Уч.-изд.\ л.\ 15,0. Тираж 100 экз.\\
\ \\
Заказ №\,3951\\
\ \\
Издательство <<ТОРУС ПРЕСС>>, Москва 121614, ул. Крылатская, 29-1-43\\
torus@torus-press.ru; http://www.torus-press.ru\\
\ \\
Отпечатано в Академиздатцентре <<Наука>> РАН с готовых файлов\\
Москва 121099, Шубинский пер., д.~6\\
\end{center}

%\vspace*{-60pt} {\small
{\baselineskip=9.1pt
\section*{Правила подготовки рукописей статей для публикации в журнале
<<Информатика и её применения>>}

\thispagestyle{empty}

 Журнал <<Информатика и её применения>> публикует
теоретические, обзорные и дискуссионные статьи, посвященные научным
исследованиям и разработкам в области информатики и ее приложений. Журнал
издается на русском языке. По специальному решению редколлегии отдельные статьи,
в виде исключения, могут печататься на английском языке.
Тематика журнала охватывает следующие направления:
\begin{itemize}
\item теоретические основы информатики; %\\[-13.5pt]
\item математические методы исследования сложных систем и процессов; %\\[-13.5pt]
\item информационные системы и сети; %\\[-13.5pt]
\item информационные технологии; %\\[-13.5pt]
\item архитектура и программное
обеспечение вычислительных комплексов и сетей.
\end{itemize}
\begin{enumerate}
\item В журнале печатаются результаты, ранее не
опубликованные и не предназначенные к одновременной публикации в других
изданиях. Публикация не должна нарушать закон об авторских правах. Направляя
свою рукопись в редакцию, авторы автоматически передают учредителям и
редколлегии неисключительные права на издание данной статьи на русском языке и
на ее распространение в России и за рубежом. При этом за авторами сохраняются
все права как собственников данной рукописи. В связи с этим авторами должно
быть представлено в редакцию письмо в следующей форме:
Соглашение о передаче права на публикацию:

\textit{<<Мы, нижеподписавшиеся, авторы рукописи <<$\qquad\qquad$>>, передаем
учредителям и редколлегии журнала <<Информатика и её применения>>
неисключительное право опубликовать данную рукопись статьи на русском языке как
в печатной, так и в электронной версиях журнала. Мы подтверждаем, что данная
публикация не нарушает авторского права других лиц или организаций. Подписи
авторов: (ф.\,и.\,о., дата, адрес)>>.}

Указанное соглашение может быть представлено 
как в бумажном виде, так и в виде отсканированной копии (с подписями авторов).


Редколлегия вправе запросить у авторов экспертное заключение о возможности
опубликования представленной статьи в открытой печати. %\\[-13.5pt]
\item Статья
подписывается всеми авторами. На отдельном листе представляются данные автора
(или всех авторов): фамилия, полные имя и отчество, телефон, факс, e-mail,
почтовый адрес. Если работа выполнена несколькими авторами, указывается фамилия
одного из них, ответственного за переписку с редакцией. %\\[-13.5pt]
\item Редакция журнала
осуществляет самостоятельную экспертизу присланных статей. Возвращение рукописи
на доработку не означает, что статья уже принята к печати. Доработанный вариант
с ответом на замечания рецензента необходимо прислать в редакцию. %\\[-13.5pt]
\item Решение
редакционной коллегии о принятии статьи к печати или ее отклонении сообщается
авторам. Редколлегия не обязуется направлять рецензию авторам отклоненной
статьи; дискуссия с авторами по поводу отклоненных статей не ведется. %\\[-13.5pt]
\item Корректура статей высылается авторам для просмотра. Редакция
просит авторов присылать свои замечания в кратчайшие сроки. %\\[-13.5pt]
\item При
подготовке рукописи в MS Word рекомендуется использовать следующие настройки.
Параметры страницы: формат~--- А4; ориентация~--- книжная; поля (см): внутри~---
2,5, снаружи~--- 1,5, сверху~--- 2, снизу~--- 2, от края до нижнего
колонтитула~--- 1,3. Основной текст: стиль~--- <<Обычный>>: шрифт Times New
Roman, размер 14~пунктов, абзацный отступ~--- 0,5~см, 1,5 интервала,
выравнивание~--- по ширине. Рекомендуемый объем рукописи~--- не свыше
25~страниц указанного формата. Ознакомиться с шаблонами, содержащими примеры
оформления, можно по адресу в Интернете:
\textsf{http://www.ipiran.ru/journal/template.doc}.
\item К рукописи, предоставляемой в 2-х
экземплярах, обязательно прилагается электронная версия статьи (как правило, в
форматах MS WORD (.doc) или \LaTeX\ (.tex), а также~--- дополнительно~--- в
формате .pdf) на дискете, лазерном диске или по электронной почте. Сокращения
слов, кроме стандартных, не применяются. Все страницы рукописи должны быть
пронумерованы. %\\[-13.5pt]
\item Статья должна содержать следующую информацию на русском и
английском языках: название, Ф.И.О. авторов, места работы авторов и их
электронные адреса, подробные сведения об авторах, оформленные в соответствии с форматом, 
определяемым файлами {\sf http://www.ipiran.ru/journal/issues/2011\_05\_01/authors.asp} и 
{\sf http://www.ipiran.ru/journal/issues/2011\_01\_eng/authors.asp},
аннотация (не более 100~слов), ключевые слова. Ссылки на
литературу в тексте статьи нумеруются (в квадратных скобках) и располагаются в
порядке их первого упоминания. В~списке литературы не должно быть позиций, на которые нет ссылки в тексте статьи.
Все фамилии авторов, заглавия статей, названия
книг, конференций и~т.\,п.\ даются на языке оригинала, если этот язык
использует кириллический или латинский алфавит. %\\[-13.5pt]
\item Присланные в редакцию материалы авторам не возвращаются.
\item При отправке файлов по электронной
почте просим придерживаться следующих правил:
\begin{itemize}
\item указывать в поле subject (тема) название журнала и фамилию автора; %\\[-13.5pt]
\item использовать attach (присоединение); %\\[-13.5pt]
\item в случае больших объемов информации возможно
использование общеизвестных архиваторов (ZIP, RAR); %\\[-13.5pt]
\item в состав электронной версии статьи должны входить: файл, содержащий текст статьи, и файл(ы),
содержащий(е) иллюстрации. %\\[-13.5pt]
\end{itemize}
\item Журнал <<Информатика и её применения>> является некоммерческим изданием. 
Плата за публикацию с авторов не взимается, гонорар авторам не выплачивается.
\end{enumerate}
\thispagestyle{empty}
\textbf{Адрес редакции:} Москва 119333,
ул.~Вавилова, д.~44, корп.~2, ИПИ РАН\\
\hphantom{\textbf{Адрес редакции:} }Тел.: +7 (499) 135-86-92\ \
Факс:  +7 (495) 930-45-05\ \  E-mail:   rust@ipiran.ru }
}

%\include{ipi-ind}

%\tableofcontents

\end{document}

%\tableofcontents

%\end{document}

%\tableofcontents


\end{document}

\newcommand{\Ack}{\subsection*{\protect\large\bf Acknowledgments}}

\vphantom{\int\limits_0^T }

{ \begin{center}  %fig1
 \vspace*{3pt}
    \mbox{%
 \epsfxsize=79mm 
 \epsfbox{gru-1.eps}
 }

\end{center}

\noindent
{{\figurename~1}\ \ \small{
Временные зависимости данные 
}}}

\vspace*{6pt}

\setcounter{figure}{1}

$\acute{\mbox{о}}$

\linebreak