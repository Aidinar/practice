\def\stat{gon-zats}

\def\tit{ПРЕДСТАВЛЕНИЕ НОВЫХ ЛЕКСИКОГРАФИЧЕСКИХ ЗНАНИЙ В~ДИНАМИЧЕСКИХ 
КЛАССИФИКАЦИОННЫХ СИСТЕМАХ$^*$}

\def\titkol{Представление новых лексикографических знаний в~динамических 
классификационных системах}

\def\aut{А.\,А.~Гончаров$^1$, И.\,М.~Зацман$^2$, М.\,Г.~Кружков$^3$}

\def\autkol{А.\,А.~Гончаров, И.\,М.~Зацман, М.\,Г.~Кружков}

\titel{\tit}{\aut}{\autkol}{\titkol}

\index{Гончаров А.\,А.}
\index{Зацман И.\,М.}
\index{Кружков М.\,Г.}
\index{Goncharov A.\,A.}
\index{Zatsman I.\,M.}
\index{Kruzhkov M.\,G.}

{\renewcommand{\thefootnote}{\fnsymbol{footnote}} \footnotetext[1]
{Работа выполнена в~Институте проблем информатики ФИЦ ИУ РАН при поддержке РФФИ (проект  
20-012-00166).}}

\renewcommand{\thefootnote}{\arabic{footnote}}
\footnotetext[1]{Институт проблем информатики Федерального исследовательского центра <<Информатика и~управление>> 
Российской академии наук, \mbox{a.gonch48@gmail.com}}
\footnotetext[2]{Институт проблем информатики Федерального исследовательского центра <<Информатика и~управление>> 
Российской академии наук, \mbox{izatsman@yandex.ru}}
\footnotetext[3]{Институт проблем информатики Федерального исследовательского центра <<Информатика и~управление>> 
Российской академии наук, \mbox{magnit75@yandex.ru}}


%\vspace*{-12pt}


     
     \Abst{Характерная особенность динамических классификационных систем (ДКС) состоит 
     в~том, что в~процессе применения этих систем в~них в~любой момент времени могут 
добавляться новые рубрики и/или изменяться дефиниции существующих рубрик, включая 
перераспределение смыслового содержания между ними. С~одной стороны, эта 
особенность ДКС дает возможность оперативно 
отражать в~них новое знание и~сразу начинать его использовать, например в~процессе 
лингвистического аннотирования. С~другой стороны, если некоторая рубрика 
использовалась при аннотировании, а~затем была изменена, то аннотации с~этой 
рубрикой, сформированные до внесения изменений, в~ряде случаев должны быть 
реклассифицированы. Статья преследует двоякую цель, которая состоит, во-пер\-вых, 
в~сопоставлении подходов к~классификации сущностей на основе (1)~ДКС 
и~(2)~онтологий, изменяемых во времени, а~во-вто\-рых, 
в~описании специфики представления новых лексикографических знаний в~ДКС.}
     
     \KW{динамическая классификационная система; версионные онтологии; 
лингвистическое аннотирование; реклассификация аннотаций}

\DOI{10.14357/19922264210112}


\vskip 10pt plus 9pt minus 6pt

\thispagestyle{headings}

\begin{multicols}{2}

\label{st\stat}
     
\section{Введение}

Характерная особенность ДКС 
состоит в~том, что в~процессе применения в~них могут добавляться новые 
рубрики и/или изменяться дефиниции существующих рубрик, включая 
перераспределение смыслового содержания между ними. В~отличие от 
версионных классификационных систем, для которых установлен период их 
обновления (например, рубрики Международной патентной классификации 
могут меняться не чаще, чем раз в~квартал~[1]), добавления и~изменения 
в~ДКС в~случае необходимости могут быть сделаны в~любой момент 
времени. С~одной стороны, эта особенность ДКС позволяет оперативно 
отражать в~них новое знание и~сразу начинать его использовать, например 
в~процессе лингвистического аннотирования~[2]. С~другой стороны, если 
некоторая рубрика использовалась при аннотировании, а~затем была 
изменена, то аннотации с~этой рубрикой, сформированные до внесения 
изменений, в~ряде случаев должны быть реклассифицированы~[3].

Цель статьи состоит в~сопоставлении подходов к~классификации сущностей 
на основе (1)~ДКС и~(2)~онтологий, изменяемых во времени, а также 
в~описании специфики представления новых лексикографических знаний 
в~ДКС. В~качестве примера ДКС в~статье рассматривается фасетная 
классификация (ФК) надкорпусной базы данных (НБД)~[4--6]. В~проекте по 
гранту №\,20-012-00166\linebreak НБД используется для аннотирования таких 
сущностей, как употребления немецких модальных глаго\-лов (НМГ) 
в~параллельных текстах~[7--9], в~процессе которого могут 
(1)~обнаруживаться новые значения НМГ, (2)~добавляться новые рубрики 
для этих значений в~момент их обнаружения~[10, 11] и~(3)~изменяться 
дефиниции рубрик ФК~[3].

\vspace*{-14pt}

\section{Онтологии, изменяемые во~времени}

\vspace*{-4pt}

Методы аннотирования сущностей с~использованием, с~одной стороны, ДКС 
и,~с~другой стороны, онтологий, изменяемых во времени, во многом схожи. 
При аннотировании сущностей \mbox{с~по\-мощью} ДКС им присваиваются рубрики, в~том числе 
до\-бав\-лен\-ные или измененные непосредственно в~процессе 
формирования ка\-кой-ли\-бо аннотации. При классификации сущностей 
с~по\-мощью онтологии для них устанавливаются атрибуты, пред\-став\-ля\-ющие 
собой ссылки на концепты онтологии. С~течением времени могут меняться 
структура и~наполнение как ДКС, так и~онтологии. Это может быть связано 
с~изменениями (1)~самой предметной об\-ласти, (2)~экспертных знаний об этой 
области и/или (3)~стандартов ее описания~[12].

Кроме этого, при проведении исследований научные коллективы могут 
пользоваться теми классификационными системами и~онтологиями для \mbox{своей} 
предметной области, которые находятся в~откры\-том доступе, а~затем 
модифицировать их в~зависимости от целей и~задач выполняемых проектов. 
В~силу перечисленных сходств работы, связанные с~эволюцией 
и~версионностью онтологий, представляют значительный интерес для 
разработки средств актуализации ДКС.

В~[12] описывается подход к~управлению версионностью онтологий 
и~предлагается система нумерации версий онтологий, позволяющая 
определять, обладают ли версии свойством обратной \mbox{совместимости}, или 
эквивалентности (с~точ\-ностью до синтаксических различий). Обратная 
совместимость важна потому, что в~случае ее сохранения снимается 
необходимость в~реклассификации ранее сформированных аннотаций (об 
этой проблеме применительно к~ДКС на примере ФК НБД см.~[3]). 
Сохранять историю изменений понятий онтологии предлагается либо 
в~отдельной онтологии, либо в~выделенных для этого классах исходной 
онтологии. В~\cite{13-gon} представлены инструменты, позволяющие 
выделять и~визуализировать структурные различия между версиями одной 
и~той же онтологии.

В~\cite{14-gon} показано, как с~помощью онтологий могут фиксироваться 
временн$\acute{\mbox{ы}}$е данные, знания, правила и~отношения, 
и~рассмотрены подходы к~описанию изменений в~онтологиях. Темпоральная 
дескриптивная логика (Temporal Description Logics) служит основой 
дополнения онтологии логическими функциями для работы со временем, 
поз\-во\-ля\-ющи\-ми выражать и~обрабатывать такие концепты,\linebreak как <<постоянно 
в~прошлом>> или <<в~некоторый момент в~будущем>>. Для регистрации 
фактов, относящихся к~определенному периоду времени, используются 
подходы, позволяющие обходить \mbox{ограничения} большинства онтологий, 
опирающихся на язык Web Ontology Language (OWL): реификация 
(Reification), четырехмерные (4D-fluent) онтологии и~переход к~n-ар\-ным 
отношениям (N-ary relations). Поскольку по умолчанию отношения 
в~онтологиях OWL являются бинарными, тогда как для фиксации 
временн$\acute{\mbox{о}}$го интервала события необходим по крайней мере 
один дополнительный аргумент, все перечисленные подходы 
характеризуются тем, что для описания временн$\acute{\mbox{ы}}$х фактов 
в~онтологиях каждый раз создается новая сущность.\linebreak На основе 
вышеназванных подходов разработана структура SOWL (spatio-temporal OWL), пред\-на\-значенная 
для включения %\linebreak 
про\-стран\-ст\-вен\-но-вре\-мен\-н$\acute{\mbox{о}}$й информации 
в~онтологии OWL, а~также\linebreak инструмент CHRONOS, упро\-ща\-ющий создание 
и~редактирование данных об \mbox{изменениях} в~онтологиях, ис\-поль\-зу\-ющих эту 
структуру. Кроме того, для извлечения информации из SOWL-он\-то\-ло\-гий 
создан язык запросов, описанный в~[15].

В~[16] дается обзор процессов и~алгоритмов, связанных с~развитием 
онтологий (ontology evolution). Комплекс таких процессов описан как единый 
и~непрерывный цикл, который можно разделить на несколько этапов: 

\begin{enumerate}[(1)]
\item выявление потребности в~изменениях; 
\item формулировка предлагаемых 
изменений; 
\item оценка адекватности предлагаемых изменений; 
\item оценка 
последствий реализации изменений; 
\item внедрение изменений.
\end{enumerate}

 Каждый этап 
рассматривается отдельно с~учетом опыта, полученного в~ходе других 
исследований.

Следует отметить ряд особенностей подхода, предлагаемого в~настоящей 
работе, в~сравнении с~\cite{12-gon, 13-gon, 14-gon, 15-gon, 16-gon}.  
Во-пер\-вых, он нацелен не на описание любых изменений онтологий во 
времени, а~на фиксирование дополнений, изменений порядка и~смыслового 
содержания рубрик самой ДКС. Таким образом, он ближе всего к~работам, 
где описываются подходы к~управлению различными версиями онтологий 
и~развитию онтологий; работы же, посвященные описанию темпоральных 
сущностей в~онтологиях, как правило, имеют иную целевую направленность.

\begin{figure*}[b] %fig1
\vspace*{1pt}
\begin{center}
\mbox{%
\epsfxsize=88mm
\epsfbox{gon-1.eps}
}
\end{center}

\vspace*{3pt}

\noindent
{\small %\begin{center}
{Таблицы для хранения истории изменений рубрик ФК в~НБД.}
%\end{center} %\newline
%
%\vspace*{-6pt}
%
%\noindent
Поля таблицы \textbf{PropHistory}: %\protect\newline
%\begin{itemize}
%\item 
\textbf{PropId} --- уникальный идентификатор рубрики (соответствующей значению НМГ);
\textbf{Code}~--- краткое обозначение рубрики (для значений НМГ обычно имеет следующий вид: 
модальный глагол, дефис, номер значения, например, <<sollen-01>>);
%\item 
\textbf{Name}~--- дефиниция рубрики;
\textbf{isCurrent}~--- признак актуальности данного состояния рубрики (1~--- актуально, 0~--- не 
актуально);
%\item
\textbf{OperationId}~--- идентификатор операции, в~результате которой рубрика приняла вид, 
соответствующий данной строке таблицы;
%\item
\textbf{OperationAttr}~--- атрибут, который присваивается рубрике в~рамках операции с~Id, 
указанным в~поле OperationId (см.\ об атрибутах и~их значениях в~операциях в~табл.~3).
%\end{itemize}
Поля таблицы \textbf{PropOperation}:
%\begin{itemize}
%\item 
\textbf{Id}~--- идентификатор операции (обеспечивает связь между операциями и~рубриками, 
которые они затрагивают);
%\item 
\textbf{Operation}~--- задает тип операции (допустимые значения: CREATE, DELETE, MERGE, 
SPLIT, REORDER, REDISTR, REVISE);
%\item 
\textbf{UserId}~--- идентификатор пользователя, осуществившего операцию;
%\item 
\textbf{TimeDate}~--- дата и~время выполнения операции
%\end{itemize}
}
\end{figure*}

Во-вто\-рых, хотя структура онтологии обычно сложнее, чем структура ДКС, 
это отличие не имеет принципиального значения для задач  
классификации~--- описываемую здесь ФК в~качестве примера ДКС можно 
рассматривать как потенциальную составляющую онтологии знаний по 
лексикографии немецкого языка. Узкая направленность ФК позволяет 
уделить больше внимания непосредственно теме исследования в~рамках 
упомянутого проекта, что немаловажно, поскольку выделение значений НМГ 
является нетривиальной задачей, и~помимо таких операций, как слияние или 
разделение значений, порой требуется перенести часть компонентов 
смыслового содержания из одного значения в~другое, изменить нумерацию 
значений в~соответствии с~последовательностью их описания в~словарных 
статьях, предназначенных для включения в~словарь~\cite{17-gon}, и~т.\,д.

В-третьих, описываемая ФК вложена в~НБД~--- объекты аннотирования 
и~сама ФК физически располагаются в~одной и~той же базе данных, 
благодаря чему разработчикам не нужно беспокоиться о~том, что  
ка\-кие-ли\-бо внешние данные окажутся не\-со\-вмес\-ти\-мы\-ми с~той или иной 
версией ФК. Однако при внесении изменений в~ФК важно в~случае 
необходимости сразу же вносить соответствующие изменения в~аннотации, 
сформированные ранее. Если возможно, это делается автоматически, 
в~противном случае система должна помечать аннотации, затронутые 
изменениями, чтобы их реклассифицировали эксперты~\cite{3-gon}.

\vspace*{-9pt}

\section{Динамическая классификационная система}

\vspace*{-2pt}

В отличие от упомянутых выше решений по поддержке версионности 
онтологий в~предлагаемом подходе и~его реализации в~проекте не 
предусмот\-ре\-на нумерация версий или создание новой версии ФК после 
вносимых изменений. Вместо этого для каждой рубрики ФК в~НБД 
фиксируются все ее изменения и~для них проставляются временн$\acute{\mbox{ы}}$е штампы, 
а~все старые версии сохраняются. Поэтому можно проследить историю 
эволюции каждой рубрики, включая ее взаимодействие с~другими 
руб\-ри\-ка\-ми, а~также восстановить состояние ФК на любой момент времени.

В структуре НБД за сохранение изменений отвечают две связанные между 
собой таблицы: в~первой (PropHistory) хранятся все~--- текущие 
и~устаревшие~--- состояния рубрик ФК (которые в~проекте соответствуют 
значениям НМГ), а~во второй (\mbox{PropOperation})~--- все операции изменения, 
которые применялись к~этим рубрикам. Структура таблиц и~связь между 
ними показаны на рисунке.



Различаются следующие виды операций. 
\begin{enumerate}[1.]
\item CREATE~--- создание новой рубрики.
\item REORDER~--- изменение кода рубрики (соответствует номеру 
значения НМГ по словарю~\cite{17-gon}).
\item REVISE~--- изменение дефиниции рубрики (не затрагивающее 
никакие другие рубрики).
\item MERGE~--- слияние дефиниций двух рубрик, в~результате которого 
одна из рубрик удаляется.
\item DELETE~--- удаление рубрики.
\item SPLIT~--- разделение дефиниции одной рубрики на две, в~результате 
чего создается новая рубрика, а~дефиниция исходной рубрики 
перераспределяется между старой и~новой рубриками.
\item REDISTR (от англ.\ \textit{redistribute})~--- изменение дефиниций 
двух рубрик, предусматривающее перенос части компонентов смыслового 
содержания из одной рубрики в~другую.
\end{enumerate}

\begin{table*}\small %tabl1
\begin{center}
\Caption{Пример выделения компонентов смыслового содержания дефиниции рубрики}
\vspace*{2ex}

\begin{tabular}{|c|c|c|p{52mm}|p{64mm}|}
\hline
Рубрика&Id&Код&\multicolumn{1}{c|}{\tabcolsep=0pt\begin{tabular}{c}
Дефиниция рубрики\\ без структурного выделения\\ компонентов 
смыслового\\ содержания\end{tabular}} &
\multicolumn{1}{c|}{\tabcolsep=0pt\begin{tabular}{c}
Дефиниция рубрики,\\ в~которой выделены компоненты\\ ее 
смыслового содержания\end{tabular}}\\
\hline
X&482&sollen-01&Обязанность что-л. делать по чье\-\mbox{му-л.}\ указанию, по закону, по 
правилам и~т.\,п.: должен. Моральный запрет (под отрицанием): \mbox{нельзя}&$a$.~Обязанность 
что-л.\ делать по чьему-л.\ указанию, по закону, по правилам и~т.\,п.: должен\newline
$b$.~Моральный запрет (под отрицанием): \mbox{нельзя}\\
\hline
\end{tabular}
\end{center}
\end{table*}


В то время как для понимания операций~\mbox{1--6} достаточно приведенных 
определений, операция REDISTR заслуживает более детального 
рассмотрения. Для этого понадобятся условные обозначения, введенные 
в~\cite{3-gon}, а~именно:
\begin{itemize}
\item X, Y,\ \ldots~--- рубрики ФК, обозначающие смыс\-ло\-вые значения 
НМГ;
\item def$_{\mathrm{X}}$, def$_{\mathrm{Y}}$,\ \ldots~---  дефиниции 
рубрик;
\item $\mathbf{S}_{\mathrm{def}_{\mathrm{X}}}$, 
$\mathbf{S}_{\mathrm{def}_{\mathrm{Y}}}$, \ldots ~--- смысловое 
содержание дефиниций рубрик.
\end{itemize}
Кроме того, для обозначения сущностей, которые были каким-то образом 
изменены в~результате выполнения операции, используется индекс <<ch>> 
(от англ.\ \textit{changed}).

Операция REDISTR выполняется только в~том случае, если 
def$_{\mathrm{X}}$ такова, что 
в~$\mathbf{S}_{\mathrm{def}_{\mathrm{X}}}$~--- смысловом содержании 
дефиниции рубрики (соответствующей значению НМГ)~--- выделены 
компоненты (соответствующие подзначениям внутри значения 
НМГ\footnote{С~точки зрения смыслового содержания в~дефиниции рубрики могут быть 
выделены и~более мелкие части, соответствующие частям подзначений. Данная ситуация не 
рассматривается в~рамках настоящей статьи.}). В табл.~1 приводится пример  
руб\-ри\-ки~X, в~дефиниции которой структурно выделены компоненты ее 
смыслового содержания, соответствующие подзначениям,~--- $a$ и~$b$.




Если представить $\mathbf{S}_{\mathrm{def_X}}$, компонентами которого 
являются~$a$ и~$b$, в~виде множества 
$\mathbf{S}_{\mathrm{def_X}}\{a,b\}$, то изменение def$_{\mathrm{X}}$ 
может быть таким, что набор элементов этого множества: (1)~сократится: 
$\mathbf{S}_{\mathrm{def_X}}^1\hm=\{a\}$; (2)~увеличится: 
$\mathbf{S}_{\mathrm{def_X}}^2\hm=\{a, b, c, \ldots\}$.



Более того, возможна ситуация, когда изменение набора элементов затронет 
не только def$_{\mathrm{X}}$, а~одновременно def$_{\mathrm{X}}$ 
и~def$_{\mathrm{Y}}$. Рубрики до внесения изменений обозначим как~X 
и~Y, а~после их внесения~--- как X$_{\mathrm{ch}}$ и~Y$_{\mathrm{ch}}$. 
Если изменение руб\-рик~X и~Y таково, что  $\mathbf{S}_{\mathrm{def_X}} 
\cap \mathbf{S}_{\mathrm{def_Y}}^{\mathrm{ch}}\not= \emptyset$ 
(смысловое содержание def$_{\mathrm{X}}$ и~смысловое содержание 
измененной def$_{\mathrm{Y}}$ имеют один или более общих компонентов) 
и/или $\mathbf{S}_{\mathrm{def_Y}} \cap 
\mathbf{S}_{\mathrm{def_X}}^{\mathrm{ch}}\not= \emptyset$ (смысловое 
содержание def$_{\mathrm{Y}}$ и~смысловое содержание измененной 
def$_{\mathrm{X}}$ имеют один или более общих компонентов), то оно может 
быть описано с~по\-мощью операции \mbox{REDISTR\,(X, Y)}. Экспертная 
реклассификация потребуется для тех аннотаций, которые до выполнения 
операции REDISTR содержали: (1)~руб\-ри\-ку~X, смысловое содержание 
которой в~результате выполнения операции REDISTR сужается;  
(2)~руб\-ри\-ку~X или~Y, если перенос компонентов смыслового содержания 
дефиниций осуществляется как из~X в~Y, так и~из~Y в~X.


Ниже приводится пример выполнения операции REDISTR для рубрик 
с~постоянными номерами (id) 482 и~484, которые позволяют отслеживать 
историю изменений рубрики (табл.~2). В~данном примере: X~---  
руб\-ри\-ка~482 до выполнения операции, причем 
$\mathbf{S}_{\mathrm{def_X}}\hm= \{a, b\}$; Y~--- руб\-ри\-ка~484 до 
выполнения операции, причем $\mathbf{S}_{\mathrm{def_Y}}\hm= 
\{m,n,o\}$; X$_{\mathrm{ch}}$~--- руб\-ри\-ка~482 после выполнения операции, 
причем $\mathbf{S}_{\mathrm{def_X}}^{\mathrm{ch}}\hm= \{a, b, n, o\}$; 
Y$_{\mathrm{ch}}$~--- руб\-ри\-ка~484 после выполнения операции, причем 
$\mathbf{S}_{\mathrm{def_Y}}^{\mathrm{ch}}\hm=\{m\}$.


Поскольку $\mathbf{S}_{\mathrm{def_Y}}\cap 
\mathbf{S}_{\mathrm{def_X}}^{\mathrm{ch}}\hm=\{n, o\}$ 
и~$\mathbf{S}_{\mathrm{def_X}}\cap 
\mathbf{S}_{\mathrm{def_Y}}^{\mathrm{ch}}\hm=\emptyset$, смысловое 
содержание руб\-ри\-ки~482 после внесения изменения расширяется, 
а~смысловое содержание руб\-ри\-ки~484~--- сужается. Следовательно,\linebreak 
аннотации, которые до выполнения операции \mbox{REDISTR} содержали 
 руб\-ри\-ку~482, после ее выполнения не требуют экспертной 
реклассификации и~автоматически обозначаются кодом <<sollen-01>>, тогда 
как аннотации, которые до выполнения операции REDISTR содержали  
руб\-ри\-ку~484, после ее выполнения требуют экспертной реклассификации 
и~поэтому автоматически помечаются тегом <<\mbox{TBR-R}>> (от англ. \textit{To 
Be Reclassified because of Redistribution}).

\begin{table*}\small %tabl2
\begin{center}
\Caption{Исходные данные и~результат выполнения операции REDISTR}
\vspace*{2ex}

\begin{tabular}{|c|c|c|p{120mm}|}
\hline
Рубрика&Id&Код&Дефиниции двух рубрик, в~которых структурно выделены 
компоненты их смыслового содержания, до и~после выполнения операции 
\mbox{REDISTR}\\
\hline
X&482&sollen-01&$a$.~Обязанность что-л.\ делать по чьему-л.\ указанию, по закону, по 
правилам и~т.\,п.: должен\\
&&&$b$.~Моральный запрет (под отрицанием): нельзя\\
\hline
Y&484&sollen-03&$m$.~Желательность по мнению говорящего (в~формах praet conj 
и~pqp conj): следовало (бы), нужно было (бы), должно было (бы)\\
&&&$n$. Совет, рекомендация (только в~формах praet conj)\\
&&&$o$. Нежелательность (под отрицанием): не следовало (бы), нельзя\\
\hline
X$_{\mathrm{ch}}$&482&sollen-01&$a$. Обязанность что-л.\ делать по чьему-л.\ 
указанию, по закону, по правилам и~т.\,п.: должен\\
&&&$b$.~Моральный запрет (под отрицанием): нельзя\\
&&&$n$.~Совет, рекомендация (только в~формах praet conj)\\
&&&$o$.~Нежелательность (под отрицанием): не следовало (бы), нельзя\\
\hline
Y$_{\mathrm{ch}}$&484&sollen-03&$m$.~Желательность по мнению говорящего (в 
формах praet conj и~pqp conj): следовало (бы), нужно было (бы), должно было (бы)\\
\hline
\multicolumn{4}{p{162.7mm}}{\footnotesize\hspace*{3mm}\textbf{Примечания.}
Расшифровка используемых в~таблице сокращений:
\begin{itemize}
\addtolength{\itemsep}{-4pt}
\item praet conj~--- форма прошедшего времени (лат.\ \textit{praeteritum}) сослагательного 
наклонения (лат.\ \textit{conjunctivus});
\item pqp conj~--- форма предпрошедшего времени (лат.\ \textit{plusquamperfectum}) 
сослагательного наклонения (лат.\ \textit{conjunctivus}).
\end{itemize}
Примеры употребления глагола \textit{sollen} по словарю~\cite{17-gon}, иллюстрирующие каждый 
из компонентов смыслового содержания дефиниций (формы глагола \textit{sollen} выделены 
полужирным шрифтом):
\begin{itemize}
\addtolength{\itemsep}{-4pt}
\item[$a$.] ich \textbf{soll} heute noch in die Stadt fahren~--- я должен сегодня еще поехать 
в~город;
\item[$b$.] du \textbf{sollst} nicht t$\ddot{\mbox{o}}$ten!~--- не убий! 
(\textit{библейская заповедь});
\item[$m$.] das \textbf{sollte} sie doch wissen~--- это она (вообще-то) должна была (бы) 
знать;
\item[$n$.] Sie \textbf{sollten} mit dem Rauchen aufh$\ddot{\mbox{o}}$ren~--- вам следует 
бросить курить;
\item[$o$.] das \textbf{sollte} man nie tun~--- этого не следует делать.
\end{itemize}
}
\end{tabular}
\end{center}
\vspace*{-23pt}
%\end{table*}
%\begin{table*}\small %tabl3  %\multicolumn{1}{|c|}{\raisebox{-6pt}[0pt][0pt]{
\begin{center}
\Caption{Соответствия между операциями, атрибутами и~рубриками}
\vspace*{2ex}

\tabcolsep=4.2pt
\begin{tabular}{|l|c|p{105mm}|}
\hline
\multicolumn{1}{|c|}{Операция}&Атрибут&\multicolumn{1}{c|}{Рубрика}\\
\hline
{\raisebox{-6pt}[0pt][0pt]{\tabcolsep=0pt\begin{tabular}{l}CREATE (создается новая\\ 
рубрика X)\end{tabular}}}&A&Создаваемая рубрика X\\
\cline{2-3}
&C&Рубрики, которые надо перенумеровать после выполнения операции, чтобы 
освободить в~нумерации нужную позицию для X\\
\hline
{\raisebox{-6pt}[0pt][0pt]{\tabcolsep=0pt\begin{tabular}{l}REORDER (код рубрики~X\\ изменяется)\end{tabular}}}&A&Рубрика X, код которой изменяется\\
\cline{2-3}
&C&Рубрики, которые надо перенумеровать после выполнения операции, чтобы 
освободить в~нумерации нужную позицию для X\\
\hline
REVISE (изменяется def$_{\mathrm{X}}$)&A&Рубрика X, дефиниция которой изменяется\\
\hline
{\raisebox{-6pt}[0pt][0pt]{\tabcolsep=0pt\begin{tabular}{l}MERGE (объединяются\\ def$_{\mathrm{X}}$ 
и~def$_{\mathrm{Y}}$)\end{tabular}}}&А&Рубрика X, причем 
def$_{\mathrm{X}}$ поглощает def$_{\mathrm{Y}}$, а X остается в~базе данных\\
\cline{2-3}
&B&Рубрика Y, причем def$_{\mathrm{Y}}$ включается в~def$_{\mathrm{X}}$, а~Y 
удаляется из базы данных\\
\cline{2-3}
&C&Рубрики, которые напрямую не затрагиваются операцией MERGE, но которые надо 
перенумеровать после выполнения операции, чтобы заполнить пробелы в~нумерации, 
образовавшиеся из-за удаления Y\\
\hline
{\raisebox{-6pt}[0pt][0pt]{\tabcolsep=0pt\begin{tabular}{l}DELETE (рубрика~X удаля-\\ ется)\end{tabular}}}&A&Удаляемая рубрика X\\
\cline{2-3}
&C&Рубрики, которые надо перенумеровать после выполнения операции, чтобы 
заполнить пробелы в~нумерации, образовавшиеся из-за удаления X\\
\hline
{\raisebox{-6pt}[0pt][0pt]{\tabcolsep=0pt\begin{tabular}{l}SPLIT (def$_{\mathrm{X}}$ делится на две\\
 части~--- def$_{\mathrm{X}}1$ 
и~def$_{\mathrm{X}}2$)\end{tabular}}}&A&Рубрика~X, дефиниция которой делится на две части, причем 
def$_{\mathrm{X}}1$ становится новой дефиницией X\\
\cline{2-3}
&B&Рубрика~Y, которая создается в~базе данных, причем def$_{\mathrm{X}}2$ 
становится дефиницией~Y\\
\hline
{\raisebox{-24pt}[0pt][0pt]{\tabcolsep=0pt\begin{tabular}{l}REDISTR (def$_{\mathrm{X}}$ и~def$_{\mathrm{Y}}$ изме-\\
няются так, что происходит\\ 
перераспределение компо-\\ нентов смыс\-ло\-во\-го содер-\\ жания 
между~$\mathbf{S}_{\mathrm{def_X}}$ 
и~~$\mathbf{S}_{\mathrm{def_Y}}$)\end{tabular}}}&A&Рубрика X, если 
$\mathbf{S}_{\mathrm{def_X}}$ расширяется за счет переноса компонентов 
смыс\-ло\-во\-го содержания из~$\mathbf{S}_{\mathrm{def_Y}}$, причем ни один 
компонент~$\mathbf{S}_{\mathrm{def_X}}$ не переносится 
в~$\mathbf{S}_{\mathrm{def_Y}}$\\
\cline{2-3}
&B&Рубрика Y, если ~$\mathbf{S}_{\mathrm{def_Y}}$ сужается за счет переноса 
компонентов смыс\-ло\-во\-го содержания в~~$\mathbf{S}_{\mathrm{def_X}}$, причем 
ни один компонент~$\mathbf{S}_{\mathrm{def_X}}$ не переносится 
в~~$\mathbf{S}_{\mathrm{def_Y}}$\\
\cline{2-3}
&AB&Рубрики~X и~Y, если одновременно осуществляется перенос компонентов 
смыс\-ло\-во\-го содержания из~$\mathbf{S}_{\mathrm{def_X}}$ 
в~~$\mathbf{S}_{\mathrm{def_Y}}$ и~из~$\mathbf{S}_{\mathrm{def_Y}}$ 
в~~$\mathbf{S}_{\mathrm{def_X}}$\\
\hline
\end{tabular}
\end{center}
\end{table*}

Может показаться, что введение операции \mbox{REDISTR} не оправданно, так как 
рассмотренное перераспределение компонентов значений между~482 и~484 
можно описать последовательностью операций SPLIT\,(MERGE\,(X, Y)). 
Однако при таком подходе объем реклассификации может существенно 
возрасти. Объем реклассификации будет тот же, лишь если верно 
одновременно и~$\mathbf{S}_{\mathrm{def_X}}\hm\cap 
\mathbf{S}_{\mathrm{def_Y}}^{\mathrm{ch}}\hm\not= \emptyset$, 
и~$\mathbf{S}_{\mathrm{def_Y}} \cap 
\mathbf{S}_{\mathrm{def_X}}^{\mathrm{ch}}\not= \emptyset$, т.\,е.\ 
осуществляется перенос компонентов смыслового содержания одновременно 
и~из $\mathbf{S}_{\mathrm{def_X}}$ в~$\mathbf{S}_{\mathrm{def_Y}}$, 
и~из~$\mathbf{S}_{\mathrm{def_Y}}$ в~$\mathbf{S}_{\mathrm{def_X}}$. Если 
же верно или $\mathbf{S}_{\mathrm{def_X}} \cap 
\mathbf{S}_{\mathrm{def_Y}}^{\mathrm{ch}}\not= \emptyset$, 
а~$\mathbf{S}_{\mathrm{def_Y}} \cap 
\mathbf{S}_{\mathrm{def_X}}^{\mathrm{ch}} = \emptyset$, или 
$\mathbf{S}_{\mathrm{def_Y}} \cap 
\mathbf{S}_{\mathrm{def_X}}^{\mathrm{ch}}\not= \emptyset$, 
а~$\mathbf{S}_{\mathrm{def_X}} \cap 
\mathbf{S}_{\mathrm{def_Y}}^{\mathrm{ch}}=\emptyset$ (т.\,е.\ смысловое 
содержание дефиниции одной рубрики расширяется, а~другой~--- сужается), 
то в~таком случае реклассификация нужна только для аннотаций, 
содержавших рубрику с~дефиницией, смысловое
содержание которой 
в~результате изменения сужается (в~примере из табл.~2 это аннотации, 
содержавшие руб\-ри\-ку~Y). Таким образом, введение операции \mbox{REDISTR} 
оправданно, так как позволяет сократить объем реклассификации.

Рассмотренный пример (см.\ табл.~2) показывает, что операция REDISTR  
по-раз\-но\-му влияет на руб\-ри\-ки~482 и~484: смысловое содержание 
дефиниции руб\-ри\-ки~482 расширяется, а~руб\-ри\-ки~484~--- сужается. 
Для того чтобы на основе таб\-лиц НБД с~историей изменения руб\-рик иметь 
возможность определять, как именно операция повлияла на некоторую 
руб\-ри\-ку, каждой из руб\-рик, затронутых операцией, присваивается атрибут. 
В~этом примере руб\-ри\-ке~482 будет присвоен атрибут~A,  
а~руб\-ри\-ке~484~--- B. Все возможные атрибуты и~их смыс\-ло\-вое 
содержание для каждой из семи операций приведены и~расшифрованы 
в~табл.~3.

\vspace*{-3pt}

\section{Заключение}

Предлагаемый подход к~ведению ДКС дает возможность сохранять всю 
информацию об изменениях рубрик ФК, фиксировать виды изменений, время 
их внесения и~данные о~пользователе, который их внес. Он позволяет 
отслеживать совершенные изменения в~хронологическом порядке, а~также, 
при необходимости, восстанавливать состояние ДКС на любой момент 
времени в~прошлом.

\vspace*{-3pt}

{\small\frenchspacing
{%\baselineskip=10.8pt
%\addcontentsline{toc}{section}{References}
\begin{thebibliography}{99}

%\vspace*{-2pt}
\bibitem{1-gon}
\Au{Зацман И.\,М., Косарик~В.\,В., Курчавова~О.\,А.} Задачи представления личностных 
и~коллективных концептов в~цифровой среде~// Информатика и~её применения, 2008. 
Т.~2. Вып.~3. С.~54--69.
\bibitem{2-gon}
Handbook of linguistic annotation~/ Eds. N.~Ide, J.~Pustejovsky.~--- Dordrecht, The 
Netherlands: Springer Science\;+\;Business Media, 2017. 1468~p.
\bibitem{3-gon}
\Au{Гончаров А.\,А., Зацман~И.\,М., Кружков~М.\,Г.} Эволюция классификаций 
в~надкорпусных базах данных~// Информатика и~её применения, 2020. Т.~14. Вып.~4. 
С.~108--116.
\bibitem{4-gon}
\Au{Зацман И.\,М., Инькова~О.\,Ю., Кружков~М.\,Г., Попкова~Н.\,А.} Представление 
кроссязыковых знаний о~коннекторах в~надкорпусных базах данных~// Информатика 
и~её применения, 2016. Т.~10. Вып.~1. С.~106--118.
\bibitem{5-gon}
\Au{Зализняк А., Зацман~И.\,М., Инькова~О.\,Ю.} Надкорпусная база данных коннекторов: 
построение системы терминов~// Информатика и~её применения, 2017. Т.~11. Вып.~1. 
С.~100--108.
\bibitem{6-gon}
\Au{Зацман И.\,М., Кружков~М.\,Г.} Надкорпусная база данных коннекторов: развитие 
системы терминов проектирования~// Системы и~средства информатики, 2018. Т.~28. 
№\,4. С.~156--167.
\bibitem{7-gon}
\Au{Добровольский Д.\,О., Зализняк~Анна~А.} Немецкие конструкции с~модальными 
глаголами и~их русские соответствия: проект надкорпусной базы данных~//\linebreak 
Компьютерная лингвистика и~интеллектуальные технологии: По мат-лам Междунар. 
конф. <<Диалог>>.~--- М.: РГГУ, 2018. С.~172--184.
\bibitem{8-gon}
\Au{Добровольский Д.\,О.} Немецкие модальные глаголы в~параллельном корпусе и~задачи 
двуязычной лексикографии~// Германские языки: текст, корпус, перевод.~--- М.: Институт 
языкознания РАН, 2020. С.~103--116.
\bibitem{9-gon}
\Au{Добровольский Д.\,О., Зализняк~Анна~А.} Русские конструкции с~потенциально 
модальным значением по данным параллельных корпусов~// Труды Института русского 
языка им.\ В.\,В.~Виноградова, 2020. №\,3. С.~35--49.
\bibitem{10-gon}
\Au{Zatsman I.} Finding and filling lacunas in linguistic typologies~// 15th  Forum 
(International) on Knowledge Asset Dynamics Proceedings.~--- Matera: Institute of 
Knowledge Asset Management, 2020. P.~780--793.
\bibitem{11-gon}
\Au{Zatsman I.} Three-dimensional encoding of emerging meanings in AI-systems~// 21st 
European Conference on Knowledge Management Proceedings.~--- Reading: Academic 
Publishing International Ltd., 2020. P.~878--887.
\bibitem{12-gon}
\Au{Klein M., Fensel~D., De~A.} Ontology versioning on the Semantic Web~// 1st  Conference 
(International)  on Semantic Web Working Proceedings.~--- Stanford, CA, USA: Stanford 
University, 2001. P.~75--91.
\bibitem{13-gon}
\Au{Noy N., Kunnatur~S., Klein~M., Musen~M.} Tracking changes during ontology evolution~// 
International Semantic Web Conference~/ Eds. S.\,A.~McIlraith, D.~Plexousakis, F.~van Harmelen.~--- 
Lecture notes in computer science ser.~--- Springer, 2004. Vol.~3298. P.~259--273.
\bibitem{14-gon}
\Au{Preventis A., Petrakis~E.\,G.\,M., Batsakis~S.} CHRONOS Ed: A~tool for handling 
temporal ontologies in prot$\acute{\mbox{e}}$g$\acute{\mbox{e}}$~// Int. J.~Artif. 
Intell.~T., 2014. Vol.~23. No.\,4. P.~1460018-1--1460018-26. doi: 
10.1142/S0218213014600185.
\bibitem{15-gon}
\Au{Stravoskoufos K., Petrakis~E., Mainas~N., Batsakis~S., Samoladas~V.} SOWL QL: 
Querying spatio-temporal ontologies in OWL~// J.~Data Semantics, 2016. Vol.~5. No.\,4. 
P.~249--269.
\bibitem{16-gon}
\Au{Zablith F., Antoniou~G., D'Aquin~M., Flouris~G., Kondylakis~H., Motta~E., 
Plexousakis~D., Sabou~M.} Ontology evolution: A~process-centric survey~// Knowl. 
Eng. Rev., 2015. Vol.~30. No.\,1. P.~45--75.
\bibitem{17-gon}
Немецко-русский словарь актуальной лексики~/ Под ред. Д.\,О.~Добровольского.~--- М.: 
Лексрус, 2021 (в~печати).
\end{thebibliography}

}
}

\end{multicols}

\vspace*{-6pt}

\hfill{\small\textit{Поступила в~редакцию 12.01.2021}}

%\vspace*{8pt}

%\pagebreak

\newpage

\vspace*{-28pt}

%\hrule

%\vspace*{2pt}

%\hrule

%\vspace*{-2pt}

\def\tit{REPRESENTATION OF~NEW LEXICOGRAPHICAL KNOWLEDGE IN~DYNAMIC CLASSIFICATION 
SYSTEMS}

\def\titkol{Representation of new lexicographical knowledge in~dynamic classification 
systems}

\def\aut{A.\,A.~Goncharov, I.\,M.~Zatsman, and~M.\,G.~Kruzhkov}

\def\autkol{A.\,A.~Goncharov, I.\,M.~Zatsman, and~M.\,G.~Kruzhkov}

\titel{\tit}{\aut}{\autkol}{\titkol}

\vspace*{-11pt}


\noindent
Institute of Informatics Problems, Federal Research Center ``Computer Science and
Control'' of the Russian Academy of Sciences, 44-2~Vavilov Str., Moscow 119333,
Russian Federation

\def\leftfootline{\small{\textbf{\thepage}
\hfill INFORMATIKA I EE PRIMENENIYA~--- INFORMATICS AND
APPLICATIONS\ \ \ 2021\ \ \ volume~15\ \ \ issue\ 1}
}%
\def\rightfootline{\small{INFORMATIKA I EE PRIMENENIYA~---
INFORMATICS AND APPLICATIONS\ \ \ 2021\ \ \ volume~15\ \ \ issue\ 1
\hfill \textbf{\thepage}}}

\vspace*{3pt}


\Abste{The distinctive feature of dynamic classification systems is that new categories may be introduced 
in the course of their use or definitions of existing categories may be modified, including cases of 
rearranging semantic content between categories. On one hand, this feature of dynamic classification 
systems provides a possibility to integrate new knowledge on-the-fly and to start using it immediately for 
linguistic annotation. On the other hand, if a category is changed, then, in some cases, the annotations it 
has been previously applied to will have to be reclassified. This paper has a twofold purpose, which is, 
first, to compare approaches to classification of entities based on ($i$)~dynamic classification systems and 
($ii$)~ontologies that change over time; and then, second, to describe how new lexicographical knowledge is 
represented in dynamic classification systems.}

\KWE{dynamic classification system; ontology versioning; linguistic annotation; reclassification of 
annotations}




\DOI{10.14357/19922264210112}

\vspace*{-15pt}

\Ack
\noindent
The study has been conducted at the Institute of Informatics Problems, Federal Research Center 
``Computer Science and Control'' of the Russian Academy of Sciences (FRC CSC RAS) with financial 
support of the Russian Foundation for Basic Research (grant No.\,20-012-00166).
%\vspace*{6pt}

  \begin{multicols}{2}

\renewcommand{\bibname}{\protect\rmfamily References}
%\renewcommand{\bibname}{\large\protect\rm References}

{\small\frenchspacing
 {%\baselineskip=10.8pt
 \addcontentsline{toc}{section}{References}
 \begin{thebibliography}{99}
\bibitem{1-gon-1}
\Aue{Zatsman, I.\,M., V.\,V.~Kosarik, and O.\,A.~Kurchavova.} 2008. Zadachi predstavleniya 
lichnostnykh i~kollektivnykh kontseptov v~tsifrovoy srede [Representation of individual and collective 
concepts in digital medium]. \textit{Informatika i~ee Primeneniya~--- Inform. Appl.} 2(3):54--69.
\bibitem{2-gon-1}
Ide, N., and J.~Pustejovsky, eds. 2017. \textit{Handbook of linguistic annotation}. Dordrecht, The 
Netherlands: Springer Science\;+\;Business Media. 1468~p.
\bibitem{3-gon-1}
\Aue{Goncharov, A.\,A., I.\,M.~Zatsman, and M.\,G.~Kruzhkov.} 2020. Evolyutsiya klassifikatsiy 
v~nadkorpusnykh ba\-zakh dannykh [Evolution of classifications in supracorpora databases]. 
\textit{Informatika i~ee Primeneniya~--- Inform. Appl.} 14(4):108--116.
\bibitem{4-gon-1}
\Aue{Zatsman, I.\,M., O.\,Yu.~Inkova, M.\,G.~Kruzhkov, and N.\,A.~Popkova.}
 2016. Predstavlenie kross-yazykovykh znaniy o~konnektorakh v~nadkorpusnykh 
 bazakh dannykh [Representation of cross-lingual 
knowledge about connectors in suprocorpora databases]. 
\textit{Informatika i~ee Primeneniya~--- Inform. Appl.} 10(1):106--118.
\bibitem{5-gon-1}
\Aue{Zaliznyak, A.\,A., I.\,M.~Zatsman, and O.\,Yu.~In'kova.} 2017. Nadkorpusnaya basa dannykh 
konnektorov: postroenie sistemy terminov [Supracorpora database of connectives: Term system 
development]. \textit{Informatika i~ee Primeneniya~--- Inform. Appl.} 11(1):100--108.
\bibitem{6-gon-1}
\Aue{Zatsman, I.\,M., and M.\,G.~Kruzhkov.} 2018. Nadkorpusnaya baza dannykh konnektorov: razvitie 
sistemy terminov proektirovaniya [Supracorpora database of connectives: Design-oriented evolution of 
the term system]. \textit{Sistemy i~Sredstva Informatiki~--- Systems and Means of Informatics} 
28(4):156--167.
\bibitem{7-gon-1}
\Aue{Dobrovol'skiy, D.\,O., and Anna A.~Zaliznyak.} 2018. Ne\-mets\-kie konstruktsii s~modal'nymi 
glagolami i~ikh russkie sootvetstviya: proekt nadkorpusnoy bazy dannykh [German constructions with 
modal verbs and their Russian correlates: A~supracorpora database project]. \textit{Komp'yuternaya 
lingvistika i~intellektual'nyye tekhnologii: po mat-lam Mezhdunar. konf. ``Dialog'}' [Computational 
Linguistics and Intellectual Technologies. Papers from the Annual Conference (International) 
``Dialogue'']. Moscow: RSHI. 17(24):172--184. 
\bibitem{8-gon-1}
\Aue{Dobrovol'skiy, D.\,O.} 2020. Nemetskie modal'nye glagoly v~parallel'nom korpuse i~zadachi 
dvuyazychnoy leksikografii [German modal verbs in a parallel corpus and bilingual lexicography tasks]. 
\textit{Germanskie yazyki: tekst, korpus, perevod} [German languages: Text, corpus, translation].  Moscow:
Institute of Linguistics RAS. 103--116.
\bibitem{9-gon-1}
\Aue{Dobrovol'skiy, D.\,O., and Anna A.~Zaliznyak.} 2020. Russkie konstruktsii s~potentsial'no 
modal'nym znacheniem po dannym parallel'nykh korpusov [Russian constructions with potentially modal 
meanings: An analysis based on parallel corpus data]. \textit{Trudy Instituta russkogo yazyka im.\ 
V.\,V.~Vinogradova} [V.\,V.~Vinogradov Russian Language Institute Proceedings]. 35--49.
\bibitem{10-gon-1}
\Aue{Zatsman, I.} 2020. Finding and filling lacunas in linguistic typologies. \textit{15th Forum 
(International) on Knowledge Asset Dynamics Proceedings}. Matera: Institute of Knowledge Asset 
Management. 780--793.
\bibitem{11-gon-1}
\Aue{Zatsman, I.} 2020. Three-dimensional encoding of emerging meanings in AI-systems. \textit{21st 
European Conference on Knowledge Management Proceedings}. Reading: Academic Publishing 
International Ltd. 878--887.
\bibitem{12-gon-1}
\Aue{Klein, M., D.~Fensel, and A.~De.} 2001. Ontology versioning on the semantic web. \textit{1st 
Conference (International) on Semantic Web Working Proceedings}. Stanford, CA: Stanford University. 
75--91.
\bibitem{13-gon-1}
\Aue{Noy, N., S.~Kunnatur, M.~Klein, and M.~Musen.} 2004. Tracking changes during ontology 
evolution. \textit{International Semantic Web Conference}. Eds. S.\,A.~McIlraith, D.~Plexousakis, and 
F.~van Harmelen. Lecture notes in computer science ser. Springer. 3298:259--273.
\bibitem{14-gon-1}
\Aue{Preventis, A., E.\,G.\,M.~Petrakis, and S.~Batsakis.} 2014. CHRONOS Ed: A~tool for handling 
temporal ontologies in prot$\acute{\mbox{e}}$g$\acute{\mbox{e}}$. \textit{Int. J.~Artif. 
Intell.~T.} 23(4):1460018. 26~p.
doi: 
10.1142/S0218213014600185.
\bibitem{15-gon-1}
\Aue{Stravoskoufos, K., E.~Petrakis, N.~Mainas, S.~Batsakis, and V.~Samoladas.} 2016. SOWL QL: 
Querying spatio-temporal ontologies in OWL. \textit{J.~Data Semantics} 5(4):249--269.
\bibitem{16-gon-1}
\Aue{Zablith F., G.~Antoniou, M.~D'Aquin, G.~Flouris, H.~Kondylakis, E.~Motta, D.~Plexousakis, and 
M.~Sabou.} 2015. Ontology evolution: A~process-centric survey. \textit{Knowl. Eng. 
Rev.} 30(1):45--75.
\bibitem{17-gon-1}
Dobrovol'skiy, D.O., ed. 2021 (in press). \textit{Nemetsko-russkiy slovar' aktual'noy leksiki} 
[German--Russian dictionary of actual vocabulary]. Moscow: Leksrus.
\end{thebibliography}

 }
 }

\end{multicols}

\vspace*{-3pt}

  \hfill{\small\textit{Received January~12, 2021}}


%\pagebreak

%\vspace*{-8pt}

\Contr

\noindent
\textbf{Goncharov Alexander A.} (b.\ 1994)~--- junior scientist, Institute of Informatics Problems, 
Federal Research Center ``Computer Science and Control'' of the Russian Academy of Sciences,  
44-2~Vavilov Str., Moscow 119333, Russian Federation; \mbox{a.gonch48@gmail.com}

\vspace*{3pt}

\noindent
\textbf{Zatsman Igor M.} (b.\ 1952)~--- Doctor of Science in technology, Head of Department, Institute 
of Informatics Problems, Federal Research Center ``Computer Science and Control'' of the Russian 
Academy of Sciences, 44-2~Vavilov Str., Moscow 119333, Russian Federation; 
\mbox{izatsman@yandex.ru}

\vspace*{3pt}

\noindent
\textbf{Kruzhkov Mikhail G.} (b.\ 1975)~--- senior scientist, Institute of Informatics Problems, Federal 
Research Center ``Computer Science and Control'' of the Russian Academy of Sciences, 44-2~Vavilov 
Str., Moscow 119333, Russian Federation; \mbox{magnit75@yandex.ru}

\label{end\stat}

\renewcommand{\bibname}{\protect\rm Литература}