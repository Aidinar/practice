\def\xytt{(X_t, Y_t, t,\Theta)}
\def\xtt{(X_t, t,\Theta)}
\def\hx{\hat X_t}
\def\ssz{\mathop {\sum\limits^{p^\Theta}\sum\limits}}
\def\pss{\left(b_1\nu b_1^{\mathrm{T}}\right)}
\def\ps{b\nu b_1^{\mathrm{T}}}

\def\stat{sinitsin}

\def\tit{НОРМАЛЬНЫЕ СУБОПТИМАЛЬНЫЕ ФИЛЬТРЫ ДЛЯ~ДИФФЕРЕНЦИАЛЬНЫХ СТОХАСТИЧЕСКИХ СИСТЕМ,
НЕ~РАЗРЕШЕННЫХ ОТНОСИТЕЛЬНО ПРОИЗВОДНЫХ}

\def\titkol{Нормальные субоптимальные фильтры для~дифференциальных СтС, %стохастических систем,
не разрешенных относительно производных}

\def\aut{И.\,Н.~Синицын$^1$}

\def\autkol{И.\,Н.~Синицын}

\titel{\tit}{\aut}{\autkol}{\titkol}

\index{Синицын И.\,Н.}
\index{Sinitsyn I.\,N.}

%{\renewcommand{\thefootnote}{\fnsymbol{footnote}} \footnotetext[1]
%{Работа выполнена при финансовой поддержке РАН (проект АААА-А19-119091990037-5).}}

\renewcommand{\thefootnote}{\arabic{footnote}}
\footnotetext[1]{Институт проблем информатики Федерального
исследовательского центра <<Информатика и~управ\-ле\-ние>> Российской
академии наук, \mbox{sinitsin@dol.ru}}


\vspace*{-4pt}


\Abst{Рассмотрены вопросы синтеза среднеквадратичных (с.к.)\ 
нелинейных нормальных (гауссовских) субоптимальных  фильтров (НСОФ) для дифференциальных стохастических систем
(СтС), 
не разрешенных относительно производных. Представлены уравнения состояния и~наблюдения 
нелинейных дифференциальных СтС, не разрешенных относительно производных. Один из разделов 
по\-свя\-щен синтезу НСОФ при следующих условиях: (1)~отсутствуют пуассоновские шумы в~наблюдениях; 
(2)~коэффициент при гауссовском шуме не зависит от состояния. 
Рассмотрен синтез НСОФ при аддитивных 
шумах в~уравнениях со\-сто\-яния и~на\-блюде\-ния. Для иллюстрации методов синтеза 
НСОФ приведен пример. Рассмотрены вопросы качества НСОФ.}


\KW{метод аналитического моделирования (МАМ);
метод нормальной аппроксимации (МНА);
метод статистической линеаризации (МСЛ);
нормальный субоптимальный фильтр (НСОФ);
стохастическая система (СтС);
стохастические системы, не разрешенные относительно производных;
формирующий фильтр (ФФ)}

\DOI{10.14357/19922264210101}

%\vspace*{-2pt}

\vskip 10pt plus 9pt minus 6pt

\thispagestyle{headings}

\begin{multicols}{2}

\label{st\stat}



\section{Введение}

\vspace*{-3pt}

Методы  аналитического моделирования (МАМ) широкополосных стохастических процессов 
(СтП) в~нелинейных дифференциальных СтС, разрешенных относительно производных, представлены в~[1--6].

В~[7] представлено развитие методов па\-ра\-мет\-ри\-че\-ско\-го аналитического моделирования~\cite{1-s, 2-s, 5-s} 
применительно к~СтС, не разрешенным относительно производных.
Рассмотрены уравнения нелинейных дифференциальных СтС, не разрешенных 
относительных производных с~винеровскими и~пуассоновскими белыми, а~также автокоррелированными шумами.
 Дано обобщение МАМ  на основе метода нормальной аппроксимации (МНА)
  (метода статистической линеаризации (МСЛ)) на случай автокоррелированных шумов. 
 Рассмотрены общие вопросы МАМ на основе параметризации одно- и~многомерных плотностей СтП нелинейных 
 дифференциальных  СтС, не разрешенных относительно производных, приводимых к~СтС, 
 разрешенным относительно производных. Особое внимание уделено МАМ на основе метода 
 ортогональных разложений (МОР) и~метода квазимоментов (МКМ) для одно- и~многомерных плотностей в~случае 
 эталонных гауссовских и~пуассоновских плотностей.

В статье рассмотрены вопросы синтеза с.к.\ нелинейных НСОФ 
для дифференциальных СтС, не разрешенных относительно 
производных. 

В~разд.~2 представлены уравнения состояния и~наблюдения нелинейных дифференциальных 
СтС, не разрешенных относительно производных. 

Раздел~3 посвящен синтезу НСОФ при сле\-ду\-ющих условиях: 
(1)~отсутствуют пуассоновские шумы в~наблюдениях; (2)~коэффициент при гауссовском шуме не зависит от состояния. 

Синтез НСОФ при аддитивных шумах в~уравнениях состояния и~наблюдения рассмотрен в~разд.~4. 

Для иллюстрации методов синтеза НСОФ в~разд.~5 приведен пример. Вопросы качества НСОФ рассмотрены в~разд.~6. 

Заключение и~возможные обобщения содержатся в~разд.~7.

\vspace*{-6pt}

\section{Уравнения  дифференциальных стохастических систем, 
не~разрешенных относительно производных и~наблюдений}

\vspace*{-3pt}


Следуя~\cite{8-s}, рассмотрим сначала векторную дифференциальную СтС с~нелинейностями, 
описываемыми гладкими функциями:

\pagebreak

\noindent
\begin{multline}
\vrp =\vrp\left(t, \Theta, X_t, \dot X_t \tr X_t^{(k)}, U_t\right) =0\,, \\ 
X\left(t_0\right) = X_0,\\
  \dot X\left(t_0\right) = \dot X_0 \tr X^{(k)} \left(t_0\right) = X_0^{(k)}\,,
\label{e2.1-s}
\end{multline}
    при этом уравнение нелинейного формирующего фильтра (ФФ) возьмем в~виде, 
    разрешенном относительно возмущений~$U_t$:
\begin{multline}
\dot U_t = a^U \left(t, \Theta,U_t\right)  + b^U \left(t, \Theta,U_t\right) V_t^U, \\
 U\left(t_0\right) = U_0.
\label{e2.2-s}
\end{multline}
Здесь $a^U = a^U(t, \Theta,U_t)$ и~$b^U(t, \Theta,U_t)$~--- $(n^X\times 1)$- и~$(n^X\times n^V)$-мер\-ные 
функции; $V_t^U$~--- белый шум в~строгом смысле~\cite{1-s, 2-s}, 
допускающий представление в~виде гауссовской и~пуассоновской составляющих
\begin{multline*}
    V_t^U = \dot W_t^U, \\ 
    W_t^U = W_0^U (t,\Theta) + \int\limits_{R_0^q} c^U (\Theta,\rho)
     P^0 ( t,\Theta, d\rho),
     %\label{e2.3-s}
     \end{multline*}
где $\nu_t$~--- его интенсивность:
   \begin{equation*}
    \nu_t=\nu_t^W \!=\nu_t^{W_0} +\!  \int\limits_{R_0^q}\!\! c^U (\Theta,\rho) 
    \left[c^U (\Theta,\rho)\right]^{\mathrm{T}}\! \nu_P (t,\Theta,\rho)\, d\rho\,;
 %   \label{e2.4-s}
    \end{equation*}
 $c^U \hm= c^U(\Theta,\rho)$~--- 
известная векторная функция той же размерности, что и~$W_t^0$, а интеграл при любом $t\hm\ge t_0$ 
представляет собой стохастический интеграл по центрированной пуассоновской мере $P^0(t,\Theta,\mathcal{A})$,
 независимой от~$W_0^U$ и~имеющей независимые значения на попарно непересекающихся множествах; 
 $\mathcal{A}$~--- борелевское множество пространства~$R_0^q$ с~выколотым началом; 
 $\nu_t^W$, $\nu_t^{W_0}$ и~$\nu_P$~--- интенсивности СтП $W_t^U$, $W_{0}^U$ и~$P^0$. Уравнение~(\ref{e2.2-s}) 
 понимается в~смысле Ито и~имеет единственное решение в~с.к.~\cite{1-s, 2-s}.

Для гладких функций в~(\ref{e2.1-s}), допускающих стохастические производные Ито до $h$-го 
порядка и~статистическую линеаризацию по Казакову~\cite{1-s, 2-s, 3-s, 4-s},\linebreak 
выполним следующие преобразования.
Будем дифференцировать сполна по~$t$ левые части уравнений~(\ref{e2.1-s}) 
по обобщенной формуле Ито~\cite{1-s, 2-s} до тех пор, пока не появятся производные белого шума. 
В~результате получим сле\-ду\-ющую сис\-те\-му нелинейных дифференциальных уравнений:
\begin{equation}
\vrp=0\,,\enskip \dot \vrp =0\tr \vrp^{(h)}=0\,.
\label{e2.5-s}
\end{equation}
%
Затем заменим эту систему уравнений статистически линеаризованной:

\noindent
\begin{multline*}
\vrp^{(i)} \left(t, \Theta, X_t^{(i)}, U_t\right)={}\\
{}=
\vrp_{0X'}^{(i)} \left(t, \Theta, m_t^{X^i}, K_t^{X^i}, m_t^{U},  K_t^{U}\right)+{}
\end{multline*}

\noindent
\begin{multline}
{}+k_{1X'}^{(i)} \left(t, \Theta, m_t^{X^i}, K_t^{X^i}, m_t^{U},  K_t^{U}\right) X_t^{i0} + {}\\
{}+
k_{1U}^{(i)} \left(t, \Theta, m_t^{X^i}, K_t^{X^i}, m_t^{U},  K_t^{U}\right) U_t^0=0\\
 (i=1,2,\ldots,h).
 \label{e2.6-s}
 \end{multline}
Здесь $X_t' =\lk X_t^{\mathrm{T}} \dot{X}_t^{\mathrm{T}}\ldots X_t^{(k-1)T}\rk^{\mathrm{T}}$; 
$m_t^{X'}$  и~$K_t^{X'}$~--- вектор математического ожидания и~ковариационная матрица; 
$k_{1X'}^{(i)} \hm=k_{1X'}^{(i)} (t, \Theta, m_t^{X^i}, K_t^{X^i}, m_t^{U},  
K_t^{U})$ и~$k_{1U}^{(i)} \hm =k_{1U}^{(i)} (t, \Theta, m_t^{X^i}, K_t^{X^i}, m_t^{U},  K_t^{U})$~--- 
мат\-рич\-ные коэффициенты статистической линеаризации функций~(\ref{e2.6-s}). 

Далее введем вектор
$\bar X_t \hm= \lk X_t^{'\mathrm{T}} X_t^{''\mathrm{T}}\rk^{\mathrm{T}}$, со\-став\-лен\-ный из 
$X_t' \hm= \lk X_t^{\mathrm{T}} \dot{X}_t^{\mathrm{T}}\ldots X_t^{(k-1)\mathrm{T}}\rk^{\mathrm{T}}$ 
и~вспомогательного вектора 
$X_t''\hm=[X_t^{(i)}]_{i=\overline{1,h}}$ на основе~(\ref{e2.6-s}).
В~результате придем к~уравнениям, разрешенным относительно дифференциалов, следующего вида:
\begin{equation}
d\bar X_t = a^{\bar X}  dt + b^{\bar X} d W_0 + \int\limits_{R_0^q} c^{\bar X} P^0(t,\Theta,du),
\label{e2.7-s}
\end{equation}
где $a^{\bar X} = a^{\bar X}(t,\Theta,\bar X_t)$; $b^{\bar X}\hm=b^{\bar X}(t,\Theta,\bar X_t)$; 
$ c^{\bar X}\hm=c^{\bar X}(t,\Theta,\bar X_t, u)$.

Таким образом, имеем следующее утверждение~\cite{8-s}.

\smallskip

\noindent
\textbf{Теорема~1.} \textit{Пусть нелинейная негауссовская СтС}~(\ref{e2.1-s}), (\ref{e2.2-s}), 
\textit{не разрешенная относительно производных $k$-го порядка, удовлетворяет условиям}:
\begin{itemize}
\item[1$^\circ$]
\textit{функции}~(\ref{e2.1-s}) \textit{допускают обобщенные стохастические дифференциалы Ито 
вплоть до $h$-го порядка включительно и~статистическую линеаризацию по Казакову};
\item[2$^\circ$]
\textit{возмущения $U_t$ негауссовские, причем уравнение ФФ}~(\ref{e2.2-s}) 
\textit{разрешено относительно возмущений $U_t$ и~имеет единственное с.к.\ решение};
\item[3$^\circ$]
\textit{СтП $X_t'$ более гладкий, чем возмущение~$U_t$}.
\end{itemize}
\textit{Тогда система}~(\ref{e2.1-s}), (\ref{e2.2-s}) 
\textit{приводима к~дифференциальной системе, разрешенной относительно производных}~(\ref{e2.7-s}).

\smallskip

Далее, следуя~\cite{10-s}, предположим, что СтС удовле\-тво\-ря\-ет условиям теоремы~1, 
а~также пол\-ностью наблюдаема. При этом уравнения наблюдения описыва\-ют\-ся уравнениями~(\ref{e2.7-s}) 
при сле\-ду\-ющих условиях: во-пер\-вых, уравнения наблюдения не содержат пуассоновского шума 
 $(c_1\hm\equiv 0)$ и,~во-вто\-рых, коэффициент при винеровском шуме~$b_1^Y$ не зависит от состояния  
 ${\bar X}_t (b_1^Y ({\bar X}^t, Y_t))\hm = b_1^Y (Y_t, t)$. Размерность вектора~$Y_t$ примем равной~$n^Y$, 
 а~${\bar X}_t \hm= n^{\bar X}$.
В~дальнейшем для краткости будем принимать ${\bar X}_t \hm= X_t$, а~уравнения 
<<система плюс наблюдения>> записывать в~сле\-ду\-ющем виде:
\begin{align}
dX_t &= a \left(X_t, Y_t, t, \Theta\right) dt + b\left(X_t, Y_t, t, \Theta\right) dW_0 + {}\notag\\
&\hspace*{10mm}{}+ \int\limits_{R_0^q} c \left(X_t, Y_t, t, \Theta,v\right)P^0 (dt, dv, \Theta)\,;\label{e2.8-s}
\\
    dY_t &= a_1 \left(X_t, Y_t, t, \Theta\right)dt + b_1 \left( Y_t, t, \Theta\right)dW_0.\label{e2.9-s}
    \end{align}
Для  СтС (\ref{e2.8-s}), (\ref{e2.9-s}) с~аддитивными шумами имеют место следующие условия:
\begin{equation}
\left.
\begin{array}{rl}
b\left(X_t, Y_t, t, \Theta\right) &= b_0 ( t, \Theta)\,; \\[6pt] 
c\left(X_t, Y_t, t, \Theta,v\right)&=0\,,\\[6pt] 
b_1 \left( Y_t, t, \Theta\right)&=b_{10}(t,\Theta)\,.
\end{array}
\right\}
\label{e2.10-s}
\end{equation}
Условия~(\ref{e2.10-s}) будут выполняться, если в~последнем уравнении~(\ref{e2.5-s}) с~номером~$h$ 
появится белый шум, а коэффициент при нем будет зависеть только от~$t$ и~$\Theta$.


\section{Нормальный субоптимальный фильтр для стохастических систем, 
не~разрешенных относительно производных}


Как известно из нелинейной теории фильтрации~\cite{10-s}, для гауссовской СтС, поскольку
гауссовское (нормальное) распределение, аппроксимирующее
апостериорное распределение вектора~$X_t$,\linebreak полностью определяется
апостериорными математическим ожиданием~$\hat X_t$ и~ковариационной матрицей~$R_t$ вектора~$X_t$, 
при аппроксимации
апостериорного распределения вектора~$X_t$ нормальным\linebreak
распределением все математические ожидания для~$d\hat X_t$ и~$dR_t$ будут определенными
функциями~$\hat X_t$, $R_t$ и~$t$, т.\,е.\  будут
представлять собой стохастические дифференциальные уравнения,
определяющие~$\hat X_t$ и~$R_t$:
\begin{multline}
\hspace*{-1.53119pt}d\hat X_t =B\!\left(\hat X_t, Y_t, R_t, t, \Theta\right)=
 f \!\left(\hat X_t, Y_t,R_t,t, \Theta\right)dt +{}\\
{}+
    h\left(\hat X_t,Y_t, R_t,t,, \Theta\right)dt\times{}\\
    {}\times \left[ dY_t - f^{(1)} \left(\hat X_t,Y_t, R_t,t\right)dt
    \right];
    \label{e3.1-s}
    \end{multline}
    
    \vspace*{-12pt}
    
    \noindent
    \begin{multline*}
       dR_t=\Biggl\{ f^{(2)}\left(\hat X_t, Y_t,R_t,t, \Theta\right)-{}\\
       {}-
       h\left(\hat    X_t, Y_t,R_t,t, \Theta\right)
       \pss \left(Y_t,t, \Theta\right) \times{}
       \end{multline*}
       
      \noindent
       \begin{multline}
        \hspace*{-8pt}{}\times h \left({\hat X}_t, Y_t,R_t,t, \Theta\right)^{\mathrm{T}}\Biggr\} dt+
    \sum\limits_{r=1}^{n_y} \rho_r \left({\hat X}_t,Y_t, R_t,t, \Theta\right)\times{}\\
{}\times \left[
    dY_r -f_r^{(1)}\left({\hat X}_t,Y_t, R_t,t, \Theta\right) dt\right],\label{e3.2-s}
    \end{multline}
где $\nu =\nu(t)$~--- интенсивность белого шума:
\begin{multline}
f\left(\hat X_t, Y_t,R_t,t, \Theta\right)={}\\
{}=\left[\left(2\pi\right)^n \lv
    R_t\rv\right]^{-1/2}\int\limits_{-\infty}^\infty a\left(Y_t,x,t, \Theta\right)\times{}\\
    {}\times \exp \left\{ -
    \left(x^{\mathrm{T}}
    -\hat X_t^{\mathrm{T}}\right) R_t^{-1} \fr{x -\hat X_t}{2}\right\}
    dx\,;\label{e3.3-s}
    \end{multline}
    
   \vspace*{-12pt}
    
    \noindent
    \begin{multline}
    f^{(1)}\!\left(\hat X_t, Y_t,R_t,t, \Theta\right)=\left\{ f_r^{(1)} \left( \hat X_t, Y_t, R_t, t, 
    \Theta\right)\!\right\}={}\\
{}=\left[ (2\pi)^{n_x}\lv
    R_t\rv\right]^{-1/2}\int\limits_{-\infty}^\infty a_1\left(Y_t,x,t\right) \times{}\\
    {}\times \exp \left\{
    -\left(x^{\mathrm{T}} -\hat X_t^{\mathrm{T}}\right) R_t^{-1} \fr{x -\hat X_t}{2}\right\}
    dx\,;\label{e3.4-s}
    \end{multline}
    
    \vspace*{-12pt}
    
    \noindent
    \begin{multline}
h\left(\hat X_t, Y_t,R_t,t, \Theta\right)=
\Biggl\{ \lk (2\pi)^{n_x}\lv
    R_t\rv\rk^{-1/2}\times{}\\
    {}\times\int\limits_{-\infty}^\infty\!\!
    \lk x a_1(Y_t,x,t)^{\mathrm{T}} + \ps \left(Y_t,x,t, \Theta\right)\rk\times{}\\
{}\times \exp \left\{ -\left(x^{\mathrm{T}} -\hat X_t^{\mathrm{T}}\right) R_t^{-1} \fr{x -\hat X_t}{2}\right\} dx-{}\\
{}-
    \hat X_t f^{(1)}\left(\hat X_t, Y_t,R_t,t, \Theta\right)^{\mathrm{T}}\Biggr\}\times{}\\
{}\times \pss^{-1} (Y_t,t);
\label{e3.5-s}
\end{multline}

\vspace*{-12pt}
    
    \noindent
    \begin{multline}
f^{(2)}\left(\hat X_t, Y_t,R_t,t, \Theta\right)={}\\
{}=\left[ (2\pi)^{n_x}\lv
    R_t\rv\right]^{-1/2}\int\limits_{-\infty}^\infty
    \biggl\{  \left(x-\hat X_t\right) a\left(Y_t,x,t, \Theta\right)^{\mathrm{T}} + {}\\
{}+ a \left(Y_t,x,t, \Theta\right) \left(x^{\mathrm{T}}-\hat X_t^{\mathrm{T}}\right) +\ps 
\left(Y_t,x,t, \Theta\right)\biggr\}\times{}\\
{}\times \exp \left\{ -\left(x^{\mathrm{T}} -\hat X_t^{\mathrm{T}}\right) R_t^{-1} \fr{x -\hat X_t}{2}\right\}
    dx\,;\label{e3.6-s}
    \end{multline}
    
 \vspace*{-12pt}
    
    \noindent
    \begin{multline*}
    \rho_r\left(\hat X_t,Y_t, R_t,t, \Theta\right)=\left[ (2\pi)^{n_x}\lv
    R_t\rv\right]^{-1/2}\times{}\\
    {}\times \int\limits_{-\infty}^\infty
    \biggl\{  \left(x-\hat X_t\right)\left(x^{\mathrm{T}}-\hat X_t^{\mathrm{T}}\right) \alpha_r \left(Y_t,x,t, \Theta\right)+ {}
\end{multline*}

\noindent
    \begin{multline}
{}+ \left(x-\hat X_t\right) \beta_r\left(Y_t,x,t, \Theta\right)^{\mathrm{T}} \left(x^{\mathrm{T}}-\hat X_t^{\mathrm{T}}\right)+ {}\\
{}+
\beta_r \left(Y_t,x,t, \Theta\right) \left(x^{\mathrm{T}}-\hat     X_t^{\mathrm{T}}\right)\biggr\} \times{}\\
{}\times \exp \left\{ -(x^{\mathrm{T}} -\hat X_t^{\mathrm{T}}) R_t^{-1} \fr{x -\hat
    X_t}{2}\right\} dx\\
     (r=1\tr n_y).\label{e3.7-s}
    \end{multline}
Здесь $\alpha_r$~---  $r$-й элемент мат\-ри\-цы-стро\-ки  
$(a_1\hm-\hat a_1^{\mathrm{T}})(b_1\nu b_1^{\mathrm{T}})^{-1}$; $\beta_{kr}$~--- элемент  $k$-й строки 
и~$r$-го столбца матрицы $b \nu b_1^{\mathrm{T}} (b_1 \nu b_1^{\mathrm{T}})^{-1}$, при этом $\beta_r \hm= 
\lk \beta_{1r}\cdots \beta_{pr}\rk^{\mathrm{T}}$ $(r=1,\ldots, n_X)$.


Таким образом, применительно к~(\ref{e2.8-s}), (\ref{e2.9-s}) имеет место сле\-ду\-ющее утверж\-де\-ние.

\smallskip

\noindent
\textbf{Теорема~2.}\ \textit{Пусть уравнения нелинейной гауссовской
 дифференциальной СтС}~(\ref{e2.8-s}) \textit{и}~(\ref{e2.9-s})  \textit{допускают
применение}  МНА. \textit{Тогда в~основе} {МНА} \textit{одномерного
апостериорного распределения лежат уравнения}~(\ref{e3.1-s}) \textit{и}~(\ref{e3.2-s})
\textit{при условиях}~(\ref{e3.3-s})--(\ref{e3.7-s}) \textit{и начальных условиях}.


\smallskip

Количество уравнений МНА одномерного апостериорного распределения
определяется по формуле:
\begin{equation*}
    Q_{\mathrm{МНА}} = n_x + \fr{n_x (n_x+1)}{2} = \fr{n_x(n_x+3)}{2}\,.
    %\label{e3.8-s}
    \end{equation*}

За начальные значения $\hat X_t$ и~$R_t$  при интегрировании
уравнений~(\ref{e3.1-s}) и~(\ref{e3.2-s}), естественно, следует принять
условные математическое ожидание и~ковариационную матрицу величины~$X_0$ относительно~$Y_0$:
\begin{align*}
\hat X_0 &= \mathsf{M}\left[ X_0 \mid Y_0\right];\\
R_0 &= \mathsf{M} \left[ \left(X_0 -\hat X_0\right) 
\left(X_0^{\mathrm{T}} -\hat X_0^{\mathrm{T}}\right)\mid Y_0\right].
%\label{e3.9-s}
\end{align*}
 Если нет
информации об условном распределении~$X_0$ относительно~$Y_0$, то
начальные условия можно взять в~виде:
$$
\hat X_0 = \mathsf{M}\,X_0;\enskip
R_0= \mathsf{M}\left(X_0-\mathsf{M}\,X_0\right) \left(X_0^{\mathrm{T}}- \mathsf{M}\,X_0^{\mathrm{T}}\right). 
$$
Если
же и~об этих величинах нет никакой информации, то начальные
значения~$\hat X_t$ и~$R_t$ приходится задавать произвольно.

Теперь рассмотрим НСОФ для негауссовских СтС, не разрешенных относительно производных. 
В~этом случае, как показано в~\cite{10-s}, требуется ограниченность функций~$f$, $f^{(1)}$, $h$, $\rho_r$ 
и~функции
\begin{equation*}
\bar f^{(2)} = f^{(20)} + \mathsf{M}^N \left[ \int\limits_{R_0^q} cc^{\mathrm{T}} \nu_P (t,\Theta) \,dv\right]
% \label{e3.10-s}
 \end{equation*}
(\textbf{теорема~3}).



\section{Нормальный субоптимальный фильтр для~стохастических систем, 
не~разрешенных относительно производных  и~с~аддитивными шумами}


Пусть  функции $a,b, a_1$ и~$b_1$ в~(\ref{e2.8-s}), (\ref{e2.9-s}) удовлетворяют условиям:

\vspace*{4pt}

\noindent
\begin{equation}
\left.
\begin{array}{rl}
a = a\xytt&=a \xtt;\\[6pt]
 a_1 =a_1\xytt &=a_1 \xtt;
\\[6pt]
b\xytt \,dW &=b(t,\Theta) \,dW_1;\\[6pt]
 b_1 \xtt \,dW &= dW_2.
 \end{array}
 \right\}
 \label{e4.1-s}
\end{equation}

\vspace*{-2pt}

\noindent
Здесь $W_1$ и~$W_2$~--- независимые винеровские процессы
размерности  $n_{w_1}\hm = n_x$ и~$n_{w_2}\hm= n_y$. Тогда после
перехода от дифференциалов к~производным  будем
иметь

\vspace*{4pt}

\noindent
\begin{equation}
\left.
\begin{array}{rl}
\dot X_t& =a\xtt + b(t,\Theta) V_1;\\[6pt]
Z_t &=\dot Y_t =a_1 \xtt + V_2,
\end{array}
\right\}
\label{e4.2-s}
\end{equation}

\vspace*{-2pt}

\noindent
где  $V=[V_1 V_2]^{\mathrm{T}}$~--- нормальный белый шум интенсивности $\nu\hm=\mathrm{diag} \left(\nu_1,\nu_2\right)$.

Заменим~(\ref{e4.2-s}) статистически линеаризованной системой, нелинейной
относительно математических ожиданий  $m_t^x$ и~$m_t^z$ и~линейной
относительно центрированных составляющих $X_t^0 \hm= X_t \hm- m_t^x$ и~$\hx^0 \hm=\hx \hm-\hat m_t^x$:

\vspace*{-4pt}

\noindent
\begin{align}
\dot m_t^x &=a_{00} \left(m_t^x, K_t^x, t,\Theta\right);\label{e4.3-s}\\
 m_t^z &=a_{10} \left(m_t^x, K_t^x, t,\Theta\right);\label{e4.4-s}\\
\dot X_t^0 &=a_{01} \left(m_t^x, K_t^x, t,\Theta\right) X_t^0 +b(t,\Theta) V_1;\label{e4.5-s}\\
Z_t^0 &=a_{11} \left(m_t^x, K_t^x, t,\Theta\right) X_t^0 + V_2,\label{e4.6-s}
\end{align}

\vspace*{-2pt}

\noindent
где $a_{00}=a_{00} (m_t^x, K_t^x, t,\Theta)$,
$a_{10}\hm=a_{10}(m_t^x, K_t^x, t,\Theta)$, $a_{01}
(m_t^x, K_t^x, t,\Theta)\hm= \prt a_0 /\prt m_t^x$
и~$a_{11} (m_t^x, K_t^x, t,\Theta)\hm=\prt a_{10} /\prt m_t^x$~---
коэффициенты статистической линеаризации нелинейных функций~$a$ и~$a_1$, 
вычисляемые для нормального распределения~${\cal N} (m_t^x, K_t^x)$. При этом в~силу~(\ref{e4.5-s}) 
ковариационная матрица~$K_t^x$ будет определяться уравнением:

\vspace*{-1pt}

\noindent
\begin{multline}
\dot K_t^x = a_{11} \left(m_t^x, K_t^x, t,\Theta\right) K_t^x +{}\\
\!\!\!\!\!+ K_t^x a_{11} \left(m_t^x, K_t^x, t,\Theta\right)^{\mathrm{T}} \!\!\!+
   b(t,\Theta) \nu_1(t,\Theta)b(t,\Theta)^{\mathrm{T}}\!.\!\!\!\!\label{e4.7-s}
   \end{multline}
   
   \vspace*{-1pt}
   
   \noindent
Применяя к~модели~(\ref{e4.5-s}), (\ref{e4.6-s}) уравнения  линейного фильтра
Кал\-ма\-на--Бью\-си, получим искомые уравнения квазилинейного
фильтра на основе МСЛ:

\noindent
\begin{multline*}
{\dot{\hat X}}_t =a_{00} \left(m_t^x, K_t^x, t,\Theta\right) - {}\\
{}-
a_{01}\left(m_t^x, K_t^x, t,\Theta\right) m_t^x +
a_{01} \left(m_t^x, K_t^x, t,\Theta\right)\hat X_t + {}
\end{multline*}

\noindent
\begin{multline}
{}+
R_t a_{11}\left(m_t^x, K_t^x, t,\Theta\right)^{\mathrm{T}} \nu_2 (t,\Theta)^{-1}
 \left[ Z_t -{}\right.\\
 {}-a_{11}\left(m_t^x, K_t^x, t,\Theta\right) \hat X_t -
  a_{10}\left(m_t^x, K_t^x, t,\Theta\right)+{}\\
\left.  {}+
  a_{11} \left(m_t^x, K_t^x, t,\Theta\right)m_t^x\right] ,\label{e4.8-s}
  \end{multline}
где $\hat X_0 =\mathsf{M}\,X_0$;
\begin{multline}
\dot R_t ={}\\
{}=a_{01} \left(m_t^x, K_t^x, t,\Theta\right)R_t + 
R_t a_{01}\left(m_t^x, K_t^x, t,\Theta\right)^{\mathrm{T}}-{}\\
{}- R_t a_{11} \left(m_t^x, K_t^x, t,\Theta\right)^{\mathrm{T}} \nu_2 (t,\Theta)^{-1}\times{}\\
{}\times 
a_{11} \left(m_t^x, K_t^x, t,\Theta\right)R_t+ b(t,\Theta) \nu_1(t,\Theta) b(t,\Theta)^{\mathrm{T}},\\  
  R_0 =\mathsf{M}\left[ \left(X_0-\hat X_0\right) \left(X_0 -\hat X_0\right)^{\mathrm{T}}\right].
  \label{e4.9-s}
  \end{multline}
Входящие сюда $m_t^x$ и~$K_t^x$ определяются из уравнений~(\ref{e4.3-s})
и~(\ref{e4.4-s}).

Полученные результаты можно сформулировать в~виде теоремы.

\smallskip

\noindent
\textbf{Теорема~4.}\ \textit{Пусть в~условиях}~(\ref{e4.1-s}) \textit{и~теоремы~$1$ 
уравнения нелинейной гауссовской
дифференциальной СтС}~(\ref{e4.2-s}) \textit{допускают применение МСЛ, 
а~линеаризованные уравнения удовлетворяют условиям стохастической
наблюдаемости. Тогда НСОФ согласно МСЛ
определяется уравнениями}~(\ref{e4.8-s}), (\ref{e4.9-s}) \textit{при условиях}~(\ref{e4.3-s}), 
(\ref{e4.4-s}) \textit{и}~(\ref{e4.7-s}) \textit{и~соответствующих начальных
условиях}.


\smallskip

При чисто линейных наблюдениях, когда  
$$
a_1 \xtt = b_1(t,\Theta)
X_t  +b_0(t,\Theta),
$$
 уравнения~(\ref{e4.8-s}) и~(\ref{e4.9-s}) упрощаются,
поскольку 
\begin{gather*}
a_{10}\left(m_t^x, K_t^x, t,\Theta\right)= b_0(t,\Theta);\\
a_{11}\left(m_t^x, K_t^x, t,\Theta\right)= b_1(t,\Theta);\\
\beta_t = R_t b_1 (t,\Theta)^{\mathrm{T}}\nu_2 (t,\Theta)^{-1},
\end{gather*}
 и~принимают вид:
\begin{multline*}
{\dot{\hat X}}_t =a_{00} \left(m_t^x, K_t^x, t,\Theta\right) -
a_{01}\left(m_t^x, K_t^x, t,\Theta\right) m_t^x     + {}\\
{}+
a_{01}\left(m_t^x, K_t^x, t,\Theta\right)\hat X_t+{}\\
{}+ \beta_t \left[ Z_t - b_1(t,\Theta) \hat X_t - b_0 (t,\Theta) + 
b_1(t,\Theta) m_t^x \right],\\ \hat X_0 = \mathsf{M}\,X_0\,;
%\label{e4.10-s}
\end{multline*}

\vspace*{-12pt}

\noindent
\begin{multline*}
\!\!\!\!\dot R_t =a_{01} \left(m_t^x, K_t^x, t,\Theta\right) R_t + 
R_t a_{01}\left(m_t^x, K_t^x, t,\Theta\right)^{\mathrm{T}} -{}\\
{}- \beta_t  b_1 (t,\Theta) R_t +b(t,\Theta)\nu_1(t,\Theta) b(t,\Theta)^{\mathrm{T}},
\\
R_0 =\mathsf{M} \left[ \left(X_0-\hat X_0\right) \left(X_0-\hat X_0\right)^{\mathrm{T}}\right].
%\label{e4.11-s}
\end{multline*}

Коэффициенты статистической линеаризации~$a_{00}$,
$a_{01}$, $a_{10}$ и~$a_{11}$ и~вспомогательная
(инструментальная) матрица~$R_t$ размерности  $n_z\times n_z$ не
содержат результаты наблюдений~$Z_t$ и~могут быть определены
отдельно (до получения результатов наблюдений). Тогда
линеаризованные уравнения~(\ref{e4.3-s}) и~(\ref{e4.5-s}) в~силу их простоты
могут быть проинтегрированы в~реальном масштабе времени в~течение
наблюдений системы. При этом возможна априорная оценка точ\-ности
 фильтра~\cite{10-s}.


\section{Пример}

Рассмотрим скалярную нелинейную СтС вида
\begin{equation}
 \vrp=-\dot X_t^3 -X_t^3 + X_t V(\Theta),\enskip X_{t_0}=X_0.
 \label{e5.1-s}
 \end{equation}
Здесь $V(\Theta)$~--- гауссовский белый шум ин\-тен\-сив\-ности $\nu\hm=\nu(\Theta)$. Выполним с~(\ref{e5.1-s}) 
сле\-ду\-ющие преобразования. Проведем статистическую линеаризацию левой части~(\ref{e5.1-s}), положив
\begin{equation*}
\dot X_t^3 = \vrp_0 \left( m_t^{\dot X}, D_t^{\dot X}\right)+ 
k_{1} \left( m_t^{\dot X}, D_t^{\dot X}\right)\dot X_t^0,
%\label{e5.2-s}
\end{equation*}
где

\noindent
\begin{align*}
\vrp_0&=\vrp_0 \left( m_t^{\dot X}, D_t^{\dot X}\right)= D_t^{\dot X} 
\lk 3 + \fr{\left( m_t^{\dot X}\right)^2}{D_t^{\dot X}}\rk\,;\\
   k_1 &= k_1 \left( m_t^{\dot X}, D_t^{\dot X}\right)= 
   3D_t^{\dot X} \lk 1+\fr{\left( m_t^{\dot X}\right)^2}{D_t^{\dot X}}\rk.
   \end{align*} %\eqno(5.3)$$
Учитывая, что  $k_1 \ne 0$, заменим~(\ref{e5.1-s}) приближенной СтС следующего вида:
\begin{equation}
\dot X_t = \alpha - \beta X_t^3 + \gamma X_t V(\Theta), \enskip X\left(t_0\right) = X_0, 
\label{e5.4-s}
\end{equation}
где

\vspace*{-2pt}

\noindent
\begin{align*}
\alpha&= m_t^{\dot X} - \vrp_0 k_1^{-1} = m_t^{\dot X} - 
\fr{3+( m_t^{\dot X})^2/ D_t^{\dot X}}{3 \lk 1+( m_t^{\dot X})^2/ D_t^{\dot X}\rk};\\
\beta&= \gamma=k_1^{-1}=  \fr{1}{3 D_t^{\dot X}\lk 1+( m_t^{\dot X})^2/ D_t^{\dot Y}\rk}.
\end{align*}
%\eqno(5.5)

Система~(\ref{e5.4-s}) при $\alpha\hm=0$ и~$\beta\hm=\gamma\hm=1$ подробно изуче\-на в~\cite{8-s, 10-s}. 
Обобщая~\cite{8-s, 10-s} на случай $\beta\hm= \gamma\hm \ne 1$ с~точ\-ностью до вероятностных 
моментов второго порядка, придем к~следующим взаимосвязанным уравнениям для~$m_t^X, D_t^X$~\cite{8-s}:

%\pagebreak

\noindent
\begin{align*}
\dot m^X_t &= \alpha - \beta m_t^X \lk \left(m_t^X\right)^2 +3D_t^X\rk, \enskip 
m^X \left(t_0\right) = m_0^X;\\ %\eqno(5.6)$$
\dot D_t^X &= \beta^2 \lk \nu(\Theta) - 6 D_t^X\rk \lk \left(m_t^X\right)^2 + D_t^X\rk, \\ 
&\hspace*{49.5mm}D^X \left(t_0\right) = D_0^X. %\eqno(5.7)$$
\end{align*}

Пусть наблюдаемый процесс  определяется уравнением
\begin{equation*}
Z_t=\dot Y_t = X_t+V_2.
%\label{e5.8-s}
\end{equation*}
Процесс  $W(t)$  состоит из двух независимых
скалярных винеровских процессов~$W_1$ и~$W_2$, слабыми с.к.\
производными которых служат белые шумы~$V_1$ и~$V_2$
соответственно. Соответствующую структуру имеют в~этом случае
матрицы
  \begin{equation*}
    b\left(X_t,Y_t,t\right) =\left[X_t\,0\right];\enskip 
    b_1\left(Y_t,t\right) =[0\,1];\enskip
    \nu= \left[
\begin{array}{cc}
\nu_1&0\\
0&\nu_2
\end{array}
\right], %\eqno(5.9)$$
\end{equation*}
где  $\nu_1$  и~$\nu_2$~--- интенсивности белых шумов  $V_1$ и~$V_2$ соответственно. Функции  $a$ и~$a_1$
определяются формулами $ a(X_t,Y_t,t)\hm =-X_t^3$ 
и~$a_1(X_t,Y_t,t)\hm =X_t$. Уравнения~(\ref{e3.1-s}) и~(\ref{e3.2-s}) имеют вид:
   \begin{align*}
    \dot{\hat X}_t &=- \hat X_t \left(\hat X_t^2 + 3R_t\right) +\nu_2^{-1} R_t \left(Z_t-\hat
    X_t\right);\\ %\eqno(5.10)$$
  \dot R_t &=\left(\nu_1 - 6R_t\right) \left(\hat X_t^2 +R_t\right) -\nu_2^{-1}
    R_t^2. %\eqno(5.11)
    \end{align*}
 Эти
уравнения приближенно определяют с.к.\ оптимальную оценку~$\hat X_t$
 состояния системы и~апостериорную дисперсию ошибки. За начальные
 значения~$\hat X_t$ и~$R_t$ следует взять априорные \mbox{математическое}
 ожидание и~дисперсию величины~$X_0$, поскольку~$Y_0$ не задано 
 и~может быть взято совершенно произвольно, независимо от~$X_0$.

 Заметим, что полученный НСОФ упрощается, если дополнительно провести статистическую линеаризацию
 кубической функции в~(\ref{e5.1-s}). В~этом случае НСОФ будет обобщенным фильтром Калмана--Бьюси.


\section{Качество нормального субоптимального фильтра}

Алгоритмы синтеза НСОФ зависят от инструментальных параметров~$\Theta$. 
Они представляют собой случайные величины (СВ) или медленно меняющиеся функции времени. 
Применим методы теории чувствительности~\cite{11-s, 12-s} при гауссовских СВ~$\Theta$  аналогично~\cite{8-s}. 
В~результате получим для оценки качества НСОФ и~условной функции потерь $\rho\hm= R(\Theta, t)$, допускающей квадратическую аппроксимацию
  \begin{multline*}
  \rho = \rho(\Theta) = \rho \left(m^\Theta\right) + {}\\
  {}+
    \sum\limits_{i=1}^{p^\Theta} \rho_i' \left(m^\Theta\right) \Theta_i^0 + 
      \ssz_{i, j =1} \rho_{ij}'' \left(m^\Theta \right)\Theta_i^0\Theta_j^0, 
  %\label{e6.1-s}
  \end{multline*}
показатель $\varepsilon$, равный
\begin{equation*}
  \varepsilon = \varepsilon_2^{1/4},\enskip \varepsilon_2 =\mathsf{M}_{\cal N} 
  \lk \rho (\Theta)\rk^2 - \rho \left(m^\Theta\right)^2,
  %\label{e6.2-s}
  \end{equation*}
где
\begin{multline*}
\mathsf{M}_{\cal N} \lk \rho(\Theta)^2\rk = 
    \rho \left(m^\Theta\right)^2 + \rho' \left(m^\Theta\right)^{\mathrm{T}} K^\Theta \rho' \left(m^\Theta\right) + {}\\
    {}+
    2 \rho \left(m^\Theta\right) \mathrm{tr} \left[ \rho'' \left(m^\Theta\right) K^\Theta\right] +{}\\
{}+ \left\{ \mathrm{tr} \left[ \rho '' \left(m^\Theta\right) K^\Theta\right] \right\}^2 + 
2 \mathrm{tr} \left[ \rho'' \left(m^\Theta\right) K^\Theta\right]^2.
%\eqno(6.3)$$
\end{multline*}

Функции чувствительности~$\rho_i'$ и~$\rho_i''$ находятся путем решения уравнений для функций 
чувствительности~$\nabla^\Theta R_t$ и~$(\nabla^\Theta)^2 R_t$ в~силу~(\ref{e3.2-s}):
    $$
    \nabla^\Theta \dot R_t= \nabla^\Theta B\,; \quad 
    \left(\nabla^\Theta\right)^2 \dot R_t=\left(\nabla^\Theta\right)^2 B %\eqno(6.4)
    $$
при нулевых начальных условиях.

\vspace*{-6pt}

\section{Заключение}


Для нормальных (гауссовских) нелинейный дифференциальных СтС, не разрешенных 
относительно производных, разработаны методы синтеза  с.к.\
 НСОФ (теоремы~2--4). На основе принципа эквивалентности гауссовских и~негауссовских 
 шумов получено обобщение на случай негауссовских дифференциальных СтС.  
 Разработанные методы справедливы только в~случае, когда уравнение наблюдения 
 не содержит пуассоновских шумов, а~коэффициент при гауссовском белом шуме не зависит 
 от обобщенного вектора состояния. Особое внимание уделено синтезу НСОФ 
 для СтС с~аддитивными шумами. Приведен пример, иллюстрирующий методы. Разработан метод 
 оценки качества НСОФ на основе теории чувствительности.

В настоящее время идет разработка учебного экспериментального 
программного обеспечения по тематике~\cite{13-s, 14-s, 15-s, 16-s} в~случаях, когда в~уравнениях состояния
 можно пренебречь постоянными времени высокого порядка.

В качестве обобщений могут рассматриваться следующие задачи:
\begin{enumerate}[(1)]
\item синтез с.к.\ услов\-но-оп\-ти\-маль\-ных фильт\-ров, экстраполяторов и~интерполяторов 
на основе па\-ра\-мет\-ри\-за\-ции распределений (моментов, семиинвариантов, ортогональных разложений, 
эллипсоидальной аппроксимации и~др.);
\item в~том случае когда вектор возмущений и/или нелинейные функции заданы 
каноническими представлениями случайных функций, полученные результаты допускают 
обобщения, если воспользоваться~\cite{2-s, 3-s};
\item синтез  с.к.\ услов\-но-оп\-ти\-маль\-ных фильт\-ров для стохастических эредитарных 
интегродифференциальных сис\-тем, не разрешенных относительно производных.
\end{enumerate}

Наконец, теоретический и~практический интерес представляют задачи, 
когда функции~(\ref{e2.1-s}) являются разрывными, а~также стохастическими функциями отмеченных переменных.





%\vspace*{-8pt}

{\small\frenchspacing
{%\baselineskip=10.8pt
%\addcontentsline{toc}{section}{References}
\begin{thebibliography}{99}
%1
\bibitem{1-s}
\Au{Пугачёв~В.\,С., Синицын~И.\,Н.}
Стохастические дифференциальные системы. Анализ и~фильтрация.~--- М.:
Наука,  1990.  632~с. (\Au{Pugachev~V.\,S., Sinitsyn I.\,N.}
Stochastic differential systems.
Analysis and filtering.~--- Chichester\,--\,New York: Jonh Wiley \& Sons, 1987.
549~p.)

%2
\bibitem{2-s}
\Au{Пугачёв В.\,С., Синицын~И.,Н.}
Теория стохастических систем.~--- М.: Логос, 2000; 2004. 1000~с.
%(\Au{Pugachev~V.\,S., Sinitsyn I.\,N.}
%Stochastic systems. Theory and  applications.~---
%Singapore: World Scientific, 2001. 908~p.)

%3
\bibitem{3-s}
\Au{Синицын И.\,Н.}
Канонические представления случайных функций и~их применение 
в~задачах компьютерной поддержки научных исследований.~--- М.: ТОРУС
ПРЕСС, 2009. 768~с.

%4
\bibitem{4-s}
\Au{Синицын И.\,Н.,  Синицын~В.\,И.}
Лекции по нормальной и~эллипсоидальной аппроксимации распределений 
в~стохастических системах.~--- М.: ТОРУС ПРЕСС, 2013. 488~с.

%5
\bibitem{5-s}
\Au{Синицын И.\,Н.}
Параметрическое статистическое и~аналитическое моделирование распределений в~нелинейных 
стохастических системах на многообразиях~// Информатика и~её применения, 2013. Т.~7. Вып.~2. С.~4--16.

%6
\bibitem{6-s}
\Au{Синицын И.\,Н., Синицын~В.\,И.}
Аналитическое моделирование нормальных процессов в~стохастических системах со сложными нелинейностями~// 
Информатика и~её применения, 2014. Т.~8. Вып.~3. С.~2--4.

%7
\bibitem{7-s}
\Au{Синицын И.\,Н.}
Аналитическое моделирование широкополосных процессов в~стохастических системах, 
не разрешенных относительно производных~// Информатика и~её применения, 2017. Т.~11. Вып.~1. С.~3--10.

%8
\bibitem{8-s}
\Au{Синицын И.\,Н. }
Параметрическое аналитическое моделирование процессов в~стохастических  сис\-те\-мах, 
не разрешенных относительно производных~// Сис\-те\-мы и~средства информатики, 2017. Т.~27. №\,1. С.~20--45.

%%9
%\bibitem{9-s}
%\Au{Sinitsyn I.\,N.}
%Analytical modeling of ``linear'' and circular control stochastic systems~// 
%Advances in robotics and automatic control:  Reviews.~--- International Frequency Sensor
%Association, 2021 (in press). Vol.~2. Ch.~2.

%10
\bibitem{10-s}
\Au{Синицын И.\,Н.}
Фильтры Калмана и~Пугачёва.~--- 2-е изд.~--- М.: Логос, 2007. 776~с.


%11
\bibitem{11-s}
Справочник по теории автоматического управления~/ Под ред. А.\,А.~Красовского.~--- М.: Наука, 1987. 712~с.

%12
\bibitem{12-s}
\Au{Евланов А.\,Г., Константинов~В.\,М.}
Системы со случайными параметрами.~--- М.: Наука, 1976. 568~с.

%13
\bibitem{13-s}
ГОСТ 23743--88. Изделия авиационной техники. Номенклатура показателей безопасности полета, 
надежности, контролепригодности, эксплуатационной и~ремонтной технологичности.

%14
\bibitem{14-s}
\Au{Александровская Л.\,Н., Аронов~И.\,З., Круглов~В.\,И. и~др.} 
Безопас\-ность и~на\-деж\-ность технических сис\-тем.~--- 
М.: Университетская книга, Логос, 2008. 376~с.

%15
\bibitem{15-s}
\Au{Синицын И.\,Н.} 
Лекции по теории сис\-тем интегрированной логистической поддержки.~--- 
М.: ТОРУС ПРЕСС, 2019. 1072~с.

%16
\bibitem{16-s}
\Au{Sinitsyn I.\,N., Shalamov~A.\,S.}
Probabilistic modeling,\linebreak estimation and control for CALS organization-technical-economic systems~// 
Probability, combinatorics and control and control~/ Eds. A.~Kostohryzov, V.~Korolev.~--- 
Intechopen, 2019. P.~117--141. doi: 10.5772.79802.
\end{thebibliography}

}
}

\end{multicols}

\vspace*{-3pt}

\hfill{\small\textit{Поступила в~редакцию 29.06.2020}}

\vspace*{8pt}

%\pagebreak

%\newpage

%\vspace*{-28pt}

\hrule

\vspace*{2pt}

\hrule

%\vspace*{-2pt}

\def\tit{NORMAL SUBOPTIMAL FILTERING FOR~DIFFERENTIAL STOCHASTIC SYSTEMS WITH~UNSOLVED DERIVATIVES}

\def\titkol{Normal suboptimal filtering for~differential stochastic systems with~unsolved derivatives}

\def\aut{I.\,N.~Sinitsyn}

\def\autkol{I.\,N.~Sinitsyn}

\titel{\tit}{\aut}{\autkol}{\titkol}

\vspace*{-11pt}


\noindent
Institute of Informatics Problems, Federal Research Center ``Computer Science and
Control'' of the Russian Academy of Sciences, 44-2~Vavilov Str., Moscow 119333,
Russian Federation

\def\leftfootline{\small{\textbf{\thepage}
\hfill INFORMATIKA I EE PRIMENENIYA~--- INFORMATICS AND
APPLICATIONS\ \ \ 2021\ \ \ volume~15\ \ \ issue\ 1}
}%
\def\rightfootline{\small{INFORMATIKA I EE PRIMENENIYA~---
INFORMATICS AND APPLICATIONS\ \ \ 2021\ \ \ volume~15\ \ \ issue\ 1
\hfill \textbf{\thepage}}}

\vspace*{3pt}




\Abste{The article develops the series of papers dedicated to stochastic systems with unsolved derivatives. 
Methodological aspects of normal suboptimal filterings (NSOF) for stochastic systems with unsolved 
derivatives are presented. Nonlinear differential equations for state and observation 
are given at the following conditions: observation equations are Gaussian and do not depend on 
the state variable. One of sections is devoted to NSOF for Gaussian and non-Gaussian systems.
Corresponding NSOF are given for additive noises. Also, an illustrative example 
 is given. The NSOF quality analysis is considered.}


\KWE{method of analytical modeling (MAM); method of normal approximation (MNA); 
method of statistical linearization (MSL); normal suboptimal filter; 
stochastic system (StS); stochastic systems with unsolved derivatives;
shaping filter}
 


\DOI{10.14357/19922264210101}

%\vspace*{-15pt}

%\Ack
%\noindent
%The research was supported by the Russian Academy of Sciences 
%(project АААА-А19-119091990037-5).

%\vspace*{12pt}

  \begin{multicols}{2}

\renewcommand{\bibname}{\protect\rmfamily References}
%\renewcommand{\bibname}{\large\protect\rm References}

{\small\frenchspacing
 {%\baselineskip=10.8pt
 \addcontentsline{toc}{section}{References}
 \begin{thebibliography}{99}
\bibitem{1-s-1}
\Aue{Pugachev, V.\,S., and I.\,N.~Sinitsyn.} 
1987. \textit{Stochastic differential systems. Analysis and filtering.} 
Chichester\,--\,New York: John Wiley \& Sons. 549~p.

\bibitem{2-s-1}
\Aue{Pugachev, V.\,S., and I.\,N.~Sinitsyn.} 2000, 2004.
\textit{Teoriya stokhasticheskikh sistem} [Stochastic systems. Theory and  applications]. 
Moscow: Logos. 1000~p.  %[Angl. per. . -- Singapore: World Scientific, 2001].

\bibitem{3-s-1}
\Aue{Sinitsyn, I.\,N.} 2009. \textit{Kanonicheskie predstavleniya sluchaynykh funktsiy i~ikh primenenie 
v~zadachakh komp'yuternoy podderzhki nauchnykh issledovaniy} 
[Canonical expansions of random functions and their applications in computer-aided support]. 
Moscow: TORUS PRESS. 768~p.
\bibitem{4-s-1}
\Aue{Sinitsyn, I.\,N., and V.\,I.~Sinitsyn.}
 2013. \textit{Lektsii po nor\-mal'\-noy i~ellipsoidal'noy approksimatsii raspredeleniy 
 v~sto\-kha\-sti\-che\-skikh sistemakh} [Lectures on normal and ellipsoidal approximation in stochastic systems].
  Moscow: TORUS PRESS. 488~p.
\bibitem{5-s-1}
\Aue{Sinitsyn, I.\,N.} 2013. Parametricheskoe statisticheskoe 
i~analiticheskoe modelirovanie raspredeleniy v~nelineynykh stokhasticheskikh sistemakh na 
mno\-go\-ob\-ra\-zi\-yakh 
[Parametric statistical and analytical modeling of distributions in stochastic systems on manifolds]. 
\textit{Informatika i~eePrimeneniya~--- Inform. Appl.} 7(2):4--16.
\bibitem{6-s-1}
\Aue{Sinitsyn, I.\,N., and V.\,I.~Sinitsyn.}
 2014. Analiticheskoe modelirovanie normal'nykh protsessov 
 v~sto\-kha\-sti\-che\-skikh sistemakh so slozhnymi nelineynostyami 
 [Analytical modeling of normal processes in stochastic systems with complex nonlinearities]. 
 \textit{Informatika i~eePrimeneniya~--- Inform. Appl.} 8(3):2--4.
\bibitem{7-s-1}
\Aue{Sinitsyn, I.\,N.}
 2017. Analiticheskoe modelirovanie shirokopolosnykh protsessov v~stokhasticheskikh sistemakh, 
 ne razreshennykh otnositel'no proizvodnykh [Analytical modeling of wide band processes in
  stochastic systems with unsolved derivatives]. 
  \textit{Informatika i~ee Primeneniya~--- Inform. Appl.} 11(1):3--10.
\bibitem{8-s-1}
\Aue{Sinitsyn, I.\,N.} 2017. Parametricheskoe analiticheskoe modelirovanie protsessov 
v~stokhasticheskikh  sistemakh, ne razreshennykh otnositel'no proizvodnykh 
[Parametric analytical modeling of wide band processes in stochastic systems 
with unsolved derivatives]. \textit{Sistemy i~Sredstva Informatiki~--- 
Systems and Means of Informatics} 27(1):20--45.
%\bibitem{9-s-1}
%\Aue{Sinitsyn, I.\,N.} 2021 (in press.) 
%Analytical modeling of ``linear'' and circular control stochastic systems. 
%\textit{Advances in robotics and automatic control: Reviews}.
%International Frequency Sensor Association. Vol.~2. Ch.~2.

\bibitem{10-s-1}
\Aue{Sinitsyn, I.\,N.} 2007. \textit{Fil'try Kalmana i~Pugacheva} [Kalman and Pugachev filters]. 
2nd. ed. Moscow: Logos. 776~p.

\bibitem{11-s-1}
Krasovskiy, A.\,A., ed. 1987. 
\textit{Spravochnik po teorii avtomaticheskogo upravleniya} [Handbook for automatic control]. Moscow: Nauka. 
712~p.
\bibitem{12-s-1}
\Aue{Evlanov, A.\,G., and V.\,M.~Konstantinov.} 1976.
\textit{Sistemy so slozhnymi parametrami} [Systems with random parameters]. Moscow: Nauka. 568~p.
\bibitem{13-s-1}
GOST 23743-88. Izdeliya aviatsionnoy tekhniki. Nomenklatura pokazateley bezopasnosti poleta, 
na\-dezh\-nosti, kontroleprigodnosti, ekspluatatsionnoy i~remontnoy tekhnologichnosti 
[Aircraft products. Nomenclature of flight safety indicators, reliability, controllability, 
operational and repair processability].
\bibitem{14-s-1}
\Aue{Aleksandrovskaya, L.\,N., I.\,Z.~Aronov, V.\,I.~Kruglov, \textit{et al.}}
 2008. \textit{Bezopasnost' i~nadezhnost' tekhnicheskikh sistem} [Security and reliability of technical systems]. 
 Moscow: Universitetskaya kniga, Logos. 376~p.
\bibitem{15-s-1}
\Aue{Sinitsyn, I.\,N.} 2019. \textit{Lektsii po teorii sistem integrirovannoy logisticheskoy podderzhki}
 [Lectures on theory of integrated logistic support systems]. Moscow: TORUS PRESS. 1072~p.
\bibitem{16-s-1}
\Aue{Sinitsyn, I.\,N., and A.\,S.~Shalamov.} 2019. 
Probabilistic modeling, estimation and control for CALS organization-technical-economic systems. 
\textit{Probability, combinatorics and control and control.} 
Eds. A.~Kostohryzov and V.~Korolev. IntechOpen. 117--141.  doi: 10.5772.79802.
 \end{thebibliography}

 }
 }

\end{multicols}

\vspace*{-3pt}

  \hfill{\small\textit{Received June~29, 2020}}


%\pagebreak

%\vspace*{-8pt}

\Contrl

\noindent
\textbf{Sinitsyn Igor N.} (b.\ 1940)~--- 
Doctor of Science in technology, professor, Honored scientist of RF, principal scientist, 
Institute of Informatics Problems, Federal Research Center 
``Computer Science and Control'' of the Russian Academy of Sciences, 
44-2~Vavilov Str., Moscow 119333, Russian Federation; \mbox{sinitsin@dol.ru}

\label{end\stat}

\renewcommand{\bibname}{\protect\rm Литература}