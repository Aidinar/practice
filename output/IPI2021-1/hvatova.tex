\def\stat{hvatova}

\def\tit{МОДЕЛИРОВАНИЕ СТОХАСТИЧЕСКОЙ ДИНАМИКИ ИЗМЕНЕНИЯ СОСТОЯНИЙ УЗЛОВ 
И~ПЕРКОЛЯЦИОННЫХ ПЕРЕХОДОВ В~СОЦИАЛЬНЫХ СЕТЯХ С~УЧЕТОМ 
САМООРГАНИЗАЦИИ И~НАЛИЧИЯ ПАМЯТИ}

\def\titkol{Моделирование стохастической динамики изменения состояний узлов 
и~перколяционных переходов} % в~социальных сетях с~учетом самоорганизации и~наличия памяти}

\def\aut{Д.\,О.~Жуков$^1$, Т.\,Ю.~Хватова$^2$, А.\,Д.~Зальцман$^3$}

\def\autkol{Д.\,О.~Жуков, Т.\,Ю.~Хватова, А.\,Д.~Зальцман}

\titel{\tit}{\aut}{\autkol}{\titkol}

\index{Жуков Д.\,О.}
\index{Хватова Т.\,Ю.}
\index{Зальцман А.\,Д.}
\index{Zhukov D.\,O.}
\index{Khvatova T.\,Yu.}
\index{Zaltcman A.\,D.}

%{\renewcommand{\thefootnote}{\fnsymbol{footnote}} \footnotetext[1]
%{Финансовое обеспечение исследований осуществлялось из средств федерального бюджета на 
%выполнение государственного задания Карельского научного центра Российской академии наук 
%(Институт прикладных математических исследований КарНЦ РАН).}}

\renewcommand{\thefootnote}{\arabic{footnote}}
\footnotetext[1]{МИРЭА~--- Российский технологический университет, zhukov\_do@mirea.ru}
\footnotetext[2]{Санкт-Петербургский политехнический университет Петра Великого, \mbox{khvatova.ty@spbstu.ru}}
\footnotetext[3]{МИРЭА~--- Российский технологический университет, ad.zaltcman@gmail.com}


\vspace*{-6pt}



\Abst{Ообсуждаются вопросы использования подходов теоретической информатики 
и~применение ее приложений для анализа и~моделирования процессов в~социотехнических 
системах (социальных сетях). Разработана стохастическая модель динамики изменения 
состояний (настроений или мнений) пользователей (узлов) и~достижения порога перколяции 
в~социальной сети, имеющей случайные связи между узлами. Модель показывает 
возможность скачкообразных переходов между состояниями (мнений, настроений и~т.\,д.)\ 
узлов в~социальной сетевой структуре в~течение короткого времени без внешнего 
воздействия, что может быть связано с~памятью о предыдущих состояниях 
и~самоорганизацией. При создании модели были рассмотрены схемы вероятностей 
переходов между возможными состояниями узлов с~учетом предыдущих шагов 
(немарковские процессы с~наличием памяти) и~выведено нелинейное дифференциальное 
уравнение второго порядка, которое содержит член, отвечающий за возможность 
самоорганизации, а также сформулирована и~решена граничная задача для определения 
функции плотности вероятности нахождения системы в~определенном состоянии с~течением 
времени. Разработанная модель может быть связана с~полученными ее авторами ранее 
результатами описания процессов в~социальных сетевых структурах с~помощью теории 
перколяции (определение времени достижения пороговых значений доли узлов сети, при 
котором мнения или предпочтения могут беспрепятственно распространяться по сети 
в~целом).}

\KW{стохастическая динамика; состояния узлов социальной сети; самоорганизация систем; 
процессы с~памятью; перколяция в~социальных сетях}

\DOI{10.14357/19922264210114}

\vspace*{-4pt}

\vskip 10pt plus 9pt minus 6pt

\thispagestyle{headings}

\begin{multicols}{2}

\label{st\stat}

\section{Введение}

  Отличительной особенностью динамики явлений в~социотехнических 
и~социальных системах является активное воздействие на них человеческого 
фактора, который, с~одной стороны, вносит неопределенность и~создает 
стохастичность, а~с~другой стороны, создает возможности для 
самоорганизации, позволяет говорить о наличии памяти и~придает динамике 
процессов существенно нелинейный характер.
  
  Для моделирования нелинейной динамики самоорганизующихся социальных 
систем с~памятью можно и~нужно применять методы и~средства теоретической 
информатики и~кибернетики, которая, по определению Роберта Винера, 
является наукой об управлении не только техническими, но и~биологическими 
системами.
  
  Использование методов теоретической информатики, разработанных в~ней 
принципов моделирования и~ее приложений может позволить получить 
качественно новые результаты для описания сложных социальных, 
экономических и~социотехнических систем, а также создать новые методики 
прогнозирования поведения людей в~социальных и~социотехнических системах.

%\vspace*{-6pt}
  
\section{Обзор некоторых моделей описания динамики процессов 
в~социальных сетевых структурах}

%\vspace*{-2pt}

  Многие из существующих теоретических подходов к~описанию социальных 
сетевых систем имеют много общего с~кинетическим описанием физических 
систем и~распространением вирусов в~компьютерных сетях. Однако эти модели 
в~основном рассматривают цепные явления, 
где макроскопическая доля узлов с~определенным состоянием в~сети быстро возникает из некоторого 
микроскопического состояния, захватывающего все новые и~новые узлы.
  
  В более сложных моделях взаимодействие пользователей социальных сетей 
описывается теорией многоагентных систем~[1--3], а также аппаратом теории 
клеточных автоматов~[4, 5]. В этих моделях на основании некоторых правил 
переходов агенты принимают определенные состояния, образуют связанную по 
своим свойствам группу, могут сотрудничать, чтобы решить некую задачу или 
достигнуть определенной цели~\cite{1-hv}, а~временн$\acute{\mbox{а}}$я логика поведения 
агентов может зависеть от динамически меняющихся условий~\cite{2-hv}.
  
  В работе~\cite{4-hv} было изучено влияние структуры сетей (случайные 
структуры, маленькие миры, цикл, колесо, звезда, иерархическая) и~правил 
поведения клеточных автоматов на динамику процессов в~социальных сетях. 
При одинаковых правилах взаимодействия клеток динамика процессов зависит 
от топологии сети (наибольшая скорость наблюдается в~регулярных 
структурах, а наименьшая~--- в~неупорядоченных).
  
  Для описания процессов в~социальных сетях также применяются 
стохастические подходы, учитывающие зависимости изменения состояния 
узлов от времени. В~работе~[6] описана модель смешанного членства 
в~стохастически формирующихся группах, основанная на попарном 
рассмотрении присутствия или отсутствия связей между объектами. Анализ 
вероятностных изменений связей требует специальных предположений, 
например независимости или предположения непостоянства данной связи 
(смешанного членства в~стохастически формирующихся группах). Данная 
модель позволяет описать динамику кластеризации членов по группам 
и~изменение их численности.
  
  Другим направлением анализа и~прогнозирования динамики процессов 
в~сложных социальных системах является использование нестационарных 
временн$\acute{\mbox{ы}}$х рядов. Традиционный подход к~их анализу основан на том, чтобы 
с~помощью применения линейных методов свести их к~стационарным 
(например, авторегрессионные интегрированные модели скользящего 
среднего~--- ARIMA, autoregressive integrated moving average~[7]). Эти модели оперируют не функциями 
распределения, а~непосредственно элементами временн$\acute{\mbox{о}}$го ряда. Ряды, не 
укла\-ды\-ва\-ющи\-еся в~рамки регрессионного анализа, чаще\linebreak всего изучаются 
адаптивными эвристическими методами, в~которых ряды на некоторой длине 
описываются стационарной моделью типа регрессии или авторегрессии, 
а~параметры модели пересчитываются с~учетом новой информации или 
с~учетом сравнения предсказанного значения с~фактом. Недостаток этих 
подходов заключается в~том, что длина участка возможной стационарности не 
является известной величиной. При исследовании стационарных случайных 
процессов, согласно теореме Гливенко~[8] (о~сходимости эмпирической 
вероятности к~теоретическому распределению), чем больше учтено 
наблюдаемых значений, тем точнее будут получены теоретические 
характеристики распределения. Для нестационарных временн$\acute{\mbox{ы}}$х рядов данное 
условие, в~силу их специфики, не может быть выполнено, что затрудняет 
возможности прогнозирования.
  
  Следует отметить, что ни одна из существующих моделей не рассматривает 
самоорганизацию и~возможность наличия памяти. Поэтому можно сделать 
вывод о том, что для прогнозирования динамики процессов в~социотехнических 
системах, имеющих сетевую структуру, требуется продолжение разработки их 
моделей с~учетом структурных свойств, самоорганизации и~наличия памяти.

\vspace*{-3pt}
  
\section{Постановка задач исследования}

   С~позиций структурного подхода социотехнические системы представляют 
собой случайную сеть взаимосвязей и~взаимодействий пользователей, которая 
приводит к~нелинейной динамике изменения состояний узлов. При 
моделировании нелинейных динамических процессов в~социальных сетевых 
структурах необходимо ответить минимум на два важных вопроса.  
Во-пер\-вых, как учесть стохастичность, неопределенность, самоорганизацию 
процессов и~наличие памяти и~как они влияют на наблюдаемые явления.  
Во-вто\-рых, как структура сетей влияет на динамику процессов и~как она 
может быть связана со стохастичностью, неопределенностью, 
самоорганизацией процессов и~наличием памяти. Ответы на эти вопросы могут 
позволить создать эффективные алгоритмы мониторинга состояния социальных 
сетей.
   
  В~теории перколяции (теория вероятностей на графах) изучают решение 
задач узлов и~связей для сетей с~различной структурой. При решении задачи 
связей определяют долю связей, которую нужно разорвать, чтобы сеть 
распалась минимум на две несвязанные части (или, наоборот, долю 
проводящих связей в~сети, когда в~целом между любыми произвольными 
узлами появляется проводимость). В~задаче узлов определяют среднюю долю 
блокированных узлов, при которой сеть распадется на не связанные между 
собой кластеры, внутри которых сохраняются связи (или, наоборот, долю 
проводящих узлов, когда проводимость возникает). Доля блокированных узлов 
(в~задаче узлов) или разорванных связей (в~задаче связей), при которой 
исчезает проводимость (или, наоборот, появляется) между двумя произвольно 
выбранными узлами сети, называется порогом перколяции (протекания)~[9].
  
  Использование понятия долей блокированных узлов или связей эквивалентно 
понятию вероятности нахождения случайно выбранного узла (или связи) 
в~блокированном (разорванном) состоянии. Поэтому величина порога 
перколяции определяет вероятность передачи информации через всю сеть 
в~целом, если задана средняя вероятность блокирования узла или связи.
  
  Величину порога перколяции для случайной сетевой структуры можно 
определить теоретически методами численного моделирования или 
экспериментально при изучении реальных сетей \mbox{найти} с~по\-мощью 
инструментов социального сетевого анализа (SNA~--- social network analysis).
  
  Если для случайной сети социальных связей порог ее перколяции известен, 
то, описав механизмы перехода ее узлов в~блокированное или проводящее 
состояние, можно определить время его достижения, а~следовательно, 
спрогнозировать динамику распространения определенных мнений или 
взглядов.
  
\section{Перколяционные свойства случайных сетевых структур}

  Исследования~\cite{1-hv, 10-hv, 11-hv, 12-hv, 13-hv, 14-hv} показывают, что 
пороги перколяции случайных сетей зависят от среднего числа связей в~расчете 
на один узел (плотности) сети. Для задачи связей имеется линейная 
зависимость: $y\hm= -6{,}581z\hm- 0{,}203$; а~для задачи узлов: $y\hm= 4{,}39 
z\hm- 2{,}41$. Здесь $z\hm=1/x,$ где~$x$~--- плот\-ность связей; $y$~--- натуральный 
логарифм доли разорванных связей (или узлов в~задаче узлов), при которой 
исчезает проводимость всей сети  
в~целом~\cite{11-hv, 12-hv, 13-hv, 14-hv, 15-hv, 16-hv, 17-hv}.
  
  Полученные ранее  
результаты~~\cite{11-hv, 12-hv, 13-hv, 14-hv, 15-hv, 16-hv, 17-hv} 
о~перколяционных свойствах случайных сетей позволяют сделать ряд очень 
важных выводов. Например, о наличии насыщения порога перколяции, о роли 
увеличения плотности связей в~информационном влиянии сети и~ряд других. 
Следует отметить, что динамика изменения состояния узлов сетей 
в~совокупности с~их перколяционными свойствами была рассмотрена 
в~работе~\cite{10-hv}, где исследовалось распространение компьютерных 
вирусов. Однако влияние процессов самоорганизации и~наличия памяти на 
динамику изменения состояния узлов и~достижение порогов перколяции 
с~течением времени исследовано не было.
  
\section{Стохастическая динамика переходов между состояниями 
в~сетях социальных связей и~достижение порога перколяции 
с~учетом памяти и~самоорганизации}

\subsection{Построение разностных вероятностных схем переходов 
между~состояниями} %5.1

  Будем описывать социальную сеть как систему, состояния которой в~любой 
момент времени могут быть заданы параметром, принимающим непрерывные 
случайные значения с~недетерминированным законом распределения. 
Например, это может быть доля пользователей (узлов сети), раз\-де\-ля\-ющих 
и~пропагандирующих определенные взгляды или настроения.
  
  Все множество состояний будем обозначать как~$X$. Состояние, 
наблюдаемое в~момент времени~$t$, можно обозначить как~$x_i$ ($x_i\hm\in  
{X}$).
  
  Введем интервал времени~$\tau_0$, за который возможно изменение 
состояния~$x_i$. В~данном случае любое значение текущего времени 
$t\hm=h\tau_0$, где~$h$~--- номер шага перехода между состояниями 
(процесс перехода между состояниями становится квазинепрерывным 
с~бесконечно малым временным интервалом~$\tau_0$); $h\hm=0, 1, 2, \ldots, 
N$. Текущее состояние~$x_i$ на шаге~$h$ после перехода на шаг $h\hm+1$ 
может \textit{за счет случайно возникающих факторов} увеличиваться на 
некоторую величину~$\varepsilon$ или уменьшаться на величину~$\xi$, т.\,е.\ 
оказаться равным $x_i\hm+\varepsilon$ или $x_i\hm- \xi$ 
соответственно. Введем понятие вероятности нахождения системы в~том или 
ином состоянии: после некоторого числа шагов~$h$ про описываемую систему 
можно сказать, что ${\sf P}(x\hm- \varepsilon, h)$~--- вероятность того, что она 
находится в~состоянии $(x\hm - \varepsilon)$; ${\sf P}(x,h)$~--- вероятность того, что 
она находится в~состоянии~$x$; ${\sf P}(x\hm+\xi, h)$~--- вероятность того, что она 
находится в~состоянии $(x\hm + \xi)$.
  
  После каждого шага состояние~$x_i$ (далее индекс~$i$ для краткости 
опустим) может изменяться на величину~$\varepsilon$ или~$\xi$. Вероятность 
${\sf P}(x, h\hm+1)$ того, что на следующем, $(h\hm+1)$-м, шаге система (или 
процесс) окажется в~состоянии~$x$ описывается уравнением (рис.~1):
  \begin{equation}
  {\sf P}(x, h+1)= {\sf P}(x - \varepsilon, h)+{\sf P}(x+\xi, h) - {\sf P}(x, h)\,.
  \label{e1-hv}
  \end{equation}



  Поясним уравнение~(1) и~представленную на рис.~1 схему. Вероятность 
перехода в~состояние~$x$ на\linebreak\vspace*{-12pt}

{ \begin{center}  %fig1
 \vspace*{-1pt}
    \mbox{%
\epsfxsize=77.502mm
\epsfbox{hva-1.eps}
}

\end{center}

\noindent
{{\figurename~1}\ \ \small{
Схема возможных переходов между состояниями системы (или процесса) на ($h + 1$)-м шаге
}}}


\vspace*{14pt}

\addtocounter{figure}{1}

\noindent
 ($h\hm+1$)-м шаге ${\sf P}(x, h\hm+1)$ определяется 
суммой вероятностей перехода в~это состояние из состояний ($x\hm-
\varepsilon$): ${\sf P}(x-\varepsilon, h)$ и~$(x+\xi)$: ${\sf P}(x+\xi, h)$, в~которых 
находилась система на шаге~$h$ за вычетом вероятности ${\sf P}(x, h)$ перехода 
системы из состояния~$x$ (в~котором она находилась на шаге~$h$) в~любое 
другое состояние на $(h\hm+1)$-м шаге. В~реальности в~социальной сети 
всегда остается память о предыдущих состояниях. Для учета этого определим 
вероятности ${\sf P}(x\hm- \varepsilon,h)$, ${\sf P}(x\hm+\xi, h)$ и~${\sf P}(x, h)$ через 
состояния на $(h\hm-1)$-м шаге. Схемы соответствующих переходов можно 
изобразить аналогично схеме, представленной на рис.~1, и~получить для 
вероятности перехода следующее алгебраическое уравнение:
  \begin{multline*}
  {\sf P}(x,h+2)={}\\
  {}=\{ {\sf P}(x-2\varepsilon, h)+{\sf P}(x-\varepsilon+\xi, h) -{\sf P}(x-\varepsilon, 
h)\}+{}\\
  {}+ \{ {\sf P}(x+\xi-\varepsilon,h)+ {\sf P}(x+\xi,h) -{\sf P}(x+\xi,h)\} -{}\\
  {}-{\sf P}(x-\varepsilon,h)- {\sf P}(x+\xi, h-1)+ {\sf P}(x,h)\,.
 % \label{e2-hv}
\end{multline*}
Далее, учитывая, что $t\hm=h\tau_0$, перейдем от~$h$ к~$t$, а~затем проведем 
соответствующие разложения в~ряд Тейлора и~получим:
\begin{equation}
\fr{d{\sf P}(x,t)}{dt} =a\fr{ d^2{\sf P}(x,t)}{dx^2} -b\fr{ d{\sf P}(x,t)}{dx} -c\fr{ 
d^2{\sf P}(x,t)}{dt^2}\,,
\label{e3-hv}
\end{equation}
где 
$$
a=\fr{\varepsilon^2-\varepsilon\xi+\xi^2}{\tau_0}\,;\enskip b=\fr{\varepsilon -
\xi}{\tau_0}\,;\enskip c-\tau_0\,.
$$
  
  Член уравнения $d{\sf P}(x,t)/dx$ описывает упорядоченный переход либо 
в~состояние, когда оно увеличивается ($\varepsilon\hm > \xi$), либо когда оно 
уменьшается ($\varepsilon \hm <\xi$); член $d^2{\sf P}(x,t)/dx^2$ описывает 
случайное изменение состояния (неопределенность изменения). Член 
$d{\sf P}(x,t)/dt$ определяет скорость общего изменения состояния системы 
с~течением времени; член $d^2{\sf P}(x,t)/dt^2$ описывает процесс, при котором 
состояния сами становятся источниками возникновения других состояний 
(\textit{самоорганизация} и~ускорение упорядоченных и~случайных переходов).
  
  Сравним полученный результат с~су\-щест\-ву\-ющи\-ми моделями анализа 
и~описания поведения нестационарных временн$\acute{\mbox{ы}}$х рядов. В~настоящее время 
в~качестве аппроксимаций выборочных распределений чаще всего 
используются диффузионные уравнения, включая нелинейную 
диффузию~\cite{18-hv}:
  $$
  \fr{\partial \rho(x,t)}{\partial t} =\fr{D(t)\partial^2\rho^{(n-1)/(n+1)}(x,t)}{\partial x^2}\,,
  $$
  где $n$~--- числовой параметр модели; $(n\hm-1)/(n\hm+1)$~--- показатель 
степени функции плотности распределения~$\rho(x,t)$ (это уравнение 
учитывает только случайные переходы); уравнение Лиувилля~\cite{18-hv}: 
  $$
  \fr{\partial \rho(x,t)}{\partial t}=-\fr{\partial \{ U(x,t)\rho(x,t)\}}{\partial x}\,,
  $$
  которое определяет упорядоченный перенос; уравнение  
Фок\-ке\-ра--План\-ка~\cite{19-hv}: 
  $$
  \fr{\partial\rho(x,t)}{\partial t}=\fr{D(t)}{2}\,\fr{\partial^2\rho(x,t)}{\partial 
x^2} - \fr{\partial \{ U(x,t)\rho(x,t)\}}{\partial x}\,,
  $$
  где $U(x,t)$~--- скорость <<сноса>>; $D(t)$~--- коэффициент диффузии (это 
уравнение учитывает не только случайное изменение (член 
$\partial^2\rho(x,t)/\partial x^2$), но и~упорядоченные переходы (член $\partial 
\{ U(x,t)\rho(x,t)\}/\partial x$), или <<снос>>) и~ряд других. Однако ни одна из 
этих моделей не рассматривает самоорганизацию и~память. Разработанная 
авторами модель обобщает другие модели, и~при равенстве нулю некоторых 
коэффициентов в~уравнении~(\ref{e3-hv}) оно переходит в~уравнения 
Лиувилля или  
Фок\-ке\-ра--План\-ка, которые выступают ее частными случаями.
  
\subsection{Формулировка и~решение краевой~задачи}

  Считая функцию~${\sf P}(x,t)$ непрерывной, можно перейти от вероятности 
${\sf P}(x,t)$ (уравнение~(\ref{e3-hv})) к~плотности вероятности $\rho(x,t)\hm= 
=d{\sf P}(x,t)/dx$ и~сформулировать граничную задачу, решение которой и~будет 
описывать процесс перехода между состояниями. Предположим, что 
необходимо, чтобы доля пользователей (узлов) социальной сети, имеющих 
негативное мнение, не превышала определенного значения (т.\,е.\ величина 
доли негативных настроений должна находиться на отрезке от~0 до величины 
порога перколяции~$l$ для данной сети).
  
  \textbf{Первое граничное условие}. Состояние $x\hm=0$ определяет полное 
отсутствие негативных мнений (доля равна~0). Сама вероятность обнаружить 
такое состояние может быть отлична от~0, однако плотность вероятности, 
определяющую поток в~состоянии $x\hm=0$, необходимо положить равной~0 
(состояния системы не могут быть отрицательными): $\rho(x,t)_{x=0}\hm=0$.
  
  \textbf{Второе граничное условие}. Вероятность обнаружить состояние 
с~максимально возможной долей негативно настроенных пользователей 
$x\hm=L\hm=1$ отлична от~0. Однако плотность вероятности, определяющая 
поток в~этом состоянии, необходимо положить равной~0 (величина состояния 
не может быть больше, чем максимально возможная доля): 
$\rho(x,t)_{x=L}\hm=0$.
  
  Поскольку в~момент времени $t\hm=0$ состояние системы уже может быть 
равно некоторому значению~$x_0$, то начальное условие зададим в~виде:
  \begin{multline*}
  \rho(x,t=0)=\delta(x-x_0)= {}\\
  {}=\begin{cases} 
  \displaystyle \int \delta(x-x_0)\,dx=1\,, &x=x_0\,;\\
  0\,, & x\not= x_0\,.
  \end{cases}
  \end{multline*}
  
  Второе начальное условие можно задать в~виде: 
  $$
  \fr{\partial G(x,t)}{\partial t}\Big\vert_{t=0}=0\,,
  $$ 
  так как начальное условие содержит дель\-та-функ\-цию; кроме того, ее 
наличие приводит к~тому, что решение для $\rho(x,t)$ разбивается на две 
области при $x\hm> x_0$ и~при $x\hm\leq x_0$. Используя методы 
операционного исчисления для плотности вероятности $\rho_1(x,t)$ 
и~$\rho_2(x,t)$ обнаружения состояния системы в~одном из значений на отрезке 
от~0 до~$L$, можно получить следующую систему уравнений:
  \begin{multline*}
  \mbox{при } x\geq x_0:\ \rho_1(x,t)= -\fr{2}{L}e^{-t/(2\tau_0)} e^{k(x-x_0)}\times{}\\
  {}\times \sum\limits^\infty_{n=1} \fr{\sin ( \pi 
n x_0/L) \sin (\pi n (L-x)/L)}{\cos(\pi n)}\times{}\\
{}\times \mathrm{ch}\left( 
\fr{t}{\tau_0}\sqrt{\fr{k\varepsilon\xi}{2(\varepsilon-\xi)}- \fr{\pi^2 n^2 
(\varepsilon-\xi)}{2kL^2}}\right)\,;
  \end{multline*}
  
  \vspace*{-12pt}
  
  \noindent
  \begin{multline*}
  \mbox{при } x < x_0:\ \rho_2(x,t)= -\fr{2}{L}e^{-t/(2\tau_0)} e^{k(x-x_0)} \times{}\\
  {}\times \sum\limits^\infty_{n=1} \fr{\sin (\pi 
n (L-x_0)/L)\sin (\pi n x/L)}{\cos (\pi n)}\times{}\\
{}\times \mathrm{ch} 
\left(\fr{t}{\tau_0}\sqrt{\fr{k\varepsilon\xi}{2(\varepsilon-\xi)}- \fr{\pi^2 n^2 
(\varepsilon-\xi)}{2kL^2}}\right)\,,
  \end{multline*}
где 
$$
k=\fr{\varepsilon -\xi}{2(\varepsilon^2 -\varepsilon\xi +\xi^2)}\,.
$$ 
Если вычислить интеграл 
\begin{equation}
{\sf P}(l,t)=\int\limits_0^{x_0} \rho_2(x,t)\,dx+\int\limits^l_{x_0}\rho_1(x,t)\,dx\,,
\label{e4-hv}
\end{equation}
то функция ${\sf P}(l,t)$ будет задавать вероятность того, что состояние системы 
к~моменту времени~$t$ будет находиться на отрезке от~0 до~$l$, т.\,е.\ порог 
перколяции~$l$ не будет достигнут. Соответственно, вероятность~$Q_i(t)$ того, 
что порог перколяции~$l$ окажется к~моменту времени~$t$ достигнутым или 
превзойденным, будет равна:
\begin{equation}
Q(l,t)=1-{\sf P}(l,t)\,.
\label{e5-hv}
\end{equation}

\vspace*{-12pt}

\subsection{Моделирование динамики и~самоорганизации состояний узлов~социальной сети}

  При анализе модели необходимо задать приемлемые величины значений 
порогов перколяции случайной сети. Плотность связей можно определить 
экспериментально, а затем, используя зависимости величины порогов 
перколяции от среднего числа связей, приходящегося на один  
узел~~\cite{11-hv, 12-hv, 13-hv, 14-hv, 15-hv, 16-hv, 17-hv}, рассчитать их 
допустимые величины (см.\ уравнения в~разд.~4).
  
  Для моделирования примем, что начальная доля~$x_0$ негативных мнений 
равна~5\% ($x_0\hm=0{,}05$), величину~$\tau_0$ примем равной одной 
условной единице времени ($\tau_0\hm =1$), $\varepsilon\hm=0{,}02$ (2\%) 
и~$\xi\hm=0{,}01$ (1\%). Результаты моделирования времени достижения 
порога перколяции~(\ref{e5-hv}) с~использованием~(\ref{e4-hv}) при заданном 
выше в~качестве примера наборе параметров модели представлены 
в~графическом виде на рис.~2.  Кривые~\textit{1} и~\textit{2} на рис.~2 показывают, что чем 
ближе значение величины начального со\-сто\-яния сис\-те\-мы~$x_0$ в~момент 
времени $t\hm=0$ к~пороговому значению, тем быстрее возрастает вероятность 
перехода и~тем сильнее вероятность его достижения приближается к~единице. 
Кривая~\textit{4} на рис.~2 показывает, что при большой разности между 
величиной порогового значения и~$x_0$ вероятность его достижения имеет 
осциллирующий характер, при этом она сначала снижается с~течением времени, 
а затем показывает рост, причем чем дальше значение величины~$x_0$ от 
порогового значения, тем сильнее проявляются осцилляции. Кроме того, 
существует и~отличная от нуля вероятность достижения порогового значения 
при $t\hm=0$ (мгновенная реализация).
  

\setcounter{figure}{1}
\begin{figure*} %fig2
\vspace*{1pt}
\begin{center}
\mbox{%
\epsfxsize=163mm
\epsfbox{hva-2.eps}
}
\end{center}
\vspace*{-14pt}
\begin{minipage}[t]{79mm}
\Caption{Графическое представление результатов моделирования преодоления порогов 
перколяции для распространения негативных мнений в~социальной сети: \textit{1}~--- $l\hm=0{,}1$; \textit{2}~--- 0,15;
\textit{3}~--- 0,20; \textit{4}~--- $l\hm=0{,}25$}
\end{minipage}
\hfill
%\end{figure*}
%\begin{figure*} %fig3
%\vspace*{1pt}
%\begin{center}
%\mbox{%
%\epsfxsize=77.502mm
%%\epsfbox{hva-3.eps}
%}
%\end{center}
\vspace*{-14pt}
\begin{minipage}[t]{79mm}
\Caption{Графическое представление результатов моделирования преодоления порогов 
перколяции для распространения негативных мнений в~социальной сети при $\varepsilon\hm=\xi\hm= 0{,}02$:
\textit{1}~--- $l\hm=0{,}1$; \textit{2}~--- 0,15; \textit{3}~--- 0,20; \textit{4}~--- $l\hm=0{,}25$}
\end{minipage}
\vspace*{6pt}
\end{figure*}

  Рост вероятности перехода через пороговое значение имеет ступенчатый 
характер, а~про\-тя\-жен\-ность ступени во времени зависит от того, насколько 
начальная величина со\-сто\-яния сис\-те\-мы~$x_0$
 \mbox{близка} к~пороговому значению. 
Процесс достижения порогового значения имеет протяженное во времени 
плато, величина которого (в~единицах ве\-ро\-ят\-ности) зависит от~$x_0$.
  
  Увеличение значения величин ~$\varepsilon$ и~$\xi$ (при выполнении 
условия $\varepsilon \hm >\xi$) изменяет величину плато (горизонтальный 
участок зависимости ве\-ро\-ят\-ности перехода через пороговое значение до 
\mbox{второго} участка резкого рос\-та) на рис.~2, однако общая за\-ви\-си\-мость 
ве\-ро\-ят\-ности перехода от времени качественно не изменяется.
  
  Ход кривых на рис.~2 показывает возможность роста вероятности 
достижения порогового значения состояния системы практически сразу после 
начала процесса. Вероятность перехода через пороговое значение отлична от 
нуля уже после первого шага и~нелинейно возрастает с~течением времени. Это 
следствие того, что не только величины~$\varepsilon$ и~$\xi$ определяют 
изменение состояния~$x$, но и~сами состояния~$x$ служат источником изменения 
вследствие наличия памяти о предыдущих состояниях и~самоорганизации, за 
которую отвечает в~дифференциальном уравнении член $d^2 {\sf P}(x,t)/dt^2$. 
Арифметический расчет показывает, что число шагов (обозначим его 
как~$q_0$), за которое можно достичь порогового значения~$l$, должно быть 
не меньше чем $q\hm=(l\hm - x_0)/(\varepsilon \hm - \xi)$. Например, для 
пороговых значений состояния системы $l\hm=0{,}1$ и~0,2 при ее начальном 
состоянии $x_0\hm=0{,}05$, $\varepsilon\hm=0{,}02$ и~$\xi\hm=0{,}01$ 
для~$q$ получим соответственно~5 и~15. Результаты (см.\ рис.~2) показывают, 
что это не так, т.\,е.\ происходит самоорганизация.
  
  При равенстве~$\varepsilon$ и~$\xi$ (например, 
$\varepsilon\hm=\xi\hm=0{,}02$)\linebreak характер хода кривых, описывающих 
вероятность достижения пороговых значений, изменяется (рис.~3).  
В~част\-ности, не наблюдается протяженного во времени плато 
с~последующим плавным ростом вероятности достижения пороговых значений 
до единицы, а рост вероятности имеет характер резкого скачка. Это связано 
с~тем, что коэффициент~$b$ в~уравнении~(\ref{e3-hv}) окажется равен нулю 
и~упорядоченные переходы будут невозможны, а член $d^2\rho(x,t)/dt^2$ будет 
ускорять только случайные переходы $d^2\rho(x,t)/dx^2$.

\vspace*{-4pt}

\section{Алгоритм мониторинга состояний социальной сетевой~структуры}

\vspace*{-2pt}

  Разработанная модель позволяет создать практически реализуемый алгоритм 
мониторинга состояния социальной сети.\\[-12pt]
\begin{enumerate}[1.]
\item Определяем с~помощью социологического мониторинга плотность сети 
и~долю узлов~$x_0$ с~определенным мнением или настроением (состояние 
узла) в~данный момент времени $t\hm=0$.\\[-15pt]
\item Спустя одну выбранную условную единицу времени $\tau\hm=1$ 
(например, одна неделя) снова находим долю узлов с~определенным мнением 
или настроением в~данный момент времени~$x_1$ ($t\hm=0\hm +\tau$). 
Находим величину $\varepsilon\hm=x_1 \hm- x_0$, а~величину~$\xi$ считаем 
равной~0. Если $\varepsilon\hm<0$, то считаем $\xi\hm= x_1 \hm- x_0$, 
а~$\varepsilon\hm=0$.\\[-15pt]
\item На основании данных о~среднем числе связей рассчитываем порог 
перколяции данной сети. Используя уравнения~(\ref{e4-hv}) и~(\ref{e5-hv}), по 
определенным в~пп.~1 и~2 значениям параметров $x_0$, $\varepsilon$, 
$\xi$ и~порогу перколяции~$l$ моделируем поведение от условного времени 
вероятности перехода поро-\linebreak\vspace*{-12pt}
\end{enumerate}

\begin{enumerate}
\item[\,]
га перколяции и~определяем допустимый лимит 
времени для изменения ситуации.
\end{enumerate}

\vspace*{-6pt}

\section{Заключение}

  Создана новая модель описания динамики изменения состояния узлов 
и~перколяционных переходов в~социальных сетях с~учетом самоорганизации 
и~наличия памяти, которая вносит \mbox{значительный} вклад в~развитие тео\-рии 
управления сложными системами.
  
  Результаты анализа разработанной модели могут быть связаны 
с~полученными ранее результатами описания процессов в~социальных сетевых 
структурах с~помощью теории перколяции (это необходимо для определения 
времени достижения пороговых значений доли узлов социальной сети, когда 
определенные мнения или предпочтения могут беспрепятственно 
распространяться по сети в~целом).
  
  Полученные результаты существенно отличаются от применяемых 
в~настоящее время моделей для описания нестационарных процессов на 
основе тео\-рии хаоса, диффузионных подходов, уравнений Лиувилля  
и~Фок\-ке\-ра--План\-ка. Все это в~целом представляется абсолютно новым 
и~оригинальным, а также вносит вклад в~развитие теории управления 
сложными системами.
  
  Разработанная модель во взаимосвязи с~методами теории перколяции 
существенно расширяет возможности применения уравнений математической 
физики и~теоретической информатики для моделирования социальных систем.

\vspace*{-6pt}
  
{\small\frenchspacing
{\baselineskip=10.85pt
%\addcontentsline{toc}{section}{References}
\begin{thebibliography}{99}
\bibitem{1-hv}
\Au{Gasser L.} Social conceptions of knowledge and action: DAI foundations and open system 
semantics~// Artif. Intell., 1991. Vol.~47. No.\,1-3. P.~107--138.
\bibitem{2-hv}
\Au{Jennings N.\,R., Faratin~P., Lomuscio~A.\,R., Parsons~S., Sierra~C., Wooldridge~M.} 
Automated negotiation: Prospects, methods and challenges~// Group Decis.
Negot., 2001. Vol.~10. No.\,2. P.~199--215.
\bibitem{3-hv}
\Au{Plikynas D., Raudys~A., Raudys~S.} Agent-based modelling of excitation propagation in 
social media groups~// J.~Exp. Theor. Artif. In., 2015. Vol.~27. No.\,4. 
P.~373--388.

\bibitem{5-hv} %4
\Au{Hay J., Flynn D.} The effect of network structure on individual behavior~// Complex Systems, 
2014. Vol.~23. No.\,4. P.~295--311.

\bibitem{4-hv} %5
\Au{Hay J., Flynn D.} How external environment and internal structure change the behavior of 
discrete systems~// Complex Systems, 2016. Vol.~25. No.\,1. P.~39--49.

\bibitem{6-hv}
\Au{Airoldi~E.\,M., Blei~D.\,M., Fienberg~S.\,E., Xing~E.\,P.} Mixed membership stochastic 
blockmodels~// J.~Mach. Learn. Res, 2008. Vol.~9. P.~1981--2014.
\bibitem{7-hv}
\Au{Бокс Дж., Дженкинс~Г.} Анализ временных рядов. Прогноз и~управление~/ Пер. 
с~англ.~--- М.: Мир, 1974. 553~с. (\Au{Box~G.\,E.\,P., Jenkins~G.\,M.} Time series analysis: 
Forecasting and control.~--- Holden-day, 1970. 553~p.)
\bibitem{8-hv}
\Au{Гнеденко Б.\,В.} Курс теории вероятностей.~--- М.: Физматлит, 1961. 406~с.
\bibitem{9-hv}
\Au{Grimmet G.\,R.} Percolation.~--- New York, NY, USA: Springer-Verlag, 1989. 296~p.

\bibitem{13-hv} %10
\Au{Zhukov D., Lesko S.} Percolation models of information dissemination in social networks~// 
IEEE Conference (International) on Smart City/SocialCom/SustainCom 
Together with DataCom Proceedings.~--- 
IEEE, 2015. P.~213--216.
\bibitem{14-hv} %11
\Au{Khvatova T., Block~M., Zhukov~D., Lesko~S.} Studying the structural topology of the 
knowledge sharing network~// 11th European Conference on Management Leadership and 
Governance Proceedings.~--- Lisbon, Portugal: Academic Conferences and Publishing 
Itnternational Ltd., 2015. P.~20--27.

\bibitem{12-hv} %12
\Au{Khvatova T.\,Yu., Zaltsman~A.\,D., Zhukov~D.\,O.} Information processes in social networks: 
Percolation and stochastic dynamics~// CEUR Workshop Procee., 2017. Vol.~2064. 
P.~277--288.

\bibitem{11-hv} %13
\Au{Zhukov D., Khvatova~T., Lesko~S., Zaltsman~A.} Managing social networks: Applying 
percolation theory methodology to understand individuals' attitudes and moods~// Technol. 
Forecast. Soc., 2018. Vol.~129. P.~297--307.


\bibitem{10-hv} %14
\Au{Лесько С.\,А., Алёшкин~.\,С., Филатов~В.\,В.} Стохастические и~перколяционные 
модели динамики блокировки вычислительных сетей при распространении эпидемий 
эволюционирующих компьютерных вирусов~// Российский технологический~ж., 2019. 
Т.~7. №\,3(29). С.~7--27.

\bibitem{16-hv} %15
\Au{Khvatova T., Block~M., Zhukov~D., Lesko~S.} How to measure trust: The percolation model 
applied to intraorganisational knowledge sharing networks~// J.~Knowl. Manag., 2016. Vol.~20. 
No.\,5. P.~918--935.

\bibitem{15-hv} %16
\Au{Жуков Д.\,О., Хватова~Т.\,Ю., Лесько~С.\,А., Зальцман~А.\,Д.} Влияние плотности 
связей на кластеризацию и~порог перколяции при распространении информации 
в~социальных сетях~// Информатика и~её применения, 2018. Т.~12. Вып.~2. С.~90--97.


\bibitem{17-hv}
\Au{Zhukov D.\,O., Khvatova~T.\,Y., Millar~C., Zaltcman~A.} Modelling the stochastic dynamics 
of transitions between states in social systems incorporating self-organisation and memory~// 
Technol. Forecast. Soc., 2020. Vol.~158. Art. ID: 120134.
\bibitem{18-hv}
\Au{Орлов Ю.\,Н., Федоров~С.\,Л.} Генерация нестационарных траекторий временного ряда 
на основе уравнения Фок\-ке\-ра--План\-ка~// Труды МФТИ, 2016. Т.~8. №\,2(30).  
С.~126--133.
\bibitem{19-hv}
\Au{Fuentes M.} Non-linear diffusion and power law properties of heterogeneous systems: 
Application to financial time series~// Entropy, 2018. Vol.~20. Iss.~9. Art. No.~649.
\end{thebibliography}

}
}

\end{multicols}

\vspace*{-10pt}

\hfill{\small\textit{Поступила в~редакцию 18.06.2019}}

%\vspace*{8pt}

%\pagebreak

\newpage

\vspace*{-28pt}

%\hrule

%\vspace*{2pt}

%\hrule

%\vspace*{-2pt}

\def\tit{MODELING OF~THE~STOCHASTIC DYNAMICS OF~CHANGES IN~NODE STATES AND~PERCOLATION 
TRANSITIONS IN~SOCIAL NETWORKS WITH~SELF-ORGANIZATION AND~MEMORY}

\def\titkol{Modeling of the~stochastic dynamics of~changes in~node states and~percolation 
transitions in~social networks} % with~self-organization and~memory}

\def\aut{D.\,O.~Zhukov$^1$, T.\,Yu.~Khvatova$^2$, and~A.\,D.~Zaltcman$^1$}

\def\autkol{D.\,O.~Zhukov, T.\,Yu.~Khvatova, and~A.\,D.~Zaltcman}

\titel{\tit}{\aut}{\autkol}{\titkol}

\vspace*{-11pt}


\noindent
$^1$Russian Technological University (MIREA), 78~Vernadskogo Ave., Moscow 119454, Russian Federation

\noindent
$^2$Peter the Great St.\ Petersburg Polytechnic University, 29~Polytechnicheskaya Str., St.\ Petersburg 195251, 
Russian\linebreak
$\hphantom{^1}$Federation

\def\leftfootline{\small{\textbf{\thepage}
\hfill INFORMATIKA I EE PRIMENENIYA~--- INFORMATICS AND
APPLICATIONS\ \ \ 2021\ \ \ volume~15\ \ \ issue\ 1}
}%
\def\rightfootline{\small{INFORMATIKA I EE PRIMENENIYA~---
INFORMATICS AND APPLICATIONS\ \ \ 2021\ \ \ volume~15\ \ \ issue\ 1
\hfill \textbf{\thepage}}}

\vspace*{3pt}





\Abste{This paper explores the use of theoretical informatics applied for analyzing and modeling 
the processes in sociotechnical systems (social networks). A~stochastic model of users' (network nodes) dynamic changes of states (opinions 
or moods) and the percolation threshold in a~social network with random connections among nodes was developed. 
This model demonstrates the opportunity for jump-like transitions in states (opinions, moods) of the nodes in a~social 
network over a short period of time without external influence. While developing the model, the probabilistic schemes 
of state-to-state transitions of nodes (users having certain opinions and views) were considered; 
a~second-order  
nonlinear differential equation was derived; the boundary for calculating the probability density function for a~system 
being in a certain state depending on the time interval was formulated. The differential equation of the model contains 
a~member representing the opportunity for self-organization; it also considers the presence of memory. The results of 
analysis of the stochastic model support those previously obtained by the authors when investigating social network 
processes using the percolation theory. This theory was used for defining the time of reaching the threshold values for 
the share of social network nodes when certain opinions or preferences can spread freely within the whole social 
network.}

\KWE{stochastic dynamics; states of social network nodes; system self-organization; processes involving memory; 
percolation in social networks}

\DOI{10.14357/19922264210114}

%\vspace*{-15pt}

%\Ack
%\noindent

%\vspace*{6pt}

  \begin{multicols}{2}

\renewcommand{\bibname}{\protect\rmfamily References}
%\renewcommand{\bibname}{\large\protect\rm References}

{\small\frenchspacing
 {%\baselineskip=10.8pt
 \addcontentsline{toc}{section}{References}
 \begin{thebibliography}{99}

\bibitem{1-hv-1}
\Aue{Gasser, L.} 1991. Social conceptions of knowledge and action: DAI foundations and open 
system semantics. \textit{Artif. Intell.} 47(1-3):107--138.
\bibitem{2-hv-1}
\Aue{Jennings, N.\,R., P.~Faratin, A.\,R.~Lomuscio, S.~Parsons, C.~Sierra, and M.~Wooldridge}. 
2001. Automated negotiation: Prospects, methods and challenges. \textit{Group Decis. 
Negot.} 10(2):199--215.
\bibitem{3-hv-1}
\Aue{Plikynas, D., A.~Raudys, and S.~Raudys.} 2015. Agent-based modelling of excitation 
propagation in social media groups. \textit{J.~Exp. Theor. Artif. In.} 
27(4):373--388.

\bibitem{5-hv-1}
\Aue{Hay, J., and D.~Flynn.} 2014. The effect of network structure on individual behavior. 
\textit{Complex Systems} 23(4):295--311.

\bibitem{4-hv-1}
\Aue{Hay, J., and D.~Flynn.} 2016. How external environment and internal structure change the 
behavior of discrete systems. \textit{Complex Systems} 25(1):39--49.

\bibitem{6-hv-1}
\Aue{Airoldi, E.\,M., D.\,M.~Blei, S.\,E.~Fienberg, and E.\,P.~Xing.} 2008. Mixed membership 
stochastic blockmodels. \textit{J.~Mach. Learn. Res.} 9:1981--2014.
\bibitem{7-hv-1}
\Aue{Box, G.\,E.\,P., and G.\,M.~Jenkins.} 1970. \textit{Time series analysis: Forecasting and 
control.} Holden-day. 553~p.
\bibitem{8-hv-1}
\Aue{Gnedenko, B.\,V.} 1961. \textit{Kurs teorii veroyatnostey} [Probability theory]. Moscow: Fizmatlit. 406~p.
\bibitem{9-hv-1}
\Aue{Grimmet, G.\,R.} 1989. \textit{Percolation}. New York, NY: Springer-Verlag. 296~p.

\bibitem{13-hv-1} %10
\Aue{Lesko, S.\,A., and D.\,O.~Zhukov.} 2015. Percolation models of information dissemination in 
social networks. \textit{IEEE Conference (International) on Smart City/SocialCom/\linebreak SustainCom 
Together with DataCom Proceedings}. IEEE. 213--216.
\bibitem{14-hv-1} %11
\Aue{Block, M., T.~Khvatova, D.~Zhukov, and S.~Lesko.} 2015. Studying the structural topology 
of the knowledge sharing network. \textit{11th European Conference on Management, Leadership 
and Governance Proceedings}. Lisbon, Portugal: Academic Conferences and Publishing 
International Ltd. 20--27.

\bibitem{12-hv-1} %12
\Aue{Khvatova, T.\,Yu., A.\,D.~Zaltcman, and D.\,O.~Zhukov.} 2017. Information processes in 
social networks: Percolation and stochastic dynamics. CEUR Workshop Procee.
2064:277--288.

\bibitem{11-hv-1} %13
\Aue{Zhukov, D., T.~Khvatova, S.~Lesko, and A.~Zaltcman.} 2018. Managing social networks: 
Applying Percolation theory methodology to understand individuals' attitudes and moods. 
\textit{Technol. Forecast. Soc.} 129:297--307.


\bibitem{10-hv-1} %14
\Aue{Lesko, S.\,A., A.\,S.~Alyoshkin, and V.\,V.~Filatov.} 2019. Stokhasticheskie 
i~perkolyatsionnye modeli dinamiki blokirovki vychislitel'nykh setey pri rasprostranenii epidemiy 
evolyutsioniruyushchikh komp'yuternykh virusov [Stochastic and percolating models of blocking 
computer networks dynamics during distribution of epidemics of evolutionary computer viruses]. 
\textit{Rossiyskiy tekh\-no\-lo\-gi\-che\-skiy zh.} [Russian Technological~J.] 7(3):7--27.


\bibitem{16-hv-1} %15
\Aue{Khvatova, T., M.~Block, D.~Zhukov, and S.~Lesko.} 2016. How to measure trust: The 
percolation model applied to intraorganisational knowledge sharing networks. \textit{J.~Knowl. 
Manag.} 20 (5):918--935.

\bibitem{15-hv-1} %16
\Aue{Zhukov, D.\,O., T.\,Yu.~Khvatova, S.\,A.~Les'ko, and A.\,D.~Zal'tsman.} 2018. Vliyanie 
plotnosti svyazey na klasterizatsiyu i~porog perkolyatsii pri rasprostranenii informatsii 
v~sotsial'nykh setyakh [The influence of the connections density on clusterization and percolation 
threshold during information distribution in social networks]. \textit{Informatika i~ee 
Primeneniya~---Inform. Appl.} 12(2):90--97.

\bibitem{17-hv-1}
\Aue{Zhukov, D.\,O., T.\,Y.~Khvatova, C.~Millar, and A.~Zaltcman.} 2020. Modelling the 
stochastic dynamics of transitions between states in social systems incorporating self-organisation 
and memory. \textit{Technol. Forecast. Soc.} 158:120134.
\bibitem{18-hv-1}
\Aue{Orlov, Yu.\,N., and S.\,L.~Fedorov.} 2016. Generatsiya ne\-sta\-tsio\-nar\-nykh traektoriy 
vremennogo ryada na osnove uravneniya Fokkera--Planka [Generating nonstationary trajectories 
of a time series based on Fokker--Plank equation]. \textit{Trudy MFTI} [MIPT Proceedings~J.] 
8(2):126--133.
\bibitem{19-hv-1}
\Aue{Fuentes, M.} 2018. Non-linear diffusion and power law properties of heterogeneous systems: 
Application to financial time series. \textit{Entropy} 20(9):649. 8~p.
\end{thebibliography}

 }
 }

\end{multicols}

\vspace*{-3pt}

  \hfill{\small\textit{Received June~18, 2019}}


%\pagebreak

%\vspace*{-8pt}

\Contr

\noindent
\textbf{Zhukov Dmitry O.} (b.\ 1965)~--- 
Doctor of Science in technology, professor, Head of Department, Russian Technological 
University (MIREA), 78~Vernadskogo Ave., Moscow 119454, Russian Federation; 
\mbox{zhukov\_do@mirea.ru}

\vspace*{6pt}

\noindent
\textbf{Khvatova Tatiana Yu.} (b.\ 1971)~--- Doctor of Science in economics, professor, Peter the 
Great St.\ Petersburg Polytechnic University, 29~Polytechnicheskaya Str., St.\ Petersburg 195251, 
Russian Federation; \mbox{khvatova.ty@spbstu.ru}

\vspace*{6pt}

\noindent
\textbf{Zaltcman Anastasia D.} (b.\ 1989)~--- lecturer, Russian Technological University 
(MIREA), 78~Vernadskogo Ave., Moscow 119454, Russian Federation; 
\mbox{ad.zaltcman@gmail.com}

\label{end\stat}

\renewcommand{\bibname}{\protect\rm Литература} 