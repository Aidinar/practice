%есть таблицы и рисунки

\def\stat{sokolov}

\def\tit{ПОВЫШЕНИЕ СБОЕУСТОЙЧИВОСТИ\\ САМОСИНХРОННЫХ СХЕМ$^*$}

\def\titkol{Повышение сбоеустойчивости самосинхронных 
схем}

\def\aut{И.\,А.~Соколов$^1$, Ю.\,А.~Степченков$^2$, Ю.\,Г.~Дьяченко$^3$, 
Ю.\,В.~Рождественский$^4$}

\def\autkol{И.\,А.~Соколов, Ю.\,А.~Степченков, Ю.\,Г.~Дьяченко, 
Ю.\,В.~Рождественский}

\titel{\tit}{\aut}{\autkol}{\titkol}

\index{Соколов И.\,А.}
\index{Степченков Ю.\,А.}
\index{Дьяченко Ю.\,Г.}
\index{Рождественский Ю.\,В.}
\index{Sokolov I.\,A.}
\index{Stepchenkov Yu.\,A.}
\index{Diachenko Yu.\,G.}
\index{Rogdestvenski Yu.\,V.}


{\renewcommand{\thefootnote}{\fnsymbol{footnote}} \footnotetext[1]
{Исследование выполнено при финансовой поддержке Минобрнауки (проект 075-15-2020-799).}}


\renewcommand{\thefootnote}{\arabic{footnote}}
\footnotetext[1]{Федеральный исследовательский центр <<Информатика и управление>> 
Российской академии наук, \mbox{ISokolov@ipiran.ru}}
\footnotetext[2]{Федеральный исследовательский центр <<Информатика и управление>> 
Российской академии наук, \mbox{YStepchenkov@ipiran.ru}}
\footnotetext[3]{Федеральный исследовательский центр <<Информатика и управление>> 
Российской академии наук, \mbox{diaura@mail.ru}}
\footnotetext[4]{Федеральный исследовательский центр <<Информатика и управление>> 
Российской академии наук, \mbox{YRogdest@ipiran.ru}}


\vspace*{-12pt}
  
  \Abst{Анализируется проблема устойчивости самосинхронных (СС) схем, изготовленных 
по технологии комплементарный ме\-талл--ди\-элект\-рик--по\-лу\-про\-вод\-ник (КМДП), 
к~кратковременным логическим сбоям (ЛС), генерируемым внешними воздействиями: ядерными 
частицами, космическими лучами, электромагнитными наводками.\ Практические СС-схе\-мы 
реализуются в виде конвейера с за\-прос-от\-вет\-ным взаимодействием между его ступенями 
и двухфазной дисциплиной работы с чередованием рабочей фазы и спейсера.\ 
Комбинационная часть ступени конвейера использует парафазное со спейсером кодирование 
информационных сигналов.\ Индикаторная подсхема ступени конвейера подтверждает 
окончание переключения всех элементов ступени, возбужденных в~текущей фазе работы, 
и~формирует сигналы управ\-ле\-ния за\-прос-от\-вет\-ным взаимодействием ступеней конвейера.\ 
Рассмотрены физические причины появления ЛС и~проанализированы типы 
сбоев, возможных в~КМДП-СС-схе\-мах с~проектными нормами~65~нм и~ниже.\ 
Сравниваются характеристики сбоеустойчивости разных вариантов СС-ре\-гист\-ров 
хранения.\ Предлагаются схемотехнические и~топологические методы повышения 
сбоеустойчивости СС-кон\-вей\-ера.\ Даются оценки  
сбое\-устой\-чи\-вости СС-кон\-вей\-ера в~зависимости от места появления~ЛС.}
  
  \KW{самосинхронная схема; сбоеустойчивость; конвейер; рабочая фаза; спейсер}

\DOI{10.14357/19922264200409} 
  
%\vspace*{9pt}


\vskip 10pt plus 9pt minus 6pt

\thispagestyle{headings}

\begin{multicols}{2}

\label{st\stat}

\section{Введение}

  Широкое использование цифровых микросхем в условиях неблагоприятной 
радиационной обстановки и переход к субмикронным технологиям их 
изготовления сделали актуальной задачу повышения устойчивости цифровых 
микросхем к~ЛС. Логический сбой~--- это изменение 
уровня сигнала в узле схемы из-за кратковременной причины~--- пролета 
через тело полупроводника мик\-ро\-схе\-мы одинокой ядерной частицы (ЯЧ), 
мощного электромагнитного импульса, сильной помехи по шинам питания 
или по сигнальным линиям и~т.\,д.
  
  В комбинационных схемах, находящихся в статическом состоянии, ЛС 
прекращается сам собой по окончании действия физической причины его 
появления. Но в конвейерных синхронных схемах даже кратковременный ЛС 
может успеть записаться в выходной регистр и испортить результат 
обработки данных. Повышение быстродействия цифровых схем усугубляет 
эту проблему. 
  
  Элементы с памятью (триггеры, ячейки памяти) более чувствительны к ЛС, 
поскольку ЛС в~них может инвертировать хранимый бит данных, который 
самостоятельно не восстановится после исчезновения причины сбоя.
  
  Самосинхронные схемы~[1] обладают высокой устойчивостью к 
ЛС~[2] за счет избыточного кодирования информационных сигналов и зап\-рос-от\-вет\-ной дисциплины взаимодействия функциональных СС-бло\-ков. 
Однако индикаторная часть СС-схе\-мы традиционно реализуется на 
элементах с па\-мятью~--- гистерезисном триггере 
  (Г-триг\-ге\-ре~[1]). Кроме того, практические СС-схе\-мы имеют 
конвейерную реализацию, аналогично синхронным аналогам, с регистром в 
каждой ступени \mbox{конвейера}. В~результате уровень сбоеустойчивости СС-схем 
зависит и от устойчивости  
к~ЛС Г-триг\-ге\-ра и разряда регистра хранения данных.
       Поэтому задача повышения устойчивости СС-кон\-вей\-ера 
     к~кратковременным одиночным ЛС является актуальной. 

Данная статья 
анализирует естественную сбоеустойчивость СС-кон\-вей\-ера в~КМДП-базисе 
с проектными нормами не более 65~нм и~предлагает схемотехнические 
методы ее повышения.

 \begin{figure*}[b] %fig1
\vspace*{1pt}
    \begin{center}  
  \mbox{%
 \epsfxsize=122.326mm 
 \epsfbox{dya-1.eps}
 }
\end{center}
\vspace*{-11pt}
\Caption{Схема СС-конвейера}
\end{figure*}
   
   
   \vspace*{-4pt}
   

\section{Типы логических сбоев в~самосинхронных схемах}

      \vspace*{-4pt}

  Физической причиной ЛС в КМДП-схе\-мах служит индуцирование 
избыточных носителей заряда\linebreak\vspace*{-12pt}

\pagebreak

%\begin{table*}\small %tabl1
\begin{center}

\parbox{73mm}{{{\tablename~1}\ \ \small{
Возможные изменения ПФС сигнала из-за~ЛС
}}}


\vspace*{2ex}

{\small \begin{tabular}{|c|c|c|c|c|}
\hline
№ 
&\multicolumn{2}{c|}{Спейсер <<00>>}&\multicolumn{2}{c|}{Спейсер <<11>>}\\
\cline{2-5}
п/п&До ЛС&После ЛС&До ЛС&После ЛС\\
\hline
1&00&01&11&01\\
2&00&10&11&10\\
3&00&11&11&00\\
4&01&11&01&00\\
5&10&11&10&00\\
6&01&00&01&11\\
7&10&00&10&11\\
\hline
\end{tabular}}
\end{center}
%\end{table*}

\vspace*{12pt}

\noindent
 (элект\-рон\-но-ды\-роч\-ных пар) в теле 
полупроводника и сигнальных трассах из-за внешнего воздействия или 
сильных помех. Под действием электрического поля электроны и дырки в 
полупроводнике разлетаются в противоположных направлениях, порождая 
ток ионизации (ТИ) в узле схемы. В~первом приближении импульс ТИ 
описывается формулой~[3]:
  \begin{equation*}
  I_{\mathrm{ТИ}}(t)=\fr{Qk}{\tau_{\mathrm{сп}}-\tau_{\mathrm{н}}} \left( e^{-
t/\tau_{\mathrm{сп}}} - e^{-t/\tau_{\mathrm{н}}}\right)\,,
  %\label{e1-step}
  \end{equation*}
где $Q$~--- интегральный заряд, образовавшийся в~объеме полупроводника; 
$\tau_{\mathrm{н}}$ и~$\tau_{\mathrm{сп}}$~--- постоянные времени нарастания 
и спада импульса ТИ; $k$~--- коэффициент, характеризующий часть общего 
заряда~$Q$, попавшего в данный узел схемы. Интегральный заряд~$Q$ 
оценивается по формуле:
\begin{equation*}
Q=\fr{q\rho L_{\mathrm{тр}} \theta \mathrm{LET}}{E_{\mathrm{eh}}}\,,
%\label{e2-step}
\end{equation*}
где $q$~--- заряд электрона; $\rho$~--- плотность полупроводника; 
$L_{\mathrm{тр}}$~--- длина трека ЯЧ в полупроводнике; $\theta$~--- угол 
падения частицы; $\mathrm{LET}$~--- потери энергии частицы; $E_{\mathrm{eh}}$~--- энергия 
образования элект\-рон\-но-ды\-роч\-ной пары.


  
  В наихудшем случае ТИ приводит к кратковременной инверсии 
логического уровня на выходе сбойного элемента на время рассасывания 
заряда~$Q$, но следующая часть 
  СС-схе\-мы может воспринять ЛС и зафиксировать в регистре. 
  
  Сильные помехи по сигнальным линиям и шинам питания и земли за счет 
паразитных емкостных связей наводят положительные или отрицательные 
импульсы напряжения на соседние трассы. При определенных условиях 
(амплитуде помехи, соотношении паразитных емкостей трасс-<<агрес\-со\-ров>> 
и трас\-сы-<<жерт\-вы>>) перепад напряжения \mbox{может} инвертировать 
логический уровень на трас\-се-<<жертве>>.
  
  В комбинационных СС-схе\-мах информационные сигналы представлены в 
парафазном коде со спейсером~[1]. Парафазный сигнал (ПФС) формируется 
парой дуальных логических ячеек. В~работе~\cite{2-step} было 
показано, что в СС-схе\-мах, изготовленных по объемной КМДП-тех\-но\-ло\-гии 
с проектными нормами 65~нм и ниже, при надлежащем размещении в 
топологии дуальных ячеек и~трасс ПФС ЛС может привести к изменению 
текущего состояния ПФС с~нулевым (<<00>>) или единичным (<<11>>) 
спейсером в~соответствии с~табл.~1.
   
  Однако свойства СС-схем и индикация состояния, инверсного спейсеру 
(антиспейсера, АС), как второго спейсера~[2] маскируют часть~ЛС.

%-\linebreak\vspace*{-12pt}



\section{Сбоеустойчивость самосинхронных схем}

  Вероятность распространения ЛС в СС-схе\-ме зависит от ряда факторов: 
типа ЛС; текущей фазы (рабочая фаза или спейсер); времени появления ЛС до 
момента срабатывания индикаторных выходов схемы; маскирования 
сбойного ПФС остальными сигналами 
  СС-схе\-мы; длительности ЛС; места появления ЛС.
  
  На практике СС-схе\-мы реализуются в виде конвейера, пример которого 
показан на рис.~1. Здесь Ст1, Ст2 и~Ст3~--- ступени конвейера; 
Г~--- Г-триг\-гер, формирующий фазовый сигнал управления ре\-гист\-ром.
   
  Вероятность распространения кратковременного одиночного ЛС в 
СС-кон\-вей\-ере в~первом приближении рассчитывается по формуле:

\noindent
  \begin{multline}
  P_{\mathrm{ЛС}}=P_{\mathrm{РФ}}\left[ \left( P_{\mathrm{вх}} 
P_{\mathrm{зн1}}+P_{4\mbox{\_}7}P_{\mathrm{зн2}}\right) 
P_{\mathrm{ЗР1}}+P_{\mathrm{И1}}\right]+{}\\
  {}+ P_{\mathrm{СФ}}\left[ \left( P_{\mathrm{вх}} 
P_{\mathrm{зн3}}+P_{1\mbox{\_}3}P_{\mathrm{зн4}}\right) 
P_{\mathrm{ЗР2}}+P_{\mathrm{И2}}\right]\,,
  \label{e3-step}
  \end{multline}
  
  \vspace*{-2pt}
  
  \noindent
где $ P_{\mathrm{РФ}}$ и~$P_{\mathrm{СФ}}$~--- вероятности пребывания 
СС-схе\-мы в рабочей фазе и в спейсере в момент появления ЛС; 
$P_{\mathrm{вх}}$~--- вероятность того, что данный ЛС появился в выходном 
регистре предыдущей ступени; $P_{\mathrm{ЗР1}},\ldots , 
P_{\mathrm{ЗР4}}$~--- вероятности того, что данный ЛС не будет 
замаскирован логикой ступени; $P_{1\mbox{\_}3}$ и~$P_{4\mbox{\_}7}$~--- вероятности 
принадлежности ЛС к типам~1--3 или 4--7 из табл.~1 соответственно; 
$P_{\mathrm{ЗР1}}$\linebreak и~$P_{\mathrm{ЗР2}}$~--- вероятности записи ЛС в выходной 
ре-\linebreak гистр ступени; $P_{\mathrm{И1}}$ и~$P_{\mathrm{И2}}$~--- вероятности того, что 
данный ЛС появился в индикаторной части сту\-пени.

  
  Формула~(\ref{e3-step}) дает оценку сверху. Она учитывает вклад трех 
частей ступени 
  СС-кон\-вей\-ера: комбинационной части (КЧ), индикаторной части (ИЧ) и 
выходного регистра (ВР),~--- но не учитывает времени появления ЛС в 
рабочем цикле СС-схе\-мы и~соотношений длительностей ЛС, рабочей и 
спейсерной фаз разных ступеней конвейера.
  
  Сбоеустойчивость СС-конвейера зависит от сбоеустойчивости всех его 
частей: КЧ, ИЧ и ВР. Рассмотрим каждую из них, оценивая их устойчивость с 
помощью вероятностного подхода~\cite{2-step}. Критическими считаются 
ЛС, приводящие к искажению обрабатываемых данных или остановке 
конвейера.

\vspace*{-4pt}

\subsection{Комбинационная часть ступени самосинхронного конвейера}

  Анализ влияния одиночного кратковременного ЛС, возникшего в КЧ 
ступени 
  СС-кон\-вей\-ера, на работоспособность конвейера показал, что благодаря 
двухфазной дисциплине работы, парафазному со спейсером кодированию 
информационных сигналов и индикации окончания переключения всех 
элементов схемы, возбужденных на данной фазе, КЧ конвейера устойчива к 
85,5\% ЛС~[2]. Схемотехнические и топологические методы, предложенные в 
работе~\cite{2-step}, повышают устойчивость КЧ 
  СС-кон\-вей\-ера к одиночным ЛС, приведенным в табл.~1, до уровня 
98,9\% при условии индикации АС как второго спейсера. 

\vspace*{-4pt}
  
\subsection{Регистр ступени самосинхронного конвейера}
  
  В СС-конвейерах с ПФС разряд регистра ступени традиционно реализуется 
на двух 
  Г-триг\-ге\-рах и~индикаторном элементе~\cite{4-step}, обеспечивая 
хранение рабочего состояния и спейсера ПФС при минимальных аппаратных 
затратах. Но такая реализация обладает низкой сбоеустойчивостью.
  
  Возможные варианты разряда СС-ре\-гист\-ра хранения представлены на 
рис.~2. Все они имеют парафазные вход (R, S) и 
выход (Q, QB) с нулевым спейсером, сигнал разрешения записи (E), 
регулирующий фазовые переходы регистра, и~индикаторный выход~(I). 
Рисунок~3 иллюстрирует реализацию Г-триг\-ге\-ра GI2AT (из разряда 
регистра на рис.~2,\,\textit{ж}) с нулевым спейсером~\cite{5-step}, 
устойчивого к АС на \mbox{входе}. 
  
  Таблица~2 содержит оценки устойчивости разрядов СС-ре\-гист\-ра 
хранения, приведенных на рис.~2, к ЛС из табл.~1 и показатели сложности их 
реализации  
в КМДП-тран\-зис\-то\-рах. Сравнение характеристик вариантов разряда СС-ре\-гист\-ра хранения показывает, что максимально защищенным от ЛС 
оказывается разряд вида рис.~2,\,\textit{а}. Наилучшим отношением показателя 
сбоеустойчивости к сложности реализации обладает разряд вида рис.~2,\,\textit{ж}.
  
\begin{figure*} %fig2
\vspace*{1pt}
    \begin{center}  
  \mbox{%
 \epsfxsize=157.389mm 
 \epsfbox{dya-2.eps}
 }
\end{center}
\vspace*{-11pt}
  \Caption{Варианты сбоеустойчивого разряда регистра СС-конвейера}
  \end{figure*}

  Таким образом, классический Г-триг\-гер не рекомендуется использовать 
для реализации разряда регистра СС-кон\-вей\-ера, предназначенного для 
эксплуатации в условиях активных неблагоприятных внешних воздействий.

\begin{figure*} %fig3
\vspace*{1pt}
    \begin{center}  
  \mbox{%
 \epsfxsize=117.113mm 
 \epsfbox{dya-3.eps}
 }
\end{center}
\vspace*{-11pt}
\Caption{Схема Г-триггера, защищенного от АС на входе}
\vspace*{-8pt}
\end{figure*}

\setcounter{table}{1}
\begin{table*}\small %tabl2
\begin{center}
\parbox{100mm}{\Caption{Вероятность внесения ошибки в обрабатываемые данные  
из-за ЛС в ступени конвейера}
}

\vspace*{2ex}

\begin{tabular}{|c|l|c|c|c|c|}
\hline
№ &\multicolumn{1}{c|}{Тип разряда} &\multicolumn{3}{c|}{Устойчивость к ЛС, \%}
&Число \\
\cline{3-5}
п/п&\multicolumn{1}{c|}{регистра}&В КЧ&В регистре&Общая&транзисторов\\
\hline
1&\hspace*{10pt}Рис. 2,\,\textit{а}&98,76&93,62&97,78&44\\
2&\hspace*{10pt}Рис. 2,\,\textit{б}&97,66&96,27&98,04&54\\
3&\hspace*{10pt}Рис. 2,\,\textit{в}&89,84&89,99&92,42&62\\
4&\hspace*{10pt}Рис. 2,\,\textit{г}&86,36&58,59&82,83&30\\
5&\hspace*{10pt}Рис. 2,\,\textit{д}&97,92&94,53&97,59&48\\
6&\hspace*{10pt}Рис. 2,\,\textit{е}&98,18&92,43&97,20&70\\
7&\hspace*{10pt}Рис. 2,\,\textit{ж}&97,88&91,41&96,76&32\\
\hline
\end{tabular}
\end{center}
\vspace*{-3pt}
\end{table*}

\vspace*{-6pt}

\subsection{Индикаторная часть ступени самосинхронного конвейера}

\vspace*{-2pt}

  Индикаторная часть~--- наиболее чувствительная часть СС-кон\-вей\-ера. Критическая 
ситуация возможна при преж\-де\-вре\-мен\-ном переключении \mbox{Г-триг}\-ге\-ра, 
формирующего сигнал управления регистром ступени. В~результате регистр 
может раньше времени перейти в~спейсер (рабочее состояние) и~тем самым 
помешать правильному переключению сле\-ду\-ющей ступени конвейера.
  
  Однако использование сбоеустойчивой DICE-по\-доб\-ной реализации 
  Г-триг\-ге\-ров~\cite{2-step} обеспечивает их абсолютную устойчивость 
  к~ЛС, указанным в~табл.~1, и~предотвращает появление критической ситуации 
  в~конвейере. В~результате ЛС может привести лишь к~временной приостановке 
конвейера: по окончании ЛС конвейер продолжит нормальную работу.

\vspace*{-6pt}
  
\subsection{Общая сбоеустойчивость самосинхронного конвейера}

\vspace*{-2pt}

  В реальных СС-схемах КЧ ступени конвейера обычно имеет площадь 
топологической реализации в~несколько раз больше, чем регистр ступени. 
Пусть, например, это соотношение равно двум. Площадь ИЧ также меньше 
площади
КЧ примерно в 2~раза. Тогда при равномерном распределении сбоев по 
площади СБИС вероятность их появления в~КЧ в~2~раза больше вероятности 
их появления в~регистре и~ИЧ. С~учетом отсутствия критических ситуаций 
при ЛС в~ИЧ вероятность распространения ЛС по СС-конв\-ей\-еру для 
разных вариантов реализации разряда регистра будет соответствовать 
результатам, приведенным в табл.~2 в~графе <<общая устойчивость>>.

\vspace*{-6pt}

\section{Заключение}

\vspace*{-2pt}
  
  Индикация антиспейсера ПФС как второго спейсера и использование 
  DICE-по\-доб\-но\-го Г-триг\-ге\-ра с~<<самолечением>> АС 
обеспечивают существенное повышение сбоеустойчивости комбинационной 
и~индикаторной частей    СС-кон\-вей\-ера.
  
  Классический вариант разряда регистра СС-конвейера на обычных Г-триг\-ге\-рах имеет сравнительно низкую сбоеустойчивость~--- на уровне 83\%. 
  
  Наилучшую сбоеустойчивость по отношению к~кратковременным 
одиночным ЛС демонстрирует вариант СС-кон\-вей\-ера, регистр в~котором 
реализуется на однотактном 
  RS-триг\-ге\-ре с разрешени-\linebreak ем записи. Он устойчив к~98\%~ЛС в~ступени 
кон-\linebreak вейера. 

\vspace*{-8pt}

{\small\frenchspacing
 {%\baselineskip=10.8pt
 %\addcontentsline{toc}{section}{References}
 \begin{thebibliography}{9}
 
 \vspace*{-2pt}
  
  \bibitem{1-step}
  \Au{Kishinevsky M., Kondratyev~A., Taubin~A., Varshavsky~V.} Concurrent hardware: The 
theory and practice of self-timed design.~--- New York, NY, USA: J.~Wiley \& Sons, 1994. 368~p.
  \bibitem{2-step}
  \Au{Stepchenkov Y.\,A., Kamenskih~A.\,N., Diachenko~Y.\,G., Rogdestvenski~Y.\,V., 
Diachenko~D.\,Y.} Improvement of the natural self-timed circuit tolerance to short-term soft errors~// 
Advances Science Technology Engineering Systems~J., 2020. Vol.~5. Iss.~2. P.~44--56.
  \bibitem{3-step}
  \Au{Nicolaidis M.} Soft errors in modern electronic systems.~--- New York, NY, USA: Springer, 
2011. 316~p.
  \bibitem{4-step}
  \Au{Степченков Ю.\,А., Дьяченко~Ю.\,Г., Рождественский~Ю.\,В., Морозов~Н.\,В., 
Степченков~Д.\,Ю., Рож\-дественскене~А.\,В., Сурков~А.\,В.} Самосинхронный умножитель с 
накоплением: варианты реализации~//\linebreak Системы и средства информатики, 2014. Т.~24. №\,3. 
С.~63--77.
  \bibitem{5-step}
  \Au{Соколов И.\,А., Захаров В.\,Н., Степченков~Ю.\,А., Дьяченко~Ю.\,Г.} 
 Устройство сбоеустойчивого разряда самосинхронного регистра хранения: Заявка на 
изобретение №\,2020140031 от 28.02.20.
\end{thebibliography}

 }
 }

\end{multicols}

\vspace*{-8pt}

\hfill{\small\textit{Поступила в~редакцию 13.03.2020}}


%\vspace*{8pt}

%\pagebreak

\newpage

\vspace*{-28pt}

%\hrule

%\vspace*{2pt}

%\hrule

%\vspace*{-2pt}

\def\tit{IMPROVEMENT OF SELF-TIMED CIRCUIT SOFT ERROR TOLERANCE}


\def\titkol{Improvement of self-time circuit soft error tilerance}


\def\aut{I.\,A.~Sokolov, Yu.\,A.~Stepchenkov, Yu.\,G.~Diachenko, 
and~Yu.\,V.~Rogdestvenski}

\def\autkol{I.\,A.~Sokolov, Yu.\,A.~Stepchenkov, Yu.\,G.~Diachenko, 
and~Yu.\,V.~Rogdestvenski}

\titel{\tit}{\aut}{\autkol}{\titkol}

\vspace*{-11pt}

  \noindent
  Federal Research Center ``Computer Science and Control'' of the Russian Academy of Sciences, 
  44-2~Vavilov Str., Moscow 119333, Russian Federation
  
 

\def\leftfootline{\small{\textbf{\thepage}
\hfill INFORMATIKA I EE PRIMENENIYA~--- INFORMATICS AND
APPLICATIONS\ \ \ 2020\ \ \ volume~14\ \ \ issue\ 4}
}%
 \def\rightfootline{\small{INFORMATIKA I EE PRIMENENIYA~---
INFORMATICS AND APPLICATIONS\ \ \ 2020\ \ \ volume~14\ \ \ issue\ 4
\hfill \textbf{\thepage}}}

\vspace*{3pt} 
  
  \Abste{The paper considers a~tolerance of self-timed (ST) circuits 
  fabricated with complementary metal--oxide--semiconductor (CMOS) process 
  to short-term soft errors generated by external causes, namely, nuclear particles, 
  cosmic rays, electromagnetic pulses, and noises. Pipeline implementation is usual 
  for practical ST-circuits. Its control bases on handshake between pipeline stages
   and two-phase operation discipline with a~sequence of the working phase and spacer one. 
   Combinational part of the pipeline stage uses dual-rail information signal coding with a~spacer. 
   The pipeline stage indication part acknowledges a~switching completion of all stage cells, 
   fired at the current operation phase, and generates handshake signals in ST-pipeline stages control. 
   The paper discusses the physical causes of the short-term soft errors. It analyzes 
   soft error types that may appear in CMOS ST-circuits fabricated with 65-nanometer and below 
   standard bulk process. The tolerance level of the proposed soft error hardened 
   ST-register bits is discussed and compared. The paper suggests circuitry and layout 
   techniques improving ST-pipeline soft error tolerance and estimates soft error immunity 
   level for all pipeline parts depending on soft error location.}
   
   \vspace*{2pt}
  
  \KWE{self-timed circuit; tolerance; pipeline; working phase; spacer}
  
\DOI{10.14357/19922264200409} 

\vspace*{-15pt}

  \Ack
  
  \vspace*{-3pt}
  
  \noindent
  The research was supported by the Ministry of Science and Higher Education of the Russian Federation, 
project No.\,075-15-2020-799.

\vspace*{6pt}

  \begin{multicols}{2}

\renewcommand{\bibname}{\protect\rmfamily References}
%\renewcommand{\bibname}{\large\protect\rm References}

{\small\frenchspacing
 {%\baselineskip=10.8pt
 \addcontentsline{toc}{section}{References}
 \begin{thebibliography}{9}
  
  
  \bibitem{1-step-1}
  \Aue{Kishinevsky, M., A.~Kondratyev, A.~Taubin, and V.~Varshavsky.} 1994. \textit{Concurrent 
hardware: The theory and practice of self-timed design}. New York, NY: J.~Wiley \& Sons. 368~p.
  \bibitem{2-step-1}
  \Aue{Stepchenkov, Y.\,A., A.\,N.~Kamenskih, Y.\,G.~Diachenko, Y.\,V.~Rogdestvenski, and 
D.\,Y.~Diachenko}. 2020. Improvement of the natural self-timed circuit tolerance to short-term soft 
errors. \textit{Advances Science Technology Engineering Systems~J.} 5(2):44--56.
  \bibitem{3-step-1}
  \Aue{Nicolaidis, M.} 2011. \textit{Soft errors in modern electronic systems}. New York, NY: 
Springer. 316~p.
  \bibitem{4-step-1}
  \Aue{Stepchenkov, Y., Y.~Diachenko, Y.~Rogdestvenski, N.~Morozov, D.~Stepchenkov, 
A.~Rogdestvenskene, and A.~Surkov.} 2014. Samosinkhronnyy umnozhitel' s~nakopleniem: varianty 
realizatsii [Self-timed fused multiply-add unit: Implementation variants]. \textit{Sistemy i~Sredstva 
Informatiki~--- Systems and Means of Informatics} 24(3):63--77.
  \bibitem{5-step-1}
  \Aue{Sokolov, I.\,A., V.\,N.~Zakharov, Yu.\,A.~Stepchenkov, and Yu.\,G.~Diachenko.} 2020. 
Ustroystvo sboeustoychivogo razryada samosinkhronnogo registra khraneniya [Fault-tolerance bit device 
for self-timed storage register]. Application for Patent RF No.\,2020140031.
\end{thebibliography}

 }
 }

\end{multicols}

\vspace*{-3pt}

  \hfill{\small\textit{Received March~13, 2020}}


%\pagebreak

\vspace*{-18pt}
  
  
  \Contr
  
  \vspace*{-2pt}
  
  \noindent
  \textbf{Sokolov Igor A.} (b.\ 1954)~--- Doctor of Science in technology, Academician of RAS, 
  director, Federal Research Center ``Computer Science and Control'' of the 
Russian Academy of Sciences, 44-2~Vavilov Str., Moscow 119133, Russian Federation; 
\mbox{isokolov@ipiran.ru}
  
  \vspace*{3pt}
  
  \noindent
  \textbf{Stepchenkov Yuri A.} (b.\ 1951)~--- Candidate of Science (PhD) in technology, leading 
scientist, Federal Research Center ``Computer Science and Control'' of the Russian Academy of 
Sciences, 44-2~Vavilov Str., Moscow 119133, Russian Federation; \mbox{YStepchenkov@ipiran.ru}
  
  \vspace*{3pt}
  
  \noindent
  \textbf{Diachenko Yuri G.} (b.\ 1958)~--- Candidate of Science (PhD) in technology, senior scientist, 
Federal Research Center ``Computer Science and Control'' of the Russian Academy of Sciences, 44-
2~Vavilov Str., Moscow 119133, Russian Federation; \mbox{diaura@mail.ru}
  
  \vspace*{3pt}
  
  \noindent
  \textbf{Rogdestvenski Yuri V.} (b.\ 1952)~--- Candidate of Science (PhD) in technology, leading 
scientist, Federal Research Center ``Computer Science and Control'' of the Russian Academy of 
Sciences, 44-2~Vavilov Str., Moscow 119333, Russian Federation, Moscow 119333, Russian 
Federation; \mbox{YRogdest@ipiran.ru}
\label{end\stat}

\renewcommand{\bibname}{\protect\rm Литература} 
  
  