\def\stat{danilishin}

\def\tit{ОЦЕНКА СТОИМОСТИ ОПЦИОНОВ НА~ОСНОВЕ 
МОДЕЛЕЙ ARIMA--GARCH С~ОШИБКАМИ, РАСПРЕДЕЛЕННЫМИ 
ПО~ЗАКОНУ~$S_U$ ДЖОНСОНА}

\def\titkol{Оценка стоимости опционов на основе моделей  
ARIMA--GARCH с~ошибками, распределенными по закону~JSU} %$S_U$  Джонсона}

\def\aut{А.\,Р.~Данилишин$^1$, Д.\,Ю.~Голембиовский$^2$}

\def\autkol{А.\,Р.~Данилишин, Д.\,Ю.~Голембиовский}

\titel{\tit}{\aut}{\autkol}{\titkol}

\index{Данилишин А.\,Р.}
\index{Голембиовский Д.\,Ю.}
\index{Danilishin A.\,R.}
\index{Golembiovsky D.\,Yu.}


%{\renewcommand{\thefootnote}{\fnsymbol{footnote}} \footnotetext[1]
%{Работа выполнена при частичной поддержке РФФИ (проект 19-07-00187-A).}}


\renewcommand{\thefootnote}{\arabic{footnote}}
\footnotetext[1]{Московский государственный университет имени М.\,В.~Ломоносова, факультет 
вычислительной математики и~кибернетики, \mbox{danilishin-artem@mail.ru}}
\footnotetext[2]{Московский государственный университет имени М.\,В.~Ломоносова, факультет 
вычислительной математики и~кибернетики; Московский фи\-нан\-со\-во-про\-мыш\-лен\-ный университет 
<<Синергия>>, \mbox{golemb@cs.msu.su}}

%\vspace*{-12pt}


\Abst{В продолжение статьи <<Риск-нейтральная динамика для модели ARIMA--GARCH 
с~ошибками, распределенными по закону $S_U$ Джонсона>> в~данной работе приводятся 
результаты экспериментов для моделей ARIMA--GARCH 
(autoregressive integrated moving average\,--\,generalized autoregressive conditional heteroskedasticity)
с~нормальными (N), 
экспоненциальными бета второго типа (EGB2) и~$S_U$ Джонсона (JSU) распределениями 
ошибок. Стоимость европейских опционов оценивается методом Мон\-те-Кар\-ло на основе 
результатов, полученных в~указанной статье при помощи расширенного принципа 
Гирсанова. Параметры моделей ARIMA--GARCH-N, ARIMA--GARCH-EGB2 и~ARIMA--GARCH-JSU 
были найдены методом квазимаксимального правдоподобия. Эффективность 
полученных риск-нейтральных моделей исследовалась на примере биржевых европейских 
опционов PUT и~CALL на базовые активы DAX  (Deutscher Aktienindex)
и~Light Sweet Crude Oil. }

\KW{ARIMA; GARCH; риск-нейтральная мера; расширенный принцип Гирсанова; 
распределение $S_U$ Джонсона; ценообразование опционов}

\DOI{10.14357/19922264200412} 
  
%\vspace*{9pt}


\vskip 10pt plus 9pt minus 6pt

\thispagestyle{headings}

\begin{multicols}{2}

\label{st\stat}


\section{Введение}

Данная работа является продолжением \mbox{статьи}~[1]. В~указанной статье была 
введена модель  
ARIMA$\,(p,d,q)$--GARCH$\,(P,Q)$-JSU$\,(\xi,\lambda, \gamma,\delta)$ для 
доходности базового актива в~следующем виде (для распределения JSU 
$\tilde{Y}_t\hm= S_t/S_{t-1}\hm -1$, для N- и~EGB2-рас\-пре\-де\-ле\-ний 
$Y_t\hm= \ln(S_t/S_{t-1})$)~[2--5]:
\begin{equation}
\left.
\begin{array}{l}
\!\!\!\Delta^d Y_t=m_t +\sqrt{h_t}\,\varepsilon_t\,,\ \varepsilon_t\vert \sim 
\mathrm{JSU}(\xi,\lambda,\gamma,\delta)\,;\\[12pt]
\!\!\!m_t=\mathbb{E}\left[ \Delta^dY_t\vert \mathcal{F}_{t-1}\right]=
\phi_0+\cdots+\phi_p\Delta^d Y_{t-p} +{}\\[6pt]
\!\!\!\hspace*{10mm}{}+\theta_1\sqrt{h_{t-1}}\,\varepsilon_{t-1}+
\cdots+\theta_q \sqrt{h_{t-q}}\,\varepsilon_{t-q}\,;\\[12pt]
\!\!\!h_t=\mathrm{Var}\left[ \Delta^d Y_t\vert \mathcal{F}_{t-1}\right]=
\alpha_0+\cdots+\alpha_P h_{t-P} +{}\\[6pt]
\!\!\!\hspace*{15mm}{}+\beta_1h_{t-1}\varepsilon^2_{t-1}+\cdots+\beta_Q h_{t-Q} 
\varepsilon^2_{t-Q}
\end{array} \!\!
\right\} \!\!
\label{e1-dan}
\end{equation}
с ограничениями на математическое ожидание и~дисперсию ошибки
\begin{align*}
\mathbb{E}[\varepsilon_t] &=\xi-\lambda e^{1/(2\delta^2)}\mathrm{sinh}\left(
\fr{\gamma}{\delta}\right)=0\,;\\
\mathrm{Var}\left[\varepsilon_t\right]&=\fr{\lambda^2}{2}\left( e^{1/\delta^2}-1\right)\times{}\\
&\hspace*{10mm}{}\times \left( 
e^{1/\delta^2} \mathrm{cosh}\left( \fr{2\gamma}{\delta}\right) +1\right)=1\,.
\end{align*}

%\label{e2-dan}

  Далее к~разностному (стационарному) ряду~(\ref{e1-dan}) применялся 
расширенный принцип Гирсанова~[6, 7], а~в~случае распределения~JSU~---
 его найденная модификация. Для распределения JSU %$S_U$ Джонсона 
полученные коэффициенты модели, обес\-пе\-чи\-ва\-ющие риск-ней\-т\-раль\-ную 
динамику процесса даются сле\-ду\-ющи\-ми соотношениями:
  \begin{equation}
  \left.
  \begin{array}{rl}
  Y_t&=r+\delta_t\fr{1+r}{1+m_t}\,\varepsilon_t\,;\\[9pt]
   \varepsilon_{t\vert
  \mathcal{F}_{t-1}} &\sim \mathrm{JSU}\left( 
\tilde{\xi},\tilde{\lambda},\gamma,\delta\right);\\[9pt]
  \tilde{\xi}&=\tilde{\lambda}e^{1/(2\delta^2)}\mathrm{sinh}\left( \fr{\gamma} 
{\delta} \right);\\[9pt] 
\tilde{\lambda}&=\sqrt{2}\left( \left( e^{1/\delta^2}-1\right)\times{}\right.\\[9pt]
&\hspace*{3mm}\left.{}\times\left(
  e^{1/\delta^2}\mathrm{cosh}\left( \fr{2\gamma}{\delta}\right)+1\right)\right)^{-1/2}.
  \end{array}
  \right\}
  \label{e3-dan}
  \end{equation}
  
  Риск-нейтральные коэффициенты для моделей ARIMA--GARCH-EGB2  
и~ARIMA--GARCH-N представлены соотношениями~\cite{7-dan}:
  \begin{multline}
  Y_t=r-\ln\fr{B(\alpha+\delta_t\overline{\delta},\beta-
\delta_t\overline{\delta})}{B(\alpha,\beta)}+{}\\
{}+\delta_t\overline{\delta}\overline{\omega}(\alpha,\beta)+
  \delta_t\varepsilon_t\,;
  \label{e4-dan}
  \end{multline}
  
  \noindent
  \begin{equation}
  \left.
  \begin{array}{c}
  \varepsilon_{t\vert\mathcal{F}_{t-1}}\sim 
\mathrm{EGB2}\left(\alpha,\beta,\overline{\delta},\overline{\mu}\right);\\[6pt]
  \overline{\delta}=\fr{1}{\sqrt{l(\alpha,\beta)}};\quad
   \overline{\mu}=-
\fr{\overline{\omega}(\alpha,\beta)}{\sqrt{l(\alpha,\beta)}};\\[12pt]
  Y_t=r-\fr{1}{2}\,\delta^2_t+\delta_t \varepsilon_t\,; \quad 
\varepsilon_{t\vert\mathcal{F}_{t-1}} \sim N(0,1)\,.
  \end{array}
  \right\}
  \label{e5-dan}
  \end{equation}
  
  В данной статье приводятся результаты чис\-лен\-ных экспериментов, 
подтверждающие корректность теоретических результатов первой работы 
и~эффективность полученных риск-ней\-траль\-ных моделей  
ARIMA--GARCH-N, ARIMA--GARCH-EGB2 и~ARIMA--GARCH-JSU. 
  
  Работа построена следующим образом. В~разд.~2 приводятся формулы для 
оценки справедливой стоимости опционов CALL и~PUT методом Мон\-те-Кар\-ло. 
В~разд.~3 описываются два набора данных для проведения 
численных экспериментов: цены закрытия торгов по опционам на индекс 
немецких акций DAX и~на нефть Light Sweet Crude Oil, а~также 
соответствующие ряды цен базовых активов. В~разд.~4 даются формулы 
оценки параметров моделей ARIMA--GARCH методом квазимаксимального 
правдоподобия~\cite{8-dan}. В~разд.~5 приводятся тесты, определяющие 
спецификации моделей и~результаты оценок параметров моделей. Раздел~6 
содержит результаты оценки справедливой стоимости опционов, полученные 
при использовании различных моделей динамики базовых активов. 
В~заключении формулируются выводы исследования.

\vspace*{-6pt}

\section{Оценка справедливой стоимости опциона методом  
Монте-Карло}

\vspace*{-3pt}

  Оценка справедливой стоимости опционов проводится по методу Мон\-те-Кар\-ло~\cite{9-dan}. 
  Европейские опционы CALL и~PUT с~ценой 
исполнения~$X$ и~стоимостью базового актива~$S_T$ в~день 
экспирации~$T$ характеризуются функциями выплат $b_c (S_T,X)\hm= \max 
(S_T\hm - X,0)$ и~$b_p(S_T, X)\hm= \max (X\hm- S_T,0)$. Стоимость опционов 
определяется как среднее значение соответствующей функции выплаты 
относительно риск-ней\-траль\-ной меры~$\mathbb{Q}$, приведенное 
к~текущему моменту времени~\cite{10-dan}:
  \begin{equation}
\! \fr{p_{\mathrm{call}}}{p_{\mathrm{put}}} \!=\!\fr{\mathbb{E}^{\mathbb{Q}} [b_{c/p}(S_T,X)]} {(1+r)^T} 
\!=\! \fr{\mathbb{E}^{\mathbb{P}} [b_{c/p}(S_T,X)d\mathbb{Q}/d\mathbb{P}]} 
{(1+r)^T},\!\!
  \label{e6-dan}
  \end{equation}
где $d\mathbb{Q}/d\mathbb{P}$~--- производная Ра\-до\-на--Ни\-ко\-ди\-ма  
риск-ней\-т\-раль\-ной меры (в~рамках данной работы это мера, полученная на 
основе расширенного принципа Гирсанова либо его модификации) 
относительно физической меры~$\mathbb{P}$~\cite{11-dan, 12-dan}. Метод 
Мон\-те-Кар\-ло позволяет по реализациям построенного процесса ARIMA--GARCH 
оценить среднее значение относительно  
риск-ней\-т\-раль\-ной меры~$\mathbb{Q}$:
\vspace*{-6pt}

\noindent
\begin{multline}
\fr{1}{M}\sum\limits_{m=1}^M b_{c/p}\left( S_T, X\right) 
\fr{d\mathbb{Q}}{d\mathbb{P}}\,(m) 
\xrightarrow[M\to\infty]{P}{} \\
{}\xrightarrow[M\to\infty]{P}
\mathbb{E}^{\mathbb{P}} \left[ b_{c/p}\left( 
S_T,X\right)\fr{d\mathbb{Q}}{d\mathbb{P}}\right]\,.
\label{e7-dan}
\end{multline}

\vspace*{-9pt}

\section{Описание данных}

\vspace*{-3pt}

  Данные для проведения численных экспериментов состоят из двух 
однотипных наборов цен закрытия торгов по опционам на 3~июня 2019~г. По 
дате экспирации опционы были поделены на самые ближние и~дальние 
с~учетом ликвидности.
  
  Первый набор данных представлен европейскими опционами на фондовый 
индекс DAX. Индекс отражает суммарный доход по 
капиталу, поэтому при его расчете учитываются полученные дивидендные 
доходы по акциям, которые, как предполагается, реинвестируются в~акцию, 
по которой был получен дивиденд. Рас\-смат\-ри\-ва\-ют\-ся 19~опционов CALL 
и~19~опционов PUT, величина страйка варьируется от~9\,400 до~13\,000 
с~шагом~200. Базовым активом выступает фьючерс с~датой экспирации, 
соответствующей дате экспирации самого опциона. Дата экспирации ближних 
опционов~--- 22~июня 2019~г., а~дальних~--- 22~декабря 2023~г. Валюта 
опционов~--- евро. Данные взяты с~сайта {\sf www.eurexchange.com}.
  
  Во второй набор данных вошли 10 европейских опционов CALL и~PUT на 
фьючерс, базовым активом которого выступает нефть (Light Sweet Crude Oil). 
Величина страйка варьируется от~51,0 до~55,5 с~шагом~0,5. Дата экспирации 
ближних опционов~--- 20~июня 2019~г., дальних~--- 22~июня 2020~г. 
Валюта~--- доллар США. Источник данных: {\sf www.cmegroup.com}.
  
  В представленных данных фигурируют два вида валют; соответственно, при 
расчете справедливой стоимости опционов использовались две\linebreak безрисковые 
процентные ставки. В~качестве таковых были взяты соответствующие ставки 
LIBOR. На дату расчета (03~июня 2019~г.)\ ставка USD \mbox{LIBOR} равнялась 
2,36075\%, а~ставка EUR \mbox{LIBOR} со\-став\-ля\-ла~0,46614\%. Источник данных: 
{\sf www.global-rates.com}.

\vspace*{-12pt}

\section{Калибровка моделей ARIMA--GARCH}

\vspace*{-3pt}

\begin{table*}[b] %\small %tabl1
\vspace*{-3pt}
\begin{minipage}[t]{80mm}
\begin{center}
{\small 
\Caption{Результаты оценивания моделей ARIMA(0,0,1)--GARCH(1,1) для 
$Y_t^{\mathrm{N/EGB2}}\hm=\ln (S_t/S_{t-2})$  
и~$\tilde{Y}_t^{\mathrm{JSU}} \hm= S_t/S_{t-2}\hm-1$ фондового индекса DAX}
\vspace*{2ex}

\tabcolsep=8.15pt
\begin{tabular}{|c|c|c|c|}
\hline
\tabcolsep=0pt\begin{tabular}{c}Распре-\\ деление\\ ошибок\end{tabular}&N&EGB2&JSU \\
\hline
$L_n(\hat{v})$&1652,609&1658,26&1658,657\\
\hline
\tabcolsep=0pt \begin{tabular}{c} $\phi_0$\\ Std.\ error\\ $t$-value\end{tabular}&
\tabcolsep=0pt \begin{tabular}{c} $-$0,000100\\ (0,000798)\\ $-$0,12480 \end{tabular}&
\tabcolsep=0pt \begin{tabular}{c} $-$0,00006\\ (0,000786)\\ $-$0,076381 \end{tabular}&
\tabcolsep=0pt \begin{tabular}{c}  $-$0,000074\\ (0,000792)\\ $-$0,093104 \end{tabular}\\
\hline
\tabcolsep=0pt \begin{tabular}{c} $\theta_1$\\ Std.\ error\\ $t$-value\end{tabular}&
\tabcolsep=0pt \begin{tabular}{c} 0,950806\\ (0,013106)\\ 72,54849 \end{tabular}&
\tabcolsep=0pt \begin{tabular}{c} 0,948445\\ (0,015226)\\ 62,292279 \end{tabular}&
\tabcolsep=0pt \begin{tabular}{c} 0,949900\\ (0,013229)\\ 71,806222 \end{tabular}\\
\hline
\tabcolsep=0pt \begin{tabular}{c} $\alpha_0$\\ Std.\ error\\ $t$-value\end{tabular}&
\tabcolsep=0pt \begin{tabular}{c} 0,000003\\ (0,000005)\\ 0,55453 \end{tabular}&
\tabcolsep=0pt \begin{tabular}{c} 0,000003\\ (0,000004)\\ 0,687438 \end{tabular}&
\tabcolsep=0pt \begin{tabular}{c} 0,000003\\ (0,000004)\\ 0,718274 \end{tabular}\\
\hline
\tabcolsep=0pt \begin{tabular}{c} $\alpha_1$\\ Std.\ error\\ $t$-value \end{tabular}&
\tabcolsep=0pt \begin{tabular}{c} 0,071761\\ (0,031491)\\ 2,27881 \end{tabular}&
\tabcolsep=0pt \begin{tabular}{c} 0,067895\\ (0,027377)\\ 2,480017 \end{tabular}&
\tabcolsep=0pt \begin{tabular}{c} 0,067271\\ (0,030275)\\ 2,222013 \end{tabular}\\
\hline
\tabcolsep=0pt \begin{tabular}{c} $\beta_1$\\ Std.\ error\\ $t$-value \end{tabular}&
\tabcolsep=0pt \begin{tabular}{c} 0,894232\\ (0,050380)\\ 17,74964 \end{tabular}&
\tabcolsep=0pt \begin{tabular}{c} 0,902903\\ (0,037822)\\ 23,872549 \end{tabular}&
\tabcolsep=0pt \begin{tabular}{c} 0,900226\\ (0,041630)\\ 21,624622 \end{tabular}\\
\hline
AIC&$-$6,5773&$-$6,5919&$-$6,5934\\
\hline
BIC&$-$6,5352&$-$6,5369&$-$6,5385\\
\hline
$\xi$&---&$-$0,224916&$-$0,543876\\
\hline
$\kappa$&---&2,897165&2,298773\\
\hline
\end{tabular}
}
\end{center}
\end{minipage}
%\end{table*}
\hfill
%\begin{table*}%{\small %tabl2
\begin{minipage}[t]{80mm}
\begin{center}
{\small 
\Caption{Результаты оценивания моделей ARIMA(2,0,0)--GARCH(1,1) 
для $Y_t^{\mathrm{N/EGB2}}\hm= \ln (S_t/S_{t-2})$ 
и $\tilde{Y}_t^{\mathrm{JSU}}\hm= S_t/S_{t-2}\hm-1$ (Light Sweet Crude Oil)}
\vspace*{2ex}

\tabcolsep=8.14pt
\begin{tabular}{|c|c|c|c|}
\hline
\tabcolsep=0pt\begin{tabular}{c}Распре-\\ деление\\ ошибок\end{tabular}&N&EGB2&JSU\\
\hline
$L_n(\hat{v})$&1261,086&1267,017&1268,022\\
\hline
\tabcolsep=0pt \begin{tabular}{c} $\phi_1$\\ Std.\ error\\ $t$-value \end{tabular}&
\tabcolsep=0pt \begin{tabular}{c} 0,667116\\ (0,043985)\\ 15,1668 \end{tabular}&
\tabcolsep=0pt \begin{tabular}{c} 0,657088\\ (0,043122)\\ 15,2379 \end{tabular}&
\tabcolsep=0pt \begin{tabular}{c} 0,659745\\ (0,043152)\\ 15,2888\end{tabular}\\
\hline
\tabcolsep=0pt \begin{tabular}{c} $\phi_2$\\ Std.\ error\\ $t$-value \end{tabular}&
\tabcolsep=0pt \begin{tabular}{c} $-$0,313186\\ (0,043636)\\ $-$7,1773 \end{tabular}&
\tabcolsep=0pt \begin{tabular}{c} $-$0,323465\\ (0,042236)\\ $-$7,6585 \end{tabular}&
\tabcolsep=0pt \begin{tabular}{c} $-$0,322464\\ (0,042439)\\ $-$7,5983 \end{tabular}\\
\hline
\tabcolsep=0pt \begin{tabular}{c} $\alpha_0$\\ Std.\ error\\ $t$-value \end{tabular}&
\tabcolsep=0pt \begin{tabular}{c} 0,000015\\ (0,000002)\\ 8,341500 \end{tabular}&
\tabcolsep=0pt \begin{tabular}{c} 0,000013\\ (0,000001)\\ 16,4056 \end{tabular}&
\tabcolsep=0pt \begin{tabular}{c} 0,000013\\ (0,000001)\\ 16,1163\end{tabular}\\
\hline
\tabcolsep=0pt \begin{tabular}{c} $\alpha_1$\\ Std.\ error\\ $t$-value \end{tabular}&
\tabcolsep=0pt \begin{tabular}{c} 0,049707\\ (0,007002)\\ 7,0988\end{tabular}&
\tabcolsep=0pt \begin{tabular}{c} 0,042324\\ (0,005701)\\ 7,424 \end{tabular}&
\tabcolsep=0pt \begin{tabular}{c} 0,041834\\ (0,005506)\\ 7,5984\end{tabular}\\
\hline
\tabcolsep=0pt \begin{tabular}{c} $\beta_1$\\ Std.\ error\\ $t$-value \end{tabular}&
\tabcolsep=0pt \begin{tabular}{c} 0,913128\\ (0,013747)\\ 66,424300 \end{tabular}&
\tabcolsep=0pt \begin{tabular}{c} 0,92728\\ (0,011196)\\ 82,8231 \end{tabular}&
\tabcolsep=0pt \begin{tabular}{c} 0,928020\\ (0,011169)\\ 83,0925\end{tabular}\\
\hline
AIC&$-$4,9944&1267,752&$-$5,0140\\ 
\hline
BIC&$-$4,9524&$-$4,9542&$-$4,9553\\
\hline
$\xi$&&$-$0,70877&$-$0,27232\\
\hline
$\kappa$&&4,405384&2,663401\\
\hline
\end{tabular}
}
\end{center}
\end{minipage}
\vspace*{-3pt}
\end{table*}

  Пусть $\ln(L_n(v))$~--- логарифм функции правдоподобия модели  
ARIMA--GARCH для доходности базового актива с~вектором параметров 
$v\hm\in \Theta$. В~случае модели  
ARIMA$\,(p,d,q)$--GARCH$\,(P,Q)$-JSU$\,(\xi,\lambda,\gamma,\delta)$~(1) 
имеется вектор параметров 
\begin{multline*}
v= \left(\gamma, \delta, \phi_0, \phi_1, \ldots , \phi_p, 
\theta_1, \ldots , \theta_q, \alpha_0, \alpha_1, \ldots , \alpha_P,\right.\\
\left. \beta_1, \ldots , 
\beta_Q\right).
\end{multline*}
 Число параметров равно $n\hm= p\hm+ 1\hm+q\hm+ P\hm+Q\hm+2$. 
Оптимальные параметры определяются исходя из максимума функции  
правдоподобия~\cite{8-dan}:
  \begin{multline}
  \hat{v}_n=\argmax\limits_{v\in\Theta} \ln (L_n(v))={}\\[-3pt]
  {}=\argmax\limits_{v\in \Theta} \sum\limits^T_{t=0} \left( \ln(\delta)-
\ln \left(\sqrt{\fr{2h_t}{A}}\right)-{}\right.\\[-1pt]
\left.{}- \fr{1}{2}\ln\left(1+\left( 
\fr{\varepsilon_t}{\sqrt{2h_t/A}}-B\right)^{\!2}\right) \right.-{}\\[-3pt]
  \left.{}-\fr{1}{2}\left(\gamma+\delta\mathrm{sinh}^{-1}\left( 
\fr{\varepsilon_t}{\sqrt{2h_t/A}}-B\right)\right)^{\!2}\right)\,,\\[-1pt]
\delta, \alpha_0>0,\alpha_1,\ldots , \alpha_P, \beta_1,\ldots , \beta_Q\geq 0\,,
\label{e8-dan}
\end{multline}
где 
%\vspace*{-6pt}


\noindent
$$
A=\left(e^{1/\delta^2}-1\right)\left(e^{1/\delta^2}\mathrm{cosh}\left(\fr{2\gamma}{\delta}\right)+1\right);
$$ 
$$
B=e^{1/(2\delta^2)}\mathrm{sinh}\left(\fr{\gamma}{\delta}\right); \enskip
\sum\limits_{i=1}^P 
\alpha_i+ \sum\limits^Q_{j=1} \beta_j <1\,.
$$ 
  
  Для случая EGB2-распределения число па\-ра\-мет\-ров $v\hm= (\alpha, \beta, 
\phi_0, \ldots , \phi_p, \theta_1, \ldots , \theta_q, \alpha_0, \alpha_1, \ldots$\linebreak $\ldots , \alpha_P, 
\beta_1, \ldots , \beta_Q)$ такое же, как в~предыдущем случае. 
Оптимизационная задача имеет следующий вид:

\vspace*{-6pt}

\noindent
  \begin{multline}
  \hat{v}_n=\argmax\limits_{v\in \Theta} \ln (L_n(v))={}\\
  {}=\argmax\limits_{v\in \Theta} \sum\limits^T_{t=0} \left( 
  \vphantom{\fr{\varepsilon_t\sqrt{l(\alpha,\beta)}}{\sqrt{h_t}}}
  \ln \left( \sqrt{l(\alpha,\beta)}\right) -
\ln(B(\alpha,\beta)) +{}\right.\\
{}+ \alpha\overline{\omega}(\alpha,\beta)+ 
\fr{\alpha\varepsilon_t\sqrt{l(\alpha,\beta)}}{\sqrt{h_t}}
- \ln\left(\sqrt{h_t}\right)-{}\\
{} -(\alpha+\beta) \ln \left( 1+\exp \left( 
\fr{\varepsilon_t\sqrt{l(\alpha,\beta)}}{\sqrt{h_t}}
%+{}\right.\right.\\\left.\left.
\left.{}+\overline{\omega}(\alpha,\beta)
 \vphantom{\fr{\varepsilon_t\sqrt{l(\alpha,\beta)}}{\sqrt{h_t}}}
 \right)\right)\right)\!,\\
  \alpha,\beta,\alpha_0>0\,, \enskip \alpha_1,\ldots, \alpha_P,\beta_1,\ldots , \beta_Q\geq 0\,,
  \label{e9-dan}
  \end{multline}
\vspace*{-6pt}
  
\noindent
где $\Gamma(c)$~--- гамма-функция;
\begin{align*}
\overline{\omega}(\alpha,\beta)&=\left.\fr{d\ln\Gamma(c)}{dc}\right\vert_{c=\alpha}-
\left.\fr{d\ln\Gamma(c)}{dc}\right\vert_{c=\beta}\,;\\
l(\alpha,\beta)&=\left.\fr{d^2\ln\Gamma(c)}{dc^2}\right\vert_{c=\alpha} +
\left.\fr{d^2\ln\Gamma(c)}{dc^2}\right\vert_{c=\beta}\,;
\end{align*}
$$
\sum\limits^P_{i=1}\alpha_i+\sum\limits^Q_{j=1} \beta_j<1\,.
$$
  
\begin{figure*}[b] %fig1
\vspace*{1pt}
    \begin{center}  
  \mbox{%
 \epsfxsize=162.997mm 
 \epsfbox{dan-1.eps}
 }
\end{center}
\vspace*{-12pt}
\Caption{Графики ACF~(\textit{а}) и~PACF~(\textit{б}) для $Y_t\hm= \ln(S_t/S_{t-2})$ (DAX)}
%\end{figure*}
%\begin{figure*} %fig2
\vspace*{5pt}
    \begin{center}  
  \mbox{%
 \epsfxsize=162.997mm 
 \epsfbox{dan-2.eps}
 }
\end{center}
   \vspace*{-12pt}
  \Caption{Графики ACF~(\textit{а}) и~PACF~(\textit{б}) для $Y_t\hm= \ln (S_t/S_{t-2})$ (Light Sweet Crude Oil)}
  \end{figure*}

  Для случая нормального распределения использовались функции 
библиотеки rugarch среды~R~\cite{8-dan}.

\vspace*{-9pt}
   
\section{Калибровка параметров моделей ARIMA--GARCH}

\vspace*{-3pt}

  В табл.~1 и~2 представлены результаты оценки параметров моделей  
ARIMA--GARCH. Для обоих временн$\acute{\mbox{ы}}$х рядов был проведен Q-тест  
Льюн\-га--Бок\-са для разного числа лагов (нулевая гипотеза заключается 
в~отсутствии автокорреляций для первых~$k$~лагов), который показал, что 
автокорреляционные связи для рядов $\ln (S_t/S_{t-1})$ и~$S_t/S_{t-1}\hm-1$ 
отсутствуют с~вероятностью 99\% (значение статистики составило~30,123 для 
индекса DAX и~24,576 для Light Sweet Crude Oil, в~то время как критическое 
значение для $k\hm=30$~лагов и~уровня значимости~99\% равно~50,892). 
В~результате были рассмотрены ряды $\ln(S_t/S_{t-2})$  
и~$S_t/S_{t-2}\hm- 1$ и~на основе их коррелограмм ACF и~PACF (рис.~1 и~2) 
был сделан вывод о спецификации ARIMA части моделей ARIMA--GARCH.
{\looseness=-1

}

  Для коррелограмм индекса DAX характерна модель ARIMA$(\,(0,0,1)$ 
(ненулевое значение автокорреляции первого лага и~затухающая динамика 
значений частных автокорреляций). Коррелограммы цен на нефть определяют 
модель ARIMA$(\,2,0,0)$  (ненулевые значения частных автокорреляций двух 
первых лагов и~затухающее поведение автокорреляций). 
{\looseness=-1

}
  
  Коэффициенты всех трех моделей имеют одинаковый знак и~порядок. Все 
полученные коэффициенты моделей статистически значимы на уровне 
значимости~99\% (следует из $t$-кри\-те\-рия). Можно также заметить, что 
коэффициенты, соответствующие моделям с~распределениями ошибок EGB2 
и~JSU, почти идентичны.

  
\begin{figure*}[b] %fig3
\vspace*{1pt}
 \begin{center}  
 \mbox{%
 \epsfxsize=161.099mm 
 \epsfbox{dan-3.eps}
 }
\end{center}
\vspace*{-11pt}
\Caption{Абсолютные ошибки цен опционов CALL~(\textit{а}) и~PUT~(\textit{б})
 на индекс DAX со сроком 22~июня 2019~г.: \textit{1}~---  
ARIMA--GARCH-N; \textit{2}~--- 
ARIMA--GARCH-EGB2; \textit{3}--- ARIMA--GARCH-JSU }
%\end{figure*}
%\begin{figure*} %fig4
\vspace*{5pt}
    \begin{center}  
  \mbox{%
 \epsfxsize=162.999mm 
 \epsfbox{dan-5.eps}
 }
\end{center}
\vspace*{-11pt}
\Caption{Абсолютные ошибки цен опционов CALL~(\textit{а}) и~PUT~(\textit{б})
 на индекс DAX со сроком 22~декабря 2023~г.: \textit{1}~--- 
ARIMA--GARCH-N; \textit{2}~--- 
ARIMA--GARCH-EGB2; \textit{3}--- ARIMA--GARCH-JSU}
\end{figure*}

  Предсказательная сила моделей оценивалась с~помощью информационных 
критериев Акаике (AIC) и~Байеса (BIC), статистики которых рассчитываются 
по следующим формулам:

\noindent
  \begin{align*}
    \mathrm{AIC} &= 2k-2\ln\left( L_n\!\left( \hat{v}\right)\right)\,;\\[6pt]
  \mathrm{BIC} &= k\ln(N) -2\ln\left( L_n\!\left( \hat{v}\right)\right)\,,
  %  \label{e10-dan}
  \end{align*}
где $N$~--- объем выборки; $k$~--- число параметров; $L_n(\hat{v})$~--- 
значение функции правдоподобия для найден\-ных оптимальных 
параметров~$\hat{v}$. Таблицы~1 и~2 показывают, что асимметричные 
распределения EGB2 и~JSU лучше моделируют ряд, чем нормальное 
распределение.


\section{Ценообразование опционов}

  В рамках данной работы справедливая стоимость опционов оценивалась 
  с~помощью метода Мон\-те-Кар\-ло в~соответствии с~формулами~(\ref{e6-dan}) 
и~(\ref{e7-dan}) по риск-ней\-т\-раль\-ным траекториям моделей  
ARIMA--GARCH~(\ref{e3-dan})--(\ref{e5-dan}), где число 
реализаций доходности базового актива $M\hm= 10\,000$. Эффективность 
каждой модели ARIMA--GARCH оценивалась по абсолютной ошибке (AE):
  \begin{multline*}
  \mathrm{АО}(\mathrm{Moneyness})={}\\
\!=\!\,\left\vert p^m_{\mathrm{call}}/p^m_{\mathrm{put}}
  \left( \mathrm{Moneyness}\right) -
p_{\mathrm{call}}/p_{\mathrm{put}}\left( \mathrm{Moneyness}\right)
  \!\,\right\vert\!,\hspace*{-5.9pt}
  %\label{e11-dan}
  \end{multline*}
где $p^m_{\mathrm{call}}/p^m_{\mathrm{put}}$~---  рыночные котировки опционов; 
$\mathrm{Moneyness}\hm=X/s_0$. 



  Абсолютные ошибки оценивания опционов иллюстрируются на рис.~3--6. 
Для опционов на индекс DAX наиболее близкие значения к~рыночным 
котировкам дает модель с~распределением ошибок JSU, модель 
с~нормальным распределением хуже всего находит стоимость опционов DAX. 
При этом стоит также отметить, что для опционов с~датой экспирации 
22~декабря 2023~г.\ расхождения между моделями становятся существенней, 
показывая неэффективность использования моделей с~нормальным 
распределением ошибок на длинных временн$\acute{\mbox{ы}}$х
горизонтах. 
\pagebreak

\end{multicols}


\begin{figure*} %fig5
\vspace*{.1pt}
    \begin{center}  
  \mbox{%
 \epsfxsize=162.238mm 
 \epsfbox{dan-7.eps}
 }
\end{center}
\vspace*{-12.5pt}
\Caption{Абсолютные ошибки цен опционов CALL~(\textit{а}) и~PUT~(\textit{б})
 Light Sweet Crude Oil (20~июня 2019~г.): \textit{1}~---  
ARIMA--GARCH-N; \textit{2}~--- 
ARIMA--GARCH-EGB2; \textit{3}--- ARIMA--GARCH-JSU}
%\end{figure*}
%\begin{figure*} %fig6
\vspace*{5pt}
    \begin{center}  
  \mbox{%
 \epsfxsize=161.542mm 
 \epsfbox{dan-9.eps}
 }
\end{center}
\vspace*{-12.5pt}
\Caption{Абсолютные ошибки цен опционов CALL~(\textit{а}) и~PUT~(\textit{б})
 Light Sweet Crude Oil (22~июня 2020~г.): \textit{1}~---  
ARIMA--GARCH-N; \textit{2}~--- 
ARIMA--GARCH-EGB2; \textit{3}--- ARIMA--GARCH-JSU}
\vspace*{-1pt}
\end{figure*}


\begin{multicols}{2}

Результаты 
оценки спра\-вед\-ли\-вой стои\-мости опционов на индекс DAX аналогичны 
результатам работы~\cite{7-dan}, в~которой среди рас\-смат\-ри\-ва\-емых опционов 
присутствовали опционы на индекс S\&P~500 и~делался вывод о~том, что 
модели ARIMA--GARCH с~ошибками, распределенными по закону EGB2, дают 
лучшие оценки стоимости опционов в~деньгах (для опционов CALL 
$\mathrm{Moneyness}\hm< 1$, для PUT $\mathrm{Moneyness}\hm>1$) по сравнению с~моделями, 
где ошибки распределены нормально. 
  
  Опционы на базовый актив Light Sweet Crude Oil также характеризуются 
существенным преимуществом модели JSU. Однако теперь уже модель 
с~распределением EGB2 показывает результаты хуже, чем модель с~нормальным 
распределением.
  
\vspace*{-7pt}

\section{Заключение}
\vspace*{-2pt}

  В данной статье были рассмотрены три альтернативные модели временн$\acute{\mbox{ы}}$х 
рядов с~условным средним ARIMA и~условной дисперсией GARCH 
структуры. Оценка справедливой стоимости опционов проводилась по методу  
Мон\-те-Кар\-ло~(\ref{e6-dan}), (\ref{e7-dan}) на основе риск-ней\-т\-раль\-ной 
динамики моделей в~соответствии с~формулами~(\ref{e3-dan})--(\ref{e5-dan}). 
Калибровка моделей осуществлялась методом 
квазимаксимального правдоподобия в~соответствии 
с~соотношениями~(\ref{e8-dan}) и~(\ref{e9-dan}). Определение порядка моделей 
было выполнено на основе Q-тес\-та  
Льюн\-га--Бок\-са и~графиков коррелограмм ACF и~PACF. 
  
  Результаты эмпирических исследований показывают, что на небольших 
временн$\acute{\mbox{ы}}$х промежутках модели обеспечивают близкие значения 
справедливой стоимости опционов. Для опционов с~дальней датой 
экспирации (больше года) модели показывают существенно разные 
результаты. Модель, построенная для распределения JSU, дает 
оценки стоимости, максимально близкие к~ценам закрытия биржевых торгов 
для всех рас\-смат\-ри\-ва\-емых опционных контрактов. 
  
{\small\frenchspacing
 {%\baselineskip=10.8pt
 %\addcontentsline{toc}{section}{References}
 \begin{thebibliography}{99}
  
\bibitem{1-dan}
\Au{Данилишин А.\,Р., Голембиовский~Д.\,Ю.} Риск-нейт\-раль\-ная динамика для модели 
ARIMA--GARCH с~ошиб\-ка\-ми, распределенными по закону $S_U$ Джонсона~// 
Информатика и~её применения, 2020. Т.~14. Вып.~1. С.~48--55.

\bibitem{5-dan} %2
\Au{Bollerslev T.} A~conditionally heteroskedastic time series model for speculative prices and 
rates of return~//  Rev. Econ. Stat., 1987. Vol.~69. Iss.~3. P.~542--547. doi: 10.2307/1925546.
\bibitem{3-dan} %3
\Au{Akgiray V.} Conditional heteroscedasticity in time series of stock returns: Evidence and 
forecasts~// J.~Bus., 1989. Vol.~62. Iss.1. P.~55--80. doi: 10.1086/296451.
\bibitem{4-dan} %4
\Au{Follmer H., Schied A.} Stochastic finance: An introduction in discrete time.~--- Berlin: 
Walter de Gruyter, 2002.\linebreak 422~p.
\bibitem{2-dan} %5
\Au{Terasvirta T.} An introduction to univariate GARCH models~//  Handbook of 
financial time series~/
Eds. T.\,G.~Andersen, R.\,A.~Davis, J.-P.~Kreiss, Th.\,V.~Mikosch.~--- Berlin--Heidelberg: 
Springer, 2009. Vol.~10.  
P.~17--42. doi: 10.1007/978-3-540-71297-8\_1.
\bibitem{6-dan} %6
\Au{Elliott R.\,J., Madan D.\,B.} A~discrete time equivalent martingale measure~// Math. 
Financ., 1998. Vol.~8. Iss.~2. P.~127--152. doi: 10.1111/1467-9965.00048.
\bibitem{7-dan}
\Au{Yi Xi.} Comparison of option pricing between ARMA-GARCH and GARCH-M models.~--- 
London, Ontario, Canada: University of Western Ontario, 2013. MoS Thesis. 73~p.
\bibitem{8-dan}
\Au{Christian F., Francq M.} GARCH models: Structure, statistical inference and financial 
applications.~--- New York, NY, USA: Wiley, 2019. 504~p.
\bibitem{9-dan}
\Au{Boyle T.} A~Monte Carlo approach~// J.~Financ. Econ., 2012. Vol.~4. P.~323--338. 
doi: 10.1016/0304-405x(77)90005-8.
\bibitem{10-dan}
\Au{Hull J.} Options, futures, and other derivatives.~--- 10th ed.~--- Pearson, 2018. 896~p.
\bibitem{11-dan}
\Au{Williams D.} Probability with martingales.~--- Cambridge: Cambridge University Press, 
1991. 251~p.
\bibitem{12-dan}
\Au{Bell D.} Transformations of measure on an infinite dimensional vector space~// Seminar on 
stochastic processes, 1990~/ Eds. \mbox{E.~{\ptb{\c{C}}}inlar}, P.\,J.~Fitzsimmons, 
R.\,J.~Williams.~--- Progress in probability book ser.~--- Boston, MA, USA:
Birkh$\ddot{\mbox{a}}$user, 1991. 
Vol.~24. P.~15--25. doi: 10.1007/978-1-4684-0562-0\_3.
\end{thebibliography}

 }
 }

\end{multicols}

\vspace*{-12pt}

\hfill{\small\textit{Поступила в~редакцию 01.10.19}}

\vspace*{6pt}

%\pagebreak

%\newpage

%\vspace*{-28pt}

\hrule

\vspace*{2pt}

\hrule

\vspace*{-2pt}

\def\tit{ESTIMATING THE FAIR VALUE OF~OPTIONS\\
 BASED~ON~ARIMA--GARCH MODELS WITH~ERRORS\\ DISTRIBUTED ACCORDING 
TO~THE~JOHNSON'S $S_U$ LAW}


\def\titkol{Estimating the fair value of options based on~ARIMA--GARCH models 
with~errors distributed according to~the~Johnson's $S_U$ law}


\def\aut{A.\,R.~Danilishin$^1$ and D.\,Yu.~Golembiovsky$^{1,2}$}

\def\autkol{A.\,R.~Danilishin and D.\,Yu.~Golembiovsky}

\titel{\tit}{\aut}{\autkol}{\titkol}

\vspace*{-11pt}


\noindent
$^1$Department of Operations Research, Faculty of Computational Mathematics and Cybernetics, 
M.\,V.~Lomonosov\linebreak
$\hphantom{^1}$Moscow State University,  
1-52~Leninskie Gory, Moscow 119991, GSP-1, Russian Federation

\noindent
$^2$Department of Banking, Sinergy University, 80-G~Leningradskiy Prosp., Moscow 125190, Russian 
Federation


\def\leftfootline{\small{\textbf{\thepage}
\hfill INFORMATIKA I EE PRIMENENIYA~--- INFORMATICS AND
APPLICATIONS\ \ \ 2020\ \ \ volume~14\ \ \ issue\ 4}
}%
 \def\rightfootline{\small{INFORMATIKA I EE PRIMENENIYA~---
INFORMATICS AND APPLICATIONS\ \ \ 2020\ \ \ volume~14\ \ \ issue\ 4
\hfill \textbf{\thepage}}}

\vspace*{6pt} 


\Abste{In continuation of the article ``Risk-neutral dynamics for the ARIMA--GARCH 
(autoregressive integrated moving average\,--\,generalized autoregressive conditional heteroskedasticity)
random process with errors distributed according to the Johnson's $S_U$ law,'' this 
paper presents the experimental results for the ARIMA--GARCH 
(autoregressive integrated moving average\,--\,generalized autoregressive conditional heteroskedasticity)
models with normal (N), exponential beta of the second type 
(EGB2), and $S_U$ Johnson (JSU) error distributions. The fair value of European 
options is estimated by the Monte-Carlo method based on the results obtained in the 
specified article by using the extended Girsanov principle. The parameters of the 
ARIMA--GARCH-N,  
ARIMA--GARCH-EGB2, and ARIMA--GARCH-JSU models were found by the 
quasi-maximum likelihood method. The efficiency of the resulting risk-neutral models 
was studied using the example of European exchange-traded options PUT and CALL 
on basic assets DAX and Light Sweet Crude Oil.}

\KWE{ARIMA; GARCH; risk-neutral measure; Girsanov extended principle; 
Johnson's $S_U$ distribution; option pricing}

\DOI{10.14357/19922264200412} 

%\vspace*{-20pt}

%\Ack
%\noindent


%\vspace*{-6pt}

  \begin{multicols}{2}

\renewcommand{\bibname}{\protect\rmfamily References}
%\renewcommand{\bibname}{\large\protect\rm References}

{\small\frenchspacing
 {%\baselineskip=10.8pt
 \addcontentsline{toc}{section}{References}
 \begin{thebibliography}{99}
\vspace*{-4pt}

\bibitem{1-dan-1}
\Aue{Danilishin, A.\,R., and D.\,Y. Golembiovsky.} 2020. Risk-neytral'naya dinamika 
dlya ARIMA--GARCH modeli
%\linebreak
%\vspace*{-12pt}
%\columnbreak
%\noindent
s~oshib\-ka\-mi, raspredelennymi po zakonu $S_U$ 
Dzhonsona. [Risk-neutral dynamics for ARIMA--GARCH random process with errors 
distributed according to the Johnson's $S_U$ law]. \textit{Informatika i~ee 
Primeneniya~--- Inform. Appl.} 14(1):56--62.

\bibitem{5-dan-1} %2
\Aue{Bollerslev, T.} 1987. A~conditionally heteroskedastic time series model for 
speculative prices and rates of return. \textit{Rev. Econ. Stat.} 69(3):542--547. doi: 
10.2307/ 1925546.
{\looseness=1

}
\bibitem{3-dan-1} %3
\Aue{Akgiray, V.} 1989. Conditional heteroscedasticity in time series of stock returns: 
Evidence and forecasts. \textit{J.~Bus.} 62(1):55--80. doi: 10.1086/296451.
\bibitem{4-dan-1} %4
\Aue{Follmer, H., and A. Schied.} 2002. \textit{Stochastic finance: An introduction in 
discrete time}. Berlin: Walter de Gruyter. 422~p.

\bibitem{2-dan-1} %5
\Aue{Terasvirta, T.} 2009. An introduction to univariate GARCH models. 
\textit{Handbook of financial time series}. Eds. T.\,G.~Andersen, R.\,A.~Davis, 
J.-P.~Kreiss, and Th.\,V.~Mikosch. Berlin--Heidelberg: Springer. 10:17--42. doi:  
10.1007/978-3-540-71297-8\_1.

\bibitem{6-dan-1}
\Aue{Elliott, R.\,J., and D.\,B. Madan.} 1998. A~discrete time equivalent martingale 
measure. \textit{Math. Financ.} 8(2):127--152. doi: 10.1111/1467-9965.00048.
\bibitem{7-dan-1}
\Aue{Yi, Xi.} 2013. Comparison of option pricing between ARMA--GARCH and 
GARCH-M models. London, Ontario, Canada: University of Western Ontario. MoS 
Thesis. 73~p.
\bibitem{8-dan-1}
\Aue{Christian, F., and M. Francq.} 2019. \textit{GARCH models: Structure, 
statistical inference and financial applications.} New York, NY: Wiley. 504~p.
\bibitem{9-dan-1}
\Aue{Boyle, P.} 2012. Options: A~Monte Carlo approach. \textit{J.~Financ. 
Econ.} 4(3):323--338.  doi: 10.1016/0304-405x(77)90005-8.
\bibitem{10-dan-1}
\Aue{Hull, J.} 2018. \textit{Options, futures, and other derivatives}. 10th ed. Pearson. 
896~p.
\bibitem{11-dan-1}
\Aue{Williams, D.} 1991. \textit{Probability with martingales}. Cambridge: 
Cambridge University Press. 251~p.
\bibitem{12-dan-1}
\Aue{Bell, D.} 1991. Transformations of measure on an infinite dimensional vector 
space. \textit{Seminar on stochastic processes, 1990}. Eds. \mbox{E.~{\ptb{\c{C}}}inlar}, 
P.\,J.~Fitzsimmons, and R.\,J.~Williams. Progress in probability book ser. Boston, MA: 
Birkh$\ddot{\mbox{a}}$user. 
24:15--25. doi: 10.1007/978-1-4684-0562-0\_3.
\end{thebibliography}

 }
 }

\end{multicols}

\vspace*{-3pt}

\hfill{\small\textit{Received October 1, 2019}}

%\pagebreak

%\vspace*{-24pt}


\Contr

\noindent
\textbf{Danilishin Artem R.} (b.\ 1992)~--- PhD student, Department of Operations Research, Faculty of 
Computational Mathematics and Cybernetics, M.\,V.~Lomonosov Moscow State University, 1-52~Leninskie 
Gory, GSP-1, Moscow 119991, Russian Federation; \mbox{danilishin-artem@mail.ru}

\vspace*{3pt}

\noindent
\textbf{Golembiovsky Dmitry Y.} (b.\ 1960)~--- Doctor of Science in technology, professor, Department of 
Operation Research, Faculty of Computational Mathematics and Cybernetics, M.\,V.~Lomonosov Moscow 
State University, 1-52~Leninskie Gory, GSP-1, Moscow 119991, Russian Federation; professor, Department 
of Banking, Sinergy University, 80-G~Leningradskiy Prosp., Moscow 125190, Russian Federation; 
\mbox{golemb@cs.msu.su}
\label{end\stat}

\renewcommand{\bibname}{\protect\rm Литература} 
      
 