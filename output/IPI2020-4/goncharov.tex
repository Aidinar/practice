

\def\stat{goncharov}

\def\tit{ЭВОЛЮЦИЯ КЛАССИФИКАЦИЙ\\ В~НАДКОРПУСНЫХ 
БАЗАХ ДАННЫХ$^*$}

\def\titkol{Эволюция классификаций в~надкорпусных базах данных}

\def\aut{А.\,А.~Гончаров$^1$, И.\,М.~Зацман$^2$, М.\,Г.~Кружков$^3$}

\def\autkol{А.\,А.~Гончаров, И.\,М.~Зацман, М.\,Г.~Кружков}

\titel{\tit}{\aut}{\autkol}{\titkol}

\index{Гончаров А.\,А.}
\index{Зацман И.\,М.}
\index{Кружков М.\,Г.}
\index{Goncharov A.\,A.}
\index{Zatsman I.\,M.}
\index{Kruzhkov M.\,G.}


{\renewcommand{\thefootnote}{\fnsymbol{footnote}} \footnotetext[1]
{Работа выполнена в~Институте проблем информатики ФИЦ ИУ РАН при поддержке РФФИ (проекты 
18-07-00192  
и~20-012-00166).}}


\renewcommand{\thefootnote}{\arabic{footnote}}
\footnotetext[1]{Институт проблем информатики Федерального исследовательского центра 
<<Информатика и~управление>> Российской академии наук, \mbox{a.gonch48@gmail.com}}
\footnotetext[2]{Институт проблем информатики Федерального исследовательского центра 
<<Информатика и~управление>> Российской академии наук, \mbox{izatsman@yandex.ru}}
\footnotetext[3]{Институт проблем информатики Федерального исследовательского центра 
<<Информатика и~управление>> Российской академии наук, \mbox{magnit75@yandex.ru}}

\vspace*{-9pt}

\Abst{Рассматривается задача фиксации изменений, вносимых в~описания значений 
немецких модальных глаголов в~процессе аннотирования параллельных немецко-русских 
текстов с~использованием надкорпусной базы данных (НБД). На примере этой задачи 
проанализирована специфика использования динамической классификационной системы 
(ДКС) в~информационных системах. Специфика ДКС состоит в~том, что смысловое 
содержание ее рубрик может меняться в~процессе аннотирования, а~это нередко влечет за 
собой потребность в~реклассификации ранее сформированных аннотаций с~измененными 
впоследствии рубриками. Основная цель статьи состоит в~поиске ответов на два вопроса: 
(1)~от каких факторов зависит необходимость редактирования и~реклассификации 
аннотаций, сформированных до изменения рубрик; (2)~с~по\-мощью каких операций 
можно вносить изменения в~дефиниции рубрик ДКС. В~статье определены семь типов 
возможных изменений дефиниций и~описаны соответствующие им операции, применяемые 
к~рубрикам ДКС в~процессе аннотирования. Операции распределены на три группы 
в~зависимости от того, требуют ли они программной или экспертной реклассификации ранее 
сформированных аннотаций.}

      \KW{динамическая классификация; фасетная классификация; реклассификация; 
надкорпусная база данных}
      
\DOI{10.14357/19922264200415} 
  
\vspace*{-3pt}

\vskip 10pt plus 9pt minus 6pt

\thispagestyle{headings}

\begin{multicols}{2}

\label{st\stat}
      
\section{Введение}

     Современные компьютерные технологии и~информационные ресурсы 
служат основой развития многих направлений лингвистической науки, 
в~частности корпусной лингвистики~[1--4]. Одним из таких ресурсов стали 
НБД (см.~[5], где впервые было использовано это 
понятие), создаваемые в~ИПИ ФИЦ ИУ РАН.
     
Надкорпусные базы данных можно определить как <<``надстроенный'' над корпусом [текстов] 
информационный ресурс, позволяющий последовательно фиксировать$\ldots$ 
употребления исследуемых языковых единиц (ЯЕ), снабжая их описаниями, 
структурированными в~соответствии с~задачами исследования>>~[6]. Процесс 
описания употреблений ЯЕ называется лингвистическим аннотиро\-ванием~[7], 
а~результатом аннотирования одного употребления ЯЕ становится аннотация, 
спектр исполь\-зуемых видов которых описан в~[8]. Анно\-тация содержит 
информацию о~том, к~каким руб\-ри\-кам фасетной классификации~[9, 10] 
относится некоторое употребление ЯЕ. Структурированность аннотаций 
позволяет использовать их для получения разнообразных статистических 
данных~[11, 12].
     
     Одна из НБД была разработана для исследо\-вания немецких модальных 
глаголов (НМГ),\linebreak которые характеризуются развитой полисемией~--- имеют три 
и~более значений, согласно~\cite{13-gon}. Соответствующая база данных далее 
обозначается как НБД НМГ. Она обеспечивает аннотирование употреблений 
НМГ, их переводов на русский язык и~переводных соответствий в~целом. 
Материалом исследования стали тексты параллельного немецкого подкорпуса 
Национального корпуса русского языка ({\sf 
https://ruscorpora.ru/new/index.html}) для направления перевода <<не\-мец\-кий--рус\-ский>>
 общим объемом более 2,6~млн словоупотреблений.
     
     В процессе аннотирования (1)~отбираются пары фрагментов 
параллельных текстов, где в~оригинале употреблен модальный глагол; 
(2)~отобранные фрагменты оригинала и~его перевода копируются в~аннотацию 
(табл.~1, первый и~третий столбцы) и~снабжаются рубриками фасетной 
классификации для оригинала, перевода и~переводного соответствия 
(табл.~1, второй, четвертый и~пятый столбцы);\linebreak
\vspace*{-12pt}

\pagebreak

\end{multicols}

\begin{table*}\small %tabl1
\begin{center}
\Caption{Пример аннотации, сформированной в~НБД НМГ}
\vspace*{2ex}

\begin{tabular}{|p{30mm}|p{53pt}|p{126pt}|p{94pt}|p{15mm}|}
\hline
\multicolumn{1}{|c|} 
{\tabcolsep=0pt\begin{tabular}{c}Контекст НМГ\\ в~оригинале\end{tabular}}&
\multicolumn{1}{c|} {\tabcolsep=0pt\begin{tabular}{c} 
Коды рубрик\\ оригинала\end{tabular}}&
\multicolumn{1}{c|} {\tabcolsep=0pt\begin{tabular}{c}
Перевод\\ на русский язык\end{tabular}}&
\multicolumn{1}{c|} {\tabcolsep=0pt\begin{tabular}{c}
Коды рубрик\\ перевода\end{tabular}}&
\multicolumn{1}{c|} {\tabcolsep=0pt\begin{tabular}{c}
Коды рубрик\\ переводного\\ соответствия\end{tabular}}\\
\hline
\textbf{Soll} ich Gef$\ddot{\mbox{u}}$hle, einen Glauben \textit{heucheln}, die ich nicht habe?\newline
[H.~B$\ddot{\mbox{o}}$ll. Ansichten eines Clowns (1963)]&\textbf{sollen}\newline 
$\langle$sollen-10$\rangle$\newline
$\langle$Praes$\rangle$\ %newline
$\langle$1sg$\rangle$\newline
$\langle$+Inf I$\rangle$\newline
$\langle$Inversion$\rangle$\newline
$\langle$Interrog$\rangle$&--- \textbf{Хотите}, чтобы я притворялся верующим, изображая чувства, 
которых у~меня нет?\newline
[Г.~Белль. Глазами клоуна (пер.\ Л.\,Б.~Черная, 1964)]&
\hspace*{-1pt}{\raisebox{-2pt}{\epsfxsize=2.5mm
\epsfbox{gon-1.eps}}}
$\langle$хотеть$\rangle$\newline
\hspace*{-1pt}{\raisebox{-2pt}{\epsfxsize=2.5mm
\epsfbox{gon-2.eps}}}
$\langle$Praes$\rangle$\newline
\hspace*{-1pt}{\raisebox{-2pt}{\epsfxsize=2.5mm
\epsfbox{gon-2.eps}}}
$\langle$Вы$\rangle$\newline
\mbox{\hspace*{-1pt}{\raisebox{-2pt}{\epsfxsize=2.5mm
\epsfbox{gon-2.eps}}}~$\langle$+Subord\;+\;<<чтобы>>$\rangle$}\newline 
\hspace*{-1pt}{\raisebox{-2pt}{\epsfxsize=2.5mm
\epsfbox{gon-2.eps}}}
$\langle$Interrog$\rangle$&SubjCh\\
\hline
\multicolumn{5}{p{463pt}} %158mm
{\footnotesize \textbf{Примечания.}\newline
Расшифровка кодов рубрик оригинала:\newline
\vspace*{-9pt}
\begin{itemize}
\addtolength{\itemsep}{-3pt}
\item \textbf{sollen}~--- в~данном контексте исследуемым модальным глаголом 
немецкого языка является <<sollen>>;
\item $\langle$sollen-10$\rangle$~--- глагол <<sollen>> употреблен в~десятом 
значении согласно порядку описания значений в~словарной статье в~\cite{13-gon};
\item $\langle$Praes$\rangle$~--- глагол употреблен в~форме настоящего времени 
изъявительного наклонения;
\item $\langle$1sg$\rangle$~--- глагол употреблен в~форме первого лица 
единственного числа;
\item $\langle$+Inf~I$\rangle$~--- глаголу подчинен инфинитив настоящего времени 
(см.\ слово <<heucheln>> в~оригинале, выделенное курсивом);
\item $\langle$Inversion$\rangle$~--- инверсия подлежащего и~сказуемого, т.\,е.\ 
сказуемое~--- в~данном случае модальный глагол~--- занимает начальную позицию 
в~предложении;
\item $\langle$Interrog$\rangle$~--- глагол употреблен в~вопросительном 
предложении.
\end{itemize}
%
Расшифровка кодов рубрик перевода:
\begin{itemize}
\addtolength{\itemsep}{-3pt}
\item \hspace*{-1pt}{\raisebox{-2pt}{\epsfxsize=2.5mm
\epsfbox{gon-1.eps}}}
 $\langle$хотеть$\rangle$~--- носителем модального значения, передающего 
значение немецкого модального глагола в~переводе на русский язык, служит глагол 
<<хотеть>> (в~аннотации символом <<\hspace*{-1pt}{\raisebox{-2pt}{\epsfxsize=2.5mm
\epsfbox{gon-1.eps}}}>> отмечаются носители модального 
значения в~русском языке);
\item \hspace*{-1pt}{\raisebox{-2pt}{\epsfxsize=2.5mm
\epsfbox{gon-2.eps}}}
 $\langle$Praes$\rangle$~--- глагол употреблен в~форме настоящего времени 
изъявительного наклонения;
\item \hspace*{-1pt}{\raisebox{-2pt}{\epsfxsize=2.5mm
\epsfbox{gon-2.eps}}}
 $\langle$Вы$\rangle$~--- глагол употреблен в~форме второго лица 
множественного числа для вежливого обращения к~одному лицу;
\item \hspace*{-1pt}{\raisebox{-2pt}{\epsfxsize=2.5mm
\epsfbox{gon-2.eps}}}
 $\langle$+Subord\;+\;<<чтобы>>$\rangle$~--- глаголу подчинено 
придаточное предложение, присоединенное с~помощью союза <<чтобы>> (<<чтобы я 
притворялся верующим$\ldots$>>);
\item \hspace*{-1pt}{\raisebox{-2pt}{\epsfxsize=2.5mm
\epsfbox{gon-2.eps}}}
 $\langle$Interrog$\rangle$~--- глагол употреблен в~вопросительном 
предложении.
\end{itemize}
%
Расшифровка кодов рубрик переводного соответствия:
\begin{itemize}
\item SubjCh~--- при переводе подлежащее при модальном глаголе было изменено 
(в~оригинале~--- <<должен ли~я>>, в~переводе~--- <<хотите [ли Вы]>>).
\end{itemize}
}
\end{tabular}
\end{center}
\vspace*{-17pt}
\end{table*}

\begin{multicols}{2}

\noindent
(3)~затем сформированная 
аннотация записывается в~НБД НМГ (подробнее о~методологии 
аннотирования в~НБД см.~\cite{6-gon}).
     
     Примерами фасетов, которые используются для аннотирования 
немецкоязычного оригинала, служат:
     \begin{enumerate}[(1)]
     \item грамматическое время, например: $\langle$Praes$\rangle$~--- 
настоящее время изъявительного наклонения; $\langle$Praet$\rangle$~--- 
простое прошедшее время изъявительного наклонения и~т.\,д.;
\item форма лица и~числа, например: $\langle$1sg$\rangle$~--- 1-е лицо 
ед.\ числа; $\langle$1pl$\rangle$~--- 1-е лицо множ.\ числа; 
$\langle$2sg$\rangle$~--- 2-е лицо ед.\ числа и~т.\,д.;
\item значение НМГ: в~качестве отправной точки в~этот фасет были 
внесены все описания значений рассматриваемых модальных глаголов 
в~соответствии  
с~не\-мец\-ко-рус\-ским словарем~\cite{13-gon}, например:  
$\langle$sollen-01$\rangle$~--- первое значение глагола <<sollen>>; 
$\langle$m$\ddot{\mbox{u}}$ssen-01$\rangle$~--- первое значение 
глагола <<m$\ddot{\mbox{u}}$ssen>>;  
$\langle$m$\ddot{\mbox{u}}$ssen-02$\rangle$~--- второе значение 
глагола <<m$\ddot{\mbox{u}}$ssen>> и~т.\,д.
\end{enumerate}

Классификационная система, используемая в~процессе аннотирования, 
отражает состояние знания лингвистов на определенные моменты времени. 
В~процессе работы их знания могут меняться, что иногда влечет 
и~перераспределение классифицируемых объектов по классам. Все это 
необходимо отражать в~самой классификационной системе.

Цель настоящей статьи заключается в~описании (1)~специфики динамических 
классификаций; (2)~типов возможных изменений рубрик этих классификаций 
на примере фасета <<Значения немецких модальных глаголов>> (<<Значения 
НМГ>>); (3)~связанных с~такими изменениями потребностей 
в~редактировании ранее сформированных аннотаций и~их 
реклассификации\footnote{О классификации и, соответственно, реклассификации 
аннотаций представляется возможным говорить потому, что совокупности рубрик, 
присваиваемых аннотациям, представляют собой классификационные коды (подобные кодам 
Универсальной десятичной классификации), отражающие содержание оригинала, перевода и~переводного соответствия этих 
аннотаций. Так, классификационный код аннотации из табл.~1 будет включать коды рубрик 
из столбцов~2, 4 и~5 этой таблицы. Следовательно, при изменении дефиниции одной из 
рубрик соответствующим образом изменятся содержание аннотации и, возможно, ее код.}. 
Средства отражения динамики этих изменений в~НБД обсуждаются 
в~минимальной степени, так как их планируется детально рассмотреть 
в~отдельной статье.

\vspace*{-6pt}

\section{Эволюция классификационных систем 
и~реклассификация}

\vspace*{-2pt}

 Классификационные системы делятся на две\linebreak категории~--- 
\textit{статические} и~\textit{динамические}~--- в~зависимости от 
потенциальной возможности изменения рубрик на рассматриваемом интервале 
времени. Рассмотрим различия между ними на примере\linebreak лингвистического 
аннотирования ЯЕ, выполня\-емого в~течение заданного 
интервала времени.
     
     \textit{Статическая классификационная система} (СКС) остается 
неизменной на протяжении всего заданного интервала времени и~должна 
содержать полный перечень необходимых рубрик до начала аннотирования. 
Если же при аннотировании возникает потребность включить в~СКС новые 
рубрики и/или изменить существующие, то это можно сделать только по 
окончании заданного интервала времени. Поэтому до его окончания неполнота 
СКС может привести к~появлению незавершенных аннотаций или аннотаций 
с~нерелевантными рубриками, которые не позволяют адекватно представить 
аннотируемый материал.
     
     Чтобы избежать появления незавершенных и~нерелевантных аннотаций, 
можно использовать \textit{динамическую классификационную систему}, 
которая допускает внесение изменений в~существующие\linebreak руб\-ри\-ки 
и~добавление новых рубрик в~течение интервала времени аннотирования. 
Использование ДКС обусловлено динамикой знания лингвистов в~процессе 
проведения ими исследований в~об\-ласти корпусной семантики. Поскольку 
НБД ориентированы именно на отражение динамики, то концепция, лежащая 
в~основе проектирования НБД, в~общем случае предполагает использование 
фасетных динамических классификаций, которые можно редактировать 
в~процессе лингвистического аннотирования~\cite{8-gon, 9-gon}.
      
     Эксперимент по аннотированию значений НМГ показал, что знание 
лингвистов об этих значениях действительно достаточно часто меняется 
в~процессе работы с~НБД~\cite{14-gon}. Хотя большинство используемых при 
аннотировании фасетов классификации стабильны и~не изменяются 
в~процессе работы (грамматическое время, лицо, число и~т.\,д.), фасет, 
обеспечивающий аннотирование семантики употребления и~значений НМГ, 
таковым не является. Рубрики именно этого фасета находятся в~центре 
внимания проводимых с~использованием НБД исследований, так как 
<<уточнение номенклатуры значений немецких модальных глаголов и~условий 
их реализации>> обозначено в~качестве одной из целей создания НБД 
НМГ~\cite[с.~173]{15-gon} (подробнее об аннотировании НМГ 
см.~\cite[с.~175--181]{15-gon}).
     
     Лингвисты в~ходе аннотирования текстов параллельного немецкого 
подкорпуса нередко сталкиваются в~этих текстах с~примерами использования 
модальных глаголов, которые не соотносятся ни с~одним из ранее описанных 
значений этих глаголов. Поэтому в~интересах завершенности аннотирования 
они вырабатывают новые или уточняют существующие дефиниции значений 
прямо в~процессе семантического анализа таких примеров, используя ДКС. 
Иными словами, знание лингвистов об исследуемых ЯЕ 
эволюционирует в~процессе аннотирования, что влечет за собой изменение 
дефиниций рубрик ДКС. Результатом ее применения становятся не только 
более точные и~завершенные аннотации, но и~обновленная ДКС, отражающая 
обнаруженное в~ходе аннотирования и~эксплицированное новое 
лингвистическое знание.
     
     Существенная особенность использования ДКС заключается в~том, что 
изменения дефиниций руб\-рик могут потребовать редактирования 
соответствующих полей тех аннотаций, которые включали рубрики ДКС 
с~предыдущими версиями их дефиниций, что неизбежно замедлит процесс 
аннотирования. Такое редактирование по сути представляет собой 
\textit{реклассификацию} ранее сформированных аннотаций, так как меняются 
их классификационные коды.
     
     Реклассификация может быть определена как \textit{процесс перехода от 
одной классификационной сис\-те\-мы к~другой сис\-те\-ме или к~новой версии той 
же сис\-те\-мы}. Проблема реклассификации была впервые осознана, видимо,  
в~ин\-фор\-ма\-ци\-он\-но-биб\-лио\-теч\-ной науке. Она активно обсуждалась 
уже в~первой половине XX~в., тогда как <<в том или ином виде 
реклассификация существовала с~появления каталогизации и~самой 
классификации>>~\cite[с.~249]{16-gon}. Причинами перехода на новую 
классификацию или новую версию старой классификации могут быть 
<<недостаточная приемлемость используемой классификационной системы 
и~появление более приемлемой классификационной 
системы>>~\cite[с.~83]{17-gon}.

\begin{table*}[b]\small %tabl2
\begin{center}
\Caption{Фрагмент списка рубрик фасета <<Значения НМГ>> в~НБД (на 23.04.2020)}
\vspace*{2ex}

\begin{tabular}{|c|c|p{368pt}|}
\hline
\tabcolsep=0pt\begin{tabular}{c}Id\\ 
рубрики\end{tabular}&\tabcolsep=0pt\begin{tabular}{c}Код\\ рубрики 
\end{tabular}&\multicolumn{1}{c|}{Дефиниция рубрики}\\
\hline
482&sollen-01&Обязанность что-л.\ делать по чьему-л.\ указанию, по закону, по правилам 
и~т.\,п.: должен. Моральный запрет (под отрицанием): нельзя\\
\hline
483&sollen-02&Совет или требование, исходящее от других лиц, а~также пожелание самого 
говорящего (в~придаточных дополнительных, также в~формах conj; часто в~неполных 
синтаксических конструкциях, как правило, с~опущенным глаголом перемещения)\\
\hline
484&sollen-03&Желательность по мнению говорящего (в формах praet conj и~pqp conj): 
следовало (бы), нужно было (бы), должно было (бы). Совет, рекомендация (только в~формах 
praet conj). Нежелательность (под отрицанием): не следовало (бы), нельзя\\
\hline
485&sollen-04&Передача чужого мнения, приводимого со слов других лиц (в формах praes): 
говорят, полагают\\
\hline
486&sollen-05&Отнесенность действия к~будущему, снятие категоричности (в формах praes 
ind; также в~вопросительных предложениях). Косвенный императив. Неуверенное согласие. 
Запрос информации для выполнения последующего действия (в~вопросительных 
предложениях без вопросительного слова). Угроза и~возмущение\\
\hline
\end{tabular}
\end{center}
%\vspace*{-3pt}
\end{table*}
     
     Сегодня проблема реклассификации вышла\linebreak далеко за рамки  
ин\-фор\-ма\-ци\-он\-но-биб\-лио\-теч\-ной науки. Наглядным примером ДКС 
может служить Международная патентная классификация (МПК). Чтобы 
избежать реклассификации изобретений из-за ускорения роста и~динамики 
технического знания, начиная с~2006~г.\ при простановке рубрики МПК 
в~описаниях изобретений одновременно указывается дата (год и~месяц), когда 
она была согласована экспертами и~утверждена Международным бюро 
Всемирной организации интеллектуальной собственности (МБ ВОИС)~\cite{18-gon}. 
Однако в~линг\-вистическом аннотировании применение такого подхода 
вряд ли возможно, так как в~лингвистике и~лексикографии нет 
международного директивного органа, аналогичного МБ ВОИС.
     
     В 2008~г.\ К.~Ньоли опубликовал статью <<Ten Long-Term Research 
Questions in Knowledge Organization>>, где были сформулированы 10 
актуальных вопросов, относящихся к~проблематике организации знания, поиск 
ответов на которые, с~его точки зрения, важен в~долгосрочной перспективе. 
Седьмой из поставленных Ньоли вопросов~--- <<Как справиться с~проблемой 
изменения знания при организации знания?>> (<<How can KO deal with changes 
     in knowledge?>>)~\cite[с.~142--143]{19-gon}. Поскольку одним из средств 
организации знания служат классификационные системы, вопрос, заданный 
Ньоли, предполагает и~поиск ответа на вопрос: <<Как справиться с~проблемой 
изменения знания при его отражении с~использованием классификационных 
рубрик?>>
     
     Проблема роста и~изменения знания, влеку\-щая за собой пересмотр 
классификационных\linebreak сис\-тем и~предполагающая разработку методов 
реклассификации, актуальна для широкого спектра отраслей знания, в~которых 
создаются и~используются классификационные системы и~онтологии для 
индексирования и~рубрицирования объектов исследования и/или их описаний. 
В~информатике с~этой проблемой сталкиваются, например, при создании баз 
данных и~баз знаний с~динамическими классификационными системами или 
онтологиями~\cite{20-gon}. Эти системы и~онтологии устаревают со временем, 
и~поэтому они, как правило, регулярно обновляются, чтобы отразить рост 
и~динамику научного знания. Подобное обновление нередко влечет 
необходимость реклассификации тех объектов исследования и/или их 
описаний, которым индексы или рубрики были присвоены ранее.
     
     Как было отмечено выше, необходимость в~обновлении используемой 
классификационной сис\-те\-мы проявилась и~при развитии НБД НМГ. При этом 
ряд операций по изменению дефиниций рубрик ДКС НБД, а~следовательно, по 
об\-нов\-ле\-нию фасетной классификации может требовать реклассификации ранее 
сформированных аннотаций, в~которых использовались рубрики, измененные 
впоследствии. Рассмотрим типы возможных изменений рубрик ДКС НБД, 
которая, как отмечено выше, является фасетной классификацией.

\vspace*{-6pt}


\vspace*{-6pt}

\section{Типы изменений рубрик}
\vspace*{-2pt}

 
     Каждая рубрика ДКС НБД имеет: (1)~уникальный идентификатор; (2)~код; 
(3)~дефиницию. В~табл.~2 приводится фрагмент списка рубрик фасета 
<<Значения НМГ>>, соответствующих первым пяти значениям глагола 
<<sollen>> согласно словарю~\cite{13-gon}. Столбец~1 содержит уникальные 
неизменяемые идентификаторы рубрик. Столбец~2 содержит коды рубрик: для 
фасета <<Значения НМГ>> они отражают порядок, в~котором описания 
значений слова приводятся в~рамках словарной статьи в~\cite{13-gon}. Эти 
коды могут меняться при изменении порядка описания значений. Столбец~3 
содержит дефиниции рубрик. Так, исходные версии дефиниций рубрик фасета 
<<Значения НМГ>> были сформулированы на основании описаний значений 
модальных глаголов в~\cite{13-gon} и~внесены в~НБД до начала 
аннотирования в~сокращенной (по сравнению  
с~\cite{13-gon}) и~структурированной форме. Структура дефиниции рубрики 
фасета <<Значения НМГ>> такова: $\langle$стилистические пометы$\rangle$, $\langle$описание 
значения НМГ$\rangle$, $\langle$комментарий к~употреблению НМГ$\rangle$, $\langle$варианты перевода 
на русский язык$\rangle$. Другие зоны, присутствующие в~описании значения 
в~словаре,~--- зона примеров, зона фразеологии и~т.\,д.~--- в~НБД на данный 
момент не отражаются.
     

     
     Так, дефиниция рубрики~482 из табл.~2 состоит из двух блоков, и~ее 
структура может быть эксплицирована следующим образом.
     
Блок 1:
\begin{enumerate}[{1.}1]
\item Стилистические пометы:~--- (здесь отсутствуют).
\item Описание значения НМГ: \textit{Обязанность что-л.\ делать по 
чьему-л.\ указанию, по закону, по правилам и~т.\,п.}
\item Комментарий к~употреблению НМГ:~--- (здесь отсутствует).
\item Варианты перевода на русский язык: \textit{должен}.
\end{enumerate}

Блок 2:
\begin{enumerate}[2.1]
\item Стилистические пометы:~--- (здесь отсутствуют).
\item Описание значения НМГ: \textit{Моральный запрет}.
\item Комментарий к~употреблению НМГ: \textit{под отрицанием}.
\item Варианты перевода на русский язык: \textit{нельзя}.
\end{enumerate}

     В результате наблюдения за процессом лингвистического аннотирования 
было выделено~7~типов изменений рубрик фасетной классификации. Чтобы 
отражать эти изменения в~НБД, была преду\-смотрена возможность применения 
со\-от\-вет\-ст\-ву\-ющих~7~операций к~рубрикам, входящим в~фасет <<Значения 
НМГ>>. Для их описания введем сле\-ду\-ющие обозначения:
\begin{itemize}
\item X, Y, Z, \ldots~--- рубрики фасета <<Значения НМГ>>;
\item def$_{\mathrm{X}}$, def$_{\mathrm{Y}}$, def$_{\mathrm{Z}}$, 
\ldots~--- дефиниции рубрик фасета <<Значения НМГ>> (def~--- сокр.\ от 
англ.\ \textit{definition});
\item $\mathrm{S_{def_X}}$, $\mathrm{S_{def_Y}}$, 
$\mathrm{S_{def_Z}}$, \ldots~--- смысловое содержание дефиниций или 
семантика рубрик фасета <<Значения НМГ>> (S~---  сокр.\ от англ.\ 
\textit{semantics}).
\end{itemize}

Операции, применяемые к~рубрикам фасета <<Значения НМГ>>:
\begin{enumerate}[(1)]
\item CREATE~--- создание новой рубрики~X;
\item REORDER~--- изменение кода рубрики~X;
\item REVISE~--- изменение def$_{\mathrm{X}}$ без 
сужения~$\mathrm{S_{def_X}}$, при условии что изменение 
def$_{\mathrm{X}}$ не связано с~изменением дефиниций других рубрик;
\item MERGE~--- слияние def$_{\mathrm{X}}$ и~def$_{\mathrm{Y}}$, 
при котором def$_{\mathrm{Y}}$ включается в~def$_{\mathrm{X}}$, 
после чего рубрика~${\mathrm{Y}}$ удаляется;
\item DELETE~--- удаление рубрики~X;
\item SPLIT~--- разделение def$_{\mathrm{X}}$ на две части~--- 
def$_{\mathrm{X}}$1 и~def$_{\mathrm{X}}$2, в~результате которого 
def$_{\mathrm{X}}$1 становится новой дефиницией рубрики~X 
и~создается новая рубрика~Y, дефиницией которой становится 
def$_{\mathrm{X}}$2;
\item REDISTR (сокр.\ от англ.\ \textit{redistribute})~--- изменение 
def$_{\mathrm{X}}$ и~def$_{\mathrm{Y}}$, при котором происходит 
перераспределение смыслового содержания 
между~$\mathrm{S_{def_X}}$ и~$\mathrm{S_{def_Y}}$.
     \end{enumerate}
     
     В рамках НБД НМГ операции~4, 6 и~7 могут быть применены максимум 
к~двум рубрикам. Если требуется объединить более двух дефиниций, разделить 
одну дефиницию более чем на две или перераспределить смысловое 
содержание более двух дефиниций, то это можно сделать путем 
последовательного выполнения нескольких операций (соответственно MERGE, 
SPLIT или REDISTR).
     
\begin{table*}\small %tabl3
\begin{center}
\Caption{Пример выполнения операции REVISE}
\vspace*{2ex}

\begin{tabular}{|c|c|l|c|}
\hline
\tabcolsep=0pt\begin{tabular}{c}Id\\ рубрики\end{tabular}&
\tabcolsep=0pt\begin{tabular}{c}Код\\ рубрики\end{tabular}&
\multicolumn{1}{c|}{Дефиниция рубрики}&
\tabcolsep=0pt\begin{tabular}{c}Время\\ записи\end{tabular}\\
\hline
485&sollen-04&\tabcolsep=0pt\begin{tabular}{l}
Передача {\bfseries\textit{чужого}} мнения, {\bfseries\textit{приводимого со 
слов}} других лиц (в~формах praes):\\ говорят, полагают.\end{tabular}&
\tabcolsep=0pt\begin{tabular}{c}13/04/2020\\ 18:02:32\end{tabular}\\
\hline
485&sollen-04&Передача мнения других лиц (в~формах praes): говорят, полагают 
{\bfseries\textit{и~т.\,п.}}&
\tabcolsep=0pt\begin{tabular}{c}29/05/2020\\ 13:37:21\end{tabular}\\
\hline
\end{tabular}
\end{center}
\vspace*{-3pt}
\end{table*}



Отражение в~НБД динамики изменения рубрик фасетной классификации 
преследует как минимум две цели. 

%z
Первая из них~--- обеспечить возможность 
определения объема работ по редактированию и~реклассификации аннотаций 
при изменении ДКС. 

%z
Вторая~--- обеспечить лингвисту возможность увидеть, 
соответствует ли конкретная аннотация последней версии ДКС или нет. В~этом 
случае при работе с~НБД (например, при получении статистики) пользователь 
сможет исключить из рассмотрения те примеры, которые еще не были 
приведены в~соответствие с~последней версией ДКС методом 
реклассификации\footnote{Следует отметить, что в~работе~\cite{17-gon} при описании примеров 
библиотек, в~которых проводилась реклассификация, особо отмечены те случаи, когда библиотека не 
прекращала функционировать во время работ по реклассификации и~ее фонды были в~полном объеме 
доступны для посетителей. По этой причине представляется важным обеспечение возможности полноценного 
использования НБД вне зависимости от того, подлежат ли некоторые аннотации редактированию 
и~реклассификации.}. На данный момент функции отражения истории изменений 
реализованы в~НБД только для фасета <<Значения НМГ>> как наиболее 
динамичного.

     Перечисленные выше~7~операций рассмотрим с~точки зрения того, 
требуется ли после их выполнения реклассификация аннотаций, 
сформированных с~использованием рубрик более ранней версии ДКС. На этом 
основании операции могут быть поделены на три группы: 
\begin{enumerate}[(1)]
\item  реклассификация не осуществляется \mbox{(CREATE,}
REORDER, REVISE)

     В результате выполнения операции CREATE(X) в~НБД не появляется 
таких аннотаций, которые требовали бы реклассификации.
     
     При выполнении операций REORDER(X) или REVISE(X) 
реклассификация также не осуществляется, а~в~аннотациях, в~которых~X 
была проставлена ранее, начинает отображаться результат выполнения 
соответственно \mbox{REORDER(X)}~--- меняется код рубрики~--- или  
REVISE(X)~--- меняется дефиниция рубрики (см.\ пример выполнения 
операции REVISE в~табл.~3; несовпадающие фрагменты дефиниций выделены 
полужирным курсивом);
     
\item  реклассификация осуществляется программно 
(MERGE)

     При выполнении MERGE(X,\,Y) def$_{\mathrm{Y}}$ включается 
в~def$_{\mathrm{X}}$, после чего~Y удаляется. Лингвист, который вносит 
данное изменение в~НБД, может самостоятельно определить поглощающую 
рубрику~X и~поглощаемую рубрику~Y. В~результате выполнения операции 
MERGE(X,\,Y) во всех аннотациях, где ранее была проставлена~Y, 
автоматически проставляется идентификатор и~код обновленной рубрики~X;

\item реклассификация осуществляется лингвистами 
экспертно (DELETE, SPLIT, REDISTR)

     При выполнении DELETE(X) аннотации, которые ранее были отнесены 
к~X, автоматически помечаются рубрикой <<TBR-D>> (от англ.\ \textit{To Be 
Reclassified because of Deletion}), указывающей, что требуется их 
реклассификация лингвистами, а~сама~X удаляется.
     
     При выполнении SPLIT(X) аннотации, которые ранее были отнесены 
к~рубрике~X, автоматически помечаются рубрикой <<TBR-S>> (от англ.\ 
\textit{To Be Reclassified because of Split}), указывающей, что требуется их 
реклассификация лингвистами, а~также создается новая рубрика~Y.
     
     Операция REDISTR необходима потому, что нередко приходится иметь 
дело со случаями, когда одновременно меняются def$_{\mathrm{X}}$ 
и~def$_{\mathrm{Y}}$, причем эти изменения оказываются взаимосвязанными. 
Дело в~том, что описания разных значений НМГ в~рамках одной и~той же 
словарной статьи не автономны: они должны позволять разделить все 
употребления НМГ на непересекающиеся семантические классы, определения 
которых указаны в~столбце~3 табл.~1, т.\,е.\ каждой дефиниции соответствует 
один класс. Все аннотации, которые подлежат реклассификации лингвистами 
после выполнения этой операции, автоматически помечаются рубрикой 
<<TBR-R>> (от англ.\ \textit{To Be Reclassified because of Redistribution}).
\end{enumerate}

\vspace*{-8pt}

\section{Заключение}

\vspace*{-2pt}

     На примере задачи фиксации изменений, вносимых в~описания значений 
НМГ, определены семь операций, необходимых для 
использования ДКС в~НБД. Этот набор операций выявлен в~результате 
эксперимента, в~рамках которого с~2018~г.\ по настоящее время 
регистрировались все изменения, вносимые в~дефиниции рубрик фасета 
<<значения НМГ>>. Если вернуться к~упомянутому в~статье вопросу 
К.~Ньоли: <<Как справиться с~проблемой изменения знания при 
организации знания?>>~--- то можно сказать, что предлагаемый набор 
операций представляется одним из возможных методов решения этой 
проб\-ле\-мы. Из-за ускорения роста и~динамики знания, которое необходимо 
учитывать в~задачах классификации, расширяется сфера применения ДКС 
в~информационных системах. Поэтому важной задачей становится разработка 
методов и~операций изменения ДКС как инструментов организации знания, 
учитывающих его быстрый рост и~динамику. В~продолжение исследования 
операций изменения ДКС планируется рассмотреть следующие вопросы: 
(1)~отражение истории изменений дефиниций рубрик ДКС в~НБД; 
(2)~определение объема аннотаций, которые должны быть 
реклассифицированы после внесения изменений в~дефиниции ранее 
использованных рубрик.
     
{\small\frenchspacing
 {%\baselineskip=10.8pt
 %\addcontentsline{toc}{section}{References}
 \begin{thebibliography}{99}
 
 \bibitem{3-gon} %1
\Au{Захаров В.\,П., Богданова~С.\,Ю.} Корпусная лингвистика.~--- Иркутск: ИГЛУ, 2011. 161~с.
  
\bibitem{1-gon} %2
\Au{McEnery T., Hardie~A.}  Corpus linguistics: Method, theory and practice.~--- Cambridge: 
Cambridge University Press, 2012. 310~p.

\bibitem{4-gon} %3
\Au{Копотев М.} Введение в~корпусную лингвистику.~--- Прага: Animedia Co., 2014. 196~с.

\bibitem{2-gon} %4
\Au{K$\ddot{\mbox{u}}$bler S., Zinsmeister~H.} Corpus linguistics and linguistically annotated 
corpora.~--- London/New York: Bloomsbury, 2015. 320~p.

\bibitem{5-gon}
\Au{Кружков М.\,Г.} Информационные ресурсы контрастивных лингвистических 
исследований: электронные корпуса текстов~// Системы и~средства информатики, 2015. 
Т.~25. №\,2. С.~140--159.
\bibitem{6-gon}
\Au{Гончаров А.\,А., Инькова~О.\,Ю., Кружков~М.\,Г.} Методология аннотирования 
в~надкорпусных базах данных~// Системы и~средства информатики, 2019. Т.~29. №\,2. 
С.~148--160.
\bibitem{7-gon}
Handbook of linguistic annotation~/ Eds. N.~Ide, J.~Pustejovsky.~--- Dordrecht: Springer 
Science\;+\;Business Media, 2017. 1568~p.
\bibitem{8-gon}
\Au{Зализняк А.\,А., Зацман И.\,М., Инькова~О.\,Ю.} Надкорпусная база данных 
коннекторов: построение системы терминов~// Информатика и~её применения, 2017. 
Т.~11. Вып.~1. С.~100--108.

\bibitem{10-gon} %9
\Au{Зацман И.\,М., Инькова~О.\,Ю., Кружков~М.\,Г., Попкова~Н.\,А.} Представление 
кроссязыковых знаний о~коннекторах в~надкорпусных базах данных~// Информатика 
и~её применения, 2016. Т.~10. Вып.~1.  
С.~106--118.
\bibitem{9-gon} %10
\Au{Зацман И.\,М., Инькова~О.\,Ю., Нуриев~В.\,А.} По\-стро\-ение классификационных схем: 
методы и~технологии экспертного формирования~// На\-уч\-но-тех\-ни\-че\-ская 
информация. Сер.~2: Информационные процессы и~сис\-те\-мы, 2017. №\,1. С.~8--22.
\bibitem{11-gon}
\Au{Inkova O., Popkova N.} Statistical data as information source for linguistic analysis of 
Russian connectors~// Информатика и~её применения, 2017. Т.~11. Вып.~3. С.~123--131.
\bibitem{12-gon}
\Au{Зацман И., Кружков~М., Лощилова~Е.} Методы и~средства информатики для 
описания структуры неоднословных коннекторов~// Структура коннекторов и~методы ее 
описания~/ Под ред. О.\,Ю.~Иньковой.~--- М.: ТОРУС ПРЕСС, 2019. С.~205--230.
\bibitem{13-gon}
Немецко-русский словарь: актуальная лексика~/ Под ред.\ Д.\,О.~Добровольского.~--- М.: 
Лексрус, 2020 (в~печати).
\bibitem{14-gon}
\Au{Гончаров А.\,А., Зацман~И.\,М., Кружков~М.\,Г.} Темпоральные данные 
в~лексикографических базах знаний~// Информатика и~её применения, 2019. Т.~13. №\,4. 
С.~90--96.
\bibitem{15-gon}
\Au{Добровольский Д.\,О., Зализняк Анна~А.} Немецкие конструкции с~модальными 
глаголами и~их русские соответствия: проект надкорпусной базы данных~// 
Компьютерная лингвистика и~интеллектуальные \mbox{технологии:} По мат-лам Междунар. 
конф. <<Диалог>>.~--- М.: РГГУ, 2018. Вып.~17(24). С.~172--184.
\bibitem{16-gon}
\Au{Bentz D.\,M., Cavender~T.\,P.} Reclassification and recataloging~// Libr. Trends, 1953. 
Vol.~2. Iss.~2.  
P.~249--263.
\bibitem{17-gon}
\Au{Kumbhar R.} Library classification trends in the 21st century.~--- Oxford: Chandos 
Publishing, 2012. 186~p.
\bibitem{18-gon}
\Au{Зацман И.\,М., Косарик~В.\,В., Курчавова~О.\,А.} Задачи представления личностных 
и~коллективных концептов в~цифровой среде~// Информатика и~её применения, 2008. 
Т.~2. Вып.~3. С.~54--69.
\bibitem{19-gon}
\Au{Gnoli C.} Ten long-term research questions in knowledge organization~// Knowl. 
Organ., 2008. Vol.~35. Iss.~2/3. P.~137--149.
\bibitem{20-gon}
\Au{Зацман И.\,М.} Проб\-лем\-но-ори\-ен\-ти\-ро\-ван\-ная верификация полноты 
темпоральных онтологий и~заполнение понятийных лакун~// Информатика и~её 
применения, 2020. Т.~14. Вып.~3. С.~119--128.
\end{thebibliography}

 }
 }

\end{multicols}

\vspace*{-3pt}

\hfill{\small\textit{Поступила в~редакцию 05.10.20}}

\vspace*{8pt}

%\pagebreak

%\newpage

%\vspace*{-28pt}

\hrule

\vspace*{2pt}

\hrule

%\vspace*{-2pt}

\def\tit{EVOLUTION OF CLASSIFICATIONS\\ IN~SUPRACORPORA DATABASES}


\def\titkol{Evolution of classifications in supracorpora databases}


\def\aut{A.\,A.~Goncharov, I.\,M.~Zatsman, and~M.\,G.~Kruzhkov}

\def\autkol{A.\,A.~Goncharov, I.\,M.~Zatsman, and~M.\,G.~Kruzhkov}

\titel{\tit}{\aut}{\autkol}{\titkol}

\vspace*{-11pt}


\noindent
Institute of Informatics Problems, Federal Research Center ``Computer Science and Control'' of the Russian 
Academy of Sciences, 44-2~Vavilov Str., Moscow 119333, Russian Federation

\def\leftfootline{\small{\textbf{\thepage}
\hfill INFORMATIKA I EE PRIMENENIYA~--- INFORMATICS AND
APPLICATIONS\ \ \ 2020\ \ \ volume~14\ \ \ issue\ 4}
}%
 \def\rightfootline{\small{INFORMATIKA I EE PRIMENENIYA~---
INFORMATICS AND APPLICATIONS\ \ \ 2020\ \ \ volume~14\ \ \ issue\ 4
\hfill \textbf{\thepage}}}

\vspace*{6pt} 


\Abste{The paper examines the task of recording changes to descriptions of meanings of German 
modal verbs in the process of annotating parallel German-Russian texts within a supracorpora database. This 
task was used as a case study to analyze the specifics of using dynamic classification systems (DCS) in 
information systems. The distinctive feature of a DCS is that semantic content of its concepts may change in 
the process of annotation which often entails the need to reclassify previously annotated data according to 
the changes made. This paper 
aims to answer the following questions: ($i$)~What factors may have an impact 
on the need to edit and/or reclassify
the annotations created prior to the concept changes? 
and ($ii$)~What kind of operations may be used to represent\linebreak
\vspace*{-12pt}}

\Abstend{the changes to concepts in the DCS? The paper describes seven types of 
possible changes and enumerates the corresponding operations applied to the DCS concepts in the process of 
annotation. The operations are grouped in three categories depending on how they affect the need to 
reclassify the previously created annotations.}

\KWE{dynamic classification; faceted classification; reclassification; supracorpora databases}

\DOI{10.14357/19922264200415} 

\vspace*{-18pt}

\Ack
\noindent
The study has been conducted at the Institute of Informatics Problems, Federal Research Center ``Computer 
Science and Control'' of the Russian Academy of Sciences with financial aid from the 
Russian Foundation for Basic Research (grants Nos.\,18-07-00192 and 20-012-00166).

\vspace*{6pt}

  \begin{multicols}{2}

\renewcommand{\bibname}{\protect\rmfamily References}
%\renewcommand{\bibname}{\large\protect\rm References}

{\small\frenchspacing
 {%\baselineskip=10.8pt
 \addcontentsline{toc}{section}{References}
 \begin{thebibliography}{99}
 
 \bibitem{3-gon-1} %1
\Aue{Zakharov, V.\,P., and S.\,Yu.~Bogdanova.} 2011. \textit{Korpusnaya lingvistika} 
[Corpus linguistics]. 
Irkutsk: IGLU. 161~p.

\bibitem{1-gon-1} %2
\Aue{McEnery, T., and A.~Hardie.} 2012. \textit{Corpus linguistics: Method, theory and practice}. 
Cambridge: Cambridge University Press. 310~p.


\bibitem{4-gon-1} %3
\Aue{Kopotev, M.} 2016. \textit{Vvedenie v~korpusnuyu lingvistiku} [Introduction to corpus linguistics].
Praha: 
Animedia Company. 196~p.

\bibitem{2-gon-1} %4
\Aue{K$\ddot{\mbox{u}}$bler, S., and H.~Zinsmeister.} 2015. \textit{Corpus linguistics and linguistically 
annotated corpora}. London/New York: Bloomsbury. 320~p.

\bibitem{5-gon-1}
\Aue{Kruzhkov, M.\,G.} 2015. Informatsionnye resursy kontrastivnykh lingvisticheskikh issledovaniy: 
elektronnye korpusa tekstov [Information resources for contrastive\linebreak studies: Electronic text corpora]. 
\textit{Sistemy i~Sredstva Informatiki~--- Systems and Means of Informatics} 25(2):140--159.
\bibitem{6-gon-1}
\Aue{Goncharov, A.\,A., O.\,Yu.~Inkova, and M.\,G.~Kruzhkov.} 2019. Metodologiya annotirovaniya v 
nadkorpusnykh bazakh dannykh [Annotation methodology of supracorpora databases]. \textit{Sistemy 
i~Sredstva Informatiki~--- Systems and Means of Informatics} 29(2):148--160.
\bibitem{7-gon-1}
Ide, N., and J. Pustejovsky, eds. 2017. \textit{Handbook of linguistic annotation}. Dordrecht: Springer 
Science\;+\;Business Media. 1568~p.
\bibitem{8-gon-1}
\Aue{Zaliznyak, A.\,A., I.\,M.~Zatsman, and O.\,Yu.~Inkova.} 2017. Nadkorpusnaya baza dannykh 
konnektorov: postroenie sistemy terminov [Supracorpora database on connectives: Term system 
development]. \textit{Informatika i~ee  
Primeneniya~--- Inform. Appl.} 11(1):100--108.

\bibitem{10-gon-1} %9
\Aue{Zatsman, I.\,M., O.\,Yu.~Inkova, M.\,G.~Kruzhkov, and N.\,A.~Popkova.} 2016. Predstavlenie 
krossyazykovykh znaniy o~konnektorakh v~nadkorpusnykh bazakh dannykh [Representation of cross-lingual 
knowledge about connectors in suprocorpora databases]. \textit{Informatika i~ee Primeneniya~--- Inform. 
Appl.} 10(1):106--118.

\bibitem{9-gon-1} %10
\Aue{Zatsman, I.\,M., O.\,Yu.~Inkova, and V.\,A.~Nuriev}. 2017. The construction of classification 
schemes: Methods and technologies of expert formation. \textit{Automatic Documentation Mathematical Linguistics} 51(1): 27--41.

\bibitem{11-gon-1}
\Aue{Inkova, O., and N.~Popkova.} 2017. Statistical data as information source for linguistic analysis of 
Russian connectors. \textit{Informatika i~ee Primeneniya~--- Inform. Appl.} 11(3):123--131.
\bibitem{12-gon-1}
\Aue{Zatsman, I., M. Kruzhkov, and E.~Loshchilova.} 2019. Metody i~sredstva informatiki dlya opisaniya 
struktury neodnoslovnykh konnektorov [Methods and means of informatics for multiword connectives 
structure description]. \textit{Struktura konnektorov i~metody ee opisaniya} [Connectives structure and 
methods of its description]. Ed. O.\,Yu.~Inkova. Moscow: TORUS PRESS. 205--230.
\bibitem{13-gon-1}
Dobrovol'skiy, D.\,O., ed. 2020 (in press). \textit{Nemetsko-russkiy slovar': aktual'naya leksika} 
[German--Russian dictionary: Actual vocabulary]. Moscow: Leksrus.

\bibitem{14-gon-1}
\Aue{Goncharov, A.\,A., I.\,M.~Zatsman, and M.\,G.~Kruzhkov.} 2019. Temporal'nye dannye 
v~leksikograficheskikh bazakh znaniy [Temporal data in lexicographic databases]. \textit{Informatika i~ee 
Primeneniya~--- Inform. Appl.} 13(4):90--96.
\bibitem{15-gon-1}
\Aue{Dobrovol'skiy, D.\,O., and Anna A.~Zalizniak.} 2018. Nemetskie konstruktsii s~modal'nymi 
glagolami i~ikh russkie sootvetstviya: proekt nadkorpusnoy bazy dannykh [German constructions with 
modal verbs and their Russian correlates: A supracorpora database project]. \textit{Komp'yuternaya 
lingvistika i~intellektual'nye tekhnologii: po mat-lam Mezhdunar. konf. ``Dialog''} [Computer Linguistic 
and Intellectual Technologies: Conference (International) ``Dialog'' Proceedings]. Moscow. 17(24):172--184.
\bibitem{16-gon-1}
\Aue{Bentz, D.\,M., and T.\,P.~Cavender.} 1953. Reclassification and recataloging. \textit{Libr. Trends} 
2(2):249--263.
\bibitem{17-gon-1}
\Aue{Kumbhar, R}. 2012. \textit{Library classification trends in the 21st century}. Oxford: Chandos 
Publishing. 186~p.
\bibitem{18-gon-1}
\Aue{Zatsman, I.\,M., V.\,V.~Kosarik, and O.\,A.~Kurchavova.} 2008. Zadachi predstavleniya lichnostnykh 
i~kollektivnykh kontseptov v~tsifrovoy srede [Representation of individual and collective concepts in digital 
medium]. \textit{Informatika i~ee Primeneniya~--- Inform. Appl.} 2(3):54--69.
\bibitem{19-gon-1}
\Aue{Gnoli, C.} 2008. Ten long-term research questions in knowledge organization. \textit{Knowl. 
Organ.} 35(2/3):137--149.
\bibitem{20-gon-1}
\Aue{Zatsman, I.\,M.} 2020. Problemno-orientirovannaya ve\-ri\-fi\-ka\-tsiya polnoty temporal'nykh ontologiy 
i~zapolnenie ponyatiynykh lakun [Problem-oriented verifying the completeness of temporal ontologies and 
filling conceptual lacunas]. \textit{Informatika i~ee Primeneniya~--- Inform. Appl.} 14(3):119--128.
\end{thebibliography}

 }
 }

\end{multicols}

\vspace*{-9pt}

\hfill{\small\textit{Received October 5, 2020}}

%\pagebreak

%\vspace*{-24pt}

\Contr

\noindent
\textbf{Zatsman Igor M.} (b.\ 1952)~--- Doctor of Science in technology, Head of Department, Institute of 
Informatics Problems, Federal Research Center ``Computer Science and Control'' of the Russian Academy 
of Sciences, 44-2~Vavilov Str., Moscow 119333, Russian Federation; \mbox{izatsman@yandex.ru}

\vspace*{6pt}

\noindent
\textbf{Kruzhkov Mikhail G.} (b.\ 1975)~--- senior scientist, Institute of Informatics Problems, Federal 
Research Center ``Computer Science and Control'' of the Russian Academy of Sciences, 44-2~Vavilov Str., 
Moscow 119333, Russian Federation; \mbox{magnit75@yandex.ru}

\vspace*{6pt}

\noindent
\textbf{Goncharov Alexander A.} (b.\ 1994)~--- junior scientist, Institute of Informatics Problems, Federal 
Research Center ``Computer Science and Control'' of the Russian Academy of Sciences, 44-2~Vavilov Str., 
Moscow 119333, Russian Federation
\label{end\stat}

\renewcommand{\bibname}{\protect\rm Литература} 
      