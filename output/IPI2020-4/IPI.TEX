

\documentclass[10pt]{book}
\usepackage[utf8]{inputenc}

\usepackage{latexsym,amssymb,amsfonts,amsmath,amsxtra,dsfont,
indentfirst,shapepar,%fleqn,%
picinpar,shadow,floatflt,enumerate,multicol,colortbl,moreverb,cite,ipi}

\usepackage{rotating}
\usepackage{mathrsfs}
\usepackage[noend]{algorithmic}
\usepackage{ulem}
\usepackage{graphicx}
%\usepackage{algorithm2e}
\usepackage[linesnumbered,boxed,ruled]{algorithm2e}
%\usepackage{xypic}
\usepackage{oldgerm}
\usepackage{epic}
\usepackage{eepic}

\SetAlgorithmName{Algorithm}{алгоритм}{Список алгоритмов}

%из Дюковой

\newcommand{\algKeyword}[1]{{\bf #1}}
\newcommand{\Proc}[1]{\text{\tt #1}}
\def\CALL{\algKeyword{call}~}

\newenvironment{AlgProcedure}[1]
{
\small
\medskip
%    \hrule
\medskip
\algKeyword{PROCEDURE} #1
\begin{algorithmic}[1]}
{\end{algorithmic}
%    \hrule
\bigskip
}

\def\CALL{\algKeyword{call}~}

%конец для Дюковой

%\RequirePackage[ruled]{algorithm}


\input{epsf}

%\nofiles

%\includeonly{avtor}             %+pdf+
%\includeonly{obchak,avtor}
%\includeonly{pred}                 %+
%\includeonly{podgot-rus-site,podgot-eng-site}  
%\includeonly{podgot-rus,podgot-eng}  
%\includeonly{ocherk} 
%\includeonly{ipi-ind} 
%\includeonly{index14}
%\includeonly{toc-rus, toc-en}
%\includeonly{toc-rus}
%\includeonly{toc-en} 

%ИИЕП 2020-4

%\includeonly{grusho-zab}                       %01+pdf+авт+
%\includeonly{moskaleva}                        %02+pdf+авт+повторно отправили+авт+
%\includeonly{kochetkova}                       %03+pdf+авт+
%\includeonly{razumcik}                         %04+pdf+авт+повт отпр+авт+
%\includeonly{korolev}                          %05+pdf+авт
%\includeonly{naumov}            %06+pdf+авт+повторно отпр
%\includeonly{popkovi}                          %07+pdf+авт
%\includeonly{potanin-strijov}                  %08+pdf+авт
%\includeonly{sokolov-dyachenko} %09+pdf+авт+повторно отпр
%\includeonly{budzko-sochenkov}                 %10+pdf+авт+
%\includeonly{rum-kir}                          %11+pdf+авт+
%\includeonly{danilishin}                       %12+pdf+авт+
%\includeonly{abgaryan}          %13+pdf
%\includeonly{betelin}                          %14+pdf+авт+повт отпр+авт+повт отпр+
%\includeonly{goncharov}                        %15+pdf+авт+



%\includeonly{obchak}
%\includeonly{rekl}
%\includeonly{rekl-1}
%\includeonly{reshal}  %
%\includeonly{cover3}

\usepackage{acad}
%\usepackage{courier}
\usepackage{decor}
\usepackage{newton}
\usepackage{pragmatica}
\usepackage{zapfchan}
\usepackage{petrotex}
\usepackage{bm}                     % полужирные греческие буквы
\usepackage{upgreek}                % прямые греческие буквы \upalpha
\usepackage{eufrak}
\usepackage{verbatim}

\renewcommand{\bottomfraction}{0.99}
\renewcommand{\topfraction}{0.99}
\renewcommand{\textfraction}{0.01}

\setcounter{secnumdepth}{1} %здесь - 3 + chapter = 4

\arraycolsep=1.5pt

%\usepackage[pdftex]{graphicx}

%\usepackage{oz}

%NEW COMMANDS


\renewcommand*{\hm}[1]{#1\nobreak\discretionary{}%
            {\hbox{$\mathsurround=0pt #1$}}{}} %% Дублирует знаки операций
                               %при переносе в формуле (перед знаком, который
                               %надо продублировать ставится команда \hm)

%\newcommand{\endproof}{\hfill$\Box$}
\renewcommand{\r}{\mathbb{R}}
%\newcommand{\I}{{\rm I\hspace{-0.7mm}I}}
%\newcommand{\Ikl}{{\tt{1}}\hspace*{-1.44mm}\mathtt{1}}
\newcommand{\Ik}{\mbox{{\small \tt {1}}\hspace{-1.3mm}{\tt 1}}}
\newcommand{\argmin}{\mathop{\mathrm{arg}\,\mathrm{min}}}
\newcommand{\argmax}{\mathop{\mathrm{arg}\,\mathrm{max}}}
%\newcommand{\capr}{\mathop{\cap\,}}
%\newcommand{\cupr}{\mathop{\cup\,}}
%\def\argmin{\mathop{arg\,min}}

\def\vrp{\varphi}
\def\prt{\partial}
\def\mm{{\sf M}}
\def\modnop#1{\mathop{#1}\limits_{n}}
\def\eam{\mathbin{{\mathop{=}\limits^{\mathrm{def}}}}}
\def\dey#1#2{#1 (#2)}
\def\deyc#1#2{#1 \cdot  #2}
\def\ra#1{\;\mathop{\to}\limits^{#1}\;}
\def\raz#1{\;\mathop{\longrightarrow}\limits^{\!\!\!#1}\;}
\def\ral#1{\;\mathop{\longrightarrow}\limits^{#1}\;}

\newcommand{\Nor}{\mathcal{N}}
\newcommand{\T}{\mathbb{T}}
\newcommand{\Z}{\mathbb{Z}}



\newcommand{\il}[2]{\int\limits_{#1}^{#2}}%интеграл с пределами #1 и #2

\def\sm2{\mathop {\sum\limits^{n^\Theta}\sum\limits^{n^\Theta}}}
\def\sss{\sum\limits}
\def\tr{,\,\ldots\,,\,}
\def\rk{\right]}
\def\lk{\left[}
\def\rf{\right\}}
\def\lf{\left\{}
\def\lv{\,\left\vert}
\def\rv{\right\vert\,}
\def\iii{\int\limits}
\def\iin{\int\limits_{-\infty}^\infty}
\def\rrv{\right\vert}


\def\ee{{\cal E}}
\def\ww{{\cal W}}
\def\yy{{\cal Y}}
\def\vv{{\cal V}}

\newcommand{\R}{\mathbb R}
\newcommand{\E}{\mathbb E}
\newcommand{\N}{\mathbb N}

\renewcommand{\P}{\mathbb{P}}

\newcommand{\h}{{\bf H}}
\newcommand{\p}{{\sf P}}  % вероятность

\newcommand{\e}{{\sf E}}  % мат. ожидание
\newcommand{\D}{{\sf D}}  % дисперсия
\newcommand{\eps}{\varepsilon}
\newcommand{\vp}{{\mathbf p}}
\newcommand{\vz}{{\mathbf z}}
\newcommand{\vx}{{\mathbf x}}
\newcommand{\vf}{{\mathbf f}}
\newcommand{\F}{{\mathcal F}}
\def\ap{{\mathrm{ЭР}}}
\newcommand{\ud}{\Delta_n} %uniform ditance
\newcommand{\nud}{\Delta_n(x)}
%\renewcommand{\Re}{\mathrm{Re}\,}

\newcommand{\abs}[1]{\left\vert#1\right\vert}

\newcommand{\norm}[1]{\left\Vert#1\right\Vert}
\def\da{(\Delta_t,A)}

\newcommand{\corr}{\mathrm{corr}}

\newcommand{\cov}{\mathrm{cov}}
\newcommand{\Expect}{\mathbb{E}}

\def\w{\omega}
\def\W{\Omega}

\def\inh{\int\limits_{nh}^{(n+1)h}}

\def\sumin{\sum_{i=1}^N}


\def\bxt{(Y,t)}
\def\xt{(y,t)}

\def\ovth{{\fr{\tau-nh}{h}}}
\def\ov{\overline}
\def\tm{\tilde m}
\def\tl{\tilde\lambda}
\def\tB{\widetilde B}
\def\tb{\tilde b}
\def\ld{\ldots}
\def\cd{\cdots}


\DeclareMathOperator{\sign}{sign}

%\newcommand{\gr}{{\geqslant}}


\newcommand{\g}{\mbox{\textit{g}}}

\renewcommand{\la}{\lambda}
\newcommand{\si}{\sigma}
\newcommand{\alp}{\alpha}

\newcommand{\pto}{\stackrel{P}{\longrightarrow}} % сходимость по веpоятности

\newcommand{\eqd}{\stackrel{\mathrm{d}}{=}} % равенство по pаспpеделению
\newcommand{\eqdelta}{\stackrel{\triangle}{=}} % равенство по pаспpеделению

\def\be#1{\begin{equation}\label{#1}}
\def\ee{\end{equation}}
\def\re#1{(\ref{#1})}

\def\bn{\begin{enumerate}}
\def\en{\end{enumerate}}
\def\bi{\begin{itemize}}
\def\ei{\end{itemize}}
%\def\i{\item}

%\newcommand{\kp}{\kappa}
%\def\Q{{\cal Q}} \def\H{{\cal H}}
%\newcommand{\bet}{\beta_{2+\delta}}


%\newtheorem{definition}{Определение}
%\renewcommand{\thedefinition}{\arabic{definition}.}
%END NEW COMMANDS

%\renewcommand{\baselinestretch}{1.2}

%\pagestyle{myheadings}

\setlength{\textwidth}{167mm}      % 122mm
\setlength{\textheight}{658pt}
%\setlength{\textheight}{635.6pt}
\setlength{\columnsep}{4.5mm}

\setcounter{secnumdepth}{4}

%\addtolength{\headheight}{2pt}
%\addtolength{\headsep}{-2mm}

\addtolength{\topmargin}{-7mm}  % for printing


%\hoffset=-30mm  % From Yap
\hoffset=-23mm  % From Acrobat

%\voffset=0mm % From Yap
\voffset=-5mm   % From Acrobat

%\addtolength{\evensidemargin}{-2.5mm} % for printing
%\addtolength{\oddsidemargin}{2.5mm}  % for printing

\addtolength{\evensidemargin}{-12mm} % for printing
\addtolength{\oddsidemargin}{8mm}  % for printing

%\renewcommand{\thefootnote}{\fnsymbol{footnote}}
%\renewcommand{\thefootnote}{\arabic{footnote}}
\renewcommand{\figurename}{\protect\bf Рис.}
\renewcommand{\tablename}{\protect\bf Таблица}

\newcommand{\Caption}[1]{\caption{\protect\small %\baselineskip=2.5ex
#1}}

\renewcommand{\thefigure}{\arabic{figure}}
\renewcommand{\thetable}{\arabic{table}}
\renewcommand{\theequation}{\arabic{equation}}
\renewcommand{\thesection}{\arabic{section}}

\renewcommand{\contentsname}{СОДЕРЖАНИЕ}
\newcommand{\fr}[2]{\displaystyle\frac{\displaystyle #1\mathstrut}{\displaystyle #2\mathstrut}}

%\renewcommand{\thefootnote}{\fnsymbol{footnote}}
%\newcommand{\g}{\mbox{\textit{g}}}

%\newcommand{\Caption}[1]{\caption{\protect\small\baselineskip=2ex #1}}
\newcounter{razdel}
\setcounter{razdel}{0}

\def\god{2020}
\def\tom{14}
\def\vyp{4}


\newcommand{\titel}[4]{%
\

\vspace*{5pt}

\ifodd\therazdel {\raggedright\noindent\Large\textrm\textbf
 \lineskip .75em
  \baselineskip=3.2ex #1 \par}
\vskip 1em {\noindent\large\textrm\textbf #2 \par}
\addcontentsline{toc}{subsection}{{\textrm\textbf #1}\protect\newline #2}
\def\rightheadline{\underline{\noindent\hbox to \textwidth{\hfill\small\textrm{#4}
%\hfill \large\bf\thepage
}}}
\def\leftheadline{\underline{\noindent\parbox{\textwidth}{
%\raggedleft\large\bf\thepage \hfill
\small\textit{#3}\hfill}}}
\def\leftfootline{\small{\textbf{\thepage}
\hfill ИНФОРМАТИКА И ЕЁ ПРИМЕНЕНИЯ\ \ \ том~\tom\ \ \ выпуск~\vyp\ \ \ \god}
}%
 \def\rightfootline{\small{ИНФОРМАТИКА И ЕЁ ПРИМЕНЕНИЯ\ \ \ том~\tom\ \ \ выпуск~\vyp\ \ \ \god
\hfill \textbf{\thepage}}}
\vskip 2em \setcounter{figure}{0}
\setcounter{table}{0}
\setcounter{equation}{0}
\setcounter{section}{0}
\setcounter{subsection}{0}
\setcounter{subsubsection}{0}
\setcounter{footnote}{0}
\setcounter{razdel}{0}
%\end{flushleft}
\else {
 \raggedright\noindent\Large\textrm\textbf
 \lineskip .75em
\baselineskip=3.2ex #1 \par} \vskip 1em
%\begin{flushleft}
{\noindent\large\textrm\textbf #2 \par}
\addcontentsline{toc}{subsection}{{\textrm\textbf #1}\protect\newline #2}
\def\rightheadline{\underline{\noindent\hbox to \textwidth{\hfill\small\textrm{#4}
%\hfill \large\bf\thepage
}}}
\def\leftheadline{\underline{\noindent\parbox{\textwidth}{%\raggedleft\large\bf\thepage \hfill
\small\textit{#3}\hfill}}}
\def\leftfootline{\small{\textbf{\thepage}
\hfill ИНФОРМАТИКА И ЕЁ ПРИМЕНЕНИЯ\ \ \ том~\tom\ \ \ выпуск~\vyp\ \ \ \god}
}%
 \def\rightfootline{\small{ИНФОРМАТИКА И ЕЁ ПРИМЕНЕНИЯ\ \ \ том~14\ \ \ выпуск~\vyp\ \ \ 2020
\hfill \textbf{\thepage}}} \vskip 2em \setcounter{figure}{0}
\setcounter{table}{0} \setcounter{equation}{0} \setcounter{section}{0}
\setcounter{subsection}{0} \setcounter{subsubsection}{0}
\setcounter{footnote}{0}
%\end{flushleft}
\fi}

\newcommand{\titelr}[2]{%
\

\vspace*{5pt}

\ifodd\therazdel {\raggedright\noindent%\Large\textrm\textbf
 \lineskip .75em
  \baselineskip=3.2ex #1 \par}
\vskip 1em {\noindent\normalsize\textrm\textbf #2 \par}
\else {
 \raggedright\noindent\Large\textrm\textbf
 \lineskip .75em
\baselineskip=3.2ex #1 \par} \vskip 1em
%\begin{flushleft}
{\noindent\large\textrm\textbf #2 \par
%\noindent\normalsize\textrm\textbf #2 \par
} \fi}

\newcommand{\titele}[5]{%
\

%\vspace*{5pt}

\ifodd\therazdel {\raggedright\noindent\large
\textrm\textbf
 \lineskip .75em
%  \baselineskip=3.2ex
#1 \par}
\vskip .5em {\noindent\large\textrm\textbf #2 \par}
\vskip .5em
 {\noindent\textrm #3 \par}
\addcontentsline{toc}{subsection}{{\textrm\textbf #1}\protect\newline #2}
\def\rightheadline{\underline{\noindent\hbox to \textwidth{\hfill\small\textrm{#4}
%\hfill \large\bf\thepage
}}}
\def\leftheadline{\underline{\noindent\parbox{\textwidth}{
%\raggedleft\large\bf\thepage \hfill
\small\textrm{#5}\hfill}}}
\def\leftfootline{\small{\textbf{\thepage}
\hfill ИНФОРМАТИКА И ЕЁ ПРИМЕНЕНИЯ\ \ \ том~14\ \ \ выпуск~4\ \ \ 2020}
}%
 \def\rightfootline{\small{ИНФОРМАТИКА И ЕЁ ПРИМЕНЕНИЯ\ \ \ том~14\ \ \ выпуск~4\ \ \ 2020
\hfill \textbf{\thepage}}} \vskip 1em \setcounter{figure}{0}
\setcounter{table}{0} \setcounter{equation}{0} \setcounter{section}{0}
\setcounter{subsection}{0} \setcounter{subsubsection}{0}
\setcounter{footnote}{0} \setcounter{razdel}{0}
%\end{flushleft}
\else {
 \raggedright\noindent\large
 \textrm\textbf
 \lineskip .75em
%\baselineskip=3.2ex
#1 \par} \vskip .5em
%\begin{flushleft}
{\noindent\large\textrm\textbf #2 \par} \vskip .5em
 {\noindent\textrm #3 \par}
\addcontentsline{toc}{subsection}{{\textrm\textbf #1}\protect\newline #2}
\def\rightheadline{\underline{\noindent\hbox to \textwidth{\hfill\small\textrm{#4}
%\hfill \large\bf\thepage
}}}
\def\leftheadline{\underline{\noindent\parbox{\textwidth}{%\raggedleft\large\bf\thepage \hfill
\small\textrm{#5}\hfill}}}
\def\leftfootline{\small{\textbf{\thepage}
\hfill ИНФОРМАТИКА И ЕЁ ПРИМЕНЕНИЯ\ \ \ том~14\ \ \ выпуск~4\ \ \ 2020}
}%
 \def\rightfootline{\small{ИНФОРМАТИКА И ЕЁ ПРИМЕНЕНИЯ\ \ \ том~14\ \ \ выпуск~4\ \ \ 2020
\hfill \textbf{\thepage}}} \vskip 1em \setcounter{figure}{0}
\setcounter{table}{0} \setcounter{equation}{0} \setcounter{section}{0}
\setcounter{subsection}{0} \setcounter{subsubsection}{0}
\setcounter{footnote}{0}
%\end{flushleft}
\fi}

\def\Abst#1{
\begin{center}\small\nwt
\parbox{150mm}{%\baselineskip=2.5ex
\textbf{Аннотация:}\ \
%\hspace*{\parindent}
#1}
\end{center}}
\def\Abste#1{
\begin{center}\small\nwt
\parbox{150mm}{%\baselineskip=2.5ex
\textbf{Abstract:}\ \
%\hspace*{\parindent}
#1}
\end{center}}

\def\DOI#1{
\begin{center}\small\nwt
\parbox{150mm}{%\baselineskip=2.5ex
\textbf{DOI:}\ \
%\hspace*{\parindent}
#1}
\end{center}}

\def\Abstend#1{
\begin{center}\small\nwt
\parbox{150mm}{%\baselineskip=2.5ex
%\hspace*{\parindent}
#1}
\end{center}}


\def\KW#1{
\begin{center}\small\nwt
\parbox{150mm}{%\baselineskip=2.5ex
\textbf{Ключевые слова:}\ \ #1}
\end{center}}

\def\KWE#1{
\begin{center}\small\nwt
\parbox{150mm}{%\baselineskip=2.5ex
\textbf{Keywords:}\ \ #1}
\end{center}}


\def\KWN#1{
%\begin{center}
%\small
%\parbox{150mm}\end{center}
}

\newcommand{\Avtors}[1]{%\smallskip
%\vspace*{.5pt}
\hangindent=23pt\noindent
%\nwt
{\bfseries#1}\
}


\renewcommand{\thesubsection}{\thesection.\arabic{subsection}\hspace*{-5pt}}
\renewcommand{\thesubsubsection}{\thesubsection\hspace*{5pt}.\arabic{subsubsection}\hspace*{-3pt}}

\newcommand{\Ack}{\section*{\protect\rmfamily Acknowledgments}\noindent}
\newcommand{\Contr}{\section*{\protect\rmfamily Contributors}\noindent}
\newcommand{\Contrl}{\section*{\protect\rmfamily Contributor}\noindent}

\makeindex


\begin{document}
\Rus

\nwt
%\ptb


%\renewcommand{\contentsname}{\protect\Large\bf Содержание}

\setcounter{tocdepth}{2}

%\tableofcontents

\renewcommand{\bibname}{\protect\rmfamily Литература}
  \def\Au#1{{\it #1}}
    \def\Aue#1{{#1}}

%\newcommand{\No}{№}
  \newcommand{\tg}{\,\mathrm{tg}\,}
    \newcommand{\ctg}{\,\mathrm{ctg}\,}
  \newcommand{\arctg}{\,\mathrm{arctg}\,}

\def\forallb{\mathop{\forall}}
\def\cupb{\mathop{\cup}}
\def\existsb{\mathop{\exists}}


\newpage
\addtocounter{razdel}{1}
%\def\razd{РЕГУЛИРУЕМЫЙ ЭЛЕКТРОПРИВОД ДЛЯ ЭЛЕКТРОЭНЕРГЕТИКИ}


\setcounter{page}{3}

%   { %\Large  
   { %\baselineskip=16.6pt
   
   \vspace*{-48pt}
   \begin{center}\LARGE
   \textit{Предисловие}
   \end{center}
   
   %\vspace*{2.5mm}
   
   \vspace*{25mm}
   
   \thispagestyle{empty}
   
   { %\small 

    
Вниманию читателей журнала <<Информатика и её применения>> предлагается 
очередной тематический выпуск <<Вероятностно-статистические методы и 
задачи информатики и информационных технологий>>. Предыдущие тематические 
выпуски журнала по данному направлению вышли в 2008~г.\ (т.~2, вып.~2), 
в 2009~г.\ (т.~3, вып.~3) и в 2010~г.\ (т.~4, вып.~2). 

Статьи, собранные в данном журнале, посвящены разработке новых вероятностно-статистических 
методов, ориентированных на применение к решению конкретных задач информатики и информационных 
технологий, а также~--- в ряде случаев~--- и других прикладных задач. Проблематика, охватываемая 
публикуемыми работами, развивается в рамках научного сотрудничества между Институтом проблем 
информатики Российской академии наук (ИПИ РАН) и Факультетом вычислительной математики и 
кибернетики Московского государственного университета им.\ М.\,В.~Ломоносова в ходе работ 
над совместными научными проектами (в том числе в рамках функционирования 
Научно-образовательного центра <<Вероятностно-статистические методы анализа рисков>>). 
Многие из авторов статей, включенных в данный номер журнала, являются активными участниками 
традиционного международного семинара по проблемам устойчивости стохастических моделей, 
руководимого В.\,М.~Золотаревым и В.\,Ю.~Королевым; регулярные сессии этого семинара 
проводятся под эгидой МГУ и ИПИ РАН (в 2011~г.\ указанный семинар проводится в октябре 
в Калининградской области РФ). 

Наряду с представителями ИПИ РАН и МГУ в число авторов данного выпуска журнала входят 
ученые из Научно-исследовательского института системных исследований РАН, Института 
проблем технологии микроэлектроники и особочистых материалов РАН, Института 
прикладных математических исследований Карельского НЦ РАН, Московского 
авиационного института, Вологодского государственного педагогического университета, 
НИИММ им.\ Н.\,Г.~Чеботарева, Казанского государственного университета, Дебреценского 
университета (Венгрия).

Несколько статей выпуска посвящено разработке и применению стохастических методов и 
информационных технологий для решения различных прикладных задач. В~работе В.\,Г.~Ушакова 
и О.\,В.~Шестакова рассмотрена задача определения вероятностных характеристик случайных 
функций по распределениям интегральных преобразований, возникающих в задачах эмиссионной 
томографии. В~статье Д.\,О.~Яковенко и М.\,А.~Целищева рассмотрены некоторые вопросы 
математической теории риска и предложен новый подход к диверсификации инвестиционных 
портфелей. Работа И.\,А.~Кудрявцевой и А.\,В.~Пантелеева посвящена построению и 
исследованию математической модели, описывающей динамику сильноионизованной плазмы. 
В~статье П.\,П.~Кольцова изучается качество работы ряда алгоритмов сегментации изображений. 
Статья А.\,Н.~Чупрунова и И.~Фазекаша посвящена вероятностному анализу числа без\-оши\-бочных 
блоков при помехоустойчивом кодировании; получены усиленные законы больших чисел для указанных 
величин.

В данном выпуске традиционно присутствует тематика, весьма активно разрабатываемая в течение 
многих лет специалистами ИПИ РАН и МГУ,~--- методы моделирования и управления для 
информационно-телекоммуникационных и вычислительных систем, в частности методы 
теории массового обслуживания. В~статье А.\,И.~Зейфмана с соавторами рассматриваются 
модели обслуживания, описываемые марковскими цепями с непрерывным временем в случае 
наличия катастроф. В~работе М.\,М.~Лери и И.\,А.~Чеплюковой рассматриваются случайные 
графы Интернет-типа, т.\,е.\ графы, степени вершин которых имеют степенные распределения; 
такие задачи находят применение при исследовании глобальных сетей передачи данных. 
Работа Р.\,В.~Разумчика посвящена исследованию систем массового обслуживания специального 
вида~--- с отрицательными заявками и хранением вытесненных заявок.

Ряд статей посвящен развитию перспективных теоретических 
вероятностно-статистических методов, которые находят широкое применение в различных 
задачах информатики и информационных технологий. В~работе В.\,Е.~Бенинга, А.\,К.~Горшенина 
и В.\,Ю.~Королева рассмотрена задача статистической проверки гипотез о числе компонент 
смеси вероятностных распределений, приводится конструкция асимптотически наиболее мощного 
критерия. Результаты этой работы найдут применение в ряде прикладных задач, использующих 
математическую модель смеси вероятностных распределений (в информатике, моделировании 
финансовых рынков, физике турбулентной плазмы и~т.\,д.). В~статье В.\,Ю.~Королева, 
И.\,Г.~Шевцовой и С.\,Я.~Шоргина строится новая, улучшенная оценка точности нормальной 
аппроксимации для пуассоновских случайных сумм; как известно, указанные случайные суммы 
широко используются в качестве моделей многих реальных объектов, в том числе в информатике, 
физике и других прикладных областях. Работа В.\,Г.~Ушакова и Н.\,Г.~Ушакова посвящена 
исследованию ядерной оценки плотности распределения; эти результаты могут применяться, 
в част\-ности, при анализе трафика в телекоммуникационных системах. Серьезные приложения 
в статистике могут получить результаты работы О.\,В.~Шестакова, в которой доказаны оценки 
скорости сходимости распределения выборочного абсолютного медианного отклонения к нормальному 
закону. 

\smallskip

Редакционная коллегия журнала выражает надежду, что данный тематический  выпуск 
будет интересен специалистам в области теории вероятностей и математической статистики 
и их применения к решению задач информатики и информационных технологий.
     
     %\vfill 
     \vspace*{20mm}
     \noindent
     Заместитель главного редактора журнала <<Информатика и её 
применения>>,\\
     директор ИПИ РАН, академик  \hfill
     \textit{И.\,А.~Соколов}\\
     
     \noindent
     Редактор-составитель тематического выпуска,\\
     профессор кафедры математической статистики факультета\\
      вычислительной математики и кибернетики МГУ им.\ М.\,В.~Ломоносова,\\
     ведущий научный сотрудник ИПИ РАН,\\ 
доктор физико-математических наук \hfill
      \textit{В.\,Ю.~Королев}
     
     } }
     }

\def\stat{gr+zab}

\def\tit{ФОРМИРОВАНИЕ КОНЦЕПТОВ НА~ОСНОВЕ МАЛЫХ ВЫБОРОК$^*$}

\def\titkol{Формирование концептов на основе малых выборок}

\def\aut{А.\,А.~Грушо$^1$, М.\,И.~Забежайло$^2$, Н.\,А.~Грушо$^3$, 
Е.\,Е.~Тимонина$^4$}

\def\autkol{А.\,А.~Грушо, М.\,И. Забежайло, Н.\,А.~Грушо, 
Е.\,Е.~Тимонина}

\titel{\tit}{\aut}{\autkol}{\titkol}

\index{Грушо А.\,А.}
\index{Забежайло М.\,И.}
\index{Грушо Н.\,А.} 
\index{Тимонина Е.\,Е.}
\index{Grusho A.\,A.}
\index{Zabezhailo M.\,I.}
\index{Grusho N.\,A.}
\index{Timonina E.\,E.}



{\renewcommand{\thefootnote}{\fnsymbol{footnote}} \footnotetext[1]
{Работа частично поддержана РФФИ (проект 18-29-03081).}}


\renewcommand{\thefootnote}{\arabic{footnote}}
\footnotetext[1]{Институт проблем информатики Федерального исследовательского центра <<Информатика и~управление>> 
Российской академии наук, \mbox{grusho@yandex.ru}}
\footnotetext[2]{Институт проблем информатики Федерального исследовательского центра <<Информатика и~управление>> 
Российской академии наук, m.zabezhailo@yandex.ru}
\footnotetext[3]{Институт проблем информатики Федерального исследовательского центра <<Информатика и~управление>> 
Российской академии наук, info@itake.ru}
\footnotetext[4]{Институт проблем информатики Федерального исследовательского центра <<Информатика и~управление>> 
Российской академии наук, eltimon@yandex.ru}

\vspace*{-12pt}
  
  
  \Abst{Системы мониторинга информационной безопасности ин\-фор\-ма\-ци\-он\-но-вы\-чис\-ли\-тель\-ных систем получают информацию в~виде цепочек коротких сообщений, которые 
можно считать цепочками малых выборок. Часто в~силу инерционности информационных 
систем эти цепочки отражают близкие состояния вычислительной системы или сети. 
Предполагается, что работу системы можно представить в~виде конечного набора режимов, 
которые называются концептами. Нарушения безопасности выявляются с~помощью аномалий, 
которые ассоциируются с~появлением новых концептов. 
  Известные технологии выявления аномалий основаны на построении модели нормального 
поведения системы. Концепты соответствуют нормальным типам поведения системы. В~работе 
рассмотрена задача построения концептов на основе машинного обучения, опирающегося на 
цепочки малых выборок. Построен алгоритм формирования концептов и~доказана его 
эффективность.}
   
  \KW{мониторинг информационной безопасности; малые выборки; обучение на малых 
выборках; формирование концептов}

\DOI{10.14357/19922264190413} 
  
%\vspace*{1pt}


\vskip 10pt plus 9pt minus 6pt

\thispagestyle{headings}

\begin{multicols}{2}

\label{st\stat}
  
  
\section{Введение }

  Многие системы мониторинга информационной безопасности~[1, 2] и~других 
аспектов работы ин\-фор\-ма\-ци\-он\-но-вы\-чис\-ли\-тель\-ных систем получают 
информацию в~виде коротких сообщений, которые можно считать малыми 
выборками. В~силу инерционности информационных систем часто эти 
сообщения поступают сериями, отражая близкие состояния вычислительной 
системы или сети. 
  
  Целью работы мониторинговых систем ставится выявление аномалий в~работе 
отслеживаемых объектов. Известны технологии выявления аномалий, основанные 
на построении моделей нормального поведения~[3]. Однако поступающие 
сообщения не всегда имеют простую структуру~[4]. Не всегда методы 
регрессии~[3] можно применять: например, когда сеть изменяет свое поведение, 
происходит изменение многих параметров функционирования сети. Если 
устройство демонстрирует несколько режимов работы, их описание необходимо 
строить на основе анализа поступающих малых выборок с~помощью процедур 
машинного обучения. 
  
  В последнее время методы машинного обучения получили большое развитие 
(см., например,~[5, 6]). Методы машинного обучения на основе малых выборок 
также подробно изучались~[7]. Один из главных сценариев в~таком обучении 
основан на Concept Learning. Цель этого подхода состоит в~распознавании 
концептов по небольшому числу малых выборок на основе ранее наблюденных 
концептов. Вторая цель этого подхода состоит в~формировании множества 
концептов. В~дальнейшем будет использована терминология теории обучения на 
малых выборках, где под концептами понимаются классы выборок, 
принадлежность к~которым необходимо определять для вновь поступающих 
малых выборок. 
  
  Далее будем предполагать, что данные поступают с~помощью цепочек малых 
выборок. Каждая цепочка однозначно связана с~некоторым кон\-цеп\-том. При этом 
чис\-ло кон\-цеп\-тов неизвестно, но оно конечно. Каждый концепт будет описываться 
множеством выборок. Как отмечалось в~обзоре~[7], наиболее сложная задача 
состоит в~формировании концептов. 
  
  В статье построен и~описан алгоритм формирования концептов и~доказана его 
эффективность.

\vspace*{-6pt}
  
  \section{Математическая модель}
  
  \vspace*{-2pt}
  
  Будем считать, что каждая малая выборка есть слово длины~$N$ в~алфавите 
из~$m$~букв. Каждая цепочка малых выборок конечна, и~для простоты все 
цепочки имеют одинаковую длину~$n$. Концепты формируются с~помощью 
кластеров. 
  
  Цель работы~--- построение корректного алгоритма определения числа 
концептов и~самих концептов. 
  
  Примем следующие условия.\\[-14pt]
  \begin{enumerate}[1.]
\item Каждая малая выборка относится к~одному и~только к~одному концепту.\\[-14pt]
\item Концепты не пересекаются между собой.\\[-14pt]
\item Любая цепочка малых выборок относится только к~одному концепту.\\[-14pt]
  \end{enumerate}
  
  Поскольку концепты будут формироваться постепенно на основании текущей 
кластерной структуры, то все изолированные кластеры будем называть 
\textit{промежуточными концептами}. Каждый промежуточный концепт состоит из:\\[-14pt]
  \begin{itemize}
\item \textit{видимого концепта}, т.\,е.\ малых выборок, которые в~него 
вошли;\\[-14pt] 
\item \textit{невидимого концепта}, т.\,е.\ малых выборок, которые можно было 
бы отнести к~данному промежуточному концепту, но они ранее не встретились;\\[-14pt] 
\item \textit{запретов}, т.\,е.\ малых выборок, которые в~принципе не могут 
входить в~данных концепт.\\[-14pt]
\end{itemize}

  Из сделанных ранее предположений вытекают следующие выводы.\\[-14pt] 
  \begin{enumerate}[1.]
\item Если в~цепочке есть хотя бы одна малая выборка из существующего 
промежуточного концепта, то вся цепочка относится к~этому промежуточному 
концепту, хотя почти все ее элементы могут быть невидимыми для данного 
промежуточного концепта.\\[-14pt]
\item Если в~цепочке встретились по крайней мере две малые выборки, 
принадлежащие разным промежуточным концептам, то эти два промежуточных 
концепта объединяются в~единый промежуточный концепт. При этом остальные 
элементы цепочки принадлежат этому объединенному промежуточному 
концепту.\\[-14pt]
\item Если в~полученной цепочке нет ни одной малой выборки, принадлежащей 
одному из существующих промежуточных концептов, то такая цепочка образует 
новый промежуточный концепт. При этом надо помнить, что эта цепочка может 
состоять из невидимых элементов какого-то существующего промежуточного 
концепта.\\[-14pt] 
\end{enumerate}
  
  Указанные шаги~1--3 фактически формируют алгоритм обработки цепочек 
малых выборок и~преобразования промежуточных концептов. 
  
  Предположим, что существует некоторое семейство концептов $M_1,\ldots , 
M_k$ такое, что каждая малая выборка из пространства возможных малых 
выборок принадлежит одному из этих концептов. 
  
  Покажем, что предложенный алгоритм позволяет определить число концептов 
и~определить их содержание. Для простоты будем считать, что $n\hm= 2$. Выбор 
цепочек осуществляется случайно следующим образом. Сначала выбирается 
концепт, из которого выбирается цепочка для простоты в~соответствии 
с~равномерным распределением на множестве целых чисел $\{1,\ldots , k\}$. По 
условию каждая цепочка выбирается из одного концепта. 
  
  Обозначим через $\vert M_i\vert \hm=s_i$, $i\hm=1,\ldots , k$. При равномерном 
выборе малой выборки из кон\-цеп\-та~$M_i$ получим, что вероятность 
$$
{\sf P}\left(x_1,x_2\right)=\fr{1}{s_i(s_i-1)}\,,
$$
 где $x_, x_2\hm\in M_i$. Тогда $1\hm- 1/(s_i(s_i\hm-
1))$~--- вероятность того, что данная цепочка не встретится на фиксированном 
месте в~последовательности выбора цепочек из~$M_i$, $i\hm=1,\ldots , k$. 
  
  Пусть в~последовательности длины~$t$ выбранных из~$M_i$ цепочек ни разу 
не встретится цепочка $(x_1,x_2)$. Вероятность такого события равна $(1-
1/(s_i(s_i\hm-1)))^t$. 
  
  Из леммы Бореля--Кан\-тел\-ли и~полученных выше оценок следует, что 
в~бесконечной последовательности выборок из множества~$M_i$ появление 
цепочки $(x_1,x_2)$ произойдет бесконечное число раз. Кроме того, из той же 
леммы следует, что с~вероятностью~1 существует бесконечная 
последовательность появления концепта~$M_i$ в~указанной выше вероятностной 
схеме. 
  
  Рассмотрим бесконечную схему преобразования данных в~кластеры с~целью 
построения концептов. Пусть время дискретно и~в данный момент сформированы 
кластеры $K_1,\ldots ,K_r$ промежуточных концептов. Пусть получена очередная 
цепочка $(x_1,x_2)$. Тогда:
  \begin{enumerate}[(1)]
\item если~$x_1$ и~$x_2$ принадлежат некоторому клас\-те\-ру~$K_i$, то 
кластерная структура не изменяется; 
\item если один элемент $x_1$ или~$x_2$ ранее не встречался, а~второй 
элемент принадлежит промежуточному концепту~$K_i$, то клас\-тер~$K_i$ 
увеличивается на один ранее не встречавшийся элемент; 
\item если элемент~$x_1$ принадлежит некоторому клас\-те\-ру~$K_i$, 
а~элемент~$x_2$ принадлежит некоторому кластеру~$K_j$, $i\not= j$, то 
в~новой клас\-тер\-ной структуре вместо кластеров~$K_i$ и~$K_j$ появляется 
новый кластер $K_i\cup K_j$;
\item если ни один из элементов цепочки $(x_1,x_2)$ ранее не встречался 
и~не принадлежит ни одному кластеру, то элементы этой цепочки образуют 
новый кластер. 
\end{enumerate}
  
  Пусть первая цепочка, полученная для по\-стро\-ения концептов,~--- $(x_1, x_2)$, 
где $x_1, x_2\hm\in M_1$. Тогда для каждого элемента $x_3\hm\in M_1$ пара 
$(x_1, x_3)$ встречается с~ве\-ро\-ят\-н\-остью~1. Поскольку концепт~$M_1$ по 
определению конечен, то элемент~$x_1$ встретится в~сочетании со всеми 
элементами~$M_1$ с~ве\-ро\-ят\-ностью~1. Таким образом, концепт~$M_1$ будет 
однозначно восстановлен. Так как каждый из концептов выбирается 
в~бесконечной последовательности бесконечное число раз, то с~вероятностью~1 
будут восстановлены все другие концепты. 
  
  Докажем, что полученная структура не может быть противоречивой. Пусть 
цепочка $(x_1, x_2)$ такова, что $x_1\hm\in M_1$, а~$x_2\hm\in M_2$. Это 
противоречит условию, что каждая цепочка выбирается из одного концепта. 
  
  Докажем, что ни одна выборка~$x_1$ не может быть пропущена в~результате 
работы алгоритма. Пусть элемент~$x_2$ принадлежит тому же концепту, что 
и~$x_1$. Тогда, как было показано выше, цепочка $(x_1, x_2)$ появляется 
с~вероятностью~1 в~последовательности цепочек из~$M_1$. Если при этом 
известно, что $x_2\hm\in M_1$, то и~$x_1\hm\in M_1$, т.\,е.~$x_1$ не может быть 
пропущен.

\vspace*{-6pt}
  
 \section{Эффективность алгоритма построения концептов}
 
 \vspace*{-2pt}
  
  Рассмотрим задачу определения того, что клас\-тер\-ная структура соответствует 
структуре концептов. Определим граф~$G$ с~ребрами $(x_i, x_j)$, которые 
соответствуют появившимся цепочкам малых выборок. После того как все 
концепты~$M_i$, $i\hm=1,\ldots ,k$, определены, граф~$G$ представляет 
собой~$k$~компонентов связности, каждый их которых является полным графом. 
Таким образом, появление полных графов в~кластерной структуре компонентов 
связности служит признаком (недостаточным) того, что концепты построены. 
Изоляция кластеров, которые являются полными графами,~--- признак 
восстановления структуры концептов. 
  
  Число ребер в~графе, соответствующем концепту~$M_i$, равно 
$\begin{pmatrix} s_i\\ 2\end{pmatrix}$. Минимальное число цепочек, которое 
необходимо для появления признаков формирования концептов, равно 

\noindent
  $$
  R= \sum\limits_{i=1}^k \begin{pmatrix} s_i\\ 2\end{pmatrix}\,. 
  $$
  
  \vspace*{-2pt}
  
  Устойчивость структуры восстановленных концептов будет видна, когда число 
цепочек станет равным~$rR$, где $r\hm>1$. Таким образом, получена нижняя 
оценка сложности алгоритма восстановления концептов, и~она имеет 
квадратичный порядок.

\vspace*{-6pt}
  
  \section{Заключение }
  
  Построен алгоритм формирования концептов, не использующий семантический 
анализ содержания малых выборок. Это делает алгоритм универсальным 
в~подобных задачах. 
  
  Условие принадлежности малой выборки одному концепту можно заменить на 
меру близости, связанной с~содержанием малой выборки. Тогда при построении 
концептов возможны ошибки. Однако предложенный контроль при построении 
полного графа может скомпенсировать эти ошибки.
  
  В данной работе не рассмотрена задача снижения сложности алгоритма 
построения концептов. Возможно, что рассмотренную выше идею можно 
реализовать с~меньшей сложностью.

\vspace*{-6pt}
  
{\small\frenchspacing
 {%\baselineskip=10.8pt
 \addcontentsline{toc}{section}{References}
 \begin{thebibliography}{9}

\bibitem{2-gz}
\Au{Грушо А., Грушо~Н., Тимонина~Е., Шоргин~С.} Возможности построения безопасной 
архитектуры для динамически изменяющейся информационной системы~// Системы и~средства 
информатики, 2015. Т.~25. №\,3. С.~78--93.
\bibitem{1-gz}
\Au{Grusho A., Grusho~N., Timonina~E.}
 The bans in finite probability spaces and the problem of 
small samples~// Distributed computer and communication networks~/ Eds. V.\,M.~Vishnevskiy, K.\,E.~Samouylov, 
D.\,V.~Kozyrev.~--- Lecture notes in computer 
science ser.~--- Springer,  2019. Vol.~11965. P.~578--590.

\bibitem{3-gz}
\Au{Тьюки Дж.} Анализ результатов наблюдений. Разведочный анализ приложения~/ Пер. 
с~англ.~--- М.: Мир, 1981. 694~с. (\Au{Tukey~J.\,W.} Exploratory data analysis.~--- Addison 
Wesley, 1977. 711~р.)
\bibitem{4-gz}
\Au{Grusho A., Grusho~N., Timonina~E.} Detection of anomalies in non-numerical data~// 8th 
Congress (International) on Ultra Modern Telecommunications and Control Systems and Workshops 
Proceedings.~--- Piscataway, NJ, USA: IEEE, 2016. P.~273--276.
\bibitem{5-gz}
\Au{Jordan M.\,I., Mitchell~T.\,M.} Machine learning: Trends, perspectives, and prospects~// Science, 
2015. Vol.~349. Iss.~6245. P.~255--260.
\bibitem{6-gz}
\Au{Bramley N.\,R.} Constructing the world: Active causal learning in cognition.~--- 
London: University College London, 2017.  PhD thesis. 361~p. 
\bibitem{7-gz}
\Au{Shu~J., Zongben~X., Deyu~M.} Small sample learning in big data era~// arXiv.org, 
2018. 76~p. arXiv:1808.04572v3 [cs.LG].
 \end{thebibliography}

 }
 }

\end{multicols}

\vspace*{-6pt}

\hfill{\small\textit{Поступила в~редакцию 30.09.19}}

%\vspace*{8pt}

\pagebreak

\newpage

\vspace*{-28pt}

%\hrule

%\vspace*{2pt}

%\hrule

%\vspace*{-2pt}

\def\tit{CONCEPTS FORMING ON~THE~BASIS OF~SMALL SAMPLES}


\def\titkol{Concepts forming on~the~basis of~small samples}

\def\aut{A.\,A.~Grusho, M.\,I.~Zabezhailo, N.\,A.~Grusho, and~E.\,E.~Timonina}

\def\autkol{A.\,A.~Grusho, M.\,I.~Zabezhailo, N.\,A.~Grusho, and~E.\,E.~Timonina}

\titel{\tit}{\aut}{\autkol}{\titkol}

\vspace*{-11pt}


 \noindent
   Institute of Informatics Problems, Federal Research Center ``Computer Sciences and 
Control'' of the Russian Academy of Sciences; 44-2~Vavilov Str., Moscow 119133, 
Russian Federation

\def\leftfootline{\small{\textbf{\thepage}
\hfill INFORMATIKA I EE PRIMENENIYA~--- INFORMATICS AND
APPLICATIONS\ \ \ 2019\ \ \ volume~13\ \ \ issue\ 4}
}%
 \def\rightfootline{\small{INFORMATIKA I EE PRIMENENIYA~---
INFORMATICS AND APPLICATIONS\ \ \ 2019\ \ \ volume~13\ \ \ issue\ 4
\hfill \textbf{\thepage}}}

\vspace*{3pt}  


      
   
   \Abste{Monitoring systems of information security of information systems obtain information in 
the form of chains of short messages which can be considered as chains of small samples. Often, 
owing to an inertance of information systems, these chains reflect close statuses of the computing 
system or network. In the paper, it is supposed that work of the system can be presented in the form of 
a finite set of modes which are called concepts. Violations of security are detected by means of 
anomalies that are associated with emergence of new concepts. 
   The known technologies of identification of anomalies are based on creation of a model of a 
normal system's behavior. Concepts correspond to normal types of a~system's behavior. In the paper, 
the problem of creation of concepts on the basis of machine learning based on chains of small samples 
is considered. The algorithm of concepts forming is constructed and its efficiency is proved.} 
   
   \KWE{information security monitoring; small samples; small sample learning; concepts forming}
   
  

\DOI{10.14357/19922264190413} 

%\vspace*{-14pt}

 \Ack
   \noindent
   The paper was partially supported by the Russian Foundation for Basic Research (project  
18-29-03081).


%\vspace*{-6pt}

  \begin{multicols}{2}

\renewcommand{\bibname}{\protect\rmfamily References}
%\renewcommand{\bibname}{\large\protect\rm References}

{\small\frenchspacing
 {%\baselineskip=10.8pt
 \addcontentsline{toc}{section}{References}
 \begin{thebibliography}{9}

\bibitem{2-gz-1}
\Aue{Grusho, A., N.~Grusho, E.~Timonina, and S.~Shorgin.} 2015. Vozmozhnosti 
postroeniya 
bezopasnoy arkhitektury dlya dinamicheski izmenyayushcheysya informatsionnoy sistemy 
[Possibilities of secure architecture creation for dynamically changing information systems]. 
\textit{Sistemy i~Sredstva Informatiki~--- Systems and Means of Informatics} 25(3):78--93.

\bibitem{1-gz-1}
\Aue{Grusho, A., N.~Grusho, and E.~Timonina.} 2019. 
The bans in finite probability spaces and 
the problem of small samples. \textit{Distributed computer and communication networks}.
Eds. V.\,M.~Vishnevskiy, 
K.\,E.~Samouylov, and D.\,V.~Kozyrev. Lecture notes
in computer science ser. Springer. 11965:578--590.



\bibitem{3-gz-1}
\Aue{Tukey, J.\,W.} 1977. \textit{Exploratory data analysis}. Addison Wesley. 
711~р.
\bibitem{4-gz-1}
\Aue{Grusho, A., N.~Grusho, and E.~Timonina.} 2016. Detection of anomalies in non-numerical 
data. \textit{8th  Congress (International) on Ultra Modern Telecommunications and Control 
Systems and Workshops Proceedings}. Piscataway, NJ: IEEE. 273--276.

\vspace*{2pt}

\bibitem{5-gz-1}
\Aue{Jordan, M.\,I., and T.\,M.~Mitchell.} 2015. 
Machine learning: Trends, perspectives, 
and prospects. \textit{Science} 349(6245):\linebreak 255--260.

\vspace*{2pt}

\bibitem{6-gz-1}
\Aue{Bramley, N.\,R.} 2017. Constructing the world: 
Active causal learning in cognition.  London: University College London. PhD  Thesis. 361~p.

\vspace*{2pt}

\bibitem{7-gz-1}
\Aue{Shu, J., X.~Zongben, and M.~Deyu.} 2018. Small sample learning in big data era. Available 
at: {\sf https://arxiv.org/ abs/1808.04572} (accessed October~9, 2019).
\end{thebibliography}

 }
 }

\end{multicols}

\vspace*{-6pt}

\hfill{\small\textit{Received September 30, 2019}}

%\pagebreak

\vspace*{-22pt}

\Contr


\noindent
\textbf{Grusho Alexander A.} (b.\ 1946)~--- Doctor of Science in physics and 
mathematics, professor, principal scientist, Institute of Informatics Problems, Federal 
Research Center ``Computer Sciences and Control'' of the Russian Academy of 
Sciences; 44-2~Vavilov Str., Moscow 119133, Russian Federation; 
\mbox{grusho@yandex.ru}

\vspace*{3pt} 

\noindent
\textbf{Zabezhailo Michael I.} (b.\ 1956)~--- Doctor of Science in physics and 
mathematics, principal scientist, Institute of Informatics Problems, Federal Research 
Center ``Computer Sciences and Control'' of the Russian Academy of Sciences;  
44-2~Vavilov Str., Moscow 119133, Russian Federation; 
\mbox{m.zabezhailo@yandex.ru} 

\vspace*{3pt}

\noindent
\textbf{Grusho Nikolai A.} (b.\ 1982)~--- Candidate of Science (PhD) in physics 
and mathematics, senior scientist, Institute of Informatics Problems, Federal 
Research Center ``Computer Sciences and Control'' of the Russian Academy of 
Sciences; 44-2~Vavilov Str., Moscow 119133, Russian Federation; 
\mbox{info@itake.ru} 
 
\vspace*{3pt}

\noindent
\textbf{Timonina Elena E.} (b.\ 1952)~--- Doctor of Science in technology, professor, 
leading scientist, Institute of Informatics Problems, Federal Research Center 
``Computer Sciences and Control'' of the Russian Academy of Sciences; 44-2~Vavilov 
Str., Moscow 119133, Russian Federation; \mbox{eltimon@yandex.ru}
\label{end\stat}

\renewcommand{\bibname}{\protect\rm Литература}          %01
\def\stat{moskaleva}

\def\tit{ВЛИЯНИЕ ПАРАМЕТРОВ ИЗОЛЯЦИИ НА~РАЗДЕЛЕНИЕ РЕСУРСОВ ПРИ~НАРЕЗКЕ 
СЕТИ$^*$}

\def\titkol{Влияние параметров изоляции на~разделение ресурсов при~нарезке 
сети}

\def\aut{Ф.\,А.~Москалева$^1$, Ю.\,В.~Гайдамака$^2$, В.\,С.~Шоргин$^3$}

\def\autkol{Ф.\,А.~Москалева, Ю.\,В.~Гайдамака, В.\,С.~Шоргин}

\titel{\tit}{\aut}{\autkol}{\titkol}

\index{Москалева Ф.\,А.}
\index{Гайдамака Ю.\,В.}
\index{Шоргин В.\,С.}
\index{Moskaleva F.\,A.}
\index{Gaidamaka Yu.\,V.}
\index{Shorgin V.\,S.}


{\renewcommand{\thefootnote}{\fnsymbol{footnote}} \footnotetext[1]
{Публикация выполнена при поддержке Программы стратегического академического лидерства РУДН 
и при финансовой поддержке РФФИ (проекты 19-07-00933 и 20-07-01064).}}


\renewcommand{\thefootnote}{\arabic{footnote}}
\footnotetext[1]{Российский университет дружбы народов, moskaleva-fa@rudn.ru}
\footnotetext[2]{Российский университет дружбы народов; Институт проблем информатики 
Федерального исследовательского центра <<Информатика и~управ\-ле\-ние>> Российской академии 
наук, \mbox{gaidamaka-yuv@rudn.ru}}
\footnotetext[3]{Институт проблем информатики Федерального исследовательского центра 
<<Информатика и~ управ\-ле\-ние>> Российской академии наук, \mbox{vshorgin@ipiran.ru}}
%
\vspace*{-15pt}

  
  \Abst{Технология нарезки радиоресурсов сети определяется как один из основных 
компонентов пятого поколения мобильных коммуникаций, способных решить проблему 
колоссального роста объема трафика данных в сотовых сетях. Ключевая особенность 
нарезки радиоресурсов сети, или сетевого слайсинга, позволяющая ограничить влияние 
одного слайса на другой, заключается в обеспечении изолированных гарантий 
производительности для предоставления высокого качества обслуживания (QoS, Quality of Service). 
В~статье  
с~по\-мощью аппарата тео\-рии массового обслуживания построена модель разделения 
ресурсов при нарезке сети, позволяющая исследовать разделение ресурсов в соответствии 
с различными стратегиями справедливости. Задача разделения ресурсов сформулирована 
в~форме задачи оптимизации относительно зависящей от параметра изоляции весовой 
функции ресурса системы, занятого заявками каждого слайса. Проведенный численный 
анализ показал существенное влияние параметра изоляции на изменение характеристик 
производительности сис\-темы.}
  
  \KW{нарезка сети; справедливое разделение ресурсов; изоляция слайсов; параметр 
изоляции}
\DOI{10.14357/19922264200402} 
  
\vspace*{-3pt}


\vskip 10pt plus 9pt minus 6pt

\thispagestyle{headings}

\begin{multicols}{2}

\label{st\stat}

\section{Введение}

  Нарезка радиоресурсов сети (\textit{англ}.\ network slicing)~--- ключевая 
технология, позволяющая операторам сети предоставлять свою физическую 
инфраструктуру для поддержки услуг с различными\linebreak требованиями~[1]. 
Определенный набор услуг\linebreak может быть связан с логически независимой 
сквоз\-ной сетью, т.\,е.\ слайсом. Слайс (\textit{англ}.\ slice)~--- логи-\linebreak ческая 
сеть, обеспечивающая определенные\linebreak функциональные возможности и 
сетевые харак\-те\-ри\-сти\-ки~[2]. Слайсы настраиваются и управ\-ля\-ют\-ся 
арендаторами (\textit{англ}.\ tenants), например виртуальными операторами 
(Virtual Network Operator, VNO), которым мобильный оператор, владеющий 
инфраструктурой, делегирует контроль над использованием ресурсов и 
качеством предоставления услуг внутри слайса. 

Концепция нарезки сети 
подразумевает автоматизацию создания и настройки слайса, изоляцию 
слайсов (независимость показателей качества обслуживания в слайсе от 
трафика в других слайсах, а~также безопас\-ность и~т.\,п.), эластичность 
нарезки (справедливое~\cite{4-mos} и эффективное использование ресурсов, 
адаптация к~условиям), иерархию управ\-ления (самоуправление в слайсе), 
возможность назначать приоритетные слайсы~\cite{3-mos}.
  
  Гарантии качества обслуживания в логической сети обеспечиваются 
изоляцией слайса, так что никакие изменения в других слайсах не могут 
повлиять на показатели качества обслуживания (QoS). При 
этом выбор стратегии изоляции и разделения ресурсов на 
радиоинтерфейсе~--- достаточно сложная  
задача~\cite{5-mos}. 

Из-за стохастической природы беспроводной среды и 
высокой изменчивости трафика во времени и пространстве идеальную 
изоляцию можно обеспечить лишь в случае резервирования ресурсов 
в~соответствии с наихудшими ожидаемыми условиями, что ведет к 
неэффективному использованию ресурсов в~большинстве случаев. 
Попытка 
учесть стохастическую природу беспроводной среды и ее особенности 
с~использованием аппарата цепей Маркова сделана в~\cite{6-mos}. 

Как показано в~\cite{7-mos}, устанавливая взаимосвязь параметров сети, 
можно достичь баланса между изоляцией и эф\-фек\-тив\-ностью, что позволяет 
прио\-ри\-тизировать и~настраивать каждый слайс в со\-от\-вет\-ствии с 
конкретными задачами, для которого он используется. При этом управ\-ле\-ние 
ресурсами как внутри слайса, так и~меж\-ду разными слайсами должно 
гарантировать не только изоляцию слайса, но и справедливость разделения 
ресурсов между пользователями~\cite{8-mos}. 

В~\cite{9-mos} механизмы 
нарезки сети с учетом гарантий для различных типов трафика исследованы 
при фиксированных значениях параметров изоляции.
  
  В статье построена математическая модель разделения радиоресурсов 
соты между двумя виртуальными операторами (далее~--- операторами), 
которая иллюстрирует влияние параметров изоляции на мет\-рики 
производительности сети. Положим, что на базовой станции сети связи 
пятого поколения с~технологией радиодоступа New Radio активированы два 
слайса, принадлежащие разным операторам, и модуль нарезки делит между 
ними общий ресурс ем\-костью~$C$  единиц ресурса. Примером единицы 
ресурса может быть герц для полосы радиочастот, бит в секунду для 
ско\-рости или ресурсный блок LTE/NR (long-term evolution\,/\,new radio). 
Каждый оператор осуществляет  
предостав\-ле\-ние некоторой услуги связи, пред\-по\-ла\-га\-ющей 
непрерывную передачу пользователю потокового трафика на выделенном 
ресурсе не менее $b_{\min}, d_{\min}\hm >0$ и~не более $b_{\max}, 
d_{\max}\hm>0$ единиц ресурса для первого и второго слайсов 
соответственно. Длительность предостав\-ле\-ния услуги пользователю 
(длительность пользовательской сессии) определяется объемом 
выделенного для обслуживания сессии ресурса, который зависит от числа 
активных сессий в каж\-дом слайсе. Считаем, что ресурсы каждого слайса 
делятся между его пользователями поровну. Для обеспечения 
справедливого разделения ресурсов введены параметры изоляции, 
определяющие чис\-ло пользователей в каждом слайсе, которым оператор 
обязуется предоставить услугу с~минимальным качеством. При низ\-кой 
за\-гру\-жен\-ности соты каждой сессии\linebreak выделяется ресурс, достаточный 
для получения пользователем услуги на максимальной ско\-рости. 
С~увеличением нагрузки, создаваемой запросами пользователей обоих 
операторов, вы\-де\-ля\-емый каж\-дой сессии ресурс снижается, пока не 
достигнет уровня минимальной ско\-рости, тре\-бу\-емой для 
предоставления услуги. При дальнейшем росте нагрузки начинает работать 
концепция нарезки сети, согласно которой для приема в сис\-те\-му заявки 
слайса, не достигшего заданного па\-ра\-мет\-ром изоляции предела, долж\-но 
быть прервано обслуживание одной или нескольких заявок второго 
слайса, т.\,е.\ <<нарушителя>> (\textit{англ}.\ violator), который пользовался 
простаивающим ресурсом недогруженного слайса. 

В~сле\-ду\-ющих разделах 
построена модель в виде\linebreak сис\-те\-мы массового обслуживания (СМО), 
позво\-ля\-ющая исследовать за\-ви\-си\-мость показателей качества 
обслуживания от параметров изоляции,\linebreak предложен алгоритм разделения 
радиоресурсов с~по\-мощью методов теории оптимизации, приведен 
пример  
чис\-лен\-но\-го анализа полученных результатов.
  
  \section{Математическая модель системы с~двумя слайсами} %2
  
  Пусть в многолинейную СМО поступают два пуассоновских потока 
заявок с интенсивностями~$\lambda_1$ и~$\lambda_2$ (рис.~1), 
соответствующие запросам на установление сессии от пользователей двух 
операторов. Длительности обслуживания заявок 1-го и~2-го потоков 
распределены по экспоненциальному закону с~параметрами~$\mu_1$ 
и~$\mu_2$ соответственно. Чис\-ло единиц ресурса, выделяемых заявке, 
принятой на обслуживание, зависит от общей за\-гру\-жен\-ности сис\-те\-мы 
и~варьируется в~диапазонах $[b_{\min}, b_{\max}]$ и~$[d_{\min}, d_{\max}]$ 
для 1-го и~2-го потоков со\-от\-вет\-ст\-венно.
  
  В системе предусмотрены параметры изоляции~$\overline{M}$ и 
$\overline{N}$, характеризующие максимальные значения числа заявок 1-го 
и~2-го потока, для которых обслуживание гарантировано.

  Состояние системы описывает случайный процесс 
$\boldsymbol{X}(t)\hm= \left(M(t), N(t)\right)$, где $M(t)$~--- число заявок 1-го потока; 
$N(t)$~--- число заявок 2-го потока в~момент~$t$, с пространством 
состояний
\begin{multline*}
  \mathbb{X}=\left\{ (m,n): m=0,\ldots , \left\lfloor \fr{C}{b_{\min}}\right\rfloor\,,\right.\\ 
n=0,\ldots ,\enskip
\left. \left\lfloor \fr{C}{d_{\min}}\right\rfloor \,,\ mb_{\min}+nd_{\min}\leq 
C\right\}\,.
  \end{multline*}
  
  Введем величины $b(m,n)$ и $d(m,n)$, обо\-зна\-ча\-ющие число единиц 
ресурса, выделенных для обслуживания одной заявки 1-го и 2-го потоков 
соответственно в состоянии $(m,n)\hm\in\mathbb{X}$. Заметим, что 
$b_{\min}\hm\leq b(m,n)\hm\leq b_{\max}$, $d_{\min}\hm\leq d(m,n)\hm\leq 
d_{\max}$, при этом значения~$b(m,n)$ и~$d(m,n)$ являются решением 
задачи оптимизации, которая сформулирована в разд.~3 статьи. 
  
  Интенсивности обслуживания заявок 1-го и 2-го потока определяются как 
$mb(m,n)\mu_1$ и~$nd(m,n)\mu_2$ соответственно.
  

  Параметры изоляции $\overline{M}$ и~$\overline{N}$ управляют при\-емом 
в систему поступающих заявок и~прерывани-\linebreak
\vspace*{-12pt}

{ \begin{center}  %fig1
 \vspace*{3pt}
    \mbox{%
 \epsfxsize=79mm 
 \epsfbox{mos-1.eps}
 }
\vspace*{3pt}

\noindent
{{\figurename~1}\ \ \small{Модель СМО}}
\end{center}
}

%\vspace*{6pt}


\setcounter{figure}{1}
\begin{figure*} %fig2
\vspace*{1pt}
    \begin{center}  
  \mbox{%
 \epsfxsize=161.667mm 
\epsfbox{mos-2.eps}
 }
\end{center}
\vspace*{-11pt}
\Caption{Центральное состояние диаграммы интенсивностей переходов случайного 
процесса~$\mathbb{X}(t)$}
\end{figure*}



\pagebreak

\noindent
ем обслуживания ранее 
принятых заявок следу\-ющим образом. Если при поступлении заявки 1-го 
потока в состоянии $(m,n)\hm\in \mathbb{X}$, где $m\hm<\overline{M}$, 
$n\hm>\overline{N}$,  
в~сис\-т\-еме недостаточно свободного ресурса для обслуживания $(m+1)$ 
заявок 1-го потока с выделением каждой минимального числа~$b_{\min}$ 
единиц ресурса, поступающая заявка 1-го потока вытесняет одну или 
несколько заявок 2-го потока, чтобы встать на обслуживание (рис.~2). 
Число $k(m,n)$ таких заявок, которые после прерывания обслуживания 
покинут систему, не оказывая влияния на ее дальнейшее функционирование, 
вычисляется следующим образом:
  \begin{multline*}
  k(m,n)=\left\lceil \fr{(m+1)b_{\min}+nd_{\min}-C}{d_{\min}}\right\rceil\,,\\ 
(m,n)\in\mathbb{X}\,,\enskip (m+1,n)\not\in\mathbb{X}\,,\enskip k(m,n)\leq n\,.
\end{multline*}
  
  Аналогично для случая $m\hm>\overline{M}$, $n\hm> \overline{N}$ для 
при\-ема заявки 2-го потока будет прервано обслуживание $s(m,n)$ заявок 1-го потока:
  \begin{multline*}
  s(m,n)= \left\lceil \fr{mb_{\min}+(n+1)d_{\min}-C}{b_{\min}}\right\rceil,\\ 
(m,n)\in\mathbb{X}\,,\enskip (m,n+1)\not\in\mathbb{X}\,,\enskip s(m,n)\leq m\,.
 \end{multline*}
  
  Для случая $m<\overline{M}$, $n<\overline{N}$ и для случая 
$m\hm>\overline{M}$, $n\hm>\overline{N}$ поступающая заявка теряется, 
если в системе недостаточно свободного ресурса для ее приема в систему.
  
  Таким образом, множества потери $\mathbb{B}_1^{\mathrm{arr}}$ заявок \mbox{1-го}
потока и~$\mathbb{B}_2^{\mathrm{arr}}$ заявок 2-го потока при поступлении 
(\textit{англ}.\ arrival) и~множества прерывания 
обслуживания~$\mathbb{B}_1^{\mathrm{pr}}$ заявок 1-го потока 
и~$\mathbb{B}_2^{\mathrm{pr}}$ заявок 2-го потока при вытеснении (\textit{англ}.\ 
preemption)~\cite{10-mos} имеют сле\-ду\-ющий вид:
  \begin{align*}
&  \mathbb{B}_1^{\mathrm{arr}}\!=\!\left\{\! (m,n)\in \mathbb{X}: 
(m+1,n)\not\in\mathbb{X}\,,\,m\geq \overline{M}\right\}\!;\\[3pt]
&  \mathbb{B}_2^{\mathrm{arr}}\!=\!\left\{\! (m,n)\in \mathbb{X}: 
(m, n+1)\not\in\mathbb{X}\,,\,n\geq \overline{N}\right\}\!;\\[3pt]
&  \mathbb{B}_1^{\mathrm{pr}}\!=\!\left\{\! (m,n)\in \mathbb{X}: 
(m,n+1)\not\in\mathbb{X}\,,\,m>\overline{M},\,n<\overline{N}\right\}\!;\\[3pt] \hspace*{-.3pt}
&\mathbb{B}_2^{\mathrm{pr}}\!=\!\left\{\! (m,n)\in \mathbb{X}: 
(m+1,n)\not\in\mathbb{X}\,,\,m<\overline{M},\,n>\overline{N}\right\}\!. \hspace*{-.3pt}
\end{align*}

Распределение стационарных вероятностей~$\mathbf{p}$ получаем путем 
решения системы линейных уравнений:
\begin{align*}
\mathbf{pA} &=\mathbf{0}\,;\\
\mathbf{p1} &=\mathbf{1}\,,
\end{align*}
где $\mathbf{A}$~--- инфинитезимальная матрица, элементы которой $a\left( 
(m,n),(m^\prime n^\prime)\right)$ записаны ниже:
\begin{multline*}
 a\left( (m,n), \left( m^\prime,n^\prime\right)\right) ={}\\
{}=\! 
\begin{cases}
\lambda_1\,,  & m+1,\ n^\prime=n\left( m^\prime, n^\prime\right)\in \mathbb{X}\,;\\[3pt]
\lambda_1\,, & m^\prime=m+1\,,\ n^\prime=n-k\,,\\
   & (m+1,n)\notin\mathbb{X}\,,\ 
m<\overline{M},\ n\geq \overline{N}\,;\\[3pt]
\lambda_2\,, &m^\prime=m,\ n^\prime=n+1,\
   \left(m^\prime,n^\prime\right)\in\mathbb{X}\,;\\[3pt]
\lambda_2\,, & m^\prime=m-s\,,\ n^\prime=n+1\,,\\
   & (m,n+1)\notin \mathbb{X}\,,\ 
m\geq\overline{M}\,,\ n< \overline{N}\,;\\[3pt]
mb(m,n)\mu_1\,,\hspace*{-6pt}& m^\prime=m-1\,,\ n^\prime=n\,,\ m^\prime>0\,;\\[3pt]
nd(m,n)\mu_2\,,\hspace*{-6pt}& m^\prime=m\,,\ n^\prime=n-1\,,\ n^\prime>0 \,;\\[3pt]
A\,, & m^\prime=m\,,\ n^\prime=n\,;\\[3pt]
0 &\mbox{иначе}.
\end{cases}\hspace*{-8.8pt}
\end{multline*}
Здесь диагональные элементы $a\left( (m,n), m,n)\right)\hm=A$ имеют 
следующий вид:
\begin{multline*}
A=-\lambda_1 I\left((m+1,n)\in\mathbb{X},n\geq N\right) -{}\\
{}-\lambda_2 
I\left( (m,n+1)\in\mathbb{X},m\geq M\right)-{}\\
{}- mb(m,n)\mu_1 I(m>0)-nd(m,n)\mu_2 I(n>0)\,.
%\label{e3-mos}
\end{multline*}

  Получив распределение вероятностей, можно найти некоторые 
характеризующие производительность системы метрики для заявок 1-го и 
2-го потока:
  \begin{itemize}
  \item вероятность потери заявки при поступлении и~вероятность 
прерывания обслуживания заявки при вытеснении ($s=1, 2$)
  \begin{align*}
  B_s^{\mathrm{arr}}&=\sum\limits_{(m,n)\in \mathbb{B}_s^{\mathrm{arr}}} p(m,n)\,;\\
  B_s^{\mathrm{pr}}&=\sum\limits_{(m,n)\in \mathbb{B}_s^{\mathrm{pr}}} p(m,n)\,;
  \end{align*}
\item среднее время обслуживания заявки
\begin{multline*}
\hspace*{-15pt}S_1={}\\
\hspace*{-15pt}{}={N_1} \!\left(\!\lambda_1(1-B_1^{\mathrm{arr}})- %\vphantom{\sum\limits_{(m,n)}}{}\right.\\[-9pt]
\lambda_2\!\!\! \sum\limits_{(m,n)\in \mathbb{B}_1^{\mathrm{arr}}}\!\!\! 
s(m,n)p(m,n)\!\right)^{\!-1}\!;\hspace*{-9pt}
\end{multline*}

\vspace*{-12pt}

\noindent
\begin{multline*}
\hspace*{-15pt}S_2={}\\
\hspace*{-15pt}{}={N_2} \!\left(\!\lambda_2(1-B_2^{\mathrm{arr}})- %\vphantom{\sum\limits_{(m,n)}}{}\right.\\[-9pt]
\lambda_1 \!\!\!\sum\limits_{(m,n)\in 
\mathbb{B}_2^{\mathrm{arr}}} \!\!\!k(m,n)p(m,n)\!\right)^{\!-1}\!;\hspace*{-9pt}
\end{multline*}
\item вероятность нарушения
$$
V_1=\sum\limits_{m>\overline{M}} p(m,n)\,;\quad 
V_2=\sum\limits_{n>\overline{N}} p(m,n)\,.
$$
  \end{itemize}
  
  В следующем разделе сформулирована задача оптимизации для 
вычисления значений  
величин $b(m,n)$ и~$d(m,n)$, обеспечивающая эффективное использование 
ресурса системы.
  
  \section{Решение задачи разделения ресурсов} %3
  
  Для состояний $(m,n)\in \mathbb{X}$, в которых $mb_{\max}\hm+ 
nd_{\max}\hm> C$, необходимо определить значения величин $b(m,n)$ 
и~$d(m,n)$, позволяющие не только полностью использовать ресурс 
системы, но и максимизировать некоторую функцию по\-лез\-ности. Для 
примера выберем стратегию max-min-справедливости при разделении 
ресурсов (\textit{англ.}\  
max-min fairness), функция полезности при кото-\linebreak
\vspace*{-12pt}
\columnbreak

\noindent
рой~\cite{11-mos} относится 
к~логарифмическому типу, является возрастающей, строго вогнутой и 
непрерывно дифференцируемой и определяется как $U(x)\hm= \ln x$, $\min 
\left( b_{\min}, d_{\min}\right)\hm\leq x\hm\leq \max\left( b_{\max}, 
d_{\max}\right)$. 
  
  Разделение ресурса между заявками в системе соответствует решению 
следующей задачи оптимизации:
  \begin{equation}
  \left.
  \begin{array}{c}
  w_1(m,n) mU(b(m,n))+{}\hspace*{60pt}\\[6pt]
  \hspace*{20pt}{}+w_2(m,n) nU(d(m,n))\to \max\\[6pt]
  \mbox{s.t.}\ mb(m,n)+nd(m,n)=C\,;\\[6pt]
  b_{\min} \leq b(m,n)\leq b_{\max}\,;\\[6pt]
  d_{\min}\leq d(m,n)\leq d_{\max}\,.
  \end{array}
  \right\}
  \label{e4-mos}
  \end{equation}

Весовые функции $w_1(m,n)$ и~$w_2(m,n)$ будем вычислять по формулам: 
\begin{align*}
w_1(m,n) &=\begin{cases}
1\,, & m\leq \overline{M}\,;\\
\fr{1}{m-\overline{M}+1}\,,& m>\overline{M}\,;
\end{cases}\\
w_2(m,n) &=\begin{cases}
1\,, & n\leq \overline{N}\,;\\
\fr{1}{n-\overline{N}+1}\,, & n>\overline{N}\,.
\end{cases}
\end{align*}
  
  Такой выбор весовых функций обеспечивает max-min-спра\-вед\-ли\-вое 
распределение ресурсов для пользователей до тех пор, пока их количество 
в~соответствующих слайсах не превышает зарезервированное, 
и~<<штрафует>>  
сре\-зы-на\-ру\-ши\-те\-ли уменьшением их веса. 
  
  Таким образом, стационарная точка задачи оптимизации имеет 
координаты
  \begin{equation}
  b(m,n)=\fr{w_1C}{w_1m+w_2n}\,;\enskip
  d(m,n)=\fr{w_2C}{w_1m+w_2n}
  \label{e5-mos}
  \end{equation}
и расположена на пересечении прямых $C\hm= mb\hm+ nd$ 
и~$w_1/b(m,n)\hm= w_2/d(m,n)$.
  
\begin{figure*}[b] %fig3
\vspace*{6pt}
    \begin{center}  
  \mbox{%
 \epsfxsize=162.998mm 
\epsfbox{mos-3.eps}
 }
\end{center}
\vspace*{-12pt}
\Caption{Разделение ресурсов между заявками: (\textit{а})~$X(t)\hm=(4,4)$;
(\textit{б})~$X(t)\hm=(5,1)$; (\textit{в})~$X(t)\hm= (9,5)$}
%\end{figure*}
%\begin{figure*} %fig4
\vspace*{15pt}
    \begin{center}  
  \mbox{%
 \epsfxsize=163mm 
 \epsfbox{mos-4.eps}
 }
\end{center}
\vspace*{-12pt}
\Caption{Вероятности блокировки~$B_s$~(\textit{а}) и~потери при 
поступлении~$B_s^{\mathrm{arr}}$ (залитые значки) и~прерывания обслуживания при 
вытеснении~$B_s^{\mathrm{pr}}$~(пустые значки)~(\textit{б}), $s\hm=1$ (сплошные кривые) 
и~2 (штриховые кривые)}
\end{figure*}
  
  Решение~(\ref{e5-mos}) задачи оптимизации~(\ref{e4-mos}) обеспечивает 
одновременно как изоляцию слайсов  
с~по\-мощью па\-ра\-мет\-ров~$\overline{M}$ и~$\overline{N}$, так и 
эластичность нарезки для эффективного использования ресурса 
системы~\cite{3-mos}.
  
  \section{Пример численного анализа}  %4
  
  Проиллюстрируем зависимость метрик, характеризующих 
производительность системы, от пара\-метров изоляции при нарезке сети. 
Предполагаем, что два оператора делят между собой 50~Мбит/с 
($C\hm=50$) согласно решению задачи оптимизации~(\ref{e4-mos}). 
Минимальные скорости передачи данных равны 5~Мбит/с 
($b_{\min}\hm=5$) и 1~Мбит/с ($d_{\min}\hm=1$),\linebreak
\vspace*{-12pt}
\pagebreak

\noindent
 максимальные~--- 
8~Мбит/с ($b_{\max}\hm=8$) и~50~Мбит/с ($d_{\max}\hm=50$), эти 
диапазоны показаны на рис.~3 тем\-но-се\-рым цветом. Параметры 
изоляции: $\overline{M}\hm=5$ и~$\overline{N}\hm=25$ заявок, 
интенсивности поступления: $\lambda_1\hm=1/150$ и~$\lambda_2\hm=1/120$, 
интенсивности обслуживания: $\mu_1\hm= 1/1920$ и~$\mu_2\hm=1/4000$, 
средние размеры файлов: 1,2~ГБ и~500~МБ для услуг 1-го и 2-го операторов 
соответственно. Указанные значения параметров сис\-те\-мы близ\-ки 
к~реальным и~соответствуют услуге буферизуемого потокового видео 
в~высоком разрешении для 1-го оператора, и услуге загрузки файлов, 
например
%  \begin{align*}
%&  \mathbb{B}_1^{\mathrm{arr}}=\left\{\! (m,n)\in\mathbb{X}: (m+1,n)\not=\mathbb{X}, 
%m\geq \overline{M}\right\}\!;\\
%&  \mathbb{B}_2^{\mathrm{arr}}=\left\{\! (m,n)\in\mathbb{X}: (m,n+1)\not=\mathbb{X}, 
%n\geq \overline{N}\right\}\!;\\
%&  \mathbb{B}_1^{\mathrm{pr}} = \left\{\! (m,n)\in \mathbb{X}: (m+1,n)\not\in \mathbb{X}, 
%m>\overline{M}, n<\overline{N}\right\}\!;\\
%&  \mathbb{B}_2^{\mathrm{pr}} = \left\{\! (m,n)\in \mathbb{X}: (m,n+1)\not\in \mathbb{X}, 
%m<\overline{M}, n>\overline{N}\right\}
%  \end{align*}
при обновлении программного обеспечения, для 2-го оператора.
  
  На рис.~3 штриховыми линиями показано решение~(\ref{e5-mos}) задачи 
оптимизации~(\ref{e4-mos}) для  
max-min-спра\-вед\-ли\-во\-го разделения ресурсов: на рис.~3,\,\textit{а} всем 
пользователям обоих операторов выделен одинаковый ресурс~6,2~Мбит/с; 
на рис.~3,\,\textit{б} все 5~пользователей 1-го оператора получили по 
$b_{\max}\hm=8$~Мбит/с, а~единственный пользователь 2-го оператора~--- 
оставшиеся~10~Мбит/с; на рис.~3,\,\textit{в} все пользователи обоих 
операторов получили минимальный требуемый ресурс.


\begin{figure*} %fig5
\vspace*{1pt}
    \begin{center}  
  \mbox{%
 \epsfxsize=160.393mm 
 \epsfbox{mos-5.eps}
 }
\end{center}
\vspace*{-11pt}
\Caption{Вероятность нарушения $V_s$~(\textit{а}) и среднее время 
обслуживания~$S_s$~(\textit{б}), $s\hm=1$ (сплошные кривые) и~2 (штриховые кривые)}
\end{figure*}
  
  На рис.~4 показаны графики вероятностей блокировки, потери при 
поступлении и~прерывания обслуживания заявки при вытеснении для 
обоих операторов. Вероятность блокировки заявки вы\-чис\-ля\-ет\-ся по 
формуле: $B_s\hm= B_s^{\mathrm{arr}}\hm+ B_s^{\mathrm{pr}}$, $s\hm=1,2$.
%o
 Сравнение рис.~4,\,\textit{а} и рис.~4,\,\textit{б} показывает, что потери 
при поступлении вносят основной вклад в~блокировку заявки, которая 
соответствует отказу\linebreak пользователю в получении услуги. При этом с~рос\-том 
параметра изоляции первого слайса, т.\,е.\ гарантированного 
числа~$\overline{M}$ принимаемых заявок \mbox{1-го} потока, вероятность 
прерывания обслуживания заявки 1-го потока~$B_1^{\mathrm{pr}}$ падает, 
а~вероятность потери заявки 2-го потока при поступлении~$B_2^{\mathrm{arr}}$ растет 
попарно симметрично. Переломным значением оказывается 
$\overline{M}\hm=5$, при котором ресурса системы становится 
недостаточно, чтобы удовлетворять одновременно гарантиям 1-го и 2-го 
слайса, т.\,е.\ $\overline{M} b_{\min} \hm+ \overline{N} d_{\min}\hm>C$.
  
  На рис.~5,\,\textit{а} показано, что существенное влияние параметр 
изоляции оказывает на вероятность~$V_s$ пребывания слайса в состоянии 
нарушителя ($m\hm> \overline{M}$ для первого слайса, $n\hm> \overline{N}$ 
для второго). С~ростом параметра изоляции~$\overline{M}$ 
вероятность~$V_1$ для 1-го слайса стремится к нулю, а вероятность~$V_2$ 
для 2-го слайса меняется незначительно, что свидетельствует об 
обеспечении изоляции. С~увеличением гарантии для 1-го слайса до 
значения $\overline{M}\hm=5$ среднее время~$S_1$ обслуживания заявок 
1-го потока (рис.~5,\,\textit{б}) падает, после чего стабилизируется, поскольку 
при переходе через значение $\overline{M}\hm=5$ параметр изоляции 
перестает оказывать влияние на вероятность приема заявок, 
а~следовательно, и на среднее время обслуживания. Противоположный 
характер поведения кривой наблюдается для второго слайса. 
  
  Таким образом, можно сделать вывод, что изменение параметра 
изоляции оказывает существенное влияние на характеристики 
про\-из\-во\-ди\-тель\-ности сис\-те\-мы, при этом построенная модель способна 
обеспечить изоляцию слайсов.

\vspace*{-10pt}
  
  \section{Заключение}
  
  Построенная в работе модель разделения ресурсов при нарезке 
радиоресурса сети с обеспечением изоляции слайсов и разработанный 
алгоритм разделения ресурсов, включающий решение задачи оптимизации, 
позволяют справедливо разделить ресурсы сети для их эффективного 
использования\linebreak двумя виртуальными сетевыми операторами, арендующими 
радиоресурс у~мобильного оператора.\linebreak Заметим, что замена  
max-min-стра\-те\-гии справедливости на стратегию пропорциональной 
справедливости или более общий случай  
$\alpha$-спра\-вед\-ли\-вости повлияет лишь на вид целевой функции 
в~задаче оптимизации. Модель может быть применена при поиске 
комбинации па\-ра\-мет\-ров изоляции, удерживающих сис\-те\-му в~целевой  
об\-ласти значений мет\-рик ее производительности. Целью дальнейших 
исследований модели разделения ресурсов при нарезке сети может стать 
развитие построенной модели на случай отсутствия минимальных 
требований к~ресурсам, что позволит исследовать элас\-тич\-ный трафик, 
а~также расширение модели до произвольного чис\-ла виртуальных сетевых 
операторов.
{\looseness=1

}
  
  \bigskip
  
  Авторы благодарят Н.\,В.~Яркину и Е.\,Ю.~Лисовскую за полезные 
обсуждения в ходе работы над статьей.
  
{\small\frenchspacing
 {%\baselineskip=10.8pt
 %\addcontentsline{toc}{section}{References}
 \begin{thebibliography}{99}
\vspace*{3pt}

\bibitem{1-mos}
NGMN 5G White Paper. 2015. {\sf http://www.ngmn.de/\linebreak 5gwhite-paper.html}.

\bibitem{2-mos}
5G: System Architecture for the 5G System (Release~15). Version 15.2.0. ETSI 3GPP TS 23.501, 2018. 
{\sf https://\linebreak
www.etsi.org/deliver/etsi\_ts/123500\_123599/123501/ 15.02.00\_60/ts\_123501v150200p.pdf}. 

\bibitem{4-mos} %3
\Au{Jain R., Chiu D.-M., Hawe~W.\,R.} A quantitative measure of fairness and discrimination 
for resource allocation in shared computer system~// arXiv.org, 1998. 
\mbox{arXiv}: cs/9809099 [cs.NI].

\bibitem{3-mos} %4
Network slicing~--- use case requirements.~--- \mbox{GSMA},\linebreak 2018. {\sf  
https://www.gsma.com/futurenetworks/wp-content/uploads/2018/07/Network-Slicing-Use-Case-Requirements-fixed.pdf}.

\bibitem{5-mos}
\Au{Richart M., Baliosian~J., Serrat~J., Gorricho~J.-L.} Resource slicing in virtual wireless 
networks: A~survey~// IEEE~T. Netw. Serv. Man., 2016. Vol.~13. Iss.~3.  
P.~462--476.
\bibitem{6-mos}
\Au{Vila I., P$\acute{\mbox{e}}$rez-Romero J., Sallent~O., Umbert~A.} Characterisation 
of radio access network slicing scenarios with 5G\linebreak
\vspace*{-12pt}
\pagebreak

\noindent
QoS provisioning~// IEEE Access, 2020. 
Vol.~8. P.~51414--51430. doi: 10.1109/access.2020.2980685.
\bibitem{7-mos}
\Au{Marabissi D., Fantacci~R.} Highly flexible RAN slicing approach to manage isolation, 
priority, efficiency~// IEEE Access, 2019. Vol.~7. P.~97130--97142. doi: 
10.1109/ \mbox{access}.2019.2929732.

\bibitem{8-mos}
\Au{Lieto A., Malanchini~I., Capone~A.} Enabling dynamic resource sharing for slice 
customization in 5G networks~// Conference and Exhibition on Global Telecommunications 
Proceedings.~--- IEEE, 2018. P. 1--7. doi: 
10.1109/\mbox{GLOCOM}.2018.8647249. 
\bibitem{9-mos}
\Au{Агеев К.\,А., Сопин Э.\,С., Яркина~Н.\,В., Самуйлов~К.\,Е., Шоргин~С.\,Я.} 
Анализ механизмов нарезки сети с учетом гарантий для различных типов 
трафика~// Информатика и~её применения, 2020. Т.~14. Вып.~3. С.~94--100.
\bibitem{10-mos}
\Au{Yarkina N., Gaidamaka Y., Correia~L.\,M., Samouylov~K.} An analytical model for 
5G network resource sharing with flexible SLA-oriented slice isolation~// Mathematics, 2020. 
Vol.~8. Iss.~7. Art. No.\,1177.
\bibitem{11-mos}
\Au{Kelly F.} Charging and rate control for elastic traffic~// Eur.~T. Telecommun., 1997. 
Vol.~8. P.~33--37.
\end{thebibliography}

 }
 }

\end{multicols}

\vspace*{-5pt}

\hfill{\small\textit{Поступила в~редакцию 11.10.20}}

\vspace*{8pt}

%\pagebreak

%\newpage

%\vspace*{-28pt}

\hrule

\vspace*{2pt}

\hrule

%\vspace*{-2pt}

\def\tit{IMPACT OF THE ISOLATION PARAMETERS ON~RESOURCE ALLOCATION 
IN~THE~NETWORK SLICING MODEL}
                  
\def\titkol{Impact of the isolation parameters on resource allocation in the 
network slicing model}


\def\aut{F.\,A.~Moskaleva$^1$, Yu.\,V.~Gaidamaka$^{1,2}$, and~V.\,S.~Shorgin$^2$}

\def\autkol{F.\,A.~Moskaleva, Yu.\,V.~Gaidamaka, and~V.\,S.~Shorgin}

\titel{\tit}{\aut}{\autkol}{\titkol}

\vspace*{-9pt}


\noindent
$^1$Peoples' Friendship University of Russia (RUDN University), 6~Miklukho-Maklaya Str., Moscow 
117198, Russian\linebreak
$\hphantom{^1}$Federation


\noindent
$^2$Institute of Informatics Problems, Federal Research Center ``Computer Science and Control'' of the 
Russian\linebreak
$\hphantom{^1}$Academy of Sciences, 44-2 Vavilov Str., Moscow 119333, Russian Federation


\def\leftfootline{\small{\textbf{\thepage}
\hfill INFORMATIKA I EE PRIMENENIYA~--- INFORMATICS AND
APPLICATIONS\ \ \ 2020\ \ \ volume~14\ \ \ issue\ 4}
}%
 \def\rightfootline{\small{INFORMATIKA I EE PRIMENENIYA~---
INFORMATICS AND APPLICATIONS\ \ \ 2020\ \ \ volume~14\ \ \ issue\ 4
\hfill \textbf{\thepage}}}

\vspace*{6pt} 


\Abste{Network slicing technology is defined as one of the main components of the fifth generation of 
mobile communications, capable of solving the problem of the colossal growth of data traffic in cellular 
networks. A~key feature of slicing that limits the impact of one slice on another is to provide isolated 
performance guarantees to deliver high quality of service. In this article, a model of resource allocation 
during slicing is developed using the queuing theory. The main task of the work is to determine how 
network resources should be fairly shared between two slices in the system. The resource allocation 
problem is formulated as an optimization problem. For the constructed model, a numerical analysis was 
carried out showing the significant effect of the isolation parameters on the performance characteristics 
of the system.}

\KWE{network slicing; fairness resource allocation; isolation of slices; isolation parameter}


\DOI{10.14357/19922264200402} 

\vspace*{-12pt}

\Ack
\noindent
The paper has been supported by the RUDN University Strategic Academic Leadership Program and 
funded by the Russian Foundation for Basic Research according to the research projects 
No.\,19-07-00933 and No.\,20-07-01064.
\vspace*{3pt}

  \begin{multicols}{2}

\renewcommand{\bibname}{\protect\rmfamily References}
%\renewcommand{\bibname}{\large\protect\rm References}

{\small\frenchspacing
 {%\baselineskip=10.8pt
 \addcontentsline{toc}{section}{References}
 \begin{thebibliography}{99}

\bibitem{1-mos-1}
NGMN 5G white paper. 2015. Available at: {\sf http:// www.ngmn.de/5gwhite-paper.html} (accessed 
Octo-\linebreak ber~22, 2020).
\bibitem{2-mos-1}
TS 23.501. 2018. Technical Specification: System architecture for the 5G system (Release~15).
Version 15.2.0.\linebreak Available at: {\sf 
https://www.etsi.org/deliver/etsi\_ts/\linebreak
 123500\_123599/123501/15.02.00\_60/ts\_123501v
 150200p.pdf} 
(accessed October~22, 2020).

\bibitem{4-mos-1}
\Aue{Jain, R., D.-M. Chiu, and W.\,R.~Hawe.} 1998. A~quantitative measure of fairness and 
discrimination for resource allocation in shared computer system. arXiv:cs/9809099 [cs.NI]. Available 
at: {\sf https://arxiv.org/abs/cs/9809099} (accessed October~22, 2020).


\bibitem{3-mos-1} %4
GSMA. 2018. Network slicing~--- use case requirements.  Available at: {\sf  
https://www.gsma.com/futurenetworks/
wp-content/uploads/2018/07/Network-Slicing-Use-Case-Requirements-fixed.pdf}
(accessed October~22, 2020).

\bibitem{5-mos-1}
\Aue{Richart, M., J. Baliosian, J.~Serrat, and J.-L.~Gorricho.} 2016. Resource slicing in virtual 
wireless networks: A~survey. \textit{IEEE~T. Netw. Serv. Man.} 13(3):462--476.
\bibitem{6-mos-1}
\Aue{Vila, I., J.~P$\acute{\mbox{e}}$rez-Romero, O.~Sallent, and A.~Umbert.} 2020. 
Characterisation of radio access network slicing scenarios with 5G QoS provisioning. \textit{IEEE Access} 
8:51414--51430. doi: 10.1109/\mbox{access}.2020.2980685.
\bibitem{7-mos-1}
\Aue{Marabissi, D., and R.~Fantacci.} 2019. Highly flexible RAN slicing approach to manage 
isolation, priority, efficiency. \textit{IEEE Access} 7:97130--97142. doi: 
10.1109/\linebreak access.2019.2929732.
\bibitem{8-mos-1}
\Aue{Lieto, A., I. Malanchini, and A.~Capone.} 2018. Enabling dynamic resource sharing for slice 
customization in 5G\linebreak networks. \textit{Conference and Exhibition on Global Tele\-communications 
Proceedings}. IEEE. 1--7. doi: 10.1109/\linebreak GLOCOM.2018.8647249.
\bibitem{9-mos-1}
\Aue{Ageev, K.\,A., E.\,S.~Sopin, N.\,V.~Yarkina, K.\,E.~Samuylov, and S.\,Ya.~Shorgin.} 2020. 
Analiz mekhanizmov narezki seti s~uchetom garantiy dlya razlichnykh tipov trafika [Analysis of network 
slicing mechanisms with guarantees for various types of traffic]. \textit{Informatika i~ee 
Primeneniya~--- Inform. Appl.} 14(3):94--100.
\bibitem{10-mos-1}
\Aue{Yarkina, N., Yu. Gaidamaka, L.\,M.~Correia, and K.~Samouylov.} 2020. An analytical model 
for 5G network resource sharing with flexible SLA-oriented slice isolation. \textit{Mathematics} 
8(7):1177.
\bibitem{11-mos-1}
\Aue{Kelly, F.} 1997. Charging and rate control for elastic traffic. \textit{Eur.~T. Telecommun.} 
8:33--37.
\end{thebibliography}

 }
 }

\end{multicols}

\vspace*{-3pt}

\hfill{\small\textit{Received October 11, 2020}}

%\pagebreak

%\vspace*{-24pt}


\Contr

\noindent
\textbf{Moskaleva Faina A.} (b.\ 1996)~--- PhD student, Department of Applied Probability and 
Informatics, Peoples' Friendship University of Russia (RUDN University), 6~Miklukho-Maklaya Str., 
Moscow 117198, Russian Federation; \mbox{moskaleva-fa@rudn.ru}

\vspace*{3pt}

\noindent
\textbf{Gaidamaka Yuliya V.} (b.\ 1971)~--- Doctor of Science in physics and mathematics, 
professor, Department of Applied Probability and Informatics, Peoples' Friendship University of Russia 
(RUDN University), 6~Miklukho-Maklaya Str., Moscow 117198, Russian Federation; senior scientist, 
Institute of Informatics Problems, Federal Research Center ``Computer Science and Control'' of the 
Russian Academy of Sciences, 44-2~Vavilov Str., Moscow 119333, Russian Federation;  
\mbox{gaydamaka-yuv@rudn.ru}

\vspace*{3pt}

\noindent
\textbf{Shorgin Vsevolod S.} (b.\ 1978)~--- Candidate of Science (PhD) in technology, senior 
scientist, Institute of Informatics Problems, Federal Research Center ``Computer Science and 
Control'' of the Russian Academy of Sciences, 44-2~Vavilov Str., Moscow 119333, Russian 
Federation; \mbox{vshorgin@ipiran.ru}
\label{end\stat}

\renewcommand{\bibname}{\protect\rm Литература}          %02
\def\stat{kochetkova}

\def\tit{ВЕРОЯТНОСТНАЯ МОДЕЛЬ ЗАТУХАНИЯ МОЩНОСТИ СИГНАЛА В СЦЕНАРИЯХ 3GPP TR 38.901 
РАЗВЕРТЫВАНИЯ СЕТИ 5G$^*$}

\def\titkol{Вероятностная модель затухания мощности сигнала в~сценариях 3GPP TR 38.901 
развертывания сети 5G}

\def\aut{Е.\,Д.~Макеева$^1$, И.\,А.~Кочеткова$^2$, С.\,Я.~Шоргин$^3$}

\def\autkol{Е.\,Д.~Макеева, И.\,А.~Кочеткова, С.\,Я.~Шоргин}

\titel{\tit}{\aut}{\autkol}{\titkol}

\index{Макеева Е.\,Д.}
\index{Кочеткова И.\,А.}
\index{Шоргин С.\,Я.}
\index{Makeeva E.\,D.}
\index{Kochetkova I.\,A.}
\index{Shorgin S.\,Ya.}


{\renewcommand{\thefootnote}{\fnsymbol{footnote}} \footnotetext[1]
{Публикация выполнена в~рамках проекта №\,025319-2-000 Системы грантовой 
поддержки научных проектов РУДН.}}


\renewcommand{\thefootnote}{\arabic{footnote}}
\footnotetext[1]{Российский университет дружбы народов имени Патриса Лумумбы; 
Институт проблем управления имени В.\,А.~Трапезникова Российской академии наук, 
\mbox{elena-makeeva-96@mail.ru}}
\footnotetext[2]{Российский университет дружбы народов имени Патриса Лумумбы; 
Федеральный исследовательский центр <<Информатика и~управ\-ле\-ние>> Российской 
академии наук, \mbox{kochetkova-ia@rudn.ru}}
\footnotetext[3]{Федеральный исследовательский центр <<Информатика 
и~управ\-ле\-ние>> Российской академии наук, \mbox{sshorgin@ipiran.ru}}

\vspace*{2pt}






\Abst{Сети пятого (5G) и~последующих поколений будут использовать терагерцевый диапазон 
радиочастот, что обеспечит сверхвысокую скорость передачи данных. Однако при 
этом возможны потери сигнала при прохождении через препятствия. Поэтому 
становится крайне важным моделирование распространения сигнала с~по\-мощью 
стохастической геометрии и~использование актуальных моделей затухания сигнала. 
Модели для описания затухания сигнала для различных сценариев развертывания сети 
5G в~виде эмпирических формул содержатся в~спецификации 3GPP TR 38.901. Тем не 
менее обычно для построения моделей стохастической геометрии используются 
упрощенные виды формул. В~\mbox{статье} представлена функция распределения (ФР) затухания 
мощности сигнала при случайном расположении пользователей в~соответствии со 
сценариями, описанными в~3GPP TR 38.901. На численных примерах показано, что 
разница значений с~упрощенной формулой значительна и~может привести к~занижению 
оценки пропускной способности сети.}

\KW{беспроводная сеть; 5G; 3GPP TR 38.901; мощ\-ность затухания сигнала; прямая 
видимость; непрямая ви\-ди\-мость; сто\-ха\-сти\-че\-ская гео\-метрия}

\DOI{10.14357/19922264240204}{EKLCAP}
  
%\vspace*{-6pt}


\vskip 10pt plus 9pt minus 6pt

\thispagestyle{headings}

\begin{multicols}{2}

\label{st\stat}



\section{Введение}

Сети пятого и~последующих поколений будут использовать терагерцевый 
диапазон радиочастот, чтобы обеспечить сверхвысокую скорость передачи данных и~пропускную способность. Однако использование миллиметровых волн связано со 
сложностями из-за потери сигнала при про\-хож\-де\-нии препятствий. Таким образом, для 
обеспечения производительности сетей~5G становится крайне важным моделирование 
распространения сигнала. Формула Шен\-но\-на--Харт\-ли с~формулой Фрииса задают 
пропускную способность канала
$$
C=B \log_2 \left(1+\fr{P_t G_t G_r}{(N+I) \mathrm{PL}}\right),
$$
 где $B$~--- полоса 
пропускания канала; $P_t$~--- мощ\-ность передающей антенны; $G_t$~--- коэффициент 
усиления передающей антенны; $G_r$~--- коэффициент усиления приемной антенны; $N$~--- мощ\-ность шума; 
$I$~--- мощ\-ность интерференции; $\mathrm{PL}$~--- мощ\-ность затухания 
сигнала (path loss, PL) на расстоянии от передающей антенны до приемной 
антенны~\cite{Moltchanov2022a}.
Пропускная способность канала уже далее используется в~управ\-ле\-нии занятием 
радиоресурсов базовой станции (БС) для соблюдения необходимого качества обслуживания 
пользователей по требуемой ско\-рости передачи данных.

 Ввиду того что пользователи находятся на разных расстояниях от БС, 
значения мощностей затухания сигнала будут случайными. Как показано в~работе~\cite{Hmamouche2021}, для учета влияния на пропускную\linebreak
 способность канала 
случайного положения пользователей в~соте применяется стохастическая гео\-мет\-рия. 
Рассмотреть совместное занятие радиоресурсов и~случайный характер поведения 
пользователей позволяет модель на основе аппарата \mbox{ресурсных} сис\-тем массового 
обслуживания~\cite{Naumov2016, Gorbunova2018}. Такие модели применяются для 
исследования различных сценариев развертывания сетей 
5G~\cite{Moltchanov2022b, Markova2019}, например при анализе совместного 
обслуживания трафика со сверхнизкой задержкой и~широкополосного трафика~\cite{Kochetkova2021}.

\begin{figure*}[b] %fig1
\vspace*{-6pt}
      \begin{center}
     \mbox{%
\epsfxsize=124.62mm 
\epsfbox{koc-1.eps}
}
\end{center}
\vspace*{-9pt}
\Caption{Схема системной модели}
\label{fig1}
\end{figure*}

Модели для описания мощности $\mathrm{PL}$ затухания сигнала для разных сценариев 
отражены в~спецификации 3GPP TR 38.901~\cite{3GPP38901}. И~если при проведении 
имитационного моделирования исследователи по большей части полностью реализуют 
эти модели~\cite{Bolla2023}, то при построении моделей стохастической гео\-мет\-рии 
зачастую применяется упрощенный вид формул.
В~обзоре~\cite{Hmamouche2021} рассмотрены различные виды функциональной 
зависимости затухания мощ\-ности сигнала от расстояния между пользователем 
и~БС, которые применяют исследователи. Например, для простоты 
расчетов в~работе~\cite{Moltchanov2022b} используются упрощенные формулы без 
учета ку\-соч\-но-за\-дан\-но\-го вида функции для прямой видимости и~максимума нескольких 
величин мощностей PL для непрямой видимости при по\-стро\-ении~ФР.

В данной статье получена ФР затухания мощ\-ности сигнала при случайном 
расположении пользователей в~соответствии со сценариями 3GPP TR~38.901 
развертывания сети 5G. Использованы формулы из этой спецификации, где приведены 
зависимости PL от расстояния между пользователем и~БС. В~данной 
статье закон распределения пользователей в~соте взят произвольный, а~для 
численного анализа~--- в~соответствии с~типовыми рекомендованными значениями 
параметров сценариев.



\section{Затухание сигнала как функция от параметров сценария 3GPP} \label{sec2}

При исследовании распространения сигнала необходимо учитывать множество 
параметров сети, таких как частота, основные характеристики местности, высота 
принимающей и~передающей антенн, конфигурация антенн и~другие факторы. Для 
упрощения расчетов мощности PL затухания сигнала стандартом 3GPP TR~38.901~\cite{3GPP38901} 
были выделены основные сценарии развертывания сети~5G: 
мак\-ро\-со\-та в~городе (urban macro, UMa), микросота в~городе (urban micro, UMi), 
мак\-ро\-со\-та в~сельской местности (rural macro, RMa), точка доступа внут\-ри 
помещения (indoor hotspot, InH) и~крытая фабрика (indoor factory, InF),~--- 
и~путем экспериментов были получены эмпирические модели затухания сигнала для них. 
На основе этих моделей и~в~предположении случайного характера поведения 
пользователей в~данном разделе получена ФР мощности затухания сигнала с~учетом 
особенностей, описанных в~данной спецификации.


Рассмотрим общее описание предлагаемых сценариев (рис.~\ref{fig1}). Пусть 
передающая антенна БС расположена на высоте~$h_{\mathrm{BS}}$, 
использует несущую частоту~$f_c$ и~создает покрытие радиуса~$R$. 
Пользовательские устройства (ПУ) находятся на высоте~$h_{\mathrm{UT}}$, а~проекция 
расстоянии от ПУ до БС со\-став\-ля\-ет $d$.

В зависимости от своего расположения ПУ может находиться в~зоне прямой видимости 
(line-of-sight, LOS) с~устойчивым уровнем сигнала или вне этой зоны (non-line-of-sight, NLOS). 
Если ПУ расположено на расстоянии~$d$, то ве\-ро\-ят\-ность того, что 
ПУ находится в~зоне прямой видимости, пред\-став\-ля\-ет собой кусочно-заданную 
функцию:
\begin{equation}
\label{eq1}
{\mathrm{Pr}}_{\mathrm{LOS}}(d)=
\begin{cases}
{\mathrm{Pr}}_1^{\mathrm{LOS}}(d), & 0=r_0 \leq d < r_1; \\
{\mathrm{Pr}}_2^{\mathrm{LOS}}(d), & r_1 \leq d < r_2; \\
\cdots & \cdots \\
{\mathrm{Pr}}_I^{\mathrm{LOS}}(d), & r_{I-1} \leq d \leq r_I=R,
\end{cases}
\end{equation}
где радиусы $R_i$ определяют границы интервалов. 

Тогда мощ\-ность $\mathrm{PL}(d)$ 
затухания сигнала примет вид:
\begin{multline}
\label{eq2}
\mathrm{PL}\,(d)= \mathrm{PL}_{\mathrm{LOS}}(d)  {\mathrm{Pr}}_{\mathrm{LOS}}(d) + {}\\
{}+\mathrm{PL}_{\mathrm{NLOS}}(d)  \left[1-
{\mathrm{Pr}}_{\mathrm{LOS}}(d)\right].
\end{multline}

Мощность затухания сигнала в~условиях прямой видимости LOS описывается ку\-соч\-но-за\-дан\-ной функцией
\begin{multline}
\label{eq3}
\mathrm{PL}^{\mathrm{LOS}}(d)={}\\
{}=
\begin{cases}
\mathrm{PL}_1^{\mathrm{LOS}}(d), & 0=d_0 \leq d < d_1;\\
\mathrm{PL}_2^{\mathrm{LOS}}(d), & d_1 \leq d < d_2;\\
\cdots & \cdots \\
\mathrm{PL}_J^{\mathrm{LOS}}(d), & d_{J-1} \leq d \leq d_J=R,
\end{cases}
\end{multline}
где $d_j$~--- границы интервалов (break point distance), а~в~условиях непрямой 
видимости NLOS пред\-став\-ля\-ет собой максимум
\begin{multline}
\label{eq4}
\mathrm{PL}_{\mathrm{NLOS}}(d) = {}\\
\!\!{}=\!
\max\left(\mathrm{PL}^{\mathrm{LOS}}(d),\mathrm{PL}^{\mathrm{NLOS}}_1(d), \ldots, \mathrm{PL}^{\mathrm{NLOS}}_K(d)\right)\!.\!\!
\end{multline}

Каждая из компонент функций для случаев LOS и~NLOS имеет схожую структуру:
\begin{multline}
 \mathrm{PL}^{l}_m(d)[\mathrm{dB}] = \alpha_m^{l}[\mathrm{dB}]+\beta_m^{l}[\mathrm{dB}]\log_{10}{D(d)},
 \\
 \mathrm{PL}^{l}_m(d) = \alpha_m^{l} \cdot D^{\beta_m^{l}}(d),
\\
 l=\begin{cases}
 \mbox{``}\mathrm{LOS}\mbox{''}, & m=j=\overline{0,J}\,; \\
 \mbox{``}\mathrm{NLOS}\mbox{''},& m=k=\overline{0,K}\,,
 \end{cases}
\label{eq5}
\end{multline}
где $D(d)=\sqrt{d^2+(h_{\mathrm{BS}}\hm-h_{\mathrm{UT}})^2}$~--- расстояние от ПУ до БС в~трехмерном 
пространстве; $\alpha$ и~$\beta$~--- коэффициенты модели затухания сигнала~--- 
константы для каждого из сценариев 3GPP TR~38.901.



\section{Функция распределения затухания сигнала при~случайном расположении 
пользователей} \label{sec3}

Примем теперь, что расстояние между ПУ и~БС~--- случайная величина (СВ)~$\xi_d$ 
со значениями~$d$ и~ФР~$F_{\xi_d}(x)$. Тогда расстояние от ПУ до БС в~трехмерном 
пространстве~$\xi_D$ будет функцией от СВ~$\xi_d$ с~ФР

\noindent
\begin{multline*}
F_{\xi_D}(x)  =
\mathrm{Pr}\,(\xi_D \leq x) ={}\\
{}=
\mathrm{Pr}\left(\sqrt{\xi_d^2+(h_{\mathrm{BS}}- h_{\mathrm{UT}})^2} \leq x \right) ={} \\
{} = \mathrm{Pr}\left(\xi_d \leq \sqrt{x^2-(h_{\mathrm{BS}}- h_{\mathrm{UT}})^2} \right) ={}\\
{}=
F_{\xi_d}\left(\sqrt{x^2-(h_{\mathrm{BS}}- h_{\mathrm{UT}})^2} \right).
\end{multline*}

Случайная величина $\xi_m^l$~--- компонента функции затухания сигнала~--- зависит от СВ $\xi_D$ и~по формуле~(\ref{eq5}) имеет ФР
\begin{multline*}
F_{\xi_m^l}(x)  =
\mathrm{Pr}\,(\xi_m^l \leq x) =
\mathrm{Pr}\left(\alpha_m^{l}  ({\xi}_D)^{\beta_m^{l}} \leq x \right) = {}\\
{}=
\mathrm{Pr}\left(\xi_D \leq \left(\fr{x}{\alpha_m^{l}}\right)^{{1}/{\beta_m^{l}}} \right) = 
F_{\xi_D}\left( \left( \fr{x}{\alpha_m^{l}}\right)^{{1}/{\beta_m^{l}}} 
\right) ={}\\
{}=  F_{\xi_d}\left( \sqrt{ 
\left(\fr{x}{\alpha_m^{l}}\right)^{{2}/{\beta_m^{l}}} - \left(h_\mathrm{BS}-h_\mathrm{UT}\right)^2 }\right), \\
 l=\begin{cases}
 \mbox{``}\mathrm{LOS}\mbox{''}, &  m=j=\overline{0,J}\,; \\
 \mbox{``}\mathrm{NLOS}\mbox{''}, & m=k=\overline{0,K}\,.
 \end{cases}
\end{multline*}

Для затухания сигнала в~условиях прямой видимости ФР СВ~$\xi_{\mathrm{LOS}}$ по 
формуле~(\ref{eq3}) примет вид:
\begin{multline}
F_{\xi_{\mathrm{LOS}}}(x) =
\mathrm{Pr}\,(\xi_{\mathrm{LOS}} \leq x) = {} \\
{}= \sum\limits_{j=1}^J  \mathrm{Pr}\left(\xi_{\mathrm{LOS}} \leq x \mid d_{j-1} \leq \xi_d < d_j\right) \times{}\\
{}\times \mathrm{Pr}\left(d_{j-1} \leq  \xi_d < d_j\right) = {}\\
{} = \sum\limits_{j=1}^J F_{\xi_j^{\mathrm{LOS}}}(x) \left[ F_{\xi_d}(d_j) - 
F_{\xi_d}(d_{j-1}) \right], 
\label{eq6}
\end{multline}
а для непрямой видимости ФР СВ $\xi_{\mathrm{NLOS}}$ по формуле~(\ref{eq4}) 
и~с~учетом~\cite{Ventzel2018} запишем как
\begin{multline}
F_{\xi_{\mathrm{NLOS}}}(x)  =
\mathrm{Pr}\,(\xi_{\mathrm{NLOS}} \leq x) ={}\\
{}=
\mathrm{Pr}\left(\max{\left(\xi_{\mathrm{LOS}}, \; \xi^{\mathrm{NLOS}}_1, \ldots, \xi^{\mathrm{NLOS}}_K \right)} 
\leq x \right) = {} \\
{} = \mathrm{Pr}\left(\xi_{\mathrm{LOS}} \leq x, \; \xi^{\mathrm{NLOS}}_1 \leq x, \ldots, \xi^{\mathrm{NLOS}}_K 
\leq x \right) = {}\\
{} = \mathrm{Pr}\,(\xi_{\mathrm{LOS}} \leq x)  \prod\limits_{k=1}^K{\mathrm{Pr}\left(\xi^{\mathrm{NLOS}}_k \leq x 
\right)} = {}\\
{}=
F_{\xi_{\mathrm{LOS}}}(x)  \prod\limits_{k=1}^K F_{\xi_k^{\mathrm{NLOS}}}(x). 
\label{eq7}
\end{multline}

Наконец, ФР СВ $\xi_{\mathrm{PL}}$ затухания сигнала по формуле~(\ref{eq2}) запишем 
следующим образом:
\begin{multline*}
F_{\xi_{\mathrm{PL}}}(x) =
\mathrm{Pr}\,(\xi_{\mathrm{PL}} \leq x) ={}\\
{}=
\mathrm{Pr}\left(\xi_{\mathrm{LOS}}  \xi_{\mathrm{Pr}_{\mathrm{LOS}}} + \xi_{\mathrm{NLOS}} \left[1-
\xi_{\mathrm{Pr}_{\mathrm{LOS}}}\right] \leq x \right),
\end{multline*}
где $\xi_{\mathrm{Pr}_{\mathrm{LOS}}}$~--- СВ вероятности расположения ПУ в~зоне прямой 
видимости~(\ref{eq1}).
Функция распределения $F_{\xi_{\mathrm{PL}}}(x)$ будет приближенно представлять собой свертку.



\section{Численный анализ}


\begin{figure*}[b]\small
\begin{center}
\tabcolsep=4pt
\begin{tabular}{|c|l|c|c|}

\multicolumn{4}{c}{Коэффициенты модели затухания сигнала для UMa и~UMi}\\
\multicolumn{4}{c}{\ }\\[-6pt]
\hline
 Зона& \multicolumn{1}{c|}{[dB]} & UMa & UMi \\
\hline
&&&\\[-9pt]
 & $\alpha_1^{\mathrm{LOS}}$ & $28+20 \log_{10}{f_c}$ & $32{,}4+20\log_{10}{f_c}$\\
% \cline{2-4}
 LOS& $\beta_1^{\mathrm{LOS}}$ & $22$ & $21$\\
% \cline{2-4}
 & $\alpha_2^{\mathrm{LOS}}$ & $28+20 \log_{10}{f_c}-9\log_{10}{\left(d_1^2+(h_{\mathrm{BS}}-h_{\mathrm{UT}})^2\right)} $ 
 & $32{,}4+20\log_{10}{f_c}- 9{,}5\log_{10}{\left(d_1^2+(h_{\mathrm{BS}}-h_{\mathrm{UT}})^2\right)}$\\
% \cline{2-4}
 & $\beta_2^{\mathrm{LOS}}$ & $40$ & $40$\\
\hline
&&&\\[-9pt]
& $\alpha_1^{\mathrm{NLOS}}$ & $13{,}54+20\log_{10}{f_c}-0{,}6(h_{\mathrm{UT}}-1,5)$ & 
$22{,}4+21{,}3\log_{10}{f_c} - 0{,}3(h_{\mathrm{UT}}-1{,}5)$ \\
 %\cline{2-4}
NLOS  & $\beta_1^{\mathrm{NLOS}}$ & $39{,}08$ & $35{,}3$\\
 %\cline{2-4}
 & $\alpha_{\mathrm{Opt}}$ & $32{,}4+20\log_{10}{f_c}$ & $32{,}4+20\log_{10}{f_c} $\\
 %\cline{2-4}
 & $\beta_{\mathrm{Opt}}$ & $30$ & $31{,}9$\\
 \hline
\end{tabular}
\end{center}
%\end{table*}
%\begin{figure*}[b] %fig2
\setcounter{figure}{1}
\vspace*{12pt}
      \begin{center}
     \mbox{%
\epsfxsize=163mm 
\epsfbox{koc-2.eps}
}
\end{center}
\vspace*{-10pt}
  \Caption{Функции распределения PL для LOS~(\textit{а}) и ~NLOS~(\textit{б}) для сценариев UMa (черные кривые) и~UMi~(серые кривые):
  \textit{1}~--- расчет по формулам~(6) для LOS и~(7) для NLOS; \textit{2}~--- расчет по упрощенным формулам}
 \label{fig:1}
 \end{figure*}

 В спецификации 3GPP TR 38.901 указаны основные диапазоны значений параметров 
для сценариев развертывания сетей~5G. Рассмотрим сценарии макросоты UMa и~микросоты UMi в~городе со следующим набором исходных данных: радиус действия БС 
$R\hm=5000$~м, центральная частота $f_c\hm=6$~ГГц, высота ПУ
$h_{\mathrm{UT}}\hm=1{,}5$~м, высоты БС для UMa $h_{\mathrm{BS}}\hm=25$ м и~для UMi $h_{\mathrm{BS}}\hm=10$~м. 
Предположим, что пользователи распределены равномерно в~об\-ласти действия БС 
радиусом~$R$.

Согласно упрощенным формулам, подобным тем, что описаны в~работе~\cite{Moltchanov2022b}, ФР мощности PL затухания сигнала в~условиях 
прямой и~непрямой видимости могут быть представлены как 
$$
F_{\xi_\mathrm{LOS}}(x) = 
F_{\xi_1^{\mathrm{LOS}}}(x)\,;
$$

\noindent
\begin{multline*}
F_{\xi_\mathrm{NLOS}}(x)=F_{\mathrm{Opt}}(x)={}\\
{}= F_{\xi_d}\left( \sqrt{ 
\left(\fr {x}{\alpha_{\mathrm{Opt}}}\right)^{{2}/{\beta_{\mathrm{Opt}}}} - 
(h_\mathrm{BS}-h_\mathrm{UT})^2 }\,\right).
\end{multline*}

\noindent
 Коэффициенты модели затухания сигнала $\alpha_m^{l}$ 
[dB] и~$\beta_m^{l}$ [dB] для сценариев UMa и~UMi, согласно~\cite{3GPP38901}, 
представлены в~таб\-лице.


 
 \begin{figure*} %fig3
 \vspace*{1pt}
      \begin{center}
     \mbox{%
\epsfxsize=163.204mm 
\epsfbox{koc-3.eps}
}
\end{center}
\vspace*{-15pt}
 \Caption{Разница значений ФР PL для LOS~(\textit{а}) и~NLOS~(\textit{б}):
 \textit{1}~--- UMa; \textit{2}~--- UMi}
 \label{fig:2}
  \vspace*{-3pt}
 \end{figure*}

 
 

В рамках данного численного анализа покажем графики ФР моделей PL затухания 
сигнала для сценариев UMa и~UMi в~условиях прямой и~непрямой видимости по 
формулам~(\ref{eq6}) и~(\ref{eq7}) и~упрощенным формулам. Графики с~полученными 
результатами представлены на рис.~\ref{fig:1} и~3. Во всех случаях 
график ФР по упрощенным формулам идет выше, что представляет собой 
верхнюю оценку. Однако при дальнейших расчетах с~использованием упрощенных 
формул пропускная спо\-соб\-ность и~максимальное число обслуженных пользователей 
в~соте оказываются занижены. Таким образом, авторы рекомендуют при использовании 
модели затухания сигнала как компоненты, например в~ресурсных сис\-те\-мах массового 
обслуживания для анализа беспроводных сетей, использовать формулы (\ref{eq6}) и~(\ref{eq7}).


\section{Заключение}

В статье исследована модель затухания сигнала по формулам сценариев 3GPP TR 
38.901. Была получена функция распределения (ФР) мощности затухания сигнала при 
случайном (произвольный закон) расположении пользователей в~зоне покрытия 
БС. Она учитывает ку\-соч\-но-за\-дан\-ный вид функции для LOS и~максимум 
нескольких величин для NLOS. Проведен численный анализ для данных из 
спецификации 3GPP TR 38.90 для сценариев макро- и~микросот в~городе для 
сравнения ФР, представленных в~данной работе, и~ФР, полученных с~по\-мощью 
упрощенных формул. Результат анализа показал, что ФР по упрощенным формулам дает 
оценку сверху, что может понижать точность расчетов пропускной способности 
канала. Отметим, что авторы статьи не ставили перед собой задачу аналитического 
сравнения двух ФР, а~хотели бы обратить внимание на несложный вид полученных 
формул, которые рекомендуют для использования как компоненту в~ресурсных 
системах массового обслуживания при моделировании беспроводных сетей 5G/6G. 
Задачей дальнейшего исследования станет разработка ресурсной системы массового 
обслуживания с~учетом случайного расположения пользователей в~соте через 
пред\-став\-лен\-ную ФР для оценки схемы приоритетного обслуживания узкополосного 
трафика и~прерывания обслуживания широкополосного трафика в~се\-ти~5G.


\vspace*{-12pt}

{\small\frenchspacing
 {\baselineskip=10.5pt
 %\addcontentsline{toc}{section}{References}
 \begin{thebibliography}{99}
 
 \vspace*{-2pt}
 
\bibitem{Moltchanov2022a}
\Au{Молчанов~Д.\,А., Бегишев~В.\,О., Самуйлов~К.\,Е., Кучерявый~Е.\,А.}
Сети 5G/6G: архитектура, технологии, методы анализа и~расчета.~--- 
М.: РУДН, 2022. 516~с.

\bibitem{Hmamouche2021}
\textit{Hmamouche~Y., Benjillali~M., Saoudi~S., Yanikomeroglu~H., Renzo~M.\,D.}
New trends in stochastic geometry for wireless networks: A~tutorial and survey~//
P.~IEEE, 2021. Vol.~109. No.\,7. P.~1200--1252.
doi: 10.1109/JPROC.2021.3061778.

\bibitem{Naumov2016}
\Au{Наумов~В.\,А., Самуйлов~К.\,Е.}
О~связи ресурсных сис\-тем массового обслуживания с~сетями Эрланга~//
Информатика и~её применения, 2016. Т.~10. Вып.~3. С.~9--14.
doi: 10.14357/19922264160302.

\bibitem{Gorbunova2018} %4
\Au{Горбунова~А.\,В., Наумов~В.\,А., Гайдамака~Ю.\,В., Самуйлов~К.\,Е.}
Ресурсные системы массового обслуживания как модели беспроводных сис\-тем связи~//
Информатика и~её применения, 2018. Т.~12. Вып.~3. С.~48--55.
doi: 10.14357/19922264180307.

\bibitem{Markova2019} %5
\Au{Маркова~Е.\,В., Гольская~А.\,А., Дзантиев~И.\,Л., Гудкова~И.\,А., 
Шоргин~С.\,Я.}
Сравнительный анализ показателей эффективности модели беспроводной сети 
межмашинного взаимодействия, работающей в~рамках двух политик разделения 
радиоресурсов~//
Информатика и~её применения, 2019. Т.~13. Вып.~1. С.~108--116.
doi: 10.14357/19922264190115.

\bibitem{Moltchanov2022b} %6
\Au{Moltchanov~D.\,A., Sopin~E.\,S., Begishev~V.\,O., Sa\-muylov~A.\,K., Koucheryavy~Y.\,A., 
Sa\-mouylov~K.\,E.}
A~tutorial on mathematical modeling of 5G/6G millimeter wave and terahertz 
cellular systems~//
IEEE Commun. Surv.  Tut., 2022. Vol.~24. No.\,2. P.~1072--1116.
doi: 10.1109/ COMST.2022.3156207.



\bibitem{Kochetkova2021}
\Au{Кочеткова~И.\,А., Кущазли~А.\,И., Харин~П.\,А., Шоргин~С.\,Я.}
Модель схемы приоритетного доступа трафика URLLC и~eMBB в~сети пятого поколения в~виде ресурсной сис\-те\-мы массового обслуживания~//
Информатика и~её применения, 2021. Т.~15. Вып.~4. С.~87--92.
doi: 10.14357/19922264210412.

\bibitem{3GPP38901}
3GPP TR 38.901. Study on channel model for frequencies from~0.5 to~100~GHz, 
2024. Release 17.1.0.

\bibitem{Bolla2023}
\Au{Bolla~R., Bruschi~R., Lombardo~C., Mohammadpour~A., Trivisonno~R., 
Poe~W.\,Y.}
A~5G multi-gNodeB simulator for ultra-reliable 0.5--100~GHz communication in 
indoor Industry~4.0 environments~//
Comput. Netw., 2023. Vol.~237. Art. No.\,110103.
doi: 10.1016/j.comnet.2023.11010.

\bibitem{Ventzel2018}
\textit{Вентцель~Е.\,С., Овчаров~Л.\,А.}
Теория вероятностей и~ее инженерные приложения.~--- М.: Юстиция, 
2018. 480~c.

\end{thebibliography}

 }
 }

\end{multicols}

\vspace*{-10pt}

\hfill{\small\textit{Поступила в~редакцию 15.03.24}}

%\vspace*{10pt}

%\pagebreak

\newpage

\vspace*{-28pt}

%\hrule

%\vspace*{2pt}

%\hrule



\def\tit{STOCHASTIC PATH LOSS MODEL IN~5G~NETWORK DEPLOYMENT SCENARIOS: A~STUDY BASED ON~3GPP~TR~38.901}


\def\titkol{Stochastic Path Loss Model in 5G Network Deployment Scenarios: A~Study Based on 3GPP TR 38.901}


\def\aut{E.\,D.~Makeeva$^{1,2}$, I.\,A.~Kochetkova$^{1,3}$, and~S.\,Ya.~Shorgin$^{3}$}

\def\autkol{E.\,D.~Makeeva, I.\,A.~Kochetkova, and~S.\,Ya.~Shorgin}

\titel{\tit}{\aut}{\autkol}{\titkol}

\vspace*{-8pt}


\noindent
$^1$RUDN University, 6 Miklukho-Maklaya Str., Moscow 117198, Russian Federation

\noindent
$^2$V.\,A.~Trapeznikov Institute of Control Science of the Russian Academy of 
Sciences, 65~Profsoyuznaya Str., Moscow\linebreak
$\hphantom{^1}$117997, Russian Federation

\noindent
$^3$Federal Research Center ``Computer Science and Control'' of the Russian 
Academy of Sciences, 44-2~Vavilov\linebreak
$\hphantom{^1}$Str., Moscow 119333, Russian Federation

\def\leftfootline{\small{\textbf{\thepage}
\hfill INFORMATIKA I EE PRIMENENIYA~--- INFORMATICS AND
APPLICATIONS\ \ \ 2024\ \ \ volume~18\ \ \ issue\ 2}
}%
 \def\rightfootline{\small{INFORMATIKA I EE PRIMENENIYA~---
INFORMATICS AND APPLICATIONS\ \ \ 2024\ \ \ volume~18\ \ \ issue\ 2
\hfill \textbf{\thepage}}}

\vspace*{3pt}




\Abste{The fifth-generation (5G) and beyond networks will utilize radio frequencies in the terahertz 
spectrum, enabling extremely high data transmission rates. However, signal loss may occur when signals 
pass through obstacles, making it crucial to simulate signal propagation using stochastic geometry 
and apply up-to-date models for signal attenuation. The 3GPP TR~38.901 specification includes models that describe signal
 attenuation in various 5G~network scenarios using empirical formulas. Nevertheless, simpler formulas are typically employed 
 to create models based on stochastic geometry. The authors present the cumulative distribution function 
 for path loss at random user locations according to the scenarios described in 3GPP TR~38.901. In numerical examples, it is shown that the difference 
in values with the simplified formula can be significant and lead to underestimation of the network's capacity}

\KWE{wireless network; 5G; 3GPP TR 38.901; path loss; line-of-sight (LOS); non-line-of-sight (NLOS); stochastic geometry}



\DOI{10.14357/19922264240204}{EKLCAP}

\vspace*{-12pt}

\Ack

\vspace*{-3pt}

\noindent
The publication has been supported by the RUDN University Scientific Projects 
Grant System, project No.\,025319-2-000.


  \begin{multicols}{2}

\renewcommand{\bibname}{\protect\rmfamily References}
%\renewcommand{\bibname}{\large\protect\rm References}

{\small\frenchspacing
 {%\baselineskip=10.8pt
 \addcontentsline{toc}{section}{References}
 \begin{thebibliography}{99} 
\bibitem{Moltchanov2022a-1}
\Aue{Moltchanov,~D.\,A., V.\,O.~Begishev, K.\,E.~Samouylov, and Y.\,A.~Koucheryavy.}
2022.
\textit{Seti 5G/6G: arkhitektura, tekhnologii, metody analiza i~rascheta}
[The 5G/6G networks: Architecture, technologies, analysis methods, and calculations].
Moscow: RUDN University. 516~p.

\bibitem{Hmamouche2021-a}
\Aue{Hmamouche,~Y., M.~Benjillali, S.~Saoudi, H.~Yanikomeroglu, and 
M.\,D.~Renzo.}
2021.
New trends in stochastic geometry for wireless networks: A~tutorial and survey.
\textit{P.~IEEE}. 109(7):1200--1252.
doi: 10.1109/ JPROC.2021.3061778.

\bibitem{Naumov2016-1}
\Aue{Naumov,~V.\,A., and K.\,E.~Samouylov.}
2016. O~svyazi resursnykh sistem massovogo obsluzhivaniya s~setyami Erlanga
[On relationship between queuing systems with resources and Erlang networks].
\textit{Informatika i~ee Primeneniya~--- Inform Appl.} 10(3):9--14.
doi: 10.14357/ 19922264160302.

\bibitem{Gorbunova2018-1} %4
\Aue{Gorbunova,~A.\,V., V.\,A.~Naumov, Yu.\,V.~Gaidamaka, and K.\,E.~Samouylov.}
2018. Resursnye sistemy massovogo obsluzhivaniya kak modeli besprovodnykh sistem svyazi
[Resource queuing systems as models of wireless communication systems].
\textit{Informatika i~ee Primeneniya~--- Inform. Appl.} 12(3):48--55.
doi: 10.14357/19922264180307.



\bibitem{Markova2019-1} %5
\Aue{Markova,~E.\,V., A.\,A.~Golskaia, I.\,L.~Dzantiev, I.\,A.~Gudkova, and 
S.\,Ya.~Shorgin.}
2019. Sravnitel'nyy analiz pokazateley effektivnosti modeli besprovodnoy seti mezhmashinnogo 
vzaimodeystviya, rabotayushchey v~ramkakh dvukh politik razdeleniya radioresursov
[Comparative analysis of performance measures for a~wireless machine-to-machine 
network model operating within two radio resource management policies].
\textit{Informatika i~ee Primeneniya~--- Inform. Appl}. 13(1):108--116.
doi: 10.14357/ 19922264190115.


\bibitem{Moltchanov2022b-1} %6
\Aue{Moltchanov,~D.\,A., E.\,S.~Sopin, V.\,O.~Begishev, A.\,K.~Sa\-muy\-lov, 
Y.\,A.~Koucheryavy, and K.\,E.~Samouylov.}
2022.
A tutorial on mathematical modeling of 5G/6G millimeter wave and terahertz 
cellular systems.
\textit{IEEE Commun. Surv. Tut.} 24(2):1072--1116.
doi: 10.1109/COMST. 2022.3156207.

\bibitem{Kochetkova2021-1} %7
\Aue{Kochetkova,~I.\,A., A.\,I.~Kushchazli, P.\,A.~Kharin, and S.\,Ya.~Shorgin.}
2021. Model' skhemy prioritetnogo do\-stu\-pa trafika URLLC i~eMBB v~seti pyatogo pokoleniya v~vide resursnoy sistemy massovogo obsluzhivaniya
[Model for analyzing priority admission control of URLLC and eMBB communications 
in 5G networks as a~resource queuing system].
\textit{Informatika i~ee Primeneniya~--- Inform. Appl}. 15(4):87--92.
doi: 10.14357/19922264210412.

\bibitem{3GPP38901-1}
3GPP TR 38.901. 2023. Study on channel model for frequencies from~0.5 to~100~GHz, 
Release 17.1.0.

\bibitem{Bolla2023-1}
\Aue{Bolla,~R., R.~Bruschi, C.~Lombardo, A.~Mohammadpour, R.~Trivisonno, and 
W.\,Y.~Poe.}
2023.
A 5G multi-gNodeB\linebreak\vspace*{-12pt}

\pagebreak

\noindent
 simulator for ultra-reliable 0.5--100~GHz communication in 
indoor Industry 4.0 environments.
\textit{Comput. Netw.} 37:110103. doi: 10.1016/j.comnet.2023.11010.

\bibitem{Ventzel2018-1}
\Aue{Ventzel,~E.\,S. and L.\,A.~Ovcharov.}
2018.
\textit{Teoriya veroyatnostey i~ee inzhenernye prilozheniya}
[Probability theory and its engineering applications].
Moscow: Justice. 480~p.

\end{thebibliography}

 }
 }

\end{multicols}

\vspace*{-6pt}

\hfill{\small\textit{Received March 15, 2024}} 

\vspace*{-12pt}


\Contr

\vspace*{-3pt}

\noindent
\textbf{Makeeva Elena D.} (b.\ 1996)~--- PhD student, Department of Probability 
Theory and Cyber Security, RUDN University, 6~Miklukho-Maklaya Str., Moscow 
117198, Russian Federation; junior scientist, V.\,A.~Trapeznikov Institute of 
Control Science of the Russian Academy of Sciences, 65~Profsoyuznaya Str., 
Moscow 117997, Russian Federation; \mbox{elena-makeeva-96@mail.ru}

\vspace*{3pt}

\noindent
\textbf{Kochetkova Irina A.} (b.\ 1985)~--- Candidate of Science (PhD) in physics 
and mathematics, associate professor, Department of Probability Theory and Cyber 
Security, RUDN University, 6~Miklukho-Maklaya Str., Moscow 117198, Russian 
Federation; senior scientist, Federal Research Center ``Computer Science and 
Control'' of the Russian Academy of Sciences, 44-2~Vavilov Str., Moscow 119333, 
Russian Federation; \mbox{kochetkova-ia@rudn.ru}

\vspace*{3pt}

\noindent
\textbf{Shorgin Sergey Ya.} (b.\ 1952)~--- Doctor of Science in physics and 
mathematics, professor, principal scientist, Federal Research Center ``Computer Science and Control'' of the Russian Academy of 
Sciences, 44-2~Vavilov Str., Moscow 119133, Russian Federation; 
\mbox{sshorgin@ipiran.ru}





\label{end\stat}

\renewcommand{\bibname}{\protect\rm Литература}         %03
%без проверки Юлей
\renewcommand{\d}{\mathrm{d}}
%\renewcommand{\i}{\mathrm{i}}
%\def\ld{\ldots}
%\def\cd{\cdots}
%\def\b{\overline b}

\def\stat{razumcik}

\def\tit{СТАЦИОНАРНЫЕ ХАРАКТЕРИСТИКИ СИСТЕМЫ Geo$/G/1/\infty$ 
С~НЕОРДИНАРНЫМ ВХОДЯЩИМ ПОТОКОМ, УПРАВЛЯЮЩИМ~РАЗМЕРОМ ОЧЕРЕДИ$^*$}

\def\titkol{Стационарные характеристики системы Geo$/G/1/\infty$ 
с~неординарным входящим потоком}
%, управляющим размером очереди}

\def\aut{С.\,И.~Матюшенко$^1$, Р.\,В.~Разумчик$^2$}

\def\autkol{С.\,И.~Матюшенко, Р.\,В.~Разумчик}

\titel{\tit}{\aut}{\autkol}{\titkol}

\index{Матюшенко С.\,И.}
\index{Разумчик Р.\,В.}
\index{Matyushenko S.\,I.}
\index{Razumchik R.\,V.}

{\renewcommand{\thefootnote}{\fnsymbol{footnote}} \footnotetext[1]
{Исследование выполнено при финансовой поддержке РФФИ  
(проект 20-07-00804) и~в~соответствии с~программой Московского центра 
фундаментальной и~прикладной математики.}}


\renewcommand{\thefootnote}{\arabic{footnote}}
\footnotetext[1]{Российский
университет дружбы народов, matyushenko\_si@pfur.ru}
\footnotetext[2]{Институт проблем информатики Федерального 
исследовательского
центра <<Информатика и~управ\-ле\-ние>> Российской академии наук,
\mbox{rrazumchik@ipiran.ru}}

%\vspace*{-10pt}


\Abst{Рассматривается функционирующая в~дискретном времени
система массового обслуживания (СМО) с~одним прибором,
очередью неограниченной емкости и~неординарным геометрическим потоком 
заявок.
В~системе реализован специальный механизм управления очередью:
в~момент поступления в~систему новой группы заявок ее размер сравнивается 
с~текущим общим числом заявок в~системе
и,~если число заявок в~новой группе превышает общее число заявок 
в~системе, новая группа целиком принимается в~систему,
вытесняя при этом все прежде находившиеся в~ней заявки;
в~противном случае новая группа покидает систему, не оказывая на нее 
никакого воздействия.
Заявки обслуживаются прибором по одной. В предположении, что заявки 
в~группе независимы, а
распределения чисел заявок в~группе и~времени обслуживания являются 
произвольными дискретными,
найдены основные стационарные характеристики функционирования.}

\KW{дискретное время; неординарный поток; управление очередью; выходящий 
поток}

\DOI{10.14357/19922264200404} 
 
\vspace*{2pt}


\vskip 10pt plus 9pt minus 6pt

\thispagestyle{headings}

\begin{multicols}{2}

\label{st\stat}


\section{Введение}

\vspace*{-5pt}

Дисциплины обслуживания очередей, которые позволяют
повысить эффективность работы %систем массового обслуживания 
СМО
путем использования доступной информации (известной либо точно, либо 
приближенно)
о~размерах (временах обслуживания и~т.\,п.)\
поступающих в~них заявок, продолжают оставаться предметом активных
научных исследований~\cite{i2, i3, i4, i5, i6}.
При этом внимание часто сосредоточено вокруг наиболее известной
из всех специальных дисциплин обслуживания~---
дисциплины преимущественного обслуживания заявки минимальной
остаточной длины (shortest remaining time first, SRPT).

В~недавней работе~\cite{i1} авторами рассмотрена новая однолинейная СМО 
с~групповым потоком и~специальным механизмом обработки очереди,
по которому поступающая группа всегда вытесняет из системы
все находящиеся в~ней заявки, если размер группы достаточно велик.
Побудительным мотивом\footnote[3]{Появление описанного механизма, по-видимому,
также связано с~результатами работ~\cite{i7,i8}, в~которых показано, что
одна его более простая разновидность приводит к~мультипликативному виду
совместного стационарного распределения в~соответствующим образом
образованных сетях массового обслуживания. Отметим, что сети из СМО, 
рассмотренных
в~\cite{i1} и~в~этой работе, таким свойством уже не обладают.}
к~изучению этого механизма (в отличие от SRPT) послужил
поиск путей максимизации загрузки системы.

В предположениях, что поток групп пуассоновский, заявки в~группе
независимы, число заявок в~группе имеет произвольное,
а~времена обслуживания~--- экспоненциальное распределение,
в~\cite{i1} найде\-ны с~помощью метода обращения времени основные 
стационарные характеристики сис\-те\-мы.
В~данной статье рассматривается аналогичная~\cite{i1} СМО, но 
функционирующая в~дискретном времени и~при этом
в~более общих предположениях о~распределении времени обслуживания (оно 
допускается произвольным дискретным).
Предложен основанный на вероятностных соображениях из~\cite{n4}
метод\footnote[4]{Здесь же необходимо отметить,
что в~ряде частных случаев (например, для геометрического
распределения времени обслуживания) для вывода стационарного 
распределения,
как и~в~непрерывном времени, может быть применен метод обращения 
времени.}
нахождения совместного стационарного распределения числа заявок в~системе 
и~остаточных длин заявок в~очереди.
Стандартными методами изучен и~ряд других стационарных характеристик: 
период занятости,
время пребывания заявки в~системе, выходящий поток потерянных заявок.

Прежде чем переходить к~подробному описанию системы, отметим, что 
отличительная особенность рассмотренной СМО заключается в~том, что 
управ\-ля\-ющие размером очереди решения принимаются по результатам
сравнения не остаточных времен обслуживания (см., например,~\cite{tata,i10,i11}), 
а~остаточных размеров групп заявок. Несмотря на то что в~такой СМО 
поступление заявок
может привести к~потере уже находящихся в~системе заявок,
она не относится к~типу СМО с~отрицательными за\-яв\-ка\-ми/сиг\-на\-ла\-ми 
(см., например,~\cite{i12}).



\section{Описание системы}

Рассматривается функционирующая в~дискретном времени\footnote{Дискретное 
время вводится обычным образом
(см., например, \cite{distime}).} однолинейная СМО
%система массового обслуживания 
с~очередью неограни\-чен\-ной емкости,
в~которую поступает неординарный геометрический поток заявок,
определяемый следующим образом. На каждом такте
(далее будем называть тактом как интервал времени
между соседними изменениями состояния сис\-те\-мы, так
и~сами моменты, в~которые происходят эти изменения)
с~вероятностью~$a$ приходит
группа заявок случайного размера, не зависящего
от всего процесса функционирования сис\-те\-мы. При
этом в~каждой поступившей группе имеется $i\hm\ge 1$ заявок 
с~вероятностью~$l_i$. 
Заявки обслуживаются прибором по одной,
причем время обслуживания заявки становится известным
в~момент ее поступления на прибор.
Распределение времени обслуживания заявки является произвольным
дискретным с~вероятностью~$b_i$, $i\hm\ge0$, того, что
обслуживание заявки продлится~$i$~так\-тов (предполагается, что
$b_0\hm=0$).

% (предполагается, что $l_i>0$ при всех $i \ge 1$) и~$b_i>0$ при всех $i \ge 1$

Будем использовать следующие обозначения:
\begin{description}
\item[\,]
$\overline{a}=1-a$ --- вероятность непоступления заявки на такте;
\item[\,]
$B_i= \sum\nolimits_{j=i}^\infty b_j$, $i \hm\ge 0$,~--- вероятность 
того,
что обслуживание заявки продлится не менее~$i$~тактов;
\item[\,]
$\mathsf{E}b^k=\sum\nolimits_{i=1}^\infty i^k b_i$~--- $k$-й момент 
времени обслуживания;
\item[\,]
$L_i =\sum\nolimits_{j=i}^\infty l_j$, $i\hm\ge 1$,~---
вероятность того, что в~поступившей группе окажется не менее~$i$~заявок.
Очевидно, $L_1=1$;  
\item[\,]

$\mathsf{E}l^k=\sum\nolimits_{i=1}^\infty i^k l_i$~--- $k$-й момент 
размера группы;
\item[\,]
$\beta(z)=\sum\nolimits_{j=1}^\infty z^j b_j$~--- производящая функция 
(ПФ)
времени обслуживания заявки.
\end{description}


В системе реализован следующий механизм управления очередью. В~момент
поступления в~сис\-те\-му новой группы заявок ее размер $x$ сравнивается
с текущим общим числом заявок в~системе~$y$. Та из групп
заявок, длина которой больше, остается в~системе, а~другая
покидает систему. Другими словами, если $x\hm>y$, то все~$y$~заявок 
мгновенно
уходят из системы; новая группа заявок размера~$x$
целиком помещается в~очередь, и~одна заявка из группы немедленно занимает 
прибор.
Если же $x \hm\le y$, то поступающая группа заявок теряется,
не оказывая на систему никакого воздействия.

Примем, что все изменения состояния СМО происходят в~конце такта 
в~следующем порядке\footnote{В зарубежной литературе это схема EAS-IA 
(см., например, \cite[с.~2--3]{nobel}).}:
\begin{itemize}
\item если на этом такте завершилось обслуживание заявки на приборе, то 
она покидает систему
и~на прибор сразу же поступает следующая заявка из очереди;
\item затем с~вероятностью~$a$ в~систему поступает группа заявок и, если 
система оказалась непустой, происходит потеря либо поступившей группы, 
либо всех тех заявок, которые находились в~системе (до момента 
поступления).
\end{itemize}

Далее будем предполагать, что $L_i\hm>0$ и~$B_i\hm>0$ при всех~$i$ 
и~выполнено условие~\eqref{stab} (см.\ разд.~4),
необходимое и~достаточное для существования стационарного режима.
\vspace*{-.5pt}

\section{Период занятости}

Рассмотрим случайный процесс $\{ \eta(t)\hm=(\nu(t),\xi(t)), \ t \hm\ge 
0\}$,
где $\nu(t)$~--- общее число заявок в~системе,
а $\xi(t)$~--- остаточное время обслуживания (далее~--- длина) заявки на 
приборе
непосредственно после такта~$t$. При $\nu(t)\hm=0$
координата~$\xi(t)$ не определяется.
Процесс ${\{\eta(t), \ t\ge0\}}$ является \mbox{цепью} Маркова, причем множество 
ее состояний~$\mathcal{X}$ имеет вид:
$$\mathcal{X}\hm=\{0\} \bigcup \{ (n,i), \ n \hm\ge 1, i \hm\ge 1 \},
$$
где $n$~--- число заявок в~системе; $i$~--- остаточная длина заявки
на приборе.

Пусть после очередного такта в~системе оказалось $n \hm\ge 1$ заявок 
и~обслуживание заявки только началось.
Обозначим через~$\mathcal{U}_{n}(z)$, $0\hm<z\hm\le 1$, ПФ числа тактов 
до того момента, когда в~системе впервые окажется $n\hm-1$ заявок.
Если в~момент начала функционирования в~системе находится $n \hm\ge 1$ 
заявок, то
распределение ее периода занятости (ПЗ) в~терминах ПФ имеет вид 
$\prod\nolimits_{j=1}^n \mathcal{U}_{j}(z)$;
иначе~--- $\sum\nolimits_{n=1}^\infty l_n \mathcal{U}_{n}(z)$.
Воспользовавшись формулой полной вероятности,
получаем систему уравнений для~$\mathcal{U}_{n}(z)$:
%\vspace*{-.5pt}

%\pagebreak

\noindent
\begin{multline}
\label{eq1}
\mathcal{U}_{n}(z)
=\mathcal{D}_{n}(z)
\mathcal{U}_{n-1}(z)
+\mathcal{E}_{n}(z)
\mathcal{U}_{n}(z)+{}\\
{}+ \mathcal{F}_{n}(z)
\sum\limits_{j=n+1}^\infty
al_j \mathcal{U}_{j}(z),
\enskip n \ge 1.
\end{multline}
Здесь используется соглашение $\mathcal{U}_{0}(z) \hm\equiv 1$ и
обозначения:
$$
\mathcal{D}_{n}(z)
=
\fr{z_{n-1}}{z_{n}}\,
\beta\left( z z_n \right);
\quad
\mathcal{E}_{n}(z)
=
\fr{al_{n}}{z_{n}}\,
\beta\left( z z_n \right);
$$$$
\mathcal{F}_{n}(z)
=
z\fr{ 1- \beta\left( z z_n \right)}{1-z z_n}\,,
$$
где $z_n=1-aL_{n+1}$.

Заметим, что 
$$\mathcal{D}_{n}(z) \hm+ \mathcal{E}_{n}(z)\hm + 
\mathcal{F}_{n}(z) \fr{1\hm-z z_n}{z}\hm=1\,.$$
Системе~\eqref{eq1} можно придать следующий вид:
$$
\mathcal{U}_{n}(z)=
\sum\limits_{j=1}^\infty \mathcal{T}_{nj}(z) \mathcal{U}_{j}(z)
+\mathcal{B}_n(z), \enskip n \ge 1\,,
$$

\noindent 
где $\mathcal{B}_1(z)\hm=\mathcal{D}_{1}(z)$, $\mathcal{B}_n(z)\hm=0$, 
$n\hm\ge 2$;
$\mathcal{T}_{nj}(z)$~--- соответствующим образом подобранные 
по~\eqref{eq1} коэффициенты.
Поскольку при $0<z<1$ в~каждой строке $\sum\nolimits_{j=1}^\infty 
\mathcal{T}_{nj}(z)\hm<1$
и, очевидно, свободные члены удовлетворяют условию
$\mathcal{B}_n(z) \hm\le K (1\hm-\sum\nolimits_{j=1}^\infty 
\mathcal{T}_{nj}(z))$
при некоторой постоянной $K\hm>0$,
то система~\eqref{eq1} имеет ограниченное решение, которое
может быть найдено методом последовательных приближений
(см., например,~\cite[теоремы~Ia,~IVa]{kry} или~\cite[теорема~1]{wil}).
При $z\hm=1$ единственное решение\footnote{Ввиду громоздкости выкладок 
лишь заметим, что это можно показать и~прямыми вычислениями,
если воспользоваться представлением решения~\eqref{eq1}, данным 
в~\cite[соотн.~(7)]{Car},
и выписать явный вид входящих в~него слагаемых.}~\eqref{eq1}~--- это 
$\mathcal{U}_{n}(1)\hm=1$ при всех~$n$, если параметры системы таковы, 
что
стационарный режим функционирования существует (т.\,е. 
выполняется~\eqref{stab}, см.\ разд.~4).


\section{Стационарное распределение очереди}

Введем обозначения:
\begin{description}
\item[\,]
$P_0=\lim\nolimits_{t \rightarrow \infty } {\sf P} (\nu(t)\hm=0)$~---
стационарная вероятность того, что непосредственно после очередного такта 
система будет
пуста;
\item[\,]
$p_{ni}=\lim\nolimits_{t \rightarrow \infty } {\sf P} (\nu(t)=n, 
\xi(t)\hm=i)$, $n \hm\ge 1$, $i \hm\ge 1$,~---
стационарная вероятность того, что непосредственно после очередного такта 
в~системе будет $n$
заявок и~до окончания обслуживания заявки на приборе останется $i$ 
тактов.
\end{description}

Положим
$$
P_n=\sum\limits_{i=1}^\infty p_{ni}, \enskip n \ge 1; \enskip {\overline 
P}_n=\sum\limits_{i=0}^n P_{i}, \enskip
 n \ge 0\,.
$$

Из системы уравнений равновесия (СУР) стандартным образом находится
двойная ПФ $\mathcal{P}(u,v)\hm=P_0+\sum\nolimits_{n=1}^\infty 
\sum\nolimits_{i=1}^\infty u^n v^i p_{ni}$,
$0\hm<u,v\hm\le 1$:
\begin{multline*}
%\label{eq2}
\mathcal{P}(u,v)=
\sum\limits_{n=1}^\infty \sum\limits_{i=1}^\infty p_{ni}\times{}\\
{}\times
\left( (1-aL_{n+1}) u^n v^{i-1} + \beta(v) \sum\limits_{j=n+1}^\infty a 
l_j u^j \right)+{}\\
{}+\sum\limits_{n=1}^\infty p_{n1}
\left( (1-aL_{n}) u^{n-1} \left(\beta(v)-u \right) +{}\right.\\
\!\!\left.{}+ a l_n u^n (\beta(v)\!-\!1) \right)+
P_0\! \left(\!\beta(v) \sum\limits_{j=1}^\infty a l_j u^j+\!1\!-a \beta(v)\!\!\right)\!\!.\!\!
\end{multline*}

\noindent 
Однако ее вид малопригоден для проведения анализа.
Поэтому поступим следующим образом.
Введем новую СМО с~конечным числом $n$ мест для ожидания,
отличающуюся от исходной только тем, что
если в~очереди находится $n$ заявок и~поступает
новая группа заявок размера больше $n$,
то все находящиеся в~системе заявки покидают ее,
а~(любые)~$n$~заявок из новой группы принимаются в~систему.
Воспользовавшись приемом, введенном в~\cite{n4} и~подробно изложенном 
в~\cite{distime},
можно показать, что стационарные вероятности состояний в~исходной и~новой 
СМО
отличаются лишь на постоянный множитель.
Это дает возможность записать следующую СУР:
\begin{multline}
\label{eq3r}
p_{ni}=(1-aL_{n+1})
p_{n,i+1}
+a l_n b_i p_{n1}
+a l_n b_i
{\overline P}_{n-1}+{}\\
{}+ {\overline P}_n
\sum\limits_{j=n+1}^{\infty} a l_j q_{n,i,j},
\enskip n \ge 1\,, \enskip i \ge 1\,,
\end{multline}

\noindent  
где $q_{n,i,j}$~--- условная вероятность того, что, когда в~системе 
впервые
окажется $n$ заявок, остаточное время обслуживания заявки
на приборе будет равно~$i$ при условии, что
после очередного такта в~сис\-те\-ме оказалось $j \hm\ge n\hm+1$ заявок
и~обслуживание заявки на приборе только началось.
Если система находится в~стационарном режиме, то $q_{n,i,j}\hm=b_i$,
$n\hm\ge 1$, $j \hm\ge n\hm+1$.

Переходя к~ПФ $\mathcal{P}_n(z)\hm=\sum\nolimits_{i=1}^\infty z^i 
p_{ni}$, $0 \hm< z \hm\le 1$,
из~\eqref{eq3r} получаем:
\begin{multline}
\label{eq5}
P_n(z) \fr{z - z_n }{z}=
\left(a l_n \beta(z) - z_n \right) p_{n1}+{}\\
{}+a L_n \beta(z) {\overline P}_{n-1}
+a L_{n+1} \beta(z) P_n, \enskip n \ge 1\,.
\end{multline}

\noindent  
Подставляя $z\hm=1$, находим 
$${p_{n1}\hm={\overline P}_{n-1}\,\fr{1\hm-z_{n-1}}{z_{n-1}}}.
$$
Теперь, воспользовавшись теоремой Руше, с~учетом найденного вида~$p_{n1}$ 
из~\eqref{eq5} имеем:
%\begin{equation}
\begin{multline*}
%\label{eq6}
P_n
=c_n {\overline P}_{n-1}\,,\\
c_n=\fr{z_{n}}{1-z_{n}}\,
\fr{1-z_{n-1}}{z_{n-1}}\,
\fr{1- \beta(z_n)}{\beta(z_n)}\,,
\enskip n \ge 1\,.
\end{multline*}
%\end{equation}

\noindent 
Отсюда, с~учетом соотношения ${\overline P}_n\hm={\overline P}_{n-
1}\hm+P_n$, следует, что 
$${{\overline P}_{n} \hm= P_0 \prod\limits_{i=1}^n
(1+c_i)},\enskip {n \hm\ge 1}\,.
$$
Используя теперь условие нормировки
$\lim\limits_{n \rightarrow \infty } {\overline P}_n=1$,
окончательно получаем, что
\begin{equation}
\label{sn4-1}
\left.
\begin{array}{rl}
P_0 &=\left( \prod\limits_{i=1}^\infty (1+c_i) \right)^{-1};\\
P_n &= \fr{c_n}{\prod\nolimits_{i=n}^\infty
(1+c_i)}\,, \enskip
 n \ge 1\,.
\end{array}
\right\} 
\end{equation}

Из эргодической теоремы Фостера следует, что необходимым и~достаточным
условием существования стационарного режима является сходимость 
произведения
$\prod\nolimits_{i=1}^\infty (1+c_i)$, которая эквивалентна условию
\begin{equation}
\label{stab}
\sum\limits_{i=1}^\infty
\fr{z_{i}}{1-z_{i}}\,
\fr{1-z_{i-1}}{z_{i-1}}\,
\fr{1- \beta(z_i)}{\beta(z_i)}
< \infty\,.
\end{equation}

\noindent 
Для выполнения \eqref{stab} достаточно\footnote{Действительно,
так как $\beta(z_n) \ge 1- (1-z_n) \mathsf{E}b$ 
и~$\beta(z_n)\hm<\beta(z_{n+1})$,
то $c_n\hm\le a L_n \mathsf{E}b/(\beta(z_1)(1\hm- a L_n))$,
а ряд $\sum\nolimits_{i=1}^\infty aL_i /(1\hm- a L_i)$
сходится, если $\sum\nolimits_{i=1}^\infty aL_i \hm= a \mathsf{E}l 
\hm<\infty$.}, чтобы $\mathsf{E}b \, \mathsf{E}l \hm< \infty$.
При расчете моментов стационарного распределения по~\eqref{sn4-1}
необходимо быть уверенным, что соответствующие ряды сходятся;
достаточным условием существования $\mathsf{E}\nu^k$
является существование соответствующего момента размера группы.
Для расчета же совместного
стационарного распределения числа заявок в~системе и~остаточного
времени обслуживания заявки на приборе можно воспользоваться
формулой\footnote{Здесь и~далее используется соглашение
$\sum\nolimits_{j=1}^0\hm=0$.}:
\begin{multline*}
p_{ni}={\overline P}_{n-1}
\left(\fr{1-z_{n-1}}{z_{n-1} z_n^{i-1}}-
\vphantom{\sum\limits_{j=1}^{i-1}} 
{}\right.\\
\left.{}-\left(
\fr{a l_n}{z_{n-1}}
+aL_{n+1} (1+c_n)
\right)
\sum\limits_{j=1}^{i-1}
\fr{b_{i-j}}{z_n^{j}}
\right)\!, \enskip n,i \ge 1\,,
\end{multline*}

\noindent  
которая получается путем обращения ПФ~\eqref{eq5}. Отсюда, поскольку 
времена обслуживания заявок в~группе
предполагаются независимыми, немедленно
следует совместное стационарное распределение общего числа
заявок в~системе, остаточного времени обслуживания
заявки на приборе и~каждой заявки в~очереди.

\section{Некоторые характеристики производительности}

Остановимся на выводе формул для вероят\-ностей потери заявки.
Обозначим через~$\pi_1$ и~$\pi_2$ соответственно вероятность
потери произвольной заявки при поступлении и~во время пребывания 
в~системе.
Для этого необходимо перейти от стационарных вероятностей~$P_n$
по тактам к~стационарным вероятностям по моментам
поступления заявок в~систему (которые будем обозначать~$P^*_n$),
а~также зафиксировать порядок выбора заявок на обслуживание из очереди.

Нетрудно видеть, что 
$$P^*_0\hm=P_0\hm+p_{11}\,;\quad P^*_n\hm=P_n\hm-
p_{n1}\hm+p_{n+1,1}\,,\enskip n \hm\ge 1\,.
$$
Поскольку случайно выбранная заявка с~вероятностью $kl_k/\mathsf{E}l$
принадлежит группе размера~$k$, по формуле полной вероятности
получаем\footnote{А вероятность потери
поступающей группы равна $\sum\nolimits_{n=1}^\infty P^*_n (1-L_{n+1})$.}: 
$$\pi_1=\sum\limits_{n=1}^\infty P^*_n \sum\limits_{j=1}^n \fr{j l_j }{\mathsf{E}l}\,.
$$
Предположим, что заявки обслуживаются из очереди в~порядке поступления.
Обозначим через $\pi_{2,k,j}$, $k\hm\ge 1$, $1\hm \le j \hm\le k$, 
условную вероятность того, что заявка будет потеряна, при условии что 
она принята в~сис\-те\-му в~группе размера~$k$ и
оказалась в~группе на $j$-м месте. Величины~$\pi_{2,k,j}$ могут быть
вычислены рекуррентно по формулам:
\begin{align*}
\pi_{2,k,1} &= 1- \fr{\beta(z_k)}{z_k}, \enskip k \ge 1\,;\\
\pi_{2,k,j} &=\pi_{2,k,1}+ (1-\pi_{2,k,1})(aL_k + z_{k-1} \pi_{2,k-1,j-1}),\\ 
&\hspace*{42mm}
 1 \le j \le k\,, \enskip k \ge 2\,.
\end{align*}

\noindent 
Усредняя $\pi_{2,k,j}$ по распределению
размера принятой в~систему группы, содержащей случайно выбранную заявку,
и~предполагая, что, поступая в~группе размера~$k$, заявка
может равновероятно оказаться на любом из $k$ мест, находим:
$$
\pi_2= \sum\limits_{n=1}^\infty 
\fr{l_n {\overline P}^*_{n-1} }{\sum\nolimits_{k=1}^\infty k l_k 
{\overline P}^*_{k-1} }
\sum\limits_{j=1}^n \pi_{2,n,j},
\ \ {\overline P}^*_n=\sum\limits_{i=0}^n P^*_{i}.
$$

Остановимся теперь на стационарных распределениях
времен пребывания в~системе
обслуженной и~потерянной заявки.
Обозначим через~$V_{1,j,k}(z)$ ПФ условных вероятностей того, что
заявка будет обслужена и~время ее пребывания в~системе равно~$i$~тактам,
при условии что она принята в~систему в~группе размера~$k$ и
оказалась в~группе на $j$-м месте. При $j\hm=1$ заявка будет обслужена,
если за время ее пребывания на приборе в~систему не поступит
группа размера больше~$k$. Поэтому 
$$V_{1,1,k}(z)\hm=\sum\limits_{j=1}^\infty z^j b_j z_k^{j-1}\hm=
\fr{\beta(z z_k)}{z_k}\,.
$$
Если выделенная заявка оказалась не на первом месте в~группе, то
ее время пребывания зависит от времени обслуживания находящихся перед ней 
заявок.
Так как эти времена по предположению независимы, то в~терминах ПФ имеем: 
$$V_{1,j,k}(z)\hm= V_{1,1,k}(z)z_{k-1}V_{1,j-1,k-1}(z)\,,\enskip
2 \hm\le j \hm\le k\,.
$$
Воспользовавшись теперь формулой полной вероятности, получаем следующее
выражение для ПФ~$V_1(z)$ стационарного распределения времени пребывания
обслуженной заявки в~системе:
\begin{equation}
\label{v1}
V_1(z)=
\sum\limits_{n=1}^\infty \fr {l_n {\overline P}^*_{n-1} 
}{\sum\nolimits_{k=1}^\infty k l_k {\overline P}^*_{k-1} }
\sum\limits_{j=1}^n V_{1,n,j}(z).
\end{equation}

\noindent 
Вводя $V_{2,j,k}(z)$~--- ПФ условных вероятностей того, что
заявка не будет обслужена и~время ее пребывания в~системе равно $i$ 
тактам,
при условии что она принята в~систему в~группе размера~$k$ и
оказалась в~группе на $j$-м месте,~--- 
и~рассуждая аналогичным образом, нетрудно по формуле полной вероятности
получить следующие соотношения:
\begin{align*}
V_{2,1,k}(z)&=\fr{1-z_k}{z_k}\,\fr { z z_k - \beta(z z_k)}{1- z z_k}\,;\\
V_{2,j,k}(z)&=V_{2,1,k}(z)+V_{1,1,k}(z)
\left(1-z_{k-1} +{}\right.\\
&\hspace*{12mm}\left.{}+ z_{k-1} V_{2,j-1,k-1}(z)\right), \enskip 2 \le j \le k\,.
\end{align*}

\noindent Безусловная ПФ~$V_2(z)$ стационарного распределения времени 
пребывания в~системе
принятой, но в~итоге потерянной заявки рассчитывается по 
формуле~\eqref{v1}
с~заменой~$V_{1,n,j}(z)$ на $V_{2,n,j}(z)$.


\section{Выходящий поток потерянных заявок}

При изучении рассмотренной системы в~связке с~другими СМО
важны характеристики выходящего из нее потока потерянных заявок.
В~общем случае, очевидно, он не является ординарным.
Не является он и~геометрическим:
числа заявок, покидающих систему на соседних интервалах,
представляют собой зависимые случайные величины.


Рассмотрим последовательные моменты~$\tau^-_n$,\linebreak $n\hm\ge 1$,
потерь заявок и~введем вложенную цепь Маркова
$\nu_n\hm=\nu(\tau^-_n)$~--- общее число заявок в~системе
непосредственно после момента~$\tau^-_n$.
Положим $l_n\hm=\tau^-_{n+1}-\tau^-_n$
и~обозначим через ${h_{i,t_1,t_2}(n)\hm={\sf P} (\nu_n\hm=i, l_n\hm=t_1, 
l_{n+1}\hm=t_2)}$, $i, t_1,t_2 \hm\ge 1$,
вероятность того, что после $n$-й потери общее число заявок в~системе 
будет равно~$i$ и~длины интервалов между последующими двумя потерями 
равны~$t_1$ и~$t_2$ тактам соответственно. 
Положим
\begin{align*}
p_i^-(n)&=\sum\limits_{t_1=1}^\infty \sum\limits_{t_2=1}^\infty h_{i,t_1,t_2}(n)\,;\\
h_{t}(n)&=\sum\limits_{i=1}^\infty\sum\limits_{t_2=1}^\infty 
h_{i,t,t_2}(n)\,;\\
h_{t_1,t_2}(n)&=\sum\limits_{i=1}^\infty h_{i,t_1,t_2}(n)\,.
\end{align*}
При выполнении условия существования стационарного режима
существуют и~стационарные вероятности
$p_i^-\hm=\lim\nolimits_{n \rightarrow \infty } p_i^-(n)$, $i\hm\ge 1$, 
того,
что непосредственно после момента потери в~системе будет~$i$~заявок,
а также стационарные вероятности
\begin{align*}
h_{t}&=\lim\limits_{n \rightarrow \infty } h_{t}(n)\,,\enskip t \hm\ge 1\,;\\
h_{t_1,t_2}&=\lim\limits_{n \rightarrow \infty } h_{t_1,t_2}(n)\,,\enskip 
t_1,t_2 \hm\ge 1\,.
\end{align*}

Cистема уравнений Колмогорова--Чепмена для вероятностей $p_i^-(n)$, $n 
\hm\ge 2$, $i \hm\ge 1$, имеет вид:
\begin{multline}
p_i^-(n+1)=p_i^-(1)\sum\limits_{j=1}^\infty
p_j^-(n) \fr{\beta(\overline{a})^j}{\overline{a}}+{}\\
{}+\left(al_i \sum\limits_{j=1}^{i-1}
p_j^-(n) +p_i^-(n)\sum\limits_{j=1}^{i}al_j\right)
\left( 1-\fr{\beta(\overline{a}) }{\overline{a}}\right)+{}
\\
{}+
\sum\limits_{k=1}^\infty
\left(
al_i \sum\limits_{j=k+1}^{k+i-1}
p_j^-(n) +
p_{k+i}^-(n) \sum\limits_{j=1}^{i}
al_j \right)\times{}\\
{}\times
\fr{\beta(\overline{a})^k (1-\beta(\overline{a})) }{\overline{a}}, 
\enskip n \ge 1\,, \enskip i \ge 1\,,
\label{loss1}
\end{multline}

\noindent к~которому необходимо добавить условие нормировки 
$\sum\nolimits_{i=1}^{\infty} p_i^-(n+1)\hm=1$.
Вероятности $p_i^-(1)$, $i \hm\ge 1$, того, что после первой потери 
в~системе окажется
$i$ заявок, также могут быть найдены из~\eqref{loss1}, если зафиксировать
общее число заявок в~системе в~начальный момент функционирования.
Так, если изначально система пуста,
для нахождения $\{ p_i^-(1), \ i \hm\ge 1\}$ достаточно положить 
в~\eqref{loss1}
$n\hm=0$ и~$p_i^-(0)\hm=a \overline{a}^{i-1}$, $i \hm\ge 1$. Устремляя 
в~\eqref{loss1} $n \hm\rightarrow \infty$,
получаем систему уравнений для стационарных вероятностей~$p_i^-$, $i 
\hm\ge 1$,
решение которой при сделанных предположениях
о~распределениях $\{ b_i, i \hm\ge 0\}$ и~$\{ l_i, i \hm\ge 1\}$
(см.\ разд.~2) может быть найдено чис\-лен\-но.

Перейдем к~нахождению распределений $\{h_{t},\linebreak t \hm\ge 1\}$ 
и~$\{h_{t_1,t_2}, t_1,t_2 \hm\ge 1\}$.
Обозначим через $h^{(1)}_{t,i,j}$, $t,i,j \hm\ge 1$, условную вероятность 
того, что
очередная потеря произойдет через~$t$~тактов, при условии что изначально
в~системе находится~$i$~заявок и~остаточное время обслуживания
заявки на приборе равно~$j$~тактам. Положим 

\noindent
$$
h^{(1)}_{t,i}\hm=\sum\limits_{j=1}^\infty b_j h^{(1)}_{t,i,j}\,.
$$
Воспользовавшись формулой полной вероятности, находим

\vspace*{-2pt}

\noindent
\begin{align}
\label{loss2}
h^{(1)}_{t,1,j}&=
{\mathbf{1}_{(1 \le j \le t-1)}}
\sum\limits_{k=0}^{t-j-1} \overline{a}^{k+j-1}
\sum\limits_{m=1}^{\infty} a l_m h^{(1)}_{t-j-k,m}
+ {}\notag\\[6pt]
&\hspace*{25mm}{}+{\mathbf{1}_{(j \ge t+1)}}
\overline{a}^{t-1} a\,, \enskip j\ge 1\,;\\[9pt]
\label{loss3}
h^{(1)}_{t,i,j}&=
{\mathbf{1}_{(1 \le j \le t-1)}}
\overline{a}^{j} h^{(1)}_{t-j,i-1}
+
{\mathbf{1}_{(j \ge t)}}
\overline{a}^{t-1} a\,,\notag\\[6pt]
&\hspace*{41mm}\enskip i \ge 2\,, \enskip j\ge 1\,,
\end{align}

\vspace*{-2pt}

\noindent где ${\mathbf{1}_{(A)}}$~--- индикатор множества~$A$.
Соотношения~\eqref{loss2} и~\eqref{loss3} позволяют
последовательно по~$t$, начиная с~$t\hm=1$, определять\footnote{Расчет 
необходимо
вести в~следующем порядке: $h^{(1)}_{t,1,1}, h^{(1)}_{t,1,2},\dots$,
$h^{(1)}_{t,1},h^{(1)}_{t,2},\dots$} вероятности $h^{(1)}_{t,i,j}$
и~$h^{(1)}_{t,i}$. Усредняя $h^{(1)}_{t,i}$ по распределению $\{ p_i^-, i 
\hm\ge 1\}$, получаем
выражение для стационарного распределения длины
интервала между последовательными потерями:

\noindent
\begin{equation}
\label{loss4}
h_{t}
=\sum\limits_{i=1}^{\infty} h^{(1)}_{t,i} p_i^-, \enskip t \ge 1\,.
\end{equation}

\vspace*{-2pt}

\noindent Формулы для совместного стационарного распределения длин~$k$, 
$k\hm\ge2$, последовательных
интервалов между потерями могут быть получены аналогичным образом. 
Остановимся на случае $k\hm=2$. Обозначим через $h^{(2)}_{t_1,t_2,i}$, 
$t_1,t_2,i \hm\ge 1$, условную вероятность того, что
очередная потеря произойдет через~$t_1$ тактов, а~последующая~--- 
через~$t_2$ так-\linebreak\vspace*{-12pt}

\columnbreak

\noindent
тов, при условии что сразу после очередной потери 
в~сис\-те\-ме находится~$i$~заявок. Вводя обозначение

\vspace*{2pt}

\noindent
$$
\Delta_{t,i,j}=
h^{(1)}_{t,i,j-1} \sum\limits_{m=1}^{i} a l_m 
+\sum\limits_{m=i+1}^{\infty} a l_m
h^{(1)}_{t,m}
$$

\noindent  
и применяя формулу полной вероятности, получаем, что
$h^{(2)}_{t_1,t_2,i}$ могут быть рассчитаны на основе
$h^{(1)}_{t,i}$ рекуррентно по следующим формулам:

\vspace*{-3pt}

\noindent
\begin{align*}
%\label{loss5}
h^{(2)}_{t_1,t_2,1} &=
\sum\limits_{j=1}^{t_1-1}b_j
\!\sum\limits_{k=0}^{t_1-j-1}\!
\overline{a}^{k+j-1}
\sum\limits_{m=1}^{\infty} a l_m
h^{(2)}_{t_1-j-k,t_2,m}+{}\\
&\hspace*{24mm}{}+\sum\limits_{j=t_1+1}^{\infty}
b_j \overline{a}^{t_1-1}
\Delta_{t_2,1,j-t_1+1}\,;\\
%\end{multline*}
%\begin{multline*}
%\label{loss6}
h^{(2)}_{t_1,t_2,i} &=
\sum\limits_{j=1}^{t_1-1} b_j
\overline{a}^{j-1} h^{(2)}_{t_1-j,t_2,i-1}+{}\\
&{} + b_{t_1}
\overline{a}^{t_1-1}
\sum\limits_{m=1}^{\infty} a l_m
h^{(1)}_{t_2,\max(m,i-1)}+{}\\
&\hspace*{24mm}{}+ \sum\limits_{j=t_1+1}^{\infty}
b_j \overline{a}^{t_1-1}
\Delta_{t_2,i,j-t_1+1}\,.
\end{align*}

\vspace*{-2pt}

\noindent Усредняя $h^{(2)}_{t_1,t_2,i}$, как в~\eqref{loss4}, получаем
безусловное распределение длин последовательных двух
интервалов между потерями в~стационарном режиме.
%Справедливости ради необходимо отметить, что вычисления по полученным формулам
%явлются очень трудоемкими.

\vspace*{-6pt}

\section{Заключение}

В связи с~найденным видом стационарного распределения встает вопрос 
о~точности вычислений.
При расчете~$P_0$ по~\eqref{sn4-1} нельзя сказать,
когда нужно оборвать вычисления, чтобы гарантировать заданную точность. 
Этот вопрос,
встающий особенно остро, когда распределение
$\{b_i, i \hm\ge 0\}$ имеет тяжелый хвост, требует дополнительных 
исследований.
Некоторым ориентиром на практике могут служить двусторонние
оценки для~$P_0$, например $e^{-\sum\nolimits_{i=1}^\infty c_i} \hm\le 
P_0 \hm\le 
(1+\sum\nolimits_{i=1}^\infty c_i)^{-1}$ (см.\ далее, 
например,~\cite{klam}). Полезными могут оказаться и~приближенные формулы 
(см., например,~(8) в~\cite{i1}). В~плане дальнейших исследований 
несомненный интерес
представляет обобщение использованного метода на несколько СМО, связанных
рассмотренной дисциплиной обслуживания, а~также снятие наложенных
на входящий поток ограничений.

\vspace*{-6pt}


{\small\frenchspacing
 {%\baselineskip=10.8pt
 %\addcontentsline{toc}{section}{References}
 \begin{thebibliography}{99}

\bibitem{i5} %1
\Au{Schroeder B., Harchol-Balter~M.}
Web servers under overload: How scheduling can help~//
ACM~T. Internet Techn., 2006. Vol.~6. Iss.~1. P.~20--52.

\bibitem{i4} %2
\Au{Pradhan S., Gupta~U.\,C.}
Modeling and analysis of an infinite-buffer batch-arrival
queue with batch-size-dependent service~// Perform. Evaluation, 2017. 
Vol.~108. P.~16--31.

\bibitem{i2} %3
\Au{Grosof I., Scully~Z., Harchol-Balter~M.}
SRPT for multiserver systems~// Perform. Evaluation, 2018. Vol.~127-128. 
P.~154--175.

\bibitem{i3} %4
\Au{Marin A., Mitrani~I., Elahi~B.\,M., Williamson~C.}
Control and optimization of the SRPT service policy by frequency 
scaling~// 
Conference (International) on Quantitative Evaluation of Systems~/ Eds. 
A.~McIver, A.~Horvath.~--- Lecture notes in computer science ser.~--- 
Springer, 2018. Vol.~11024. P.~257--272.

\bibitem{i6} %5
\Au{Scully Z., Harchol-Balter~M., Scheller-Wolf~A.}
Simple near-optimal scheduling for the $M/G/1$~//
SIGMETRICS Perform. Eval. Rev., 2019. Vol.~47. Iss.~2. P.~24--26.


\bibitem{i1} %6
\Au{Marin A., Rossi~S.} A~queueing model that works
only on biggest jobs~// European Workshop on Performance Engineering~/ 
Eds. M.~Gribaudo, M.~Iacono, T.~Phung-Duc, R.~Razumchik.~--- Lecture 
notes in computer science ser.~--- Springer, 2020. Vol.~12039. P.~118--132.

%Marin A., Rossi S. (2020) A Queueing Model that Works Only on the 
%Biggest Jobs. In: Gribaudo M., Iacono M., Phung-Duc  
%T., Razumchik R. (eds) Computer Performance Engineering. EPEW 2019. 
%Lecture Notes in Computer Science, vol 12039.  
%Springer, Cham. https://doi.org/10.1007/978-3-030-44411-2_8

\bibitem{i8} %7
\Au{Pittel~B.\,G.}
Closed exponential networks of queues with saturation: The Jackson-type 
stationary distribution and its asymptotic analysis~//
Math. Oper. Res., 1979. Vol. 4. Iss.~4. P.~357--378.

\bibitem{i7} %8
\Au{Balsamo~S., Harrison~P., Marin~A.}
A~unifying approach to product-forms in
networks with finite capacity constraints~//
SIGMETRICS Perform. Eval. Rev., 2010. Vol.~38. Iss.~1. P.~25--36.

\bibitem{n4} %9
\Au{Печинкин А.\,В.} Об одной
инвариантной системе массового обслуживания~//
Math.\ Operationsforsch.\ Statist.
Ser.\ Optimization, 1983. Vol.~14. No.\,3. P.~433--444.

%Pechinkin, A.V. 1983.
%Ob odnoy invariantnoy sisteme massovogo
%obsluzhivaniya
%[On an Invariant Queuing System].
%{\it Math.\ Operationsforsch.\ Statist.
%Ser.\ Optimization} 14(3): 433--444.


\bibitem{tata} %10
\Au{Таташев~А.\,Г.}
Многоканальная система массового обслуживания с~потерями кратчайших 
требований~// Автоматика и~телемеханика, 1991. №\,7. С.~187--189.


\bibitem{i10} %11
\Au{Милованова Т.\,А.} Система BMAP${/G/1}$ с~инверсионным порядком 
обслуживания и~вероятностным приоритетом~// Автоматика и~телемеханика, 2009. 
№\,5. С.~155--168.

%Milovanova, T. A. 2009. ${BMAP/G/1/\infty}$ system with last
%come first served probabilistic priority. \textit{Automat. Rem.
%Contr.} 70(5): 885--896.

\bibitem{i11}
\Au{Мейханаджян Л.\,А.}
Стационарные вероятности состояний в~системе обслуживания конечной 
емкости с~инверсионным порядком обслуживания и~обобщенным вероятностным 
приоритетом~// Информатика и~её применения, 2016. Т.~10. Вып.~2. С.~123--131.
%Meykhanadzhyan, L. A. 2016. Statsionarnyye veroyatnosti sostoyaniy v sisteme 
%obsluzhivaniya konechnoy emkosti s inversionnym poryadkom 
%obsluzhivaniya i obobshchennym veroyatnostnym prioritetom
%[Stationary Characteristics of the Finite Capacity Queueing System with 
%Inverse Service Order and Generalized Probabilistic Priority].
%\textit{Informatika i ee Primeneniya --- Inform. Appl.} 10(62): 123--131.

\bibitem{i12}
\Au{Бочаров П.\,П., Гаврилов~Е.\,В., Печинкин~А.\,В.}
Экспоненциальная сеть массового обслуживания с~зависимым обслуживанием, 
отрицательными заявками и~изменением типа заявок~// Автоматика и~телемеханика, 
2004. №\,7. С.~35--59.

%P. P. Bocharov, E. V. Gavrilov, A. V. Pechinkin, ``Exponential queuing 
%network with dependent servicing, negative  
%customers, and modification of the customer type'', Autom. Remote 
%Control, 65:7 (2004), 1066--1088.

\bibitem{distime}
\Au{Печинкин А.\,В., Разумчик~Р.\,В.}
Системы массового обслуживания в~дискретном времени.~--- M.: Физматлит, 
2018. 432~с.

%Pechinkin, A. V., and R. V. Razumchik. 2018.
%\textit{Sistemy massovogo obsluzhivaniya v diskretnom vremeni}
%[Discrete Time Queuing Systems]. Moscow: Fizmatlit. 432 p.

\bibitem{nobel}
\Au{Nobel R.}
Retrial queueing models in discrete time: A~short survey of some late 
arrival models~//
Ann. Oper. Res., 2015. Vol. 247. Iss.~1. P.~37--63.

\bibitem{kry}
\Au{Канторович Л.\,В., Крылов~В.\,И.} Приближенные методы высшего 
анализа.~--- 5-е изд.~--- М.--Л.: Физматлит, 1962. 708~с.

\bibitem{wil}
\Au{Shivakumar P.\,N., Williams~J.\,J.}
An iterative method with trunction for infinite linear systems~// J.~Comput. Appl. Math., 1988.
Vol.~24. P.~199--207.

\bibitem{Car}
\Au{Carmichael R.\,D.} On non-homogeneous equations with an infinite 
number of variables~// Am. J.~Math., 1914. Vol.~36. Iss.~1. P.~13--20.

\bibitem{klam}
\Au{Klamkin M.\,S., Newman~D.\,J.}
Extensions of the Weierstrass product inequalities~//
Math. Mag., 1970. Vol.~43. Iss.~3. P.~137--141.
\end{thebibliography}

 }
 }

\end{multicols}

\vspace*{-6pt}

\hfill{\small\textit{Поступила в~редакцию 15.10.20}}

\vspace*{8pt}

%\pagebreak

%\newpage

%\vspace*{-28pt}

\hrule

\vspace*{2pt}

\hrule

%\vspace*{-2pt}

\def\tit{STATIONARY CHARACTERISTICS\\ 
OF~DISCRETE-TIME~Geo$/G/1/\infty$ QUEUE\\ 
WITH~BATCH ARRIVALS AND~ONE QUEUE SKIPPING POLICY}


\def\titkol{Stationary characteristics of discrete-time Geo$/G/1/\infty$ 
queue with~batch arrivals and~one queue skipping policy}


\def\aut{S.\,I.~Matyushenko$^1$ and R.\,V.~Razumchik$^2$}

\def\autkol{S.\,I.~Matyushenko and R.\,V.~Razumchik}

\titel{\tit}{\aut}{\autkol}{\titkol}

\vspace*{-11pt}


\noindent
$^1$Peoples' Friendship University of Russia (RUDN University), 
6~Miklukho-Maklaya Str., Moscow 117198,\\
$\hphantom{^1}$Russian Federation

\noindent
$^2$Institute of Informatics Problems, Federal Research Center 
``Computer Science and Control'' of the Russian\\
$\hphantom{^1}$Academy of Sciences, 
44-2~Vavilov Str., Moscow 119333, Russian Federation

\def\leftfootline{\small{\textbf{\thepage}
\hfill INFORMATIKA I EE PRIMENENIYA~--- INFORMATICS AND
APPLICATIONS\ \ \ 2020\ \ \ volume~14\ \ \ issue\ 4}
}%
 \def\rightfootline{\small{INFORMATIKA I EE PRIMENENIYA~---
INFORMATICS AND APPLICATIONS\ \ \ 2020\ \ \ volume~14\ \ \ issue\ 4
\hfill \textbf{\thepage}}}

\vspace*{3pt} 

\Abste{
Consideration is given to the discrete-time single-server system 
with one queue of infinite capacity and the geometric (Bernoulli) 
input flow. Customers are homogeneous, arrive in batches, 
and are served one by one in FIFO (first in, first out) manner. 
The sizes of arriving batches as well as the service times are assumed 
to be independent and identically distributed random variables with arbitrary discrete distributions.
The queue skipping policy is implemented in the system:
upon arrival of a~batch, its size is compared with the current
total number of customers in the system. If the size of the batch 
is larger than the system content, all customers residing in the\linebreak\vspace*{-12pt}}

\Abstend{system
(including the one in server) are lost and the arrived batch enters the 
system; otherwise, the new batch leaves the system having no effect on it. 
Main stationary system performance characteristics, including 
those of the flow of lost customers, are obtained.}


\KWE{discrete-time; queueing system; batch arrivals; queue skipping 
policy}


\DOI{10.14357/19922264200404} 

\vspace*{-16pt}

\Ack
\noindent
The reported 
study was funded by RFBR (project number 20-07-00804) and conducted
in accordance with the Program of Moscow Center for Fundamental and 
Applied Mathematics.


%\vspace*{6pt}

 \begin{multicols}{2}

\renewcommand{\bibname}{\protect\rmfamily References}
%\renewcommand{\bibname}{\large\protect\rm References}

{\small\frenchspacing
 {%\baselineskip=10.8pt
 \addcontentsline{toc}{section}{References}
 \begin{thebibliography}{99}

\bibitem{i5-1} %1
\Aue{Schroeder, B., and M.~Harchol-Balter.} 2006. 
Web servers under overload: How scheduling can help.
\textit{ACM~T. Internet Techn.} 6(1):20--52.

\bibitem{i4-1}%2
\Aue{Pradhan, S., and U.\,C.~Gupta.} 2017.
Modeling and analysis of an infinite-buffer batch-arrival
queue with batch-size-dependent service.
\textit{Perform. Evaluation} 108:16--31.

\bibitem{i2-1} %3
\Aue{Grosof, I., Z.~Scully, and M.~Harchol-Balter.} 2018.
SRPT for multiserver systems. \textit{Perform. Evaluation} 127-128:154--175.

\bibitem{i3-1} %4
\Aue{Marin, A., I. Mitrani, B.\,M.~Elahi, and C.~Williamson}. 2018.
Control and optimization of the SRPT service policy by frequency scaling.
\textit{Conference (International) on Quantitative Evaluation of 
Systems}. 
Eds. A.~McIver, and A.~Horvath. Lecture notes in computer science ser. 
Springer. 11024:257--272.

\bibitem{i6-1}
\Aue{Scully, Z., M.~Harchol-Balter, and A.~Scheller-Wolf}. 2019. 
Simple near-optimal scheduling for the $M/G/1$.
\textit{\mbox{SIGMETRICS} Perform. Eval. Rev.} 47(2):24--26.

\bibitem{i1-1}
\Aue{Marin, A., and S.~Rossi.} 2020. A~queueing model that works
only on biggest jobs. \textit{European Workshop on Performance 
Engineering}.
Eds. M.~Gribaudo, M.~Iacono, T.~Phung-Duc, and R.~Razumchik.
Lecture notes in computer science ser. Springer. 12039:118--132.

\bibitem{i8-1} %7
\Aue{Pittel, B.\,G.} 1979.
Closed exponential networks of queues with saturation: The Jackson-type 
stationary distribution and its asymptotic analysis. 
\textit{Math. Oper. Res.} 4(4):357--378.

\bibitem{i7-1} %8
\Aue{Balsamo, S., P.~Harrison, and A.~Marin.} 2010. 
A~unifying approach to product-forms in
networks with finite capacity constraints.
\textit{SIGMETRICS Perform. Eval. Rev.} 38(1):25--36.

\bibitem{n4-1} 
\Aue{Pechinkin, A.\,V.} 1983.
Ob odnoy invariantnoy sisteme massovogo
obsluzhivaniya [On an invariant queuing system]. \textit{Math.\ 
Operationsforsch.\ Statist. Ser.\ Optimization} 14(3):433--444.

\bibitem{tata-1}
\Aue{Tatashev, A.\,G.} 1991.
A~queueing system with invariant discipline.
\textit{Automat. Rem. Contr.} 52(7):1034--1037.

\bibitem{i10-1}
\Aue{Milovanova, T.\,A.} 2009. BMAP${/G/1/\infty}$ system with last
come first served probabilistic priority. \textit{Automat. Rem.
Contr.} 70(5):885--896.

\bibitem{i11-1}
\Aue{Meykhanadzhyan, L.\,A.} 2016. Statsionarnye veroyatno\-sti sostoyaniy 
v~sisteme obsluzhivaniya konechnoy emkosti s~inversionnym poryadkom 
obsluzhivaniya i~obobshchennym veroyatnostnym prioritetom
[Stationary characteristics of the finite capacity queueing system with 
inverse service order and generalized probabilistic priority]. 
\textit{Informatika i~ee Primeneniya --- Inform. Appl.} 10(62):123--131.

\bibitem{i12-1}
\Aue{Bocharov, P. P., E.\,V.~Gavrilov, and A.\,V.~Pechinkin.} 2004. 
Exponential queuing network with dependent servicing, negative customers, 
and modification of the customer type.
\textit{Automat. Rem. Contr.} 65(7):1066--1088.

\bibitem{distime-1}
\Aue{Pechinkin, A.\,V., and R.\,V.~Razumchik.} 2018.
\textit{Sistemy massovogo obsluzhivaniya v~diskretnom vremeni}
[Discrete time queuing systems]. Moscow: Fizmatlit. 432~p.

\bibitem{nobel-1}
\Aue{Nobel, R.} 2015.
Retrial queueing models in discrete time: A~short survey of some late 
arrival models.
\textit{Ann. Oper. Res}. 247(1):37--63.

\bibitem{kry-1}
\Aue{Kantorovich, L.\,V., and V.\,I.~Krylov}. 1962.
\textit{Priblizhennye metody vysshego analiza}.
Moscow--Saint-Petersburg: Fizmatlit. 708~p. 

\bibitem{wil-1}
\Aue{Shivakumar, P.\,N., and J.\,J.~Williams}. 1988.
An iterative method with trunction for infinite linear systems. 
\textit{J.~Comput. Appl. Math.}
24:199--207.

\bibitem{Car-1}
\Aue{Carmichael, R.\,D.} 1914.
On non-homogeneous equations with an infinite number of variables.
\textit{Am. J.~Math}. 36(1):13--20.

\bibitem{klam-1}
\Aue{Klamkin, M.\,S., and D.\,J.~Newman.} 1970. 
Extensions of the Weierstrass product inequalities.
\textit{Math. Mag.} 43(3):137--141.
\end{thebibliography}

 }
 }

\end{multicols}

\vspace*{-6pt}

\hfill{\small\textit{Received October 15, 2020}}

%\pagebreak

\vspace*{-20pt}


\Contr

\vspace*{-4pt}

\noindent
\textbf{Matyushenko Sergey I.} (b.\ 1963)~---
Candidate of Science (PhD) in physics and mathematics, associate 
professor,
Department of Applied Informatics and Probability Theory,
Peoples' Friendship University of Russia (RUDN University), 
6~Miklukho-Maklaya Str., Moscow 117198, Russian Federation; 
\mbox{matyushenko\_si@pfur.ru}

\vspace*{3pt}

\noindent
\textbf{Razumchik Rostislav V.} (b.\ 1984)~---
Candidate of Science (PhD) in physics and mathematics, leading scientist,
Institute of Informatics Problems, Federal Research Center ``Computer 
Science and Control'' of the Russian Academy of Sciences, 44-2~Vavilov 
Str., Moscow 119333, Russian Federation; \mbox{rrazumchik@ipiran.ru}

\label{end\stat}

\renewcommand{\bibname}{\protect\rm Литература}           %04
\def\stat{kor-kor}



\def\tit{МОДИФИЦИРОВАННЫЙ СЕТОЧНЫЙ МЕТОД РАЗДЕЛЕНИЯ ДИСПЕРСИОННО-СДВИГОВЫХ
СМЕСЕЙ НОРМАЛЬНЫХ ЗАКОНОВ$^*$}



\def\titkol{Модифицированный сеточный метод разделения дисперсионно-сдвиговых
смесей нормальных законов}

\def\aut{В.\,Ю.~Королев$^1$,  А.\,Ю.~Корчагин$^2$}

\def\autkol{В.\,Ю.~Королев,  А.\,Ю.~Корчагин}

\titel{\tit}{\aut}{\autkol}{\titkol}

{\renewcommand{\thefootnote}{\fnsymbol{footnote}} \footnotetext[1]
{Работа поддержана Российским научным фондом (проект 14-11-00364).}}


\renewcommand{\thefootnote}{\arabic{footnote}}
\footnotetext[1]{Факультет
вычислительной математики и кибернетики Московского государственного
университета им.\ М.\,В.~Ломоносова; Институт проблем информатики
Российской академии наук; victoryukorolev@yandex.ru}
\footnotetext[2]{Факультет вычислительной математики и кибернетики
Московского государственного университета им.\ М.\,В.~Ломоносова;
sasha.korchagin@gmail.com}

%\vspace*{2pt}



\Abst{Описывается модифицированный двухэтапный
сеточный метод разделения дис\-пер\-си\-он\-но-сдви\-го\-вых смесей нормальных
законов, представляющий собой альтернативу чистому ЕМ (expectation-maximization)
ал\-го\-рит\-му. На
первом этапе этого алгоритма строится дискретная аппроксимация для
смешивающего распределения, на втором этапе подбирается абсолютно
непрерывное распределение из заранее заданного семейства, например,
обобщенных обратных гауссовских законов, ближайшее к~дискретному
распределению, полученному на первом этапе. Обсуждаются вопросы
сходимости этого двухэтапного алгоритма. Доказана монотонность
сеточного итерационного метода, используемого на первом этапе.
Подробно обсуждается вопрос оптимального выбора параметров метода,
прежде всего сетки, накидываемой на носитель смешивающего
распределения. С~этой целью предложены статистические оценки
квантилей смешивающего распределения. Эффективность метода
иллюстрируется примерами конкретных вычислений оценок параметров
обобщенных гиперболических распределений.}

\KW{смесь распределений вероятностей;
дис\-пер\-си\-он\-но-сдви\-го\-вая смесь нормальных законов; обобщенное
гиперболическое распределение; ЕМ-ал\-го\-ритм; сеточный метод
разделения смесей}

\vspace*{1pt}

%\vspace*{2pt}

\DOI{10.14357/19922264140402}


\vskip 12pt plus 9pt minus 6pt

\thispagestyle{headings}

\begin{multicols}{2}

\label{st\stat}

\section{Введение}

При {\it практическом} решении задачи моделирования и исследования
волатильности (изменчивости) хаотических стохастических процессов
ключевым этапом является статистическое разделение смесей
вероятностных распределений. Задача разделения смесей~---
статистического оценивания параметров смесей вероятностных
распределений~--- в~деталях разобрана, например, в~книге~\cite{k2011}.

Для решения задачи разделения смесей вероятностных распределений
традиционно используются итерационные процедуры типа ЕМ-ал\-го\-рит\-ма.
К~сожалению, классический ЕМ-ал\-го\-ритм обладает рядом серьезных
недостатков при его применении к~смесям нормальных законов, а~именно:
он демонстрирует крайнюю неустойчивость по отношению к~исходным
данным и~начальным приближениям.

Для преодоления этих недостатков
предложено много модификаций ЕМ-ал\-го\-рит\-ма (см., например,~\cite{k2011}).
Вместе с тем в~указанной книге предложен и~исследован
принципиально новый~--- сеточный~--- метод приближенного решения
задачи разделения смесей. В~работе~\cite{n2013} подробно исследованы
вопросы сходимости сеточных методов разделения смесей.

В соответствии с подходом к~статистическому анализу хаотических
стохастических процессов, в~частности к~решению задачи декомпозиции
волатильности таких процессов, развитом в~книге~\cite{k2011},
в~общем случае на практике приходится решать задачу разделения
конечных смесей нормальных законов с~произвольно большим числом
неизвестных параметров (параметров компонент и~их весов).
И~хотя в~большинстве приложений возникают смеси не более чем с~пятью--семью
компонентами, даже при использовании таких смесей, скажем, в~задачах
анализа и~прогнозирования финансовых рисков приходится моделировать
траекторию движения точки в~пространствах, размерность которых
соответственно лежит в~пределах от~14 (для пятикомпонентных смесей)
до~20 (для семикомпонентных смесей), что существенно увеличивает
вычислительные и~временн$\acute{\mbox{ы}}$е ресурсы, необходимые для практического
решения указанных задач.

Поскольку во многих ситуациях (например,
при прогнозировании на основе высокочастотных данных) эти задачи
необходимо решать в~режиме, близком к~реальному времени, для
создания эффективных методов статистического анализа на основе
смешанных моделей на первый план выходит проб\-ле\-ма снижения
размерности решаемой задачи, т.\,е.\ параметрического пространства.

Одним из возможных подходов к~снижению размерности является
априорное сужение классов допусти\-мых смесей. К~примеру, при решении
многих задач, связанных с~анализом процессов атмосферной или
плазменной турбулентности, а~так\-же процессов, описывающих эволюцию
различных финансовых индексов, высочайшую адекватность
продемонстрировали модели, основанные на дис\-пер\-си\-он\-но-сдви\-го\-вых
смесях нормальных законов. Класс таких смесей очень обширен
и,~в~част\-ности, включает в~себя обобщенные гиперболические распределения,
которые были введены О.-Е.~Барн\-дорфф-Ниль\-се\-ном в~1977--1978~гг.\ как
класс специальных сдвиг-мас\-штаб\-ных смесей нормальных законов~\cite{BN1977, BN1978}.
Пусть $\alpha\hm\in\r$, $\beta\hm\in\r$. Если
функцию распределения обобщенного гиперболического закона
с~параметрами~$\alpha$, $\beta$, $\nu$, $\mu$, $\lambda$ обозначить
$P_{GH}(x;\alpha,\beta,\nu,\mu,\lambda)$, то по определению
\begin{multline}
P_{GH}(x;\alpha,\beta,\nu,\mu,\lambda)={}\\
{}=
\int\limits_{0}^{\infty}\Phi\left(\fr{x-\beta-\alpha
z}{\sqrt{z}}\right)\,p_{GIG}(z;\nu,\mu,\lambda)\,dz\,,\\
x\in\r\,,
\label{e1-kor}
\end{multline}
где $\Phi(x)$~--- стандартная нормальная функция распределения:
$$
\Phi(x)=\int\limits_{-\infty}^{x}\varphi(z)\,dz\,,\enskip
\varphi(x)=\fr{1}{\sqrt{2\pi}}e^{-x^2/2}\,,\enskip  x\in\mathbb{R}\,;
$$
$p_{GIG}(x;\nu,\mu,\lambda)$~--- плот\-ность обобщенного обратного
гауссовского распределения:
\begin{multline*}
p_{GIG}(x;\nu,\mu,\lambda)={}\\
{}=\fr{\lambda^{\nu/2}}{2\mu^{\nu/2}
K_{\nu}\left(\sqrt{\mu\lambda}\right)}\,
x^{\nu-1}\exp\left\{-\fr{1}{2}\left(\fr{\mu}{x}+\lambda
x\right)\right\}\,,\\ x>0\,.
\end{multline*}
Здесь $\nu\in\r$;
$$
\begin{array}{lll}
\mu>0\,, & \lambda\geqslant0\,, & \mbox{если }\nu<0\,;\\[6pt]
\mu>0\,, & \lambda>0\,, & \mbox{если }\nu=0\,;\\[6pt]
\mu\geqslant0\,, & \lambda>0\,, & \mbox{если }\nu>0\,;
\end{array}
$$
$K_{\nu}(z)$~--- модифицированная бесселева функция третьего рода
порядка~$\nu$:

\noindent
\begin{multline*}
K_{\nu}(z)=\fr{1}{2}\int\limits_{0}^{\infty}y^{\nu-1}\exp
\left\{-\fr{z}{2}\left(y+\fr{1}{y}\right)\right\}\,dy\,,\\
z\in\mathbb{C}\,,\enskip \mathrm{Re}\,z>0\,.
\end{multline*}
Обратим внимание, что в~(1) смешивание происходит одновременно и~по
параметру сдвига, и~по параметру масштаба, но так как эти параметры
в~(1)  связаны жесткой зависимостью, так что параметр сдвига
смешиваемого распределения пропорционален его дисперсии, то
фактически смесь~(1) является {\it однопараметрической} и~поэтому
называется {\it дис\-пер\-си\-он\-но-сдви\-го\-вой} (см., например,~\cite{BN1982}).

Другим примером дис\-пер\-си\-он\-но-сдви\-го\-вых смесей нормальных законов
являются обобщенные дисперсионные гам\-ма-рас\-пре\-де\-ле\-ния, в~которых
смешивающими являются обобщенные гам\-ма-рас\-пре\-де\-ле\-ния~\cite{ks2012, zk2013}.

В указанных семействах смесей число неизвестных параметров равно
пяти или шести (если\linebreak учитывать неслучайный сдвиг). Вместе
с~тем у~подоб\-ных моделей имеются довольно серьезные тео\-ре\-ти\-че\-ские
обоснования: в~работах~\cite{zk2013, k2013} показано, что указанные
модели являются асимптотическими аппроксимациями в~простой
предельной схеме случайного суммирования и~потому могут успешно
применяться для анализа процессов типа остановленных случайных
блужданий. Эти выводы подтверждены статистическим анализом
вы\-со\-ко\-час\-тот\-ных финансовых данных, в~результате которого выявлен
синхронизированный характер изменения интенсивностей потоков заявок
в~сис\-те\-мах электронных торгов, что естественно приводит к~синхронизированному
поведению па\-ра\-мет\-ров сдвига и~диффузии в~соответствующих моделях вида смесей
нормальных законов~\cite{kckg2013}.

\section{Описание моди\-фи\-ци\-ро\-ван\-но\-го
сеточного ме\-то\-да разделения дисперсионно-сдвиговых смесей
нормальных законов и~его свойства}

Оказывается, что сеточные методы разделения смесей довольно
эффективны не только при разделении конечных смесей нормальных
законов, но и~при разделении произвольных дис\-пер\-си\-он\-но-сдви\-го\-вых
смесей нормальных законов. Поясним сказанное на примере задачи
оценивания па\-ра\-мет\-ров обобщенных гиперболических распределений.

Для решения задачи оценивания параметров обобщенных гиперболических
распределений традиционно используется метод, предложенный в~статье~\cite{p2004}
и~по сути являющийся классическим ЕМ-ал\-го\-рит\-мом,
приспособленным к~конкретной задаче, и,~соответственно, наследующий
присущие ЕМ-ал\-го\-рит\-мам недостатки.

Рассмотрим следующий альтернативный двухэтапный метод. На первом
этапе на поло\-жи\-тельной полупрямой выделим основную часть носителя
смешивающего распределения, т.\,е.\ \mbox{ограниченный} интервал,
вероятность которого, вычисленная в~соответствии со смешивающим
распределением, практически равна единице. На этот интервал накинем
конечную сетку, содержащую, возможно, очень много {\it известных}
узлов $u_1,\ldots,u_K$. Считая параметр сдвига~$\beta$ равным нулю,
приблизим искомое обобщенное гиперболическое распределение конечной
смесью нормальных законов:

\noindent
\begin{multline}
P_{GH}(x;\,\alpha,0,\nu,\mu,\lambda)\approx{}\\
{}\approx \sum\limits_{i=1}^K
p_i\Phi\left(\fr{x-\alpha u_i}{\sqrt{u_i}}\right)\,,\enskip
x\in\mathbb{R}\,.\label{e2-kor}
\end{multline}
В смеси, стоящей в~правой части соотношения~(2), неизвестными
являются только параметры $p_1,\ldots,p_{K-1}$ и~$\alpha$. Пусть
$x_1,\ldots,x_n$~--- анализируемая выборка значений случайной
величины с~оцениваемым обобщенным гиперболическим распределением.
Итерационный процесс, определяющий сеточный ЕМ-ал\-го\-ритм для данной
задачи, задается следующим образом. Пусть
$p_1^{(m)},\ldots,p_{K-1}^{(m)}$ и~$\alpha^{(m)}$~--- оценки параметров
$p_1,\ldots,p_{K-1}$ и~$\alpha$ на $m$-й итерации,
$p_K^{(m)}\hm=1\hm-p_1^{(m)}-\cdots-p_{K-1}^{(m)}$. Обозначим

\noindent
\begin{align*}
\varphi_{ij}^{(m)}&=\fr{1}{\sqrt{u_i}}\varphi\left(\fr{x_j-\alpha^{(m)}u_i}{\sqrt{u_i}}\right)\,;
\\
g_{ij}^{(m)}&=\fr{p_i^{(m)}\varphi_{ij}^{(m)}}{\sum\limits_{r=1}^K
p_r^{(m)}\varphi_{rj}^{(m)}}\,,\\
&\hspace*{14mm}i=1,\ldots,K\,;\enskip j=1,\ldots,n\,.
\end{align*}
Тогда, используя стандартные рассуждения, определяющие
вычислительные формулы EM-ал\-го\-рит\-ма для параметров конечной смеси
нормальных законов (см, например,~[1, разд.~5.3.7--5.3.8]),
следует положить

\noindent
\begin{equation}
p_i^{(m+1)}=\fr{1}{n}\sum\limits_{j=1}^n g_{ij}^{(m)}\,, \enskip
i=1,\ldots,K\,.\label{e3-kor}
\end{equation}
Обозначим $\overline{x}=(1/n)\sum\limits_{j=1}^nx_j$. Используя
соотношение~(5.3.24) в~\cite{k2011}, с~учетом очевидного равенства
$\sum\limits_{i=1}^K g_{ij}^{(m)}\hm=1$ можно заметить, что уточненная
оценка параметра~$\alpha$ имеет вид:

\columnbreak

\noindent
\begin{equation}
\alpha^{(m+1)}=\fr{\overline{x}}{\sum\limits_{i=1}^K u_ip_i^{(m+1)}}\,,
\label{e4-kor}
\end{equation}
т.\,е.\ равна отношению генерального выборочного среднего и~текущего
эмпирического среднего смешивающего распределения, что вполне
согласуется с~тем, что в~соответствии с~приводимым ниже соотношением~(\ref{e5-kor})
в~данном случае ${\sf E}X\hm=\alpha{\sf E}U$.

В силу монотонности классического ЕМ-ал\-го\-рит\-ма справедливо следующее
утверждение.

\smallskip

\noindent
\textbf{Теорема~1.} {\it Пусть узлы $u_1,\ldots,u_K$ сетки различны,
неотрицательны и~известны. Тогда итерационный процесс $(3)$--$(4)$
является монотонным, т.\,е.\ каждая его итерация не уменьшает
целевую сеточную функцию правдоподобия}
\begin{multline*}
L(p_1,\ldots,p_K,\alpha;x_1,\ldots,x_n)={}\\
{}=
\prod\nolimits_{j=1}^n\left[\sum\nolimits_{i=1}^K
\fr{p_i}{\sqrt{u_i}}\,\varphi\left(\fr{x_j-\alpha^{(m)}u_i}{\sqrt{u_i}}\right)\right].
\end{multline*}

\smallskip

\noindent
\textbf{Замечание~1.} В~разд.~5.7.4 книги~\cite{k2011} показано, что
при каждом фиксированном значении параметра~$\alpha$ сеточная
функция правдоподобия\linebreak
$L(p_1,\ldots,p_{K-1},\alpha;\,x_1,\ldots,x_n)$ вогнута по
аргументам $p_1,\ldots,p_{K-1}$. Поэтому на каждом шаге
итерационного процесса вместо соотношения~(3) можно\linebreak использо\-вать
любой более быстрый алгоритм максимизации функции
$L(p_1,\ldots,p_{K-1},\alpha^{(m)};\,x_1,\ldots$\linebreak $\ldots,x_n)$ по переменным
$p_1,\ldots,p_{K-1}$. Например, оценки весов $p_1,\ldots,p_K$ можно
искать методом условного градиента~\cite{k2011, kn2010}.

\smallskip

Таким образом, на первом этапе получаются оценки параметра~$\alpha$
и~весов всех узлов~$u_i$ конечной сетки, накинутой на носитель
смешивающего обобщенного обратного гауссовского распределения
$P_{\mathrm{GIG}}(z;\,\nu,\mu,\lambda)$.

На втором этапе остается применить ка\-кой-ли\-бо стандартный метод
подгонки обобщенного обратного гауссовского распределения
$P_{\mathrm{GIG}}(z;\,\nu,\mu,\lambda)$ к~эмпирическим данным типа
гистограммы $(u_1, p_1),\ldots, (u_K, p_K)$. Например, параметры~$\nu$,
$\mu$ и~$\lambda$ можно оценить, минимизируя соответствующую
статистику хи-квад\-рат. Или же, например, можно решить задачу
наименьших квад\-ратов:
\begin{multline*}
(\nu^*,\mu^*,\lambda^*)={}\\
{}=\arg\min\limits_{\nu,\mu,\lambda}\sum\limits_{i=1}^K
\left[p_i- \!\!\!\!\!
\int\limits_{(1/2)\left(u_{i-1}+u_i\right)}^{(1/2)(u_i+u_{i+1})}\!\!\!\!\!\!\!\!\!\!\!\!\!\!\!
p_{GIG}(u;\,\nu,\mu,\lambda)\,du\right]^2,
\end{multline*}
где $u_0=0$; $u_{K+1}\hm=\infty$.

На практике хорошие результаты показал подход с решением задачи
наименьших квадратов. Для поиска параметров использовался алгоритм
ns2sol, описанный в~книге~\cite{DSch1983}. Указанный алгоритм
доступен во многих статистических пакетах, отличается высоким
быстродействием и~возможностью при желании задавать разумные
интервалы для поиска параметров.

%\vspace*{-9pt}

\section{О практическом выборе сетки
на~первом этапе моди\-фи\-ци\-ро\-ван\-но\-го
сеточного метода разделения дисперсионно-сдвиговых смесей нормальных
законов}

Естественно, что при использовании указанного двухэтапного метода
в~динамическом режиме крайне важным становится вопрос о~выборе
наиболее эффективных и~быстродействующих численных процедур и~их
параметров. В~частности, исключительную важность приобретает
правильный выбор сетки на первом этапе. Рассмотрим этот вопрос
подробнее.

Формально рассматриваемая задача выглядит так: по наблюдаемым
значениям $x_1,\ldots,x_n$ требуется построить статистическую оценку
верхней границы квантилей заданного порядка сме\-ши\-ва\-юще\-го закона так,
чтобы как можно точнее оценить носитель смешивающего распределения.

В дальнейшем будем считать, что $x_1,\ldots,x_n$~--- независимые
реализации случайной величины $X\hm=Y\sqrt{U}+\alpha U$, где $Y$~---
случайная величина со стандартным нормальным распределением, а~$U$~---
независимая от нее случайная величина с~обобщенным обратным
гауссовским распределением. Тогда, очевидно, распределение случайной
величины~$X$ имеет вид~(1). Предположим, что у~случайной величины~$U$
существуют моменты первых двух порядков. Тогда, как несложно видеть,
\begin{equation}
{\sf E}X={\sf E}Y\cdot{\sf E}\sqrt{U}+\alpha{\sf E}U=\alpha{\sf
E}U\,.\label{e5-kor}
\end{equation}
При этом по усиленному закону больших чисел с~вероятностью единица
$\overline x\hm\longrightarrow {\sf E}X$ $(n\hm\to\infty)$, так что при
больших~$n$ справедливо приближенное равенство ${\sf E}X\hm\approx\overline x$
и~с учетом~(\ref{e5-kor})
\begin{equation}
{\sf E}U\approx\fr{\overline x}{\alpha}\,.\label{e6-kor}
\end{equation}
Далее, очевидно,

\columnbreak

\noindent
\begin{multline}
{\sf E}X^2={\sf E}Y^2\cdot{\sf E}U+2\alpha{\sf E}X\cdot{\sf E}U^{3/2}+{}\\
{}+
\alpha^2{\sf E}U^2={\sf E}U+\alpha^2{\sf E}U^2\,.
\label{e7-kor}
\end{multline}

\noindent
Поэтому, обозначив
$$
m^2=\fr{1}{n}\sum\limits_{i=1}^nx_i^2\,,
$$
получаем приближенное равенство ${\sf E}X^2\hm\approx m^2$, так что
с~учетом~(\ref{e6-kor}) и~(\ref{e7-kor}) имеем:
\begin{equation}
{\sf E}U^2\approx\fr{1}{\alpha^2}\left(m^2-\fr{\overline
x}{\alpha}\right)\,.\label{e8-kor}
\end{equation}
Если параметр~$\alpha$ известен, то для определения верхней границы~$u^*$
сетки, накидываемой на носитель распределения случайной
величины~$U$, можно задать малое положительное число~$\varepsilon$
и~воспользоваться требованием
\begin{equation}
{\sf P}(U\geqslant u^*)\leqslant\varepsilon\,.\label{e9-kor}
\end{equation}
А~для гарантированного выполнения требования~(\ref{e9-kor}) можно использовать
неравенство Маркова:
$$
{\sf P}(U\geqslant u^*)\leqslant\fr{{\sf E}U^2}{(u^*)^2}\leqslant \varepsilon\,,
$$
откуда с учетом~(\ref{e8-kor})
$$
(u^*)^2\geqslant\fr{{\sf E}U^2}{\varepsilon}\approx
\fr{1}{\alpha^2\varepsilon}\left( m^2-\fr{\overline x}{\alpha}\right)
$$
или
\begin{equation}
u^*\approx\fr{1}{\alpha\sqrt{\varepsilon}}\sqrt{m^2-
\fr{\overline x}{\alpha}}\,.\label{e10-kor}
\end{equation}

\begin{figure*}[b] %fig1
\vspace*{1pt}
 \begin{center}
 \mbox{%
 \epsfxsize=161.718mm
 \epsfbox{kor-1.eps}
 }
 \end{center}
 \vspace*{-9pt}
\Caption{Примеры применения модифицированного двухэтапного сеточного
ЕМ-ал\-го\-рит\-ма для подгонки обобщенного гиперболического распределения
к искусственным данным, $\beta\hm=0$: (\textit{a})~$n\hm=1000$, $\alpha\hm=0{,}3$,
$\nu\hm=1{,}3$, $\mu\hm=1{,}6$, $\lambda\hm=0{,}2$;
(\textit{б})~$n\hm=1000$, $\alpha\hm=0{,}5$, $\nu\hm=1$, $\mu\hm=1$,
$\lambda\hm=3$;
(\textit{в})~$n\hm=1000$, $\alpha\hm=3$,
 $\nu\hm=1{,}3$, $\mu\hm=1{,}6$, $\lambda\hm=2$;
(\textit{г})~$n\hm=10\,000$,
$\alpha\hm=0{,}3$, $\nu\hm=1{,}3$, $\mu\hm=1{,}6$, $\lambda\hm=0{,}2$}
\end{figure*}


Если же параметр~$\alpha$, определяющий асим\-мет\-рию распределения
случайной величины~$X$, неизвестен, то можно воспользоваться
следующими рассуждениями. Обозначим
$$
q_n=\fr{1}{n}\sum\limits_{i=1}^n{\bf 1}(x_i<0)\,,
$$
где ${\bf 1}(A)$~--- индикаторная функция множества (события)~$A$.
При этом по усиленному закону больших чисел с~вероятностью единица
$q_n\hm\longrightarrow {\sf P}(X\hm<0)$ $(n\hm\to\infty)$, так что при
больших~$n$ справедливо приближенное равенство
\begin{equation}
q_n\approx{\sf P}(X<0)\,.\label{e11-kor}
\end{equation}
Но
\begin{multline}
{\sf P}(X<0)=\int\limits_{0}^{\infty}\Phi
\left(-\alpha\sqrt{u}\right) p_{\mathrm{GIG}}(u;\nu,\mu,\lambda)\,du={}\\
{}=
{\sf E}\Phi\left(-\alpha\sqrt{U}\right)\,.\label{e12-kor}
\end{multline}

\pagebreak

\noindent
Предположим сначала, что $q_n\hm<1/2$. Если~$n$ достаточно велико,
то можно с~большой степенью
 уверенности утверж\-дать, что тогда
$\overline x\hm>0$ и~$-\alpha\hm<0$, т.\,е.
 $\alpha\hm>0$ и,~стало быть, на
положительной полуоси значений аргумента~$u$ функция $\Phi(\alpha u)$
вогнута, т.\,е.\ выпукла вверх. Тогда из~(\ref{e11-kor}) и~(\ref{e12-kor}), дважды
применяя неравенство Иенсена, в~силу монотонности функции~$\Phi$
получаем:
\begin{multline}
1-q_n\approx 1-{\sf E}\Phi\left(-\alpha\sqrt{U}\right)=
          {\sf E}\Phi\left(\alpha\sqrt{U}\right)\leqslant{}\\
          {}\leqslant\Phi
          \left(\alpha{\sf E}\sqrt{U}\right)\leqslant
          \Phi\left(\alpha\sqrt{{\sf E}U}\right)\,.\label{e13-kor}
\end{multline}
Если теперь для $t\hm\in(0,1)$ символом~$v_t$ обозначить $t$-кван\-тиль
стандартного нормального закона, то из~(\ref{e13-kor}) и~(\ref{e6-kor}) вытекает
<<приближенное неравенство>>
$$
v_{1-q_n}\hm\leqslant \alpha\sqrt{{\sf E}U}\,,
$$
т.\,е.
$$
\alpha\geqslant\fr{v_{1-q_n}}{\sqrt{{\sf E}U}}\approx
\fr{v_{1-q_n}\sqrt{\alpha}}{\sqrt{\overline x}}\,,
$$
откуда получаем, что при достаточно больших~$n$
\begin{equation}
\alpha\geqslant\fr{v_{1-q_n}^2}{\overline x}\,.\label{e14-kor}
\end{equation}
Если теперь задать малое положительное число~$\varepsilon$, то
для определения верхней границы~$u^*$ сетки, накидываемой на
носитель распределения случайной величины~$U$, можно воспользоваться
требованием~(\ref{e9-kor}), для гарантированного выполнения которого
с~учетом~(\ref{e6-kor}) и~(\ref{e14-kor}) можно использовать неравенство Маркова:
$$
{\sf P}(U\geqslant u^*)\leqslant \fr{{\sf E}U}{u^*}\approx\fr{\overline
x}{\alpha u^*}\leqslant \fr{(\overline x)^2}{v_{1-q_n}^2 u^*}\leqslant
\varepsilon\,,
$$
откуда окончательно вытекает оценка
\begin{equation}
u^*\approx\fr{(\overline x)^2}{v_{1-q_n}^2 \varepsilon}\,.\label{e15-kor}
\end{equation}

\begin{figure*}[b] %fig2
\vspace*{18pt}
 \begin{center}
 \mbox{%
 \epsfxsize=162.433mm
 \epsfbox{kor-3.eps}
 }
 \end{center}
 \vspace*{-9pt}
\Caption{Примеры применения модифицированного двухэтапного
сеточного ЕМ-ал\-го\-рит\-ма для подгонки обобщенного гиперболического
распределения к~искусственным данным, $n=10\,000$, $\beta\hm=0$:
(\textit{а})~$\alpha\hm=0{,}3$,
$\nu\hm=2$, $\mu\hm=2$, $\lambda\hm=2{,}5$;
(\textit{б})~$\alpha\hm=0{,}5$,  $\nu\hm=1$, $\mu\hm=1$, $\lambda\hm=3$;
(\textit{в})~$\alpha\hm=0{,}8$,
$\nu\hm=1{,}3$, $\mu\hm=1{,}6$, $\lambda\hm=2$;
(\textit{г})~$\alpha\hm=1{,}3$, $\nu\hm=2$, $\mu\hm=2$, $\lambda\hm=2{,}5$}
\end{figure*}



В случае $q_n\hm\geqslant1/2$, если $n$ достаточно велико, то можно
с~большой степенью уверенности утверж\-дать, что $\overline x\hm\leqslant 0$
и~$-\alpha\hm\geqslant 0$, т.\,е.\ на положительной\linebreak\vspace*{-12pt}

\pagebreak

%\end{multicols}


%\begin{multicols}{2}

\noindent
 полуоси значений аргумента~$u$
функция $\Phi(-\alpha u)$ вогнута, т.\,е.\ выпукла вверх. Тогда
из~(\ref{e11-kor}) и~(\ref{e12-kor}), дважды применяя неравенство Иенсена, в~силу
монотонности функции~$\Phi$ получаем
$$
q_n\approx {\sf E}\Phi\left(-\alpha\sqrt{U}\right)\leqslant
\Phi\left(-\alpha\sqrt{{\sf E}U}\right)\,,
$$
откуда вытекает <<приближенное неравенство>> $v_{q_n}\hm \leqslant
-\alpha\sqrt{{\sf E}U}$,
т.\,е.
$$
-\alpha\geqslant\fr{v_{q_n}}{\sqrt{{\sf E}U}}\approx
\fr{v_{q_n}\sqrt{|\alpha|}}{\sqrt{|\overline x|}}
$$
и при достаточно больших~$n$
\begin{equation}
|\alpha|\geqslant\fr{v_{q_n}^2}{|\overline x|}\,.\label{e16-kor}
\end{equation}
Для определения верхней границы~$u^*$ сетки, накидываемой на
носитель распределения случайной величины~$U$, снова зададим малое
положительное число~$\varepsilon$ и~потребуем, чтобы было
справедливо условие~(\ref{e9-kor}), для гарантированного выполнения которого
с~учетом~(\ref{e6-kor}) и~(\ref{e16-kor}) используем неравенство Маркова и~тот факт, что
$\mathrm{sign}\, \overline x\hm=\mathrm{sign}\,\alpha$ при достаточно
больших~$n$:
\begin{multline}
{\sf P}(U\geqslant u^*)\leqslant \fr{{\sf E}U}{u^*}\approx
\fr{\overline x}{\alpha u^*}=
\fr{|\overline x|}{|\alpha| u^*} \leqslant{}\\
{}\leqslant
\fr{(\overline x)^2}{v_{q_n}^2 u^*}\leqslant
\varepsilon\,.\label{e17-kor}
\end{multline}
В силу симметричности нормального распределения $v_{t}\hm=-v_{1-t}$ для
любого $t\hm\in(0,1)$, поэтому $v_{q_n}^2\hm=v_{1-q_n}^2$ и~в~случае
$q_n\hm\geqslant1/2$ соотношение~(\ref{e17-kor}) снова приводит к~оценке~(\ref{e15-kor}).

Справедливости ради необходимо отметить, что оценки~(\ref{e10-kor}) и~(\ref{e15-kor})
являются завышенными, но они гарантируют, что
$(1-\varepsilon)$-почти-весь носитель распределения случайной
величины~$U$ будет лежать внутри интервала $[0, u^*]$.

\section{Результаты численных экспериментов}

Приводимые в~данном разделе графики иллюстрируют качество работы
модифицированного сеточного метода разделения дис\-пер\-си\-он\-но-сдви\-го\-вых
смесей нормальных законов на примере его\linebreak применения к~оцениванию
параметров обоб\-щенных гиперболических распределений с~ис\-поль\-зованием
указанного алгоритма выбора сетки\linebreak с~умеренным чис\-лом узлов $K\hm=40$.
Для вы\-чис\-ле\-ний использовались искусственно сгенерированные выборки
объемов $n\hm=1000$ и~$n\hm=10\,000$ с~разными наборами параметров, значения
которых указаны на рисунках. На рис.~1 и~2 изображены гистограммы
(серые столбики) и~графики
истинной плот\-ности (штриховые линии), промежуточной
оценки, полученной сеточным ЕМ-ал\-го\-рит\-мом (пунктирные линии)
и~итоговой оценки (непрерывные линии). На рис.~1 и~2 так\-же указаны
значения полученных оценок параметров. Как видно из приводимых
рисунков, параметры~$\alpha$ оцениваются очень точно. Точность
оценок остальных параметров удовлетворительная и~может быть повышена
за счет использования более частых сеток и~более чувствительных
критериев остановки ЕМ-ал\-го\-рит\-ма на первом этапе. Следует отметить,
что даже в~тех случаях, в~которых наблюдаются заметные расхождения
оценок параметров и~их точных значений, оценки самих плотностей
довольно \mbox{точны}.




{\small\frenchspacing
 {%\baselineskip=10.8pt
 \addcontentsline{toc}{section}{References}
 \begin{thebibliography}{99}
\bibitem{k2011}
\Au{Королев В.\,Ю.} Ве\-ро\-ят\-но\-ст\-но-ста\-ти\-сти\-че\-ские методы
декомпозиции волатильности хаотических процессов.~--- М.: Изд-во
Московского ун-та, 2011.

\bibitem{n2013}
\Au{Назаров А.\,Л.} Приближенные методы разделения смесей
вероятностных распределений: Дисс.\ \ldots\  канд. физ.-мат. наук.~--- М.:
МГУ им.\ М.\,В.~Ломоносова, 2013.

\bibitem{BN1977}
\Au{Barndorff-Nielsen~O.-E.} Exponentially decreasing distributions
for the logarithm of particle size~// Proc. Roy. Soc. Lond.~A,
1977. Vol.~353. P.~401--419.

\bibitem{BN1978}
\Au{Barndorff-Nielsen~O.-E.} Hyperbolic distributions and
distributions of hyperbolae~// Scand. J. Statist., 1978. Vol.~5.
P.~151--157.

\bibitem{BN1982}
\Au{Barndorff-Nielsen~O.-E., Kent~J., S\!{\!\ptb{\!\o}}\,rensen~M.} Normal
variance-mean mixtures and $z$-distributions~// Int. Statist. Rev.,
1982. Vol.~50. No.\,2. P.~145--159.

\bibitem{ks2012}
\Aue{Королев В.\,Ю., Соколов И.\,А.} Скошенные распределения
Стьюдента, дисперсионные гам\-ма-рас\-пре\-де\-ле\-ния и~их обобщения как
асимптотические аппроксимации~// Информатика и~её применения, 2012.
Т.~6. Вып.~1. С.~2--10.

\bibitem{zk2013}
\Au{Закс Л.\,М., Королев В.\,Ю.} Обобщенные дисперсионные
гам\-ма-рас\-пре\-де\-ле\-ния как предельные для случайных сумм~// Информатика
и её применения, 2013. Т.~7. Вып.~1. С.~105--115.

\bibitem{k2013}
\Au{Королев В.\,Ю.} Обобщенные гиперболические
распределения как предельные для случайных сумм~// Тео\-рия
вероятностей и~ее применения, 2013. Т.~58. Вып.~1. С.~117--132.

\bibitem{kckg2013}
\Au{Королев В.\,Ю., Черток А.\,В., Корчагин~А.\,Ю.,
Горшенин~А.\,К.} Ве\-ро\-ят\-но\-ст\-но-ста\-ти\-сти\-че\-ское моделирование
информационных потоков в~сложных финансовых системах на основе
высокочастотных данных~// Информатика и~её применения, 2013. Т.~7.
Вып.~1. С.~12--21.

\bibitem{p2004}
\Au{Protassov R.\,S.} EM-based maximum likelihood parameter
estimation for a~multivariate generalized hyperbolic distribution
with fixed~$\lambda$~// Statistics Computing, 2004. Vol.~14.
P.~67--77.

\bibitem{kn2010}
\Au{Королев В.\,Ю., Назаров А.\,Л.} Разделение смесей
вероятностных распределений при помощи сеточных методов моментов и~максимального правдоподобия~//
Автоматика и~телемеханика, 2010. Вып.~3. С.~98--116.

\bibitem{DSch1983}
\Au{Dennis J.\,E., Schnabel R.\,B.} Numerical methods for
unconstrained optimization and nonlinear equations.~--- Englewood
Cliffs: Prentice-Hall, 1983. 378~p.
 \end{thebibliography}

 }
 }

\end{multicols}

\vspace*{-6pt}

\hfill{\small\textit{Поступила в редакцию 01.10.14}}

\newpage

%\vspace*{12pt}

%\hrule

%\vspace*{2pt}

%\hrule

%\vspace*{12pt}

\def\tit{A MODIFIED GRID METHOD FOR~STATISTICAL SEPARATION
OF~NORMAL VARIANCE-MEAN MIXTURES}

\def\titkol{A modified grid method for statistical separation
of~normal variance-mean mixtures}

\def\aut{V.\,Yu.~Korolev$^{1,2}$ and~A.\,Yu.~Korchagin$^1$}

\def\autkol{V.\,Yu.~Korolev and~A.\,Yu.~Korchagin}

\titel{\tit}{\aut}{\autkol}{\titkol}

\vspace*{-9pt}


\noindent
$^1$Faculty of Computational Mathematics and Cybernetics,
M.\,V.~Lomonosov Moscow State University,\linebreak
$\hphantom{^1}$1-52 Leninskiye Gory, GSP-1, Moscow 119991, Russian Federation


\noindent
$^2$Institute of Informatics Problems, Russian Academy of Sciences,
44-2~Vavilov Str., Moscow 119333, Russian\linebreak
$\hphantom{^1}$Federation

\def\leftfootline{\small{\textbf{\thepage}
\hfill INFORMATIKA I EE PRIMENENIYA~--- INFORMATICS AND
APPLICATIONS\ \ \ 2014\ \ \ volume~8\ \ \ issue\ 4}
}%
 \def\rightfootline{\small{INFORMATIKA I EE PRIMENENIYA~---
INFORMATICS AND APPLICATIONS\ \ \ 2014\ \ \ volume~8\ \ \ issue\ 4
\hfill \textbf{\thepage}}}

\vspace*{3pt}

\Abste{A~modified two-stage grid method for
statistical separation of normal variance-mean mixtures is described
as an alternative to a pure EM (expectation-maximization) algorithm.
At the first stage of this
algorithm, a~discrete approximation is constructed to the mixing
distribution. At the second stage, the obtained discrete
distribution is approximated by an absolutely continuous
distribution from a~predetermined family, say, by a generalized
inverse Gaussian distribution. The convergence of this two-stage
procedure is discussed. The monotonicity of the grid procedure used
at the first stage is proved. The problem of the optimal choice of
the parameters of the method is discussed in detail. First of all,
the problem of the optimal choice of the grid thrown on the support
of the mixing distribution is considered. Statistical estimators are
proposed for the quantiles of the mixing law. The efficiency of the
method is illustrated by examples of its application to the
estimation of the parameters of generalized hyperbolic
distributions.}

\smallskip

\KWE{mixture of probability distributions; normal
variance-mean mixture; generalized hyperbolic distribution;
EM-algorithm; grid method of separation of mixtures}

\DOI{10.14357/19922264140402}

\Ack
\noindent
The research was supported by the Russian Science Foundation (project 14-11-00364).

%\vspace*{3pt}

  \begin{multicols}{2}

\renewcommand{\bibname}{\protect\rmfamily References}
%\renewcommand{\bibname}{\large\protect\rm References}



{\small\frenchspacing
 {%\baselineskip=10.8pt
 \addcontentsline{toc}{section}{References}
 \begin{thebibliography}{99}
 \bibitem{k2011eng}
 \Aue{Korolev, V.\,Yu.} 2011.
\textit{Veroyatnostno-statisticheskie metody dekompozitsii
volatil'nosti khaoticheskikh protsessov}
[Probabilistic and statistical methods for the decomposition of volatility
of chaotic processes].
Moscow: Moscow University Press. 510~p.

\bibitem{n2013eng}
\Aue{Nazarov, A.\,L.} 2013.
{Priblizhennye metody razdeleniya smesey veroyatnostnykh raspredeleniy}
[Approximate methods for the decomposition of volatility of chaotic processes].
Ph.D. Thesis. Moscow: Moscow State University.

\bibitem{BN1977eng}
\Aue{Barndorff-Nielsen, O.\,E.} 1977.
Exponentially decreasing distributions for the logarithm of particle size.
\textit{Proc. Roy. Soc. Lond. A} 353:401--419.

\bibitem{BN1978eng}
\Aue{Barndorff-Nielsen, O.\,E.} 1978.
Hyperbolic distributions and distributions of hyperbolae.
\textit{Scand. J. Statist.} 5:151--157.

\bibitem{BN1982eng}
\Aue{Barndorff-Nielsen, O.\,E., J.~Kent, and M.~S\!{\ptb{\o}}rensen}. 1982.
Normal variance-mean mixtures and $z$-distributions.
\textit{Int. Statist. Rev.} 50(2):145--159.

\bibitem{ks2012eng}
\Aue{Korolev, V.\,Yu., and I.\,A. Sokolov}. 2012.
{Skoshennye raspredeleniya St'yudenta, dispersionnye
gam\-ma-ras\-pre\-de\-le\-niya i~ikh obobshcheniya kak asimptoticheskie
approksimatsii}
[Skewed Student's distributions, variance gamma distributions, and their
generalizations as asymptotic approximations].
\textit{Informatika i ee Primeneniya}~--- \textit{Inform. Appl.} 6(1):2--10.

\bibitem{zk2013eng}
\Aue{Korolev, V.\,Yu., and L.\,M.~Zaks}. 2013.
{Obobshchennye dispersionnye gam\-ma-ras\-pre\-de\-le\-niya kak
predel'nye dlya sluchaynykh summ}
[Generalized variance gamma distributions as limiting for random sums].
\textit{Informatika i ee Primeneniya}~--- \textit{Inform. Appl.} 7(1):105--115.

\bibitem{k2013eng} \Aue{Korolev, V.\,Yu.} 2013.
{Obobshchennye giperbolicheskie raspredeleniya kak predel'nye dlya sluchaynykh summ}
[Generalized hyperbolic distributions as limiting for random sums]
\textit{Theory Probab. Appl.} 58(1):117--132.

\bibitem{kckg2013eng}
\Aue{Korolev, V.\,Yu., A.\,V. Chertok, A.\,Yu.~Korchagin, and A.\,K.~Gorshenin}.
2013. {Ve\-ro\-yat\-no\-st\-no-sta\-ti\-sti\-che\-skoe
mo\-de\-li\-ro\-va\-nie informatsionnykh potokov v~slozhnykh finansovykh sistemakh
na osnove vysokochastotnykh dannykh}
[Probability and statistical modeling of information flows in complex
financial systems from high-frequency data].
\textit{Informatika i~ee Primeneniya}~--- \textit{Inform.  Appl.} 7(1):12--21.

\bibitem{p2004eng-1}
\Aue{Protassov, R.\,S.} 2004.
EM-based maximum likelihood parameter estimation for a multivariate
generalized hyperbolic distribution with fixed~$\lambda$.
\textit{Statistics Computing} 14:67--77.

\bibitem{kn2010eng-1}
\Aue{Korolev, V.\,Yu., and A.\,L.~Nazarov}. 2010.
{Razdelenie smesey veroyatnostnykh raspredeleniy pri pomoshchi
setochnykh metodov momentov i~maksimal'nogo pravdopodobiya}
[Separation of mixtures using grid moment-based methods and maximum likelihood].
\textit{Avtomatika i~Telemekhanika} [Automatics and Telemechanics] 3:98--116.

\bibitem{DSch1983eng}
\Aue{Dennis, J.\,E., and R.\,B.~Schnabel}. 1983.
\textit{Numerical methods for unconstrained optimization and nonlinear equations}.
Englewood Cliffs: Prentice-Hall. 378~p.


\end{thebibliography}

 }
 }

\end{multicols}

\vspace*{-6pt}

\hfill{\small\textit{Received October 01, 2014}}

\vspace*{-18pt}

\Contr

\noindent
\textbf{Korolev Victor Yu.} (b.\ 1954)~---
Doctor of Science in physics and mathematics, professor,
Department of Mathematical Statistics, Faculty of Computational Mathematics
and Cybernetics, M.\,V.~Lomonosov Moscow State University,
1-52 Leninskiye Gory, GSP-1, Moscow 119991, Russian Federation;
leading scientist, Institute of Informatics Problems,
Russian Academy of Sciences, 44-2~Vavilov Str., Moscow 119333, Russian
Federation; victoryukorolev@yandex.ru

\vspace*{3pt}

\noindent
\textbf{Korchagin Alexander Yu.} (b.\ 1989)~---
PhD student, Faculty of Computational Mathematics and Cybernetics,
M.\,V.~Lomonosov Moscow State University,
1-52 Leninskiye Gory, GSP-1, Moscow 119991, Russian Federation;
sasha.korchagin@gmail.com


\label{end\stat}

\renewcommand{\bibname}{\protect\rm Литература}           %05
\def\stat{naumov}

\def\tit{О МАРКОВСКИХ И~РАЦИОНАЛЬНЫХ ПОТОКАХ 
СЛУЧАЙНЫХ СОБЫТИЙ.~II$^*$} % Часть~2$^*$}

\def\titkol{О марковских и рациональных потоках случайных 
событий. II} %Часть 2}

\def\aut{В.\,А.~Наумов$^1$, К.\,Е.~Самуйлов$^2$}

\def\autkol{В.\,А.~Наумов, К.\,Е.~Самуйлов}

\titel{\tit}{\aut}{\autkol}{\titkol}

\index{Наумов В.\,А.}
\index{Самуйлов К.\,Е.}
\index{Naumov V.\,A.}
\index{Samouylov К.\,Е.}


{\renewcommand{\thefootnote}{\fnsymbol{footnote}} \footnotetext[1]
{Исследование выполнено при финансовой поддержке РФФИ в рамках научного проекта №\,19-17-50126.}}


\renewcommand{\thefootnote}{\arabic{footnote}}
\footnotetext[1]{Исследовательский институт инноваций, г.~Хельсинки, Финляндия, 
\mbox{valeriy.naumov@pfu.fi}}
\footnotetext[2]{Российский университет дружбы народов; Институт проблем информатики Федерального 
исследовательского центра <<Информатика и~управ\-ле\-ние>> Российской академии наук, \mbox{samouylov-ke@rudn.ru}}

%\vspace*{6pt}

  \Abst{Статья представляет собой вторую часть обзора, выполненного в рамках проекта 
РФФИ 
  №\,19-17-50126. Цель обзора~--- ознакомление заинтересованных читателей с основами 
теории марковских потоков событий для более подробного изучения и облегчения 
применения этих моделей на практике. В~первой части приведены свойства общих 
марковских потоков событий и показана их связь с марковскими аддитивными процессами и 
процессами марковского восстановления. Во второй части обзора рассмотрены важные для 
приложений частные случаи таких потоков~--- подклассы марковских потоков событий, а~именно:
 простые и групповые потоки однородных и неоднородных событий. Показано, 
как свойства марковских потоков событий связаны с мультипликативностью стационарных 
распределений марковских систем. Обсуждаются  
мат\-рич\-но-экс\-по\-нен\-ци\-аль\-ные распределения и рациональные потоки событий, 
расширяющие возможности марковских потоков для моделирования сложных систем, при 
этом сохраняющие удобство их анализа с помощью вычислительной техники.}
  
  \KW{марковские процессы; марковские аддитивные процессы; потоки без последействия; 
  МС-по\-то\-ки}
  
\DOI{10.14357/19922264200406} 
  
\vspace*{6pt}


\vskip 10pt plus 9pt minus 6pt

\thispagestyle{headings}

\begin{multicols}{2}

\label{st\stat}


\section{Введение}

Настоящий обзор, состоящий из двух частей, включает изложение основ 
теории марковских потоков и снабжен ссылками на большое число работ, 
посвященных марковским и~рациональным потокам событий. Он начался с 
рассмотрения в первой части случайных величин фазового типа, определения 
марковских потоков общего вида и их связи с~марковскими аддитивными 
процессами и процессами марковского восстановления. Во второй части 
обзора  перейдем к важным для приложений подклассам марковских потоков 
однородных и неоднородных событий в разд.~2, а~в~завершение в~разд.~3 
обсудим  
мат\-рич\-но-экс\-по\-нен\-ци\-аль\-ные распределения и~в~разд.~4 
рациональные потоки событий, которые расширяют возможности марковских 
потоков для моделирования сложных систем и~при этом сохраняют удобство 
их анализа. 

Как и в первой части обзора, далее в работе жирные строчные буквы 
обозначают векторы, а~жирные прописные буквы обозначают матрицы. 
Кроме того, используются следующие обозначения: 
$$
\delta(i,j)= \begin{cases}
1, &\mbox{если } i=j\,;\\
0 & \mbox{в~противном\ случае};
\end{cases}
$$
 у~вектора~$\mathbf{e}_i$ 
$i$-я координата равна единице, а остальные равны нулю; $\mathbf{I}\hm= \left[ 
\delta(i,j)\right]$~--- единичная матрица; $\mathbf{u}$~---  
век\-тор-стол\-бец из единиц; $\boldsymbol{\mathcal{N}}^K$~--- множество 
неотрицательных целочисленных векторов длины~$K$, 
$\boldsymbol{\mathcal{N}}^K _0\hm= \boldsymbol{\mathcal{N}}^K \backslash 
\{\mathbf{0}\}$. Для краткости вмес\-то <<наступило $n_1$ событий типа~1, 
$n_2$ событий ти-\linebreak па~2,~\ldots , $n_K$ событий типа~$K$>> будем писать 
<<наступило $\mathbf{n}$ событий>>, где $\mathbf{n}\hm= \left( n_1, n_2, 
\ldots , n_K\right)$.

\section{Важные для~приложений частные случаи марковских потоков 
событий}

\subsection{Простой марковский поток однородных событий}

  Рассмотрим некоторый поток случайных неоднородных событий и 
обозначим через $N_k(t)$ чис\-ло событий типа~$k$, наступивших за время~$t$, 
$\mathbf{N}(t)\hm= (N_1(t), N_2(t), \ldots, N_K(t))$. Поток случайных событий 
называется марковским, если для некоторого случайного процесса~$X(t)$ с 
конечным \mbox{множеством} состояний $\boldsymbol{\mathcal{X}}\hm= \{1,2,\ldots , 
L\}$ процесс $\xi(t)\hm= (X(t), \mathbf{N}(t))$ является марковским процессом, 
однородным во времени и по второй компоненте, т.\,е.\ если для любых~$t, 
h\hm>0$ справедливы равенства
  \begin{multline*}
  {\sf P}\left(X(h+t)=j, \mathbf{N}(h+t)=\mathbf{k}+\mathbf{n}\vert X(h)=i, \right.\\
\left.\mathbf{N}(h)=\mathbf{k}\right)=p_{\mathbf{n}}(i,j,t)\,,\enskip
  \mathbf{k}, \mathbf{n} \in \boldsymbol{\mathcal{N}}^K,\enskip i,j\in 
\boldsymbol{\mathcal{X}}\,.
  \end{multline*}
Матрицы вероятностей переходов $\mathbf{P}_{\mathbf{n}}(t)\hm= 
[p_{\mathbf{n}}(i,j,t)]$ однозначно определяются матрицами интенсивностей 
переходов $\mathbf{A}_{\mathbf{n}}\hm= \left[ a_{\mathbf{n}}(i,j)\right]$, 
$\mathbf{n}\hm\geq \mathbf{0}$, где
\begin{align*}
a_{\mathbf{0}}(i,j) &=\lim\limits_{t\to0} \fr{1}{t}\left( p_{\mathbf{0}}(i,j,t)-\delta(i,j)\right)\,,\enskip
 i,j\in  \boldsymbol{\mathcal{X}}\,;\\
a_{\mathbf{n}}(i,j) &=\lim\limits_{t\to0} \fr{1}{t}\, p_{\mathbf{n}}(i,j,t)\,,\enskip i,j\in 
\boldsymbol{\mathcal{X}}\,,\enskip \mathbf{n}\in \boldsymbol{\mathcal{N}}^K_0,
\end{align*}
при этом фазовый процесс~$X(t)$ является однородным марковским 
процессом с матрицей интенсивностей переходов $\mathbf{A}\hm= 
\sum\nolimits_{\mathbf{n}\in \boldsymbol{\mathcal{N}}^K} 
\mathbf{A}_{\mathbf{n}}$.
  
  В первой части обзора определен процесс марковского восстановления 
$(X_l,\boldsymbol{\sigma}_l, \tau_l)$, где $X_l\hm= X(t_l)$~--- состояния 
фазового процесса~$X(t)$ марковского потока в моменты после наступления\linebreak 
событий потока, $X(t)\hm\in \boldsymbol{\mathcal{X}} \hm= \{1,2,\ldots ,L\}$, 
$0\hm< t_1\hm< t_2
  <\cdots$~--- моменты наступления событий, также называемые 
вызывающими моментами; $\tau_l\hm= t_l\hm- t_{l-1}$~--- длины интервалов 
между \mbox{моментами} наступления событий; $\boldsymbol{\sigma}_l$~--- вектор, 
$\boldsymbol{\sigma}_l\hm= (\sigma_{l,1}, \ldots , \sigma_{l,K})$, 
в~котором~$\sigma_{l,k}$ есть размер группы событий типа~$k$, наступивших 
в~момент~$t_l$, $l\hm=1, 2, \ldots$ Матрицы $\mathbf{G}_{\mathbf{n}}(x)\hm= 
[G_{\mathbf{n}}(i,j,x)]$, описывающие связанный с марковским потоком 
процесс марковского восстановления $(X_l, \boldsymbol{\sigma}_l, \tau_l)$, и 
их преобразования Лап\-ла\-са--Стилть\-еса имеют следующий вид:

\noindent
  \begin{align}
  \mathbf{G}_{\mathbf{n}}(x)&=\int\limits_0^x \exp 
(z\mathbf{A}_0)\mathbf{A}_{\mathbf{n}}\,dz={}\notag\\
&\hspace*{-10mm}{}=\left( \exp 
(x\mathbf{A}_{\mathbf{0}}))-\mathbf{I}\right)\mathbf{A}_0^{-1} \mathbf{A}_{\mathbf{n}}\,,\ \mathbf{n}\in 
\boldsymbol{\mathcal{N}}_0^K\,;
  \label{e1-nau}\\
  \int\limits_0^x e^{-\nu x}d\mathbf{G}_{\mathbf{n}}(x)&= (\nu\mathbf{I}-
\mathbf{A}_{\mathbf{0}})^{-1}\mathbf{A}_{\mathbf{n}}\,,\ \mathbf{n}\in 
\boldsymbol{\mathcal{N}}_0^K\,.
  \label{e2-nau}
  \end{align}
Используя матрицы $\mathbf{G}_{\mathbf{n}}(x)$, можно найти совместное 
распределение числа~$\boldsymbol{\sigma}_l$ наступивших событий и 
длин~$\tau_l$ интервалов между вызывающими моментами 
\begin{multline}
F_{\mathbf{k}_1, \mathbf{k}_2, \ldots , \mathbf{k}_m} \left(x_1, x_2, \ldots , 
x_m\right)={}\\
{}={\sf P}\left(
\boldsymbol{\sigma}_l=\mathbf{k}_l\,, \tau_l<x_l\,, l=1,2,\ldots, m\right)={}\\
{}=\bm{\alpha}\mathbf{G}_{\mathbf{k}_1}(x_1) \mathbf{G}_{\mathbf{k}_2}(x_2)\cdots 
\mathbf{G}_{\mathbf{k}_m}(x_m)\mathbf{u}\,,
\label{e3-nau}
\end{multline}
а также плотность этого распределения

\columnbreak

\noindent
\begin{multline}
f_{\mathbf{k}_1, \mathbf{k}_2, \ldots , \mathbf{k}_m} (x_1, x_2, \ldots , 
x_m)={}\\
{}=\bm{\alpha}\exp \left( x_1\mathbf{A}_{\mathbf{0}}\right) 
\mathbf{A}_{\mathbf{k}_1}\exp \left( x_2\mathbf{A}_{\mathbf{0}}\right) 
\mathbf{A}_{\mathbf{k}_2}\cdots\\
\cdots \exp \left( x_m\mathbf{A}_{\mathbf{0}}\right) 
\mathbf{A}_{\mathbf{k}_m}\mathbf{u}\,,\quad
\mathbf{k}_1, \mathbf{k}_2, \ldots , \mathbf{k}_m\in 
\boldsymbol{\mathcal{N}}_0^K\,,\\
 x_0, x_1, \ldots , x_m>0\,,\enskip m=1,2,\ldots
\label{e4-nau}
\end{multline}

\vspace*{-6pt}

\noindent
где $\bm{\alpha}$~--- начальное распределение фазового про\-цесса.


  
  Простой марковский поток однородных событий~--- это марковский поток 
событий одного типа, причем в каждый вызывающий момент наступает ровно 
одно событие. Он характеризуется двумя мат\-ри\-ца\-ми интенсивностей 
переходов $\mathbf{S}\hm= \mathbf{A}_0$ и~$\mathbf{R}\hm= \mathbf{A}_1$, 
а~остальные матрицы~$\mathbf{A}_k$, $k\hm\geq 2$, для такого потока~--- 
нулевые. Первыми работами, посвященными простым марковским потокам 
однородных событий, стали~[1--5]. Их применение к~решению задач теории 
телетрафика рассматривается  
в~\cite{6-nau, 7-nau}. Поток вызывающих моментов любого марковского 
потока~--- это простой марковский поток, характеризуемый матрицами 
$\mathbf{S}\hm= \mathbf{A}\hm-\mathbf{R}$ и~$\mathbf{R}\hm= 
\sum\nolimits_{\mathbf{n}\in \boldsymbol{\mathcal{N}}_0^K} 
\mathbf{A}_{\mathbf{n}}$. К~простым марковским потокам относятся также 
процессы восстановления фазового типа~\cite{8-nau}. Для таких потоков ранг 
матрицы~$\mathbf{R}$ равен единице и~она имеет вид $\mathbf{R}\hm= 
\mathbf{sq}$, где $\mathbf{s}\hm= -\mathbf{Su}$. Верно и~обратное~\cite{7-nau}. 
В~англоязычной литературе простые марковские потоки называют 
Markovian arrival process и~используют для их обозначения сокращение МАР 
или MArP.
  
  Простой марковский поток однородных событий является 
полумарковским, поскольку последовательность $(X_l, \tau_l)$, $l\hm=1, 
2,\ldots,$~--- процесс марковского восстановления. Из~(1) и~(2) вытекают 
следующие формулы для полумарковской матрицы $\mathbf{G}(x)\hm= \left[ 
G(i,j,x)\right]$ процесса $(X_l,\tau_l)$ марковского восстановления с 
элементами 

\vspace*{3pt}

\noindent
  $$
  G(i,j,x)={\sf P} \left( X_l=j,\ \tau_l<x\vert X_{l-1}=i\right)
  $$
  
  \vspace*{-1pt}
  
  \noindent
 и для ее преобразования Лап\-ла\-са--Стилть\-еса:
 
 \vspace*{2pt}
 
 \noindent
\begin{equation}
\left.
\begin{array}{rl}
\mathbf{G}(x)&=\left( \exp (x\mathbf{S})-\mathbf{I}\right) \mathbf{S}^{-
1}\mathbf{R}\,;\\
\displaystyle\int\limits_0^x e^{-\nu x}d\mathbf{G}(x)&=(\nu\mathbf{I}-\mathbf{S})^{-1}\mathbf{R}\,.
\end{array}
\right\}
\label{e5-nau}
\end{equation}

\vspace*{-2pt}
  
  Из~(\ref{e4-nau}) вытекает следующее выражение для плотности функции 
распределения длин интервалов~$\tau_l$ между моментами наступления 
событий простого марковского потока однородных событий:

\vspace*{-8pt}

\noindent
  \begin{multline}
  f\left( x_1, x_2, \ldots, x_m\right)={}\\
  {}=\bm{\alpha}\exp \left(x_1\mathbf{S}\right) 
\mathbf{R}\exp \left( x_2\mathbf{S}\right)\mathbf{R}\cdots \exp \left( 
x_m\mathbf{S}\right) \mathbf{Ru}\,,\\
  x_0, x_1,\ldots , x_m>0\,,\enskip m=1,2,\ldots
  \label{e6-nau}
  \end{multline}
  
  \vspace*{-2pt}
  
  Поскольку простой марковский поток является полумарковским, при 
анализе систем массового обслуживания с такими поступающими потоками 
можно использовать результаты, полученные для систем с полумарковским 
входящим потоком,  
например~[9--12].
   
  В первом разделе обзора указано, что стационарные распределения 
$\mathbf{q}\hm=[q(i)]$ и $\mathbf{q}_{\mathbf{n}}\hm= [q_{\mathbf{n}}(i)]$, 
$\mathbf{n}\hm\in \boldsymbol{\mathcal{N}}_0^K$, вложенных цепей 
Маркова~$X_l$ и~$(X_l, \boldsymbol{\sigma}_l)$ связаны со стационарным 
распределением~$\mathbf{p}$ фазового процесса~$X(t)$ следующими 
равенствами:
\begin{multline*}
  \mathbf{q}=\fr{1}{\lambda}\,\mathbf{p}\boldsymbol{\Lambda}\,,\
  \mathbf{p}=-\lambda \mathbf{q}\mathbf{A}_0^{-1}\,,\ 
  \mathbf{q}=\sum\limits_{\mathbf{n}\in \boldsymbol{\mathcal{N}}_0^K} 
\mathbf{q}_{\mathbf{n}}\,,\\
 \mathbf{q}_{\mathbf{n}}=\fr{1}{\lambda}\,\mathbf{p}
  \mathbf{A}_{\mathbf{n}}\,,\enskip \mathbf{n}\in \boldsymbol{\mathcal{N}}_0^K\,.
  \end{multline*}

  Если вектор из единиц~$\mathbf{u}$ является правым собственным 
вектором каждой из матриц~$\mathbf{A}_{\mathbf{n}}$ и выполняются 
равенства 
  \begin{equation}
  \mathbf{A}_{\mathbf{n}}\mathbf{u}=\lambda_{\mathbf{u}}\mathbf{u}\,,\quad
  \mathbf{n}\in \boldsymbol{\mathcal{N}}_0^K\,,
  \label{e7-nau}
  \end{equation}
то из~(\ref{e3-nau}) следует, что при любом начальном 
распределении~$\mathbf{s}$ марковский поток будет стационарным потоком 
без последействия. Аналогично, если вектор стационарных 
вероятностей~$\mathbf{p}$ является левым собственным вектором 
матриц~$\mathbf{A}_{\mathbf{n}}$ и выполняются равенства 
\begin{equation}
\mathbf{pA}_{\mathbf{n}}=\lambda_{\mathbf{n}}\mathbf{p}\,,\quad
\mathbf{n}\in \boldsymbol{\mathcal{N}}_0^K\,.
\label{e8-nau}
\end{equation}
    
Условия~(\ref{e7-nau}) и~(\ref{e8-nau}), достаточные для того чтобы 
марковский поток был пуассоновским, для простого марковского потока 
приобретают вид $\mathbf{Ru}\hm= \lambda\mathbf{u}$ и~$\mathbf{pR}\hm= 
\lambda\mathbf{p}$ соответственно, где $\lambda\hm= \mathbf{pRu}$~--- 
интенсивность потока. Проверка необходимых и~достаточных условий 
пуассоновости простого марковского потока более сложна и~требует знания 
собственных векторов матрицы~$\mathbf{S}$~\cite{13-nau}.
  
  Считающий процесс $N(t)$ стационарной версии простого марковского 
потока является асимптотически нормальным с~математическим ожиданием 
${\sf M}(t)\hm=\lambda t$ и дисперсией
  $$
  {\sf D}(t)=\left( 2\mathbf{d}_1\mathbf{s}-\lambda\right) t +2\left( 
\mathbf{d}_2\mathbf{s}-\lambda\right) +o(1)\,,
  $$
где векторы-столб\-цы~$\mathbf{d}_1$ и~$\mathbf{d}_2$~--- единственные 
решения систем линейных уравнений~\cite{2-nau}:
\begin{alignat*}{2}
\mathbf{d}_1\mathbf{A} &=\mathbf{p}(\lambda \mathbf{I}-\mathbf{R})\,,&\quad
\mathbf{d}_1\mathbf{u}&=1\,;\\
\mathbf{d}_2\mathbf{A}&=\mathbf{d}_1 -\mathbf{p}\,, &\quad
\mathbf{d}_2\mathbf{u}&=1\,.
\end{alignat*}
    
\subsection{Простой марковский поток неоднородных событий}

  Простой марковский поток неоднородных событий~--- это марковский 
поток событий нескольких типов, в каждый вызывающий момент которого 
наступает ровно одно событие. Такой поток характеризуется $K\hm+1$ 
матрицами интенсивностей переходов $\mathbf{S}\hm= \mathbf{A}_0$ и 
$\mathbf{R}_k\hm= \mathbf{A}_{\mathbf{e}_k}$, $k\hm=1,2,\ldots ,K$, 
а~остальные матрицы~$\mathbf{A}_{\mathbf{n}}$~--- нулевые. При этом поток 
событий одного типа, например типа~$i$, является простым марковским 
потоком однородных событий, описываемым матрицами 
$\mathbf{S}_i\hm=\mathbf{A}\hm- \mathbf{A}_{\mathbf{e}_i}$ 
и~$\mathbf{R}_i$. Первыми работами, посвященными прос\-тым марковским 
потокам неоднородных событий, считаются~[14--16]. В~англоязычной 
литературе такой поток называют Markovian Arrival Process with marked arrivals 
и~используют для его обозначения сокращение ММАР.  
Из~(\ref{e5-nau}) вытекает следующее выражение для плотности совместного 
распределения ${\sf P}(\omega_l=k_l,\tau_l<x_l, l\hm=1,2,\ldots ,m)$ типов 
$\omega_l$ событий, наступивших в~момент~$t_l$, и~длин~$\tau_l$ интервалов 
между вызывающими моментами: 
  \begin{multline}
  f_{{k}_1, {k}_2, \ldots , {k}_m}\left( x_1, x_2, \ldots , x_m\right)={}\\
  {}=\bm{\alpha}\exp \left( x_1\mathbf{S}\right)\mathbf{R}_{k_1}\exp\left( 
x_2\mathbf{S}\right) \mathbf{R}_{k_2}\cdots\\
\cdots \exp \left( x_m\mathbf{S}\right) 
\mathbf{R}_{k_m}\mathbf{u}\,,\quad
 1\leq k_1, k_2, \ldots , k_m\leq K\,,\\
x_0, x_1, \ldots , x_m>0\,,\quad  m=1,2,\ldots
 \label{e9-nau}
\end{multline}

\subsection{Марковский поток групп однородных событий}

  Марковский поток групп однородных событий~--- это марковский поток 
событий одного типа, в каждый вызывающий момент которого \mbox{может} 
наступить несколько событий. Такие марковские потоки впервые 
исследовались в~\cite{8-nau, 17-nau, 18-nau}, а их описание с помощью 
матриц~$\mathbf{A}_{\mathbf{n}}$ впервые появилось в~\cite{19-nau}. 
В~англоязычной литературе такой поток сейчас называют batch Markovian 
arrival process и используют для его обозначения сокращение BMAP. 
В~\cite{20-nau} получены формулы и асимптотики для первых двух моментов 
считающего процесса~$N(t)$, а~в~\cite{21-nau}~--- для старших моментов~$N(t)$.

\section{Матрично-экспоненциальные распределения}

  Функция распределения $F(t)$ неотрицательной случайной величины 
называется мат\-рич\-но-экс\-по\-нен\-ци\-аль\-ной, если $F(0)\hm<1$ и она 
представима в~виде 
  \begin{equation}
  F(t)=1-\mathbf{q}\exp (t\mathbf{S})\mathbf{u}
  \label{e10-nau}
  \end{equation}
с некоторым вектором~$\mathbf{q}$ и матрицей~$\mathbf{S}$, име\-ющей 
собственные числа лишь с отрицательными действительными частями. Для 
того чтобы функция распределения~$F(t)$ неотрицательной случайной 
величины была  
мат\-рич\-но-экс\-по\-нен\-ци\-аль\-ной, необходимо и достаточно, чтобы она 
имела рациональное преобразование 
Лап\-ла\-са--Стилть\-еса $\tilde{F}(\nu)$. Минимальный порядок 
матрицы~$\mathbf{S}$  
в~мат\-рич\-но-экс\-по\-нен\-ци\-аль\-ном представлении~(\ref{e10-nau}) равен 
чис\-лу полюсов функции $\tilde{F}(\nu)$ с учетом их кратности. Представление 
с~матрицей~$\mathbf{S}$ минимального порядка называется минимальным. 

  В некоторых работах по мат\-рич\-но-экс\-по\-нен\-ци\-аль\-ным функциям  
распределения~\cite{22-nau, 23-nau, 24-nau}, а~также в книгах~\cite{25-nau, 26-nau}, 
чтобы подчеркнуть аналогию с экспоненциальными функциями 
\mbox{распределения},  
вмес\-то~(\ref{e10-nau}) использовалось пред\-став\-ле\-ние $F(t)\hm= 1\hm - 
\mathbf{q}\exp (-t\mathbf{B})\mathbf{u}$ со знаком минус перед~$t$ 
и~мат\-ри\-цей~$\mathbf{B}$, име\-ющей собственные чис\-ла с~положительными 
действительными частями. В~настоящее\linebreak время используются только 
представления вида~(\ref{e10-nau}). Иногда допускается, что 
вектор~$\mathbf{u}$ в~(\ref{e10-nau}) может быть любым, а~не состоящим из 
единиц, как в~рас\-смат\-ри\-ва\-емом случае. Однако в~\cite{24-nau, 27-nau} 
было показано, что всегда можно подобрать мат\-рич\-но-экс\-по\-нен\-ци\-аль\-ное 
пред\-став\-ле\-ние с~$\mathbf{u}\hm=(1,1,\ldots , 1)$. 
  
  Идея матрично-экс\-по\-нен\-ци\-аль\-ных функций распределения восходит 
к работе~\cite{28-nau}, в которой показано, что рациональные преобразования  
Лап\-ла\-са--Стилть\-еса неотрицательных функций распределения 
представимы в виде:
  $$
  \tilde{F}(s)=p_0+\sum\limits^L_{l=1} q_0\cdots q_{l-1} p_l \prod\limits^l_{i=1} 
\fr{\lambda_i}{\lambda_{i}+s}\,,
  $$
где $p_i+q_i\hm=1$, $i\hm=1, \ldots , L$, $p_L\hm=1$, и~$-\lambda_i$, $i\hm=1, 
\ldots , L$,~--- полюсы~$\tilde{F}(s)$. Такое представление можно записать в 
мат\-рич\-но-экс\-по\-нен\-ци\-аль\-ном виде~(\ref{e10-nau}), полагая 
\begin{align*}
\mathbf{q}&=(1,0,\ldots ,0)\,;\\
\mathbf{S}&=\begin{bmatrix}
-\lambda_1&q_1\lambda_1&0&\cdots&0\\
0&-\lambda_2&q_2\lambda_2&\ddots &\vdots\\
0&0&\ddots& \ddots& 0\\
\vdots& \ddots& \ddots& -\lambda_{L-1}&q_{L-1}\lambda_{L-1}\\
0&\cdots & 0&0&-\lambda_L
\end{bmatrix}\,,
\end{align*}
%
  при этом элементы матрицы~$\mathbf{S}$ могут быть комплексными. 
В~\cite{22-nau} показано, что вектор~$\mathbf{q}$ и~мат\-ри\-ца~$\mathbf{S}$  
в~мат\-рич\-но-экс\-по\-нен\-ци\-аль\-ном  
пред\-ставлении~(\ref{e10-nau}) всегда могут быть выбраны действительными. 
  
  Из~(\ref{e10-nau}) вытекают формулы для начальных моментов
  \begin{equation*}
  \int\limits_0^\infty t^n dF(t)=n! \mathbf{q}(-\mathbf{S})^{-n}\mathbf{u}\,,\enskip 
n=1,2,\ldots
  %\label{e11-nau}
  \end{equation*}
и для преобразования Лап\-ла\-са--Стилть\-еса функции распределения~$F(t)$ 
\begin{multline*}
\tilde{F}(\nu)=\int\limits_0^\infty e^{-\nu t}dF(t)={}\\
{}=1-
\mathbf{q}\mathbf{u}+\mathbf{q}(\nu\mathbf{I}-\mathbf{S})^{-1} \mathbf{s}=1-
\nu\mathbf{q}(\nu\mathbf{I}-\mathbf{S})^{-1}\mathbf{u}\,,
%\label{e12-nau}
\end{multline*}
где $\mathbf{s}=-\mathbf{Su}$. Кроме того,  
мат\-рич\-но-экс\-по\-нен\-ци\-аль\-ные функции распределения обладают 
сле\-ду\-ющи\-ми свойствами~\cite{24-nau}.
\begin{enumerate}[1.]
\item Пусть $F_i(t)=1\hm- \mathbf{q}_i\exp(t\mathbf{S}_i) \mathbf{u}$, 
$i\hm=1,2$,~--- мат\-рич\-но-экс\-по\-нен\-ци\-аль\-ные функции 
распределения и $p_1\hm+p_2\hm=1$. Тогда
\begin{align*}
p_1F_1(t)+p_2F_2(t)&={}\\
&\hspace*{-15mm}{}=1-(p_1\mathbf{q}_1, p_2\mathbf{q}_2)\exp \left( 
t\begin{bmatrix} \mathbf{S}_1 & \mathbf{0}\\
\mathbf{0}& \mathbf{S}_2\end{bmatrix}
\right) \mathbf{u}\,; %\label{e13-naum}
\\
\left( F_1*F_2\right) (t) &={}\\
&\hspace*{-25mm}{}= 1-\left(\mathbf{q}_1,F_1(0) \mathbf{q}_2\right) \exp 
\left( t \begin{bmatrix}
\mathbf{S}_1 & -\mathbf{S}_1\mathbf{uq}_2\\
\mathbf{0} & \mathbf{S}_2\end{bmatrix} \right) \mathbf{u}\,.
%\label{e14-nau}
\end{align*}
\item Пусть $\tau$ и~$\gamma$~--- независимые неотрицательные случайные 
величины с функциями распределения~$F(t)$ и~$G(t)$ соответственно, 
причем~$F(t)$ имеет  
мат\-рич\-но-экс\-по\-нен\-ци\-аль\-ное представление~(\ref{e10-nau}). Тогда 
функция распределения~$H(t)$ случайной величины  $(\tau\hm-\gamma)^+$ 
имеет  
мат\-рич\-но-экс\-по\-нен\-ци\-аль\-ное пред\-став\-ление 
\begin{equation*}
H(t)=1-\mathbf{qU}\exp (t\mathbf{S})\mathbf{u}\,,
%\label{e15-nau}
\end{equation*}
где 
\begin{equation}
\mathbf{U}=\int\limits_0^\infty e^{t\mathbf{S}}dG(t)\,.
\label{e16-nau}
\end{equation}

\item Пусть $F(t)$ имеет мат\-рич\-но-экс\-по\-нен\-ци\-аль\-ное  
представление~(\ref{e10-nau}), а~у~квад\-рат\-ной мат\-ри\-цы~$\mathbf{V}$ 
все собственные числа имеют неотрицательные вещественные части. Тогда
\begin{multline*}
\int\limits_0^\infty e^{-t\mathbf{V}}dF(t)=(1-\mathbf{qu}) \mathbf{I}+
(\mathbf{q}\otimes \mathbf{I})\boldsymbol{\Psi}(\mathbf{Su}\otimes 
\mathbf{I})={}\\
{}=\mathbf{I}-(\mathbf{q}\otimes 
\mathbf{I})\boldsymbol{\Psi}(\mathbf{u}\otimes \mathbf{V})\,,
%\label{e17-nau}
\end{multline*}
где $\boldsymbol{\Psi}=(\mathbf{I}\otimes \mathbf{V}\hm- \mathbf{S}\otimes 
\mathbf{I})^{-1}$. 
  \end{enumerate}
  
  Последнее свойство можно использовать для вычисления 
матриц~$\mathbf{U}$ в~(\ref{e16-nau}) для мат\-рич\-но-экс\-по\-нен\-ци\-аль\-ных 
функций распределения~$G(t)$.
  
  Ясно, что функции распределения фазового типа являются  
мат\-рич\-но-экс\-по\-нен\-ци\-аль\-ны\-ми. Однако их  
мат\-рич\-но-экс\-по\-нен\-ци\-аль\-ные представления 
  \begin{multline*}
  F(t)-1-\mathbf{q}\exp (t\mathbf{S})\mathbf{u}\,,\enskip
  %\label{e18-nau}
    F(0)=1-\mathbf{qu}\,,\\
     \fr{d}{dt}\,F(t)= \mathbf{q}\exp 
(t\mathbf{S})\mathbf{s}\,,\ t>0\,,
  \end{multline*}
с ограничениями
\begin{equation}
\hspace*{-2mm}\left.
\begin{array}{rlrlrl}
\!\!\displaystyle 0<\sum\limits_{j\in \boldsymbol{\mathcal{X}}} q(j)&\leq 1\,,&\ q(i)&\geq0\,,&\ i&\in 
\boldsymbol{\mathcal{X}};
\\[9pt]
\!\!\displaystyle \sum\limits_{j\in \boldsymbol{\mathcal{X}}} \!\!s(i,j)&\leq 0\,,&\ s(i,j)&\geq 0\,,&\ 
i&\not= j\,,\ i, j\in \boldsymbol{\mathcal{X}},
\end{array}\!
\right\}\!\!\!\!
\label{e19-nau}
\end{equation}
где $\mathbf{S}=[s(i,j)]$, следует отличать от мат\-рич\-но-экс\-по\-нен\-ци\-аль\-ных 
представлений этих же функций, но без ограничений~(\ref{e19-nau}). 
Порядок  
мат\-рич\-но-экс\-по\-нен\-ци\-аль\-но\-го представления, удовлетворяющего 
ограничениям~(\ref{e19-nau}), будем называть числом этапов этого 
представления, а~порядок мат\-рич\-но-экс\-по\-нен\-ци\-аль\-но\-го 
представления, не удовлетворяющего этим ограничениям, следуя~\cite{28-nau}, 
будем называть\linebreak
 числом \textit{фиктивных} этапов. Необходимые и 
достаточные условия того, чтобы для функции распределения 
с~рациональным преобразованием Лап\-ла\-са--Стилть\-еса существовало 
представление, \mbox{удовлетворяющее} ограничениям~(\ref{e19-nau}), получены 
в~\cite{29-nau}. Для этого надо, чтобы (а)~функция распределения имела 
непрерывную положительную плотность на правой полуоси и~(б)~ее 
преобразование Лап\-ла\-са--Стилть\-еса имело единственный полюс 
с~максимальной вещественной частью. 

\section{Рациональные потоки событий}

  Рациональный поток групп неоднородных событий 
$(t_l,\boldsymbol{\sigma}_l)$, $l\hm=1,2,\ldots$, можно определить как поток, 
для которого совместное распределение чис\-ла~$\boldsymbol{\sigma}_l$ 
наступивших событий и~длин~$\tau_l$ интервалов между моментами~$t_l$ 
наступления событий дается формулами~(\ref{e1-nau}) и~(\ref{e3-nau}) 
с~матрицами~$\mathbf{A}_{\mathbf{n}}$, $\mathbf{n}\hm\in 
\boldsymbol{\mathcal{N}}^K $, обладающими следующими свойствами:
  \begin{enumerate}[(1)]
\item действительные части собственных чисел мат\-ри\-цы~$\mathbf{A}_{\mathbf{0}}$ 
отрицательны;
\item действительные части собственных чисел мат\-ри\-цы 
$\mathbf{A}\hm= \sum\nolimits_{\mathbf{n}\in 
\boldsymbol{\mathcal{N}}^K} \mathbf{A}_{\mathbf{n}}$ 
неположительны;
\item $\mathbf{Au}=\mathbf{0}$.
\end{enumerate}
  
  Для стационарных версий рациональных потоков дополнительно требуется, 
чтобы начальный вектор~$\bm{\alpha}$ совпадал с решением~$\mathbf{p}$ 
системы линейных уравнений $\mathbf{pA}\hm=0$, $\mathbf{pu}\hm=1$.
  
  Простой рациональный поток однородных событий, также называемый  
мат\-рич\-но-экс\-по\-нен\-циальным потоком~\cite{30-nau},~--- это поток 
событий одного типа, в каждый вызывающий момент которого наступает 
ровно одно событие и для которого плотность совместного распределения 
длин~$\tau_l$ интервалов между моментами наступления событий дается 
формулой~(\ref{e6-nau}) с матрицами~$\mathbf{S}$ и~$\mathbf{R}$, 
обладающими следующими свойствами~\cite{31-nau}:
\begin{itemize}
\item[(а)] вещественные части собственных чисел матрицы~$\mathbf{S}$ 
отрицательны;
\item[(б)] вещественные части собственных чисел матрицы 
$\mathbf{S}\hm+\mathbf{R}$ неположительны; 
\item[(в)] $(\mathbf{S}+\mathbf{R})\mathbf{u}=\mathbf{0}$. 
  \end{itemize}
  
  Примерами простых рациональных потоков однородных событий могут 
служить полумарковские потоки~\cite{22-nau} и процессы 
восстановления~\cite{27-nau}  
с~мат\-рич\-но-экс\-по\-нен\-ци\-аль\-ны\-ми функциями распределения длин 
интервалов между наступлениями событий.
  
  Рациональный поток неоднородных событий~--- это поток событий 
нескольких типов, в каждый вызывающий момент которого наступает ровно 
одно событие. Для такого потока совместное распределение типов 
наступивших событий~$\omega_l$ и длин~$\tau_l$ интервалов между 
моментами наступления событий дается формулой~(\ref{e9-nau}), а на 
матрицы~$\mathbf{S}$ 
и~$\mathbf{R}\hm=\mathbf{R}_1\hm+\mathbf{R}_2+\cdots  + \mathbf{R}_K$ 
накладываются перечисленные выше ограничения~(a)--(в)~\cite{32-nau}. 

\section{Заключение}

  Метод этапов Эрланга~\cite{33-nau} более 100~лет применяется при анализе 
стохастических систем. К~его широкому распространению привело открытие  
мат\-рич\-но-экс\-по\-нен\-ци\-аль\-но\-го представления для функций 
распределения фазового типа~\cite{34-nau} и моделей марковских потоков 
событий~\cite{1-nau, 17-nau}. Эти модели хорошо подходят для анализа 
стохастических систем с~по\-мощью вычислительной техники, 
приспособленной к обработке векторов и матриц, что привело к развитию 
специальных матричных методов анализа стохастических систем.
  
  Метод фиктивных этапов, предложенный в~\cite{28-nau}, позволил 
распространить метод Эрланга на любые распределения с рациональным 
преобразованием 
  Лап\-ла\-са--Стилть\-еса. Использование мат\-рич\-но-экс\-по\-нен\-ци\-аль\-ных 
  представлений для функций распределения~\cite{22-nau, 23-nau, 25-nau} 
  и~потоков случайных событий~\cite{31-nau} с произвольными рациональными 
преобразованиями 
  Лап\-ла\-са--Стилть\-еса упрощает применение метода фиктивных этапов. 
Формальное применение метода фиктивных этапов приводит\linebreak 
к~решению, 
в~котором вероятности, со\-от\-вет\-ст\-ву\-ющие фиктивным этапам, могут оказаться 
отрицательными, б$\acute{\mbox{о}}$льшими единицы или даже 
комплекс\-ны\-ми. Однако вероятности, соответствующие \mbox{не\-фик\-тив\-ным} 
состояниям, будут неотрицательными числами, не превосходящими единицы. 
Существуют различные интерпретации понятий отрицательных вероятностей 
и интенсивностей \mbox{переходов} \cite{35-nau, 36-nau, 37-nau, 38-nau}. Более 
детально ознакомиться с~{марковским} и~рациональным потоками событий, 
а~также с~матричными методами анализа стохастических систем можно 
 в~обзорах~\cite{39-nau, 40-nau, 41-nau, 42-nau, 43-nau, 44-nau}  
и~монографиях~[18, 25, 26, 45--57]. 

{\small\frenchspacing
 {%\baselineskip=10.8pt
 %\addcontentsline{toc}{section}{References}
 \begin{thebibliography}{99}

\bibitem{1-nau}
\Au{Наумов В.\,А.} О~независимой работе подсистем сложной системы~// Тр.~III 
Всесоюзной  
шко\-лы-се\-ми\-на\-ра по теории массового обслуживания.~--- 
М.: МГУ, 1976. №\,2. С.~169--177.
\bibitem{2-nau}
\Au{Бочаров П.\,П., Наумов В.\,А.} Анализ гиперэкспоненциальной двухфазной системы с 
ограниченным накопителем~// Информационные сети и их структура.~--- М.: Наука, 
1976.  
С.~168--180.
\bibitem{3-nau}
\Au{Наумов В.\,А.} Об обслуженной и избыточной нагрузках полнодоступного пучка с 
ограниченной очередью~// Численные методы решения задач математической физики и 
теории систем.~--- М.: УДН, 1977. С.~51--55.
\bibitem{4-nau}
\Au{Наумов В.\,А.} Исследование некоторых многофазных систем массового 
обслуживания: Дис.\ \ldots\ канд. физ.-мат. наук.~--- М.: УДН, 1978.  98~с.
\bibitem{5-nau}
\Au{Lucantoni D.\,M., Meier-Hellstern~K., Neuts M.\,F.} A~single-server queue with server 
vacations and a class of non-renewal arrival processes~// Adv. Appl. Probab., 1990. 
Vol.~22. Iss.~3. P.~676--705.
\bibitem{6-nau}
\Au{Башарин Г.\,П., Кокотушкин~В.\,А., Наумов~В.\,А.} О~методе эквивалентных замен 
расчета фрагментов сетей связи для ЦВМ~// Изв. АН \mbox{СССР}. Техническая кибернетика, 1979. №\,6. С.~92--99.
\bibitem{7-nau}
\Au{Basharin G., Naumov V.} Simple matrix description of peaked and smooth traffic and 
its applications~// 3rd ITC Specialist Seminar on Fundamentals of Teletraffic Theory.~--- M.: 
VINITI, 1984. P.~38--44. 
\bibitem{8-nau}
\Au{Neuts M.\,F.} Renewal processes of phase type~// Nav. Res. Logist.~Q., 1978. 
Vol.~25. Iss.~3. P.~445--454.
\bibitem{9-nau}
\Au{Cinlar E.} Queues with semi-Markovian arrivals~// J.~Appl. Probab., 1967. Vol.~4. Iss.~2.  
P.~365--379.
\bibitem{10-nau}
\Au{Franken P.} Erlangsche Formeln f$\ddot{\mbox{u}}$r semimarkowschen Eingang // 
Elektronische Informationsverarbeitung Kybernetik, 1968. Vol.~4. Iss.~3. P.~197--204.
\bibitem{11-nau}
\Au{Franken P., Kerstan~J.} Bedienungssysteme mit unendlich vielen Bedienungsapparaten~// 
Operationsforschung Mathematische Statistik.~--- Berlin: Akademie-Verlag, 1968. Vol.~I. 
P.~67--76.
\bibitem{12-nau}
\Au{Neuts M.\,F., Chen~S.-Z.} The infinite server queue with semi-Markovian arrivals and negative 
exponential services~// J.~Appl. Probab., 1972. Vol.~9. Iss.~1. P.~178--184.
\bibitem{13-nau}
\Au{Bean N.\,G., Green D.\,A., Taylor~P.\,G.} When is a MAP poisson?~// 2nd Australia--Japan 
Workshop on Stochastic Models in Engineering,
Technology 
and Management Proceedings~/ Eds. J.~Wilson, D.\,N.\,P.~Murthy, S.~Osaki.~--- 
Brisbane: Technology Management Center, University of Queensland, 1996. P.~34--43.
\bibitem{14-nau}
\Au{Наумов В.\,А.} Матричный аналог формулы Эрланга~// Модели распределения 
информации и методы их анализа.~--- М.: ВИНИТИ, 1988. C.~39--43.
\bibitem{15-nau}
\Au{He Q.-M.} Queues with marked customers~// Adv. Appl. Probab., 1996. Vol.~28. 
Iss.~2. P.~567--587.
\bibitem{16-nau}
\Au{He Q.-M., Neuts M.\,F.} Markov chains with marked transitions~// Stoch. Proc. 
Appl., 1998. Vol.~74. P.~37--52.
\bibitem{17-nau}
\Au{Neuts M.\,F.} A versatile Markovian point process.~--- Newark, DE: 
University of Delaware, Department of Statistics and Computer Science, 1977.
 Technical Report 77/13. 29~p.
\bibitem{18-nau}
\Au{Neuts M.\,F.} Structured stochastic matrices of $M/G/1$ type and their applications.~--- New 
York, NY, USA: Marcel Dekker, 1989. 512~p.
\bibitem{19-nau}
\Au{Lucantoni D.\,M.} New results on the single server queue with a batch Markovian arrival 
process~// Communications Statistics. Stochastic Models, 1991. Vol.~7. Iss.~1. P.~1--46. 
\bibitem{20-nau}
\Au{Narayana S., Neuts M.\,F.} The first two moment matrices of the counts for the Markovian 
arrival processes~// Communications Statistics. Stochastic Models, 1992. Vol.~8. Iss.~3. P.~459--477. 
\bibitem{21-nau}
\Au{Nielsen B.\,F., Nilsson L.\,A.\,F., Thygesen~U.\,H., Beyer~J.\,E.} Higher order moments and 
conditional asymptotics of the batch Markovian arrival process~// Stoch. Models, 2007. Vol.~23. 
Iss.~1. P.~1--26.
\bibitem{22-nau}
\Au{Бочаров П.\,П., Наумов В.\,А.} O~некоторых системах массового обслуживания 
конечной емкости~// Проб\-ле\-мы передачи информации, 1977. Т.~13. №\,4. С.~96--104.
\bibitem{23-nau}
\Au{Наумов В.\,А.} Об однолинейной системе с ограниченным накопителем и заявками 
нескольких видов~// Модели систем распределения информации и их анализ.~--- М.: 
Наука, 1982. C.~77--82.
\bibitem{24-nau}
\Au{Наумов В.\,А.} О~функциях распределения с рациональным преобразованием  
Лап\-ла\-са--Стилть\-еса~// Анализ информационно-вычислительных систем.~--- М.: 
УДН, 1986. С.~47--56.
\bibitem{25-nau}
\Au{Бочаров П.\,П., Печинкин~А.\,В.} Теория массового обслуживания.~--- М.: РУДН, 
1995. 528~с.
\bibitem{26-nau}
\Au{Bocharov P.\,P., D'Apice~C., Pechinkin~A.\,V., Salerno~S.} Queueing theory.~--- Utrecht--Boston: 
VSP, 2004. 446~p.
\bibitem{27-nau}
\Au{Asmussen S., Bladt M.} Renewal theory and queueing algorithms for matrix-exponential 
distributions~// Matrix-analytic methods in stochastic models~/
Eds. A.\,S.~Alfa, S.~Chakravarty.~--- New York, NY, USA: Marcel 
Dekker, 1996. P.~313--341.
\bibitem{28-nau}
\Au{Cox D.\,R.} A use of complex probabilities in the theory of stochastic processes~// Math. 
Proc. Cambridge, 1955. Vol.~51. Iss.~2. P.~313--319. 
\bibitem{29-nau}
\Au{O'Cinneide C.\,A.} Characterization of phase-type distributions~// Communications Statistics. 
Stochastic Models, 1990. Vol.~6. Iss.~1. P.~1--57.
\bibitem{30-nau}
\Au{Bodrog L., Horv$\acute{\mbox{a}}$th~A., Telek~M.} On the properties of moments of matrix 
exponential distributions and matrix exponential processes~// Dagstuhl Seminar Proceedings, 2008. 
Vol.~07461. Paper~1394.
\bibitem{31-nau}
\Au{Asmussen S., Bladt M.} Point processes with finite-dimensional conditional probabilities~// 
Stoch. Proc. \mbox{Appl.}, 1999. Vol.~82. Iss.~1. P.~127--142.
\bibitem{32-nau}
\Au{Horvath G., Telek M.} Acceptance-rejection methods for generating random variables from 
matrix exponential distribution and rational arrival processes~// Matrix-analytic methods in stochastic 
models~/ Eds. G.~Latouche, V.~Ramaswami, J.~Sethuraman, \textit{et al.}~--- 
New York, NY, USA: Springer, 2012. P.~123--144.
\bibitem{33-nau}
\Au{Erlang A.\,K.} \mbox{L{\!\ptb{\o}}sning} af nogle Problemer fra Sandsynlighedsregningen af 
Betydning for de automatiske Telefoncentraler~// Elektroteknikeren, 1917. Iss.~13. P.~5--13.
\bibitem{34-nau}
\Au{Neuts M.\,F.} Probability distribution of phase type~// Liber Amicorum Professor Emeritus 
H.~Florin.~--- Ottignies-Louvain-la-Neuve, Belgium: University of Louvain, Department of Mathematics, 
1975. P.~173--206.
\bibitem{35-nau}
\Au{Bartlett M.\,S.} Negative probability~// Math. Proc. Cambridge, 1945. Vol.~41. Iss.~1. P.~71--73.
\bibitem{36-nau}
\Au{Cox D.\,R.} The analysis of non-Markovian stochastic processes by the inclusion of 
supplementary variables~// Math. Proc. Cambridge, 1955. Vol.~51. Iss.~3. P.~433--441. 
\bibitem{37-nau}
\Au{Bladt M., Neuts M.\,F.} Matrix-exponential distributions: Calculus and interpretations via flows~// 
Stoch. Models, 2003. Vol.~19. Iss.~1. P.~113--124.
\bibitem{38-nau}
\Au{Khrennikov A.} Interpretations of probability.~--- 2nd ed.~--- Berlin: Walter de Gruyter, 2009. 
237~p.
\bibitem{39-nau}
\Au{Наумов В.\,А.} Марковские модели потоков требований~// Системы массового 
обслуживания и информатика.~--- М.: УДН, 1987. С.~67--73.
\bibitem{40-nau}
\Au{Asmussen S.} Matrix-analytic models and their analysis~// Scand. J.~Stat., 2000. 
Vol.~27. Iss.~2. P.~193--226.
\bibitem{41-nau}
\Au{Bladt M.} A~review on phase-type distributions and their use in risk theory~// ASTIN Bull., 
2005. Vol.~35. No.\,1. P.~145--161.
\bibitem{42-nau}
\Au{Artalejo J.\,R., G$\acute{\mbox{o}}$mez-Corral~A.} Markovian arrivals in stochastic 
modelling: A~survey and some new results~// SORT~--- Stat. Oper. Res.~T., 2010. 
Vol.~34. Iss.~2. P.~101--144.
\bibitem{43-nau}
\Au{Вишневский В.\,М., Дудин~А.\,Н.} Системы массового обслуживания с 
коррелированными входными потоками и их применение для моделирования 
телекоммуникационных сетей~// Автоматика и телемеханика, 2017. №\,8. С.~3--59.
\bibitem{44-nau}
\Au{Basharin G., Naumov~V., Samouylov~K.} On Markovian modelling of arrival processes~// 
Stat. Pap., 2018. Vol.~59. Iss.~4. P.~1533--1540. 
\bibitem{45-nau}
\Au{Neuts M.\,F.} Matrix-geometric solutions in stochastic models: An algorithmic approach.~--- 
Baltimore, MA, USA: The John Hopkins University Press, 1981. 332~p.
\bibitem{46-nau}
\Au{Latouche G., Ramaswami~V.} Introduction to matrix analytic methods in stochastic modeling.~--- 
Philadelphia, PA, USA: ASA \& SIAM, 1999. 334~p.
\bibitem{47-nau}
\Au{Asmussen S.} Applied probability and queues.~--- New York, NY, USA: Springer, 2003. 
438~p.
\bibitem{48-nau}
\Au{Breuer L., Baum D.} An introduction to queueing theory and matrix-analytic methods.~--- 
Dordrecht: Springer, 2005. 272~p.
\bibitem{49-nau}
\Au{Bini D.\,A., Latouche~G., Meini~B.} Numerical methods for structured Markov chains.~--- 
New York, NY, USA: Oxford University Press, 2005. 336~p.
\bibitem{50-nau}
\Au{Asmussen S., O'Cinneide~C.\,A.} Matrix-exponential distributions~// Encyclopedia of statistical 
sciences~/ Eds. S.~Kotz, C.\,B.~Read, N.~Balakrishnan, 
B.~Vidakovic, N.\,L.~Johnson.~--- Hoboken, NJ, USA: John Wiley \& Sons, 2006. Vol.~3. P.~1--5.
doi: 10.1002/0471667196.ess1092.
\bibitem{51-nau}
\Au{Li Q.-L.} Constructive computation in stochastic models with applications.~--- Berlin: Springer-Verlag, 2009. 650~p.
\bibitem{52-nau}
\Au{Lipsky L.} Queueing theory: A~linear algebraic approach.~--- 2nd ed.~--- New York, NY, 
USA: Springer, 2009. 548~p.
\bibitem{53-nau}
\Au{Alfa A.\,S.} Queueing theory for telecommunications.~--- New York, NY, USA: Springer, 2010. 
238~p.
\bibitem{54-nau}
\Au{He Q.-M.} Fundamentals of matrix-analytic methods.~--- New York, NY, USA: Springer, 2014. 
349~p.
\bibitem{55-nau}
\Au{Buchholz P., Kriege~J., Felko~I.} Input modeling with phase-type distributions and Markov 
models. Theory and applications.~--- New York, NY, USA: Springer, 2014. 127~p.
\bibitem{56-nau}
\Au{Наумов В.\,А., Самуйлов~В.\,А., Гайдамака~Ю.\,В.} Мультипликативные решения 
конечных цепей Маркова.~--- М.: РУДН, 2015. 159~с.
\bibitem{57-nau}
\Au{Bladt M., Nielsen B.\,F.} Matrix-exponential distributions in applied probability.~--- Boston, MA, USA: 
Springer, 2017. 736~p.
\end{thebibliography}

 }
 }

\end{multicols}

\vspace*{-12pt}

\hfill{\small\textit{Поступила в~редакцию 02.07.20}}

\vspace*{8pt}

%\pagebreak

\newpage

\vspace*{-28pt}

%\hrule

%\vspace*{2pt}

%\hrule

%\vspace*{-2pt}

\def\tit{ON MARKOVIAN AND RATIONAL ARRIVAL PROCESSES.~II}


\def\titkol{On Markovian and rational arrival processes.~II}


\def\aut{V.\,A.~Naumov$^1$ and~К.\,Е.~Samouylov$^{2,3}$}

\def\autkol{V.\,A.~Naumov and~К.\,Е.~Samouylov}

\titel{\tit}{\aut}{\autkol}{\titkol}

\vspace*{-11pt}


   \noindent
   $^1$Service Innovation Research Institute, 8A Annankatu, Helsinki 00120, Finland

\noindent
$^2$Peoples' Friendship University of Russia (RUDN University), 6~Miklukho-Maklaya Str., Moscow 
117198, Russian\linebreak
$\hphantom{^1}$Federation

\noindent
$^3$Institute of Informatics Problems, Federal Research Center ``Computer Science and Control'' 
of the Russian\linebreak
$\hphantom{^1}$Academy of Sciences, 44-2~Vavilov Str., Moscow 119333, Russian Federation

  

\def\leftfootline{\small{\textbf{\thepage}
\hfill INFORMATIKA I EE PRIMENENIYA~--- INFORMATICS AND
APPLICATIONS\ \ \ 2020\ \ \ volume~14\ \ \ issue\ 4}
}%
 \def\rightfootline{\small{INFORMATIKA I EE PRIMENENIYA~---
INFORMATICS AND APPLICATIONS\ \ \ 2020\ \ \ volume~14\ \ \ issue\ 4
\hfill \textbf{\thepage}}}

\vspace*{3pt} 
  
  
   
   
  \Abste{This article is the second part of the review carried out within the framework of the RFBR 
project No.\,19-17-50126. The purpose of this review is to get the interested readers familiar with the 
basics of the theory of Markovian arrival processes to facilitate the application of these models in practice 
and, if necessary, to study them in detail. In the first part of the review, the properties of the general 
Markovian arrival processes are presented and their relationship with Markov additive processes and 
Markov renewal processes is shown. In the second part of the review, the important for applications 
subclasses of Markovian arrival processes, i.\,e., simple and batch arrival processes of homogeneous and 
heterogeneous arrivals, are considered. It is shown how the properties of Markovian arrival processes are 
associated with the product form of stationary distributions of Markov systems. In conclusion, 
matrix-exponential distributions and rational arrival processes are discussed that expand the capabilities of 
Markovian arrival processes for modeling complex systems, while preserving the convenience of analyzing 
them using computations.}
  
  \KWE{Markov chain; Markovian arrival process; Markov additive process; MAP; MArP}
  
  
\DOI{10.14357/19922264200406} 

%\vspace*{-20pt}

  \Ack
  \noindent
  The reported study was funded by RFBR, project No.\,19-17-50126. 
  

%\vspace*{6pt}

  \begin{multicols}{2}

\renewcommand{\bibname}{\protect\rmfamily References}
%\renewcommand{\bibname}{\large\protect\rm References}

{\small\frenchspacing
 {%\baselineskip=10.8pt
 \addcontentsline{toc}{section}{References}
 \begin{thebibliography}{99}
  
  \bibitem{1-nau-1}
  \Aue{Naumov, V.\,A.} 1976. O~nezavisimoy rabote podsistem slozhnoy sistemy [About independent 
operation of subsystems of a complex system]. \textit{Tr. III Vsesoyuznoy shkoly-seminara po teorii 
massovogo obsluzhivaniya} [3th All-Russian School-Seminar of Queuing Theory Proceedings]. 
Moscow. 2:169--177.
  \bibitem{2-nau-1}
  \Aue{Bocharov, P.\,P., and V.\,A.~Naumov.} 1976. Analiz gipereksponentsial'noy dvukhfaznoy sistemy 
s~ogranichennym nakopitelem [Analysis of a hyperexponential two-phase system with a limited storage]. 
\textit{Informatsionnye seti i~ikh struktura} [Information networks and their structure]. Moscow: 
Nauka. 
  168--180.
  \bibitem{3-nau-1}
  \Aue{Naumov, V.\,A.} 1977. Ob obsluzhennoy i~izbytochnoy nagruzkakh polnodostupnogo puchka 
s~ogranichennoy ochered'yu [About serviced and excessive loads of a fully accessible bundle with a limited 
queue]. \textit{Chislennye metody resheniya zadach matematicheskoy fiziki i~teorii system} 
[Numerical methods for solving problems of mathematical physics and systems theory]. Moscow: UDN. 
51--55.
  \bibitem{4-nau-1}
  \Aue{Naumov, V.\,A.} 1978. Issledovanie nekotorykh mnogofaznykh sistem massovogo obsluzhivaniya 
[Research of some multiphase queuing systems].  Moscow: UDN.  PhD Thesis. 98~p.
  \bibitem{5-nau-1}
  \Aue{Lucantoni, D.\,M., K.~Meier-Hellstern, and M.\,F.~Neuts.} 1990. A single-server queue with 
server vacations and a~class of non-renewal arrival processes. \textit{Adv. Appl. Probab.} 
22(3):676--705.
  \bibitem{6-nau-1}
  \Aue{Basharin, G.\,P., V.\,A.~Kokotushkin, and V.\,A.~Naumov.} 1979. O~metode ekvivalentnykh 
zamen rascheta fragmentov setey svyazi dlya TsVM [On the method of equivalent substitutions for 
calculating fragments of communication networks for a central computer]. \textit{Engineering Cybernetics}
 6:92--99.
  \bibitem{7-nau-1}
  \Aue{Basharin, G.\,P., and V.\,A.~Naumov.} 1984. Simple matrix description of peaked and smooth 
traffic and its applications. \textit{3rd ITC Specialist Seminar on Fundamentals of Teletraffic Theory}. 
Moscow: VINITI. 
  38--44. 
  \bibitem{8-nau-1}
  \Aue{Neuts, M.\,F.} 1978. Renewal processes of phase type. \textit{Nav. Res. Logist.~Q.} 25(3):445--454.
  \bibitem{9-nau-1}
  \Aue{Cinlar, E.} 1967. Queues with semi-Markovian arrivals. \textit{J.~Appl. Probab.} 4(2):365--379.
  \bibitem{10-nau-1}
  \Aue{Franken, P.} 1968. Erlangsche Formeln f$\ddot{\mbox{u}}$r semimarkowschen Eingang. 
\textit{Elektronische Informationsverarbeitung  Kybernetik} 4(3):197--204.
  \bibitem{11-nau-1}
  \Aue{Franken, P., and J.~Kerstan.} 1968. Bedienungssysteme mit unendlich vielen 
Bedienungsapparaten. \textit{Operationsforschung Mathematische Statistik} 1:67--76.
  \bibitem{12-nau-1}
  \Aue{Neuts, M.\,F., and S.-Z.~Chen.} 1972. The infinite server queue with semi-Markovian arrivals 
and negative exponential services. \textit{J.~Appl. Probab.} 9(1):178--184.
  \bibitem{13-nau-1}
  \Aue{Bean, N.\,G., D.\,A.~Green, and P.\,G.~Taylor.} 1996. When is a MAP poisson? \textit{2nd 
  Australia--Japan Workshop on Stochastic Models in Engineering, Technology and Management 
Proceedings}. Eds. J.~Wilson, D.\,N.\,P.~Murthy, and S.~Osaki. 
Brisbane: Technology Management Center, University of Queensland. 34--43.
  \bibitem{14-nau-1}
  \Aue{Naumov, V.\,A.} 1988. Matrichnyy analog formuly Erlanga [The matrix analogue of a formula of 
Erlang]. \textit{Modeli raspredeleniya informatsii i~metody ikh analiza} [Information distribution 
models and methods for their analysis]. Moscow: VINITI. 39--43.
  \bibitem{15-nau-1}
  \Aue{He, Q.-M.} 1996. Queues with marked customers. \textit{Adv. Appl. Probab.} 
28(2):567--587.
  \bibitem{16-nau-1}
  \Aue{He, Q.-M., and M.\,F. Neuts.} 1998. Markov chains with marked transitions. \textit{Stoch. 
Proc. Appl.} 74:37--52.
  \bibitem{17-nau-1}
  \Aue{Neuts, M.\,F.} 1977. A~versatile Markovian point process.  
Newark, DE: University of Delaware, Department of Statistics and Computer Science.
Technical Report 77/13. 29~p.
  \bibitem{18-nau-1}
  \Aue{Neuts, M.\,F.} 1989. \textit{Structured stochastic matrices of $M/G/1$ type and their 
applications}. New York, NY: Marcel Dekker. 512~p.
  \bibitem{19-nau-1}
  \Aue{Lucantoni, D.\,M.} 1991. New results on the single server queue with a batch Markovian arrival 
process. \textit{Communications Statistics. Stochastic Models} 7(1):1--46. 
  \bibitem{20-nau-1}
  \Aue{Narayana, S., and M.\,F.~Neuts.} 1992. The first two moment matrices of the counts for the 
Markovian arrival processes. \textit{Communications Statistics. Stochastic Models} 8(3):459--477. 
  \bibitem{21-nau-1}
  \Aue{Nielsen, B.\,F., L.\,A.\,F.~Nilsson, U.\,H.~Thygesen, and J.\,E.~Beyer}. 2007. Higher order 
moments and conditional asymptotics of the batch Markovian arrival process. \textit{Stoch. Models} 
23(1):1--26.
  \bibitem{22-nau-1}
  \Aue{Bocharov, P.\,P., and V.\,A.~Naumov.} 1977. O~nekotorykh sistemakh massovogo 
obsluzhivaniya konechnoy emkosti [On some queueing systems of finite capacity]. \textit{Problemy 
peredachi informatsii} [Problems of Information Transmission] 13(4):96--104.
  \bibitem{23-nau-1}
  \Aue{Naumov, V.\,A.} 1982. Ob odnolineynoy sisteme s~ogranichennym nakopitelem i~zayavkami 
neskol'kikh vidov [About a single-line system with limited storage and multiple types of requests]. 
\textit{Modeli sistem raspredeleniya informatsii i~ikh analiz} [Models of information distribution 
systems and methods for their analysis]. Moscow: Nauka. 77--82.
  \bibitem{24-nau-1}
  \Aue{Naumov, V.\,A.} 1986. O~funktsiyakh raspredeleniya s~ratsio\-nal'nym preobrazovaniem  
Laplasa--Stilt'esa [On distribu\-tion functions with rational Laplace--Stiltjes transformation]. \textit{Analiz 
  informatsionno-vychislitel'nykh \mbox{system}}
   [\mbox{Analysis} of information and computing systems]. Moscow: 
UDN. 47--56.
  \bibitem{25-nau-1}
  \Aue{Bocharov, P.\,P., and A.\,V.~Pechinkin.} 1995. \textit{Teoriya massovogo obsluzhivaniya} 
[Queueing theory]. Moscow: RUDN. 528~p.
  \bibitem{26-nau-1}
  \Aue{Bocharov, P.\,P., C.~D'Apice, A.\,V.~Pechinkin, and S.~Salerno.} 2004. \textit{Queueing 
theory}. Utrecht--Boston: VSP. 446~p.
  \bibitem{27-nau-1}
  \Aue{Asmussen, S., and M.~Bladt}. 1996. Renewal theory and queueing algorithms for 
  matrix-exponential distributions. \textit{Matrix-analytic methods in stochastic models}. 
  Eds. A.\,S.~Alfa and 
S.~Chakravarty. New York, NY: Marcel Dekker. 313--341.
  \bibitem{28-nau-1}
  \Aue{Cox, D.\,R.} 1955. A~use of complex probabilities in the theory of stochastic processes. 
\textit{Math. Proc. Cambridge} 51(2):313--319.
  \bibitem{29-nau-1}
  \Aue{O'Cinneide, C.\,A.} 1990. Characterization of phase-type distributions. \textit{Communications 
Statistics. Stochastic Models} 6(1):1--57.
  \bibitem{30-nau-1}
  \Aue{Bodrog, L., A.~Horv$\acute{\mbox{a}}$th, and M.~Telek.} 2008. On the properties of 
moments of matrix exponential distributions and matrix exponential processes. 
\textit{Dagstuhl Seminar Proceedings} 07461:1394.
  \bibitem{31-nau-1}
  \Aue{Asmussen, S., and M.~Bladt.} 1999. Point processes with finite-dimensional conditional 
probabilities. \textit{Stoch. Proc. Appl.} 82(1):127--142.
  \bibitem{32-nau-1}
  \Aue{Horvath, G., and M.~Telek.} 2012. Acceptance-rejection methods for generating random 
variables from matrix exponential distribution and rational arrival processes. \textit{Matrix-analytic 
methods in stochastic models.} Eds. G.~Latouche, V.~Ramaswami, J.~Sethuraman, \textit{et al.} New York, NY: Springer. 123--144.
  \bibitem{33-nau-1}
  \Aue{Erlang, A.\,K.} 1917. \mbox{L{\!\ptb{\o}}sning} af nogle Problemer fra 
Sandsynlighedsregningen af Betydning for de automatiske Telefoncentraler. \textit{Elektroteknikeren} 
13:5--13.
  \bibitem{34-nau-1}
  \Aue{Neuts, M.\,F.} 1975. Probability distribution of phase type. \textit{Liber Amicorum Professor 
Emeritus H.~Florin}. Ottignies-Louvain-la-Neuve, Belgium: University of Louvain, Department of 
Mathematics.  
173--206.
  \bibitem{35-nau-1}
  \Aue{Bartlett, M.\,S.} 1945. Negative probability. \textit{Math. Proc. 
Cambridge} 41(1):71--73.
  \bibitem{36-nau-1}
  \Aue{Cox, D.\,R.} 1955. The analysis of non-Markovian stochastic processes by the inclusion of 
supplementary variables. \textit{Math. Proc. Cambridge} 
51(3):433--441.
  \bibitem{37-nau-1}
  \Aue{Bladt, M., and M.\,F.~Neuts.} 2003. Matrix-exponential distributions: Calculus and 
interpretations via flows. \textit{Stoch. Models} 19(1):113--124.
  \bibitem{38-nau-1}
  \Aue{Khrennikov, A.} 2009. \textit{Interpretations of probability}. 2nd ed. Berlin: Walter de 
Gruyter. 237~p.
  \bibitem{39-nau-1}
  \Aue{Naumov, V.\,A.} 1987. Markovskie modeli potokov trebovaniy [Markov models of demand 
flows]. \textit{Sistemy massovogo obsluzhivaniya i~informatika} [Queuing systems and computer 
science]. Moscow: UDN. 67--73.
  \bibitem{40-nau-1}
  \Aue{Asmussen, S.} 2000. Matrix-analytic models and their analysis. \textit{Scand. 
J.~Stat.} 27(2):193--226.
  \bibitem{41-nau-1}
  \Aue{Bladt, M.} 2005. A~review on phase-type distributions and their use in risk theory. \textit{ASTIN 
Bull.} 35(1):145--161.
  \bibitem{42-nau-1}
  \Aue{Artalejo, J.\,R., and A.~G$\acute{\mbox{o}}$mez-Corral.} 2010. Markovian arrivals in 
stochastic modelling: A~survey and some new results. \textit{SORT~--- Stat. Oper. Res.~T.}  
34(2):101--144.
  \bibitem{43-nau-1}
  \Aue{Vishnevskiy, V.\,M., and A.\,N.~Dudin.} 2017. Queueing systems with correlated arrival flows 
and their applications to modeling telecommunication networks. \textit{Automat. Rem. Contr.} 
78(8):1361--1403.
  \bibitem{44-nau-1}
  \Aue{Basharin, G., V.~Naumov, and K.~Samouylov.} 2018. On Markovian modelling of arrival 
processes. \textit{Stat. Pap.} 59(4):1533--1540.
  \bibitem{45-nau-1}
  \Aue{Neuts, M.\,F.} 1981. \textit{Matrix-geometric solutions in stochastic models: An algorithmic 
approach.} Baltimore, MA: The John Hopkins University Press. 332~p.
  \bibitem{46-nau-1}
  \Aue{Latouche, G., and V.~Ramaswami.} 1999. \textit{Introduction to matrix analytic methods in 
stochastic modeling}. Philadelphia, PA: ASA \& SIAM. 334~p.
  \bibitem{47-nau-1}
  \Aue{Asmussen, S.} 2003. \textit{Applied probability and queues}. New  York, NY: Springer. 
438~p.
  \bibitem{48-nau-1}
  \Aue{Breuer, L., and D.~Baum.} 2005. \textit{An introduction to queueing theory and 
  matrix-analytic methods.} Dordrecht: Springer. 272~p.
  \bibitem{49-nau-1}
  \Aue{Bini, D.\,A., G.~Latouche, and B.~Meini.} 2005. \textit{Numerical methods for structured 
Markov chains}. New  York, NY: Oxford University Press. 336~p.
  \bibitem{50-nau-1}
  \Aue{Asmussen, S., and C.\,A.~O'Cinneide}. 2006. Matrix-exponential distributions. 
\textit{Encyclopedia of statistical sciences.} Eds. S.~Kotz, C.\,B.~Read, N.~Balakrishnan, 
B.~Vidakovic, and N.\,L.~Johnson. Hoboken, NJ: John Wiley \&~Sons. 3:1--5. doi: 10.1002/0471667196.ess1092.pub2.
  \bibitem{51-nau-1}
  \Aue{Li, Q.-L.} 2009. \textit{Constructive computation in stochastic models with applications}. 
Berlin: 
  Springer-Verlag. 650~p.
  \bibitem{52-nau-1}
  \Aue{Lipsky, L.} 2009. \textit{Queueing theory: A~linear algebraic approach}. 2nd ed. New York, 
NY: Springer. 548~p.
  \bibitem{53-nau-1}
  \Aue{Alfa, A.\,S.} 2010. \textit{Queueing theory for telecommunications}. New York, NY: 
Springer. 238 p.
  \bibitem{54-nau-1}
  \Aue{He, Q.-M.} 2014. \textit{Fundamentals of matrix-analytic methods.} New York, NY: 
Springer. 349 p.
  \bibitem{55-nau-1}
  \Aue{Buchholz, P., J.~Kriege, and I.~Felko.} 2014. \textit{Input modeling with phase-type 
distributions and Markov models. Theory and applications.} New York, NY: Springer. 127~p.
  \bibitem{56-nau-1}
  \Aue{Naumov, V.\,A., K.\,E.~Samuylov, and Yu.\,V.~Gaidamaka.} 2015. \textit{Mul'tiplikativnye 
resheniya konechnykh tsepey Markova} [Multiplicative solutions of finite Markov chains]. Moscow: 
RUDN. 159~p.
  \bibitem{57-nau-1}
  \Aue{Bladt, M., and B.\,F.~Nielsen.} 2017. \textit{Matrix-exponential distributions in applied 
probability}. Boston, MA: Springer. 736~p.
\end{thebibliography}

 }
 }

\end{multicols}

\vspace*{-3pt}

\hfill{\small\textit{Received July 2, 2020}}

%\pagebreak

  %\vspace*{-24pt}
  
  \Contr
  
  \noindent
  \textbf{Naumov Valeriy A.} (b.\ 1950)~--- Candidate of Science (PhD) in physics and mathematics, 
scientific director, Service Innovation Research Institute, 8A~Annankatu, Helsinki 00120, Finland; 
\mbox{valeriy.naumov@pfu.fi}
  
  \vspace*{3pt}
  
  \noindent
  \textbf{Samouylov Konstantin E.} (b.\ 1955)~--- Doctor of Science in technology, professor, Head of 
Department,  Peoples' Friendship 
University of Russia (RUDN University), 6~Miklukho-Maklaya Str., Moscow 117198, Russian 
Federation; senior scientist, Institute of Informatics Problems, Federal Research Center ``Computer 
Science and Control'' of the Russian Academy of Sciences, 44-2~Vavilov Str., Moscow 119333, Russian 
Federation; 
  \mbox{samuylov-ke@rudn.university}
  
\label{end\stat}

\renewcommand{\bibname}{\protect\rm Литература} 
  
              %06

\def\stat{popkovi}

\def\tit{МЕТОДЫ ДЕТЕРМИНИРОВАННЫХ И~РАНДОМИЗИРОВАННЫХ ЭНТРОПИЙНЫХ ПРОЕКЦИЙ 
ДЛЯ~РЕДУКЦИИ РАЗМЕРНОСТИ МАТРИЦЫ ДАННЫХ$^*$}

\def\titkol{Методы детерминированных и рандомизированных энтропийных проекций 
для редукции размерности матрицы} % данных}

\def\aut{Ю.\,С.~Попков$^1$, А.\,Ю.~Попков$^2$, Ю.\,А.~Дубнов$^3$}

\def\autkol{Ю.\,С.~Попков, А.\,Ю.~Попков, Ю.\,А.~Дубнов}

\titel{\tit}{\aut}{\autkol}{\titkol}

\index{Попков Ю.\,С.}
\index{Попков А.\,Ю.}
\index{Дубнов Ю.\,А.}
\index{Popkov Y.\,S.}
\index{Popkov A.\,Y.}
\index{Dubnov Y.\,A.}


{\renewcommand{\thefootnote}{\fnsymbol{footnote}} \footnotetext[1]
{Работа выполнена при поддержке РФФИ (проекты 17-29-03119 и 20-07-00470).}}


\renewcommand{\thefootnote}{\arabic{footnote}}
\footnotetext[1]{Федеральный исследовательский центр <<Информатика и 
управление>> Российской академии наук; Институт проб\-лем управ\-ле\-ния 
им.~В.\,А.~Трапезникова Российской академии наук; ОРТ Брауде Колледж, 
Кармиель, Израиль, \mbox{popkov@isa.ru}}
\footnotetext[2]{Федеральный исследовательский центр <<Информатика и 
управление>> Российской академии наук, \mbox{apopkov@isa.ru}}
\footnotetext[3]{Федеральный исследовательский центр <<Информатика и 
управление>> Российской академии наук; Национальный исследовательский 
университет <<Высшая школа экономики>>, \mbox{yury.dubnov@phystech.edu}}

%\vspace*{-12pt}


\Abst{Предложены методы детерминированного и рандомизированного 
проектирования, ориентированные на решение задачи понижения размерности. В 
случае детерминированного проектирования развивается параллельная процедура 
сжатия матрицы данных, минимизирующая кросс-энтропию Куль\-ба\-ка--Лейб\-ле\-ра 
с~учетом ограничения на информационную емкость, основанная на методе 
проекции градиента. Для рандомизированного проектирования рассматривается 
задача понижения размерности признакового пространства. Идея применения 
процедур проектирования для сжатия матрицы данных реализуется в предлагаемом 
методе рандомизированного энтропийного проектирования, где используется 
принцип сохранения среднего расстояния между многомерными и маломерными 
точками в соответствующих пространствах. Задача поиска оптимальных проекторов 
сводится к поиску распределения вероятностей, максимизирующего информационную 
энтропию Ферми при ограничении на среднее расстояние между точками 
многообразия, которые отображаются матрицами данных и оптимальной проекции.}

\KW{понижение размерности; кросс-энтропия Кульбака--Лейблера; энтропия}


\DOI{10.14357/19922264200407} 
  
%\vspace*{9pt}


\vskip 10pt plus 9pt minus 6pt

\thispagestyle{headings}

\begin{multicols}{2}

\label{st\stat}

\section{Введение}

Во многих прикладных задачах обработки данных последние присутствуют в виде 
прямоугольных матриц $U_{(m \times s)}$. Например, в задачах машинного
обучения элементами строк матрицы данных выступают признаки объекта ($s$), а 
строки служат характеристиками $m$ объектов в~признаковом \mbox{$s$-про}\-стран\-ст\-ве. 
По разным причинам возникает \mbox{необходимость} <<сжать>> матрицу данных, 
т.\,е.\ использовать для обучения матрицу размерности $(m \times r)$ или $(n 
\times r)$, $n\hm < m$, $r \hm< s$. Содержательно это сводится к уменьшению 
числа признаков или уменьшению числа признаков и объектов, на массиве которых 
проводится обучение.

Данная проблема вложена в более общую: приближение заданного набора 
многомерных точек маломерным аффинным многообразием~\cite{Bruckstein_2009}. 
Здесь следует отметить метод главных компонент 
(МГК)~\cite{Kendall_1973, Jolliffe_2011} и его робастные 
версии~\cite{Polyak_2017}, а также метод случайных 
проекций~\cite{Bingham_2001, Vempala_2005}.

В~\cite{Popkov_2018_at_a_en} был предложен энтропийный метод одномерного 
(столбцы или строки) детерминированного сжатия матрицы данных (EDR-ме\-тод), 
основанный на <<прямом>> и <<обратном>> проектировании. 
Мат\-ри\-цы-про\-ек\-то\-ры определяются путем минимизации кросс-эн\-тро\-пий\-но\-го функционала.

В данной работе EDR-метод развивается для детерминированного параллельного 
сжатия матрицы данных на основе процедуры минимизации  кросс-эн\-тро\-пий\-но\-го 
функционала специального вида при ограничениях. Последние связаны с 
информационной емкостью мат\-ри\-цы-про\-ек\-ции.

Идея применения процедур проектирования для сжатия матрицы данных реализуется 
в предлагаемом методе рандомизированного энтропийного проектирования (REDR-ме\-тод). 
Здесь используется принцип сохранения среднего расстояния между 
многомерными и маломерными точками в соответствующих пространствах.


\section{Параллельное проектирование с~ограничениями (EDR-метод)}

Параллельная реализация  процедуры <<прямого>> и <<обратного>> 
проектирования, примененная к матрице данных $U \hm> 0$,  приводит к 
следующей цепочке матричных равенств:
\begin{itemize}
  \item <<прямая>> проекция
\begin{equation}
\left.
\begin{array}{rl}
U_{(m \times s)} Q_{(s \times r)} &= Y_{(m \times r)}\,;\\[6pt]
B_{(n \times m)}  Y_{(m \times r)} &= Z_{(n \times r)},
\end{array}
\right\}
\label{2_1}
\end{equation}
  \item <<обратная>> проекция
\begin{equation}
\left.
\begin{array}{rl}
Z_{(n \times r)} W_{(r \times s)} &= D_{(n \times s)}\,;\\[6pt]
 E_{(m \times n)}  D_{(n \times s)}& = X_{(m \times s)}.
\end{array}
\right\}
\label{2_1a}
\end{equation}
\end{itemize}
Матрицы-проекторы $Q$, $B$, $W$ и $E$~--- неотрицательные.
Равенства~(\ref{2_1}) преобразуют матрицу~$U_{(m \times s)}$ в~<<сжатую>> 
матрицу~$Z_{(n \times r)}$, где $n \hm< m$, $r \hm< s$. Равенство 
(\ref{2_1a}) преобразует мат\-ри\-цу~$Z_{(n \times r)}$ в~мат\-ри\-цу~$X_{{m \times 
s}}$ той же размерности, что и исходная матрица данных $U_{(m \times s)}$.

Из равенств (\ref{2_1}) и~(\ref{2_1a}) имеем:
\begin{multline*}
%\label{2_2}
X_{(m \times s)} = {}\\[3pt]
{}=E_{(m \times n)} \left\{\left[B_{(n \times m)} \left(U_{(m 
\times s)} Q_{(s \times r)}\right)\right] W_{(r \times s)}\right\} > 0\,.\hspace*{-3.59842pt}
\end{multline*}
Скобки в этом равенстве указывают на последовательность операций 
проектирования: 
$$
(\bullet) \hm\rightarrow [\bullet] \hm\rightarrow 
\left\{\bullet\right\}.
$$

Элементы матрицы-проекции $Z_{(n \times s)}$ имеют вид:
\begin{equation*}
%\label{2_3a}
z_{\mu,\nu} = \sum\limits^m_{\beta=1} b_{\mu,\beta} \sum\limits^r_{\alpha=1} 
u_{\beta,\alpha}\,q_{\alpha,\nu},\enskip \mu = \overline{1,n},\ \nu = 
\overline{1,r}.
\end{equation*}

Элементы матрицы $X_{(m \times s)}$ имеют вид:
\begin{multline}
\label{2_3}
x_{ij} = \sum\limits^n_{\mu=1} e_{i,\mu} \sum\limits^r_{\nu=1} w_{\nu,j} 
\sum\limits^m_{\beta=1} b_{\mu,\beta} \sum^s_{\alpha=1} u_{\beta,\alpha} 
q_{\alpha,\nu} > 0\,,\\[2pt] 
i = \overline{1,m},\enskip 
j = \overline{1,s}\,.
\end{multline}
Для измерения отклонения преобразованной матрицы $X_{(m \times s)}$ от 
исходной $U_{(m \times s)}$ воспользуемся \textit{информационной кросс-эн\-тро\-пи\-ей}~\cite{Kullback_1951}:
\begin{equation}
\label{2_4}
\mathcal{H}(X\,|\,U) = \sum\limits^m_{i=1}\sum^s_{j=1} s_{ij}(X\,|\,U)\,,
\end{equation}
где
\begin{equation*}
%\label{2_5}
s_{ij} = x_{ij} \ln \fr{x_{ij}}{u_{ij}}\,.
\end{equation*}
С учетом равенства~(\ref{2_3}) нетрудно видеть, что информационная 
кросс-эн\-тро\-пия~(\ref{2_4}) есть скалярная функция от матрицы данных $U > 0$ и 
мат\-риц-про\-ек\-то\-ров $(Q, B, W, E) \hm\ge 0$, т.\,е.
\begin{equation*}
%\label{2_6}
\mathcal{H} = \mathcal{H}(U\,|\,Q,B,W,E).
\end{equation*}

Важным показателем качества процедуры редукции служит оптимальное снижение 
информационной емкости редуцированной матрицы $Z_{(n \times r)}$ по сравнению 
с информационной емкостью исходной матрицы данных $U_{(m \times 
s)}$~\cite{Popkov_2019_dan}.

\textit{Информационная емкость} измеряется в~энтропийных терминах:
\begin{align*}
%\label{2_9}
\mathcal{I}_Z &= \sum\limits^{n,r}_{(i,j)=1} z_{ij}(Q,B) \ln z_{ij}(Q,B) + 
e^{-1} nr\,;\\
\mathcal{I}_U &= \sum\limits^{m,s}_{(i,j)=1} u_{ij} \ln u_{ij} + e^{-1} 
ms\,. %\notag
\end{align*}
Различие в указанных информационных емкостях будем характеризовать 
квадратичным функционалом:
\begin{equation*}
\mathcal{J}(Q,B) \!= \!\left(\sum\limits^{n,r}_{(i,j)=1}\!\! z_{ij}(Q,B) \ln 
z_{ij}(Q,B) - A\! \right)^{\!2}\!\!,\!\!\!
%\label{2_9a}
\end{equation*}
где
\begin{equation*}
%\label{2_9b}
A = e^{-1}(ms - nr) + \sum\limits^{m,s}_{(i,j)=1} u_{ij} \ln u_{ij}.
\end{equation*}
Образуем обобщенный функционал
\begin{equation*}
%\label{2_9c}
\mathcal{F}(U\,|\,Q,B,W,E) = \mathcal{H}(U\,|\,Q,B,W,E) + \mathcal{J}(Q,B)
\end{equation*}
и оптимальные значения неотрицательных элементов мат\-риц-про\-ек\-то\-ров 
будем определять, минимизируя функционал $\mathcal{F}(U\,|\,Q,B,W,E)$:
\begin{multline}
\label{2_10}
(Q^*,B^*,W^*,E^*) ={}\\
{}+ \argmin\limits_{(Q,B,W,E) \ge 0} 
\mathcal{F}(U\,|\,Q,B,W,E).
\end{multline}

\smallskip

\noindent
\textbf{Замечание.} В задаче~(\ref{2_10}) условие близости информационных 
емкостей матрицы данных и редуцированной матрицы может быть реализовано ввиду 
соответствующего ограничения.
Тогда оптимальные значения неотрицательных элементов мат\-риц-про\-ек\-то\-ров 
определяются решением следующей за\-дачи:
\begin{equation*}
%\label{2_11}
(Q^*,B^*,W^*,E^*) = \argmin\limits_{(Q,B,W,E) \in \Omega} 
\mathcal{H}(U\,|\,Q,B,W,E)\,,
\end{equation*}
где
\begin{multline*}
\Omega = \left\{(Q,B,W,E): (Q,B,W,E) \ge 0; \right.\\
\left.\mathcal{I}_Z(Q,B) \ge 
\delta\,\mathcal{I}_U\right\},\enskip \delta \in (0,1).
\end{multline*}
Допустимый уровень снижения информационной емкости редуцированной матрицы 
регулируется параметром~$\delta$.

\section{Алгоритм решения задачи~(\ref{2_10})}

Задача~(\ref{2_10}) представляет собой задачу минимизации функционала на 
неотрицательном ортанте. Для ее решения можно применить метод проекций 
градиента, предварительно осуществив векторизацию соответствующих матриц.

Введем блочные векторы $\mathbf{v}$ и~$\mathbf{c}$, каждый размерности
$N\hm = (sr \hm+ nm),$
блоками которых являются векторы $\mathbf{q}$ и~$\mathbf{b}$ для вектора 
$\mathbf{v}$ и векторы~$\mathbf{w}$ и~$\mathbf{e}$ для вектора $\mathbf{c}$. 
Все векторы~--- результат векторизации мат\-риц-про\-ек\-то\-ров~$Q$, $B$, $W$ и~$E$ 
соответственно.

Представим (\ref{2_10})  в следующем виде:
\begin{equation*}
%\label{3_3}
\mathcal{F}(\mathbf{v}, \mathbf{c}\,|\,\mathbf{u}) = \mathcal{H}(\mathbf{v}, 
\mathbf{c}\,|\,\mathbf{u}) + \mathcal{J}(\mathbf{v}) \Rightarrow \min,
\end{equation*}
где
\begin{gather*}
\mathcal{H}(\mathbf{v}, \mathbf{c}\,|\,\mathbf{u}) = 
\langle\,\mathbf{x}(\mathbf{v}, \mathbf{c}), \mathbf{y}(\mathbf{v}, 
\mathbf{c}\,|\,\mathbf{u})\rangle_{R^{ms}}\,;\\
\mathcal{J}(\mathbf{v}) = \langle\,\mathbf{z}(\mathbf{v}),  
\mathbf{g}(\mathbf{v}) \rangle_{R^{nr}}\,;\\
\mathbf{v} \geq  \mathbf{0};\enskip \mathbf{c} \geq  \mathbf{0}.
\end{gather*}

Здесь приняты следующие обозначения \cite{Magnus_1988}:
\begin{itemize}
\item
вектор $\mathbf{u}$~--- результат векторизации матрицы данных $U$, его 
размерность $(ms)$; вектор $\mathbf{x}$ размерности $(ms)$ с 
компонентами~(\ref{2_3});

\item
вектор $\mathbf{y}$ размерности $(ms)$ с компонентами
\begin{equation*}
%\label{3_4}
y_k = ln \fr{x_k}{u_k},\qquad k = \overline{1,ms}\,;
\end{equation*}

\item
вектор $\mathbf{g}$ размерности $(nr)$ с компонентами
\begin{equation*}
%\label{3_4}
g_k = ln z_k,\qquad k = \overline{1,nr}\,.
\end{equation*}
\end{itemize}

Для численного решения этой задачи применим покоординатную схему метода 
проекций градиента.

В параллельной процедуре вектор $\mathbf{v}$ объединяет элементы матриц~$Q$
и~$B$, с~по\-мощью которых проводится <<сжатие>> матрицы данных по одному 
измерению. В~вектор $\mathbf{c}$ входят элементы мат\-риц~$W$ и~$E$, с~по\-мощью 
которых проводится <<сжатие>> по второму измерению. Такое разделение векторов 
удобно для применения покоординатного ал\-го\-ритма.

Итерационный шаг покоординатной схемы
метода проекций градиента состоит из двух последовательно реализуемых этапов: 
на одном осуществляется итерация по  $\mathbf{v}$-про\-ек\-ци\-ям градиента, 
а на другом~--- по 
$\mathbf{c}$-про\-ек\-ци\-ям градиента функционала $\mathcal{F}(\mathbf{v}, 
\mathbf{c}\,|\,\mathbf{u})$.
Обозначим градиенты по этим векторам:
\begin{align*}
%\left.
%\begin{array}{rl}
\nabla_{\mathbf{v}} \mathcal{F} &= \nabla_{\mathbf{v}} \mathcal{H} + 
\nabla_{\mathbf{v}} \mathcal{I}\,;\\[6pt]
\nabla_{\mathbf{c}} \mathcal{F} &= \nabla_{\mathbf{c}} \mathcal{H}.
%\end{array}
%\right\}
%\label{3_5}
\end{align*}

Алгоритм минимизации функционала~$\mathcal{F}$ имеет следующий вид:
\begin{itemize}
\item[(\textit{a})] \textit{начальный шаг}

\noindent
$$
\mathbf{v}^0 > \mathbf{0},\quad \mathbf{c}^0 > \mathbf{0};
$$
\item[(\textit{б})] \textit{$i$-й итерационный шаг}
$$
X^i = E^i \left\{\left[ B^i \left( U Q^i\right)\right] W^i \right\};
$$
$$
\mathcal{F}^i = \mathcal{H}^i(\mathbf{v}^i, \mathbf{c}^i\,|\,\mathbf{u}) + 
\mathcal{J}^i(\mathbf{v}^i);
$$
$$
\hspace*{-5mm}\mathbf{v}^{(i+1)}\! =
\!\begin{cases}
\mathbf{v}^{i} + \gamma_{\mathbf{v}} \left(\nabla_{\mathbf{v}} 
\mathcal{H}^i(\mathbf{v}^i, \mathbf{c}^i\,|\,\mathbf{u}) + 
\nabla_{\mathbf{v}} \mathcal{J}^i(\mathbf{v}^i)\right), &\\
&\hspace*{-30mm}\mbox{если} \, 
\mathbf{v}^{(i+1)} \geq \mathbf{0};\\
\mathbf{v}^{i}, &\hspace*{-30mm}\mbox{если} \,\mathbf{v}^{(i+1)} < \mathbf{0};
\end{cases}
$$
$$
 \mathbf{v}^{(i+1)} \Rightarrow Q^{(i+1)}, B^{(i+1)};
$$
$$
\mathbf{c}^{(i+1)} =
\begin{cases}
\mathbf{c}^{n} + \gamma_{\mathbf{c}} \nabla_{\mathbf{c}} 
\mathcal{H}(\mathbf{v}^i, \mathbf{c}^i\,|\,\mathbf{u}) , &\\
&\hspace*{-15mm}\mbox{если} \, 
\mathbf{c}^{(i+1)} \geq \mathbf{0},\\
\mathbf{c}^{i}, &\hspace*{-15mm}\mbox{если}\, \mathbf{c}^{(i+1)} < \mathbf{0};
\end{cases}
$$
$$
 \mathbf{c}^{(n+1)} \Rightarrow W^{(i+1)}, E^i{(i+1)};
$$
$$
X^{(i+1)} = E^{(i+1)} \left\{\left[B^{(i+1)} \left(U Q^{(i+1)}\right)\right] 
W^{(i+1)}\right\};
$$
$$
\mathcal{F}^{(i+1)} = \mathcal{H}^{(i+1)} + \mathcal{J}^{(i+1)};
$$

\item[(\textit{в})] \textit{условие остановки}:
$$
\mbox{если } \mathcal{F}^{(i+1)} - \mathcal{F}^{(i)} \le \Delta,\, \mbox{то } 
\mbox{STOP}.
$$
\end{itemize}

\vspace*{-18pt}

\section{Энтропийно-рандомизированное проектирование (REDR-метод)}

\vspace*{-2pt}

Рассмотрим матрицу данных $U_{(m \times s)}$. В~пространстве~$R^s$ ее 
отображает множество точек\linebreak $\mathfrak{U} \hm= \{\mathbf{u}^{(1)}, \dots, 
\mathbf{u}^{(m)}\}$. Будем придерживаться использованной ранее интерпретации 
<<объекты ($m$)\,--\,приз\-на\-ки ($s$)>>. Объекты обычно выбираются из одного 
класса, что позволяет выдвинуть гипотезу о~том, что расстояние между точками 
в множестве~$\mathfrak{U}$ флуктуирует несильно, т.\,е.\ точки образуют 
достаточно <<компактную>> группу.

Определим \textit{индикатор} этой группы (матрицы данных) в виде:

\noindent
\begin{equation}
\label{4_1}
\rho_U = \fr{2}{m(m-1)} \sum\limits^m_{(\alpha,\beta)=1} 
\varrho(\mathbf{u}^{\alpha}, \mathbf{u}^{\beta}).
\end{equation}

Матрицу данных $U_{(m \times s)}$ трансформируем 
в мат\-ри\-цу-про\-ек\-цию $Z_{(n \times r)}$, $n \hm< m$, $r \hm< s$, 
с~по\-мощью левых и правых мат\-риц-про\-ек\-то\-ров:
\begin{equation*}
%\label{4_2}
Z_{(n \times r)} = B_{(n \times m)} U_{(m \times s)} Q_{(s \times r)}.
\end{equation*}
Матрицы $B$ и~$Q$~--- случайные, интервального типа:
\begin{equation*}
%\label{4_3}
Q \in \mathcal{Q} = [Q^-, Q^+]\,;\qquad B \in \mathcal{B} = [B^-, B^+]\,.
\end{equation*}

\noindent
Совместная функция плотности распределения ве\-ро\-ят\-ности (ПРВ)~$P(Q,B)$  определена на 
носителе~$\mathcal{Z}$:
\begin{equation*}
%\label{4_4}
(Q,B) \in \mathcal{Z} = \mathcal{Q} \bigcap \mathcal{B}.
\end{equation*}
Элементы матрицы-проекции $\mathcal{Z}$ имеют вид:
\begin{multline*}
\label{4_5}
z_{\mu,\nu}(Q,B) = \sum\limits^m_{\beta=1} b_{\mu,\beta} 
\sum\limits^r_{\alpha=1} u_{\beta,\alpha}\,q_{\alpha,\nu},\\ \mu = 
\overline{1,n},\,\nu = \overline{1,r}.
\end{multline*}
По аналогии с~(\ref{4_1}) определим индикатор мат\-ри\-цы-про\-ек\-ции $Z_{(n 
\times s)}$ в виде:
\begin{equation*}
%\label{4_6}
\rho_Z(Q,B) = \fr{2}{n(n-1)}\sum\limits^n_{(\eta, 
\kappa)=1}\!\!\varrho(\mathbf{z}^{(\eta)}(Q,B), \mathbf{z}^{(\kappa)}(Q,B)).
\end{equation*}
Поскольку элементы мат\-риц-про\-ек\-то\-ров~--- случайные, индикатор $\rho_Z(Q,B)$ 
является функцией случайных переменных. Его математическое ожидание:
\begin{equation*}
%\label{4_6}
G[P(Q,B)] = \int\limits_{\mathcal{Z}} P(Q,B) \rho_Z(Q,B)\,dQ dB\,.
\end{equation*}

Для определения функции ПРВ $P(Q,B)$ будем использовать оценку максимальной 
энтропии~\cite{Popkov_2020_dan}
\begin{multline}
\label{4_7}
\mathcal{H[P(Q,B)]} ={}\\
{}= - \int\limits_{\mathcal{Z}} P(Q,B) \ln P(Q,B)\, dQ dB 
\Rightarrow \max
\end{multline}
при ограничениях:
\begin{multline}
%\left.
%\begin{array}{l}
\displaystyle \int\limits_{\mathcal{Z}} P(Q,B)\,dQ dB = 1;\ G[P(Q,B)] = \delta \rho_U,\\
0 < \varepsilon \le \delta \le \theta < 1.
%\end{array}
%\right\}
\label{4_8}
\end{multline}

Задача (\ref{4_7})--(\ref{4_8}) относится к классу ляпуновских 
задач~\cite{Joffe_1974}, для которых условия оптимальности формулируются в 
терминах стационарности функционала Лагранжа:
\begin{equation*}
%\label{4_9}
\mathcal{L}[P(Q,B), \lambda] = \mathcal{H[P(Q,B)]} + \lambda \left(\delta 
\rho_U - G[P(Q,B)]\right),
\end{equation*}
где $\lambda$~--- скалярный множитель Лагранжа.

Получим:
\begin{equation}
\label{4_10}
P^*(Q,B) = 
\fr{\exp\left(- \lambda \rho_Z(Q,B) \right)}{\mathcal{P}(\lambda)}\,,
\end{equation}
где

\noindent

\begin{equation*}
%\label{4_11}
\mathcal{P}(\lambda) = \int\limits_{\mathcal{Z}} \exp\left(- \lambda 
\rho_Z(Q,B)\right)\,dQ dB\,.
\end{equation*}
Множитель Лагранжа $\lambda$ определяется из следующего уравнения:
\begin{equation*}
%\label{4_12}
\fr{\int\nolimits_{\mathcal{Z}} \exp\left(- \lambda \rho_Z(Q,B) \right) 
\rho_Z(Q,B)\, dQ dB}{\mathcal{P}(\lambda)} = \delta \rho_U.
\end{equation*}
Таким образом, энтропийно-оптимальная функция ПРВ~$P^*(Q,B)$~(\ref{4_10}) 
позволяет, путем ее семплирования, генерировать 
мат\-ри\-цы-про\-ек\-то\-ры~$Q$ и~$B$, сохраняющие <<в среднем>> расстояние 
между точками (векторами $\mathbf{z}^{(\alpha)}$) 
мат\-ри\-цы-про\-ек\-ции~$\mathcal{Z}$.

\section{Энтропийные случайные матрицы-проекторы с~заданными значениями 
элементов}

Рассмотрим матрицу данных $U_{(m \times s)}$, которую нужно <<сжать>> по 
переменной~$s$ до размера~$r$:
\begin{equation}
\label{5_1}
Y_{(m \times r)} = U_{(m \times s)} Q_{(s \times r)}.
\end{equation}
Матрица $U_{(m \times s)} \ge 0$ и имеет нормированные элементы ($0\hm \le 
u_{ij} \hm\le 1$). Введем полезные обозначения:
\begin{itemize}
\item вектор-столбец $^{(\diamond)}\bullet$, век\-тор-стро\-ка 
$\bullet^{(\diamond)}$;
\item векторы-строки: $\mathbf{u}^{(i)} \hm= \{u_{i,1}, \dots, u_{i,s}\}$, $i 
\hm= \overline{1,m}$;
\item векторы-строки  $\mathbf{y}^{(k)} \hm= \{y_{k,1}, \dots, y_{k,r}\}$, 
$k\hm = \overline{1,m}$;
\item векторы-столбцы  $^{(l)}\mathbf{q}\hm = \{q_{1,l}, \dots, q_{s,l}\}$, 
$l\hm = \overline{1,s}$.
\end{itemize}
Тогда равенство~(\ref{5_1}) представим в виде:

%\pagebreak

\noindent
\begin{multline*}
%\label{5_2}
\mathbf{y}^{(i)}(Q) = \left\{^{(1)}\mathbf{q}^\intercal\,\mathbf{u}^{(i)}, 
\dots, ^{(r)}\mathbf{q}^\intercal\,\mathbf{u}^{(i)}\right\} \in R^r,\\
 i =  \overline{1,m}.
\end{multline*}
Определим индикатор матрицы-проекции $Y_{(m \times r)}$ в~виде:
\begin{equation}
\label{5_3}
\!\!\!\rho_Y(Q) = \fr{2! (m-2)!}{m!} \sum\limits^m_{(i,j)=1}\!\!\|\mathbf{y}^{(i)}(Q) - 
\mathbf{y}^{(j)}(Q)\|^2.\!\!
\end{equation}

Рассмотрим случай, когда элементы матрицы $Q_{(s \times r)}$ могут принимать 
значения~$0$ или~$1$ и размещение их в матрице~--- случайное. Заменим матрицу 
строкой длины~$sr$. Число различных последовательностей из~$0$ и~$1$ равно $N 
\hm= 2^{rs}$. В~качестве примера для $s \hm= 3$ и~$r \hm= 1$ таких реализаций 
будет~8:
$$
000, 100, 010, 001, 110, 011, 101, 111;
$$
для $s = 4$ и~$r = 1$ их будет~16:
$$
0000, 1000, 0100, 0010, 0001, 1100, 0110, 0011,
$$
$$
1001, 0101, 1010, 1111, 1110, 0111, 1011, 1101.
$$
Для матрицы-проектора $Q_{(s \times r)}$ существует конечное число ее 
$(0,1)$-реализаций:
\begin{equation*}
%\label{5_4}
Q^{(1)}, \dots,  Q^{(N)},\qquad N = 2^{sr}.
\end{equation*}
Полагая, что реализации~--- случайные, их вероятностные свойства будем 
характеризовать  функцией распределения вероятностей (ДРВ) с дискретным 
носителем~$W(\alpha)$, $\alpha \hm= \overline{1,N}$, где
\begin{equation*}
%\label{5_4a}
W(\alpha) = w_{\alpha},\quad 0 \le w_{\alpha} \le 1,\quad l = \overline{1,N}.
\end{equation*}
Математическое ожидание индикатора~(\ref{5_3}):
\begin{equation*}
%\label{5_3a}
\mathcal{M}_{W} = \sum\limits^N_{\alpha=1} w_{\alpha} \rho(Q^{(\alpha)}).
\end{equation*}
Функцию ДРВ $W(\alpha)$ будем искать в классе функций, максимизирующих 
функцию информационной энтропии Ферми~\cite{Popkov_1995}:
\begin{multline}
\label{5_5}
\!\!F(W) = - \sum\limits^N_{\alpha=1} w_{\alpha} \ln w_{\alpha} + (1 - 
w_{\alpha})\ln(1 - w_{\alpha}) \Rightarrow{}\\
{}\Rightarrow \max\,,
\end{multline}
при ограничении математического ожидания индикатора~(\ref{5_3}):
\begin{multline}
\label{5_6}
G[\mathbf{w}\,|\,Q^{(1)}, \dots, Q^{(N)}] = \sum\limits^N_{\alpha=1} 
w_{\alpha} \rho(Q^{\alpha}) = \delta\,\rho_U,\\
 0 < \varepsilon \le 
\delta \le 1\,.
\end{multline}
Задача (\ref{5_5})--(\ref{5_6}) представляет собой конечномерную задачу на 
условный экстремум с вогнутой целевой функцией и квадратичным ограничением. 
Хотя последнее требует специального исследования, но, поскольку оно одно, 
решение несложно найти чис\-ленно.

Рассмотрим функцию Лагранжа:
\begin{equation*}
%\label{5_7}
L(\mathbf{w}, \lambda) = F(\mathbf{w}) + \lambda \left(\delta\,\rho_U - 
\sum\limits^N_{\alpha=1} w_{\alpha} \rho(Q^{\alpha})\right).
\end{equation*}
Условия стационарности этой функции имеют вид $(\alpha \hm= \overline{1,N})$:
\begin{align*}
%\!\left.
%\begin{array}{rl}
\!\displaystyle\fr{\partial L}{\partial w_{\alpha}} &= - \ln \fr{w_{\alpha}}{1 - w_{\alpha}} 
- \lambda \rho(Q^{\alpha}) = 0\,;\\[6pt]
\!\displaystyle\fr{\partial L}{\partial \lambda} &= \displaystyle\left(\delta\,\rho_U - 
\sum\limits^N_{\alpha=1} w_{\alpha} \rho(Q^{\alpha})\right) = 0\,.
%\end{array}\!
%\right\}\!\!\!\!
%\label{5_8}
\end{align*}
Отсюда получаем, что энтропийно-оптимальное распределение вероятностей имеет 
вид:
\begin{equation}
\label{5_9}
w^*_{\alpha} = \fr{\exp\left(- \lambda \rho(Q^{\alpha}) \right)}{1 + 
\exp\left(- \lambda \rho(Q^{\alpha}) \right)},\qquad \alpha = \overline{1,N},
\end{equation}
где параметр $\lambda$ определяется из следующего уравнения:
\begin{equation*}
%\label{5_10}
\sum\limits^N_{\alpha=1}\frac{\exp(- \lambda \rho(Q^{(\alpha)})) 
\rho(Q^{(\alpha)})}{1 + \exp(- \lambda  \rho(Q^{(\alpha)}))} = \delta \rho_U.
\end{equation*}
Таким образом, равенство~(\ref{5_9}) определяет распределение вероятностей 
мат\-риц-про\-ек\-то\-ров с~элементами $\{0,1\}$. Имеет смысл выбрать 
мат\-ри\-цу-про\-ектор:
\begin{equation*}
%\label{5_11}
Q^{(\alpha^*)} \Rightarrow \alpha^* = \max_{1 \le \alpha \le N} w^*_{\alpha},
\end{equation*}
хотя возможны и другие стратегии.

\begin{center}
\textbf{Алгоритм}
\end{center}

\noindent
\textbf{Шаг 0.}\ \textit{Нормировка матрицы данных:}
$$
u_{ij} := \fr{u_{ij} - u_{\min}}{u_{\max} - u_{min}},\qquad i = 
\overline{1,m};\,j = \overline{1,s}.
$$

\noindent
\textbf{Шаг 1.}\ \textit{Вычисление индикатора матрицы данных:}
$$
\rho_U := \fr{2!(m-2)!}{m!} \sum\limits^m_{\beta, \gamma = 1,\,\gamma \ne 
\beta} \varrho(\mathbf{u}^{(\beta)}, \mathbf{u}^{(\gamma)}).
$$


\noindent
\textbf{Шаг 2.}\ \textit{Генерация множества $\mathfrak{Q}$ мат\-риц-про\-ек\-то\-ров с 
элементами $0,1$}:
$$
Q^{(\alpha)},\qquad \alpha = \overline{1,N},\,\,\,N = 2^{rs}.
$$

\begin{figure*} %fig1
\vspace*{1pt}
\begin{minipage}[t]{80mm}
    \begin{center}  
  \mbox{%
 \epsfxsize=78.264mm 
 \epsfbox{pop-1.eps}
 }
\end{center}
\vspace*{-11pt}
\Caption{Множество $\mathfrak{U}$ трехмерных точек}
\label{fig1a}
\end{minipage}
%\end{figure*}
\hfill
%\begin{figure*} %fig2
\vspace*{1pt}
\begin{minipage}[t]{80mm}
    \begin{center}  
  \mbox{%
 \epsfxsize=78.952mm 
 \epsfbox{pop-2.eps}
 }
\end{center}
\vspace*{-11pt}
\Caption{Множество $\mathfrak{Y}$ двумерных точек с $\rho_Y\hm = 
\rho_U$}\label{fig1b}
\end{minipage}
\vspace*{6pt}
\end{figure*}

\noindent
\textbf{Шаг 3.}\ \textit{Формирование множества $\mathfrak{Y}$ мат\-риц-про\-ек\-ций с 
элементами}
$$
y^{(\alpha)}_{i,k} := \sum\limits^s_{\nu=1} u_{i,\nu} 
q^{(\alpha)}_{\nu,k},\qquad i = \overline{1,m},\,k = \overline{1,r},
$$
\textit{и векторами-строками}
$$
\mathbf{y}^{\alpha}_{\nu} := \left\{y^{(\alpha)}_{\nu,1}, \dots, 
y^{(\alpha)}_{\nu,r} \right\},\qquad \nu = \overline{1,m}.
$$

\noindent
\textbf{Шаг 4.}\ \textit{Вычисление индикатора мат\-риц-про\-ек\-ций}:
$$
\rho^{(\alpha)}_Y := \fr{2!(m-2)!}{m!} \sum\limits^m_{\nu, \mu = 1,\,\nu \ne 
\mu} \!\!\varrho(\mathbf{y}^{(\nu)}, \mathbf{y}^{(\mu)}), \ \alpha = 
\overline{1,N}\,.
$$

\noindent
\textbf{Шаг 5.} \textit{Определение значения множителя Лагранжа~$\lambda^*$ из 
уравнения}:
$$
\sum\limits^N_{\alpha=1}\fr{\exp(- \lambda\, \rho^{(\alpha)}_Y)\,\, 
\rho^{(\alpha)}_Y}{1 + \exp(- \lambda\,\rho^{(\alpha)}_Y)} = \delta \rho_U.
$$

\noindent
\textbf{Шаг 6.}\  \textit{Определение в множестве~$\mathfrak{Q}$ наиболее вероятной 
мат\-ри\-цы-про\-ек\-тора:}
$$
Q^{(\alpha^*)}: \,\alpha^* = \argmax\limits_{\alpha} w(\alpha\,|\,\lambda^*).
$$

\noindent
\textbf{Шаг 7.}\ \textit{Определение элементов наиболее вероятной 
мат\-ри\-цы-про\-ек\-ции:}
$$
Y^{(\alpha^*)}_{(m \times r)} = U_{(m \times s)} Q^{(\alpha^*)}_{(s \times 
r)}.
$$

\noindent
{STOP} 

\smallskip

\noindent
\textbf{Пример.} В~качестве иллюстрации предлагаемого алгоритма рассмотрим 
матрицу данных $U_{(100 \times 3)}$, элементами которой служат случайные 
числа из интервала $[0,1]$. В~пространстве~$R^3$ эта матрица отображается в 
множество~$\mathfrak{U}$ трехмерных точек, характеризуемых векторами-строками 
$\mathbf{u}^{(i)}$, $i \hm= \overline{1,100}$. Это множество изображено на 
рис.~\ref{fig1a}.
Расстояния между точками~--- евклидовы, и индикатор матрицы данных $\rho_U 
\hm= 0{,}677$.

Применяя рассмотренный выше алгоритм с параметром $\delta \hm= 1{,}0$, 
генерируем множество~$\mathfrak{Y}$ двумерных точек с $\rho_Y \hm= \rho_U$, 
показанное на рис.~\ref{fig1b}. Множество~$\mathfrak{Y}$ локализовано в 
квадрате со стороной~1,7.



\section{Заключение}

Предложены процедуры детерминированного и~рандомизированного проектирования, 
ориентированные на редукцию размерности матрицы данных. В~случае 
детерминированного проектирова\-ния развивается параллельная процедура сжатия 
матрицы данных, минимизирующая кросс-энт\-ро\-пию Куль\-ба\-ка--Лейб\-ле\-ра с 
учетом ограничения на информационную емкость.  Предложен алгоритм условной 
минимизации, использующий метод проекций градиента.

Для рандомизированного проектирования рассмотрена задача редукции матрицы 
данных по одному измерению и с заданными элементами 
мат\-риц-про\-ек\-то\-ров, в частности со случайными\linebreak
 $(0,1)$-эле\-мен\-та\-ми. Задача сводится к 
поиску распределения вероятностей, максимизирующего\linebreak информационную энтропию 
Ферми при ограничении на среднее расстояние между точками многообразия, 
которые отображаются матрицами данных и~оптимальной проекции. Предложен 
алгоритм для решения этой задачи.

{\small\frenchspacing
 {%\baselineskip=10.8pt
 %\addcontentsline{toc}{section}{References}
 \begin{thebibliography}{99}

\bibitem{Bruckstein_2009}
\Au{Bruckstein A.\,M., Donoho~D.\,L., Elad~M.} From sparse solutions of 
systems of equations to sparse modeling of signals and images~// SIAM Rev., 
2009. Vol.~51. Iss.~1. P.~34--81.

\bibitem{Kendall_1973}
\Au{Кендалл М., Стьюарт~А.} Статистические выводы и~связи~/
Пер. с~англ.~--- М.:~Наука, 
1973. 896~с. (\Au{Kendall~M.\,G., Stuart~A.} {The advanced theory of statistics.}~--- 
London: Charles Griffin, 1961.  Vol.~2. 676~p.)

\bibitem{Jolliffe_2011}
\Au{Jolliffe I.} Principal component analysis.~--- New York, NY, USA: 
Springer, 2011. 488~p. doi: 10.1007/b98835.

\bibitem{Polyak_2017}
\Au{Поляк Б.\,Т., Хлебников~М.\,В.} Метод главных компонент: робастные 
версии~// Автоматика и телемеханика, 2017. №\,3. C.~130--148.

\bibitem{Bingham_2001} %5
\Au{Bingham E., Mannila~H.} Random projection in dimensionality reduction: 
Applications to image and text data~// 7th ACM SIGKDD Conference 
(International) on Knowledge Discovery and Data Mining Proceedings.~--- New 
York, NY, USA: ACM, 2001. P.~245--250. doi: 10.1145/502512.502546.

\bibitem{Vempala_2005} %6
\Au{Vempala S.\,S.} The random projection method.~--- \mbox{DIMACS} ser. in discrete 
mathematics and theoretical computer science.~--- Providence, RI, USA: 
American Mathematical Society, 2004. Vol.~65. 105~p.

\bibitem{Popkov_2018_at_a_en}
\Au{Попков Ю.\,С., Дубнов~Ю.\,А., Попков~А.\,Ю.} Энтропийная редукция 
размерности в задачах рандомизированного машинного обучения~// Автоматика и 
телемеханика, 2018. №\,11. С.~106--122.

\bibitem{Kullback_1951} %8
\Au{Kullback S., Leibler~R.\,A.} On information and sufficiency~// Ann. 
Math. Stat., 1951. Vol.~22. Iss.~1. P.~79--86.

\bibitem{Popkov_2019_dan}
\Au{Попков Ю.\,С., Попков~А.\,Ю.} Кросс-энтропийная оптимальная редукция 
размерности матрицы данных с ограничением информационной емкости~// Докл. 
Акад. наук, 2019. Т.~488. С.~21--23. doi: 10.31857/S0869-5652488121-23.

\bibitem{Magnus_1988}
\Au{Magnus J.\,R., Neudecker~H.} Matrix differential calculus with 
applications in statistics and econometrics.~--- 
Chichester\,--\,New York\,--\,Brisbane\,--\,Toronto\,--\,Singapore: John 
Wiley \& Sons, 1988. 393~p.

\bibitem{Popkov_2020_dan}
\Au{Попков Ю.\,С.} Асимптотическая эффективность оценок максимальной 
энтропии~// Докл. Акад. наук, 2020. Т.~493. С.~104--107. {doi: 
10.31857/ S2686954320040165}.

\bibitem{Joffe_1974}
\Au{Иоффе А.\,Д., Тихомиров~В.\,М.} Теория экстремальных задач.~--- М.: 
Наука, 1984. 481~с.

\bibitem{Popkov_1995}
\Au{Popkov Yu.\,S.} Macrosystems theory and its applications.~--- Lecture notes 
in control and information sciences ser.~--- Berlin--Heidelberg: 
Springer-Verlag, 1995. Vol.~203. 327~p.

\end{thebibliography}

 }
 }

\end{multicols}

\vspace*{-6pt}

\hfill{\small\textit{Поступила в~редакцию 25.12.19}}

\vspace*{7pt}

%\pagebreak

%\newpage

%\vspace*{-28pt}

\hrule

\vspace*{2pt}

\hrule

%\vspace*{-2pt}

\def\tit{DETERMINISTIC AND~RANDOMIZED METHODS OF~ENTROPY 
PROJECTION FOR~DIMENSIONALITY REDUCTION PROBLEMS}


\def\titkol{Deterministic and randomized methods of entropy projection for 
dimensionality reduction problems}


\def\aut{Y.\,S.~Popkov$^{1,2,3}$, A.\,Y.~Popkov$^1$, and~Y.\,A.~Dubnov$^{1,4}$}

\def\autkol{Y.\,S.~Popkov, A.\,Y.~Popkov, and Y.\,A.~Dubnov}

\titel{\tit}{\aut}{\autkol}{\titkol}

\vspace*{-11pt}


\noindent
$^1$Federal Research Center ``Computer Science and Control'' of the Russian Academy of Sciences, 
44-2~Vavilov\linebreak
$\hphantom{^1}$Str., Moscow 119333, Russian Federation

\noindent
$^2$V.\,A. Trapeznikov Institute of Control Sciences, Russian Academy of Sciences, 
65~Profsoyuznaya Str., Moscow\linebreak
$\hphantom{^1}$117997, Russian Federation

\noindent
$^3$ORT Braude College, Karmiel 2161002, Israel

\noindent
$^4$National Research University Higher School of Economics, 20~Myasnitskaya Str., Moscow 
101000, Russian\linebreak
$\hphantom{^1}$Federation


\def\leftfootline{\small{\textbf{\thepage}
\hfill INFORMATIKA I EE PRIMENENIYA~--- INFORMATICS AND
APPLICATIONS\ \ \ 2020\ \ \ volume~14\ \ \ issue\ 4}
}%
 \def\rightfootline{\small{INFORMATIKA I EE PRIMENENIYA~---
INFORMATICS AND APPLICATIONS\ \ \ 2020\ \ \ volume~14\ \ \ issue\ 4
\hfill \textbf{\thepage}}}

\vspace*{3pt} 


\Abste{The work is devoted to development of methods for deterministic and randomized projection 
aimed at dimensionality reduction problems. In the deterministic case, the authors develop the parallel 
reduction procedure minimizing Kullback--Leibler cross-entropy target to condition on information 
capacity based on the gradient projection method. In the randomized case, the authors solve the 
problem of reduction of feature space. The idea of application of projection procedures for reduction 
of data matrix is implemented in the proposed method of randomized entropy projection where the 
authors use the principle of keeping average distances between high- and low-dimensional 
points in the corresponding spaces. The problem leads to searching of a~probability distribution maximizing 
Fermi entropy target to average distance between points.}

\KWE{dimensionality reduction; Kullback--Leibler cross-entropy; entropy}

\DOI{10.14357/19922264200407} 

\vspace*{-18pt}

\Ack

\vspace*{-2pt}

\noindent
This work was supported by RFBR, projects Nos.\,17-29-03119 and 20-07-00470.

%\vspace*{6pt}

  \begin{multicols}{2}

\renewcommand{\bibname}{\protect\rmfamily References}
%\renewcommand{\bibname}{\large\protect\rm References}

{\small\frenchspacing
 {%\baselineskip=10.8pt
 \addcontentsline{toc}{section}{References}
 \begin{thebibliography}{99}

\bibitem{1-pop}
\Aue{Bruckstein, A.\,M., D.\,L.~Donoho, and M.~Elad.} 2009. From sparse solutions of systems of 
equations to sparse modeling of signals and images. \textit{SIAM Rev.}
 51(1):34--81.
\bibitem{2-pop}
\Aue{Kendall, M.\,G., and A.~Stuart.} 1961. \textit{The advanced theory of statistics.} 
London: Charles Griffin.  Vol.~2. 676~p.

\columnbreak


\bibitem{3-pop}
\Aue{Jolliffe, I.} 2011. \textit{Principal component analysis}. New York, NY: Springer. 488~p. 
doi: 10.1007/b98835.
\bibitem{4-pop}
\Aue{Polyak, B.\,T., and M.\,T.~Khlebnikov.} 2017. Principal component analysis: Robust versions. 
\textit{Automat. Rem. Contr.} 78:490--506.
\bibitem{5-pop}
\Aue{Bingham, E., and H.~Mannila.} 2001. Random projection in dimensionality reduction: 
Applications to image and text data. \textit{7th ACM SIGKDD Conference (International) on 
Knowledge Discovery and Data Mining Proceedings}. ACM. 245--250. doi: 10.1145/502512.502546.

\bibitem{6-pop}
\Aue{Vempala, S.\,S.} 2004. \textit{The random projection method}. \mbox{DIMACS} ser. in discrete 
mathematics and theoretical computer science.
Providence, RI: 
American Mathematical Society. Vol.~65. 105~p. 
\bibitem{7-pop}
\Aue{Popkov, Y.\,S., Y.\,A.~Dubnov, and A.\,Y.~Popkov.} 2018. Entropy dimension reduction 
method for randomized machine learning problems. \textit{Automat. Rem. Contr.} 79(11): 2038--
2051. 
\bibitem{8-pop}
\Aue{Kullback, S., and R.\,A.~Leibler.} 1951. On information and sufficiency. \textit{Ann. 
Math. Stat.} 22(1):79--86.


\bibitem{9-pop}
\Aue{Popkov, Y.\,S., and A.\,Y.~Popkov.} 2019. Cross-entropy optimal dimensionality reduction 
with a condition on information capacity. \textit{Dokl. Math.}
 100:420--422.
 
 \columnbreak
 
\bibitem{10-pop}
\Aue{Magnus, J.\,R., and H.~Neudecker.} 1988. \textit{Matrix differential calculus with 
applications in statistics and econometrics}. 
Chichester\,--\,New York\,--\,Brisbane\,--\,Toronto\,--\,Singapore: John Wiley \& Sons. 393~p.
\bibitem{11-pop}
\Aue{Popkov, Y.\,S.} 2020.
 Asymptotic efficiency of maximum entropy estimates. \textit{Dokl. Math.} 102:350--352.
 doi: org/ 10.1134/S106456242004016X.
\bibitem{12-pop}
\Aue{Joffe, A.\,D., and V.\,M.~Tikhomirov.} 1984. \textit{Teoriya eks\-tre\-mal'-nykh zadach} [Theory 
of extreme problems]. Moscow: Nauka. 481~p.
\bibitem{13-pop}
\Aue{Popkov, Yu.\,S.} 1995. \textit{Macrosystems theory and its applications}.
Lecture notes 
in control and information sciences ser.
 Berlin--Heidelberg: Springer-Verlag. Vol.~203.\linebreak 327~p.
 
 
\end{thebibliography}

 }
 }

\end{multicols}

\vspace*{-3pt}

\hfill{\small\textit{Received December 25, 2019}}

%\pagebreak

%\vspace*{-24pt}


\Contr

\noindent
\textbf{Popkov Yuri S.} (b.\ 1937)~--- Doctor of Science in technology, professor, Academician of 
RAS, principal scientist, Federal Research Center ``Computer Science and Control'' of the Russian 
Academy of Sciences, 44-2~Vavilov Str., Moscow 119333, Russian Federation; principal scientist, 
V.\,A.~Trapeznikov Institute of Control Sciences, Russian Academy of Sciences, 65~Profsoyuznaya 
Str., Moscow 117997, Russian Federation; senior research fellow, ORT Braude College, Karmiel 
2161002, Israel; \mbox{popkov@isa.ru}

\vspace*{3pt}

\noindent
\textbf{Popkov Alexey Y.} (b.\ 1978)~--- Candidate of Science (PhD) in technology, leading 
scientist, Federal Research Center ``Computer Science and Control'' of the Russian Academy of 
Sciences, 44-2~Vavilov Str., Moscow 119333, Russian Federation; \mbox{apopkov@isa.ru}

\vspace*{3pt}

\noindent
\textbf{Dubnov Yuri A.} (b.\ 1990)~--- scientist, Federal Research Center ``Computer Science and 
Control'' of the Russian Academy of Sciences, 44-2~Vavilov Str., Moscow 119333, Russian 
Federation; senior lecturer, National Research University Higher School of Economics, 
20~Myasnitskaya Str., Moscow 101000, Russian Federation; \mbox{yury.dubnov@phystech.edu}

\label{end\stat}

\renewcommand{\bibname}{\protect\rm Литература}            %07
\include{potanin-strijov}   %08
\include{sokolov-dyachenko} %09
\include{budzko-sochenkov}  %10
   \def\stat{rum-kir}
   
   \def\tit{МЕТОД ВИЗУАЛЬНОГО ПРЕДСТАВЛЕНИЯ КОНФЛИКТОВ 
В~ГИБРИДНЫХ ИНТЕЛЛЕКТУАЛЬНЫХ МНОГОАГЕНТНЫХ 
СИСТЕМАХ}
   
   \def\titkol{Метод визуального представления 
конфликтов в~гибридных интеллектуальных 
многоагентных системах}
   
   \def\aut{С.\,Б.~Румовская$^1$, И.\,А.~Кириков$^2$}
   
   \def\autkol{С.\,Б.~Румовская, И.\,А.~Кириков}
   
   \titel{\tit}{\aut}{\autkol}{\titkol}
   
   \index{Румовская С.\,Б.}
  \index{Кириков И.\,А.}
   \index{Rumovskaya S.\,B.}
  \index{Kirikov I.\,A.}
   
   
   %{\renewcommand{\thefootnote}{\fnsymbol{footnote}} \footnotetext[1]
   %{Работа выполнена при частичной поддержке РФФИ (проект 19-07-00187-A).}}
   
   
   \renewcommand{\thefootnote}{\arabic{footnote}}
   \footnotetext[1]{Калининградский филиал Федерального исследовательского центра 
<<Информатика и~управление>> Российской академии наук, \mbox{sophiyabr@gmail.com 
}}
   \footnotetext[2]{Калининградский филиал Федерального исследовательского центра 
<<Информатика и~управление>> Российской академии наук, 
\mbox{baltbipiran@mail.ru}}
   
   %\vspace*{-12pt}

   \Abst{Малые коллективы экспертов, включающие специалистов различных направлений, 
эффективно решают сложные проблемы благодаря их анализу с~различных точек зрения 
и~получению более качественного интегрированного решения. Конфликт в~малых коллективах 
экспертов может как завести в~тупик процесс принятия решения, так и~породить позитивные 
изменения: развитие группы, диагностику отношений, сплачивание группы. Конфликт 
порождает дискуссии, позволяющие получить более продуманные и~согласованные 
решения. Подобные коллективы эффективно решают проблемы, и~моделирование их работы, 
в~частности возможной конфликтной ситуации и~процесса управления ею, позволяет 
вырабатывать метод решения, релевантный сложной задаче. Визуализация конфликтной 
ситуации делает возникшие противоречия контрастными, видимыми. В~работе коллектив 
агентов-экспертов представляется в~виде неориентированного взвешенного графа 
и~рассматриваются методы визуализации (укладки) графов. Для визуализации  
проб\-лем\-но- и~про\-цес\-сно-ори\-ен\-ти\-ро\-ван\-ных конфликтов в~рамках гибридных 
интеллектуальных многоагентных систем (ГиИМАС) предложен метод, разработанный на базе 
пружинной модели укладки графов.}
    
  \KW{коллектив экспертов; конфликт агентов; визуализация конфликта}

   \DOI{10.14357/19922264200411} 
     
   \vspace*{2pt}
   
   
   \vskip 10pt plus 9pt minus 6pt
   
   \thispagestyle{headings}
   
   \begin{multicols}{2}
   
   \label{st\stat}

\section{Введение}

  Малые коллективы экспертов, включающие специалистов различных 
направлений, эффективно решают проблемы благодаря их всестороннему 
анализу, получая интегрированное компромиссное качественное решение~[1]. 
Как следствие, исследование методов коллективного решения проблем и~их 
моделирование~--- важное направление научных исследований в~области 
системного анализа, имеющее большое практическое значение для медицины, 
транспорта и~логистики, экономики и~т.\,д. %\linebreak 
В~[2] разработана и~описана 
модель самоорганизации в~коллективе агентов (экспертов) \mbox{гибридными}
интеллектуальными многоагентными системами, алгоритм 
функционирования которых динамически перестраивается, вырабатывая 
релевантный проблеме метод решения. 
  
  С другой стороны, при решении проблем малыми коллективами снижается 
скорость принятия решений из-за процессов распределения задач 
и~интеграции частных решений, а также возникновения конфликтов, которые 
могут носить как деструктивный, так и~конструктивный характер. 

Конструктивные конфликты позволяют получить более продуманные 
и~согласованные решения~[3], обеспечивают уникальность и~автономность 
каждого из взаимодействующих субъектов, а также развитие отношений 
между ними, предоставляют информацию о~возможностях 
противодействующих субъектов, высвобождают накапливающееся внут\-рен\-нее 
напряжение, сохраняя связи, актуализируют разные позиции и~мнения по 
поводу возникающих проблем, способствуя поиску оптимальных способов их 
решения, усиливают групповую идентичность и~сплоченность. В~связи с~этим 
для повышения релевантности и~качества принимаемых решений 
гибридного и~синергетического искусственного интеллекта, моделирующего работу 
малого коллектива экспертов, в~[4] предложена модель \mbox{ГиИМАС} 
с~проблемно- и~про\-цес\-сно-ори\-ен\-ти\-ро\-ван\-ны\-ми конфликтами, 
а~в~[5] описан метод идентификации конфликтов между агентами в~рамках 
предложенной в~[4] модели. 

Управление конфликтами в~\mbox{ГиИМАС} 
позволит по аналогии с~реальными малыми коллективами экспертов подавлять 
деструктивные проявления конфликта и~стимулировать конструктивные~[3].
   
  В~[3] рассмотрены методы моделирования и~визуального представления 
конфликта в~коллективе экспертов при решении проблем, по результатам 
которого было решено использовать опыт моделирования конфликтов  
ло\-ги\-ко-струк\-тур\-ным методом~[6, 7] с~помощью графов. 

Цель настоящей 
работы~--- разработка метода визуализации конфликтов на базе предложенной 
модели~[4] и~метода их идентификации~[5], который сделает возникшие 
противоречия между агентами \mbox{ГиИМАС} контрастными, видимыми 
для пользователя системы. При визуализации конфликтов в~\mbox{ГиИМАС} 
будем рассматривать ее как неориентированный взвешенный полный граф без 
петель, множество вершин которого взаимно однозначно соответствует 
множеству агентов \mbox{ГиИМАС}, реб\-ра представляют отношения между 
ними, а~вес ре\-бер~--- напряженность возникающего конфликта~[5]. 
{\looseness=-1

}

\vspace*{-6pt}
  
\section{Методы и~модели визуализации графов}

\vspace*{-2pt}

  К наиболее известным методам рисования (визуализации, укладки) графов 
относятся:
  \begin{itemize}
\item основанные на физических аналогиях (П.~Эйдэс~[8], Т.~Камада 
и~С.~Каваи~[9], T.~Фрухтерман и~Э.~Рейнгольд~[10])~--- обладают 
наибольшим потенциалом;\\[-14pt] 
\item поуровневый подход (К.~Сигуяма с~соавторами~[11]) и~восходящее или нисходящее 
представление~[12] для ориентированных графов;\\[-14pt] 
\item рисование деревьев (Э.~Рейнгольд и~Дж.~Тилфорд~[13]);\\[-14pt]
\item планарные укладки графов без пересечения ребер (В.\,Т.~Татт~[14]);\\[-14pt] 
\item ортогональные изображения графов для снижения числа пересечений ребер  
(Ди~Батиста с~соавторами~\cite{15-rum})~--- ребра изображаются прямыми, 
параллельными осям координат;\\[-14pt]
\item произвольное представление графов~\cite{12-rum};\\[-14pt] 
\item прямолинейное (ребра представляются отрезками), сеточное 
и~полигональное (для отображения ребер используются ломаные), главная 
цель которых~--- снижение числа пересечений ребер~\cite{12-rum}.
\end{itemize}

  Для укладки неориентированных взвешенных полных графов без петель 
хорошо 
 зарекомендовали себя~\cite{12-rum}  алгоритмы, основанные на физических  
аналогиях~\cite{8-rum, 9-rum, 10-rum}, в~которых строится специальная 
модель~--- вершины и~ребра графа соответствуют\linebreak <<реальным>> физическим 
взаимодействующим объектам, вводится функция энергии. Лучшая укладка\linebreak
\vspace*{-12pt} 

\columnbreak

\noindent
графа соответствует минимуму энергии сис\-те\-мы. Выделяют силовой 
(force-directed) алгоритм рисования графов и~пружинный (spring), который 
эквивалентен методу многомерного шкалирования (MDS, multidimensional 
scaling)~\cite{16-rum}. 
  
  Силовая и~пружинная модели при определенном наборе параметров дают 
совпадение минимизируемых функций энергии и,~соответственно, \mbox{похожие} 
укладки (совпадают с~точ\-ностью до поворотов и~масштабирования). Поэтому 
иногда их не различают, называя общим термином <<методы, основанные на 
физических аналогиях>> (force-directed techniques). Тем не менее силовая 
модель более <<гибкая>> за счет большего числа настраиваемых параметров: 
возможность регулировать веса объектов позволяет учитывать 
дополнительные атрибуты вершин и~ребер. Также силовую модель проще 
интегрировать и~модифицировать для учета пересечения ребер, размера 
доступного пространства для рисования и~т.\,д. Однако пружинная модель 
проще и~обладает большей производительностью. 
  
  Для дальнейшей работы выбираем пружинную модель~\cite{17-rum}, так как  
для достижения поставленной цели нет необходимости в~учете множества 
дополнительных параметров, связанных со свойствами вершин и~ребер, 
помимо весов последних. Реб\-ра графа заменяют пружинами, при растяжении и~сжатии которых 
возникают силы упругости, действующие по закону Гука и~стремящиеся 
вернуть пружине ее первоначальную длину. Энергия системы 
пружин прямо пропорциональна расстоянию между вершинами $\vert p_i\hm- 
p_j\vert$:
  $$
  E:\ \sum\limits_{(i,j)\in n} \left\vert p_i -p_j\right\vert^2\,,
  $$
где $p_i=(x_i,y_i)\hm\in R^2$ и~$p_j\hm= (x_j,y_j)\hm\in R^2$~--- образы $i$-й 
и~$j$-й вершин в~$R^2$, их позиции (координаты), $i,j\hm\in [1,n]$ (ребра 
графа отображаются на прямые, соединяющие соответствующие вершины).
  
  Для неориентированных взвешенных графов, число вершин и~ребер 
которых не превышает~200 (малые коллективы, моделируемые 
\mbox{ГиИМАС}, содержат не более 20~экспертов, преимущественно 
менее~10), наиболее часто применяется пружинная модель Т.~Камада 
и~С.~Каваи~\cite{9-rum}, на базе которой описан предлагаемый метод 
визуализации конфликтов (МВК).
     
     \vspace*{-8pt}
     
\section{Метод визуализации конфликтов между агентами} 

\vspace*{-2pt}
  
  В~[4] было введено понятие процесса управления конфликтами: 
  \begin{equation*}
  \mathrm{cnfm}=\left\langle \mathbf{CNF}, \mathrm{cnfcl}, \mathrm{cmkb}, 
  \mathrm{act_{cnfm}}, 
\mathrm{ACT_{agcr}}\right\rangle\,.
  %\label{e1-rum}
  \end{equation*}

\pagebreak

\noindent
Здесь $\mathbf{CNF}$~--- матрица, описывающая конфликт между каждой парой 
агентов кортежем, представленным выражением:
\begin{equation*}
  \mathrm{cnf}_{ij\,\mathrm{cnft}}=\left\langle \mathrm{id}_i, \mathrm{id}_j, 
  \mathrm{cnfin}, \mathrm{cnft}, \mathrm{ACT}_{\mathrm{agcr}\,i}, 
\mathrm{ACT}_{\mathrm{agcr}\,j}\right\rangle\,,
  %\label{e2-rum}
  \end{equation*}
где $\mathrm{id}_i$ и~$\mathrm{id}_j$~--- идентификаторы аген\-тов-субъ\-ек\-тов конфликта, 
$\mathrm{cnfin}\hm\in [0,1]$~--- напряженность конфликта, $\mathrm{cnft}$~--- символьная 
переменная <<тип конфликта>>, определенная на множестве 
$\mathrm{CNFT}$\;=\;\{<<проб\-лем\-но-ори\-ен\-ти\-ро\-ван\-ный>>, 
<<про\-цес\-сно-ори\-ен\-ти\-ро\-ван\-ный>>\}, $\mathrm{ACT}_{\mathrm{agcr}\,i}, \mathrm{ACT}_{\mathrm{agcr}\,j}\hm\subseteq 
\mathrm{ACT}_{\mathrm{agcr}}$~--- множества до\-пус\-ти\-мых действий агентов~$\mathrm{ag}_i$ и~$\mathrm{ag}_j$ по 
разрешению противоречий;\linebreak
  $\mathrm{cnfcl}$~--- 
классификатор конфликтов агентов, идентифицирующий их характер 
и~оценивающий напряженность, т.\,е.\ фор\-ми\-ру\-ющий для каж\-дой пары 
агентов значение элемента мат\-ри\-цы~$\mathbf{CNF}$~\cite{5-rum}; $\mathrm{cmkb}$~--- 
база знаний об эффективности методов управ\-ле\-ния конфликтами 
в~зависимости от характеристик проб\-ле\-мы и~конфликтов между агентами; 
$\mathrm{act_{cnfm}}$~--- функция  
аген\-та-фа\-си\-ли\-та\-то\-ра <<управ\-ле\-ние конфликтом>>; $\mathrm{ACT_{agcr}}$~--- 
множество до\-пус\-ти\-мых действий агентов по разрешению противоречий.
  
  
  
  Матрица $\mathbf{CNF}$ вместе с~пороговым минимальным значением 
напряженности визуализируемого конфликта~$\eta$ (задается пользователем, 
по умолчанию $\eta\hm=0$)~--- входные данные МВК.
  
  Первый шаг МВК~--- вычисление промежуточных матриц $\mathbf{CP}$, 
$\mathbf{CPR}$ и~$\mathbf{D}$. 

Элементы $\mathrm{cp}_{ij}\hm= \Pi {p}_3(\mathrm{cnf}_{ij\,\mathrm{prob}})$ 
матрицы~$\mathbf{CP}$ описывают величину напряженности проб\-лем\-но-ори\-ен\-ти\-ро\-ван\-но\-го 
конфликта между агентами. 

Элементы $\mathrm{cpr}_{ij}\hm= \Pi 
{p}_3(\mathrm{cnf}_{ij\,\mathrm{prob}})$ матрицы~$\mathbf{CPR}$ описывают величину 
напряженности про\-цес\-сно-ори\-ен\-ти\-ро\-ван\-но\-го конфликта между 
агентами.

 Элементы матрицы~$\mathbf{D}$ расстояний графа конфликтов 
рассчитываются в~соответствии с~выражением:
  $$
  d_{ij}\!=\! \begin{cases}
  0\,, & \!\mbox{если } i=j\,;\\
  0{,}00001\,, & \!\mbox{если } \mathrm{cpr}_{ij}=\mathrm{cp}_{ij}=0\,;\hspace*{-0.12793pt}\\
  \left(0{,}5 \left( \mathrm{cp}^2_{ij}+cpr^2_{ij}\right)\right)^{0{,}5} & \!\mbox{в\ 
противном\ случае.}
  \end{cases}
  $$
  
Второй шаг МВК~--- запуск алгоритма Ка\-ма\-да--Ка\-ваи~\cite{9-rum}:
\begin{enumerate}[(1)] 
\item вершины графа, представляющие агентов \mbox{ГиИМАС}, 
помещаются в~случайные координаты~$p_i$;
\item выбирается вершина~$m$, на которую действует максимальная сила;
\item остальные вершины фиксируются, энергия системы минимизируется 
двумерным методом  
Нью\-то\-на--Раф\-со\-на, вычисляется смещение для вершины~$m$;
\item шаги 2 и~3 повторяются до достижения одного из признаков останова: 
либо заданного числа итераций, либо порога силы, действующей на 
вершины, ниже которого алгоритм не запускается.
   \end{enumerate}
   
   Алгоритм Ка\-ма\-да--Ка\-ваи на выходе даст оптимальную укладку графа, 
описывающего \mbox{ГиИМАС} (соответствует состоянию с~минимальной 
суммарной энергией системы). Энергия всей системы рассчитывается как
   $$
   E=\sum\limits_{i=1}^{n-1} \sum\limits_{j=i+1}^{n} 0{,}5 k_{ij}\left( \left\vert 
p_i-p_j\right\vert -l_{ij}\right)^2\,.
   $$
Здесь $n$~--- число вершин; $k_{ij}\hm= (d_{ij})^{-2}$~--- сила пружины между 
вершинами;  $p_i$ и~$p_j$~--- положение на плоскости вершин~$i$ и~$j$ 
соответственно; $l_{ij}\hm= L_0 d_{ij} (\max\nolimits_{i<j}  
d_{ij})^{-1}$~--- идеальная длина пружины, где~$L_0$~--- длина стороны 
квадратной области дисплея.
   
   Третий шаг МВК~--- получение результирующей визуализации (см.\ 
рисунок) конфликтующих агентов. Корректируем укладку графа:
   \begin{itemize}
\item с~учетом матриц $\mathbf{CP}$ и~$\mathbf{CPR}$: если превалирует  
проб\-лем\-но-ори\-ен\-ти\-ро\-ван\-ный конфликт между агентами 
($\mathrm{cp}_{ij}\hm\geq \mathrm{cpr}_{ij}$), то ребро между двумя вершинами (агентами) 
прорисовывается сплошной линией, а~если  
про\-цес\-сно-ори\-ен\-ти\-ро\-ван\-ный ($\mathrm{cp}_{ij}\hm< \mathrm{cpr}_{ij}$)~--- штриховой. 
Толщина и~цвет (от светло-серого до черного на рисунке) линии указывают на величину 
среднего квадратического напряженностей конфликтов между агентами;
\end{itemize}

{ \begin{center}  %fig1
 \vspace*{3pt}
    \mbox{%
    \epsfxsize=78.651mm 
\epsfbox{rum-1.eps}
 }

\end{center}

\noindent
{\small
Визуализация конфликта (на примере коллектива агентов, решающего задачу диагностики 
рака поджелудочной железы): Х~--- хирург; 
ОНЛ~--- онколог по нехирургическому лечению; 
ЛПР-Т~--- лицо, принимающее решение (терапевт); 
сУЗИ~--- специалист по ультразвуковому исследованию; 
вЛаД~--- врач лабораторной диагностики; 
сЛД~--- специалист по лучевой диагностике
}}

%\vspace*{6pt}

\begin{itemize}
\item к вершинам добавляем подписи~--- идентификаторы агентов;\\[-14pt]
\item если $\eta>0$, то для агентов с~напряженностью конфликта 
ниже~$\eta$ ребра на результирующем графе не будут отображены.\\[-14pt]
\end{itemize}

  При наведении указателя мыши на ребро отоб\-ра\-жа\-ют\-ся напряженности 
конфликтов между соответствующей парой агентов. При каждой фиксации 
изменений напряженности конфликтов в~мат\-ри\-це $\mathbf{CNF}$ будет 
запускаться МВК. Для каждой сессии работы системы весь визуальный ряд 
конфликта между агентами сохраняется, обеспечивая возможность более 
детального изучения пользователем. Таким образом, пользователь может 
отслеживать развитие конфликта с~самого начала работы системы и~до 
момента получения решения. Визуализация дает быстро воспринимаемое 
знание о том, между какими агентами возникают конфликты при решении 
проблемы, какого они типа, как меняется их напряженность в~процессе 
работы системы.  
С~по\-мощью данных знаний можно предотвратить и/или быстрее разрешить 
развитие конфликта в~естественных малых коллективах экспертов, решающих 
подобную проблему.

\vspace*{-11pt}

\section{Заключение}

\vspace*{-4pt}

  В работе представлены результаты обзора методов укладки графов, на 
основе анализа которого для разработки метода визуализации конфликтов 
в~коллективе агентов выбран метод рисования графа, основанный на 
физических аналогиях, а~именно: эффективная пружинная модель укладки 
графов Т.~Ка\-ма\-да\,--\,С.~Ка\-ваи, зарекомендовавшая себя для 
неориентированных взвешенных графов малой и~средней размерности. 
Результирующая визуализация конфликта агентов предоставляет пользо\-вателю 
быстрое понимание того, между какими\linebreak членами коллектива и~на каких этапах 
возникает конфликт, какого он типа и~напряженности, а~также на каком этапе 
он минимизируется/нивелируется. Разработанный подход к визуализации 
повышает прозрачность работы \mbox{ГиИМАС} для пользователя, не 
зависит от численности коллектива агентов и~легко реализуем.
  
\vspace*{-11pt}

{\small\frenchspacing
    {\baselineskip=10.4pt
    %\addcontentsline{toc}{section}{References}
    \begin{thebibliography}{99}

\vspace*{-2pt}

\bibitem{1-rum}
\Au{Колесников А.\,В.} Гетерогенные естественные и~искусственные системы~// 
Интегрированные модели и~мягкие вычисления в~искусственном интеллекте.~--- 
М.: Физматлит, 2013. Т.~1. С.~86--103.
\bibitem{2-rum}
   \Au{Колесников А.\,В., Кириков~И.\,А., Листопад~С.\,В.} Гиб\-рид\-ные интеллектуальные 
системы с~самоорганиза- цией: координация, согласованность, спор.~--- М.: ИПИ РАН, 2014. 
189~с.
\bibitem{3-rum}
\Au{Румовская С.\,Б., Кириков~И.\,А.} Методы моделирования и~визуального 
представления конфликта в~малом коллективе экспертов, решающих проблемы (обзор)~// 
Информатика и~её применения, 2019. Т.~13. Вып.~3. С.~122--130. doi: 
10.14357/19922264190317.
\bibitem{4-rum}
\Au{Листопад С.\,В., Кириков~И.\,А.} Моделирование конфликтов агентов в~гибридных 
интеллектуальных многоагентных системах~// Системы и~средства информатики, 2019. 
Т.~29. №\,3. С.~139--148. doi: 10.14357/08696527190312.
\bibitem{5-rum}
\Au{Листопад С.\,В., Кириков~И.\,А.} Метод идентификации конфликтов агентов 
в~гибридных интеллектуальных многоагентных системах~// Сис\-те\-мы и~средства 
информатики, 2020. Т.~30. №\,1. С.~56--65. doi: 10.14357/08696527200105.

\bibitem{7-rum} %6
\Au{Готин С.\,В., Калоша~Л.\,П.} Ло\-ги\-ко-струк\-тур\-ный подход и~его применение для 
анализа и~планирования деятельности.~--- М.: Вариант, 2007. 118~с.

\bibitem{6-rum} %7
\Au{Новиков Д.\,А.} Иерархические модели военных действий~// Управление большими 
системами, 2012. №\,37. С.~25--62.
\bibitem{8-rum}
\Au{Eades P.} A heuristic for graph drawing~// Congressus Numerantium, 1984. Vol.~42. P.~149--160.
\bibitem{9-rum}
\Au{Kamada Т., Kawai~S.} An algorithm for drawing general undirected graphs~// Inform. 
Process. Lett., 1989. Vol.~31. Iss.~1. 
 P.~7--15.  doi: 10.1016/0020-0190(89)90102-6.
\bibitem{10-rum}
\Au{Fruchterman T., Reingold~E.} Graph drawing by force-directed placement~// Software 
Pract. Exper., 1991. Vol.~21. Iss.~11. P.~1129--1164. doi: 10.1002/spe.\linebreak 4380211102.
\bibitem{11-rum}
\Au{Sugiyama К., Tagawa~S., Toda~M.} Methods for visual understanding of hierarchical system 
structures~// IEEE~T. Syst. Man Cyb., 1981. Vol.~11. Iss.~2. P.~109--125. doi: 
10.1109/TSMC.1981.4308636.
\bibitem{12-rum}
\Au{Tamassia R., Battista G.\,D., Ioannis~G., Eades~P.} Graph drawing: Algorithms for the 
visualization of graphs.~--- Englewood Cliffs, NJ, USA: Prentice Hall, 1999. 397~p. 
\bibitem{13-rum}
\Au{Reingold E., Tilford~J.} Tidier drawing of trees~// IEEE~T. Software Eng., 1981. 
Vol.~SE-7. Iss.~2. P.~223--228. doi: 10.1109/TSE.1981.234519.
\bibitem{14-rum}
\Au{Tutte W.\,T.} How to draw a graph~// P.~Lond. Math. Soc., 1963. Vol.~S3-13. Iss.~1. 
P.~743--767. doi: 10.1112/plms/s3-13.1.743.
\bibitem{15-rum}
\Au{Battista G.\,D., Liotta G., Vargiu~F.} Spirality of orthogonal representations and optimal 
drawings of series-parallel graphs and 3-planar graphs~// Algorithms and data 
structures~/ Eds. F.\,K.\,H.\,A.~Dehne, J.-R.~Sack, N.~Santoro, S.~Whikesides.~---  
Lecture notes in computer science ser. ~--- Springer, 1993. Vol.~709.  
P.~151--162. doi: 10.1007/3-540-57155-8\_244.
\bibitem{16-rum}
\Au{Kruskal J.\,В., Seery~J.\,B.} Designing network diagrams~// 1st General Conference 
(International) on Social Graphics Proceedings, 1980. P.~22--50. 
\bibitem{17-rum}
\Au{Пупырев С.\,Н.} Модели, алгоритмы и~программный комплекс визуализации сложных 
сетей:  
Дис.\ \ldots\ канд. физ.-мат. наук.~--- Екатеринбург, 2010. 136~с.
   \end{thebibliography}
   
    }
    }
   
   \end{multicols}
   
\vspace*{-6pt}
%\vspace*{-12pt}
   
\hfill{\small\textit{Поступила в~редакцию 30.09.20}}
   
%   \vspace*{8pt}
   
%\pagebreak
   
\newpage
   
\vspace*{-28pt}
   
%   \hrule
   
%   \vspace*{2pt}
   
%   \hrule
   
   %\vspace*{-2pt}
   
   \def\tit{CONFLICT VISUAL REPRESENTATION METHOD\\ 
    IN~HYBRID INTELLIGENT MULTIAGENT SYSTEMS}
   
   
   \def\titkol{Conflict visual representation method in hybrid intelligent 
multiagent systems}
  
   
   \def\aut{S.\,B.~Rumovskaya and I.\,A.~Kirikov}
   
   \def\autkol{S.\,B.~Rumovskaya and I.\,A.~Kirikov}
   
   \titel{\tit}{\aut}{\autkol}{\titkol}
   
   \vspace*{-11pt}
   
   
   \noindent
   Kaliningrad Branch of the Federal Research Center ``Computer Science and Control'' of the Russian 
Academy of Sciences, 5~Gostinaya Str., Kaliningrad 236000, Russian Federation
   
   
   \def\leftfootline{\small{\textbf{\thepage}
   \hfill INFORMATIKA I EE PRIMENENIYA~--- INFORMATICS AND
   APPLICATIONS\ \ \ 2020\ \ \ volume~14\ \ \ issue\ 4}
   }%
    \def\rightfootline{\small{INFORMATIKA I EE PRIMENENIYA~---
   INFORMATICS AND APPLICATIONS\ \ \ 2020\ \ \ volume~14\ \ \ issue\ 4
   \hfill \textbf{\thepage}}}
   
   \vspace*{9pt} 
   
   \Abste{Small collectives of experts, including specialists from different fields, effectively solve complex 
problems by analyzing them from different points of view and obtaining a better-integrated solution. 
A~conflict in small collectives of experts can both lead to a deadlock in the decision-making process and 
generate positive changes: development of the group, diagnostics of relations, and
consolidation of the group. 
A~conflict breeds debate, the depth of which allows for more thoughtful and coordinated solutions. Such 
collectives of experts solve problems effectively. Thus, modeling of their work and possible conflict situation 
with managing it allows developing a~decision-support method that is relevant to solving a complex problem. 
Visualization of a~conflict situation makes appeared contradictions contrast and observable. In the research, 
the authors represent a~collective of agents-experts in the form of an undirected weighted graph. 
The methods of 
graph visualization are considered. To visualize problem- and process-oriented conflicts within 
hybrid intelligent multiagent systems, the authors propose a~method based on the spring model of graph 
drawing.}
   
   
   \KWE{collective of experts; conflict; visualization of conflict}
   
  
   \DOI{10.14357/19922264200411} 
   
   %\vspace*{-20pt}
   
   %\Ack
   %\noindent
   
   
   \vspace*{6pt}
   
     \begin{multicols}{2}
   
   \renewcommand{\bibname}{\protect\rmfamily References}
   %\renewcommand{\bibname}{\large\protect\rm References}
   
   {\small\frenchspacing
    {%\baselineskip=10.8pt
    \addcontentsline{toc}{section}{References}
    \begin{thebibliography}{99}
   
   \bibitem{1-rum-1}
   \Aue{Kolesnikov, A.\,V.} 2013. Geterogennye estestvennye i~iskusstvennye sistemy [Natural and 
artificial heterogeneous systems]. \textit{Integrirovannye modeli i~myagkie vychisleniya 
v~iskusstvennom intellekte} [Integrated models and soft computing in artificial intelligence]. Moscow: 
Fizmatlit. 1:86--103.
   \bibitem{2-rum-1}
   \Aue{Kolesnikov, A.\,V., I.\,A.~Kirikov, and S.\,V.~Listopad.} 2014. \textit{Gibridnye intellektual'nye 
sistemy s~samoorganizatsiey: koordinatsiya, soglasovannost', spor} [Hybrid artificial systems with 
self-organization: Coordination, conformance, row]. Мoscow: IPI RAN. 189~p.
   \bibitem{3-rum-1}
   \Aue{Rumovskaya, S.\,B., and I.\,A.~Kirikov.} 2019. Metody mo\-de\-li\-ro\-va\-niya i~vizual'nogo 
predstavleniya konflikta v~ma\-lom kollektive ekspertov, reshayushchikh problemy (obzor) [Methods of 
modeling and visual representation of a~conflict in small collective of experts solving problems (review)]. 
\textit{Informatika i~ee Primeneniya~--- Inform. Appl.} 13(3):122--130. doi: 10.14357/19922264190317.
   \bibitem{4-rum-1}
   \Aue{Listopad, S.\,V., and I.\,A.~Kirikov.} 2019. Modelirovanie konfliktov agentov v~gibridnykh 
intellektual'nykh mnogoagentnykh sistemakh [Modeling of agent conflicts in hybrid intelligent multiagent 
systems]. \textit{Sistemy i~Sredstva Informatiki~--- Systems and Means of Informatics} 29(3):139--148. 
doi: 10.14357/08696527190312.
   \bibitem{5-rum-1}
   \Aue{Listopad, S.\,V., and I.\,A.~Kirikov.} 2020. Metod identifikatsii konfliktov agentov v~gibridnykh 
intellektual'nykh mnogoagentnykh sistemakh [Agent conflict identification method in hybrid intelligent 
multiagent systems]. \textit{Sistemy}

\columnbreak

\textit{i~Sredstva Informatiki~--- Systems and Means of Informatics} 30(1):56--
65. doi: 10.14357/08696527200105.
   
   \bibitem{7-rum-1} %6
   \Aue{Gotin, S.\,V., and L.\,P.~Kalosha.} 2007. \textit{Logiko-strukturnyy podkhod i~ego primenenie 
dlya analiza i~planirovaniya de\-ya\-tel'\-nosti} [Logical-structural approach and its application for the analysis 
and planning of activities]. Moscow: Variant. 118~p.

\bibitem{6-rum-1} %7
   \Aue{Novikov, D.\,A.} 2012. Ierarkhicheskie modeli voennykh deystviy [Hierarchical models of combat]. 
\textit{Upravlenie bol'shimi sistemami} [Control of Large Systems] 37:25--62.
   \bibitem{8-rum-1}
   \Aue{Eades, P.} 1984. A~heuristic for graph drawing. \textit{Congressus Numerantium} 42:149--160.
   \bibitem{9-rum-1}
   \Aue{Kamada, Т., and S.~Kawai.} 1989. An algorithm for drawing general undirected graphs. 
\textit{Inform. Process. Lett.} 31(1):7--15. doi: 10.1016/0020-0190(89)90102-6.
   \bibitem{10-rum-1}
   \Aue{Fruchterman, T., and E.~Reingold.} 1991. Graph drawing by force-directed placement. 
\textit{Software Pract. Exper.} 
 21(11):1129--1164. doi: 10.1002/spe.4380211102.
   \bibitem{11-rum-1}
   \Aue{Sugiyama, К., S.~Tagawa, and M.~Toda.} 1981. Methods for visual understanding of hierarchical 
system structures. \textit{IEEE~T. Syst. Man Cyb.} 11(2):109--125. doi: 
10.1109/TSMC.1981.4308636.
   \bibitem{12-rum-1}
   \Aue{Tamassia, R., G.\,D.~Battista, G.~Ioannis, and P.~Eades.} 1999. \textit{Graph drawing: 
Algorithms for the visualization of graphs}. Englewood Cliffs, NJ: Prentice Hall. 397~p.
   \bibitem{13-rum-1}
   \Aue{Reingold, E., and J.~Tilford.} 1981. Tidier drawing of trees. \textit{IEEE~T. Software 
Eng.} SE-7(2):223--228. doi: 10.1109/TSE.1981.234519.
   \bibitem{14-rum-1}
   \Aue{Tutte, W.\,T.} 1963. How to draw a graph. \textit{P.~Lond. Math. Soc.} S3-13(1):743--767. doi: 10.1112/plms/s3-13.1.743.

\pagebreak

   \bibitem{15-rum-1}
   \Aue{Di Battista, G., G.~Liotta, and F.~Vargiu.} 1993. Spirality of orthogonal representations and optimal 
drawings of series-parallel graphs 
 and 3-planar graphs. \textit{Algorithms and data structures}. 
 Eds. F.\,K.\,H.\,A.~Dehne, J.-R.~Sack, N.~Santoro, and S.~Whikesides.
 Lecture notes in 
computer science ser. Springer. 709:151--162.
   doi: 10.1007/3-540-57155-8\_244.
 {\looseness=1
 
 }  
   \bibitem{16-rum-1}
   \Aue{Kruskal, J.\,В., and J.\,B.~Seery.} 1980. Designing network diagrams. \textit{1st General 
Conference (International) on Social Graphics Proceedings}. 22--50. 
\vspace*{-3pt}

   \bibitem{17-rum-1}
   \Aue{Pupyrev, S.\,N.} 2010. Modeli, algoritmy i~programmnyy kompleks vizualizatsii slozhnykh setey 
[Models, algorithms and software for visualization of complex networks].  Ekaterinburg. PhD Diss.
136~p.
   \end{thebibliography}
   
    }
    }
   
   \end{multicols}
   
   \vspace*{-3pt}
   
   \hfill{\small\textit{Received September 30, 2020}}
   
   %\pagebreak
   
   %\vspace*{-24pt}
   
   
   \Contr
   
   \noindent
   \textbf{Rumovskaya Sophiya B.} (b.\ 1985)~--- Candidate of Science (PhD) in technology, scientist, 
Kaliningrad Branch of the Federal Research Center ``Computer Science and Control'' of the Russian 
Academy of Sciences, 5~Gostinaya Str., Kaliningrad 236000, Russian Federation; 
\mbox{sophiyabr@gmail.com}
   
   \vspace*{3pt}
   
   \noindent
   \textbf{Kirikov Igor A.} (b.\ 1955)~--- Candidate of Science (PhD) in technology, director, Kaliningrad 
Branch of the Federal Research Center ``Computer Science and Control'' of the Russian Academy of 
Sciences, 5~Gostinaya Str., Kaliningrad 236000, Russian Federation; \mbox{baltbipiran@mail.ru}
   \label{end\stat}
   
   \renewcommand{\bibname}{\protect\rm Литература} 
                    %11
\def\stat{danilishin}

\def\tit{ОЦЕНКА СТОИМОСТИ ОПЦИОНОВ НА~ОСНОВЕ 
МОДЕЛЕЙ ARIMA--GARCH С~ОШИБКАМИ, РАСПРЕДЕЛЕННЫМИ 
ПО~ЗАКОНУ~$S_U$ ДЖОНСОНА}

\def\titkol{Оценка стоимости опционов на основе моделей  
ARIMA--GARCH с~ошибками, распределенными по закону~JSU} %$S_U$  Джонсона}

\def\aut{А.\,Р.~Данилишин$^1$, Д.\,Ю.~Голембиовский$^2$}

\def\autkol{А.\,Р.~Данилишин, Д.\,Ю.~Голембиовский}

\titel{\tit}{\aut}{\autkol}{\titkol}

\index{Данилишин А.\,Р.}
\index{Голембиовский Д.\,Ю.}
\index{Danilishin A.\,R.}
\index{Golembiovsky D.\,Yu.}


%{\renewcommand{\thefootnote}{\fnsymbol{footnote}} \footnotetext[1]
%{Работа выполнена при частичной поддержке РФФИ (проект 19-07-00187-A).}}


\renewcommand{\thefootnote}{\arabic{footnote}}
\footnotetext[1]{Московский государственный университет имени М.\,В.~Ломоносова, факультет 
вычислительной математики и~кибернетики, \mbox{danilishin-artem@mail.ru}}
\footnotetext[2]{Московский государственный университет имени М.\,В.~Ломоносова, факультет 
вычислительной математики и~кибернетики; Московский фи\-нан\-со\-во-про\-мыш\-лен\-ный университет 
<<Синергия>>, \mbox{golemb@cs.msu.su}}

%\vspace*{-12pt}


\Abst{В продолжение статьи <<Риск-нейтральная динамика для модели ARIMA--GARCH 
с~ошибками, распределенными по закону $S_U$ Джонсона>> в~данной работе приводятся 
результаты экспериментов для моделей ARIMA--GARCH 
(autoregressive integrated moving average\,--\,generalized autoregressive conditional heteroskedasticity)
с~нормальными (N), 
экспоненциальными бета второго типа (EGB2) и~$S_U$ Джонсона (JSU) распределениями 
ошибок. Стоимость европейских опционов оценивается методом Мон\-те-Кар\-ло на основе 
результатов, полученных в~указанной статье при помощи расширенного принципа 
Гирсанова. Параметры моделей ARIMA--GARCH-N, ARIMA--GARCH-EGB2 и~ARIMA--GARCH-JSU 
были найдены методом квазимаксимального правдоподобия. Эффективность 
полученных риск-нейтральных моделей исследовалась на примере биржевых европейских 
опционов PUT и~CALL на базовые активы DAX  (Deutscher Aktienindex)
и~Light Sweet Crude Oil. }

\KW{ARIMA; GARCH; риск-нейтральная мера; расширенный принцип Гирсанова; 
распределение $S_U$ Джонсона; ценообразование опционов}

\DOI{10.14357/19922264200412} 
  
%\vspace*{9pt}


\vskip 10pt plus 9pt minus 6pt

\thispagestyle{headings}

\begin{multicols}{2}

\label{st\stat}


\section{Введение}

Данная работа является продолжением \mbox{статьи}~[1]. В~указанной статье была 
введена модель  
ARIMA$\,(p,d,q)$--GARCH$\,(P,Q)$-JSU$\,(\xi,\lambda, \gamma,\delta)$ для 
доходности базового актива в~следующем виде (для распределения JSU 
$\tilde{Y}_t\hm= S_t/S_{t-1}\hm -1$, для N- и~EGB2-рас\-пре\-де\-ле\-ний 
$Y_t\hm= \ln(S_t/S_{t-1})$)~[2--5]:
\begin{equation}
\left.
\begin{array}{l}
\!\!\!\Delta^d Y_t=m_t +\sqrt{h_t}\,\varepsilon_t\,,\ \varepsilon_t\vert \sim 
\mathrm{JSU}(\xi,\lambda,\gamma,\delta)\,;\\[12pt]
\!\!\!m_t=\mathbb{E}\left[ \Delta^dY_t\vert \mathcal{F}_{t-1}\right]=
\phi_0+\cdots+\phi_p\Delta^d Y_{t-p} +{}\\[6pt]
\!\!\!\hspace*{10mm}{}+\theta_1\sqrt{h_{t-1}}\,\varepsilon_{t-1}+
\cdots+\theta_q \sqrt{h_{t-q}}\,\varepsilon_{t-q}\,;\\[12pt]
\!\!\!h_t=\mathrm{Var}\left[ \Delta^d Y_t\vert \mathcal{F}_{t-1}\right]=
\alpha_0+\cdots+\alpha_P h_{t-P} +{}\\[6pt]
\!\!\!\hspace*{15mm}{}+\beta_1h_{t-1}\varepsilon^2_{t-1}+\cdots+\beta_Q h_{t-Q} 
\varepsilon^2_{t-Q}
\end{array} \!\!
\right\} \!\!
\label{e1-dan}
\end{equation}
с ограничениями на математическое ожидание и~дисперсию ошибки
\begin{align*}
\mathbb{E}[\varepsilon_t] &=\xi-\lambda e^{1/(2\delta^2)}\mathrm{sinh}\left(
\fr{\gamma}{\delta}\right)=0\,;\\
\mathrm{Var}\left[\varepsilon_t\right]&=\fr{\lambda^2}{2}\left( e^{1/\delta^2}-1\right)\times{}\\
&\hspace*{10mm}{}\times \left( 
e^{1/\delta^2} \mathrm{cosh}\left( \fr{2\gamma}{\delta}\right) +1\right)=1\,.
\end{align*}

%\label{e2-dan}

  Далее к~разностному (стационарному) ряду~(\ref{e1-dan}) применялся 
расширенный принцип Гирсанова~[6, 7], а~в~случае распределения~JSU~---
 его найденная модификация. Для распределения JSU %$S_U$ Джонсона 
полученные коэффициенты модели, обес\-пе\-чи\-ва\-ющие риск-ней\-т\-раль\-ную 
динамику процесса даются сле\-ду\-ющи\-ми соотношениями:
  \begin{equation}
  \left.
  \begin{array}{rl}
  Y_t&=r+\delta_t\fr{1+r}{1+m_t}\,\varepsilon_t\,;\\[9pt]
   \varepsilon_{t\vert
  \mathcal{F}_{t-1}} &\sim \mathrm{JSU}\left( 
\tilde{\xi},\tilde{\lambda},\gamma,\delta\right);\\[9pt]
  \tilde{\xi}&=\tilde{\lambda}e^{1/(2\delta^2)}\mathrm{sinh}\left( \fr{\gamma} 
{\delta} \right);\\[9pt] 
\tilde{\lambda}&=\sqrt{2}\left( \left( e^{1/\delta^2}-1\right)\times{}\right.\\[9pt]
&\hspace*{3mm}\left.{}\times\left(
  e^{1/\delta^2}\mathrm{cosh}\left( \fr{2\gamma}{\delta}\right)+1\right)\right)^{-1/2}.
  \end{array}
  \right\}
  \label{e3-dan}
  \end{equation}
  
  Риск-нейтральные коэффициенты для моделей ARIMA--GARCH-EGB2  
и~ARIMA--GARCH-N представлены соотношениями~\cite{7-dan}:
  \begin{multline}
  Y_t=r-\ln\fr{B(\alpha+\delta_t\overline{\delta},\beta-
\delta_t\overline{\delta})}{B(\alpha,\beta)}+{}\\
{}+\delta_t\overline{\delta}\overline{\omega}(\alpha,\beta)+
  \delta_t\varepsilon_t\,;
  \label{e4-dan}
  \end{multline}
  
  \noindent
  \begin{equation}
  \left.
  \begin{array}{c}
  \varepsilon_{t\vert\mathcal{F}_{t-1}}\sim 
\mathrm{EGB2}\left(\alpha,\beta,\overline{\delta},\overline{\mu}\right);\\[6pt]
  \overline{\delta}=\fr{1}{\sqrt{l(\alpha,\beta)}};\quad
   \overline{\mu}=-
\fr{\overline{\omega}(\alpha,\beta)}{\sqrt{l(\alpha,\beta)}};\\[12pt]
  Y_t=r-\fr{1}{2}\,\delta^2_t+\delta_t \varepsilon_t\,; \quad 
\varepsilon_{t\vert\mathcal{F}_{t-1}} \sim N(0,1)\,.
  \end{array}
  \right\}
  \label{e5-dan}
  \end{equation}
  
  В данной статье приводятся результаты чис\-лен\-ных экспериментов, 
подтверждающие корректность теоретических результатов первой работы 
и~эффективность полученных риск-ней\-траль\-ных моделей  
ARIMA--GARCH-N, ARIMA--GARCH-EGB2 и~ARIMA--GARCH-JSU. 
  
  Работа построена следующим образом. В~разд.~2 приводятся формулы для 
оценки справедливой стоимости опционов CALL и~PUT методом Мон\-те-Кар\-ло. 
В~разд.~3 описываются два набора данных для проведения 
численных экспериментов: цены закрытия торгов по опционам на индекс 
немецких акций DAX и~на нефть Light Sweet Crude Oil, а~также 
соответствующие ряды цен базовых активов. В~разд.~4 даются формулы 
оценки параметров моделей ARIMA--GARCH методом квазимаксимального 
правдоподобия~\cite{8-dan}. В~разд.~5 приводятся тесты, определяющие 
спецификации моделей и~результаты оценок параметров моделей. Раздел~6 
содержит результаты оценки справедливой стоимости опционов, полученные 
при использовании различных моделей динамики базовых активов. 
В~заключении формулируются выводы исследования.

\vspace*{-6pt}

\section{Оценка справедливой стоимости опциона методом  
Монте-Карло}

\vspace*{-3pt}

  Оценка справедливой стоимости опционов проводится по методу Мон\-те-Кар\-ло~\cite{9-dan}. 
  Европейские опционы CALL и~PUT с~ценой 
исполнения~$X$ и~стоимостью базового актива~$S_T$ в~день 
экспирации~$T$ характеризуются функциями выплат $b_c (S_T,X)\hm= \max 
(S_T\hm - X,0)$ и~$b_p(S_T, X)\hm= \max (X\hm- S_T,0)$. Стоимость опционов 
определяется как среднее значение соответствующей функции выплаты 
относительно риск-ней\-траль\-ной меры~$\mathbb{Q}$, приведенное 
к~текущему моменту времени~\cite{10-dan}:
  \begin{equation}
\! \fr{p_{\mathrm{call}}}{p_{\mathrm{put}}} \!=\!\fr{\mathbb{E}^{\mathbb{Q}} [b_{c/p}(S_T,X)]} {(1+r)^T} 
\!=\! \fr{\mathbb{E}^{\mathbb{P}} [b_{c/p}(S_T,X)d\mathbb{Q}/d\mathbb{P}]} 
{(1+r)^T},\!\!
  \label{e6-dan}
  \end{equation}
где $d\mathbb{Q}/d\mathbb{P}$~--- производная Ра\-до\-на--Ни\-ко\-ди\-ма  
риск-ней\-т\-раль\-ной меры (в~рамках данной работы это мера, полученная на 
основе расширенного принципа Гирсанова либо его модификации) 
относительно физической меры~$\mathbb{P}$~\cite{11-dan, 12-dan}. Метод 
Мон\-те-Кар\-ло позволяет по реализациям построенного процесса ARIMA--GARCH 
оценить среднее значение относительно  
риск-ней\-т\-раль\-ной меры~$\mathbb{Q}$:
\vspace*{-6pt}

\noindent
\begin{multline}
\fr{1}{M}\sum\limits_{m=1}^M b_{c/p}\left( S_T, X\right) 
\fr{d\mathbb{Q}}{d\mathbb{P}}\,(m) 
\xrightarrow[M\to\infty]{P}{} \\
{}\xrightarrow[M\to\infty]{P}
\mathbb{E}^{\mathbb{P}} \left[ b_{c/p}\left( 
S_T,X\right)\fr{d\mathbb{Q}}{d\mathbb{P}}\right]\,.
\label{e7-dan}
\end{multline}

\vspace*{-9pt}

\section{Описание данных}

\vspace*{-3pt}

  Данные для проведения численных экспериментов состоят из двух 
однотипных наборов цен закрытия торгов по опционам на 3~июня 2019~г. По 
дате экспирации опционы были поделены на самые ближние и~дальние 
с~учетом ликвидности.
  
  Первый набор данных представлен европейскими опционами на фондовый 
индекс DAX. Индекс отражает суммарный доход по 
капиталу, поэтому при его расчете учитываются полученные дивидендные 
доходы по акциям, которые, как предполагается, реинвестируются в~акцию, 
по которой был получен дивиденд. Рас\-смат\-ри\-ва\-ют\-ся 19~опционов CALL 
и~19~опционов PUT, величина страйка варьируется от~9\,400 до~13\,000 
с~шагом~200. Базовым активом выступает фьючерс с~датой экспирации, 
соответствующей дате экспирации самого опциона. Дата экспирации ближних 
опционов~--- 22~июня 2019~г., а~дальних~--- 22~декабря 2023~г. Валюта 
опционов~--- евро. Данные взяты с~сайта {\sf www.eurexchange.com}.
  
  Во второй набор данных вошли 10 европейских опционов CALL и~PUT на 
фьючерс, базовым активом которого выступает нефть (Light Sweet Crude Oil). 
Величина страйка варьируется от~51,0 до~55,5 с~шагом~0,5. Дата экспирации 
ближних опционов~--- 20~июня 2019~г., дальних~--- 22~июня 2020~г. 
Валюта~--- доллар США. Источник данных: {\sf www.cmegroup.com}.
  
  В представленных данных фигурируют два вида валют; соответственно, при 
расчете справедливой стоимости опционов использовались две\linebreak безрисковые 
процентные ставки. В~качестве таковых были взяты соответствующие ставки 
LIBOR. На дату расчета (03~июня 2019~г.)\ ставка USD \mbox{LIBOR} равнялась 
2,36075\%, а~ставка EUR \mbox{LIBOR} со\-став\-ля\-ла~0,46614\%. Источник данных: 
{\sf www.global-rates.com}.

\vspace*{-12pt}

\section{Калибровка моделей ARIMA--GARCH}

\vspace*{-3pt}

\begin{table*}[b] %\small %tabl1
\vspace*{-3pt}
\begin{minipage}[t]{80mm}
\begin{center}
{\small 
\Caption{Результаты оценивания моделей ARIMA(0,0,1)--GARCH(1,1) для 
$Y_t^{\mathrm{N/EGB2}}\hm=\ln (S_t/S_{t-2})$  
и~$\tilde{Y}_t^{\mathrm{JSU}} \hm= S_t/S_{t-2}\hm-1$ фондового индекса DAX}
\vspace*{2ex}

\tabcolsep=8.15pt
\begin{tabular}{|c|c|c|c|}
\hline
\tabcolsep=0pt\begin{tabular}{c}Распре-\\ деление\\ ошибок\end{tabular}&N&EGB2&JSU \\
\hline
$L_n(\hat{v})$&1652,609&1658,26&1658,657\\
\hline
\tabcolsep=0pt \begin{tabular}{c} $\phi_0$\\ Std.\ error\\ $t$-value\end{tabular}&
\tabcolsep=0pt \begin{tabular}{c} $-$0,000100\\ (0,000798)\\ $-$0,12480 \end{tabular}&
\tabcolsep=0pt \begin{tabular}{c} $-$0,00006\\ (0,000786)\\ $-$0,076381 \end{tabular}&
\tabcolsep=0pt \begin{tabular}{c}  $-$0,000074\\ (0,000792)\\ $-$0,093104 \end{tabular}\\
\hline
\tabcolsep=0pt \begin{tabular}{c} $\theta_1$\\ Std.\ error\\ $t$-value\end{tabular}&
\tabcolsep=0pt \begin{tabular}{c} 0,950806\\ (0,013106)\\ 72,54849 \end{tabular}&
\tabcolsep=0pt \begin{tabular}{c} 0,948445\\ (0,015226)\\ 62,292279 \end{tabular}&
\tabcolsep=0pt \begin{tabular}{c} 0,949900\\ (0,013229)\\ 71,806222 \end{tabular}\\
\hline
\tabcolsep=0pt \begin{tabular}{c} $\alpha_0$\\ Std.\ error\\ $t$-value\end{tabular}&
\tabcolsep=0pt \begin{tabular}{c} 0,000003\\ (0,000005)\\ 0,55453 \end{tabular}&
\tabcolsep=0pt \begin{tabular}{c} 0,000003\\ (0,000004)\\ 0,687438 \end{tabular}&
\tabcolsep=0pt \begin{tabular}{c} 0,000003\\ (0,000004)\\ 0,718274 \end{tabular}\\
\hline
\tabcolsep=0pt \begin{tabular}{c} $\alpha_1$\\ Std.\ error\\ $t$-value \end{tabular}&
\tabcolsep=0pt \begin{tabular}{c} 0,071761\\ (0,031491)\\ 2,27881 \end{tabular}&
\tabcolsep=0pt \begin{tabular}{c} 0,067895\\ (0,027377)\\ 2,480017 \end{tabular}&
\tabcolsep=0pt \begin{tabular}{c} 0,067271\\ (0,030275)\\ 2,222013 \end{tabular}\\
\hline
\tabcolsep=0pt \begin{tabular}{c} $\beta_1$\\ Std.\ error\\ $t$-value \end{tabular}&
\tabcolsep=0pt \begin{tabular}{c} 0,894232\\ (0,050380)\\ 17,74964 \end{tabular}&
\tabcolsep=0pt \begin{tabular}{c} 0,902903\\ (0,037822)\\ 23,872549 \end{tabular}&
\tabcolsep=0pt \begin{tabular}{c} 0,900226\\ (0,041630)\\ 21,624622 \end{tabular}\\
\hline
AIC&$-$6,5773&$-$6,5919&$-$6,5934\\
\hline
BIC&$-$6,5352&$-$6,5369&$-$6,5385\\
\hline
$\xi$&---&$-$0,224916&$-$0,543876\\
\hline
$\kappa$&---&2,897165&2,298773\\
\hline
\end{tabular}
}
\end{center}
\end{minipage}
%\end{table*}
\hfill
%\begin{table*}%{\small %tabl2
\begin{minipage}[t]{80mm}
\begin{center}
{\small 
\Caption{Результаты оценивания моделей ARIMA(2,0,0)--GARCH(1,1) 
для $Y_t^{\mathrm{N/EGB2}}\hm= \ln (S_t/S_{t-2})$ 
и $\tilde{Y}_t^{\mathrm{JSU}}\hm= S_t/S_{t-2}\hm-1$ (Light Sweet Crude Oil)}
\vspace*{2ex}

\tabcolsep=8.14pt
\begin{tabular}{|c|c|c|c|}
\hline
\tabcolsep=0pt\begin{tabular}{c}Распре-\\ деление\\ ошибок\end{tabular}&N&EGB2&JSU\\
\hline
$L_n(\hat{v})$&1261,086&1267,017&1268,022\\
\hline
\tabcolsep=0pt \begin{tabular}{c} $\phi_1$\\ Std.\ error\\ $t$-value \end{tabular}&
\tabcolsep=0pt \begin{tabular}{c} 0,667116\\ (0,043985)\\ 15,1668 \end{tabular}&
\tabcolsep=0pt \begin{tabular}{c} 0,657088\\ (0,043122)\\ 15,2379 \end{tabular}&
\tabcolsep=0pt \begin{tabular}{c} 0,659745\\ (0,043152)\\ 15,2888\end{tabular}\\
\hline
\tabcolsep=0pt \begin{tabular}{c} $\phi_2$\\ Std.\ error\\ $t$-value \end{tabular}&
\tabcolsep=0pt \begin{tabular}{c} $-$0,313186\\ (0,043636)\\ $-$7,1773 \end{tabular}&
\tabcolsep=0pt \begin{tabular}{c} $-$0,323465\\ (0,042236)\\ $-$7,6585 \end{tabular}&
\tabcolsep=0pt \begin{tabular}{c} $-$0,322464\\ (0,042439)\\ $-$7,5983 \end{tabular}\\
\hline
\tabcolsep=0pt \begin{tabular}{c} $\alpha_0$\\ Std.\ error\\ $t$-value \end{tabular}&
\tabcolsep=0pt \begin{tabular}{c} 0,000015\\ (0,000002)\\ 8,341500 \end{tabular}&
\tabcolsep=0pt \begin{tabular}{c} 0,000013\\ (0,000001)\\ 16,4056 \end{tabular}&
\tabcolsep=0pt \begin{tabular}{c} 0,000013\\ (0,000001)\\ 16,1163\end{tabular}\\
\hline
\tabcolsep=0pt \begin{tabular}{c} $\alpha_1$\\ Std.\ error\\ $t$-value \end{tabular}&
\tabcolsep=0pt \begin{tabular}{c} 0,049707\\ (0,007002)\\ 7,0988\end{tabular}&
\tabcolsep=0pt \begin{tabular}{c} 0,042324\\ (0,005701)\\ 7,424 \end{tabular}&
\tabcolsep=0pt \begin{tabular}{c} 0,041834\\ (0,005506)\\ 7,5984\end{tabular}\\
\hline
\tabcolsep=0pt \begin{tabular}{c} $\beta_1$\\ Std.\ error\\ $t$-value \end{tabular}&
\tabcolsep=0pt \begin{tabular}{c} 0,913128\\ (0,013747)\\ 66,424300 \end{tabular}&
\tabcolsep=0pt \begin{tabular}{c} 0,92728\\ (0,011196)\\ 82,8231 \end{tabular}&
\tabcolsep=0pt \begin{tabular}{c} 0,928020\\ (0,011169)\\ 83,0925\end{tabular}\\
\hline
AIC&$-$4,9944&1267,752&$-$5,0140\\ 
\hline
BIC&$-$4,9524&$-$4,9542&$-$4,9553\\
\hline
$\xi$&&$-$0,70877&$-$0,27232\\
\hline
$\kappa$&&4,405384&2,663401\\
\hline
\end{tabular}
}
\end{center}
\end{minipage}
\vspace*{-3pt}
\end{table*}

  Пусть $\ln(L_n(v))$~--- логарифм функции правдоподобия модели  
ARIMA--GARCH для доходности базового актива с~вектором параметров 
$v\hm\in \Theta$. В~случае модели  
ARIMA$\,(p,d,q)$--GARCH$\,(P,Q)$-JSU$\,(\xi,\lambda,\gamma,\delta)$~(1) 
имеется вектор параметров 
\begin{multline*}
v= \left(\gamma, \delta, \phi_0, \phi_1, \ldots , \phi_p, 
\theta_1, \ldots , \theta_q, \alpha_0, \alpha_1, \ldots , \alpha_P,\right.\\
\left. \beta_1, \ldots , 
\beta_Q\right).
\end{multline*}
 Число параметров равно $n\hm= p\hm+ 1\hm+q\hm+ P\hm+Q\hm+2$. 
Оптимальные параметры определяются исходя из максимума функции  
правдоподобия~\cite{8-dan}:
  \begin{multline}
  \hat{v}_n=\argmax\limits_{v\in\Theta} \ln (L_n(v))={}\\[-3pt]
  {}=\argmax\limits_{v\in \Theta} \sum\limits^T_{t=0} \left( \ln(\delta)-
\ln \left(\sqrt{\fr{2h_t}{A}}\right)-{}\right.\\[-1pt]
\left.{}- \fr{1}{2}\ln\left(1+\left( 
\fr{\varepsilon_t}{\sqrt{2h_t/A}}-B\right)^{\!2}\right) \right.-{}\\[-3pt]
  \left.{}-\fr{1}{2}\left(\gamma+\delta\mathrm{sinh}^{-1}\left( 
\fr{\varepsilon_t}{\sqrt{2h_t/A}}-B\right)\right)^{\!2}\right)\,,\\[-1pt]
\delta, \alpha_0>0,\alpha_1,\ldots , \alpha_P, \beta_1,\ldots , \beta_Q\geq 0\,,
\label{e8-dan}
\end{multline}
где 
%\vspace*{-6pt}


\noindent
$$
A=\left(e^{1/\delta^2}-1\right)\left(e^{1/\delta^2}\mathrm{cosh}\left(\fr{2\gamma}{\delta}\right)+1\right);
$$ 
$$
B=e^{1/(2\delta^2)}\mathrm{sinh}\left(\fr{\gamma}{\delta}\right); \enskip
\sum\limits_{i=1}^P 
\alpha_i+ \sum\limits^Q_{j=1} \beta_j <1\,.
$$ 
  
  Для случая EGB2-распределения число па\-ра\-мет\-ров $v\hm= (\alpha, \beta, 
\phi_0, \ldots , \phi_p, \theta_1, \ldots , \theta_q, \alpha_0, \alpha_1, \ldots$\linebreak $\ldots , \alpha_P, 
\beta_1, \ldots , \beta_Q)$ такое же, как в~предыдущем случае. 
Оптимизационная задача имеет следующий вид:

\vspace*{-6pt}

\noindent
  \begin{multline}
  \hat{v}_n=\argmax\limits_{v\in \Theta} \ln (L_n(v))={}\\
  {}=\argmax\limits_{v\in \Theta} \sum\limits^T_{t=0} \left( 
  \vphantom{\fr{\varepsilon_t\sqrt{l(\alpha,\beta)}}{\sqrt{h_t}}}
  \ln \left( \sqrt{l(\alpha,\beta)}\right) -
\ln(B(\alpha,\beta)) +{}\right.\\
{}+ \alpha\overline{\omega}(\alpha,\beta)+ 
\fr{\alpha\varepsilon_t\sqrt{l(\alpha,\beta)}}{\sqrt{h_t}}
- \ln\left(\sqrt{h_t}\right)-{}\\
{} -(\alpha+\beta) \ln \left( 1+\exp \left( 
\fr{\varepsilon_t\sqrt{l(\alpha,\beta)}}{\sqrt{h_t}}
%+{}\right.\right.\\\left.\left.
\left.{}+\overline{\omega}(\alpha,\beta)
 \vphantom{\fr{\varepsilon_t\sqrt{l(\alpha,\beta)}}{\sqrt{h_t}}}
 \right)\right)\right)\!,\\
  \alpha,\beta,\alpha_0>0\,, \enskip \alpha_1,\ldots, \alpha_P,\beta_1,\ldots , \beta_Q\geq 0\,,
  \label{e9-dan}
  \end{multline}
\vspace*{-6pt}
  
\noindent
где $\Gamma(c)$~--- гамма-функция;
\begin{align*}
\overline{\omega}(\alpha,\beta)&=\left.\fr{d\ln\Gamma(c)}{dc}\right\vert_{c=\alpha}-
\left.\fr{d\ln\Gamma(c)}{dc}\right\vert_{c=\beta}\,;\\
l(\alpha,\beta)&=\left.\fr{d^2\ln\Gamma(c)}{dc^2}\right\vert_{c=\alpha} +
\left.\fr{d^2\ln\Gamma(c)}{dc^2}\right\vert_{c=\beta}\,;
\end{align*}
$$
\sum\limits^P_{i=1}\alpha_i+\sum\limits^Q_{j=1} \beta_j<1\,.
$$
  
\begin{figure*}[b] %fig1
\vspace*{1pt}
    \begin{center}  
  \mbox{%
 \epsfxsize=162.997mm 
 \epsfbox{dan-1.eps}
 }
\end{center}
\vspace*{-12pt}
\Caption{Графики ACF~(\textit{а}) и~PACF~(\textit{б}) для $Y_t\hm= \ln(S_t/S_{t-2})$ (DAX)}
%\end{figure*}
%\begin{figure*} %fig2
\vspace*{5pt}
    \begin{center}  
  \mbox{%
 \epsfxsize=162.997mm 
 \epsfbox{dan-2.eps}
 }
\end{center}
   \vspace*{-12pt}
  \Caption{Графики ACF~(\textit{а}) и~PACF~(\textit{б}) для $Y_t\hm= \ln (S_t/S_{t-2})$ (Light Sweet Crude Oil)}
  \end{figure*}

  Для случая нормального распределения использовались функции 
библиотеки rugarch среды~R~\cite{8-dan}.

\vspace*{-9pt}
   
\section{Калибровка параметров моделей ARIMA--GARCH}

\vspace*{-3pt}

  В табл.~1 и~2 представлены результаты оценки параметров моделей  
ARIMA--GARCH. Для обоих временн$\acute{\mbox{ы}}$х рядов был проведен Q-тест  
Льюн\-га--Бок\-са для разного числа лагов (нулевая гипотеза заключается 
в~отсутствии автокорреляций для первых~$k$~лагов), который показал, что 
автокорреляционные связи для рядов $\ln (S_t/S_{t-1})$ и~$S_t/S_{t-1}\hm-1$ 
отсутствуют с~вероятностью 99\% (значение статистики составило~30,123 для 
индекса DAX и~24,576 для Light Sweet Crude Oil, в~то время как критическое 
значение для $k\hm=30$~лагов и~уровня значимости~99\% равно~50,892). 
В~результате были рассмотрены ряды $\ln(S_t/S_{t-2})$  
и~$S_t/S_{t-2}\hm- 1$ и~на основе их коррелограмм ACF и~PACF (рис.~1 и~2) 
был сделан вывод о спецификации ARIMA части моделей ARIMA--GARCH.
{\looseness=-1

}

  Для коррелограмм индекса DAX характерна модель ARIMA$(\,(0,0,1)$ 
(ненулевое значение автокорреляции первого лага и~затухающая динамика 
значений частных автокорреляций). Коррелограммы цен на нефть определяют 
модель ARIMA$(\,2,0,0)$  (ненулевые значения частных автокорреляций двух 
первых лагов и~затухающее поведение автокорреляций). 
{\looseness=-1

}
  
  Коэффициенты всех трех моделей имеют одинаковый знак и~порядок. Все 
полученные коэффициенты моделей статистически значимы на уровне 
значимости~99\% (следует из $t$-кри\-те\-рия). Можно также заметить, что 
коэффициенты, соответствующие моделям с~распределениями ошибок EGB2 
и~JSU, почти идентичны.

  
\begin{figure*}[b] %fig3
\vspace*{1pt}
 \begin{center}  
 \mbox{%
 \epsfxsize=161.099mm 
 \epsfbox{dan-3.eps}
 }
\end{center}
\vspace*{-11pt}
\Caption{Абсолютные ошибки цен опционов CALL~(\textit{а}) и~PUT~(\textit{б})
 на индекс DAX со сроком 22~июня 2019~г.: \textit{1}~---  
ARIMA--GARCH-N; \textit{2}~--- 
ARIMA--GARCH-EGB2; \textit{3}--- ARIMA--GARCH-JSU }
%\end{figure*}
%\begin{figure*} %fig4
\vspace*{5pt}
    \begin{center}  
  \mbox{%
 \epsfxsize=162.999mm 
 \epsfbox{dan-5.eps}
 }
\end{center}
\vspace*{-11pt}
\Caption{Абсолютные ошибки цен опционов CALL~(\textit{а}) и~PUT~(\textit{б})
 на индекс DAX со сроком 22~декабря 2023~г.: \textit{1}~--- 
ARIMA--GARCH-N; \textit{2}~--- 
ARIMA--GARCH-EGB2; \textit{3}--- ARIMA--GARCH-JSU}
\end{figure*}

  Предсказательная сила моделей оценивалась с~помощью информационных 
критериев Акаике (AIC) и~Байеса (BIC), статистики которых рассчитываются 
по следующим формулам:

\noindent
  \begin{align*}
    \mathrm{AIC} &= 2k-2\ln\left( L_n\!\left( \hat{v}\right)\right)\,;\\[6pt]
  \mathrm{BIC} &= k\ln(N) -2\ln\left( L_n\!\left( \hat{v}\right)\right)\,,
  %  \label{e10-dan}
  \end{align*}
где $N$~--- объем выборки; $k$~--- число параметров; $L_n(\hat{v})$~--- 
значение функции правдоподобия для найден\-ных оптимальных 
параметров~$\hat{v}$. Таблицы~1 и~2 показывают, что асимметричные 
распределения EGB2 и~JSU лучше моделируют ряд, чем нормальное 
распределение.


\section{Ценообразование опционов}

  В рамках данной работы справедливая стоимость опционов оценивалась 
  с~помощью метода Мон\-те-Кар\-ло в~соответствии с~формулами~(\ref{e6-dan}) 
и~(\ref{e7-dan}) по риск-ней\-т\-раль\-ным траекториям моделей  
ARIMA--GARCH~(\ref{e3-dan})--(\ref{e5-dan}), где число 
реализаций доходности базового актива $M\hm= 10\,000$. Эффективность 
каждой модели ARIMA--GARCH оценивалась по абсолютной ошибке (AE):
  \begin{multline*}
  \mathrm{АО}(\mathrm{Moneyness})={}\\
\!=\!\,\left\vert p^m_{\mathrm{call}}/p^m_{\mathrm{put}}
  \left( \mathrm{Moneyness}\right) -
p_{\mathrm{call}}/p_{\mathrm{put}}\left( \mathrm{Moneyness}\right)
  \!\,\right\vert\!,\hspace*{-5.9pt}
  %\label{e11-dan}
  \end{multline*}
где $p^m_{\mathrm{call}}/p^m_{\mathrm{put}}$~---  рыночные котировки опционов; 
$\mathrm{Moneyness}\hm=X/s_0$. 



  Абсолютные ошибки оценивания опционов иллюстрируются на рис.~3--6. 
Для опционов на индекс DAX наиболее близкие значения к~рыночным 
котировкам дает модель с~распределением ошибок JSU, модель 
с~нормальным распределением хуже всего находит стоимость опционов DAX. 
При этом стоит также отметить, что для опционов с~датой экспирации 
22~декабря 2023~г.\ расхождения между моделями становятся существенней, 
показывая неэффективность использования моделей с~нормальным 
распределением ошибок на длинных временн$\acute{\mbox{ы}}$х
горизонтах. 
\pagebreak

\end{multicols}


\begin{figure*} %fig5
\vspace*{.1pt}
    \begin{center}  
  \mbox{%
 \epsfxsize=162.238mm 
 \epsfbox{dan-7.eps}
 }
\end{center}
\vspace*{-12.5pt}
\Caption{Абсолютные ошибки цен опционов CALL~(\textit{а}) и~PUT~(\textit{б})
 Light Sweet Crude Oil (20~июня 2019~г.): \textit{1}~---  
ARIMA--GARCH-N; \textit{2}~--- 
ARIMA--GARCH-EGB2; \textit{3}--- ARIMA--GARCH-JSU}
%\end{figure*}
%\begin{figure*} %fig6
\vspace*{5pt}
    \begin{center}  
  \mbox{%
 \epsfxsize=161.542mm 
 \epsfbox{dan-9.eps}
 }
\end{center}
\vspace*{-12.5pt}
\Caption{Абсолютные ошибки цен опционов CALL~(\textit{а}) и~PUT~(\textit{б})
 Light Sweet Crude Oil (22~июня 2020~г.): \textit{1}~---  
ARIMA--GARCH-N; \textit{2}~--- 
ARIMA--GARCH-EGB2; \textit{3}--- ARIMA--GARCH-JSU}
\vspace*{-1pt}
\end{figure*}


\begin{multicols}{2}

Результаты 
оценки спра\-вед\-ли\-вой стои\-мости опционов на индекс DAX аналогичны 
результатам работы~\cite{7-dan}, в~которой среди рас\-смат\-ри\-ва\-емых опционов 
присутствовали опционы на индекс S\&P~500 и~делался вывод о~том, что 
модели ARIMA--GARCH с~ошибками, распределенными по закону EGB2, дают 
лучшие оценки стоимости опционов в~деньгах (для опционов CALL 
$\mathrm{Moneyness}\hm< 1$, для PUT $\mathrm{Moneyness}\hm>1$) по сравнению с~моделями, 
где ошибки распределены нормально. 
  
  Опционы на базовый актив Light Sweet Crude Oil также характеризуются 
существенным преимуществом модели JSU. Однако теперь уже модель 
с~распределением EGB2 показывает результаты хуже, чем модель с~нормальным 
распределением.
  
\vspace*{-7pt}

\section{Заключение}
\vspace*{-2pt}

  В данной статье были рассмотрены три альтернативные модели временн$\acute{\mbox{ы}}$х 
рядов с~условным средним ARIMA и~условной дисперсией GARCH 
структуры. Оценка справедливой стоимости опционов проводилась по методу  
Мон\-те-Кар\-ло~(\ref{e6-dan}), (\ref{e7-dan}) на основе риск-ней\-т\-раль\-ной 
динамики моделей в~соответствии с~формулами~(\ref{e3-dan})--(\ref{e5-dan}). 
Калибровка моделей осуществлялась методом 
квазимаксимального правдоподобия в~соответствии 
с~соотношениями~(\ref{e8-dan}) и~(\ref{e9-dan}). Определение порядка моделей 
было выполнено на основе Q-тес\-та  
Льюн\-га--Бок\-са и~графиков коррелограмм ACF и~PACF. 
  
  Результаты эмпирических исследований показывают, что на небольших 
временн$\acute{\mbox{ы}}$х промежутках модели обеспечивают близкие значения 
справедливой стоимости опционов. Для опционов с~дальней датой 
экспирации (больше года) модели показывают существенно разные 
результаты. Модель, построенная для распределения JSU, дает 
оценки стоимости, максимально близкие к~ценам закрытия биржевых торгов 
для всех рас\-смат\-ри\-ва\-емых опционных контрактов. 
  
{\small\frenchspacing
 {%\baselineskip=10.8pt
 %\addcontentsline{toc}{section}{References}
 \begin{thebibliography}{99}
  
\bibitem{1-dan}
\Au{Данилишин А.\,Р., Голембиовский~Д.\,Ю.} Риск-нейт\-раль\-ная динамика для модели 
ARIMA--GARCH с~ошиб\-ка\-ми, распределенными по закону $S_U$ Джонсона~// 
Информатика и~её применения, 2020. Т.~14. Вып.~1. С.~48--55.

\bibitem{5-dan} %2
\Au{Bollerslev T.} A~conditionally heteroskedastic time series model for speculative prices and 
rates of return~//  Rev. Econ. Stat., 1987. Vol.~69. Iss.~3. P.~542--547. doi: 10.2307/1925546.
\bibitem{3-dan} %3
\Au{Akgiray V.} Conditional heteroscedasticity in time series of stock returns: Evidence and 
forecasts~// J.~Bus., 1989. Vol.~62. Iss.1. P.~55--80. doi: 10.1086/296451.
\bibitem{4-dan} %4
\Au{Follmer H., Schied A.} Stochastic finance: An introduction in discrete time.~--- Berlin: 
Walter de Gruyter, 2002.\linebreak 422~p.
\bibitem{2-dan} %5
\Au{Terasvirta T.} An introduction to univariate GARCH models~//  Handbook of 
financial time series~/
Eds. T.\,G.~Andersen, R.\,A.~Davis, J.-P.~Kreiss, Th.\,V.~Mikosch.~--- Berlin--Heidelberg: 
Springer, 2009. Vol.~10.  
P.~17--42. doi: 10.1007/978-3-540-71297-8\_1.
\bibitem{6-dan} %6
\Au{Elliott R.\,J., Madan D.\,B.} A~discrete time equivalent martingale measure~// Math. 
Financ., 1998. Vol.~8. Iss.~2. P.~127--152. doi: 10.1111/1467-9965.00048.
\bibitem{7-dan}
\Au{Yi Xi.} Comparison of option pricing between ARMA-GARCH and GARCH-M models.~--- 
London, Ontario, Canada: University of Western Ontario, 2013. MoS Thesis. 73~p.
\bibitem{8-dan}
\Au{Christian F., Francq M.} GARCH models: Structure, statistical inference and financial 
applications.~--- New York, NY, USA: Wiley, 2019. 504~p.
\bibitem{9-dan}
\Au{Boyle T.} A~Monte Carlo approach~// J.~Financ. Econ., 2012. Vol.~4. P.~323--338. 
doi: 10.1016/0304-405x(77)90005-8.
\bibitem{10-dan}
\Au{Hull J.} Options, futures, and other derivatives.~--- 10th ed.~--- Pearson, 2018. 896~p.
\bibitem{11-dan}
\Au{Williams D.} Probability with martingales.~--- Cambridge: Cambridge University Press, 
1991. 251~p.
\bibitem{12-dan}
\Au{Bell D.} Transformations of measure on an infinite dimensional vector space~// Seminar on 
stochastic processes, 1990~/ Eds. \mbox{E.~{\ptb{\c{C}}}inlar}, P.\,J.~Fitzsimmons, 
R.\,J.~Williams.~--- Progress in probability book ser.~--- Boston, MA, USA:
Birkh$\ddot{\mbox{a}}$user, 1991. 
Vol.~24. P.~15--25. doi: 10.1007/978-1-4684-0562-0\_3.
\end{thebibliography}

 }
 }

\end{multicols}

\vspace*{-12pt}

\hfill{\small\textit{Поступила в~редакцию 01.10.19}}

\vspace*{6pt}

%\pagebreak

%\newpage

%\vspace*{-28pt}

\hrule

\vspace*{2pt}

\hrule

\vspace*{-2pt}

\def\tit{ESTIMATING THE FAIR VALUE OF~OPTIONS\\
 BASED~ON~ARIMA--GARCH MODELS WITH~ERRORS\\ DISTRIBUTED ACCORDING 
TO~THE~JOHNSON'S $S_U$ LAW}


\def\titkol{Estimating the fair value of options based on~ARIMA--GARCH models 
with~errors distributed according to~the~Johnson's $S_U$ law}


\def\aut{A.\,R.~Danilishin$^1$ and D.\,Yu.~Golembiovsky$^{1,2}$}

\def\autkol{A.\,R.~Danilishin and D.\,Yu.~Golembiovsky}

\titel{\tit}{\aut}{\autkol}{\titkol}

\vspace*{-11pt}


\noindent
$^1$Department of Operations Research, Faculty of Computational Mathematics and Cybernetics, 
M.\,V.~Lomonosov\linebreak
$\hphantom{^1}$Moscow State University,  
1-52~Leninskie Gory, Moscow 119991, GSP-1, Russian Federation

\noindent
$^2$Department of Banking, Sinergy University, 80-G~Leningradskiy Prosp., Moscow 125190, Russian 
Federation


\def\leftfootline{\small{\textbf{\thepage}
\hfill INFORMATIKA I EE PRIMENENIYA~--- INFORMATICS AND
APPLICATIONS\ \ \ 2020\ \ \ volume~14\ \ \ issue\ 4}
}%
 \def\rightfootline{\small{INFORMATIKA I EE PRIMENENIYA~---
INFORMATICS AND APPLICATIONS\ \ \ 2020\ \ \ volume~14\ \ \ issue\ 4
\hfill \textbf{\thepage}}}

\vspace*{6pt} 


\Abste{In continuation of the article ``Risk-neutral dynamics for the ARIMA--GARCH 
(autoregressive integrated moving average\,--\,generalized autoregressive conditional heteroskedasticity)
random process with errors distributed according to the Johnson's $S_U$ law,'' this 
paper presents the experimental results for the ARIMA--GARCH 
(autoregressive integrated moving average\,--\,generalized autoregressive conditional heteroskedasticity)
models with normal (N), exponential beta of the second type 
(EGB2), and $S_U$ Johnson (JSU) error distributions. The fair value of European 
options is estimated by the Monte-Carlo method based on the results obtained in the 
specified article by using the extended Girsanov principle. The parameters of the 
ARIMA--GARCH-N,  
ARIMA--GARCH-EGB2, and ARIMA--GARCH-JSU models were found by the 
quasi-maximum likelihood method. The efficiency of the resulting risk-neutral models 
was studied using the example of European exchange-traded options PUT and CALL 
on basic assets DAX and Light Sweet Crude Oil.}

\KWE{ARIMA; GARCH; risk-neutral measure; Girsanov extended principle; 
Johnson's $S_U$ distribution; option pricing}

\DOI{10.14357/19922264200412} 

%\vspace*{-20pt}

%\Ack
%\noindent


%\vspace*{-6pt}

  \begin{multicols}{2}

\renewcommand{\bibname}{\protect\rmfamily References}
%\renewcommand{\bibname}{\large\protect\rm References}

{\small\frenchspacing
 {%\baselineskip=10.8pt
 \addcontentsline{toc}{section}{References}
 \begin{thebibliography}{99}
\vspace*{-4pt}

\bibitem{1-dan-1}
\Aue{Danilishin, A.\,R., and D.\,Y. Golembiovsky.} 2020. Risk-neytral'naya dinamika 
dlya ARIMA--GARCH modeli
%\linebreak
%\vspace*{-12pt}
%\columnbreak
%\noindent
s~oshib\-ka\-mi, raspredelennymi po zakonu $S_U$ 
Dzhonsona. [Risk-neutral dynamics for ARIMA--GARCH random process with errors 
distributed according to the Johnson's $S_U$ law]. \textit{Informatika i~ee 
Primeneniya~--- Inform. Appl.} 14(1):56--62.

\bibitem{5-dan-1} %2
\Aue{Bollerslev, T.} 1987. A~conditionally heteroskedastic time series model for 
speculative prices and rates of return. \textit{Rev. Econ. Stat.} 69(3):542--547. doi: 
10.2307/ 1925546.
{\looseness=1

}
\bibitem{3-dan-1} %3
\Aue{Akgiray, V.} 1989. Conditional heteroscedasticity in time series of stock returns: 
Evidence and forecasts. \textit{J.~Bus.} 62(1):55--80. doi: 10.1086/296451.
\bibitem{4-dan-1} %4
\Aue{Follmer, H., and A. Schied.} 2002. \textit{Stochastic finance: An introduction in 
discrete time}. Berlin: Walter de Gruyter. 422~p.

\bibitem{2-dan-1} %5
\Aue{Terasvirta, T.} 2009. An introduction to univariate GARCH models. 
\textit{Handbook of financial time series}. Eds. T.\,G.~Andersen, R.\,A.~Davis, 
J.-P.~Kreiss, and Th.\,V.~Mikosch. Berlin--Heidelberg: Springer. 10:17--42. doi:  
10.1007/978-3-540-71297-8\_1.

\bibitem{6-dan-1}
\Aue{Elliott, R.\,J., and D.\,B. Madan.} 1998. A~discrete time equivalent martingale 
measure. \textit{Math. Financ.} 8(2):127--152. doi: 10.1111/1467-9965.00048.
\bibitem{7-dan-1}
\Aue{Yi, Xi.} 2013. Comparison of option pricing between ARMA--GARCH and 
GARCH-M models. London, Ontario, Canada: University of Western Ontario. MoS 
Thesis. 73~p.
\bibitem{8-dan-1}
\Aue{Christian, F., and M. Francq.} 2019. \textit{GARCH models: Structure, 
statistical inference and financial applications.} New York, NY: Wiley. 504~p.
\bibitem{9-dan-1}
\Aue{Boyle, P.} 2012. Options: A~Monte Carlo approach. \textit{J.~Financ. 
Econ.} 4(3):323--338.  doi: 10.1016/0304-405x(77)90005-8.
\bibitem{10-dan-1}
\Aue{Hull, J.} 2018. \textit{Options, futures, and other derivatives}. 10th ed. Pearson. 
896~p.
\bibitem{11-dan-1}
\Aue{Williams, D.} 1991. \textit{Probability with martingales}. Cambridge: 
Cambridge University Press. 251~p.
\bibitem{12-dan-1}
\Aue{Bell, D.} 1991. Transformations of measure on an infinite dimensional vector 
space. \textit{Seminar on stochastic processes, 1990}. Eds. \mbox{E.~{\ptb{\c{C}}}inlar}, 
P.\,J.~Fitzsimmons, and R.\,J.~Williams. Progress in probability book ser. Boston, MA: 
Birkh$\ddot{\mbox{a}}$user. 
24:15--25. doi: 10.1007/978-1-4684-0562-0\_3.
\end{thebibliography}

 }
 }

\end{multicols}

\vspace*{-3pt}

\hfill{\small\textit{Received October 1, 2019}}

%\pagebreak

%\vspace*{-24pt}


\Contr

\noindent
\textbf{Danilishin Artem R.} (b.\ 1992)~--- PhD student, Department of Operations Research, Faculty of 
Computational Mathematics and Cybernetics, M.\,V.~Lomonosov Moscow State University, 1-52~Leninskie 
Gory, GSP-1, Moscow 119991, Russian Federation; \mbox{danilishin-artem@mail.ru}

\vspace*{3pt}

\noindent
\textbf{Golembiovsky Dmitry Y.} (b.\ 1960)~--- Doctor of Science in technology, professor, Department of 
Operation Research, Faculty of Computational Mathematics and Cybernetics, M.\,V.~Lomonosov Moscow 
State University, 1-52~Leninskie Gory, GSP-1, Moscow 119991, Russian Federation; professor, Department 
of Banking, Sinergy University, 80-G~Leningradskiy Prosp., Moscow 125190, Russian Federation; 
\mbox{golemb@cs.msu.su}
\label{end\stat}

\renewcommand{\bibname}{\protect\rm Литература} 
      
         %12
\def\stat{abgaryan}

\def\tit{ПРОГРАММНЫЙ КОМПЛЕКС ДЛЯ~МНОГОМАСШТАБНОГО МОДЕЛИРОВАНИЯ 
СТРУКТУРНЫХ СВОЙСТВ КОМПОЗИЦИОННЫХ МАТЕРИАЛОВ$^*$}

\def\titkol{Программный комплекс для многомасштабного моделирования 
структурных свойств композиционных материалов}

\def\aut{К.\,К.~Абгарян~$^1$, Е.\,С.~Гаврилов$^2$}

\def\autkol{К.\,К.~Абгарян, Е.\,С.~Гаврилов}

\titel{\tit}{\aut}{\autkol}{\titkol}

\index{Абгарян К.\,К.}
\index{Гаврилов Е.\,С.}
\index{Abgaryan K.\,K.}
\index{Gavrilov E.\,S.}


{\renewcommand{\thefootnote}{\fnsymbol{footnote}} \footnotetext[1]
{Работа выполнена при поддержке Министерства науки и~высшего образования Российской Федерации (проект 
075-15-2020-799).}}


\renewcommand{\thefootnote}{\arabic{footnote}}
\footnotetext[1]{Федеральный исследовательский центр <<Информатика и~управление>> Российской академии наук, 
\mbox{kristal83@mail.ru}}
\footnotetext[2]{Федеральный исследовательский центр <<Информатика и~управление>> Российской академии наук; 
Московский авиационный институт (национальный исследовательский университет), \mbox{eugavrilov@gmail.com}}

%\vspace*{-6pt}
    
      
         
      
      \Abst{Создание новых композиционных материалов (КМ) с~прогнозируемыми свойствами 
      и~разработка способов их конструирования на сегодня стали одними из актуальных и~важнейших 
задач, связанных с~модернизацией промышленного производства в~нашей стране. Для их 
решения активно развиваются технологии многомасштабного компьютерного 
моделирования. Они стали связующим звеном между фундаментальной физикой (химией) 
и~инженерным материаловедением. В~работе представлен программный комплекс по 
моделированию структурных свойств КМ, поз\-во\-ля\-ющий решать ряд 
задач данного класса. Он ориентирован на высокопроизводительные вы\-чис\-ле\-ния. В~основе 
комплекса лежит оригинальная многомасштабная технология, которая позволяет оперативно 
проводить многовариантный анализ различных классов КМ 
и~проводить исследования по проектированию новых с~прогнозируемыми свойствами. 
Разработанные подходы в~сочетании с~экспериментальными данными могут быть использованы 
для лучшего понимания физических основ изменения свойств в~за\-ви\-си\-мости от структуры и,~как 
следствие, для удешевления и~ускорения поиска новых КМ
с~заданными свойствами.}
      
      \KW{многомасштабное моделирование; композиционные материалы; интеграционная 
платформа; программный комплекс; распределенная сис\-тема}

\DOI{10.14357/19922264220113}
  
%\vspace*{-3pt}


\vskip 10pt plus 9pt minus 6pt

\thispagestyle{headings}

\begin{multicols}{2}

\label{st\stat}

\section{Введение}

\vspace*{-3pt}

     Создание новых КМ с~прогнозируемыми 
свойствами и~разработка способов их конструирования на сегодня стали одними 
из актуальных и~важнейших задач по модернизации промышленного 
производства в~нашей стране. Особенно важны такие материалы в~областях, где 
соотношение между проч\-ностью и~массой конструкции определяет ее 
эф\-фек\-тив\-ность. На сегодня процессы создания КМ
непосредственно связаны с~этапом моделирования, включая применение наиболее 
эффективных методов многомасштабного компьютерного моделирования и~анализа данных. 
     
     Для решения данного класса задач разработан\linebreak программный комплекс по 
моделированию структурных свойств КМ. Он 
ориентирован на высокопроизводительные вы\-чис\-ле\-ния. В~осно\-ве комплекса 
лежит оригинальная многомасштабная \mbox{технология}, пред\-став\-лен\-ная в~[1, 2], 
которая позволяет оперативно проводить многовариантный анализ различных 
классов КМ. На базе разработанной технологии была 
создана распределенная информационная сис\-те\-ма для проведения 
многоуровневых исследований в~об\-ласти моделирования~КМ. 

Согласно разработанным подходам в~за\-ви\-си\-мости от типа 
мо\-де\-ли\-ру\-емо\-го КМ строится многомасштабная 
композиция и~ее схематическое представление. На ее основе в~программной среде 
формируется сценарий расчета структурных характеристик и~отдельных свойств 
рас\-смат\-ри\-ва\-емо\-го материала. Созданный программный комплекс позволяет 
автоматизировать уни\-фи\-ци\-ру\-емые этапы моделирования и~помогает 
сформировать на основе анализа полученных результатов более глубокое 
понимание физических процессов. Комплекс построен с~применением 
современных программных средств и~решений и~не уступает международному 
уровню на\-уч\-но-тех\-ни\-че\-ских разработок в~об\-ласти информационной 
поддержки для многомасштабного моделирования новых материалов. 
     
     Разработка такого средства информационной поддержки поз\-во\-ля\-ет 
обеспечить формирование информации для многопараметрического анализа 
структуры и~физических свойств различных классов су\-ще\-ст\-ву\-ющих 
КМ, рассмотреть большое чис\-ло вариантов 
в~на\-прав\-ле\-нии поиска новых материалов и,~таким образом, ускорить и~удешевить 
процесс подбора па\-ра\-мет\-ров получения материалов.  Ис-\linebreak\vspace*{-12pt}

\pagebreak

\noindent
пользование данного 
комплекса позволяет за ограниченное время строить гиб\-рид\-ные модели для 
обоснованного выбора КМ с~заданными свойствами для  
авиа\-ци\-он\-но-кос\-ми\-че\-ской и~других областей промышленности. 
     
     В связи с~тем что традиционные материалы (преимущественно металлы)
      не в~полной мере отвечают высоким фи\-зи\-ко-ме\-ха\-ни\-че\-ским, 
технологическим и~эксплуатационными свойствам, развитие производства 
современных надежных и~экономичных конструкций в~машиностроении 
основано на применении новых КМ. Под 
композиционными понимаются материалы, со\-сто\-ящие из двух или более 
физически различных компонент (фаз), возможные комбинации которых 
приводят к~появлению уникальных свойств, отличных от тех, которыми обладала 
каж\-дая из них отдельно. На сегодня для развития авиа\-ци\-он\-но-кос\-ми\-че\-ской 
отрасли, включая самолетостроение, вертолетостроение, ракетостроение, 
требуется постоянное увеличение доли полимерных КМ
с~набором заданных свойств. Современные летательные аппараты обладают 
слож\-ной конструкцией, со\-сто\-ящей из металлов и~неметаллических материалов. 
Применяются детали из алю\-ми\-ни\-евых и~сталь\-ных сплавов, коррозионностойких 
сталей, титановых сплавов и~полимерных КМ (стек\-ло-, 
угле-, органопластики и~др.). Для снижения веса и~продления срока службы 
летательных аппаратов при производстве деталей все шире применяют 
полимерные~КМ.
     
     Сегодня наиболее востребованные САЕ- (Computer-Aided Engineering) 
сис\-те\-мы, такие как ABAQUS ({\sf https://simulia.com}), \mbox{ANSYS} ({\sf 
https://\linebreak Ansys.com}), LMS Engineering innovation ({\sf https://\linebreak trademarks.justia.com}), 
Femap ({\sf https://www.cad-is.ru/femap}), MSC Software ({\sf 
http://www.mscsoftware.\linebreak ru}) включают в~себя базы данных со свойствами 
материалов. Для КМ мож\-но выбрать тип композита со 
стандартными свойствами (угле-, стекло-, органопластики на основе 
эпоксифенолформальдегидных, кремнийорганических смол, эпоксидные 
боропластики и~т.\,д.). Имеется возможность коррекции данных свойств 
и~внесения материала с~новыми свойствами в~базу данных. Следует также отметить 
российские разработки в~об\-ласти моделирования КМ, 
такие как пакет CAE-Fidesys ({\sf https://cae-fidesys.com}), программный пакет для 
моделирования полимерных материалов Multicomp ({\sf 
https://www.kintechlab.com/products}), Российский исследовательский 
и~ин\-же\-нер\-но-тех\-но\-ло\-ги\-че\-ский проект N1 Composites ({\sf 
http://n1composites.com}) и~др.
{\looseness=-1

}
     
     Программные комплексы позволяют задать\linebreak свойства материалов, из 
которых состоит КМ, такие как изотропность, 
ортотропность, анизотропность. Важная часть проектирования композиционных 
конструкций~--- преобразование модели,\linebreak созданной с~применением CAD 
(Computer-aided design, сис\-те\-мы автоматизированного проектирования) 
в~модель, пригодную для CAE-ана\-ли\-за (нетривиальная задача, тре\-бу\-ющая 
за\-час\-тую создания экспертной сис\-те\-мы). Следует отметить, что функционал всех 
мировых лидеров в~CAE-сег\-мен\-те схож. 
     %
     Так, функционал MSC позволяет встраивать разработанные пользователем 
модули в~программный комплекс (например, можно включить метод имитации 
процесса производства КМ).
     
     Помимо используемых ведущими CAE-сис\-те\-ма\-ми модулями существуют 
коммерческие сис\-те\-мы, позволяющие генерировать КМ на микроуровне, а~затем 
проводить чис\-лен\-ные эксперименты на макроуровне. К~таким сис\-те\-мам 
относятся модуль генерации и~моделирования механических характеристик 
КМ GeoDict ({\sf www.math2market.com}) с~различными типами КМ, 
ге\-не\-ри\-ру\-емы\-ми модулем GeoDict, и~программный комплекс COMSOL ({\sf 
www.comsol.ru}).
     
     В современных ведущих CAE-сис\-те\-мах учет мик\-ро\-струк\-ту\-ры 
КМ проводится после гомогенизации свойств материала 
или определения мак\-ро\-мас\-штаб\-ных свойств КМ. При этом, однако, теряются 
индивидуальные детали микроструктуры КМ~\cite{3-ab}. При определении макромасштабных свойств КМ обычно 
исходят из идеальных условий: оптимального формирования граничной 
поверхности, идеального распределения(отсутствия взаимодействия час\-тиц 
между собой) и~отсутствия влияния компонента на мат\-рицу.
     
     Однако результаты, которые на сегодня могут быть получены 
     с~использованием САЕ-систем для\linebreak воспроизведения характеристик известных 
структур, зачастую могут расходиться с~данными экспериментов~--- например, 
когда речь идет о~полимерных КМ с~на\-но\-вклю\-че\-ни\-ями 
(\mbox{нанотрубками}). \mbox{Известно} влияние до\-бав\-ле\-ния на\-но\-раз\-мер\-ных\linebreak час\-тиц 
наполнителя на изменение механических свойств КМ. 
В~литературе широко описано изменение коэффициента теп\-ло\-про\-вод\-ности 
полимерных\linebreak мат\-риц в~несколько раз при их наполнении 
нанотрубками, пред\-став\-ле\-ны тео\-ре\-ти\-че\-ские исследования с~аналогичными 
результатами~\cite{1-ab}. Использование CAE-сис\-те\-м не позволяет в~полной 
мере \mbox{оценить} фактор влияния на\-но\-час\-тиц на данные свойства. Кроме 
того, применение CAE-сис\-тем в~контексте многомасштабного моделирования 
затруднено жесткими ограничениями пакетных решений. В~настоящее время 
развиваются \mbox{системы} c~программным обеспечением для многомасштабного 
моделирования, такие как Computational Soft Materials (Comsoft) Workbench, 
поз\-во\-ля\-ющий моделировать КМ с~<<мягкой>> 
структурой (полимеры, полимерные композиты), программный пакет LAMMPS 
({\sf https://www.lammps.org}), ис\-поль\-зу\-емый для моделирования в~рамках 
классической молекулярной динамики на атомистическом и~мезомасштабном 
уровнях полимерных, металлических, биологических сис\-тем и~др. Каждый из 
разрабатываемых программных продуктов обладает своими достоинствами 
и~областями применения. В~связи с~большим разнообразием типов 
КМ и~все воз\-рас\-та\-ющи\-ми требованиями к~наборам 
свойств, которыми они должны обладать, пред\-став\-ля\-ет\-ся важ\-ным\linebreak создание 
программных средств, поз\-во\-ля\-ющих оперативно вы\-стра\-и\-вать сис\-тем\-ные 
решения в~об\-ласти\linebreak многомасштабного моделирования с~применением 
высокопроизводительных вычислений, поз\-во\-ля\-ющих проводить моделирование от  
атом\-но-крис\-тал\-ли\-че\-ско\-го до мак\-ро\-уров\-ня. Такие системы \mbox{позволят} 
генерировать и~выполнять в~автоматическом режиме сценарии проведения 
расчетов под конкретную задачу, включать в~вычислительную схему расчеты на 
всех необходимых мас\-штаб\-ных уровнях. Для предсказательного моделирования 
структурных свойств различных классов КМ такой 
подход поз\-во\-ля\-ет создавать вы\-чис\-ли\-тель\-ную среду, в~которой задействованы 
возможности CАE-сис\-тем для верх\-не\-уров\-не\-во\-го (мак\-ро-) моделирования, 
методы анализа экспериментальных и~аналитических данных, а также 
собственные разработки и~пакетные приложения для расчетов на атом\-но-крис\-тал\-ли\-че\-ском и~наноуровне.

\vspace*{-9pt}

\section{Многомасштабная модель для~расчета структурных 
свойств композиционных материалов}

     В работе~\cite{2-ab} представлена общая схема многомасштабной модели 
для расчета структурных характеристик КМ. Для ее 
описания используется тео\-ре\-ти\-ко-мно\-жест\-вен\-ный аппарат, изложенный 
в~\cite{1-ab, 2-ab}. На ее основе формируются схемы для расчета разных классов 
КМ: нанокомпозитов на основе полимерной мат\-ри\-цы, 
КМ с~металлической мат\-ри\-цей, полимерных 
КМ с~углеволокном и~др.

\vspace*{-9pt}
     
     \subsection*{Основные уровни моделирования}
     
     \vspace*{-2pt}
     
     
     \textbf{Квантово-механический}. Рассматриваются отдельные молекулы. 
Решается уравнение Шредингера, определяется атомарная струк\-ту\-ра молекул 
полимера и~наполнителя, строится электронная струк\-ту\-ра и~рас\-счи\-ты\-ва\-ет\-ся 
когезионная энергия, рас\-счи\-ты\-ва\-ют\-ся меж\-атом\-ные и~меж\-мо\-ле\-ку\-ляр\-ные силы, 
определяются отдельные фи\-зи\-ко-хи\-ми\-че\-ские свойства.
     
     \textbf{Молекулярно-динамический}. Изучаются ан\-самб\-ли из молекул. 
Решаются уравнения молекулярной динамики с~использованием потенциалов 
межатомного взаимодействия, рас\-счи\-ты\-ва\-ют\-ся структурные характеристики 
мат\-ри\-цы (полимерной, металлической и~др.), наполнителя (нанотрубки, 
волокна и~др.), физические свойства. 
     
     \textbf{Мезоскопический}. Рас\-смат\-ри\-ва\-ют\-ся крупнозернистые модели. 
Используется упрощенное строение молекул. Цель моделирования на 
мезоуровне~--- получение распределения час\-тиц \mbox{наполнителя} в~мат\-ри\-це 
(полимерной, металлической и~др.)\ с~по\-сле\-ду\-ющим расчетом инженерных 
свойств полученных сис\-тем. 

\begin{figure*}[b] %fig1
\vspace*{8pt}
  \begin{center}  
    \mbox{%
\epsfxsize=133.618mm
\epsfbox{abg-1.eps}
}

\end{center}
\vspace*{-2pt}
\Caption{Схема многомасштабной композиции $\mathbf{MK}_{0,1,2,3,4}^{(\mathrm{Ti/Mo})}$ 
для расчета структурных свойств МКМ}
\end{figure*}
     
     \textbf{Континуальный} (\textbf{макроскопический}). Проводится расчет 
инженерных свойств (механические свойства, теп\-ло\-про\-вод\-ность и~др.). Задачи 
решаются с~применением механики сплош\-ных сред, гид\-ро\-ди\-на\-ми\-ки, тео\-рии 
упру\-гости. Применяются метод конечных элементов, методы решения краевых 
задач для моделирования различных процессов. 
     
     Рассмотрим пример построения многомасштабной композиции для 
тес\-то\-во\-го рас\-че\-та структурных свойств металлического 
КМ (МКМ) на основе Ti (титана), армированного волокнами Mo 
(молибдена). На сегодня Ti и~титановые сплавы стали очень привлекательными 
материалами для перспективных сфер применения благодаря таким свойствам, 
как низкая плот\-ность, высокие механические свойства и~коррозионная стой\-кость. 
Использование данных материалов в~конструкциях самолетов (реактивный 
двигатель и~фюзеляж) и~применение в~автомобильной про\-мыш\-лен\-ности рас\-тут 
быст\-ры\-ми темпами. Одним из способов совершенствования\linebreak титановых сплавов 
стало их применение в~качестве мат\-ри\-цы для КМ, 
армированных волокнами, например из Mo, которые обладают очень \mbox{хорошими} 
механическими свойствами ({\sf http://\linebreak viam-works.ru/ru/articles?art\_id=1103}). 
     
     Задействуем четыре перечисленных выше масштабных уров\-ня (не считая 
нулевого). Используя обозначения из~\cite{1-ab, 2-ab}, для построения 
многомасштабной композиции 
$$
\mathbf{MK}_{0,1,2,3,4}^{(\mathrm{Mo}, \mathrm{Ti}; 
1{,}1; 1{,}2; 2{,}1; 2{,}2; 3{,}1; 4{,}1)}= \mathbf{MK}_{0,1,2,3,4}^{(\mathrm{Ti/Mo})}
$$ 

\vspace*{-3pt}

\noindent
приведем экземпляры базовых мо\-де\-лей-ком\-по\-зи\-ций: 

\vspace*{-9pt}

\noindent
     \begin{align*}
     \mathbf{El}_{01}^{\mathrm{Ti}}:& \left\{ V_{01}^{\mathrm{Ti}}, 
X_{01}^{\mathrm{Ti}}, \mathrm{MA}_{01}^{\mathrm{Ti}}\right\};\\[-3pt]
     \mathbf{El}_{01}^{\mathrm{Mo}}:& \left\{ V_{01}^{\mathrm{Mo}}, 
X_{01}^{\mathrm{Mo}}, \mathrm{MA}_{01}^{\mathrm{Mo}}\right\};\\[-3pt]
\mathbf{MC}_{11}^{\mathrm{Ti}}:& \left\{ V_{11}^{\mathrm{Ti}}, 
X_{11}^{\mathrm{Ti}}, \mathrm{MA}_{11}^{\mathrm{Ti}}\right\};\\[-3pt]
\mathbf{MC}_{11}^{\mathrm{Mo}}:& \left\{ V_{11}^{\mathrm{Mo}}, 
X_{11}^{\mathrm{Mo}}, \mathrm{MA}_{11}^{\mathrm{Mo}}\right\};
\end{align*}

\noindent
\begin{align*}
               \mathbf{MC}_{12}^{\mathrm{Ti}}:& \left\{ V_{12}^{\mathrm{Ti}}, 
X_{12}^{\mathrm{Ti}}, \mathrm{MA}_{12}^{\mathrm{Ti}}\right\};\\
     \mathbf{MC}_{12}^{\mathrm{Mo}}:& \left\{ V_{12}^{\mathrm{Mo}}, 
X_{12}^{\mathrm{Mo}}, \mathrm{MA}_{12}^{\mathrm{Mo}}\right\};\\
     \mathbf{MC}_{21}^{\mathrm{Ti}}:& \left\{ V_{21}^{\mathrm{Ti}}, 
X_{21}^{\mathrm{Ti}}, \mathrm{MA}_{21}^{\mathrm{Ti}}\right\};\\
     \mathbf{MC}_{21}^{\mathrm{Mo}}:& \left\{ V_{21}^{\mathrm{Mo}}, 
X_{21}^{\mathrm{Mo}}, \mathrm{MA}_{21}^{\mathrm{Mo}}\right\};\\
     \mathbf{MC}_{22}^{\mathrm{Ti}}:& \left\{ V_{22}^{\mathrm{Ti}}, 
X_{22}^{\mathrm{Ti}}, \mathrm{MA}_{22}^{\mathrm{Ti}}\right\};\\
     \mathbf{MC}_{22}^{\mathrm{Mo}}:& \left\{ V_{22}^{\mathrm{Mo}}, 
X_{22}^{\mathrm{Mo}}, \mathrm{MA}_{22}^{\mathrm{Mo}}\right\};\\
     \mathbf{MC}_{31}^{\mathrm{Ti}/\mathrm{Mo}}:& \left\{
     V_{31}^{\mathrm{Ti}/\mathrm{Mo}}, X_{31}^{\mathrm{Ti}/\mathrm{Mo}}, 
\mathrm{MA}_{31}^{\mathrm{Ti}/\mathrm{Mo}}\right\};\\
     \mathbf{MC}_{41}^{\mathrm{Ti}/\mathrm{Mo}}:& \left\{
     V_{41}^{\mathrm{Ti}/\mathrm{Mo}}, X_{41}^{\mathrm{Ti}/\mathrm{Mo}}, 
\mathrm{MA}_{41}^{\mathrm{Ti}/\mathrm{Mo}}\right\}.
     \end{align*}
     
     Согласно схематическому пред\-став\-ле\-нию (рис.~1) многомасштабная 
композиция $\mathbf{MK}_{0,1,2,3,4}^{(\mathrm{Ti/Mo})}$ со\-сто\-ит из связанных между 
собой экземпляров базовых моделей композиций, размещенных на 
со\-от\-вет\-ст\-ву\-ющих мас\-штаб\-ных уровнях. На наноуровне проводится  
мо\-ле\-ку\-ляр\-но-ди\-на\-ми\-че\-ское моделирование структурных свойств 
титановой мат\-ри\-цы и~молибденовых волокон. На мезоуровне рас\-смат\-ри\-ва\-ет\-ся 
распределение час\-тиц в~МКМ, на мак\-ро\-уров\-не проводится расчет механических 
свойств МКМ.

\setcounter{figure}{2}
\begin{figure*}[b] %fig3
\vspace*{-6pt}
  \begin{center}  
    \mbox{%
\epsfxsize=120.383mm
\epsfbox{abg-3.eps}
}

\end{center}
\vspace*{-9pt}
\Caption{Пример сценария с~цик\-лом}
\end{figure*}
%\pagebreak
     
\vspace*{-10pt}

\section{Программный комплекс}

\vspace*{-2pt}

   Программный комплекс, интегрированный с~расчетными пакетами 
и~модулями, размещается на высокопроизводительных многоядерных сис\-те\-мах, 
оснащенных мощными графическими процессорами. Это связано с~тем, что 
исполнение вычислительных экспериментов, а~так\-же обработка 
и~анализ результатов вы\-чис\-ли\-тель\-ных  экспериментов
 ориентированы на 
распределенные сис\-те\-мы сбора, хранения и~обработки больших данных. В~основе 
программного комплекса лежит интеграционная платформа для 
многомасштабного моделирования, которая объединяет информационные потоки 
на разных мас\-штаб\-ных уровнях. При решении конкретной задачи, такой как 
расчет структурных особенностей, механических или иных свойств 
КМ, при изучении процессов их де\-гра\-да\-ции 
и~разрушения и~др.\ выделяются конкретные уров\-ни моделирования, которые 
необходимо задействовать. Первоначально строится многомасштабная 
композиция~--- информационный аналог\linebreak мно\-го\-мас\-штаб\-ной  
фи\-зи\-ко-ма\-те\-ма\-ти\-че\-ской модели. Для программной реализации на базе 
интеграционной платформы~\cite{4-ab} из име\-ющих\-ся программных модулей 
формируется вы\-чис\-ли\-тель\-ный \mbox{комплекс}~\cite{5-ab, 6-ab}.
   
   Перечислим пользовательские роли в~интеграционной плат\-фор\-ме 
мно\-го\-мас\-штаб\-но\-го моделирования:
   \begin{itemize}
\item разработчик вычислительных модулей реализует расчетный модуль или 
осуществляет конфигурирование при\-клад\-но\-го па\-кета;\\[-15pt]
\item системный разработчик создает веб-сер\-ви\-сы для вы\-чис\-ли\-тель\-но\-го модуля 
и~интегрирует его в~плат\-форму;\\[-15pt]
\item разработчик расчетных сценариев создает сценарии в~среде моделирования;\\[-15pt]
\item ученый-исследователь прикладной об\-ласти запускает расчетные сценарии 
с~различными па\-ра\-мет\-ра\-ми и~анализирует ре\-зуль\-таты.
\end{itemize}
    
    Как отмечалось в~\cite{5-ab, 6-ab}, программный комплекс предназначен для 
создания и~исполнения сценариев многомасштабных расчетов для моделирования 
структурных свойств композитных материалов.
    
    Сценарий~--- программная реализация мно\-го\-мас\-штаб\-ной композиции~--- 
пред\-став\-ля\-ет собой алгоритм последовательного выполнения расчетов отдельных 
физических характеристик материалов, входящих в~со\-став композита, 
посредством интегрированных с~программным комплексом вы\-чис\-ли\-тель\-ных 
модулей. Среда моделирования сценариев поз\-во\-ля\-ет создавать или 
модифицировать сценарии, учитывая особенности конкретного 
КМ и~тре\-бу\-емые свойства.



 
    
    Среда исполнения сценариев дает возможность осуществить его запуск 
    с~заданными входными па\-ра\-мет\-ра\-ми, отслеживать его выполнение в~целом и~по 
со\-став\-ным задачам, про\-смат\-ри\-вать входные и~выходные данные (результаты 
расчетов). Интеграционная роль среды исполнения заключается\linebreak в~формировании 
входных данных для вычислительных модулей в~со\-от\-вет\-ст\-ву\-ющем формате 
и~единицах измерения, отслеживании работы модулей,\linebreak получении конечного 
результата расчета и~преобразовании его в~формат и~единицы измерения, 
до\-ступ\-ные для других модулей сценария. Таким образом, среда исполнения 
обеспечивает соответствие потока исполнения вы\-чис\-ли\-тель\-ных модулей 
заданному алгоритму в~сценарии и~це\-лост\-ность потока данных между блоками 
сценария. Кроме того,\linebreak среда исполнения предостав\-ля\-ет общие словари для\linebreak 
согласования вход\-ных-вы\-ход\-ных данных вы\-чис\-ли\-тель\-ных экспериментов, 
такие как справочник\linebreak химических элементов и~их свойств, химических формул 
веществ, ис\-поль\-зу\-емых в~композитных материалах, типы крис\-тал\-ли\-че\-ских 
сис\-тем, типы атомных радиусов, пространственные группы.

\setcounter{figure}{3}
\begin{figure*}[b] %fig4
\vspace*{-9pt}
  \begin{center}  
    \mbox{%
\epsfxsize=163mm
\epsfbox{abg-4.eps}
}

\end{center}
\vspace*{-9pt}
\Caption{Сценарий для расчета МКМ}
\end{figure*}

\vspace*{-10pt}
   
    \subsection*{Алгоритм программы}
    
    \vspace*{-2pt}
    
    Алгоритм исполнения сценария основан на стандарте BPMN~2.0 и~со\-сто\-ит из 
сле\-ду\-ющих ключевых элементов (рис.~2).

{ \begin{center}  %fig2
 \vspace*{6pt}
    \mbox{%
\epsfxsize=70.82mm
\epsfbox{abg-2.eps}
}

\vspace*{6pt}

\noindent
{{\figurename~2}\ \ \small{
Пример простого сценария
}}
\end{center}
}

%\vspace*{6pt} 

\noindent
\begin{description}
\item[Э1.]  Точка начала выполнения сценария. В~свойствах этого элемента 
указывается список кодов физических величин, которые пользователь дол\-жен 
будет ввес\-ти перед запуском сценария.

\item[Э2.] Сплошная стрелка определяет строгую по\-сле\-до\-ва\-тель\-ность 
выполнения шагов сценария.

 \begin{figure*}[b] %fig5
  \vspace*{1pt}
  \begin{center}  
    \mbox{%
\epsfxsize=131mm %.834mm
\epsfbox{abg-5.eps}
}

\end{center}
\vspace*{-9pt}
  \Caption{Сценарий для расчета механических свойств полимерного нанокомпозита}
  \end{figure*}

\item[Э3.] Вычислительная задача пред\-став\-ля\-ет\-ся в~BPMN как <<внеш\-няя 
сервисная задача>> (External Service Task). В~поле topic в~настройках 
задачи вводится название очереди задач со\-от\-вет\-ст\-ву\-юще\-го 
вы\-чис\-ли\-тель\-но\-го модуля. Например, для  
кван\-то\-во-ме\-ха\-ни\-че\-ско\-го расчета на пакете VASP вводится 
<<vasp\_topic>>. Список до\-ступ\-ных вы\-чис\-ли\-тель\-ных модулей 
с~названиями очередей хранится в~базе данных в~таб\-ли\-це <<Module>>.\\[-15pt]

\item[Э4.] Точка завершения выполнения сценария. Если в~сценарии существует 
ветвление, точек завершения может быть несколько.



    \item[Э5.] Шаг сценария, в~рамках которого выполняется скрипт, заданный 
пользователем. В~па\-ра\-мет\-рах задачи может быть указан язык скрип\-та и~сам 
скрипт. Доступны языки Groovy и~Jython (реализация языка Python на Java). 
Скрип\-ты могут использоваться для изменения входных и~выходных па\-ра\-мет\-ров, 
небольших вы\-чис\-ле\-ний на основе текущих до\-ступ\-ных данных сценария. 
В~примере на рис.~3 в~цик\-ле определяется список векторов 
крис\-тал\-ли\-че\-ской решетки, по которым будет проводиться кван\-то\-во-ме\-ха\-ни\-че\-ский 
рас\-чет деформированной решетки.\\[-19.5pt]

\begin{figure*}[b] %fig6
\vspace*{1pt}
  \begin{center}  
    \mbox{%
\epsfxsize=163mm
\epsfbox{abg-6.eps}
}

\end{center}
\vspace*{-9pt}
\Caption{Сценарий для расчета КМ с~полимерной мат\-ри\-цей 
и~наполнителем из углеволокна}
\end{figure*}
    
    \item[Э6.] Подпроцесс сценария <<цикл с~параллельным запуском>>  
(Parallel multi-instance) позволяет параллельно запустить выполнение час\-ти 
сцена- %\linebreak\vspace*{-12pt}

\columnbreak

\noindent
рия несколько раз. В~свойствах подпроцесса требуется указать коллекцию 
(Collection), по элементам которой будет проводиться ите\-ри\-ро\-ва\-ние, и~название 
переменной цикла (Element Variable). Весь элемент считается выполненным, когда 
все параллельно выполняющиеся подпроцессы завершат свою работу. Например, 
если требуется запустить кван\-то\-во-ме\-ха\-ни\-че\-ский расчет для некоторого 
множества деформированных решеток (для определения в~дальнейшем констант 
упру\-гости), предварительно в~скрип\-те перед цик\-лом формируется список 
деформированных векторов решетки и~сохраняется в~переменную процесса. 
Далее для каж\-дой деформации параллельно вызывается\linebreak\vspace*{-12pt}

\pagebreak

\noindent  
кван\-то\-во-ме\-ха\-ни\-че\-ский модуль VASP для расчета энергии и~объема 
решетки. Получившаяся таб\-ли\-ца с~данными может использоваться для расчета 
констант элас\-тич\-ности, модуля упру\-гости и~других свойств материала.
\end{description}

\vspace*{-9pt}
  
  \subsection*{Примеры тестовых сценариев для~расчета~структурных~характеристик 
  и~отдельных~свойств различных классов 
композиционных материалов}


     
     \textbf{Пример~1.} Тестовый сценарий для расчета структурных свойств 
КМ с~металлической мат\-ри\-цей (рис.~4).

\smallskip

     
     \textbf{Пример~2.} Тестовый сценарий для расчета механических свойств 
полимерного нанокомпозита (полифениленсульфид с~углеродными нанотрубками). 
На сле\-ду\-ющем этапе проекта планируется расширить сценарий для 
оценки влияния процентного содержания углеродных нанотрубок на изменение 
коэффициента теп\-ло\-про\-вод\-ности полимерного нанокомпозита (рис.~5).
  
 
     
     \textbf{Пример~3.} Тестовый сценарий для расчета механических свойств 
КМ с~полимерной мат\-ри\-цей (эпоксидная смола) и~углеволокном
(рис.~6).

\vspace*{-6pt}

\section{Выводы}

\vspace*{-2pt}

     В работе представлен программный комплекс для расчета структурных 
характеристик КМ с~тре\-бу\-емы\-ми свойствами. В~его 
основе лежит интеграционная плат\-фор\-ма для многомасштабного моделирования, 
которая объединяет информационные потоки на разных мас\-штаб\-ных уровнях. На 
ее основе формируются схемы для рас\-че\-та структурных характеристик разных 
клас\-сов КМ: нанокомпозитов на основе полимерной 
мат\-ри\-цы, КМ с~металлической мат\-ри\-цей, полимерных 
КМ с~углеволокном и~другие. Разработанные подходы 
поз\-во\-ля\-ют моделировать свойства КМ (механические, 
теп\-ло\-вые и~др.), а~так\-же многомасштабные процессы, связанные с~усталостным 
разрушением при случайных по\-вреж\-де\-ни\-ях в~ходе эксплуатации, и~другие 
динамические процессы. Программный комплекс со\-сто\-ит из программных 
модулей и~базируется на типовых сер\-ви\-сах вы\-чис\-ли\-тель\-ных модулей, общей 
интеграционной оболочки и~модулей сценариев. Про\-грам\-мные решения 
сертифицированы. В~дальнейшем планируется раз\-ра\-бо\-тать 
полнофункциональную про\-грам\-мную сис\-те\-му с~целью решения различных 
классов обратных задач в~об\-ласти наук о~материалах. Разработанные подходы 
в~сочетании с~экспериментальными данными могут быть использованы для 
лучшего понимания физических основ изменения свойств в~за\-ви\-си\-мости от 
струк\-ту\-ры и,~как след\-ст\-вие, для уде\-шев\-ле\-ния и~уско\-ре\-ния поиска новых 
КМ с~заданными свойствами.

\vspace*{-6pt}
   
{\small\frenchspacing
 {%\baselineskip=10.8pt
 %\addcontentsline{toc}{section}{References}
 \begin{thebibliography}{9}
 
 \vspace*{-2pt}
   
   \bibitem{1-ab}
   \Au{Абгарян К.\,К.} Многомасштабное моделирование в~задачах структурного 
материаловедения.~--- М.: МАКСПресс, 2017. 284~с.
\bibitem{2-ab}
\Au{Абгарян~К.\,К.} Информационная технология по\-стро\-ения многомасштабных моделей 
в~задачах вы\-чис\-ли\-тель\-но\-го материаловедения~// Сис\-те\-мы высокой до\-ступ\-ности, 2018. Т.~14. 
№\,2. С.~9--15.
\bibitem{3-ab}
\Au{Naffakh M., D$\acute{\!\mbox{{\!\ptb{\i}}}}$ez-Pascuala~A.\,M., Marcoa~C., Ellisa~G.} Morphology and thermal properties of novel poly (phenylene sulfide) 
hybrid nanocomposites based on single-walled carbon nanotubes and 8 inorganic fullerene-like WS~2 
nanoparticles~// J.~Mater. Chem., 2012. Vol.~22. No.\,4. P.~1418--1425.
\bibitem{4-ab}
\Au{Абгарян К.\,К., Гаврилов~Е.\,С.} Распределенная информационная сис\-те\-ма для расчета 
структурных свойств композиционных материалов~// Информатика и~её применения, 2021. 
Т.~15. Вып.~4. С.~50--58. doi: 10.14357/ 19922264210407.
\bibitem{5-ab}
\Au{Гаврилов Е.\,С.} Интегрированный интерфейс к~модулю сплош\-но\-сред\-но\-го взаимодействия. 
Свидетельство о~регистрации программ для ЭВМ №\,2021681058, 2021.
\bibitem{6-ab}
\Au{Гаврилов Е.\,С.} Программные средства для хранения и~обмена данными в~задачах 
моделирования композитных материалов. Свидетельство о~регистрации программ для ЭВМ 
№\,2021681762, 2021.

\end{thebibliography}

 }
 }

\end{multicols}

\vspace*{-8pt}

\hfill{\small\textit{Поступила в~редакцию 22.01.22}}

\vspace*{8pt}

%\pagebreak

%\newpage

%\vspace*{-28pt}

\hrule

\vspace*{2pt}

\hrule

%\vspace*{-2pt}

\def\tit{SOFTWARE PACKAGE FOR MULTISCALE MODELING OF~STRUCTURAL PROPERTIES 
OF~COMPOSITE MATERIALS}


\def\titkol{Software package for multiscale modeling of~structural properties 
of~composite materials}


\def\aut{K.\,K.~Abgaryan$^1$ and~E.\,S.~Gavrilov$^{1,2}$}

\def\autkol{K.\,K.~Abgaryan and~E.\,S.~Gavrilov}

\titel{\tit}{\aut}{\autkol}{\titkol}

\vspace*{-18pt}


\noindent
$^1$Federal Research Center ``Computer Science and Control'' of the Russian Academy of Sciences, 
44-2~Vavilov\linebreak
$\hphantom{^1}$Str., Moscow 119333, Russian Federation

\noindent
$^2$Moscow Aviation Institute (National Research University), 4~Volokolamskoe Shosse, Moscow 
125080, Russian\linebreak
$\hphantom{^1}$Federation

\def\leftfootline{\small{\textbf{\thepage}
\hfill INFORMATIKA I EE PRIMENENIYA~--- INFORMATICS AND
APPLICATIONS\ \ \ 2022\ \ \ volume~16\ \ \ issue\ 1}
}%
 \def\rightfootline{\small{INFORMATIKA I EE PRIMENENIYA~---
INFORMATICS AND APPLICATIONS\ \ \ 2022\ \ \ volume~16\ \ \ issue\ 1
\hfill \textbf{\thepage}}}

\vspace*{3pt} 
      
      
  
\Abste{Today, creation of new composite materials and methods of their construction with predictable 
properties is one of the urgent and most important tasks connected with modernization of 
industrial production in our country. For their solution, technologies of multiscale computer modeling 
are actively developed. They have become a~link between fundamental physics (chemistry) and 
engineering materials science. The paper presents a~software package for modeling structural 
properties of composite materials which allows solving a~number of problems of this class. It is 
focused on high-performance computations. The complex is based on an original multiscale 
technology which allows one to promptly conduct multivariate analysis of different classes of 
composite materials and conduct research on designing the new ones with predictable properties. The 
developed approaches in combination with experimental data can be used for a~better understanding of 
the physical foundations of the change of properties depending on the structure and, as a~consequence, 
for cheaper and faster search of new composite materials with predetermined properties.}

\KWE{multiscale modeling; composite materials; integration platform; software package; distributed 
system}



\DOI{10.14357/19922264220113}

\vspace*{-16pt}

\Ack
\noindent
The research was supported by the Ministry of Science and Higher Education of the Russian 
Federation (project No.\,075-15-2020-799).




%\vspace*{4pt}

  \begin{multicols}{2}

\renewcommand{\bibname}{\protect\rmfamily References}
%\renewcommand{\bibname}{\large\protect\rm References}

{\small\frenchspacing
 {%\baselineskip=10.8pt
 \addcontentsline{toc}{section}{References}
 \begin{thebibliography}{9}
\bibitem{1-ab-1}
\Aue{Abgaryan, K.\,K.} 2017. \textit{Mnogomasshtabnoe modelirovanie v~zadachakh strukturnogo 
materialovedeniya} [Multiscale modeling for structural materials science applications]. Moscow: 
MAKS Press. 284~p.

\vspace*{-2pt}

\bibitem{2-ab-1}
\Aue{Abgaryan, K.\,K.} 2018. In\-for\-ma\-tsi\-on\-naya tekh\-no\-lo\-giya po\-stro\-eniya mno\-go\-mas\-shtab\-nykh 
mo\-de\-ley v~za\-da\-chakh vy\-chis\-li\-tel'\-no\-go ma\-te\-ri\-a\-lo\-ve\-de\-niya 
[Information technology is the construction 
of multi-scale models in problems of computational materials science]. \textit{Sistemy vysokoy 
dostupnosti} [Highly Available Systems] 14(2):9--15.
\bibitem{3-ab-1}
\Aue{Naffakh, M., A.\,M.~D$\acute{\mbox{{\!\ptb{\i}}}}$ez-Pascuala, C.~Marcoa, and G.~Ellisa.} 
2012. Morphology and thermal properties of novel poly (phenylene sulfide) hybrid nanocomposites 
based on single-walled carbon nanotubes and~8~inorganic fullerene-like WS~2 nanoparticles. 
\textit{J.~Mater. Chem.}  
22(4):1418--1425.
{\looseness=1

}
\bibitem{4-ab-1}
  \Aue{Abgaryan, K.\,K., and E.\,S.~Gavrilov.} 2021. 
  Ras\-pre\-de\-len\-naya in\-for\-ma\-tsi\-on\-naya sis\-te\-ma   dlya 
ras\-che\-ta struk\-tur\-nykh svoystv kom\-po\-zi\-tsi\-on\-nykh ma\-te\-ri\-alov 
[Distributed information system for 
calculating the structural properties of composite materials]. \textit{Informatika i~ee Primeneniya~--- 
Inform. Appl.} 15(4):50--58. doi: 10.14357/19922264210407.
\bibitem{5-ab-1}
  \Aue{Gavrilov, E.\,S.} 2021. In\-teg\-ri\-ro\-van\-nyy in\-ter\-feys k~mo\-du\-lyu 
  splosh\-no\-sred\-no\-go 
vza\-imo\-dey\-stviya [Integrated interface to the solid-medium interaction module]. Certificate on official 
registration of the computer program No.\,2021681058.
\bibitem{6-ab-1}
  \Aue{Gavrilov, E.\,S.} 2021. Pro\-gram\-mnye sred\-st\-va dlya khra\-ne\-niya 
  i~ob\-me\-na dan\-ny\-mi  v~za\-da\-chakh mo\-de\-li\-ro\-va\-niya kom\-po\-zit\-nykh ma\-te\-ri\-a\-lov 
  [Software tools for data persistence and data flow in 
composite materials modeling tasks]. Certificate on official registration of the computer program 
No.\,2021681762.
\end{thebibliography}

 }
 }

\end{multicols}

\vspace*{-6pt}

\hfill{\small\textit{Received January 22, 2022}}


\Contr

\noindent
\textbf{Abgaryan Karine K.} (b.\ 1963)~--- Doctor of Science in physics and mathematics, principal 
scientist, A.\,A.~Dorodnicyn Computing Center, Federal Research Center ``Computer Science and 
Control'' of the Russian Academy of Sciences, 40~Vavilov Str., Moscow 119333, Russian Federation; 
head of department, Moscow Aviation Institute (National Research University), 4~Volokolamskoe 
Shosse, Moscow 125080, Russian Federation; \mbox{kristal83@mail.ru}

\vspace*{3pt}

\noindent
\textbf{Gavrilov Evgeny S.} (b.\ 1982)~--- scientist, A.\,A.~Dorodnicyn Computing Center, Federal 
Research Center ``Computer Science and Control'' of the Russian Academy of Sciences, 40~Vavilov 
Str., Moscow 119333, Russian Federation; senior lecturer, Moscow Aviation Institute (National 
Research University), 4~Volokolamskoe Shosse, Moscow 125080, Russian Federation; 
\mbox{eugavrilov@gmail.com}
       

\label{end\stat}

\renewcommand{\bibname}{\protect\rm Литература}           %13
\def\stat{betelin}

\def\tit{ОСНОВНЫЕ ПОНЯТИЯ ПРОГРАММИРОВАНИЯ 
В~ИЗЛОЖЕНИИ~ДЛЯ~ДОШКОЛЬНИКОВ$^*$}

\def\titkol{Основные понятия программирования в~изложении для 
дошкольников}

\def\aut{В.\,Б.~Бетелин$^1$, А.\,Г.~Кушниренко$^2$, А.\,Г.~Леонов$^3$}

\def\autkol{В.\,Б.~Бетелин, А.\,Г.~Кушниренко, А.\,Г.~Леонов}

\titel{\tit}{\aut}{\autkol}{\titkol}

\index{Бетелин В.\,Б.}
\index{Кушниренко А.\,Г.}
\index{Леонов А.\,Г.}
\index{Betelin V.\,B.}
\index{Kushnirenko A.\,G.}
\index{Leonov A.\,G.}
 

{\renewcommand{\thefootnote}{\fnsymbol{footnote}} \footnotetext[1]
{Работа выполнена по теме 0065-2019-0010 госзадания 2020~года 
в~отделе учебной информатики Федерального 
научного центра <<На\-уч\-но-ис\-сле\-до\-ва\-тель\-ский институт системных исследований>> Российской 
академии наук.}}


\renewcommand{\thefootnote}{\arabic{footnote}}
\footnotetext[1]{Федеральный научный центр <<На\-уч\-но-ис\-сле\-до\-ва\-тель\-ский институт системных исследований>> 
Российской академии наук, \mbox{betelin@niisi.msk.ru}}
\footnotetext[2]{Федеральный научный центр <<На\-уч\-но-ис\-сле\-до\-ва\-тель\-ский  институт системных исследований>> 
Российской академии наук, \mbox{agk\_@mail.ru}}
\footnotetext[3]{Московский государственный университет им.\ М.\,В.~Ломоносова; Федеральный научный центр  
<<На\-уч\-но-ис\-сле\-до\-ва\-тель\-ский институт системных исследований>> Российской академии наук; 
Московский педагогический государственный университет, \mbox{dr.l@vip.niisi.ru}}

%\vspace*{-6pt}



\Abst{Развитие информационных технологий сформировало социально-экономический 
запрос на снижение возраста знакомства детей с~программированием. В~результате 
шестилетних усилий авторам удалось разработать и~массово внедрить годовой курс 
программирования для дошкольников, построенный на метафоре программного управления. 
В~процессе развития курса удалось отобрать и~сформулировать набор основных понятий 
программирования, который может быть освоен дошкольниками возраста~6+  
в~дея\-тель\-ност\-но-иг\-ро\-вой форме. Этот набор понятий вводится на примерах программ 
управления движущимися и~неподвижными объектами с~интуитивно понятными, 
обозримыми системами команд. Курс строится на базе беcтекстовой пиктографической 
системы программирования <<ПиктоМир>> разработки ФНЦ НИИСИ РАН. Разработанное 
для курса про\-грам\-мно-ме\-то\-ди\-че\-ское наполнение позволяет каждому дошкольнику 
к~концу курса получить опыт составления и~отладки 120--150~простейших программ.}
  
  \KW{информатика; робот; программа; компьютер; язык программирования; дошкольник; 
<<ПиктоМир>>; пиктограмма}

\DOI{10.14357/19922264200308} 
 
%\vspace*{-6pt}


\vskip 10pt plus 9pt minus 6pt

\thispagestyle{headings}

\begin{multicols}{2}

\label{st\stat}
  
\section{Введение}

  Законодательными структурами власти России федерального уровня 
выдвигаются предложения по понижению возраста знакомства детей 
с~информатикой и~программированием на уровень системы дошкольного 
образования~[1]. Настоящая статья посвящена описанию конкретной методики 
такого\linebreak понижения и~суммирует шестилетний опыт разработки и~массового 
внедрения годового курса <<\mbox{Алгоритмика} для дошколят>>, проводимого 
совместными усилиями ФНЦ НИИСИ РАН и~Департамента образования 
администрации г.~Сур\-гу\-та. Начиная с~осени 2018~г.\ годовой курс проходят 
все выпускники всех подготовительных групп всех до\-школь\-но-об\-ра\-зо\-ва\-тель\-ных 
организаций г.~Сургу\-та~--- 
около 6~тыс.\ детей. В~курсе используется\linebreak
 бестекстовая учебная система 
программирования <<ПиктоМир>>, разработка которой была начата в~ФНЦ 
НИИСИ РАН около 10~лет назад~[2] и~продолжается сегодня в~работах по теме 
госзадания Минобрнауки РФ <<Разработка, реализация и~внедрение семейства 
интегрированных многоязыковых сред программирования$\ldots$>>~[3]. 
Система <<ПиктоМир>> и~методическое обеспечение курса~--- свободно 
распространяемые. Их можно загрузить с~сайта НИИСИ РАН или работать 
с~ними с~по\-мощью браузера через веб-ин\-тер\-фейс~[4].
  
  Информатизация дошкольного образования~--- очень широкая и~важная для 
современной цивилизации тема~[5]. В~данной статье, в~рамках более общей 
работы над 4-лет\-ним курсом программирования для дошкольников и~младших 
школьников, упор делается на конкретной задаче~--- организации первых 
занятий курса программирования для дошкольников. Этот вопрос особенно 
важен, поскольку именно на первых занятиях закладывается фундамент курса, 
осваивается набор основных понятий. 
  
  Накопленный опыт позволил авторам предложить набор основных понятий 
программирования, который может быть органично освоен дошкольниками 
в~дея\-тель\-ност\-но-иг\-ро\-вой форме. Этот набор понятий призван раскрыть 
одну из <<больших идей>> нашей цивилизации~--- {\bfseries\textit{принцип 
программного управ\-ления}}. 
  
  Согласно рубрикации А.\,Л.~Семенова~\cite[п.~12]{6-bet}, одной из целей 
курса информатики должно быть <<освоение информационной картины 
мира>>. Принцип программного управления и~строит такую картину 
в~терминах и~образах, понятных шестилетнему ребенку. Схематично принцип 
может быть объяснен так: \textbf{любую работу, которую человек может 
выполнить, командуя механическим по\-мощ\-ни\-ком-ро\-бо\-том, можно 
перепоручить компьютеру, если человеку удастся составить программу 
выполнения той деятельности, которую роботу надлежит выполнить}. 
  
  Методически правильным представляется двухэтапное изложение принципа 
программного управ\-ле\-ния, при котором на первом этапе излагается 
простейший вариант принципа~--- без обратной связи. Практика работы 
с~дошкольниками показала, что именно этот этап оказывается 
ос\-но\-во\-по\-ла\-гаю\-щим и~более трудным, чем второй этап, на котором вводится 
обратная связь.
  
\section{Принцип программного управления без~обратной связи}

  \textbf{Программа}~--- это план будущей деятельности,\linebreak
   в~процессе которой 
один объект~--- \textbf{компьютер}~-- управляет другим объектом~--- 
\textbf{роботом}~--- по программе, заранее составленной человеком~--- 
\textbf{программистом}~--- по заранее известным \textbf{правилам\linebreak 
составления программ} (\textbf{язык программирования}). Процесс 
\textbf{выполнения} программы компьютером состоит в~том, что компьютер, 
следуя программе, некоторым заранее установленным способом дает роботу 
\textbf{команды}, которые тот \textbf{исполняет}, докладывая компьютеру об 
окончании исполнения каждой команды и~готовности к~приему следующей 
команды. Чтобы компьютер мог выполнить программу, она должна быть ему 
предварительно сообщена (\textbf{загружена в~память} компьютера).
  
  Это описание принципа явно вводит 12~понятий: 
6~{\bfseries\textit{объектов}}, 1~{\bfseries\textit{субъект}} 
и~5~{\bfseries\textit{взаимодействий}} между объектами и~субъектом.
   
\textit{Объекты}: 

  \textbf{программа}; \textbf{компьютер}; \textbf{память компьютера}; 
\textbf{робот};  \textbf{правила составления программ} (\textbf{язык 
программирования}); \textbf{команда}.

\textit{Субъект}: 

  \textbf{программист}.

\textit{Взаимодействия}: 
\begin{itemize}
  \item программист \textbf{составляет} программу;
  \item компьютер \textbf{выполняет} программу, \textbf{давая} роботу 
команды;
\item получив команду, робот ее \textbf{исполняет} и~ждет поступления 
следующей   команды;
  \item компьютер \textbf{загружает в~свою память} сообщенную ему 
программу.
  \end{itemize}
  
  Разумеется, выбранный выше набор понятий и~акценты, сделанные 
в~объяснении принципа, могли бы быть другими. Авторы выбрали указанный 
набор из~12~понятий, исходя из сугубо практических соображений. Дело 
в~том, что на первых занятиях с~дошкольниками педагог должен одновременно 
решить две задачи:
  \begin{enumerate}[(1)]
  \item добиться интуитивного понимания детьми <<правил игры>>, 
интуитивного осознания детьми предложенной системы понятий;\\[-15pt]
  \item пополнить словарный запас детей, научить их использовать в~речевой 
практике термины, выражающие освоенные понятия.
  \end{enumerate}
  
    Предлагаемый выбор понятий позволяет педагогу вчерне решить обе 
задачи примерно за 10--15~получасовых групповых занятий и~добиться 
твердого усвоения системы понятий к~концу годового курса. Важно, что 
данные понятия образуют некоторую замкнутую систему. Скорее всего, для 
ребенка 6~лет, пришедшего на занятия алгоритмикой, данная система понятий 
окажется первой в~его жизни изученной взаимосвязанной \textbf{системой 
научных понятий}. И~это изучение должно быть организовано так, чтобы 
система понятий была понята до конца каждым ребенком.   
  
  Согласно Л.\,С.~Выготскому~\cite{7-bet}, осознание любого общего 
принципа требует комплексного освоения ребенком некоторой 
{\bfseries\textit{системы научных понятий}}: <<Научные понятия являются 
воротами, через которые осознанность входит в~царство детских понятий>>. 
  
  \textbf{Принцип программного управления может быть понят, осознан 
ребенком только после усвоения изложенной выше достаточно сложной 
системы из~12 научных понятий.}
   
  Разумеется, доведение перечисленных понятий до ребенка возраста 6--7~лет 
в~вербальной форме, путем устных объяснений взрослого, невозможно. 
Осознанное усвоение этих понятий станет возможным, только если ребенку 
будут предложены виды деятельности, позволяющие ему в~игровой форме 
<<вжиться>> \textbf{во все} эти 12~понятий, <<пропустить их через себя>>. 


  
  Подчеркнем, что принцип программного управ\-ле\-ния можно вводить 
 по-раз\-но\-му, варьируя набор понятий, которые преподносятся как основные, и~понятий, которые вводятся неявно. Предложенный выше набор 
из~12~основных понятий был подобран так, чтобы максимально облегчить 
ребенку освоение \textbf{каждого} из этих понятий и~\textbf{всей системы} 
понятий в~дея\-тель\-ност\-но-иг\-ро\-вой форме. 

  
  
  Авторам удалось построить первые занятия курса так, 
  чтобы каждый ребенок в~группе смог 
  про-\linebreak\vspace*{-12pt}
  
  \pagebreak
  
  \end{multicols}
  
  \begin{figure*} %fig1
\vspace*{1pt}
 \begin{center}
 \mbox{%
 \epsfxsize=163mm 
 \epsfbox{bet-1.eps}
 }
 \end{center}
   \vspace*{-9pt}
  \Caption{Виртуальные роботы~(\textit{а}) и~реальный робот-игрушка~(\textit{б})}
  %\end{figure*}
  %\begin{figure*} %fig2
\vspace*{18pt}
 \begin{center}
 \mbox{%
 \epsfxsize=149.668mm 
 \epsfbox{bet-3.eps}
 }
 \end{center}
   \vspace*{-9pt}
\Caption{Программа управления роботом, выложенная дошкольником 
из кубиков~(\textit{а}),
и~та же программа, составленная семиклассником~(\textit{б})}
\end{figure*}

\begin{multicols}{2}
  
  \noindent 
  ими\-ти\-ро\-вать выполнение \textbf{всех} пяти перечисленных
выше действий-взаимодействий, выполняя поочередно роль робота, 
компьютера и~программиста, и~смог поработать с~материальным воплощением 
программы, составляя ее и~загружая в~память компьютера и~сталкиваясь 
с~трудностями в~случае нарушения правил составления программы. Это 
удалось сделать за счет использования на первых занятиях следующих 
методических и~технических решений.
  
  Как было предложено Пейпертом еще полвека назад~\cite{8-bet}, дети 
работают не только с~виртуальными (экранными) роботами, но и~с~реальными  
ро\-бо\-та\-ми-иг\-руш\-ка\-ми, которые перемещаются по полу игровой 
комнаты, имитируя перемещения виртуальных роботов на экране планшета 
(рис.~1). 
  
  
   
Реальные роботы управляются звуковыми командами. Эти команды 
<<наблюдаемы>> (слышимы) детьми. 

  Программы составляются из материальных объектов, кубиков, 
с~нанесенными на их грани пикто\-грам\-ма\-ми команд, повторителей и~других 
конст\-рук\-ций языка программирования (рис.~2). \mbox{Функции} компьютера 
выполняет планшет.
  
    




  Загрузка программы в~память компьютера (планшета) состоит в~явно 
проводимом ребенком фотографировании камерой планшета выложенной на 
столе программы, т.\,е.\ некоторой конфигурации кубиков. За этим явным 
действием невидимо для ребенка, скрытно, следует <<понимание>> программы 
компьютером~--- распознавание фотографии с~кубиками процессором 
планшета с~помощью нейронных сетей. Результат этого понимания ребенок 
видит на экране планшета, а как это понимание происходит~--- с~ребенком не 
обсуждается.
  
  
  
  
  Первое синтаксическое правило составления программы из кубиков требует, 
чтобы кубики были выложены в~достаточно ровные ряды, ряды располагались 
друг под другом. Первое семантическое правило выполнения программы 
гласит: при выполнении программы пиктограммы в~рядах читаются слева 
направо, а~ряды читаются сверху вниз. 
  %
  При этом ребенок сталкивается с~тем, что компьютеру не удается <<понять>> 
программы, со\-став\-лен\-ные с~нарушением правил, т.\,е.\ компьютеру 
<<непонятны>> расположения кубиков на столе, в~которых трудно или 
невозможно мысленно разбить выложенные кубики на ряды 
(рис.~3)\footnote{Невыровненность кубиков может рассматриваться как непрерывный аналог 
<<синтаксической неправильности>> программы.}.
  
  Еще две группы правил описывают две конст\-рукции языка 
программирования: числовой повторитель и~подпрограмма с~однобуквенным 
именем.\linebreak Эти правила описывают и~синтаксис, и~семантику и~применяются 
в~ситуациях, когда ребенок, имитируя компьютер, пытается <<понять>> 
выложенную другим ребенком программу и~далее пытается <<понятую>> 
программу выполнить. Эти правила требуют, чтобы пиктограммы 
располагались в~рядах в~определенном порядке. Дети осваивают эти правила 
без затруднений.
  
  Рассматриваемый курс построен так, что на первых занятиях ребенок играет 
со сверстниками и~учебными пособиями (роботом и~набором кубиков 
с~пиктограммами команд) в~сюжетно-ролевые игры. \textbf{Компьютер на 
первых порах используется только по его главному назначению, для 
исполнения загруженных в~его память программ,} и~не используется для 
других целей: ни для составления программ, ни для генерации изображений 
виртуальных роботов и~виртуальных обстановок на экране. На первых занятиях 
курса \textbf{программа, робот и~обстановка, в~которой робот действует, 
являются реальными, а не виртуальными объектами,} и~все 
взаимодействия представляют собой реальные процессы с~участием 
материальных объектов. И~ребенок может осваивать роли объектов в~игре. 
Ребенок может выступать в~роли робота, исполняя звуковые команды, 
поступающие от компьютера или от другого ребенка, выступающего в~роли 
компьютера; ребенок может выступать в~роли компьютера, выполняя 
составленную другим ребенком программу и~командуя при этом третьим 
ребенком, играющим роль робота; ребенок может поработать программистом, 
составляя самостоятельно программу путем перемещения материальных 
кубиков на столе и~переходя позднее\linebreak\vspace*{-12pt}

{ \begin{center}  %fig3
 \vspace*{-1pt}
    \mbox{%
 \epsfxsize=79mm 
 \epsfbox{bet-5.eps}
 }

\end{center}

\noindent
{{\figurename~3}\ \ \small{
Иллюстрация художника Михаила Гладковского к~докладу А.\,П.~Ершова 
<<Программирование~--- вторая грамотность>>
}}}

\vspace*{9pt}




\noindent
 к~составлению программ путем 
псевдоматериального перемещения рукой пиктограмм на сенсорном экране 
планшета.
  
  Важно, чтобы по мере того, как основные по\-нятия программирования 
осваиваются детьми на\linebreak интуитивном уровне при работе с~реальными роботами 
на ковриках, кубиками на столе и~виртуальными роботами, ковриками 
и~программами на экранах планшетов, на бескомпьютерных половинах занятий 
происходил перевод этих интуитивных представлений в~вербальную форму. 
Под руководством воспитателя дети должны обсуждать значения слов 
\textit{программист}, \textit{робот}, \textit{программа}, значения фраз типа 
<<я~выполнил программу, которую составил Коля>>, <<программа составлена 
из 6~пиктограмм>>, <<робот выполнил 10~команд>>. Это пополнение 
словарного запаса детей и~развитие навыков монолога и~диалога 
с~использованием накопленного <<профессионального>> словарного запаса 
является столь же важной целью курса, как и~обретение навыков 
самостоятельного составления простейших программ в~учеб\-но-иг\-ро\-вой 
системе программирования.
  
\section{Принцип программного управления с~обратной связью}

  Б$\acute{\mbox{о}}$льшую часть курса <<Алгоритмика для дошколят>> 
занимает составление программ управления без обратной связи. Каждая такая 
программа решает одну задачу: обеспечивает нужное поведение робота  
в~од\-ной-един\-ст\-вен\-ной внешней обстановке, предъявляемой ребенку на 
полу в~игровой комнате или в~графической форме на экране планшета. 
Программы без обратной связи составляются с~использованием всего 
\textbf{двух явных} управляющих конструкций: числовой повторитель 
и~подпрограмма с~однобуквенным именем и~\textbf{одной неявной} 
конструкцией~--- последовательного выполнения линейного участка 
программы.
  
  В <<поминутной>> методичке годового курса <<Алгоритмика для 
дошколят>> описаны 30~занятий, еще 4~занятия предусмотрены как 
резервные. Последовательное выполнение линейного участка программы 
появляется на первом же занятии. Конструкция \textit{повторитель} впервые 
появляется на занятии №\,10 и~вводится как способ сокращения размера 
программы. Конструкция \textit{подпрограмма} впервые появляется на занятии 
№\,15 и~вводится как способ <<шифровки>> фрагментов программы. Позднее 
эта конструкция рассматривается еще и~как способ сокращения размера 
программы. Разумеется, выразительная сила двух выбранных управляющих 
конструкций невелика. Все программы, которые можно составить 
с~использованием этих конструкций, являются линейными. Однако эти 
линейные программы могут иметь достаточно сложную структуру управления. 
  
  Практика показала, что набора содержательных задач, решаемых 
с~использованием этих двух управ\-ля\-ющих конструкций, достаточно для 
удержания внимания детей в~течение года (первые 25~занятий из~30) при 
условии создания достаточного разнообразия роботов и~их графических 
представлений. В~настоящее время в~курсе <<Алгоритмика для дошколят>> 
используются 5~виртуальных роботов и~1~реальный робот (Ползун), 
имитирующий одного из виртуальных. Несмотря на весьма малую 
продолжительность компьютерной части каждого занятия курса~--- 
от~15~мин в~первом полугодии до~20~мин во втором~--- удается добиться 
того, что каждый ребенок на каждом занятии выполняет 4--5~заданий, т.\,е.\ 
\textbf{в~годовом курсе каждый дошкольник самостоятельно составляет 
120--150~простейших программ}. Самостоятельное выполнение более сотни 
упражнений представляется необходимым условием устойчивого освоения 
теоретического и~практического материала курса. 
  
  В конце первого года обучения, на последних занятиях, начинается (но не 
завершается) переход от управления без обратной связи к~управлению с~ее 
использованием:
  \begin{itemize}
  \item в~системы команд роботов вводятся ко\-ман\-ды-во\-про\-сы, 
и~в~предоставляемые системой <<ПиктоМир>> конструкции языка 
программирования добавляются ветвление и~повторение;
  \item в~систему основных понятий вводятся новый вид команды~--- 
\textbf{ко\-ман\-да-во\-прос} и~новый вид взаимодействия: робот 
\textbf{отвечает} на ко\-ман\-ду-во\-прос компьютера \textbf{да} или 
\textbf{нет}.
  \end{itemize}
  
  Параллельно с~введением понятия <<обратная связь>> в~систему основных 
понятий вводится и~понятие {\bfseries\textit{число}} (неотрицательное целое 
число). Для <<материализации>> понятия <<число>> вводится виртуальный 
исполнитель <<Волшебный кувшин с~камнями>>, играющий роль счетчика. 
Наличие счетчика позволяет с~помощью подсчета числа шагов решать задачи 
управления роботом типа <<дойти до ближайшей стены и~вернуться 
в~исходную точку>>. 
{\looseness=1

}
  
  Разумеется, введение новых понятий со\-про\-вож\-да\-ет\-ся играми. Исполняя роль 
робота, дети отвечают на ко\-ман\-ды-во\-про\-сы <<да>> или <<нет>>; 
исполняя роль кув\-ши\-на-счет\-чи\-ка, ребенок по команде до\-бав\-ля\-ет или 
удаляет камешек из реального кувшина и~отвечает на ко\-ман\-ды-во\-про\-сы 
<<кувшин пуст?>>, <<кувшин не пуст?>> и~<<сколько камней в~кувшине?>>. 
На введение обратной связи в~курсе <<Алгоритмика для дошколят>> требуется 
не менее~5~занятий. На устойчивое освоение этих понятий на следующем году 
обучения необходимо еще~15~занятий. 
{\looseness=1

}
  
   После введения обратной связи становится возможным давать задачи на 
составление <<универсальных>> программ, работающих не в~одной, 
а~в~нескольких разных обстановках. Эти задачи также даются в~графической 
форме, без словесного описания класса обстановок, в~которых должна работать 
программа. Просто на очередном уровне игры требуется составить программу, 
которая работает не в~одной, как раньше, а~в~двух или трех заданных 
обстановках. 

\vspace*{-10pt}
  
\section{Выводы}

\vspace*{-2pt}

  Опыт показал, что дети возраста 6--7~лет без труда и~с~энтузиазмом 
осваивают азы программирования с~использованием описанного выше подхода 
и~готовы продолжать занятия программированием в~школе. Авторы считают  
раннее освоение основ программирования в~описанном выше объеме
необходимым.

%\vspace*{-12pt}

{\small\frenchspacing
 {%\baselineskip=10.8pt
 \addcontentsline{toc}{section}{References}
 \begin{thebibliography}{9}
 
 %\vspace*{-4pt}
 
\bibitem{1-bet}
Глава профильного комитета Думы считает нужным ввести информатику в~дошкольную 
программу:  Мат-лы XVIII съезда <<Единой России>>~// ТАСС, 8~декабря 2018. {\sf 
https://tass.ru/obschestvo/5888487}.
\bibitem{2-bet}
\Au{Rogozhkina I., Kushnirenko~A.} PiktoMir: Teaching programming concepts to preschoolers with 
a~new tutorial environment~// Procd.  Soc. Behv., 2011. Vol.~28.  
P.~601--605. doi: 10.1016/j.sbspro.2011.11.114.
\bibitem{3-bet}
\Au{Бесшапошников Н.\,О., Кушниренко~А.\,Г., Леонов~А.\,Г., Малый~А.\,А.} Проект 
двуязыковой пик\-то\-грам\-мно-текс\-то\-вой учебной среды программирования ПиктоМир-К~// 
Свободное программное обеспечение в~высшей школе: Сб. тезисов XIV конф.~--- 
М.: МАКС Пресс, 2019. С.~64--66.
\bibitem{4-bet}
ПиктоМир: Стартовая страница проекта на сайте Федерального научного центра  
<<На\-уч\-но-ис\-сле\-до\-ва\-тель\-ский институт системных исследований>> Российской 
академии наук. {\sf https://www.niisi.ru/piktomir}.
\bibitem{5-bet}
\Au{Калаш И.} Возможности информационных и~коммуникационных технологий 
в~дошкольном образовании:\linebreak\vspace*{-12pt}

\columnbreak

\noindent
 Аналитический обзор~/ Пер. с~англ. под ред. 
А.\,Л.~Семенова.~--- Институт ЮНЕСКО по информационным технологиям 
в~дошкольном образовании, 2010. 176~с. {\sf 
https://iite.unesco.org/pics/publications/ru/files/\linebreak 3214673.pdf}.
(\Au{\mbox{Kaba{\!\!\ptb{\v{s}}}}, I.} 
Recognizing the potential of ICT in early childhood education: Analytical survey.~---
UNESCO Institute for Information Technologies in Education, 2010. 148~p. 
Available at: {\sf 
https://unesdoc.\linebreak unesco.org/ark:/48223/pf0000190433.pdf} 
(accessed September~2, 2020).)
\bibitem{6-bet}
\Au{Семёнов А.\,Л.} Концептуальные проблемы информатики, алгоритмики и~программирования 
в~школе~// Вестник кибернетики, 2016. №\,2(22). С.~12--16.
\bibitem{7-bet}
\Au{Выготский Л.\,С.} Мышление и~речь.~--- Изд. 5-е, испр.~--- М.: Лабиринт, 1999. Гл.~6. 
\bibitem{8-bet}
\Au{Пейперт С.} Переворот в~сознании: Дети, компьютеры и~плодотворные идеи~/ Пер. 
с~англ.~--- М.: Педагогика, 1989. 224~с.
(\Au{Papert~S.} Mindstorms: Children, computers and powerful ideas.~---  New York, NY, USA: Basic Books, 
1980. 252~p.)
\end{thebibliography}

 }
 }

\end{multicols}

\vspace*{-3pt}

\hfill{\small\textit{Поступила в~редакцию 20.08.19 (последняя правка 21.07.20)}}

\vspace*{10pt}

%\pagebreak

%\newpage

%\vspace*{-28pt}

\hrule

\vspace*{2pt}

\hrule

\vspace*{2pt}

\def\tit{BASIC CONCEPTS OF~PROGRAMMING EXPOUNDED FOR~PRESCHOOLERS}


\def\titkol{Basic concepts of~programming expounded for~preschoolers}

\def\aut{V.\,B.~Betelin$^1$, A.\,G.~Kushnirenko$^1$, and~A.\,G.~Leonov$^{1,2,3}$}

\def\autkol{V.\,B.~Betelin, A.\,G.~Kushnirenko, and~A.\,G.~Leonov}

\titel{\tit}{\aut}{\autkol}{\titkol}

\vspace*{-6pt}


\noindent
$^1$Federal Research Center ``Scientific Research Institute for System Analysis of the Russian Academy of 
Sciences,''\linebreak
$\hphantom{^1}$36-1~Nakhimovsky Prosp., Moscow 117218, Russian Federation

\noindent
$^2$M.\,V.~Lomonosov Moscow State University, 1~Leninskie Gory, GSP-1, Moscow 119991, Russian 
Federation

\noindent
$^3$Moscow Pedagogical State University, 1-1~Malaya Pirogovskaya Str., Moscow 119991, Russian 
Federation

\def\leftfootline{\small{\textbf{\thepage}
\hfill INFORMATIKA I EE PRIMENENIYA~--- INFORMATICS AND
APPLICATIONS\ \ \ 2020\ \ \ volume~14\ \ \ issue\ 3}
}%
 \def\rightfootline{\small{INFORMATIKA I EE PRIMENENIYA~---
INFORMATICS AND APPLICATIONS\ \ \ 2020\ \ \ volume~14\ \ \ issue\ 3
\hfill \textbf{\thepage}}}

\vspace*{6pt} 

\Abste{The development of information technology has formed 
a~socioeconomic demand for reducing the age of acquaintance 
of children with programming. As a~result of 6~years of efforts, 
the authors managed to develop and massively introduce an annual 
programming course for preschoolers built on the metaphor of program 
control. In the process of developing the course, the authors were 
able to select and formulate a~set of basic programming concepts 
which fully reveals this metaphor and, at the same time, can be 
mastered by preschool children age 6+ in an active-play form. 
This set of concepts is introduced using examples of control 
programs for moving and stationary objects with an intuitive, 
visible command system. At the beginning of the course, control 
without feedback is introduced, the concept of feedback is 
introduced and used only at the end of the course. As a~basic 
pedagogical software product, the PictoMir text-free pictographic 
system developed by the Federal Research Center 
``Scientific Research Institute for System Analysis
 of the Russian Academy of Sciences'' 
and its programmatic and methodological content is used, allowing 
each preschooler to gain experience in writing and debugging at 
least 120--150~simplest programs by the end of the course.}

\KWE{informatics; robot; program; computer; programming language; preschooler; 
PiktoMir; pictogram}

\DOI{10.14357/19922264200308} 

%\vspace*{-20pt}

\Ack
\noindent
The work was completed on the subject of the Government order 0065-2019-0010 
in 2020 in the Department 
of Educational Informatics, SRISA RAS.

%\vspace*{6pt}

 \begin{multicols}{2}

\renewcommand{\bibname}{\protect\rmfamily References}
%\renewcommand{\bibname}{\large\protect\rm References}

{\small\frenchspacing
 {%\baselineskip=10.8pt
 \addcontentsline{toc}{section}{References}
 \begin{thebibliography}{9}
\bibitem{1-bet-1}
Materialy TASS XVIII s''ezda ``Edinoy Rossii'' [TASS Materials of 
the 18th All-Russian political party 
``UNITED RUSSIA'' Congress]. Available at: {\sf https://tass.ru/
obschestvo/5888487} (accessed July~24, 
2020).
\bibitem{2-bet-1}
\Aue{Rogozhkina, I., and A.~Kushnirenko.} 2011. PiktoMir: Teaching programming concepts to 
preschoolers with a~new tutorial environment. 
\textit{Procd. Soc. Behv.}  28:601--605.  doi: 10.1016/j.sbspro.2011.11.114.
\bibitem{3-bet-1}
\Aue{Besshaposhnikov, N.\,O., A.\,G.~Kushnirenko,
 A.\,G.~Leonov, and A.\,A.~Malyy.} 2019. Proekt 
dvuyazykovoy piktogrammno-tekstovoy uchebnoy sredy programmirovaniya 
PiktoMir-K [The project of the 
bilingual pictogram-text educational environment for programming PictoMir-K]. 
\textit{Sbornik tezisov 
XIV konf. ``Svobodnoe 
programmnoe obespechenie v~vysshey shkole''} [14th Conference 
``Free Software in Higher Education'' Proceedings]. Moscow: MAKS Press. 64--66.
\bibitem{4-bet-1}
PiktoMir: Startovaya stranitsa proekta na sayte Fe\-de\-ral'\-no\-go 
nauchnogo tsentra  
``Nauchno-issledovatel'skiy\linebreak\vspace*{-12pt}

\columnbreak

\noindent
 institut sistemnykh issledovaniy'' Rossiyskoy akademii nauk [The start page of 
the PictoMir project on the website of the SRISA/NIISI RAS]. Available at: {\sf 
https://www.niisi.ru/piktomir/} (accessed July~24, 2020).
\bibitem{5-bet-1}
\Aue{\mbox{Kaba{\!\ptb{\v{s}}}}, I.} 2010.
Recognizing the potential of ICT in early childhood education: Analytical survey.
\mbox{UNESCO} Institute for Information Technologies in Education. 148~p. 
Available at: {\sf 
https://unesdoc.unesco.org/ark:/ 48223/pf0000190433.pdf} 
(accessed September~2, 2020).
\bibitem{6-bet-1}
\Aue{Semyonov, A.\,L.} 2016. Kontseptual'nye problemy informatiki, algoritmiki i~programmirovaniya 
v~shkole [Conceptual problems of teaching
 computer science, algorithm studies, and programming at school]. 
\textit{Vestnik kibernetiki} [Proceedings in Cybernetics] 2(22):11--15.
\bibitem{7-bet-1}
\Aue{Vygotskiy, L.\,S.} 1999. \textit{Myshlenie i~rech'}  
[Thinking and saying]. Moscow: Labirint. Ch.~6. 
\bibitem{8-bet-1}
\Aue{Papert, S.} 1980. \textit{Mindstorms: Children, computers and powerful 
ideas.} New York, NY: Basic 
Books. 252~p.
\end{thebibliography}

 }
 }

\end{multicols}

\vspace*{-6pt}

\hfill{\small\textit{Received August 20, 2019 (last revision July~21, 2020)}}

%\hfill{\small\textit{(last revision July~21, 2020)}}

%\pagebreak

%\vspace*{-24pt}

\Contr

\noindent
\textbf{Betelin Vladimir B.} (b.\ 1946)~--- 
Doctor of Science in physics and mathematics, professor, Academician of 
RAS, research advisor, Federal Research Center 
``Scientific Research Institute for System Analysis of the 
Russian Academy of Sciences,'' 36-1~Nakhimovsky Prosp., Moscow 117218, Russian Federation; 
\mbox{betelin@niisi.msk.ru}

\vspace*{3pt}

\noindent
\textbf{Kushnirenko Anatoliy G.} (b. 1944)~--- Candidate of Science (PhD) in 
physics and mathematics, Head of 
Department, Federal Research Center 
``Scientific Research Institute for System Analysis of the Russian 
Academy of Sciences,'' 36~b1~Nakhimovsky Prosp., Moscow 117218, Russian Federation; agk\_@mail.ru.

\vspace*{3pt}

\noindent
\textbf{Leonov Aleksandr G.} (b. 1961)~--- Candidate of Science (PhD) in physics and mathematics, associate 
professor, leading scientist, Department of Mechanics and Mathematics, 
M.\,V.~Lomonosov Moscow State 
University, 1~Leninskie Gory, GSP-1, Moscow 119991, Russian Federation; head of laboratory, Federal 
Research Center ``Scientific Research Institute for System Analysis of the Russian Academy of Sciences,''
 36-1~Nakhimovsky Prosp., Moscow 117218, Russian Federation; 
 professor, senior scientist, Moscow 
Pedagogical State University, 1-1~Malaya Pirogovskaya Str., Moscow 119991, Russian Federation; 
\mbox{dr.l@vip.niisi.ru}


\label{end\stat}

\renewcommand{\bibname}{\protect\rm Литература} 
               %14
\def\stat{goncharov}

\def\tit{ВЫРАВНИВАНИЕ ДЕКАРТОВЫХ ПРОИЗВЕДЕНИЙ УПОРЯДОЧЕННЫХ МНОЖЕСТВ$^*$}

\def\titkol{Выравнивание декартовых произведений упорядоченных множеств}

\def\aut{А.\,В.~Гончаров$^1$, В.\,В.~Стрижов$^2$}

\def\autkol{А.\,В.~Гончаров, В.\,В.~Стрижов}

\titel{\tit}{\aut}{\autkol}{\titkol}

\index{Гончаров А.\,В.}
\index{Стрижов В.\,В.}
\index{Goncharov A.\,V.}
\index{Strijov V.\,V.}


{\renewcommand{\thefootnote}{\fnsymbol{footnote}} \footnotetext[1]
{Работа выполнена при частичной финансовой поддержке РФФИ 
(проекты 19-07-1155 и~19-07-00885). Настоящая статья содержит 
результаты проекта <<Статистические методы машинного обучения>>, 
выполняемого в~рамках реализации Программы Центра компетенций 
Национальной технологической инициативы <<Центр хранения 
и~анализа больших данных>>, поддерживаемого Министерством науки 
и~высшего образования Российской Федерации по договору МГУ им.\ 
М.\,В.~Ломоносова  с~Фондом поддержки проектов Национальной 
технологической инициативы от 11.12.2018 №\,13/1251/2018.}}


\renewcommand{\thefootnote}{\arabic{footnote}}
\footnotetext[1]{Московский физико-технический институт, alex.goncharov@phystech.edu}
\footnotetext[2]{Вычислительный центр им.\ А.\,А.~Дородницына Федерального исследовательского 
центра <<Информатика и~управ\-ле\-ние>> Российской академии наук; 
Московский фи\-зи\-ко-тех\-ни\-че\-ский институт, \mbox{strijov@ccas.ru}}

%\vspace*{-12pt}



\Abst{Работа посвящена исследованию метрических методов анализа 
объектов сложной структуры. Предлагается обобщить метод динамического 
выравнивания двух временных рядов на случай объектов, определенных на 
двух и~более осях времени. В~дискретном представлении такие объекты 
являются матрицами. Метод динамического выравнивания временных рядов 
обобщается как метод динамического выравнивания матриц. Предложена 
функция расстояния, устойчивая к~монотонным нелинейным деформациям 
декартова произведения двух и~более временных шкал. Определен выравнивающий 
путь между объектами. В~дальнейшем объектом называется матрица, 
в~которой строки и~столбцы соответствуют осям времени. Исследованы 
свойства предложенной функции расстояния. Для иллюстрации метода 
решаются задачи метрической классификации объектов на модельных 
данных и~данных из набора MNIST.}

\KW{функция расстояния; динамическое выравнивание; расстояние между матрицами; 
нелинейные деформации времени; про\-стран\-ст\-вен\-но-вре\-мен\-ные ряды}

\DOI{10.14357/19922264200105} 
  
\vspace*{-3pt}


\vskip 10pt plus 9pt minus 6pt

\thispagestyle{headings}

\begin{multicols}{2}

\label{st\stat}


\section{Введение}

Временн$\acute{\mbox{ы}}$е ряды представляют собой набор измерений, упорядоченных 
по оси времени. Анализ временн$\acute{\mbox{ы}}$х рядов производится при решении задач, 
связанных с~классификацией активности человека по измерениям акселерометра 
телефона, поиском паттернов в~EEG-сиг\-на\-лах (электроэнцефалограмма), 
кластеризации набора ECoG (электрокортикограмма) данных и~во многих других 
задачах~\cite{0}. Рассматриваются объекты, для которых время между измерениями 
фиксированно. В~данной работе для построения адекватной функции 
расстояния между объектами требуется учесть нелинейные деформации 
относительно оси времени: глобальные и~локальные сдвиги, растяжения 
и~сжатия~\cite{1}.

В~\cite{2} приводятся различные методы решения задач анализа 
временн$\acute{\mbox{ы}}$х рядов: классификации, детектирования паттернов, 
кластеризации и~др. В~\cite{3} описание временных рядов 
строится с~по\-мощью анализа параметров моделей, в~\cite{4} 
используется их признаковое описание, в~\cite{5} анализируется их форма. 
Комбинации этих подходов описаны в~\cite{2}.

Метрические методы находят схожие объекты в~наборе. Используются 
функции расстояния над временн$\acute{\mbox{ы}}$ми рядами: расстояние Хаусдорфа~\cite{10}, 
MODH~\cite{11}, расстояние, основанное на HMM
(hiden Markov model)~\cite{6}, евклидово расстояние 
в~исходном пространстве или в~пространстве сниженной размерности~\cite{5}, 
\mbox{LCSS} (longest common\linebreak subsequence)~\cite{7}. Показано~\cite{8}, что в~случае локальных или глобальных 
деформаций времени при решении задач, требующих анализа исходной формы 
временн$\acute{\mbox{о}}$го ряда, метод динамического выравнивания оси времени 
DTW (Dynamic Time Warping) 
превосходит другие функции расстояния~\cite{9} по качеству итогового 
решения задачи, так как при наличии смещений двух объектов относительно 
друг друга требуется выравнивать их оптимальным образом для вычисления 
расстояния между ними.

В данной работе предлагается перейти от рас\-смот\-ре\-ния объекта~$\textbf{s}(t)$, 
временн$\acute{\mbox{о}}$го ряда, к~более общему случаю $\textbf{s}(\textbf{t})$, 
в~котором компоненты вектора~$\textbf{t}$~--- оси времени. Из-за 
существенного рос\-та вы\-чис\-ли\-тель\-ной слож\-ности при увеличении чис\-ла 
осей времени предлагается рас\-смот\-реть объекты $\textbf{s}(t_1, t_2)$, 
определенные на двух осях времени. Оси времени считаются независимыми. 
В~случае единственной дискретной и~ограниченной сверху шкалы времени 
объект представим вектором фиксированной размерности. 
Аналогично объект настоящего исследования представим мат\-ри\-цей.

Вводятся ограничения на зависимости осей времени в~декартовом 
произведении для таких объектов. Определена гипотеза порождения данных: 
объекты одного класса эквивалентности получены при помощи допустимых 
преобразований, а~именно: локальных деформаций (растяжений и~сжатий) 
каждой из осей времени по отдельности. В~дискретном случае преобразование 
представимо дуп\-ли\-ци\-ро\-ва\-ни\-ем строк и~столбцов матриц. 
В~число допустимых преобразований попадают и~глобальные деформации: 
сдвиги по осям времени, представимые добавлением и~удалением крайних 
строк и~столбцов исходных матриц. Для каждой из осей времени выполняются 
свойства времени: монотонность и~непрерывность. Похожими на описанные 
свойствами обладает, например, частотный спектр сигнала, где одна ось 
определяет время, а другая~--- частоту, величину, обратную времени.


Между двумя объектами, матрицами, в~случае допустимых преобразований 
требуется определить инвариантную к~преобразованиям осей времени функцию 
расстояния, которая сможет выделить классы эквивалентности множества 
преобразованных объектов. Работа посвящена определению такой функции 
расстояния, как обобщения метода динамического выравнивания временных рядов 
DTW для матриц.

Цель данной работы~--- построение метода, основанного на динамическом 
выравнивании осей времени для матриц. Метод динамического выравнивания 
временн$\acute{\mbox{ы}}$х рядов~\cite{33} определен только для объектов с~одной осью времени, 
что делает его неприменимым для описанного случая. Однако концепции, 
используемые на каждой стадии вы\-чис\-ле\-ния оптимального выравнивания, обобщены 
на рассматриваемый случай. Работа исследует свойства предложенного 
метода и~сравнивает результаты применения метода к~задачам классификации 
изображений~\cite{12} с~результатами функции расстояния~$L_2$.

Для иллюстрации и~анализа результатов решается задача метрической 
классификации объектов (матриц низкой размерности). Используются наборы данных: 
модельные данные, которые согласуются с~выдвинутой гипотезой порождения 
данных для временн$\acute{\mbox{ы}}$х рядов, подмножество набора MNIST сниженной 
размерности и~частотный спектр сигнала.

\vspace*{-10pt}

\section{Постановка задачи построения функции расстояния}

\vspace*{-2pt}

Рассмотрим задачу построения функции расстояния между объектами. 
Функция расстояния инвариантна к~допустимым преобразованиям осей времени: 
глобальным и~локальным линейным и~нелинейным деформациям временн$\acute{\mbox{о}}$й шкалы. 
Ниже приведены две постановки задачи, с~помощью которых определены свойства 
предложенной функции расстояния, оценено ее качество и~проведено сравнение 
нескольких функций расстояния: предложенной и~$L_2$.

Первая постановка задачи использует общее свойство функций расстояния: 
объединение схожих объектов и~разделение непохожих объектов. 
Вводится определение свойства инвариантности функции расстояния к~допустимым 
преобразованиям осей времени.
Вторая постановка задачи уточняет первую и~заключается в~проведении метрической 
классификации методом ближайшего соседа.

\textbf{Постановка задачи выбора функции расстояния между двумя объектами.}
На двух временн$\acute{\mbox{ы}}$х осях заданы объекты вида 
$\textbf{A}(t_1,t_2)\hm \in \mathbb{R}^{n \times n}$. 
Функция $G_w(\textbf{A}):\mathbb{R}^{n \times n} \hm\rightarrow 
\mathbb{R}^{\hat{n} \times \hat{n}}$ задает допустимые преобразования 
исходного объекта~$\textbf{A}$: глобальные сдвиги, локальные линейные 
и~нелинейные деформации, а~именно: растяжения и~сжатия оси времени, 
сдвиги значений по оси времени. Скалярный параметр $w \hm\in \mathbb{R}^+$
 функции~$G$ фиксирует набор этих преобразований.

Допустимым элементарным преобразованием матрицы~$\textbf{A}$ назовем 
дуплицирование случайных строк и~столбцов исходной матрицы, добавление 
или удаление крайних строк и~столбцов. Допустимым преобразованием 
примем некоторую последовательность допустимых элементарных 
преобразований матрицы~$\textbf{A}$ и~обозначим как~$G_w(\textbf{A})$.

Будем называть объект~$\textbf{B} \hm\in \mathbb{R}^{\hat{n} \times \hat{n}}$ 
полученным из объекта~$\textbf{A}$ при помощи допустимых 
преобразований~$G_{\hat{w}}$, если существует $\hat{w}\hm\in \mathbb{R}^+ : 
\textbf{B} \hm= G_{\hat{w}}(\textbf{A})$.

Функцию расстояния между двумя объектами $\rho: 
\mathbb{R}^{{n} \times {n}} \times \mathbb{R}^{\hat{n} \times \hat{n}} 
\hm\rightarrow  \mathbb{R}^+$ оценим на выборке $\mathfrak{D } \hm= 
\{ \textbf{A}_i \}_{i=1}^m$ объектов вида $\textbf{A}_i \hm\in 
\mathbb{R}^{n \times n}$.

Для каждого объекта выборки~$\textbf{A}_i$ и~объекта~$\textbf{B}_j$ его 
класса эквивалентности $\{\textbf{B}_j\}_i \hm= \{  \textbf{B} 
\hm\in \mathfrak{D} | \exists w_i,w_j: G_{w_i}(\textbf{A}_i) \hm= G_{w_j}
(\textbf{B}_j)   \}$ заданы допустимые трансформации с~параметрами~$w_i$ 
и~$w_j$, такие что $G_{w_i}(\textbf{A}_i)\hm = G_{w_j}(\textbf{B}_j)$. 
Для каждого объекта выборки~$\textbf{A}_i$ и~объекта~$\textbf{C}_j$ 
из других классов эквивалентности $\{ \textbf{C}_k\}_i \hm= 
\{  \textbf{C} \hm\in \mathfrak{D} | \nexists w_i,w_k: G_{w_i}(\textbf{A}_i)
\hm = G_{w_k}(\textbf{C})   \}$ не существует таких $ w_i, w_k : G_{w_i}
(\textbf{A}_i) \hm= G_{w_k}(\textbf{C}_k)$.

Решается задача поиска функции расстояния~$\rho$, значение
 которой на паре объектов одного класса эквивалентности меньше, 
 чем на любой паре объектов из разных: для любых $i,j,k \hm\in 
 \{1,\dots,m\}$ $\quad \rho(\textbf{A}_i,\textbf{B}_j) \hm< 
 \rho(\textbf{A},\textbf{C}_k)$. Функцию расстояния, обладающую 
 таким свойством, назовем инвариантной на классах эквивалентности.

Критерием качества для функции расстояния~$\rho$ на выборке~$\mathfrak{D}$ 
примем долю объектов, для которых указанное неравенство выполняется:
$$
S_{\rho}(\mathfrak{D}) = \fr{1}{m} \sum\limits_{i=1}^m 
\prod\limits_{\{ \textbf{B}_j\}_i} 
\prod\limits_{\{ \textbf{C}_k\}_i}  
\left[  \rho(\textbf{A}_i,\textbf{B}_j) < \rho(\textbf{A}_i,\textbf{C}_k)  
 \right].
 $$
Постановка задачи выбора функции расстояния~$\rho$ 
сводится к~задаче максимизации критерия качества.

\textbf{Прикладное использование функции расстояния.}
Задана выборка $\mathfrak{D}\hm = \{(\textbf{A}_i,y_i)\}^m_{i=1}$, 
состоящая из пар объ\-ект--от\-вет. Объектами служат объекты сложной 
структуры: $\textbf{A}_i\hm \in \mathbb{R}^{n\times n}$, 
а~ответами выступают метки класса~---~$y_i\hm \in Y \hm= \{1,\ldots,E\}$, 
где $E \hm\ll m$. Выборка разделена на обучение $\mathfrak{D}_l \hm= 
\{(\textbf{A}_i,y_i)\}^{m_1}_{i=1}$ и~контроль $\mathfrak{D}_t \hm= 
\{(\textbf{A}_i,y_i)\}_{m_1}^{m_1+m_2}$.

Модель классификации~$f$ принадлежит множеству моделей метрической 
классификации 1NN, которые классифицируемому объекту ставят 
в~соответствие метку класса ближайшего объекта из обучающей 
выборки по заданной функции расстояния~$\rho$:
$$ 
\hat{y} = f(\textbf{B} | \rho) = y \argmin\limits_{i = 1,\dots, m_1} 
\rho\left(B,A_i\right)\,.
$$
Критерий качества $S$ модели~$f$ для задачи классификации~--- 
доля правильно проставленного класса на контрольной выборке:
 $$ 
 S(f | \rho) = \fr{1}{m_2}\sum\limits_{i=m_1}^{m_1+m_2} 
 \left[f(\textbf{A}_i | \rho) = y_i\right].
 $$

Требуется выбрать функцию расстояния~$\rho$ для модели 
классификации~$f:~\mathbb{R}^{n\times n} \hm\rightarrow~Y$, 
максимизируюшую критерий качества~$S$ на контрольной выборке:
\begin{equation*}
f =  \argmax\limits_{\rho \in \{\mathrm{mDTW}, L_2\}}\left(S(f | \rho)\right).
\end{equation*}

\section{Вычисление матричного расстояния mDTW}

Предлагается использовать функцию расстояния DTW, 
модифицированную для случая выравнивания двойной шкалы времени.

\smallskip

\noindent
\textbf{Определение~1.} {Даны два объекта~$\textbf{A},\textbf{B}\hm \in 
\mathbb{R}^{n\times n}$. Тензор 
невязок~$\boldsymbol{\Omega}^{n \times n \times n \times n}$~--- 
такой тензор, что его элемент~$\boldsymbol{\Omega}(i,j,k,l)$ 
равен квадрату разности между элементами~$\textbf{A}(i,j)$ и~$\textbf{B}(k,l)$:}
\begin{equation*}
\boldsymbol{\Omega}(i,j,k,l)=(\textbf{A}(i,j) - \textbf{B}(k,l))^2.
\end{equation*}

\noindent
\textbf{Определение 2.} {Путем~$\boldsymbol{\pi}$ между двумя 
объектами $\textbf{A},\textbf{B} \hm\in \mathbb{R}^{n\times n}$ 
назовем множество индексов тензора~$\boldsymbol{\Omega}$: }
$$
\boldsymbol{\pi} = \{(i,j,k,l)\},\quad i,j,k,l \in \{1,\ldots,n\} ,
$$
\textit{удовлетворяющее следующим условиям:}

{\bfseries\textit{Частичный порядок.}}
Для элементов пути~$\boldsymbol{\pi}$ с~фиксированными значениями~$i,k$ 
задан порядок: выравнивающий путь для фиксированных строк двух 
матриц упорядочен~--- $\{(i,j_r,k,l_r))\}_{r=1}^{R} \hm\subset 
\boldsymbol{\pi}$ мощностью~$R$. Аналогично для фиксированных столбцов 
с~индексами~$j,l$.

{\bfseries\textit{Граничные условия.}}
 Пусть $(i,j,k,l) \in \boldsymbol{\pi}$, тогда $(1,j,1,l) \hm\in 
 \boldsymbol{\pi}$ и~$(i,1,k,1) \hm\in \boldsymbol{\pi}$.
Путь $\boldsymbol{\pi}$ содержит элементы тензора~$\boldsymbol{\Omega}$: 
$(1,1,1,1) \hm\in \boldsymbol{\pi}$ и~$(n,n,n,n) \hm\in \boldsymbol{\pi}$.

{\bfseries\textit{Непрерывность по направлению.}}
Для упорядоченного подмножества пути $\{(i,j_r,k,l_r)\}_{r=1}^{R}
\hm\subset\boldsymbol{\pi}$ выполняется условие непрерывности:
$$
j_{r}-j_{r-1}\leq1\,,\quad l_r-l_{r-1}\leq1\,, \quad r = 2,\ldots,R\,.
$$
На~шаге пути~$\boldsymbol{\pi}$ по фиксированному направлению времени~$i,k$ 
встречаются только соседние элементы матрицы (включая соседние по диагонали). 
Аналогично для фиксированных~$j,l$.

{\bfseries\textit{Монотонность по направлению.}}
Для упорядоченного подмножества пути  $\{(i,j_r,k,l_r)\}_{r=1}^{R}
\hm\subset\boldsymbol{\pi}$ выполняется хотя бы одно из условий 
монотонности функции выравнивания времени: 
$$
j_{r}-j_{r-1}\geq1\,,\quad l_r-l_{r-1}\geq1\,, \quad r = 2,\ldots,R\,.
$$

Свойства пути между матрицами обобщают свойства пути между двумя 
временными рядами.

\smallskip

\noindent
\textbf{Определение~3.}\ {Стоимость 
$\mathrm{Cost}\,(\textbf{A},\textbf{B},{\boldsymbol{\pi}})$ пути $\boldsymbol{\pi}$ 
между объектами $\textbf{A}, \textbf{B}$:
\begin{equation*}
\mathrm{Cost}\,(\textbf{A},\textbf{B},{\boldsymbol{\pi}}) = 
\sum\limits_{(i,j,k,l) \in \boldsymbol{\pi}}{\boldsymbol{\Omega}}(i,j,k,l).
\end{equation*}}

\noindent
\textbf{Определение~4.}\ 
{Выравнивающий путь~$\hat{\boldsymbol{\pi}}$ между 
объектами $\textbf{A},\textbf{B}$~--- путь наименьшей стоимости 
среди всех возможных путей между объектами:
\begin{equation*}
\hat{\boldsymbol{\pi}} = 
\argmin\limits_{{\boldsymbol{\pi}}} \mathrm{Cost}
\left(\textbf{A},\textbf{B},{\boldsymbol{\pi}}\right).
\end{equation*}}
Функция расстояния~$\rho (\textbf{A},\textbf{B})\hm = \mathrm{mDTW}\,
(\textbf{A},\textbf{B})$ между объектами~$\textbf{A}$ и~$\textbf{B}$ 
рассчитывается как стоимость выравнивающего пути~$\hat{\boldsymbol{\pi}}$:
\begin{equation}
\mathrm{mDTW}(\textbf{A},\textbf{B}) = \mathrm{Cost}\left(\textbf{A},
\textbf{B},\hat{\boldsymbol{\pi}}\right).
\end{equation}

\setcounter{figure}{1}
\begin{figure*}[b] %fig2
{\small 
\begin{center}
\begin{tabular}{l}
\hline
DTW(\textbf{s},\textbf{c}):\\
\hspace*{3mm}$\boldsymbol{D}$(1:n+1,1:m+1) = inf;\\
\hspace*{3mm}$\boldsymbol{D}$(1,1) = 0;\\
\hspace*{3mm}for $i = 2$: $n+1$\\
\hspace*{6mm}for $j = 2$ : $m+1$\\
\hspace*{9mm}$d = (\textbf{s}(i-1)-\textbf{c}(j-1))^2$;\\
\hspace*{9mm}$\boldsymbol{D}(i,j) = d + \min( 
[ \boldsymbol{D}(i-1,j), \boldsymbol{D}(i,j-1), \boldsymbol{D}(i-1,j-1) ])$;\\
return\ sqrt$(\boldsymbol{D}(n+1,m+1))$\\
\hline
\end{tabular}
\end{center}}
\vspace*{-9pt}

\Caption{Алгоритм вычисления DTW для временных рядов
\label{ris:dtwts}}
%\end{figure*}
%\begin{figure*} %fig3
\vspace*{6pt}
{\small 
\begin{center}
\begin{tabular}{l}
\hline
\\[-9pt]
Correction $(\overline{i,j,k,l}, \boldsymbol{\pi}(\overline{i,j,k,l})):$\\
\hspace*{3mm}if $\overline{i,j,k,l} \in \{ (i-1, j, k,l)  ;  
(i, j, k-1, l)  ;  (i-1, j, k-1, l) \}$:\\
\hspace*{6mm}$ \widehat{\pi} = \{ (\overline{i}, r, \overline{k}, f) \in 
\boldsymbol{\pi}(\overline{i, j, k, l}) \vert r, f \in \mathbb{N} \}$\\
\hspace*{3mm}elif $\overline{i,j,k,l}\in \{  
(i, j-1, k, l); (i, j, k, l-1); (i, j-1, k, l-1) \}$:\\
\hspace*{6mm}$\widehat{\pi} = \{ (r, \overline{j}, f, \overline{l}) 
\in \boldsymbol{\pi}(\overline{i, j, k, l}) \vert r, f \in \mathbb{N} \}$\\
\hspace*{3mm}elif $\overline{i,j,k,l} =  i-1,j-1,k-1,l-1:$\\
\hspace*{6mm}$\widehat{\pi} = \{ (\overline{i}, r, \overline{k}, f) 
\in \boldsymbol{\pi}(\overline{i, j, k, l}) \vert r,f \in \mathbb{N} \} \cup$\\
\hspace*{6mm}$\cup \{ (r, \overline{j}, f, \overline{l}) \in \boldsymbol{\pi}
(\overline{i, j, k, l}) \vert r,f \in \mathbb{N} \}$\\
\hspace*{3mm}$\boldsymbol{d\pi} = \{ \mathrm{element} \in \widehat{\pi}: 
\mbox{произведены\ замены\ индексов } 
\overline{i} = i,\ \overline{j} = j,\ \overline{k} = k,\ \overline{l} = l \}$\\
return $\boldsymbol{d\pi}$\\
\hline
\end{tabular}
\end{center}
}
\vspace*{-9pt}

\Caption{Алгоритм вычисления поправки $\boldsymbol{d\pi}$ 
пути $\boldsymbol{\pi}$
\label{ris:codedpi}}
\end{figure*}


\textbf{Алгоритм вычисления значения расстояния~(4).}
Построение алгоритма вычисления значения функции расстояния 
между матрицами основан на алгоритме расчета функции расстояния 
между временн$\acute{\mbox{ы}}$ми рядами. В~случае выравнивания одной\linebreak\vspace*{-12pt}

{ \begin{center}  %fig1
 \vspace*{-3pt}
    \mbox{%
 \epsfxsize=79mm 
 \epsfbox{gon-1.eps}
 }


\end{center}


\noindent
{{\figurename~1}\ \ \small{Матрица стоимости оптимального выравнивания, по обеим 
осям отложены временные отсчеты}}
}

\vspace*{12pt}


\noindent 
временн$\acute{\mbox{о}}$й шкалы
 итоговая матрица расстояний~$\boldsymbol{D}$ (рис.~1) в~каждом 
 элементе~$\boldsymbol{D}(i,j)$ содержит рас\-сто\-яние между подрядом 
 первого временн$\acute{\mbox{о}}$го ряда и~подрядом второго временн$\acute{\mbox{о}}$го ряда. 
 Рас\-смот\-рим алгоритм динамического выравнивания двух временн$\acute{\mbox{ы}}$х 
 рядов $\textbf{s} \hm\in R^n$ и~$\textbf{c} \hm\in R^m$ на рис.~2.
 
 

Элемент $\boldsymbol{D}(i,j)$ матрицы~$\boldsymbol{D}$ соответствует 
стоимости выравнивающего пути между подпоследовательностями 
исходных временн$\acute{\mbox{ы}}$х рядов: $\textbf{s}(1:i) \hm= \textbf{s}(t)$, 
$t \hm= 1,\ldots,i,$ и~$\textbf{c}(1:j) \hm= \textbf{c}(t)$, $t \hm= 1,\ldots,j$. 
Алгоритм построения наилучшего выравнивания времени 
подразумевает, что выравнивающий путь между этими 
подпоследовательностями получен одним из трех способов~--- 
если стоимость выравнивающего пути между 
подпоследовательностями~$\textbf{s}(1:\overline{i}) $ 
и~$\textbf{c}(1:\overline{j})$ минимальна для~$\overline{i,j}$ из множества
$$
\overline{i,j} \in \left\{ \{i-1,j\},\{i,j-1\},\{i-1,j-1\} \right\},$$
тогда выравнивающий путь между $\textbf{s}(1:i)$ и~$\textbf{c}(1:j)$ получен добавлением пары~$(i,j)$ к~выбранному 
выравнивающему пути с~минимальной стоимостью из трех.



Предложенный алгоритм переносит эти рас\-суж\-де\-ния на случай 
выравнивания двух матриц~$\textbf{A}$ и~$\textbf{B}$. 
Элемент~$\boldsymbol{D}(i,j,k,l)$ четырехиндексного
 тензора расстояний~$\boldsymbol{D}$ соответствует стоимости выравнивающего 
 пути между $\textbf{A}(1:i,1:j) \hm= \textbf{A}(t_1,t_2)$, 
 $t_1 \hm= 1,\ldots, i$, $t_2 \hm= 1,\ldots, j,$ 
 и~$\textbf{B}(1:k,1:l) \hm= \textbf{B}(t_1,t_2)$, $t_1 \hm= 1,\ldots, k$,
 $t_2 \hm= 1,\ldots, l$. Выравнивающий путь между этими 
 подматрицами получен одним из семи способов~--- 
 если стоимость выравнивающего пути между 
 подматрицами $\textbf{A}(1:\overline{i},1:\overline{j})$ 
 и~$\textbf{B}(1:\overline{k},1:\overline{l})$ 
 минимальна для~$\overline{i,j,k,l}$ из множества
\begin{multline*} 
\overline{i,j,k,l} \in 
\left\{ \{i-1,j,k,l\},\{i,j-1,k,l\},\right.\\
\{i,j,k-1,l\},
\{i,j,k,l-1\}, \{i-1,j,k-1,l\},\\
\left.
\{i,j-1,k,l-1\},\{i-1,j-1,k-1,l-1\}\right\},
\end{multline*}

\setcounter{figure}{3}
\begin{figure*} %fig4
{\small 
\begin{center}
\begin{tabular}{l}
\hline
$\mathrm{mDTW}\left(\textbf{A},\textbf{B}\right):$\\
\hspace*{3mm}$\textbf{D}(1:n+1,1:n+1, 1:n+1, 1:n+1) = inf$;\\
\hspace*{3mm}$\textbf{D}(1,1,1,1) = 0;$\\
\hspace*{3mm}$\boldsymbol{\pi}(1,1,1,1) = ((1,1),(1,1))$\\
\hspace*{3mm}$for\ i,j,k,l  \in \mathbb{N}^{2 : n+1} \times 
\mathbb{N}^{2 : n+1} \times \mathbb{N}^{2 : n+1} \times \mathbb{N}^{2 : n+1}:$\\
\hspace*{6mm}$\overline{i,j,k,l} = \argmin($ [ \textbf{D}(i-1, j, k, l), 
\textbf{D}(i, j-1, k, l), \textbf{D}(i, j, k-1, l), 
\textbf{D}(i, j, k, l-1),    \\
\hspace*{9mm}$\textbf{D}(i-1, j, k-1, l), \textbf{D}(i, j-1, k, l-1), 
\textbf{D}(i-1, j-1, k-1, l-1) ])$;\\
\hspace*{3mm}$\boldsymbol{d \pi} = \mathrm{Correction}\,(\overline{i,j,k,l}, 
\boldsymbol{\pi}(\overline{i,j,k,l}))$\\
\hspace*{3mm}$\boldsymbol{\pi}(i, j, k, l) = \boldsymbol{d \pi} \cup 
\{(\overline{i,j,k,l})\}$\\
\hspace*{3mm}$\mathrm{cost} = (\textbf{A}(i, j)-\textbf{B}(k, l))^2 + 
\sum\nolimits_{(r,f,t,g) \in \boldsymbol{d \pi}}
(\textbf{A}(r, f)-\textbf{B}(t, g))^2$;\\
\hspace*{3mm}$\textbf{D}(i,j,k,l) = \mathrm{cost} + \textbf{D}
(\overline{i,j,k,l})$\\
return  sqrt$(\textbf{D}(n+1,n+1,n+1,n+1))$\\
\hline
\end{tabular}
\end{center}
}
\vspace*{-9pt}

\Caption{Алгоритм вычисления расстояния между матрицами
\label{ris:matrixdtw}}
\end{figure*}

\begin{table*}[b]\small
\begin{center}
\begin{tabular}{|l|c|c|c|c|}
\multicolumn{5}{c}{Снижение расстояний при выполнении преобразований 
для различных наборов данных}\\
\multicolumn{5}{c}{\ }\\[-6pt]
\hline
 &\multicolumn{4}{c|}{Метод}\\
 \cline{2-5}
\multicolumn{1}{|c|}{Данные}  & \multicolumn{2}{c|}{$L_2$} & \multicolumn{2}{c|}{MatrixDTW} \\
\cline{2-5}
& $S(f|p)$  &  $S_{\rho}(\mathfrak{D})$ &  $S(f|p)$ & $S_{\rho}(\mathfrak{D})$ \\
\hline
Модельные данные без преобразований& 92\% & 78\% & 100\%\hphantom{9} & 85\% \\
Модельные данные с~преобразованиями & 86\% & 65\% &  100\%\hphantom{9} & 82\% \\
Модельные данные с~преобразованиями и~шумом& 69\% & 61\% &  92\% & 78\% \\
MNIST без преобразований& 95\% & --- & 95\% & --- \\
MNIST с~преобразованиями & 53\% & --- & 92\% & --- \\
Спектр сигнала& 83\% & --- & 96\% & --- \\
\hline
\end{tabular}
\end{center}
\end{table*}

\noindent
то к~выравнивающему пути между этими под\-мат\-ри\-ца\-ми 
добавляется элемент пути $(i,j,k,l)$ и~поправка~$\boldsymbol{d\pi} $ 
пути~$\boldsymbol{\pi}$, алгоритм вычисления которой приведен ниже.

Обозначим выравнивающий путь между $\textbf{A}(1:i,\linebreak 1:j)$
 и~$\textbf{B}(1:k,1:l)$ как~$\boldsymbol{\pi}(i,j,k,l)$, тогда 
 поправка~$\boldsymbol{d\pi} $ пути~$\boldsymbol{\pi}(i,j,k,l)$ 
 при фиксированных~$\overline{i,j,k,l}$ вычисляется приведенным на рис.~3 
 образом.





Алгоритм динамического выравнивания двух матриц и~вычисления 
расстояния $\mathrm{mDTW}$ между ними с~учетом приведенного выше 
алгоритма примет вид, представленный на рис.~4.





\begin{figure*} %fig5
\vspace*{1pt}
    \begin{center}  
  \mbox{%
 \epsfxsize=161.412mm 
 \epsfbox{gon-5.eps}
 }
\end{center}
\vspace*{-12.5pt}
\Caption{Выравнивание модельных данных: (\textit{а})~один класс без шума; 
(\textit{б})~разные классы без шума; 
(\textit{в})~один класс с~шумом; (\textit{г})~разные классы с~шумом
\label{ris:random}}
%\end{figure*}
%\begin{figure*} %fig6
\vspace*{1pt}
    \begin{center}  
  \mbox{%
 \epsfxsize=163mm 
 \epsfbox{gon-6.eps}
 }
\end{center}
\vspace*{-12.5pt}
\Caption{Выравнивание данных MNIST: левый столбец~--- один класс; 
правый столбец~--- разные 
классы;
(\textit{а})~$\mathrm{mDTW}\hm=720{,}1$; 
(\textit{б})~948,6;
(\textit{в})~2017,0;
(\textit{г})~$\mathrm{mDTW}\hm=2071{,}4$
\label{ris:mnist}}
\end{figure*}


Следует отметить, что алгоритм~\cite{15} имеет\linebreak высокую сложность 
вычисления~--- $O(n^4)$. Предполагается ускорение метода 
с~использованием ограниче\-ния Sakoe-Chiba band, что сократит 
вычислительную сложность алгоритма до $O(n^2k^2)$, где~$k$~--- 
параметр ограничения.


\section{Вычислительный эксперимент}

Вычислительный эксперимент проведен на модельных данных с~допустимыми 
преобразованиями и~на реальных данных: объектах коллекции MNIST с~допустимыми 
преобразованиями и~на спектрограммах зашумленных сигналов.





Решается задача метрической классификации методом ближайшего соседа. В~таблице 
приведены значения критерия качества функции расстояния 
$S_{\rho}(\mathfrak{D})$ и~критерия качества метрической классификации $S(f|p)$ 
при использовании двух функций расстояния: предложенной в~работе $\mathrm{mDTW}$ 
и~$L_2$.

Модельные данные~--- это нулевые матрицы со случайными ненулевыми 
строками, столбцами, подпрямоугольниками с~наложенным шумом. 
К~ним применены допустимые преобразования, согласованные с~гипотезой 
наличия локальных и~глобальных искажений. На рис.~\ref{ris:random} 
показан пример оптимального выравнивания двух объектов. 
Линиями показаны элементы пути~$\boldsymbol{\pi}$.

Подготовлена подвыборка набора данных MNIST. Она 
состоит из~100 объектов классов 0 и~1 сниженной размерности
 с~допустимыми преобразованиями. На рис.~\ref{ris:mnist} 
 показан пример оптимального выравнивания объектов.


Аналогичный эксперимент проведен для решения задачи метрической 
классификации спектров различных сигналов, пример которых приведен на 
рис.~\ref{ris:spectr}. На рисунке показаны примеры Фурье-спект\-ров 
этих сигналов. Спектр получен путем применения быстрого преобразования 
Фурье к~исходному сигналу для различных окон с~фиксированным размером и~сдвигом. 
Исходные временн$\acute{\mbox{ы}}$е ряды обладали свойством периодичности, период выбирался 
случайным образом.



Тестирование проведено на разного рода данных: исходных 
модельных данных без наложения\linebreak\vspace*{-12pt}

\pagebreak

\end{multicols}

\begin{figure*} %fig7
\vspace*{1pt}
    \begin{center}  
  \mbox{%
 \epsfxsize=149.062mm 
 \epsfbox{gon-7.eps}
 }
\end{center}
\vspace*{-8pt}
\Caption{Данные спектров сигнала: (\textit{а})~класс~1; (\textit{б})~спектр 
класса~1; (\textit{в})~класс~2; (\textit{г})~спектр класса~2; 
(\textit{д})~класс~3; (\textit{е})~спектр класса~3
\label{ris:spectr}}
\vspace*{9pt}
\end{figure*}

\begin{multicols}{2}

\noindent допустимых преобразований, с~ними, а~также 
на модельных данных с~наложенным поверх объектов случайным шумом.



В каждом из проведенных экспериментов была продемонстрирована 
устойчивость предложенного подхода к~допустимым преобразованиям. 
Наилучшее значение критерия качества задачи классификации было 
достигнуто при использовании предложенной функции расстояния.

\vspace*{-5pt}

\section{Заключение}

В работе предложено обобщение метода динамического выравнивания
 временн$\acute{\mbox{ы}}$х рядов для случая объектов, определенных на двух осях времени. 
 Существует теоретическое обобщение предлагаемых методов на случай 
 конечного множества осей времени. Вычислительный эксперимент позволил 
 проанализировать свойства подхода: устойчивость к~допустимым 
 преобразованиям и~разделяющая способность функции расстояния как 
 на реальных, так и~на модельных данных. Качество решения задачи 
 метрической классификации выше решения, основанного на евклидовом 
 расстоянии. Вычислительная сложность метода высокая, что ограничивает 
 его применимость на объектах высокой размерности.

\vspace*{-2pt}

{\small\frenchspacing
 {%\baselineskip=10.8pt
 \addcontentsline{toc}{section}{References}
 \begin{thebibliography}{99}
%\bibitem{Karasikov2016}
%\Au{Карасиков~М.\,Е., Стрижов~В.\,В.} Классификация временных рядов 
%в~пространстве параметров по\-рож\-да\-ющих моделей~// Информатика и~её 
%применения,~2016. T.~10. Вып.~4. С.~121--131.

\bibitem{0}
\Au{Hill~N.\,J., Lal~T.\,N., Schroder~M., Hinterberger~T., 
Wilhelm~B., Nijboer~F., Mochty~U., Widman~G., Elger~C., 
Scholkopf~B., Kubler~A., Birbaumer~N.} Classifying EEG and 
ECoG signals without subject training for fast BCI implementation: 
Comparison of nonparalyzed and completely paralyzed subjects~//  
IEEE~T. Neur. Sys. Reh., 2006. Vol.~14. 
Iss.~2. P.~183--186.

\bibitem{1}
\Au{Sakoe~H., Chiba~S.} 
A~dynamic programming approach to continuous speech recognition~// 
7th  Congress (International) on Acoustics Proceedings, 1971. Vol.~3. P.~65--69.

\bibitem{2} %3
\Au{Aghabozorgi~S., Ali~S.\,S., Wah~T.\,Y.} 
Time-series clustering~--- a~decade review~// Inform. Syst., 
2015. Vol.~53. P.~16--38.

\bibitem{3} %4
\Au{Warrenliao~T.} Clustering of time series data~--- a~survey~// 
Pattern Recogn., 2005. Vol.~38. Iss.~11. P.~1857--1874.



\bibitem{4} %5
\Au{Hautamaki~V., Nykanen~P., Franti~P.} 
Time-series clustering by approximate prototypes~// 
19th  Conference (International) on Pattern Recognition Proceedings, 2008. No.\,D. 
P.~1--4.

\bibitem{5} %6
\Au{Faloutsos~C., Ranganathan~M., Manolopoulos~Y.} 
Fast subsequence matching in time-series databases~// \mbox{SIGMOD} Rec., 1994. 
Vol.~23. Iss.~2. P.~419--429.

\bibitem{10} %7
\Au{Basalto~N., Bellotti~R., Carlo~F.\,D., Facchi~P., 
Pascazio~S.} Hausdorff clustering of financial time series~// 
Physica~A, 2007. Vol.~379. Iss.~2. P.~635--644.

\bibitem{11} %8
\Au{Gorelick~L., Blank~M., Shechtman~E., Irani~M., Basri~R.} 
Actions as space-time shapes~// IEEE~T. Pattern Anal., 
2007. Vol.~29. Iss.~12. P.~2247--2253.

\bibitem{6} %9
\Au{Smyth~P.} Clustering sequences with hidden Markov models~// 
Adv. Neural In., 1997. Vol.~9. P.~648--654.

\bibitem{7} %10
\Au{Banerjee~A., Ghosh~J.} Clickstream clustering using weighted 
longest common subsequences~// 
Workshop on Web Mining, SIAM Conference on Data Mining
Proceedings, 2001. P.~33--40.

\bibitem{8} %11
\Au{Aach~J., Church~G.M.} Aligning gene expression time series
 with time warping algorithms~// Bioinformatics, 2001. Vol.~17. Iss.~6. P.~495--508.

\bibitem{9} %12
\Au{Yi~B.\,K., Faloutsos~C.} Fast time sequence indexing 
for arbitrary $\mathcal{L}_p$ norms~// 
26th  Conference (International) on Very Large Data Bases Proceedings, 2000. P.~385--394.

\bibitem{33} %13
\Au{Goncharov~A.\,V., Strijov~V.\,V.} 
Analysis of dissimilarity set between time series~// Computational 
Mathematics Modeling, 2018. Vol.~29. Iss.~3. P.~359--366.

\bibitem{12} %14
\Au{Alon~J., Athitsos~V., Sclaroff~S.}
 Online and offline character recognition using alignment to prototypes~// 
 8th  Conference (International) on Document Analysis and Recognition, 2005. 
 Vol.~2. P.~839--843.

\bibitem{15} %15
\Au{Гончаров~А.\,В.} 
Выравнивания декартовых произведений упорядоченных множеств mDTW. 
Про\-грам\-мная реализация алгоритма, 2019. 
{\sf https://github.
com/Intelligent-Systems-Phystech/PhDThesis/tree/\linebreak  master/Goncharov2019/MatrixDTW/code}.
 \end{thebibliography}

 }
 }

\end{multicols}

\vspace*{-9pt}

\hfill{\small\textit{Поступила в~редакцию 24.04.19}}

\vspace*{6pt}

%\pagebreak

%\newpage

%\vspace*{-28pt}

\hrule

\vspace*{2pt}

\hrule

\vspace*{-4pt}

\def\tit{ALIGNMENT OF~ORDERED SET CARTESIAN PRODUCT\\[-5pt]}


\def\titkol{Alignment of~ordered set cartesian product}

\def\aut{A.\,V.~Goncharov$^1$ and~V.\,V.~Strijov$^{1,2}$}

\def\autkol{A.\,V.~Goncharov and~V.\,V.~Strijov}

\titel{\tit}{\aut}{\autkol}{\titkol}

\vspace*{-13pt}


\noindent
$^1$ Moscow Institute of Physics and Technology, 
9~Institutskiy Per., Dolgoprudny, Moscow Region 141700, Russian\linebreak
$\hphantom{^1}$Federation


\noindent
$^2$A.\,A.~Dorodnicyn Computing Center, Federal Research Center 
``Computer Science and Control'' of the Russian\linebreak
$\hphantom{^1}$Academy of Sciences, 
40~Vavilov Str., Moscow 119333, Russian Federation

\def\leftfootline{\small{\textbf{\thepage}
\hfill INFORMATIKA I EE PRIMENENIYA~--- INFORMATICS AND
APPLICATIONS\ \ \ 2020\ \ \ volume~14\ \ \ issue\ 1}
}%
 \def\rightfootline{\small{INFORMATIKA I EE PRIMENENIYA~---
INFORMATICS AND APPLICATIONS\ \ \ 2020\ \ \ volume~14\ \ \ issue\ 1
\hfill \textbf{\thepage}}}

\vspace*{2pt} 



\Abste{The work is devoted to the study of metric methods for analyzing 
objects with complex structure. It proposes to generalize the dynamic 
time warping method of two time series for the case of objects defined 
on two or more time axes. Such objects are matrices in the discrete 
representation. The DTW (Dynamic Time Warping) method of time series is generalized as 
a~method of matrices dynamic alignment. The paper proposes 
a~distance function resistant to monotonic nonlinear deformations of the 
Cartesian product of two time scales. The alignment path between objects is 
defined. An object is called a~matrix in which the rows and columns correspond 
to the axes of time. The properties of the proposed distance function 
are investigated. To illustrate the method, the problems of metric 
classification of objects are solved on model data and data from the 
MNIST dataset.}

\KWE{distance function; dynamic alignment; distance between matrices; 
nonlinear time warping; space--time series}



\DOI{10.14357/19922264200105} 

%\vspace*{-14pt}

\Ack
\noindent
This work was supported by the Russian Foundation for Basic
Research (projects 19-07-1155 and 19-07-00885). 
The paper contains results of the project Statistical 
methods of machine learning, which is carried out within the 
framework of the Program ``Center of Big Data Storage and Analysis'' 
of the National Technology Initiative Competence Center. 
It is supported by the Ministry of Science and Higher Education 
of the Russian Federation according to the agreement between the
 M.\,V.~Lomonosov Moscow State University and the Foundation 
 of project support of the National Technology Initiative from 11.12.2018, 
 No.\,13/1251/2018.
 


%\vspace*{6pt}

  \begin{multicols}{2}

\renewcommand{\bibname}{\protect\rmfamily References}
%\renewcommand{\bibname}{\large\protect\rm References}

{\small\frenchspacing
 {%\baselineskip=10.8pt
 \addcontentsline{toc}{section}{References}
 \begin{thebibliography}{99}

 \bibitem{0-1}   
\Aue{Hill, N.\,J., T.\,N.~Lal, M.~Schroder, T.~Hinterberger, B.~Wilhelm, 
F.~Nijboer, U.~Mochty, G.~Widman, C.~Elger, B.~Scholkopf, A.~Kubler, and 
N.~Birbaumer.} 2006. Classifying EEG and ECoG signals without subject 
training for fast BCI implementation: Comparison of nonparalyzed and completely 
paralyzed subjects. \textit{IEEE~T. Neur. Sys. 
Reh.} 14(2):183--186.

\bibitem{1-1}   
\Aue{Sakoe, H., and S.~Chiba.} 1971. A~dynamic programming approach 
to continuous speech recognition. \textit{7th 
 Congress (International) on Acoustics Proceedings}. 3:65--69.

\bibitem{2-1}    %2
\Aue{Aghabozorgi,~S., S.\,S.~Ali, and T.\,Y.~Wah.} 2015. 
Time-series clustering~--- a~decade review.  \textit{Inform. Syst.} 
53:16--38.

\bibitem{3-1}   %4 
\Aue{Warrenliao,~T.} 2005. Clustering of time series data~--- a~survey. 
\textit{Pattern Recogn.}
38(11):1857--1874.



\bibitem{4-1}    %5
\Aue{Hautamaki,~V., P.~Nykanen, and P.~Franti.} 2008. 
Time-series clustering by approximate prototypes. 
 \textit{19th  Conference (International) on Pattern Recognition Proceedings}. 
 D:1--4.

\bibitem{5-1}    %6
\Aue{Faloutsos,~C., M.~Ranganathan, and Y.~Manolopoulos.} 1994. 
Fast subsequence matching in time-series databases.  \textit{SIGMOD Rec}. 
23(2):419--429.

\bibitem{10-1}    %7
\Aue{Basalto, N., R.~Bellotti, F.\,D.~Carlo, P.~Facchi, and S.~Pascazio.} 
2007. Hausdorff clustering of financial time series. 
\textit{Physica~A} 379(2):635--644.

\bibitem{11-1}   %8
\Aue{Gorelick, L., M.~Blank, E.~Shechtman, M.~Irani, and R.~Basri.} 
2007. Actions as space-time shapes.
\textit{IEEE~T. Pattern Anal.} 29(12):2247--2253.

\bibitem{6-1}    %9
\Aue{Smyth, P.} 1997. 
Clustering sequences with hidden Markov models. \textit{Adv. Neural In.} 9:648--654.

\bibitem{7-1}    %10
\Aue{Banerjee,~A., and J.~Ghosh.} 2001. 
Clickstream clustering using weighted longest common subsequences.  
\textit{Workshop on Web Mining, SIAM Conference 
on Data Mining Proceedings.} 33--40.

\bibitem{8-1}    %11
\Aue{Aach, J., and G.\,M.~Church.} 2001. 
Aligning gene expression time series with time warping algorithms. 
\textit{Bioinformatics} 17(6):495--508.

\bibitem{9-1}   %12
\Aue{Yi, B.\,K., and C.~Faloutsos.} 2000. 
Fast time sequence indexing for arbitrary $\mathcal{L}_p$ norms. 
\textit{26th  Conference (International) 
on Very Large Data Bases Proceedings}. 385--394.

\bibitem{33-1}   %13 
\Aue{Goncharov,~A.\,V., and V.\,V.~Strijov.} 2018. 
Analysis of dissimilarity set between time series. 
\textit{Computational Mathematics Modeling } 29(3):359--366.



\bibitem{12-1}    %14
\Aue{Alon, J., V.~Athitsos, and S.~Sclaroff.} 2005.
 Online and offline character recognition using alignment to prototypes. 
 \textit{8th  Conference (International) on Document Analysis and Recognition}. 
 2:839--843.

\bibitem{15-1}    %15
\Aue{Goncharov, A.\,V.} Alignment of 
Ordered Set Cartesian Product mDTW. Software implementation of the algorithm. 
Available at: {\sf https://github.com/Intelligent-\linebreak 
Systems-Phystech/PhDThesis/tree/master/Goncharov\linebreak 2019/MatrixDTW/code} 
(accessed December~27, 2019).
\end{thebibliography}

 }
 }

\end{multicols}

%\vspace*{-7pt}

\hfill{\small\textit{Received April 24, 2019}}

%\pagebreak

%\vspace*{-22pt}



\Contr

\noindent
\textbf{Goncharov Alexey V.} (b.\ 1995)~--- 
PhD student, Moscow Institute of Physics and Technology, 
9~Institutskiy Per., Dolgoprudny, Moscow Region 141701, 
Russian Federation; \mbox{alex.goncharov@phystech.edu}

\vspace*{3pt}

\noindent
\textbf{Strijov Vadim V.} (b.\ 1967)~--- 
Doctor of Science in physics and mathematics, leading scientist, 
A.\,A.~Dorodnicyn Computing Centre, Federal Research Center 
``Computer Science and Control'' of the Russian Academy of Sciences, 
40~Vavilov Str., Moscow 119333, Russian Federation;
 professor, Moscow Institute of Physics and Technology, 
 9~Institutskiy Per., Dolgoprudny, Moscow Region 141701, Russian Federation; 
 \mbox{strijov@ccas.ru}
\label{end\stat}

\renewcommand{\bibname}{\protect\rm Литература}         %15



%%%%%%%%%%%%%%%%%%%%%%%%%%%%%%%%%%%%%%%%

%\def\stat{rez}
{%\hrule\par
%\vskip 7pt % 7pt
\raggedleft\Large \bf%\baselineskip=3.2ex
Р\,Е\,Ц\,Е\,Н\,З\,И\,И \vskip 17pt
    \hrule
    \par
\vskip 6pt plus 6pt minus 3pt }

%\thispagestyle{headings} %с верхним колонтитулом
%\thispagestyle{myheadings} %с нижним колонтитулом, но в верхнем РЕЦЕНЗИИ

\def\tit{НОВАЯ КНИГА И.\,Н.~СИНИЦЫНА, А.\,С.~ШАЛАМОВА <<ЛЕКЦИИ ПО ТЕОРИИ 
ИНТЕГРИРОВАННОЙ ЛОГИСТИЧЕСКОЙ ПОДДЕРЖКИ>> (М.: ТОРУС ПРЕСС, 2012. 624~с.)}

%1
\def\aut{Д.ф.-м.н., профессор С.\,Я.~Шоргин}

\def\auf{\ }

\def\leftkol{\ % РЕЦЕНЗИИ
}

\def\rightkol{ \ } 

%\def\leftkol{\ } % ENGLISH ABSTRACTS}

%\def\rightkol{\ } %ENGLISH ABSTRACTS}

%\def\leftkol{РЕЦЕНЗИИ}

%\def\rightkol{РЕЦЕНЗИИ}

\titele{\tit}{\aut}{\auf}{\leftkol}{\rightkol}
\vspace*{-18pt}


     \label{st\stat}

     \begin{multicols}{2}
     {\small
     {\baselineskip=10.1pt
     

      В книге представлено системное изложение теоретических основ одного из новейших 
направлений в \mbox{об\-ласти} экономики послепродажного обслуживания изделий наукоемкой 
продукции (ИНП) длительного пользования~--- интегрированной логистической поддержки
(ИЛП). 
{\looseness=1

}

Приведены также результаты новых работ, выполненных в Институте проблем информатики 
Российской академии наук в рамках научного направления <<Информационные технологии и 
анализ сложных сис\-тем>>.
 {%\looseness=1

}
     
      Излагаемые в книге научные подходы позво\-ляют карди\-наль\-но реформировать 
существующие системы производства и эксплуатации ИНП путем создания и внед\-ре\-ния 
методов рационального и оптимального управ\-ле\-ния процессами расходования 
вре\-мен\-н$\acute{\mbox{ы}}$х, 
мате\-ри\-аль\-ных, трудовых и других ресурсов на всех стадиях жизненного цикла изделий (ЖЦИ) по 
критериям экономической целесообразности и эф\-фек\-тив\-ности.
  {\looseness=1

}
    
      В книге приведен краткий обзор причин возник\-новения и
      развития CALS-методологии как основы 
современных международных стандартов по созданию и функционированию глобальных 
ин\-фор\-ма\-ци\-он\-но-ком\-му\-ни\-ка\-ци\-он\-ных систем, ее ключевых возможностей и эффективности 
результатов ее использования. 
Авторы %\linebreak 
предлагают ряд научных обоснований для разработки 
единой теории проектирования и управления систем ИЛП для полноценного использования 
преимуществ %\linebreak
 суще\-ст\-ву\-ющей методологии, определяют \mbox{общую} структурную схему 
комплексной системы <<ИНП-СППО>> и необходимость разработки для ее описания 
гибридных стохастических моделей.
{%\looseness=1

}

%\columnbreak
      
      Книга состоит из пяти частей, где последовательно излагается материал по каждой из 
следующих тем: <<Интегрированная логистическая поддержка>>, <<Теория гибридных 
стохастических систем и компьютерная поддержка исследований и разработок>>, <<Основы 
математического моделирования, анализа и синтеза систем послепродажного обслуживания>>, 
<<Определение и анализ показателей экспортного потенциала ИНП при проектировании>>, 
<<Задачи управления поддержкой послепродажного обслуживания>>, а также 
<<Моделирование инвестиционных процессов ИЛП в условиях неравновесных финансовых 
рынков>>. 
   
      В конце каждой главы приведены выводы и даны вопросы и задания для 
самоконтроля. В~приложениях содержатся основные определения по программам работ по 
анализу ИЛП, логистическим базам данных и компьютерным решениям, эквивалентной статистической 
линеаризации нелинейных преобразований ИЛП, справочный материал, а также развернутые 
уравнения для вероятностных характеристик.


      \def\leftkol{РЕЦЕНЗИИ}

\def\rightkol{РЕЦЕНЗИИ} 

      
      Книга заинтересует широкий круг специалистов и может быть использована научными 
проектными организациями в сфере промышленного производства ИНП. Большое количество 
иллюстраций, примеров и вопросов, обращенных к читателю, позволяет использовать книгу 
также в качестве учебного пособия для студентов и аспирантов машиностроительных, 
транспортных и~других специальностей, а также для самостоятельного изучения. 
{%\looseness=-1

}

Книга 
представляет несомненный интерес для специалистов и студентов в области прикладной 
математики и информатики.
    

}

}
\end{multicols}

%\newpage

\def\stat{authorsrus}
{%\hrule\par
%\vskip 7pt % 7pt
\raggedleft\Large \bf%\baselineskip=3.2ex
О\,Б\ \ А\,В\,Т\,О\,Р\,А\,Х \vskip 17pt
    \hrule
    \par
\vskip 21pt plus 8pt minus 4pt }


\def\tit{\ }

\def\aut{\ }

\def\auf{\ }

\def\leftkol{\ } % ENGLISH ABSTRACTS}

\def\rightkol{ОБ АВТОРАХ} %ENGLISH ABSTRACTS}

\titele{\tit}{\aut}{\auf}{\leftkol}{\rightkol}
      
            \label{st\stat}



\vspace*{24pt}

\begin{multicols}{2}




\noindent
\textbf{Архипов Олег Петрович} (р.\ 1948)~---
кандидат технических наук, директор Орловского филиала Института проб\-лем информатики
Российской академии наук
%302025, г.Орел, Московское шоссе, д.137

\vspace*{3pt}

\noindent
\textbf{Бирюкова Татьяна Константиновна} (р.\ 1968)~---
кандидат фи\-зи\-ко-ма\-те\-ма\-ти\-че\-ских наук, старший научный сотрудник Института проб\-лем информатики
Российской академии наук

\vspace*{3pt}

\noindent 
\textbf{Бобков  Сергей Геннадьевич} (р.\ 1955)~---
доктор технических наук,  заведующий отделением На\-уч\-но-ис\-сле\-до\-ва\-тель\-ско\-го 
института системных исследований Российской академии наук
%117218, Москва, Нахимовский просп., 36, к.1 

\vspace*{3pt}

\noindent \textbf{Васильев Николай Семенович} (р.\ 1952)~--- доктор 
фи\-зи\-ко-ма\-те\-ма\-ти\-че\-ских наук, профессор, 
МГТУ им.\ Н.\,Э.~Баумана 
%, Москва 105005, 2-я Бауманская ул., д.~5,

\vspace*{3pt}

\noindent
\textbf{Гершкович Максим Михайлович} (р.\ 1968)~---
старший научный сотрудник Института проб\-лем информатики
Российской академии наук

\vspace*{3pt}

\noindent 
\textbf{Дьяченко Юрий Георгиевич} (р.\ 1958)~--- кандидат технических наук, 
старший научный сотрудник Института проб\-лем информатики
Российской академии наук

\vspace*{3pt}

\noindent 
\textbf{Ерошенко Александр Андреевич} (р.\ 1989)~--- аспирант кафедры 
математической статистики факультета вычисли\-тельной математики и кибернетики 
Московского государственного университета им.\ М.\,В.~Ломоносова
%119991, Москва ГСП-1, Ленинские горы, д.\ 1, стр. 52

\vspace*{3pt}
 
\noindent 
\textbf{Захаров Виктор Николаевич} (р.\ 1948)~--- 
доктор технических наук, доцент, ученый секретарь Института проб\-лем информатики
Российской академии наук

\vspace*{3pt}

\noindent
\textbf{Зейфман Александр Израилевич} (р.\ 1954)~---
доктор фи\-зи\-ко-ма\-те\-ма\-ти\-че\-ских наук, профессор, 
заведующий кафедрой Вологодского государственного университета; 
старший научный сотрудник Института проб\-лем информатики
Российской академии наук; главный научный сотрудник ИСЭРТ Российской академии наук

\vspace*{3pt}

\noindent
\textbf{Зыкин Сергей Владимирович} (р.\ 1959)~--- 
доктор технических наук, профессор, заведующий лабораторией Института математики 
им.\ С.\,Л.~Соболева Сибирского отделения Российской академии наук, Новосибирск 
%630090, пр.\ ак.\ Коптюга, 4 

\vspace*{4pt}

\noindent
\textbf{Киреев Владимир Иванович} (р.\ 1938)~---
доктор фи\-зи\-ко-ма\-те\-ма\-ти\-че\-ских наук, профессор Московского 
государственного горного университета
%Адрес: Россия, 119991, г. Москва, Ленинский проспект, д. 6

%\columnbreak

\vspace*{4pt}

\noindent
\textbf{Козеренко Елена Борисовна} (р.\ 1959)~---
кандидат филологических наук, заведующая лабораторией Института проб\-лем информатики
Российской академии наук

\vspace*{4pt}

\noindent
\textbf{Королев Виктор Юрьевич} (р.\ 1954)~--- доктор
фи\-зи\-ко-ма\-те\-ма\-ти\-че\-ских наук, профессор кафедры математической 
статистики факультета вычисли\-тельной математики и кибернетики 
Московского государственного университета; 
ведущий научный сотрудник Института проб\-лем информатики
Российской академии наук

\vspace*{4pt}

\noindent
\textbf{Коротышева Анна Владимировна} (р.\ 1988)~---
старший преподаватель Вологодского государственного университета

\vspace*{4pt}

\noindent 
\textbf{Кун Де Турк} (р.\ 1981)~--- научный сотрудник 
исследовательской группы SMACS факультета телекоммуникаций и обработки информации
Университета Гента, Бельгия
%В-9000 Гент, Бельгия

\vspace*{4pt}

\noindent
\textbf{Лупенцов Олег Сергеевич} (р.\ 1986)~---
аспирант Омского государственного института сервиса
%Омск 644043, ул.\ Певцова 13

\vspace*{4pt}

\noindent
\textbf{Лучко Олег Николаевич} (р.\ 1961)~---
кандидат педагогических наук, профессор, заведующий кафедрой 
Омского государственного института сервиса
%Омск 644043, ул.\ Певцова 13

\vspace*{4pt}

\noindent
\textbf{Малашенко Юрий Евгеньевич} (р.\ 1946)~---
доктор фи\-зи\-ко-ма\-те\-ма\-ти\-че\-ских наук, заведующий сектором 
Вычислительного центра им.\ А.\,А.~Дородницына Российской академии наук
%Адрес: 119333, Москва, ул. Вавилова, 40,

\vspace*{4pt}

\noindent
\textbf{Маньяков Юрий Анатольевич} (р.\ 1984)~---
кандидат технических наук, научный сотрудник Орловского филиала Института проб\-лем информатики
Российской академии наук
%302025, г.Орел, Московское шоссе, д.137

\vspace*{4pt}

\noindent
\textbf{Маренко Валентина Афанасьевна} (р.\ 1951)~---
кандидат технических наук, доцент, старший научный сотрудник 
Института математики им.\ С.\,Л.~Соболева Сибирского отделения Российской академии наук
%Новосибирск 630090, пр. ак. Коптюга, 4 

\vspace*{3pt}

\noindent 
\textbf{Морозов Евсей Викторович} (р.\ 1947)~--- доктор 
фи\-зи\-ко-ма\-те\-ма\-ти\-че\-ских, профессор, ведущий научный сотрудник 
Института прикладных математических исследований Карельского научного центра Российской
академии наук; 
%%185910 Россия, Республика Карелия, г.\ Петрозаводск, ул.\ Пушкинская, 11
профессор Петрозаводского государственного университета, Петрозаводск
%185910 Россия, Республика Карелия, г.\ Петрозаводск, пр.\ Ленина, 33

%\pagebreak

\vspace*{3pt}

\noindent
\textbf{Назарова Ирина Александровна} (р.\ 1966)~---
кандидат фи\-зи\-ко-ма\-те\-ма\-ти\-че\-ских наук, 
научный сотрудник Вычислительного центра им.\ А.\,А.~Дородницына Российской академии наук 
%Адрес: 119333, Москва, ул. Вавилова, 40

\vspace*{3pt}

\noindent
\textbf{Павлов Игорь Валерианович} (р.\ 1945)~--- 
доктор фи\-зи\-ко-ма\-те\-ма\-ти\-че\-ских наук, профессор МГТУ им.\ Н.\,Э.~Баумана 
%Москва 105005, 2-я Бауманская ул., д.~5 

%\pagebreak

\vspace*{3pt}

\noindent 
\textbf{Потахина Любовь Викторовна} (р.\ 1989)~--- аспирантка
Института прикладных математических исследований Карельского научного центра
Российской академии наук; 
%%185910 Россия, Республика Карелия, г.\ Петрозаводск, ул.\ Пушкинская, 11
инженер Петрозаводского государственного университета, Петрозаводск
%185910 Россия, Республика Карелия, г.\ Петрозаводск, пр.\ Ленина, 33

\vspace*{3pt}

\noindent 
\textbf{Рождественский Юрий Владимирович} (р.\ 1952)~--- 
кандидат технических наук, заведующий сектором Института проб\-лем информатики
Российской академии наук

\vspace*{3pt}

\noindent 
\textbf{Синицын Игорь Николаевич} (р.\ 1940)~--- доктор технических наук,
профессор, заслуженный деятель\linebreak\vspace*{-12pt}

\columnbreak

\noindent
 науки РФ, заведующий отделом Института проб\-лем информатики
Российской академии наук

\vspace*{7pt}


\noindent
\textbf{Сиротинин Денис Олегович} (р.\ 1984)~---
кандидат технических наук, научный сотрудник Орловского филиала Института проб\-лем информатики
Российской академии наук
%302025, г.Орел, Московское шоссе, д.137

\vspace*{7pt}

%\columnbreak

\noindent 
\textbf{Соколов  Игорь Анатольевич} (р.\ 1954)~--- академик (действительный член) Российской 
академии наук, доктор технических наук, директор Института проб\-лем информатики
Российской академии наук

\vspace*{7pt}

\noindent
\textbf{Степченков Юрий Афанасьевич} (р.\ 1951)~---
кандидат технических наук, заведующий отделом Института проб\-лем информатики
Российской академии наук

\vspace*{7pt}

\noindent
\textbf{Сурков Алексей Викторович} (р.\ 1978)~--- 
старший научный сотрудник На\-уч\-но-ис\-сле\-до\-ва\-тель\-ско\-го 
института системных исследований Российской академии наук
%117218, Москва, Нахимовский просп., 36, к.1 

\vspace*{7pt}

\noindent 
\textbf{Шестаков Олег Владимирович} (р.\ 1976)~--- доктор 
фи\-зи\-ко-ма\-те\-ма\-ти\-че\-ских, доцент кафедры математической статистики 
факультета вычисли\-тельной математики и кибернетики Московского 
государственного университета им.\ М.\,В.~Ломоносова; 
%119991, Москва ГСП-1, Ленинские горы, д.\ 1, стр. 52
старший научный сотрудник Института проб\-лем информатики
Российской академии наук
%, Москва 119333, ул. Вавилова, д.~44, корп.~2

\vspace*{7pt}

\noindent 
\textbf{Шоргин Сергей Яковлевич} (р.\ 1952.)~--- доктор
фи\-зи\-ко-ма\-те\-ма\-ти\-че\-ских наук, профессор, заместитель директора Института 
проб\-лем информатики Российской академии наук





%%%%%%%%%%%%%%%%%%%%%%%%%%%%%%%%%%%%%%%%%%%%%%%%%%%%%%%%%%%%%%%%%%%%%%%%%%%%%%%




%\def\rightkol{ОБ АВТОРАХ}
%\def\leftkol{ОБ АВТОРАХ}

 \label{end\stat}





%\def\leftfootline{\small{\textbf{\thepage}
%\hfill ИНФОРМАТИКА И ЕЁ ПРИМЕНЕНИЯ\ \ \ том~7\ \ \ выпуск~1\ \ \ 2013}
%}%
% \def\rightfootline{\small{ИНФОРМАТИКА И ЕЁ ПРИМЕНЕНИЯ\ \ \ том~7\ \ \ выпуск~1\ \ \ 2013
%\hfill \textbf{\thepage}}}


%\thispagestyle{myheadings}



\end{multicols}

\newpage  

%\def\stat{cont}
{%\hrule\par
%\vskip 7pt % 7pt
\raggedleft\Large \bf%\baselineskip=3.2ex
А\,В\,Т\,О\,Р\,С\,К\,И\,Й\ \ У\,К\,А\,З\,А\,Т\,Е\,Л\,Ь\ \ З\,А\ \ 2\,0\,0\,7 г. \vskip 17pt
    \hrule
    \par
\vskip 21pt plus 6pt minus 3pt }

\label{st\stat}

\def\tit{\ }

\def\aut{\ }
\def\auf{\ }

\def\leftkol{\ } % ENGLISH ABSTRACTS}

\def\rightkol{\ } %ENGLISH ABSTRACTS}

\titele{\tit}{\aut}{\auf}{\leftkol}{\rightkol}


\contentsline {chapter}{\ }{Выпуск \quad Стр.} 
\contentsline {section}{\textbf{Батракова Д.\,А., Королев В.\,Ю., Шоргин С.\,Я.}\ \ Новый метод вероятностно-ста\-ти\-сти\-че\-ско\-го анализа информационных потоков в\nobreakspace {}телекоммуникационных сетях}{\qquad 1 \qquad 40} 
\contentsline {section}{\textbf{Борисов А.\,В.}\ \ Байесовское оценивание в системах наблюдения с\nobreakspace {}марковскими скачкообразными процессами: игровой подход}{\qquad 2 \qquad 65}
\contentsline {section}{\textbf{Босов А.\,В., Иванов А.\,В.}\ \ Программная инфраструктура информационного Web-пор\-тала}{\qquad 2 \qquad 50}
\contentsline {section}{\textbf{Захаров В.\,Н., Калиниченко Л.\,А., Соколов И.\,А., Ступников С.\,А.}\ \ Конструирование канонических информационных моделей для интегрированных информационных систем}{\qquad 2 \qquad 15}
\contentsline {section}{\textbf{Захаров В.\,Н., Козмидиади В.\,А.}\ \ Средства обеспечения отказоустойчивости при\-ло\-жений}{\qquad 1 \qquad 14} 
\contentsline {section}{\textbf{Иванов А.\,В.}\ \ см. Босов А.\,В.\hfill\hfill\hfill\hfill\hfill\hfill\hfill\hfill\hfill\hfill\hfill\hfill\hfill\hfill\hfill\hfill\hfill\hfill\hfill\hfill\hfill\hfill\hfill\hfill\hfill\hfill\hfill\hfill\hfill\hfill\hfill\hfill\hfill\hfill\hfill}{\ }
\contentsline {section}{\textbf{Ильин В.\,Д., Соколов И.\,А.}\ \ Символьная модель системы знаний информатики в\nobreakspace {}че\-ло\-ве\-ко-автоматной среде}{\qquad 1 \qquad 66} 
\contentsline {section}{\textbf{Калиниченко Л.\,А.}\ \ см. Захаров В.\,Н.\hfill\hfill\hfill\hfill\hfill\hfill\hfill\hfill\hfill\hfill\hfill\hfill\hfill\hfill\hfill\hfill\hfill\hfill\hfill\hfill\hfill\hfill\hfill\hfill\hfill\hfill\hfill\hfill\hfill\hfill\hfill\hfill\hfill\hfill\hfill}{\ }
\contentsline {section}{\textbf{Козеренко Е.\,Б.}\ \ Лингвистическое моделирование для систем машинного перевода и обработки знаний}{\qquad 1 \qquad 54} 
\contentsline {section}{\textbf{Козмидиади В.\,А.}\ \ см. Захаров В.\,Н.\hfill\hfill\hfill\hfill\hfill\hfill\hfill\hfill\hfill\hfill\hfill\hfill\hfill\hfill\hfill\hfill\hfill\hfill\hfill\hfill\hfill\hfill\hfill\hfill\hfill\hfill\hfill\hfill\hfill\hfill\hfill\hfill\hfill\hfill\hfill }{\ } 
\contentsline {section}{\textbf{Королев В.\,Ю.}\ \ см. Батракова Д.\,А.\hfill\hfill\hfill\hfill\hfill\hfill\hfill\hfill\hfill\hfill\hfill\hfill\hfill\hfill\hfill\hfill\hfill\hfill\hfill\hfill\hfill\hfill\hfill\hfill\hfill\hfill\hfill\hfill\hfill\hfill\hfill\hfill\hfill\hfill\hfill}{\ } 
\contentsline {section}{\textbf{Кудрявцев А.\,А., Шоргин С.\,Я.}\ \ Байесовский подход к\nobreakspace {}анализу систем массового обслуживания и\nobreakspace {}показателей надежности}{\qquad 2 \qquad 76}
\contentsline {section}{\textbf{Печинкин А.\,В., Соколов И.\,А., Чаплыгин В.\,В.}\ \ Многолинейная система массового обслуживания с конечным накопителем и ненадежными приборами}{\qquad 1 \qquad 27} 
\contentsline {section}{\textbf{Печинкин А.\,В., Соколов И.\,А., Чаплыгин В.\,В.}\ \ Стационарные характеристики многолинейной\nobreakspace {}системы массового обслуживания с\nobreakspace {}одновременными отказами приборов}{\qquad 2 \qquad 39}
\contentsline {section}{\textbf{Синицын И.\,Н.}\ \ Корреляционные методы построения аналитических информационных моделей флуктуаций полюса Земли по априорным данным}{\qquad 2 \qquad \hphantom{9}2}
\contentsline {section}{\textbf{Синицын И.\,Н.}\ \ Развитие теории фильтров Пугачева для оперативной обработки информации в стохастических системах}{{\qquad 1 \qquad \hphantom{9}3}} 
\contentsline {section}{\textbf{Соколов И.\,А.}\ \ см. Захаров В.\,Н.\hfill\hfill\hfill\hfill\hfill\hfill\hfill\hfill\hfill\hfill\hfill\hfill\hfill\hfill\hfill\hfill\hfill\hfill\hfill\hfill\hfill\hfill\hfill\hfill\hfill\hfill\hfill\hfill\hfill\hfill\hfill\hfill\hfill\hfill\hfill}{\ }
\contentsline {section}{\textbf{Соколов И.\,А.}\ \ см. Ильин В.\,Д.\hfill\hfill\hfill\hfill\hfill\hfill\hfill\hfill\hfill\hfill\hfill\hfill\hfill\hfill\hfill\hfill\hfill\hfill\hfill\hfill\hfill\hfill\hfill\hfill\hfill\hfill\hfill\hfill\hfill\hfill\hfill\hfill\hfill\hfill\hfill}{\ } 
\contentsline {section}{\textbf{Соколов И.\,А.}\ \ см. Печинкин А.\,В.\hfill\hfill\hfill\hfill\hfill\hfill\hfill\hfill\hfill\hfill\hfill\hfill\hfill\hfill\hfill\hfill\hfill\hfill\hfill\hfill\hfill\hfill\hfill\hfill\hfill\hfill\hfill\hfill\hfill\hfill\hfill\hfill\hfill\hfill\hfill}{\ } 
\contentsline {section}{\textbf{Соколов И.\,А.}\ \ см. Печинкин А.\,В.\hfill\hfill\hfill\hfill\hfill\hfill\hfill\hfill\hfill\hfill\hfill\hfill\hfill\hfill\hfill\hfill\hfill\hfill\hfill\hfill\hfill\hfill\hfill\hfill\hfill\hfill\hfill\hfill\hfill\hfill\hfill\hfill\hfill\hfill\hfill}{\ }
\contentsline {section}{\textbf{Ступников С.\,А.}\ \ см. Захаров В.\,Н.\hfill\hfill\hfill\hfill\hfill\hfill\hfill\hfill\hfill\hfill\hfill\hfill\hfill\hfill\hfill\hfill\hfill\hfill\hfill\hfill\hfill\hfill\hfill\hfill\hfill\hfill\hfill\hfill\hfill\hfill\hfill\hfill\hfill\hfill\hfill}{\ }
\contentsline {section}{\textbf{Чаплыгин В.\,В.}\ \ см. Печинкин А.\,В.\hfill\hfill\hfill\hfill\hfill\hfill\hfill\hfill\hfill\hfill\hfill\hfill\hfill\hfill\hfill\hfill\hfill\hfill\hfill\hfill\hfill\hfill\hfill\hfill\hfill\hfill\hfill\hfill\hfill\hfill\hfill\hfill\hfill\hfill\hfill}{\ } 
\contentsline {section}{\textbf{Чаплыгин В.\,В.}\ \ см. Печинкин А.\,В.\hfill\hfill\hfill\hfill\hfill\hfill\hfill\hfill\hfill\hfill\hfill\hfill\hfill\hfill\hfill\hfill\hfill\hfill\hfill\hfill\hfill\hfill\hfill\hfill\hfill\hfill\hfill\hfill\hfill\hfill\hfill\hfill\hfill\hfill\hfill}{\ }
\contentsline {section}{\textbf{Шоргин С.\,Я.}\ \ см. Батракова Д.\,А.\hfill\hfill\hfill\hfill\hfill\hfill\hfill\hfill\hfill\hfill\hfill\hfill\hfill\hfill\hfill\hfill\hfill\hfill\hfill\hfill\hfill\hfill\hfill\hfill\hfill\hfill\hfill\hfill\hfill\hfill\hfill\hfill\hfill\hfill\hfill}{\ } 
\contentsline {section}{\textbf{Шоргин С.\,Я.}\ \ см. Кудрявцев А.\,А.\hfill\hfill\hfill\hfill\hfill\hfill\hfill\hfill\hfill\hfill\hfill\hfill\hfill\hfill\hfill\hfill\hfill\hfill\hfill\hfill\hfill\hfill\hfill\hfill\hfill\hfill\hfill\hfill\hfill\hfill\hfill\hfill\hfill\hfill\hfill}{\ }
%\thispagestyle{myheadings}
\def\leftfootline{\small{\textbf{\thepage}
\hfill ИНФОРМАТИКА И ЕЁ ПРИМЕНЕНИЯ\ \ \ том~1\ \ \ выпуск~2\ \ \ 2007}
}%
 \def\rightfootline{\small{ИНФОРМАТИКА И ЕЁ ПРИМЕНЕНИЯ\ \ \ том~1\ \ \ выпуск~2\ \ \ 2007
 \hfill \textbf{\thepage}}}
 \label{end\stat} 
                     
%\def\stat{cont-e}
{%\hrule\par
%\vskip 7pt % 7pt
\raggedleft\Large \bf%\baselineskip=3.2ex
2\,0\,0\,7\ \ A\,U\,T\,H\,O\,R\ \ I\,N\,D\,E\,X \vskip 17pt
    \hrule
    \par
\vskip 21pt plus 6pt minus 3pt }

\label{st\stat}

\def\tit{\ }

\def\aut{\ }
\def\auf{\ }

\def\leftkol{\ } % ENGLISH ABSTRACTS}

\def\rightkol{\ } %ENGLISH ABSTRACTS}

\titele{\tit}{\aut}{\auf}{\leftkol}{\rightkol}


\contentsline {chapter}{\ }{Issue \quad Page} 
\contentsline {subsection}{\textbf{Batrakova D.\,A., Korolev V.\,Yu., Shorgin S.\,Ya.}\ \ A New Method for the Probabilistic and Statistical Analysis of Information Flows in Telecommunication Networks}{\qquad 1 \qquad 40} 
\contentsline {subsection}{\textbf{Borisov A.\,V.}\ \ Bayesian Estimation in\nobreakspace {}Observation Systems with\nobreakspace {}Markov Jump Processes: Game-Theoretic Approach}{\qquad 2 \qquad 65} 
\contentsline {subsection}{\textbf{Bosov A.\,V., Ivanov A.\,V.}\ \ Linguistic Simulation for Machine Translation and Knowledge Management Systems}{\qquad 2 \qquad 50} 
\contentsline {subsection}{\textbf{Chaplygin V.\,V.} see Pechinkin A.\,V.\hfill\hfill\hfill\hfill\hfill\hfill\hfill\hfill\hfill\hfill\hfill\hfill\hfill\hfill\hfill\hfill\hfill\hfill\hfill\hfill\hfill\hfill\hfill\hfill\hfill\hfill\hfill\hfill\hfill\hfill\hfill\hfill\hfill\hfill\hfill}{\ }
\contentsline {subsection}{\textbf{Chaplygin V.\,V.} see Pechinkin A.\,V.\hfill\hfill\hfill\hfill\hfill\hfill\hfill\hfill\hfill\hfill\hfill\hfill\hfill\hfill\hfill\hfill\hfill\hfill\hfill\hfill\hfill\hfill\hfill\hfill\hfill\hfill\hfill\hfill\hfill\hfill\hfill\hfill\hfill\hfill\hfill}{\ }
\contentsline {subsection}{\textbf{Ilyin V.\,D., Sokolov I.\,A.}\ \ The Symbol Model of Informatics Knowledge System in Human-Automaton Environment}{\qquad 1 \qquad 66} 
\contentsline {subsection}{\textbf{Ivanov A.\,V.} see Bosov A.\,V.\hfill\hfill\hfill\hfill\hfill\hfill\hfill\hfill\hfill\hfill\hfill\hfill\hfill\hfill\hfill\hfill\hfill\hfill\hfill\hfill\hfill\hfill\hfill\hfill\hfill\hfill\hfill\hfill\hfill\hfill\hfill\hfill\hfill\hfill\hfill}{\ }
\contentsline {subsection}{\textbf{Kalinichenko L.\,A.} see Zakharov V.\,N.\hfill\hfill\hfill\hfill\hfill\hfill\hfill\hfill\hfill\hfill\hfill\hfill\hfill\hfill\hfill\hfill\hfill\hfill\hfill\hfill\hfill\hfill\hfill\hfill\hfill\hfill\hfill\hfill\hfill\hfill\hfill\hfill\hfill\hfill\hfill}{\ }
\contentsline {subsection}{\textbf{Korolev V.\,Yu.} see Batrakova D.\,A.\hfill\hfill\hfill\hfill\hfill\hfill\hfill\hfill\hfill\hfill\hfill\hfill\hfill\hfill\hfill\hfill\hfill\hfill\hfill\hfill\hfill\hfill\hfill\hfill\hfill\hfill\hfill\hfill\hfill\hfill\hfill\hfill\hfill\hfill\hfill}{\ }
\contentsline {subsection}{\textbf{Kozerenko E.\,B.}\ \ Linguistic Simulation for Machine Translation and Knowledge Management Systems}{\qquad 1 \qquad 54} 
\contentsline {subsection}{\textbf{Kozmidiady V.\,A.} see Zakharov V.\,N.\hfill\hfill\hfill\hfill\hfill\hfill\hfill\hfill\hfill\hfill\hfill\hfill\hfill\hfill\hfill\hfill\hfill\hfill\hfill\hfill\hfill\hfill\hfill\hfill\hfill\hfill\hfill\hfill\hfill\hfill\hfill\hfill\hfill\hfill\hfill}{\ }
\contentsline {subsection}{\textbf{Kudryavtsev A.\,A., Shorgin S.\,Ya.}\ \ Bayesian Approach to Queueing Systems and Reliability Characteristics}{\qquad 2 \qquad 76} 
\contentsline {subsection}{\textbf{Pechinkin A.\,V., Sokolov I.\,A., Chaplygin V.\,V.}\ \ Multichannel Queuing System with Finite Buffer and Unreliable Servers}{\qquad 1 \qquad 27} 
\contentsline {subsection}{\textbf{Pechinkin A.\,V., Sokolov I.\,A., Chaplygin V.\,V.}\ \ Stationary Characteristics of a Multichannel Queueing System with\nobreakspace {}Simultaneous Refusals of Servers}{\qquad 2 \qquad 39} 
\contentsline {subsection}{\textbf{Shorgin S.\,Ya.} see Batrakova D.\,A.\hfill\hfill\hfill\hfill\hfill\hfill\hfill\hfill\hfill\hfill\hfill\hfill\hfill\hfill\hfill\hfill\hfill\hfill\hfill\hfill\hfill\hfill\hfill\hfill\hfill\hfill\hfill\hfill\hfill\hfill\hfill\hfill\hfill\hfill\hfill}{\ }
\contentsline {subsection}{\textbf{Shorgin S.\,Ya.} see Kudryavtsev A.\,A.\hfill\hfill\hfill\hfill\hfill\hfill\hfill\hfill\hfill\hfill\hfill\hfill\hfill\hfill\hfill\hfill\hfill\hfill\hfill\hfill\hfill\hfill\hfill\hfill\hfill\hfill\hfill\hfill\hfill\hfill\hfill\hfill\hfill\hfill\hfill}{\ }
\contentsline {subsection}{\textbf{Sinitsyn I.\,N.}\ \ Correlational Methods for Analytical Informational Models of the Earth Pole Fluctuations Design Based on a priori Data}{\qquad 2 \qquad \hphantom{9}2}
\contentsline {subsection}{\textbf{Sinitsyn I.\,N.}\ \ Development of Pugachev Filtering for Stochastic Systems}{\qquad 1 \qquad \hphantom{9}3}
\contentsline {subsection}{\textbf{Sokolov I.\,A.} see Ilyin V.\,D.\hfill\hfill\hfill\hfill\hfill\hfill\hfill\hfill\hfill\hfill\hfill\hfill\hfill\hfill\hfill\hfill\hfill\hfill\hfill\hfill\hfill\hfill\hfill\hfill\hfill\hfill\hfill\hfill\hfill\hfill\hfill\hfill\hfill\hfill\hfill}{\ }
\contentsline {subsection}{\textbf{Sokolov I.\,A.} see Pechinkin A.\,V.\hfill\hfill\hfill\hfill\hfill\hfill\hfill\hfill\hfill\hfill\hfill\hfill\hfill\hfill\hfill\hfill\hfill\hfill\hfill\hfill\hfill\hfill\hfill\hfill\hfill\hfill\hfill\hfill\hfill\hfill\hfill\hfill\hfill\hfill\hfill}{\ }
\contentsline {subsection}{\textbf{Sokolov I.\,A.} see Pechinkin A.\,V.\hfill\hfill\hfill\hfill\hfill\hfill\hfill\hfill\hfill\hfill\hfill\hfill\hfill\hfill\hfill\hfill\hfill\hfill\hfill\hfill\hfill\hfill\hfill\hfill\hfill\hfill\hfill\hfill\hfill\hfill\hfill\hfill\hfill\hfill\hfill}{\ }
\contentsline {subsection}{\textbf{Sokolov I.\,A.} see Zakharov V.\,N.\hfill\hfill\hfill\hfill\hfill\hfill\hfill\hfill\hfill\hfill\hfill\hfill\hfill\hfill\hfill\hfill\hfill\hfill\hfill\hfill\hfill\hfill\hfill\hfill\hfill\hfill\hfill\hfill\hfill\hfill\hfill\hfill\hfill\hfill\hfill}{\ }
\contentsline {subsection}{\textbf{Stupnikov S.\,A.} see Zakharov V.\,N.\hfill\hfill\hfill\hfill\hfill\hfill\hfill\hfill\hfill\hfill\hfill\hfill\hfill\hfill\hfill\hfill\hfill\hfill\hfill\hfill\hfill\hfill\hfill\hfill\hfill\hfill\hfill\hfill\hfill\hfill\hfill\hfill\hfill\hfill\hfill}{\ }
\contentsline {subsection}{\textbf{Zakharov V.\,N., Kalinichenko L.\,A., Sokolov I.\,A., Stupnikov S.\,A.}\ \ Development of Canonical Information Models for Integrated Information Systems}{\qquad 2 \qquad 15} 
\contentsline {subsection}{\textbf{Zakharov V.\,N., Kozmidiady V.\,A.}\ \ Means Providing Applications Fault Tolerance}{\qquad 1 \qquad 14} 
\def\leftfootline{\small{\textbf{\thepage}
\hfill ИНФОРМАТИКА И ЕЁ ПРИМЕНЕНИЯ\ \ \ том~1\ \ \ выпуск~2\ \ \ 2007}
}%
 \def\rightfootline{\small{ИНФОРМАТИКА И ЕЁ ПРИМЕНЕНИЯ\ \ \ том~1\ \ \ выпуск~2\ \ \ 2007
 \hfill \textbf{\thepage}}}
 \label{end\stat} 


%\end{document}

%
\def\stat{rekl}
%\label{preobr}

%\def\tit{АКАДЕМИК ПУГАЧЁВ  ВЛАДИМИР СЕМЁНОВИЧ\\
%25.03.1911--25.03.1998}


%   \vspace*{-48pt}
%   \begin{center}\LARGE
%Академик Пугачёв  Владимир Семёнович\\ (25.03.1911--25.03.1998)
%   \end{center}

   %\vspace*{2.5mm}

   \begin{center}

{\prgsh\LARGE
ЮБИЛЕИ}

\end{center}
%\hrule

\vspace*{6pt}


   \vspace*{8mm}

   \thispagestyle{empty}


%\def\stat{emel}


\section*{К 70-летию заместителя директора ИПИ РАН,\\ члена редколлегии журнала
<<Информатика и её применения>>\\ доктора технических наук В.\,И.~Будзко}

\vspace*{18pt}




          \begin{multicols}{2}

%            \label{st\stat}

\begin{center}
\vspace*{1pt}
\mbox{%
\epsfxsize=78mm
\epsfbox{bud-1.eps}
}
\end{center}

\vspace*{12pt}

      14 августа 2014~г.\ исполнилось 70~лет за\-мес\-ти\-те\-лю директора ИПИ РАН по
научной работе доктору технических наук Владимиру Игоревичу Будзко.

      Владимир Игоревич Будзко родился в г.~Москве. Высшее образование получил на факультете
элект\-рон\-но-вы\-чис\-ли\-тель\-ных устройств в Московском
ин\-же\-нер\-но-фи\-зи\-че\-ском институте
(МИФИ), который он окончил в 1968~г., после чего был на\-прав\-лен для прохождения
службы в одну из войс\-ко\-вых частей, где прошел путь от инженера до первого заместителя
командира войсковой части.

      С приходом В.\,И.~Будзко в ИПИ РАН (2001~г.)\ в институте
сформировалось новое научное на\-прав\-ле\-ние теоретических исследований~--- <<Постро\-ение
ин\-фор\-ма\-ци\-он\-но-те\-ле\-ком\-му\-ни\-ка\-ци\-он\-ных\linebreak сис\-тем
высокой до\-ступ\-ности>>. В~рамках этого
направления выполнен широкий круг фундаментальных исследований по поиску подходов и
определению принципов построения средств обеспечения доступности, конфиденциальности
и целостности современных крупномасштабных
ин\-фор\-ма\-ци\-он\-но-те\-ле\-ком\-му\-ни\-ка\-ци\-он\-ных
сис\-тем (ИТС). Разработаны основные сис\-тем\-но-тех\-ни\-че\-ские принципы и базовые
архитектурные решения построения перспективных для условий России ИТС с
централизованной обработкой и хранением информации, сочетающих в себе свойства
высокой доступности, отказо- и катастрофоустойчивости, информационной защищенности.
Определены принципы, методы и математические основы рационального построения и
оптимизации средств восстановления функционирования центров обработки данных (ЦОД)
после возникновения отказов и катастроф, передачи и хранения данных, обеспечения
информационной безопасности при достижении минимальной совокупной стоимости
владения такими системами. Результаты нашли практическое воплощение при реализации
проектов в интересах ряда отечественных государственных и негосударственных
организаций, таких как Банк России (БР), Внешторгбанк, ОАО <<ГМК <<Норильский Никель>>,
<<Газпром>>, Минэкономразвития России, Правительство Москвы, а также ряд силовых
ведомств.

      Под руководством В.\,И.~Будзко начиная с 2001~г.\ выполнен комплекс
      на\-уч\-но-ис\-сле\-до\-ва\-тель\-ских и
      опыт\-но-кон\-ст\-рук\-тор\-ских работ (свыше 100~проектов),
направленных на развитие электронной информационной технологии БР.
Разработаны концепции развития ИТС БР сначала до 2008~г., а затем до 2013~г., которые
были приняты в качестве основы проведения технической политики. За реализацию проекта
<<Катастрофоустойчивая тер\-ри\-то\-ри\-аль\-но-рас\-пре\-де\-лен\-ная
      ин\-фор\-ма\-ци\-он\-но-те\-ле\-ком\-му\-ни\-ка\-ци\-он\-ная сис\-те\-ма централизованной
обработки банковской информации>> В.\,И.~Будзко удостоен Премии Правительства РФ в
области науки и техники за 2010~г.

      В.\,И.~Будзко возглавлял и возглавляет работы по ряду других прикладных проектов,
связанных с созданием, совершенствованием и развитием крупномасштабных ИТС.

      В.\,И.~Будзко~--- генерал-майор, доктор технических наук, член-кор\-рес\-пон\-дент
Академии криптографии РФ, известный ученый в области информатики и применения
информационных технологий при построении территориально распределенных ИТС
различного назначения. Является автором свыше 250~научных работ, опубликованных в
на\-уч\-но-тех\-ни\-че\-ских и специальных изданиях.

    \thispagestyle{empty}

      В.\,И.~Будзко уделяет большое внимание подготовке научных кадров. Под его
руководством защищено 6~диссертаций на соискание ученой степени кандидата
технических наук. Свыше 30~лет он читает лекции в ИКСИ Академии ФСБ, профессор
кафедры НИЯУ МИФИ. Является членом двух диссертационных советов, главным
редактором журнала <<Системы высокой доступности>> и членом редколлегии журнала
<<Информатика и её применения>>.

      \bigskip

      Редакционный совет и Редакционная коллегия журнала <<Информатика и её
применения>> сердечно поздравляют Владимира Игоревича Будзко с 70-ле\-ти\-ем и желают
крепкого здоровья и новых научных достижений.

\end{multicols}

%Информатика и её применения
%Том 14 Выпуск 1-4 Год 2020

\def\stat{cont}
{%\hrule\par
%\vskip 7pt % 7pt
\raggedleft\Large \bf%\baselineskip=3.2ex
А\,В\,Т\,О\,Р\,С\,К\,И\,Й\ \ У\,К\,А\,З\,А\,Т\,Е\,Л\,Ь\ \ З\,А\ \ 2\,0\,2\,0 г. \vskip 17pt
 \hrule
 \par
\vskip 21pt plus 6pt minus 3pt }

\label{st\stat}

\def\tit{\ }

\def\aut{\ }
\def\auf{\ }

\def\leftkol{\ } % ENGLISH ABSTRACTS}

\def\rightkol{\ } %АВТОРСКИЙ УКАЗАТЕЛЬ ЗА 2020 г.} %ENGLISH ABSTRACTS}

\titele{\tit}{\aut}{\auf}{\leftkol}{\rightkol}
\addcontentsline{toc}{subsection}{\textrm\textbf Авторский указатель за 2020 г.}

\vspace*{-24pt}

\noindent
{\tabcolsep=3pt
\begin{tabular}{p{397pt}cc}
&\textbf{Вып.} & \textbf{Стр.}\\[6pt]
\Avtors{Абгарян~К.\,К., Гаврилов~Е.\,С.} Интеграционная платформа для многомасштабного моде-\linebreak
\\[-12pt]
\hspace*{23pt}лирования нейроморфных систем&2&104--110\\
\Avtors{Абгарян~К.\,К., Колбин~И.\,С.} Применение многомасштабного подхода и методов анализа\linebreak
\\[-12pt]
\hspace*{23pt}данных для моделирования теплопроводности в слоистых структурах&4&91--99\\
\Avtors{Агаларов~Я.\,М.} Оптимизация емкости основного накопителя в системе массового\linebreak
\\[-12pt]
\hspace*{23pt}обслуживания типа $G/M/1/K$ с дополнительным накопителем&2&72--79\\
\Avtors{Агасандян~Г.\,А.} Вычислительные аспекты применения CC-VaR на совокупности рынков&3&62--70\\
\Avtors{Агеев~К.\,А., Сопин~Э.\,С., Яркина~Н.\,В., Самуйлов~К.\,Е., Шоргин~С.\,Я.} Анализ механизмов\linebreak
\\[-12pt]
\hspace*{23pt}нарезки сети с учетом гарантий для различных типов трафика&3&\hphantom{1}94--100\\
\Avtors{Адамова~К.\,А.} см.\ Шнурков~П.\,В.&&\\
\Avtors{Базилевский~М.\,П.} Многофакторные модели полносвязной линейной регрессии без\linebreak
\\[-12pt]
\hspace*{23pt}ограничений на соотношения дисперсий ошибок переменных&2&92--97\\
\Avtors{Бахтеев~О.\,Ю.} см.\ Грабовой~А.\,В.&&\\
\Avtors{Беленков~В.\,Г.} см.\ Будзко~В.\,И.&&\\
\Avtors{Бетелин~В.\,Б., Кушниренко~А.\,Г., Леонов~А.\,Г.} Основные понятия программирования\linebreak
\\[-12pt]
\hspace*{23pt}в изложении для дошкольников&3&55--61\\
\Avtors{Бетелин~В.\,Б., Кушниренко~А.\,Г., Семенов~А.\,Л., Сопрунов~С.\,Ф.} О цифровой грамотности\linebreak
\\[-12pt]
\hspace*{23pt}и средах ее формирования&4&100--107\\
\Avtors{Борисов~А.\,В.} Численные схемы фильтрации марковских скачкообразных процессов по\linebreak
\\[-12pt]
\hspace*{23pt}дискретизованным наблюдениям II: случай аддитивных шумов&1&17--23\\
\Avtors{Борисов~А.\,В.} Численные схемы фильтрации марковских скачкообразных процессов по\linebreak
\\[-12pt]
\hspace*{23pt}дискретизованным наблюдениям III: случай мультипликативных шумов&2&10--18\\
\Avtors{Босов~А.\,В.} Управление выходом стохастической дифференциальной системы по квад-\linebreak
\\[-12pt]
\hspace*{23pt}ратичному критерию. V. Случай неполной информации о состоянии&2&19--25\\
\Avtors{Босов~А.\,В., Мартюшова~Я.\,Г., Наумов~А.\,В., Сапунова~А.\,П.} Байесовский подход к~по\-стро\-ению индивидуальной траектории пользователя в~системе дистанционного\linebreak
\\[-12pt]
\hspace*{23pt}обучения&3&86--93\\
\Avtors{Босов~А.\,В., Стефанович~А.\,И.} Управление выходом стохастической дифференциальной\linebreak
\\[-12pt]
\hspace*{23pt}системы по квадратичному критерию. IV. Альтернативное численное решение&1&24--30\\
\Avtors{Брюхов~Д.\,О., Ступников~С.\,А., Ковалёв~Д.\,Ю., Шанин~И.\,А.} Нейрофизиология как\linebreak
\\[-12pt]
\hspace*{23pt}предметная область для решения задач с интенсивным использованием данных&1&40--47\\
\Avtors{Будзко~В.\,И., Ядринцев~В.\,В., Соченков~И.\,В., Королёв~В.\,И., Беленков~В.\,Г.} Об одном подходе
 к формированию в условиях высокой неопределенности марке-\linebreak
\\[-12pt]
\hspace*{23pt}ров конфиденциальности в системах интенсивного использования данных&4&69--76\\
\Avtors{Вайсер~К.\,О.} см.\ Потанин~М.\,С.&&\\
\Avtors{Вохминцев~А.\,В., Мельников~А.\,В., Пачганов~C.\,А.} Метод навигации и составления карты в трехмерном пространстве на основе комбинированного решения вариационной\linebreak
\\[-12pt]
\hspace*{23pt}подзадачи точка--точка ICP для аффинных преобразований&1&101--112\\
\Avtors{Гаврилов~Е.\,С.} см.\ Абгарян~К.\,К.&&\\
\Avtors{Гайдамака~Ю.\,В.} см.\  Москалева~Ф.\,А.&&\\
\Avtors{Голембиовский~Д.\,Ю.} см.\ Данилишин~А.\,Р.&&\\
\Avtors{Голембиовский~Д.\,Ю.} см.\ Данилишин~А.\,Р.&&\\
\Avtors{Гончаров~А.\,А., Зацман~И.\,М., Кружков~М.\,Г.} Эволюция классификаций в надкорпусных\linebreak
\\[-12pt]
\hspace*{23pt}базах данных&4&108--116\\
\Avtors{Гончаров~А.\,В., Стрижов~В.\,В.} Выравнивание декартовых произведений упорядоченных\linebreak
\\[-12pt]
\hspace*{23pt}множеств&1&31--39\\
\end{tabular}
}

\pagebreak

\def\leftkol{АВТОРСКИЙ УКАЗАТЕЛЬ ЗА 2020 г.} % ENGLISH ABSTRACTS}

\def\rightkol{АВТОРСКИЙ УКАЗАТЕЛЬ ЗА 2020 г.} %ENGLISH ABSTRACTS}

%\thispagestyle{myheadings}
\def\leftfootline{\small{\textbf{\thepage}
\hfill ИНФОРМАТИКА И ЕЁ ПРИМЕНЕНИЯ\ \ \ том~14\ \ \ выпуск~4\ \ \ 2020}
}%
 \def\rightfootline{\small{ИНФОРМАТИКА И ЕЁ ПРИМЕНЕНИЯ\ \ \ том~14\ \ \ выпуск~4\ \ \ 2020
 \hfill \textbf{\thepage}}}


\noindent
{\tabcolsep=3pt
\begin{tabular}{p{394pt}cc}
&\textbf{Вып.} & \textbf{Стр.}\\[3pt]
\Avtors{Горшенин~А.\,К., Королев~В.\,Ю.} Аппроксимация распределений размеров частиц лунного\linebreak
\\[-12pt]
\hspace*{23pt}реголита на основе метода статистической симуляции выборок&2&50--57\\
\Avtors{Горшенин~А.\,К., Королев~В.\,Ю., Щербинина~А.\,А.} Статистическое оценивание распределений случайных коэффициентов стохастического дифференциального уравнения\linebreak
\\[-12pt]
\hspace*{23pt}Ланжевена&3&\hphantom{1}3--12\\
\Avtors{Горшенин~А.\,К., Кузьмин~В.\,Ю.} Анализ конфигураций LSTM-сетей для построения\linebreak
\\[-12pt]
\hspace*{23pt}среднесрочных векторных прогнозов&1&10--16\\
\Avtors{Грабовой~А.\,В., Бахтеев~О.\,Ю., Стрижов~В.\,В.} Введение отношения порядка на множестве\linebreak
\\[-12pt]
\hspace*{23pt}параметров аппроксимирующих моделей&2&58--65\\
\Avtors{Грушо~А.\,А., Забежайло~М.\,И., Смирнов~Д.\,В., Тимонина~Е.\,Е.} О вероятностных оценках\linebreak
\\[-12pt]
\hspace*{23pt}достоверности эмпирических выводов&4&3--8\\
\Avtors{Грушо~А.\,А., Забежайло~М.\,И., Смирнов~Д.\,В., Тимонина~Е.\,Е., Шоргин~С.\,Я.} Методы\linebreak
\\[-12pt]
\hspace*{23pt}математической статистики в задаче поиска инсайдера&3&71--75\\
\Avtors{Грушо~А.\,А., Забежайло~М.\,И., Тимонина~Е.\,Е.} О каузальной репрезентативности обуча-\linebreak
\\[-12pt]
\hspace*{23pt}ющих выборок прецедентов в задачах диагностического типа&1&80--86\\
\Avtors{Грушо~А.\,А., Тимонина~Е.\,Е., Грушо~Н.\,А., Терехина~И.\,Ю.} Выявление аномалий с по-\linebreak
\\[-12pt]
\hspace*{23pt}мощью метаданных&3&76--80\\
\Avtors{Грушо~А.\,А.} см.\ Грушо~Н.\,А.&&\\
\Avtors{Грушо~Н.\,А., Грушо~А.\,А., Забежайло~М.\,И., Тимонина~Е.\,Е.} Методы нахождения причин\linebreak
\\[-12pt]
\hspace*{23pt}сбоев в информационных технологиях  с помощью метаданных&2&33--39\\
\Avtors{Грушо~Н.\,А.} см.\ Грушо~А.\,А.&&\\
\Avtors{Данилишин~А.\,Р., Голембиовский~Д.\,Ю.} Оценка стоимости опционов на основе моделей\linebreak
\\[-12pt]
\hspace*{23pt}ARIMA--GARCH с ошибками, распределенными по закону $S_u$ Джонсона&4&83--90\\
\Avtors{Данилишин~А.\,Р., Голембиовский~Д.\,Ю.} Риск-нейтральная динамика для модели ARIMA-\linebreak
\\[-12pt]
\hspace*{23pt}GARCH с ошибками, распределенными по закону $S_U$ Джонсона&1&48--55\\
\Avtors{Диментов~А.\,В.} см.\ Краснов~Ф.\,В.&&\\
\Avtors{Донской~В.\,И.} Извлечение оптимизационных моделей из данных&3&109--118\\
\Avtors{Дубнов~Ю.\,А.} см.\ Попков~Ю.\,С.&&\\
\Avtors{Дулин~С.\,К., Дулина~Н.\,Г., Ермаков~П.\,В.} Информационный синтез документов&1&128--135\\
\Avtors{Дулина~Н.\,Г.} см.\ Дулин~С.\,К.&&\\
\Avtors{Дьяченко~Ю.\,Г.} см.\ Соколов~И.\,А.&&\\
\Avtors{Ермаков~П.\,В.} см.\ Дулин~С.\,К.&&\\
\Avtors{Ефросинин~Д.\,В.} см.\ Харин~П.\,А.&&\\
\Avtors{Жолобов~В.\,А.} см.\ Потанин~М.\,С.&&\\
\Avtors{Забежайло~М.\,И.} см.\ Грушо~А.\,А.&&\\
\Avtors{Забежайло~М.\,И.} см.\ Грушо~А.\,А.&&\\
\Avtors{Забежайло~М.\,И.} см.\ Грушо~А.\,А.&&\\
\Avtors{Забежайло~М.\,И.} см.\ Грушо~Н.\,А.&&\\
\Avtors{Захаров В. Н.} см.\ Френкель С. Л.&&\\
\Avtors{Зацман~И.\,М.} Проблемно-ориентированная верификация полноты темпоральных\linebreak
\\[-12pt]
\hspace*{23pt}онтологий и заполнение понятийных лакун&3&119--128\\
\Avtors{Зацман~И.\,М.} см.\ Гончаров~А.\,А.&&\\
\Avtors{Зацман~И.\,М.} см.\ Нуриев~В.\,А.&&\\
\Avtors{Зейфман~А.\,И.} см.\ Сатин~Я.\,А.&&\\
\Avtors{Кириков~И.\,А.} см.\ Румовская~С.\,Б.&&\\
\Avtors{Кирилюк~И.\,Л., Сенько~О.\,В.} Выбор моделей оптимальной сложности методами Монте-Карло (на примере моделей производственных функций регионов Российской\linebreak
\\[-12pt]
\hspace*{23pt}Федерации)&2&111--118\\
\Avtors{Ковалёв~Д.\,Ю.} см.\ Брюхов~Д.\,О.&&\\
\Avtors{Козеренко~Е.\,Б., Михеев~М.\,Ю., Сомин~Н.\,В., Эрлих~Л.\,И., Кузнецов~К.\,И.} Аналити\-че\-ская
текс\-тология в системах интеллектуальной обработки неструктурированных\linebreak
\\[-12pt]
\hspace*{23pt}данных&1&113--120\\
\Avtors{Колбин~И.\,С.} см.\ Абгарян~К.\,К.&&\\
\end{tabular}
}

\pagebreak

\def\leftkol{АВТОРСКИЙ УКАЗАТЕЛЬ ЗА 2020 г.} % ENGLISH ABSTRACTS}

\def\rightkol{АВТОРСКИЙ УКАЗАТЕЛЬ ЗА 2020 г.} %ENGLISH ABSTRACTS}

%\thispagestyle{myheadings}
\def\leftfootline{\small{\textbf{\thepage}
\hfill ИНФОРМАТИКА И ЕЁ ПРИМЕНЕНИЯ\ \ \ том~14\ \ \ выпуск~4\ \ \ 2020}
}%
 \def\rightfootline{\small{ИНФОРМАТИКА И ЕЁ ПРИМЕНЕНИЯ\ \ \ том~14\ \ \ выпуск~4\ \ \ 2020
 \hfill \textbf{\thepage}}}


\noindent
{\tabcolsep=3pt
\begin{tabular}{p{394pt}cc}
&\textbf{Вып.} & \textbf{Стр.}\\[3pt]
\Avtors{Королев~В.\,Ю.} О распределении отношения суммы элементов выборки, превосходящих\linebreak
\\[-12pt]
\hspace*{23pt}некоторый порог, к сумме всех элементов выборки.~I&3&35--43\\
\Avtors{Королев~В.\,Ю.} О распределении отношения суммы элементов выборки, превосходящих\linebreak
\\[-12pt]
\hspace*{23pt}некоторый порог, к сумме всех элементов выборки.~II&4&33--36\\
\Avtors{Королев~В.\,Ю.} см.\ Горшенин~А.\,К&&\\
\Avtors{Королев~В.\,Ю.} см.\ Горшенин~А.\,К.&&\\
\Avtors{Королёв~В.\,И.} см.\ Будзко~В.\,И.&&\\
\Avtors{Костина~А.\,А., Мирин~А.\,Ю., Молдовян~Д.\,Н., Фахрутдинов~Р.\,Ш.} Метод задания конечных некоммутативных ассоциативных алгебр произвольной четной размерности\linebreak
\\[-12pt]
\hspace*{23pt}для построения постквантовых криптосхем&1&\hphantom{1}94--100\\
\Avtors{Кочеткова~И.\,А.} см.\ Харин~П.\,А.&&\\
\Avtors{Краснов~Ф.\,В., Диментов~А.\,В., Шварцман~М.\,Е.} Использование тематических моделей\linebreak
\\[-12pt]
\hspace*{23pt}для парного сравнения  коллекций научных статей&3&129--135\\
\Avtors{Кривенко~М.\,П.} Последовательный анализ серий данных на основе многомерных ре-\linebreak
\\[-12pt]
\hspace*{23pt}фе\-рен\-с\-ных регионов&2&86--91\\
\Avtors{Кружков~М.\,Г.} см.\ Гончаров~А.\,А.&&\\
\Avtors{Кудрявцев~А.\,А., Шестаков~О.\,В.} Метод логарифмических моментов для оценивания\linebreak
\\[-12pt]
\hspace*{23pt}параметров гамма-экспоненциального распределения&3&49--54\\
\Avtors{Кузнецов~К.\,И.} см.\ Козеренко~Е.\,Б.&&\\
\Avtors{Кузьмин~В.\,Ю.} см.\ Горшенин~А.\,К.&&\\
\Avtors{Кушниренко~А.\,Г.} см.\ Бетелин~В.\,Б.&&\\
\Avtors{Кушниренко~А.\,Г.} см.\ Бетелин~В.\,Б.&&\\
\Avtors{Леонов~А.\,Г.} см.\ Бетелин~В.\,Б.&&\\
\Avtors{Макеева~Е.\,Д.} см.\ Харин~П.\,А.&&\\
\Avtors{Малашенко~Ю.\,Е., Назарова~И.\,А.} Аппроксимация множества достижимых потоков\linebreak
\\[-12pt]
\hspace*{23pt}многопользовательской сети&3&81--85\\
\Avtors{Мартюшова~Я.\,Г.} см.\ Босов~А.\,В.&&\\
\Avtors{Матюшенко~С.\,И., Разумчик~Р.\,В.} Стационарные характеристики системы Geo$/G/1/\infty $\linebreak
\\[-12pt]
\hspace*{23pt}с неординарным входящим потоком, управляющим размером очереди&4&25--32\\
\Avtors{Мейханаджян~Л.\,А., Разумчик~Р.\,В.} Стационарные характеристики системы $M/G/2/\infty$ с одним частным случаем дисциплины инверсионного порядка обслуживания\linebreak
\\[-12pt]
\hspace*{23pt}с обобщенным  вероятностным приоритетом&2&66--71\\
\Avtors{Мельников~А.\,В.} см.\ Вохминцев~А.\,В.&&\\
\Avtors{Мельников~С.\,Ю., Самуйлов~К.\,Е.} Статистические свойства двоичных неавтономных\linebreak
\\[-12pt]
\hspace*{23pt}регистров сдвига  с внутренним суммированием&2&80--85\\
\Avtors{Милованова~Т.\,А., Разумчик~Р.\,В.} Однолинейная система массового обслуживания с инверсионным порядком обслуживания с вероятностным приоритетом, групповым\linebreak
\\[-12pt]
\hspace*{23pt}пуассоновским потоком и фоновыми заявками&3&26--34\\
\Avtors{Мирин~А.\,Ю.} см.\ Костина~А.\,А.&&\\
\Avtors{Михеев~М.\,Ю.} см.\ Козеренко~Е.\,Б.&&\\
\Avtors{Молдовян~Д.\,Н.} см.\ Костина~А.\,А.&&\\
\Avtors{Москалева~Ф.\,А., Гайдамака~Ю.\,В., Шоргин~В.\,С.} Влияние параметров изоляции на\linebreak
\\[-12pt]
\hspace*{23pt}разделение ресурсов при нарезке сети&4&\hphantom{1}9--16\\
\Avtors{Назарова~И.\,А.} см.\ Малашенко~Ю.\,Е.&&\\
\Avtors{Наумов~А.\,В.} см.\ Босов~А.\,В.&&\\
\Avtors{Наумов~В.\,А., Самуйлов~К.\,Е.} О марковских и рациональных потоках случайных со-\linebreak
\\[-12pt]
\hspace*{23pt}бытий.~I&3&13--19\\
\Avtors{Наумов~В.\,А., Самуйлов~К.\,Е.} О марковских и рациональных потоках случайных со-\linebreak
\\[-12pt]
\hspace*{23pt}бытий.~II&4&37--46\\
\Avtors{Новиков~Д.\,А.} см.\ Шнурков~П.\,В.&&\\
\Avtors{Нуриев~В.\,А., Зацман~И.\,М.} Редуцирование спектра моделей перевода в надкорпусных\linebreak
\\[-12pt]
\hspace*{23pt}базах данных&2&119--126\\
\Avtors{Пачганов~C.\,А.} см.\ Вохминцев~А.\,В.&&\\
\end{tabular}
}

\pagebreak

\def\leftkol{АВТОРСКИЙ УКАЗАТЕЛЬ ЗА 2020 г.} % ENGLISH ABSTRACTS}

\def\rightkol{АВТОРСКИЙ УКАЗАТЕЛЬ ЗА 2020 г.} %ENGLISH ABSTRACTS}

%\thispagestyle{myheadings}
\def\leftfootline{\small{\textbf{\thepage}
\hfill ИНФОРМАТИКА И ЕЁ ПРИМЕНЕНИЯ\ \ \ том~14\ \ \ выпуск~4\ \ \ 2020}
}%
 \def\rightfootline{\small{ИНФОРМАТИКА И ЕЁ ПРИМЕНЕНИЯ\ \ \ том~14\ \ \ выпуск~4\ \ \ 2020
 \hfill \textbf{\thepage}}}


\noindent
{\tabcolsep=3pt
\begin{tabular}{p{394pt}cc}
&\textbf{Вып.} & \textbf{Стр.}\\[3pt]
\Avtors{Попков~А.\,Ю.} см.\ Попков~Ю.\,С.&&\\
\Avtors{Попков~Ю.\,С., Попков~А.\,Ю., Дубнов~Ю.\,А.} Методы детерминированных и рандомизи-\linebreak
\\[-12pt]
\hspace*{23pt}рованных энтропийных проекций для редукции размерности матрицы данных&4&47--54\\
\Avtors{Попов~Г.\,А., Симаворян~С.\,Ж., Симонян~А.\,Р., Улитина~Е.\,И.} Моделирование процесса мониторинга систем информационной безопасности на основе систем массового\linebreak
\\[-12pt]
\hspace*{23pt}обслуживания&1&71--79\\
\Avtors{Попов~М.\,В., Посыпкин~М.\,А.} Аппроксимация множества решений систем нелинейных\linebreak
\\[-12pt]
\hspace*{23pt}неравенств с использованием графических ускорителей&3&20--25\\
\Avtors{Посыпкин~М.\,А.} см.\ Попов~М.\,В.&&\\
\Avtors{Потанин~М.\,С., Вайсер~К.\,О., Жолобов~В.\,А., Стрижов~В.\,В.} Оптимизация структуры\linebreak
\\[-12pt]
\hspace*{23pt}сетей глубокого обучения&4&55--62\\
\Avtors{Разумчик~Р.\,В.} см.\ Матюшенко~С.\,И.&&\\
\Avtors{Разумчик~Р.\,В.} см.\ Мейханаджян~Л.\,А.&&\\
\Avtors{Разумчик~Р.\,В.} см.\ Милованова~Т.\,А.&&\\
\Avtors{Рождественский~Ю.\,В.} см.\ Соколов~И.\,А.&&\\
\Avtors{Румовская~С.\,Б., Кириков~И.\,А.} Метод визуального представления конфликтов в гибрид-\linebreak
\\[-12pt]
\hspace*{23pt}ных интеллектуальных многоагентных системах&4&77--82\\
\Avtors{Самуйлов~К.\,Е.} см.\ Агеев~К.\,А.&&\\
\Avtors{Самуйлов~К.\,Е.} см.\ Мельников~С.\,Ю.&&\\
\Avtors{Самуйлов~К.\,Е.} см.\ Наумов~В.\,А.&&\\
\Avtors{Самуйлов~К.\,Е.} см.\ Наумов~В.\,А.&&\\
\Avtors{Сапунова~А.\,П.} см.\ Босов~А.\,В.&&\\
\Avtors{Сатин~Я.\,А., Зейфман~А.\,И., Шилова~Г.\,Н.} О подходах к построению предельных режимов\linebreak
\\[-12pt]
\hspace*{23pt}для некоторых моделей массового обслуживания&2&3--9\\
\Avtors{Севастьянов~Л.\,А., Щетинин~Е.\,Ю.} О методах повышения точности многоклассовой\linebreak
\\[-12pt]
\hspace*{23pt}классификации на несбалансированных данных&1&63--70\\
\Avtors{Семенов~А.\,Л.} см.\ Бетелин~В.\,Б.&&\\
\Avtors{Сенько~О.\,В.} см.\ Кирилюк~И.\,Л.&&\\
\Avtors{Серебрянский~С.\,М., Тырсин~А.\,Н.} Повышение точности решения обратных задач за\linebreak
\\[-12pt]
\hspace*{23pt}счет уточнения граничных условий&1&56--62\\
\Avtors{Симаворян~С.\,Ж.} см.\ Попов~Г.\,А.&&\\
\Avtors{Симонян~А.\,Р.} см.\ Попов~Г.\,А.&&\\
\Avtors{Смирнов~Д.\,В.} см.\ Грушо~А.\,А.&&\\
\Avtors{Смирнов~Д.\,В.} см.\ Грушо~А.\,А.&&\\
\Avtors{Соколов~И.\,А., Степченков~Ю.\,А., Дьяченко~Ю.\,Г., Рождественский~Ю.\,В.} Повышение\linebreak
\\[-12pt]
\hspace*{23pt}сбоеустойчивости самосинхронных схем&4&63--68\\
\Avtors{Сомин~Н.\,В.} см.\ Козеренко~Е.\,Б.&&\\
\Avtors{Сопин~Э.\,С.} см.\ Агеев~К.\,А.&&\\
\Avtors{Сопрунов~С.\,Ф.} см.\ Бетелин~В.\,Б.&&\\
\Avtors{Соченков~И.\,В.} см.\ Будзко~В.\,И.&&\\
\Avtors{Степченков~Ю.\,А.} см.\ Соколов~И.\,А.&&\\
\Avtors{Стефанович~А.\,И.} см.\ Босов~А.\,В.&&\\
\Avtors{Стрижов~В.\,В.} см.\ Гончаров~А.\,В.&&\\
\Avtors{Стрижов~В.\,В.} см.\ Грабовой~А.\,В.&&\\
\Avtors{Стрижов~В.\,В.} см.\ Потанин~М.\,С.&&\\
\Avtors{Ступников~С.\,А.} см.\ Брюхов~Д.\,О.&&\\
\Avtors{Терехина~И.\,Ю.} см.\ Грушо~А.\,А.&&\\
\Avtors{Тимонина~Е.\,Е.} см.\  Грушо~А.\,А.&&\\
\Avtors{Тимонина~Е.\,Е.} см.\ Грушо~А.\,А.&&\\
\Avtors{Тимонина~Е.\,Е.} см.\ Грушо~А.\,А.&&\\
\Avtors{Тимонина~Е.\,Е.} см.\ Грушо~А.\,А.&&\\
\Avtors{Тимонина~Е.\,Е.} см.\ Грушо~Н.\,А.&&\\
\Avtors{Тырсин~А.\,Н.} см.\ Серебрянский~С.\,М.&&\\
\Avtors{Улитина~Е.\,И.} см.\ Попов~Г.\,А.&&\\
\end{tabular}
}

\pagebreak

\def\leftkol{АВТОРСКИЙ УКАЗАТЕЛЬ ЗА 2020 г.} % ENGLISH ABSTRACTS}

\def\rightkol{АВТОРСКИЙ УКАЗАТЕЛЬ ЗА 2020 г.} %ENGLISH ABSTRACTS}

%\thispagestyle{myheadings}
\def\leftfootline{\small{\textbf{\thepage}
\hfill ИНФОРМАТИКА И ЕЁ ПРИМЕНЕНИЯ\ \ \ том~14\ \ \ выпуск~4\ \ \ 2020}
}%
 \def\rightfootline{\small{ИНФОРМАТИКА И ЕЁ ПРИМЕНЕНИЯ\ \ \ том~14\ \ \ выпуск~4\ \ \ 2020
 \hfill \textbf{\thepage}}}


\noindent
{\tabcolsep=3pt
\begin{tabular}{p{394pt}cc}
&\textbf{Вып.} & \textbf{Стр.}\\[3pt]
\Avtors{Фахрутдинов~Р.\,Ш.} см.\ Костина~А.\,А.&&\\
\Avtors{Френкель С. Л., Захаров В. Н.} Совместная оценка предсказуемости данных и качества\linebreak
\\[-12pt]
\hspace*{23pt}предикторов&2&40--49\\
\Avtors{Харин~П.\,А., Макеева~Е.\,Д., Кочеткова~И.\,А., Ефросинин~Д.\,В., Шоргин~С.\,Я.} 
Система массового обслуживания с орбитами для анализа совместного обслуживания трафика 
с малыми задержками URLLC и~широкополосного доступа eMBB в~беспроводных\linebreak
\\[-12pt]
\hspace*{23pt}сетях пятого поколения&4&17--24\\
\Avtors{Хусаинов~А.\,А.} Производительность ограниченного конвейера&1&87--93\\
\Avtors{Шанин~И.\,А.} см.\ Брюхов~Д.\,О.&&\\
\Avtors{Шварцман~М.\,Е.} см.\ Краснов~Ф.\,В.&&\\
\Avtors{Шестаков~О.\,В.} Асимптотика оценки среднеквадратичного риска в задаче обращения\linebreak
\\[-12pt]
\hspace*{23pt}преобразования Радона по проекциям, регистрируемым на случайной сетке&2&26--32\\
\Avtors{Шестаков~О.\,В.} Асимптотическая регулярность вейвлет-методов обращения линейных однородных операторов по наблюдениям, регистрируемым в случайные моменты\linebreak
\\[-12pt]
\hspace*{23pt}времени&1&3--9\\
\Avtors{Шестаков~О.\,В.} О статистических свойствах оценки риска в задаче обращения преобра-\linebreak
\\[-12pt]
\hspace*{23pt}зования Радона при случайном объеме проекционных данных&3&44--48\\
\Avtors{Шестаков~О.\,В.} см.\ Кудрявцев~А.\,А.&&\\
\Avtors{Шилова~Г.\,Н.} см.\ Сатин~Я.\,А.&&\\
\Avtors{Шихиев~Ф.\,Ш.} см.\ Шихиев~Ш.\,Б.&&\\
\Avtors{Шихиев~Ш.\,Б., Шихиев~Ф.\,Ш.} Инкапсуляция семантических представлений в элементы\linebreak
\\[-12pt]
\hspace*{23pt}грамматики&1&121--127\\
\Avtors{Шнурков~П.\,В., Адамова~К.\,А.} Решение задачи безусловного экстремума для дробно-\linebreak
\\[-12pt]
\hspace*{23pt}линейного интегрального функционала, зависящего от параметра&2&\hphantom{1}98--103\\
\Avtors{Шнурков~П.\,В., Новиков~Д.\,А.} О концепции стохастической модели с управлением в~моменты выхода процесса на границу заданного подмножества множества\linebreak
\\[-12pt]
\hspace*{23pt}состояний&3&101--108\\
\Avtors{Шоргин~В.\,С.} см.\ Москалева~Ф.\,А.&&\\
\Avtors{Шоргин~С.\,Я.} см.\ Агеев~К.\,А.&&\\
\Avtors{Шоргин~С.\,Я.} см.\ Грушо~А.\,А.&&\\
\Avtors{Шоргин~С.\,Я.} см.\ Харин~П.\,А.&&\\
\Avtors{Щербинина~А.\,А.} см.\ Горшенин~А.\,К.&&\\
\Avtors{Щетинин~Е.\,Ю.} см.\ Севастьянов~Л.\,А.&&\\
\Avtors{Эрлих~Л.\,И.} см.\ Козеренко~Е.\,Б.&&\\
\Avtors{Ядринцев~В.\,В.} см.\ Будзко~В.\,И.&&\\
\Avtors{Яркина~Н.\,В.} см.\ Агеев~К.\,А.&&\\
\end{tabular}
}

%\thispagestyle{myheadings}
\def\leftfootline{\small{\textbf{\thepage}
\hfill ИНФОРМАТИКА И ЕЁ ПРИМЕНЕНИЯ\ \ \ том~14\ \ \ выпуск~4\ \ \ 2020}
}%
 \def\rightfootline{\small{ИНФОРМАТИКА И ЕЁ ПРИМЕНЕНИЯ\ \ \ том~14\ \ \ выпуск~4\ \ \ 2020
 \hfill \textbf{\thepage}}}

 \label{end\stat}

\newpage

\def\stat{cont-e}
{%\hrule\par
%\vskip 7pt % 7pt
\raggedleft\Large \bf%\baselineskip=3.2ex
2\,0\,2\,0\ \ A\,U\,T\,H\,O\,R\ \ I\,N\,D\,E\,X \vskip 17pt
 \hrule
 \par
\vskip 21pt plus 6pt minus 3pt }

\label{st\stat}

\def\tit{\ }

\def\aut{\ }
\def\auf{\ }

\def\leftkol{\ } %2020 AUTHOR INDEX} % ENGLISH ABSTRACTS}

\def\rightkol{\ } %2020 AUTHOR INDEX} %ENGLISH ABSTRACTS}

\titele{\tit}{\aut}{\auf}{\leftkol}{\rightkol}
\addcontentsline{toc}{subsection}{\textrm\textbf 2020 Author Index}

\def\leftfootline{\small{\textbf{\thepage}
\hfill INFORMATIKA I EE PRIMENENIYA~--- INFORMATICS AND APPLICATIONS\ \ \ 2020\
\ \ volume~14\ \ \ issue\ 4}
}%
 \def\rightfootline{\small{INFORMATIKA I EE PRIMENENIYA~--- INFORMATICS AND APPLICATIONS\ \ \ 2020\ \ \ volume~14\ \ \ issue\ 4
\hfill \textbf{\thepage}}}

\vspace*{-24pt}

\noindent
{\tabcolsep=3pt
\begin{tabular}{p{395.89pt}cc}
&\textbf{Issue} & \textbf{Page}\\[6pt]
\Avtors{Abgaryan~K.\,K. and Gavrilov~E.\,S.} Integration platform for multiscale modeling of neuromorphic\linebreak
\\[-12pt]
\hspace*{23pt}systems&2&104--110\\
\Avtors{Abgaryan~K.\,K. and Kolbin~I.\,S.} Application of multiscale approach and data sciences for\linebreak
\\[-12pt]
\hspace*{23pt}modeling thermal conductivity in layered structures&4&91--99\\
\Avtors{Adamova~K.\,A.} see Shnurkov~~P.\,V.&&\\
\Avtors{Agalarov~Ya.\,M.} Optimization of the capacity of the main storage in $G/M/1/K$ queueing system\linebreak
\\[-12pt]
\hspace*{23pt}with an additional storage device&2&72--79\\
\Avtors{Agasandyan~G.\,A.} Computational aspects of optimization on CC-VaR in a complex of markets&3&62--70\\
\Avtors{Ageev~K.\,A., Sopin~E.\,S., Yarkina~N.\,V., Samouylov~K.\,E., and Shorgin~S.\,Ya.} Analysis of the\linebreak
\\[-12pt]
\hspace*{23pt}network slicing mechanisms with guaranteed allocated resources for various traffic types&3&\hphantom{1}94--100\\
\Avtors{Bakhteev~O.\,Yu.} see Grabovoy~A.\,V.&&\\
\Avtors{Bazilevskiy~M.\,P.} Multifactor fully connected linear regression models without constraints to the\linebreak
\\[-12pt]
\hspace*{23pt}ratios of variables errors variances&2&92--97\\
\Avtors{Belenkov~V.\,G.} see Budzko~V.\,I.&&\\
\Avtors{Betelin~V.\,B., Kushnirenko~A.\,G., and Leonov~A.\,G.} Basic concepts of programming expounded\linebreak
\\[-12pt]
\hspace*{23pt}for preschoolers&3&55--61\\
\Avtors{Betelin~V.\,B., Kushnirenko~A.\,G., Semenov~A.\,L., and Soprunov~S.\,F.} About digital literacy and\linebreak
\\[-12pt]
\hspace*{23pt}environments for its development&4&100--107\\
\Avtors{Borisov~A.\,V.} Numerical schemes of Markov jump process filtering given discretized observa-\linebreak
\\[-12pt]
\hspace*{23pt}tions~II: Additive noise case&1&17--23\\
\Avtors{Borisov~A.\,V.} Numerical schemes of Markov jump process filtering given discretized observa-\linebreak
\\[-12pt]
\hspace*{23pt}tions III: Multiplicative noises case&2&10--18\\
\Avtors{Bosov~A.\,V.} Stochastic differential system output control by the quadratic criterion. V. Case of\linebreak
\\[-12pt]
\hspace*{23pt}incomplete state information&2&19--28\\
\Avtors{Bosov~A.\,V., Martyushova~Ya.\,G., Naumov~A.\,V., and Sapunova~A.\,P.} Bayesian approach to the\linebreak
\\[-12pt]
\hspace*{23pt}construction of an individual user trajectory in the system of distance learning&3&86--93\\
\Avtors{Bosov~A.\,V. and Stefanovich~A.\,I.} Stochastic differential system output control by the quadratic\linebreak
\\[-12pt]
\hspace*{23pt}criterion. IV. Alternative numerical decision&1&24--30\\
\Avtors{Briukhov~D.\,O., Stupnikov~S.\,A., Kovalev~D.\,Yu., and Shanin~I.\,A.} Neurophysiology as a subject\linebreak
\\[-12pt]
\hspace*{23pt}domain for~data intensive problem solving&1&40--47\\
\Avtors{Budzko~V.\,I., Yadrintsev~V.\,V., Sochenkov~I.\,V., Korolev~V.\,I., and Belenkov~V.\,G.} Extraction of confidentiality markers from texts under conditions of high uncertainty in systems with\linebreak
\\[-12pt]
\hspace*{23pt}data intensive usage&4&69--76\\
\Avtors{Danilishin~A.\,R. and Golembiovsky~D.\,Yu.} Estimating the fair value of options based on\linebreak
\\[-12pt]
\hspace*{23pt}ARIMA--GARCH models with errors distributed according to the Johnson's $S_u$ law&4&83--90\\
\Avtors{Danilishin~A.\,R. and Golembiovsky~D.\,Yu.} Risk-neutral dynamics for the ARIMA-GARCH\linebreak
\\[-12pt]
\hspace*{23pt}random process with errors distributed according to the Johnson's $S_u$ law&1&48--55\\
\Avtors{Diachenko~Yu.\,G.} see Sokolov~I.\,A.&&\\
\Avtors{Dimentov~A.\,V.} see Krasnov~F.\,V.&&\\
\Avtors{Donskoy~V.\,I.} Optimization models extraction from data&3&109--118\\
\Avtors{Dubnov~Y.\,A.} see Popkov~Y.\,S.&&\\
\Avtors{Dulin~S.\,K., Dulina~N.\,G., and Ermakov~P.\,V.} Information fusion of documents&1&128--135\\
\Avtors{Dulina~N.\,G.} see Dulin~S.\,K.&&\\
\Avtors{Efrosinin~D.\,V.} see Kharin~P.\,A.&&\\
\Avtors{Ehrlich~L.\,I.} see Kozerenko~E.\,B.&&\\
\Avtors{Ermakov~P.\,V.} see Dulin~S.\,K.&&\\
\end{tabular}
}
\pagebreak

\def\leftfootline{\small{\textbf{\thepage}
\hfill INFORMATIKA I EE PRIMENENIYA~--- INFORMATICS AND APPLICATIONS\ \ \ 2020\
\ \ volume~14\ \ \ issue\ 4}
}%
 \def\rightfootline{\small{INFORMATIKA I EE PRIMENENIYA~---
INFORMATICS AND APPLICATIONS\ \ \ 2020\ \ \ volume~14\ \ \ issue\ 4
\hfill \textbf{\thepage}}}

\def\leftkol{2020 AUTHOR INDEX} % ENGLISH ABSTRACTS}

\def\rightkol{2020 AUTHOR INDEX} %ENGLISH ABSTRACTS}


\noindent
{\tabcolsep=3pt
\begin{tabular}{p{395.48108pt}cc}
&\textbf{Issue} & \textbf{Page}\\[6pt]
\Avtors{Fahrutdinov~R.\,Sh.} see Kostina~A.\,A.&&\\
\Avtors{Frenkel~S.\,L. and Zakharov~V.\,N.} Joint assessment of data predictability and quality pre-\linebreak
\\[-12pt]
\hspace*{23pt}dictors&2&40--49\\
\Avtors{Gaidamaka~Yu.\,V.} see Moskaleva~F.\,A.&&\\
\Avtors{Gavrilov~E.\,S.} see Abgaryan~K.\,K.&&\\
\Avtors{Golembiovsky~D.\,Yu.} see Danilishin~A.\,R.&&\\
\Avtors{Golembiovsky~D.\,Yu.} see Danilishin~A.\,R.&&\\
\Avtors{Goncharov~A.\,V. and Strijov~V.\,V.} Alignment of ordered set Cartesian product&1&31--39\\
\Avtors{Goncharov~A.\,A., Zatsman~I.\,M., and Kruzhkov~M.\,G.} Evolution of classifications in supracorpora\linebreak
\\[-12pt]
\hspace*{23pt}databases&4&108--116\\
\Avtors{Gorshenin~A.\,K. and Korolev~V.\,Yu.} Approximation of particle size distributions of lunar regolith\linebreak
\\[-12pt]
\hspace*{23pt}based on the resampling&2&50--57\\
\Avtors{Gorshenin~A.\,K., Korolev~V.\,Yu., and Shcherbinina~A.\,A.} Statistical estimation of distributions\linebreak
\\[-12pt]
\hspace*{23pt}of random coefficients in the Langevin stochastic differential equation&3&\hphantom{1}3--12\\
\Avtors{Gorshenin~A.\,K. and Kuzmin~V.\,Yu.} Analysis of configurations of LSTM networks for medium-\linebreak
\\[-12pt]
\hspace*{23pt}term vector forecasting&1&10--16\\
\Avtors{Grabovoy~A.\,V., Bakhteev~O.\,Yu., and Strijov~V.\,V.} Ordering the set of neural network parameters&2&58--65\\
\Avtors{Grusho~A.\,A., Timonina~E.\,E., Grusho~N.\,A., and Teryokhina~I.\,Yu.} Identifying anomalies using\linebreak
\\[-12pt]
\hspace*{23pt}metadata&3&76--80\\
\Avtors{Grusho~A.\,A., Zabezhailo~M.\,I., Smirnov~D.\,V., and Timonina~E.\,E.} On probabilistic estimates of\linebreak
\\[-12pt]
\hspace*{23pt}the validity of empirical conclusions&4&3--8\\
\Avtors{Grusho~A.\,A., Zabezhailo~M.\,I., and Timonina~E.\,E.} On causal representativeness of training\linebreak
\\[-12pt]
\hspace*{23pt}samples of precedents in diagnostic type tasks&1&80--86\\
\Avtors{Grusho~A.\,A.} see Grusho~N.\,A.&&\\
\Avtors{Grusho~N.\,A., Grusho~A.\,A., Zabezhailo~M.\,I., and Timonina~E.\,E.} Methods of finding the causes\linebreak
\\[-12pt]
\hspace*{23pt}of information technology failures by means of metadata&2&33--39\\
\Avtors{Grusho~N.\,A., Zabezhailo~M.\,I., Smirnov~D.\,V., Timonina~E.\,E., and Shorgin~S.\,Ya.} Mathematical\linebreak
\\[-12pt]
\hspace*{23pt}statistics in the task of identifying hostile insiders&3&71--75\\
\Avtors{Grusho~N.\,A.} see Grusho~A.\,A.&&\\
\Avtors{Kharin~P.\,A., Makeeva~E.\,D., Kochetkova~I.\,A., Efrosinin~D.\,V., and Shorgin~S.\,Ya.} Retrial\linebreak
\\[-12pt]
\hspace*{23pt}queuing model for analyzing joint URLLC and eMBB transmission in 5G networks&4&17--24\\
\Avtors{Khusainov~A.\,A.} Performance of the bounded pipeline&1&87--93\\
\Avtors{Kirikov~I.\,A.} see Rumovskaya~S.\,B.&&\\
\Avtors{Kirilyuk~I.\,L. and Sen'ko~O.\,V.} Selection of optimal complexity models by methods of nonparametric statistics (on the example of production function model of regions of the Russian\linebreak
\\[-12pt]
\hspace*{23pt}Federation)&2&111--118\\
\Avtors{Kochetkova~I.\,A.} see Kharin~P.\,A.&&\\
\Avtors{Kolbin~I.\,S.} see Abgaryan~K.\,K.&&\\
\Avtors{Korolev~V.\,I.} see Budzko~V.\,I.&&\\
\Avtors{Korolev~V.\,Yu.} On the distribution of the ratio of the sum of sample elements exceeding\linebreak
\\[-12pt]
\hspace*{23pt}a threshold to the total sum of sample elements.~I&3&35--43\\
\Avtors{Korolev~V.\,Yu.} On the distribution of the ratio of the sum of sample elements exceeding\linebreak
\\[-12pt]
\hspace*{23pt}a threshold to the total sum of sample elements.~II&4&33--36\\
\Avtors{Korolev~V.\,Yu.} see Gorshenin~A.\,K.&&\\
\Avtors{Korolev~V.\,Yu.} see Gorshenin~A.\,K.&&\\
\Avtors{Kostina~A.\,A., Mirin~A.\,Yu., Moldovyan~D.\,N., and Fahrutdinov~R.\,Sh.} Method for defining finite noncommutative associative algebras of arbitrary even dimension for development of the\linebreak
\\[-12pt]
\hspace*{23pt}postquantum cryptoschemes&1&\hphantom{1}94--100\\
\Avtors{Kovalev~D.\,Yu.} see Briukhov~D.\,O.&&\\
\Avtors{Kozerenko~E.\,B., Mikheev~M.\,Y., Somin~N.\,V., Ehrlich~L.\,I., and Kuznetsov~K.\,I.} Analytical\linebreak
\\[-12pt]
\hspace*{23pt}textology in intelligent processing systems for unstructured data&1&113--120\\
\Avtors{Krasnov~F.\,V., Dimentov~A.\,V., and Shvartsman~M.\,E.} Using topic models for pairwise comparison\linebreak
\\[-12pt]
\hspace*{23pt}of collections of scientific papers&3&129--135\\
\end{tabular}
}
\pagebreak

\def\leftfootline{\small{\textbf{\thepage}
\hfill INFORMATIKA I EE PRIMENENIYA~--- INFORMATICS AND APPLICATIONS\ \ \ 2020\
\ \ volume~14\ \ \ issue\ 4}
}%
 \def\rightfootline{\small{INFORMATIKA I EE PRIMENENIYA~---
INFORMATICS AND APPLICATIONS\ \ \ 2020\ \ \ volume~14\ \ \ issue\ 4
\hfill \textbf{\thepage}}}

\def\leftkol{2020 AUTHOR INDEX} % ENGLISH ABSTRACTS}

\def\rightkol{2020 AUTHOR INDEX} %ENGLISH ABSTRACTS}


\noindent
{\tabcolsep=3pt
\begin{tabular}{p{395.48108pt}cc}
&\textbf{Issue} & \textbf{Page}\\[6pt]
\Avtors{Krivenko~M.\,P.} Sequential analysis of serial measurements based on multivariate reference\linebreak
\\[-12pt]
\hspace*{23pt}regions&2&86--91\\
\Avtors{Kruzhkov~M.\,G.} see Goncharov~A.\,A.&&\\
\Avtors{Kudryavtsev~A.\,A. and Shestakov~O.\,V.} Method of logarithmic moments for estimating the\linebreak
\\[-12pt]
\hspace*{23pt}gamma-exponential distribution parameters&3&49--54\\
\Avtors{Kushnirenko~A.\,G.} see Betelin~V.\,B.&&\\
\Avtors{Kushnirenko~A.\,G.} see Betelin~V.\,B.&&\\
\Avtors{Kuzmin~V.\,Yu.} see Gorshenin~A.\,K.&&\\
\Avtors{Kuznetsov~K.\,I.} see Kozerenko~E.\,B.&&\\
\Avtors{Leonov~A.\,G.} see Betelin~V.\,B.&&\\
\Avtors{Makeeva~E.\,D.} see Kharin~P.\,A.&&\\
\Avtors{Malashenko~Yu.\,E. and Nazarova~I.\,A.} Approximation of the multiuser network feasible\linebreak
\\[-12pt]
\hspace*{23pt}flows set&3&81--85\\
\Avtors{Martyushova~Ya.\,G.} see Bosov~A.\,V.&&\\
\Avtors{Matyushenko~S.\,I. and Razumchik~R.\,V.} Stationary characteristics of discrete-time Geo$/G/1/\infty$\linebreak
\\[-12pt]
\hspace*{23pt}queue with batch arrivals and one queue skipping policy&4&25--32\\
\Avtors{Melnikov~A.\,V.} see Vokhmintcev~A.\,V.&&\\
\Avtors{Melnikov~S.\,Yu. and Samouylov~K.\,E.} Statistical properties of binary nonautonomous shift\linebreak
\\[-12pt]
\hspace*{23pt}registers with internal xor&2&80--85\\
\Avtors{Meykhanadzhyan~L.\,A. and Razumchik~R.\,V.} Stationary characteristics of $M/G/2/\infty$ queue\linebreak
\\[-12pt]
\hspace*{23pt}with identical servers, LIFO service, and resampling policy&2&66--71\\
\Avtors{Mikheev~M.\,Y.} see Kozerenko~E.\,B.&&\\
\Avtors{Milovanova~T.\,A. and Razumchik~R.\,V.} A single-server queueing system with LIFO service,\linebreak
\\[-12pt]
\hspace*{23pt}probabilistic priority, batch Poisson arrivals, and background customers&3&26--34\\
\Avtors{Mirin~A.\,Yu.} see Kostina~A.\,A.&&\\
\Avtors{Moldovyan~D.\,N.} see Kostina~A.\,A.&&\\
\Avtors{Moskaleva~F.\,A., Gaidamaka~Yu.\,V., and Shorgin~V.\,S.} Impact of the isolation parameters on\linebreak
\\[-12pt]
\hspace*{23pt}resource allocation in the network slicing model&4&\hphantom{1}9--16\\
\Avtors{Naumov~A.\,V.} see Bosov~A.\,V.&&\\
\Avtors{Naumov~V.\,A. and Samouylov~К.\,Е.} On Markovian and rational arrival processes.~I&3&13--19\\
\Avtors{Naumov~V.\,A. and Samouylov~K.\,E.} On Markovian and rational arrival processes.~II&4&37--46\\
\Avtors{Nazarova~I.\,A.} see Malashenko~Yu.\,E.&&\\
\Avtors{Novikov~D.\,A.} see Shnurkov~P.\,V.&&\\
\Avtors{Nuriev~V.\,A. and Zatsman~I.\,M.} Reducing the spectrum of translation models in supracorpora\linebreak
\\[-12pt]
\hspace*{23pt}databases&2&119--126\\
\Avtors{Pachganov~S.\,A.} see Vokhmintcev~A.\,V.&&\\
\Avtors{Popkov~A.\,Y.} see Popkov~Y.\,S.&&\\
\Avtors{Popkov~Y.\,S., Popkov~A.\,Y., and Dubnov~Y.\,A.} Deterministic and randomized methods of entropy\linebreak
\\[-12pt]
\hspace*{23pt}projection for dimensionality reduction problems&4&47--54\\
\Avtors{Popov~G.\,A., Simavoryan~S.\,Zh., Simonyan~A.\,R., and Ulitina~E.\,I.} Modeling of monitoring of\linebreak
\\[-12pt]
\hspace*{23pt}information security process on the basis of queuing systems&1&71--79\\
\Avtors{Popov~M.\,V. and Posypkin~M.\,A.} Approximation of the set of solutions of systems of nonlinear\linebreak
\\[-12pt]
\hspace*{23pt}inequalities using graphic accelerators&3&20--25\\
\Avtors{Posypkin~M.\,A.} see Popov~M.\,V.&&\\
\Avtors{Potanin~M.\,S., Vayser~K.\,O., Zholobov~V.\,A., and Strijov~V.\,V.} Deep learning neural network\linebreak
\\[-12pt]
\hspace*{23pt}structure optimization&4&55--62\\
\Avtors{Razumchik~R.\,V.} see Matyushenko~S.\,I.&&\\
\Avtors{Razumchik~R.\,V.} see Meykhanadzhyan~L.\,A.&&\\
\Avtors{Razumchik~R.\,V.} see Milovanova~T.\,A.&&\\
\Avtors{Rogdestvenski~Yu.\,V.} see Sokolov~I.\,A.&&\\
\Avtors{Rumovskaya~S.\,B. and Kirikov~I.\,A.} Conflict visual representation method in hybrid intelligent\linebreak
\\[-12pt]
\hspace*{23pt}multiagent systems&4&77--82\\
\Avtors{Samouylov~K.\,E.} see Ageev~K.\,A.&&\\
\end{tabular}
}
\pagebreak

\def\leftfootline{\small{\textbf{\thepage}
\hfill INFORMATIKA I EE PRIMENENIYA~--- INFORMATICS AND APPLICATIONS\ \ \ 2020\
\ \ volume~14\ \ \ issue\ 4}
}%
 \def\rightfootline{\small{INFORMATIKA I EE PRIMENENIYA~---
INFORMATICS AND APPLICATIONS\ \ \ 2020\ \ \ volume~14\ \ \ issue\ 4
\hfill \textbf{\thepage}}}

\def\leftkol{2020 AUTHOR INDEX} % ENGLISH ABSTRACTS}

\def\rightkol{2020 AUTHOR INDEX} %ENGLISH ABSTRACTS}


\noindent
{\tabcolsep=3pt
\begin{tabular}{p{395.48108pt}cc}
&\textbf{Issue} & \textbf{Page}\\[6pt]
\Avtors{Samouylov~K.\,E.} see Melnikov~S.\,Yu.&&\\
\Avtors{Samouylov~K.\,E.} see Naumov~V.\,A.&&\\
\Avtors{Samouylov~K.\,Е.} see Naumov~V.\,A.&&\\
\Avtors{Sapunova~A.\,P.} see Bosov~A.\,V.&&\\
\Avtors{Satin~Ya.\,A., Zeifman~A.\,I., and Shilova~G.\,N.} On approaches to constructing limiting regimes\linebreak
\\[-12pt]
\hspace*{23pt}for some queuing models&2&3--9\\
\Avtors{Semenov~A.\,L.} see Betelin~V.\,B.&&\\
\Avtors{Sen'ko~O.\,V.} see Kirilyuk~I.\,L.&&\\
\Avtors{Serebryanskii~S.\,M. and Tyrsin~A.\,N.} Improvement of the accuracy of solution of tasks for the\linebreak
\\[-12pt]
\hspace*{23pt}account of the construction of boundary conditions&1&56--62\\
\Avtors{Sevastianov~L.\,A. and Shchetinin~E.\,Yu.} On methods for improving the accuracy of multiclass\linebreak
\\[-12pt]
\hspace*{23pt}classification on imbalanced data&1&63--70\\
\Avtors{Shanin~I.\,A.} see Briukhov~D.\,O.&&\\
\Avtors{Shcherbinina~A.\,A.} see Gorshenin~A.\,K.&&\\
\Avtors{Shchetinin~E.\,Yu.} see Sevastianov~L.\,A.&&\\
\Avtors{Shestakov~O.\,V.} Asymptotic regularity of the wavelet methods of inverting linear homogeneous\linebreak
\\[-12pt]
\hspace*{23pt}operators from observations recorded at random times&1&3--9\\
\Avtors{Shestakov~O.\,V.} Asymptotics of the mean-square risk estimate in the problem of inverting the\linebreak
\\[-12pt]
\hspace*{23pt}Radon transform from projections registered on a random grid&2&29--32\\
\Avtors{Shestakov~O.\,V.} On the statistical properties of risk estimate in the problem of inverting the\linebreak
\\[-12pt]
\hspace*{23pt}Radon transform with a random volume of projection data&3&44--48\\
\Avtors{Shestakov~O.\,V.} see Kudryavtsev~A.\,A.&&\\
\Avtors{Shihiev~F.\,Sh.} see Shihiev~Sh.\,B.&&\\
\Avtors{Shihiev~Sh.\,B. and Shihiev~F.\,Sh.} Incapsulation of semantic representations into elements of\linebreak
\\[-12pt]
\hspace*{23pt}a grammar&1&121--127\\
\Avtors{Shilova~G.\,N.} see Satin~Ya.\,A.&&\\
\Avtors{Shnurkov~~P.\,V. and Adamova~K.\,A.} Solution of the unconditional extremal problem for a~linear-\linebreak
\\[-12pt]
\hspace*{23pt}fractional integral functional dependent on the parameter&2&\hphantom{1}98--103\\
\Avtors{Shnurkov~P.\,V. and Novikov~D.\,A.} On the concept of a stochastic model with control at the\linebreak
\\[-12pt]
\hspace*{23pt}moments of the process at the border of a presented subset of multiple states&3&101--108\\
\Avtors{Shorgin~S.\,Ya.} see Ageev~K.\,A.&&\\
\Avtors{Shorgin~S.\,Ya.} see Grusho~N.\,A.&&\\
\Avtors{Shorgin~S.\,Ya.} see Kharin~P.\,A.&&\\
\Avtors{Shorgin~V.\,S.} see Moskaleva~F.\,A.&&\\
\Avtors{Shvartsman~M.\,E.} see Krasnov~F.\,V.&&\\
\Avtors{Simavoryan~S.\,Zh.} see Popov~G.\,A.&&\\
\Avtors{Simonyan~A.\,R.} see Popov~G.\,A.&&\\
\Avtors{Smirnov~D.\,V.} see Grusho~A.\,A.&&\\
\Avtors{Smirnov~D.\,V.} see Grusho~N.\,A.&&\\
\Avtors{Sochenkov~I.\,V.} see Budzko~V.\,I.&&\\
\Avtors{Sokolov~I.\,A., Stepchenkov~Yu.\,A., Diachenko~Yu.\,G., and Rogdestvenski~Yu.\,V.} Improvement of\linebreak
\\[-12pt]
\hspace*{23pt}self-timed circuit soft error tolerance&4&63--68\\
\Avtors{Somin~N.\,V.} see Kozerenko~E.\,B.&&\\
\Avtors{Sopin~E.\,S.} see Ageev~K.\,A.&&\\
\Avtors{Soprunov~S.\,F.} see Betelin~V.\,B.&&\\
\Avtors{Stefanovich~A.\,I.} see Bosov~A.\,V.&&\\
\Avtors{Stepchenkov~Yu.\,A.} see Sokolov~I.\,A.&&\\
\Avtors{Strijov~V.\,V.} see Goncharov~A.\,V.&&\\
\Avtors{Strijov~V.\,V.} see Grabovoy~A.\,V.&&\\
\Avtors{Strijov~V.\,V.} see Potanin~M.\,S.&&\\
\Avtors{Stupnikov~S.\,A.} see Briukhov~D.\,O.&&\\
\Avtors{Teryokhina~I.\,Yu.} see Grusho~A.\,A.&&\\
\Avtors{Timonina~E.\,E.} see Grusho~A.\,A.&&\\
\end{tabular}
}
\pagebreak

\def\leftfootline{\small{\textbf{\thepage}
\hfill INFORMATIKA I EE PRIMENENIYA~--- INFORMATICS AND APPLICATIONS\ \ \ 2020\
\ \ volume~14\ \ \ issue\ 4}
}%
 \def\rightfootline{\small{INFORMATIKA I EE PRIMENENIYA~---
INFORMATICS AND APPLICATIONS\ \ \ 2020\ \ \ volume~14\ \ \ issue\ 4
\hfill \textbf{\thepage}}}

\def\leftkol{2020 AUTHOR INDEX} % ENGLISH ABSTRACTS}

\def\rightkol{2020 AUTHOR INDEX} %ENGLISH ABSTRACTS}


\noindent
{\tabcolsep=3pt
\begin{tabular}{p{395.48108pt}cc}
&\textbf{Issue} & \textbf{Page}\\[6pt]
\Avtors{Timonina~E.\,E.} see Grusho~A.\,A.&&\\
\Avtors{Timonina~E.\,E.} see Grusho~A.\,A.&&\\
\Avtors{Timonina~E.\,E.} see Grusho~N.\,A.&&\\
\Avtors{Timonina~E.\,E.} see Grusho~N.\,A.&&\\
\Avtors{Tyrsin~A.\,N.} see Serebryanskii~S.\,M.&&\\
\Avtors{Ulitina~E.\,I.} see Popov~G.\,A.&&\\
\Avtors{Vayser~K.\,O.} see Potanin~M.\,S.&&\\
\Avtors{Vokhmintcev~A.\,V., Melnikov~A.\,V., and Pachganov~S.\,A.} Simultaneous localization and mapping method in  three-dimensional space based on the combined solution of the  point--point\linebreak
\\[-12pt]
\hspace*{23pt}variation problem ICP for an affine transformation&1&101--112\\
\Avtors{Yadrintsev~V.\,V.} see Budzko~V.\,I.&&\\
\Avtors{Yarkina~N.\,V.} see Ageev~K.\,A.&&\\
\Avtors{Zabezhailo~M.\,I.} see Grusho~A.\,A.&&\\
\Avtors{Zabezhailo~M.\,I.} see Grusho~A.\,A.&&\\
\Avtors{Zabezhailo~M.\,I.} see Grusho~N.\,A.&&\\
\Avtors{Zabezhailo~M.\,I.} see Grusho~N.\,A.&&\\
\Avtors{Zakharov~V.\,N.} see Frenkel~S.\,L.&&\\
\Avtors{Zatsman~I.\,M.} Problem-oriented verifying the completeness  of~temporal ontologies and\linebreak
\\[-12pt]
\hspace*{23pt}filling~conceptual lacunas&3&119--128\\
\Avtors{Zatsman~I.\,M.} see Goncharov~A.\,A.&&\\
\Avtors{Zatsman~I.\,M.} see Nuriev~V.\,A.&&\\
\Avtors{Zeifman~A.\,I.} see Satin~Ya.\,A.&&\\
\Avtors{Zholobov~V.\,A.} see Potanin~M.\,S.&&\\
\end{tabular}
}

%\thispagestyle{myheadings}
\def\leftfootline{\small{\textbf{\thepage}
\hfill INFORMATIKA I EE PRIMENENIYA~--- INFORMATICS AND APPLICATIONS\ \ \ 2020\
\ \ volume~14\ \ \ issue\ 4}
}%
 \def\rightfootline{\small{INFORMATIKA I EE PRIMENENIYA~---
INFORMATICS AND APPLICATIONS\ \ \ 2020\ \ \ volume~14\ \ \ issue\ 4
\hfill \textbf{\thepage}}}

 \label{end\stat}

\newpage


%\linebreak
%\\[-12pt]
%\hspace*{23pt}

%   \vspace*{-48pt}

\begin{center}
\vspace*{6pt}
\mbox{%
%\epsfxsize=50mm %56.519mm  
%\epsfbox{smu-1.eps} 

\epsfxsize=50mm %46.402 mm
\epsfbox{nec-rb.eps}
}
%\end{center}

\vspace*{9pt} %Академик


%   \begin{center}
\fbox{\large\textbf{Рустем Бадриевич Сейфуль-Мулюков}}\\[6pt]
\textbf{\large 1928--2020}
   \end{center}


   %\vspace*{2.5mm}

   \vspace*{5mm}

   \thispagestyle{empty}

%\

%\vspace*{-12pt}

  
      Редакция журнала <<Информатика и~её применения>> с глубоким 
      прискорбием сообщают, что 17~марта 2020~г.\ на 93-м~году жизни 
      скончался заведующий редакцией журнала, главный научный сотрудник Федерального исследовательского центра <<Информатика и~управление>> Российской академии наук
      Рустем Бадриевич Сейфуль-Мулюков.
           
     Всю свою жизнь Рустем Бадриевич посвятил служению науке. Закончив в~1956~г.\ аспирантуру Московского ордена Трудового Красного знамени Нефтяного института им.\ академика
     И.\,М.~Губкина, он прошел путь от заведующего отделом Института геологии зарубежных стран Министерства геологии СССР до заместителя директора ВИНИТИ
     АН СССР, доктора гео\-ло\-го-ми\-не\-ра\-ло\-ги\-че\-ских наук, профессора.
     
     С марта 2002~г.\ Рустем Бадриевич успешно применял свои знания и~организационный талант в ИПИ
     РАН (в~дальнейшем~--- ФИЦ ИУ РАН), в~котором руководил лабораторией и~отделом, занимающимися вопросами технологий информационной технической деятельности. 
Р.\,Б.~Сейфуль-Мулюков, являясь автором значительного количества научных трудов и~монографий по геологии, информационным технологиям и~теоретической информатике, осуществлял организацию издания монографий ИПИ РАН и~ФИЦ ИУ РАН, библиографий научных сотрудников Центра.
     
     Р.\,Б.~Сейфуль-Мулюков являлся заведующим редакцией журналов <<Информатика и~её применения>> и~<<Системы и~средства информатики>>, членом редколлегии журнала <<Системы и~средства информатики>>. Он вложил огромный вклад в становление и~развитие этих журналов, организацию их регистрации, функционирования, редактуры и~издания. Включение этих журналов в ряд отечественных и~зарубежных информационных баз и~систем цитирования во многом является его личной заслугой.
     
     На всех занимаемых должностях Рустем Бадриевич отличался высоким профессионализмом, преданностью делу и~вниманием к коллегам.
     
     \thispagestyle{empty}
     
     Рустема Бадриевича отличали доброта, отзывчивость, неиссякаемый
      оптимизм, простота и~сердечность.
     
     Коллеги Рустема Бадриевича запомнят его как многогранного в~своих увлечениях человека, живописца,
     эрудита и~энциклопедиста, интересующегося историей, литературой и~искусством.
     
     Выражаем глубокое
     соболезнование семье, родственникам, друзьям и~коллегам по работе в~связи с~тяжелой невосполнимой утратой.
     Светлый образ Рустема Бадриевича навсегда сохранится в~нашей памяти.
     

      

%\def\stat{cont}
{%\hrule\par
%\vskip 7pt % 7pt
\raggedleft\Large \bf%\baselineskip=3.2ex
А\,В\,Т\,О\,Р\,С\,К\,И\,Й\ \ У\,К\,А\,З\,А\,Т\,Е\,Л\,Ь\ \ З\,А\ \ 2\,0\,1\,0 г. \vskip 17pt
    \hrule
    \par
\vskip 21pt plus 6pt minus 3pt }

\label{st\stat}

\def\tit{\ }

\def\aut{\ }
\def\auf{\ }

\def\leftkol{\ } % ENGLISH ABSTRACTS}

\def\rightkol{\ } %АВТОРСКИЙ УКАЗАТЕЛЬ ЗА 2010 г.} %ENGLISH ABSTRACTS}

\titele{\tit}{\aut}{\auf}{\leftkol}{\rightkol}

\vspace*{-12pt}

{\tabcolsep=3pt
\begin{tabular}{p{388pt}rr}
&\textbf{Выпуск} & \textbf{Стр.}\\[6pt]
\hangindent=23pt\noindent\textbf{Арутюнян~А.\,Р.} Моделирование влияния деформаций отпечатков пальцев на 
точность\linebreak
\vspace*{-12pt}\\
\hspace*{23pt}дактилоскопической идентификации$\dotfill$&1&51\\
\hangindent=23pt\noindent\textbf{Архипов~О.\,П., Зыкова~З.\,П.} Интеграция гетерогенной информации о цветных 
пикселях\linebreak
\vspace*{-12pt}\\
\hspace*{23pt}и их цветовосприятии$\dotfill$&4&15\\
\hangindent=23pt\noindent\textbf{Баранов~С.\,И., Френкель~С.\,Л., Захаров~В.\,Н.} Полуформальная верификация 
цифрового устройства с конвейером, основанная на использовании алгоритмических машин\linebreak
\vspace*{-12pt}\\
\hspace*{23pt}состояния$\dotfill$&4&49\\
\textbf{Бекетова~И.\,В.} см.~Каратеев~С.\,Л.&&\\
\textbf{Белоусов~В.\,В.} см.~Синицын~И.\,Н.&&\\
\hangindent=23pt\noindent\textbf{Бенинг~В.\,Е., Королев~Р.\,А.} О предельном поведении мощностей критериев в 
случае\linebreak
\vspace*{-12pt}\\
\hspace*{23pt}распределения Лапласа$\dotfill$&2&63\\
\hangindent=23pt\noindent\textbf{Бенинг~В.\,Е., Сипина~А.\,В.} Асимптотическое разложение для мощности 
критерия,\linebreak
\vspace*{-12pt}\\
\hspace*{23pt}основанного на выборочной медиане, в случае распределения Лапласа$\dotfill$&1&18\\
\textbf{Бондаренко~А.\,В.} см.~Каратеев~С.\,Л.&&\\
\hangindent=23pt\noindent\textbf{Бородина~А.\,В., Морозов~Е.\,В.} Об оценивании асимптотики вероятности 
большого\linebreak
\vspace*{-12pt}\\
\hspace*{23pt}уклонения стационарной регенеративной очереди с одним прибором$\dotfill$&3&29\\
\hangindent=23pt\noindent\textbf{Бунтман~Н.\,В., Минель~Ж.-Л., Ле~Пезан~Д., Зацман~И.\,М.} Типология и 
компьютерное\linebreak
\vspace*{-12pt}\\
\hspace*{23pt}моделирование трудностей перевода$\dotfill$&3&77\\
\textbf{Визильтер~Ю.\,В.} см.~Каратеев~С.\,Л.&&\\
\hangindent=23pt\noindent\textbf{Гавриленко~С.\,В.} Оценки скорости сходимости распределений случайных сумм с 
безгранично делимыми индексами к нормальному закону$\dotfill$&4&81\\
\hangindent=23pt\noindent\textbf{Григорьева~М.\,Е., Шевцова~И.\,Г.} Уточнение неравенства 
Каца--Берри--Эссеена$\dotfill$&2&75\\
\hangindent=23pt\noindent\textbf{Грушо~А.\,А., Грушо~Н.\,А., Тимонина~Е.\,Е.} Поиск конфликтов в политиках 
безопасности: модель случайных графов$\dotfill$&3&38\\
\textbf{Грушо~Н.\,А.} см.~Грушо~А.\,А.&&\\
\hangindent=23pt\noindent\textbf{Гудков~В.\,Ю.} Математические модели изображения отпечатка пальца на основе 
описания линий$\dotfill$&1&58\\
\textbf{Гуртов~А.\,В.} см.~Лукьяненко~А.\,С.&&\\
\textbf{Желтов~С.\,Ю.} см.~Каратеев~С.\,Л.&&\\
\hangindent=23pt\noindent\textbf{Захаров~А.\,А., Серебряков~В.\,А.} Система управления электронной библиотекой 
LibMeta$\dotfill$&4&2\\
\textbf{Захаров~В.\,Н.} см.~Баранов~С.\,И.&&\\
\textbf{Захарова~Т.\,В.} см.~Матвеева~С.\,С.&&\\
\hangindent=23pt\noindent\textbf{Зацаринный~А.\,А., Чупраков~К.\,Г.} Некоторые аспекты выбора технологии для 
постро-\linebreak
\vspace*{-12pt}\\
\hspace*{23pt}ения систем отображения информации ситуационного центра$\dotfill$&3&59\\
\textbf{Зацман~И.\,М.} см.~Бунтман~Н.\,В.&&\\
\hangindent=23pt\noindent\textbf{Зейфман~А.\,И., Коротышева~А.\,В., Сатин~Я.\,А., Шоргин~С.\,Я.} Об 
устойчивости нестаци-\linebreak
\vspace*{-12pt}\\
\hspace*{23pt}онарных систем обслуживания с катастрофами$\dotfill$&3&9\\
\textbf{Зыкова~З.\,П.} см.~Архипов~О.\,П.&&\\
\hangindent=23pt\noindent\textbf{Илюшин~Г.\,Я., Соколов~И.\,А.} Организация управляемого доступа пользователей 
к\linebreak
\vspace*{-12pt}\\
\hspace*{23pt}разнородным ведомственным информационным ресурсам$\dotfill$&1&24\\
\hangindent=23pt\noindent\textbf{Кавагучи~Ю., Ульянов~В.\,В., Фуджикоши~Я.} Приближения для статистик, 
описывающих\linebreak
\vspace*{-12pt}\\
\hspace*{23pt}геометрические свойства данных большой размерности, с оценками 
ошибок$\dotfill$&1&12\\
\hangindent=23pt\noindent\textbf{Каратеев~С.\,Л., Бекетова~И.\,В., Ососков~М.\,В., Князь~В.\,А., 
Визильтер~Ю.\,В., Бондаренко~А.\,В., Желтов~С.\,Ю.} Автоматизированный контроль 
качества цифровых\linebreak
\vspace*{-12pt}\\
\hspace*{23pt}изображений для персональных документов$\dotfill$&1&65\\
\end{tabular}
}

\pagebreak

\def\leftkol{АВТОРСКИЙ УКАЗАТЕЛЬ ЗА 2010 г.} % ENGLISH ABSTRACTS}

\def\rightkol{АВТОРСКИЙ УКАЗАТЕЛЬ ЗА 2010 г.} %ENGLISH ABSTRACTS}

{\tabcolsep=3pt
\begin{tabular}{p{388pt}rr}
&\textbf{Выпуск} & \textbf{Стр.}\\[3pt]
\hangindent=23pt\noindent\textbf{Козеренко~Е.\,Б.} Лингвистические фильтры в статистических моделях машинного\linebreak
\vspace*{-12pt}\\
\hspace*{23pt}перевода$\dotfill$&2&83\\
\hangindent=23pt\noindent\textbf{Козеренко~Е.\,Б., Кузнецов~И.\,П.} Когнитивно-лингвистические представления в 
систе-\linebreak
\vspace*{-12pt}\\
\hspace*{23pt}мах обработки текстов$\dotfill$&3&69\\
\textbf{Князь~В.\,А.} см.~Каратеев~С.\,Л.&&\\
\hangindent=23pt\noindent\textbf{Колесников~А.\,В., Солдатов~С.\,А.} Алгоритм координации для гибридной 
интеллектуальной системы решения сложной задачи оперативно-производственного\linebreak
\vspace*{-12pt}\\
\hspace*{23pt}планирования$\dotfill$&4&61\\
\hangindent=23pt\noindent\textbf{Коновалов~М.\,Г.} О планировании потоков в системах вычислительных 
ресурсов$\dotfill$&2&3\\
\textbf{Конушин~А.\,С.} см.~Конушин~В.\,С.&&\\
\hangindent=23pt\noindent\textbf{Конушин~В.\,С., Кривовязь~Г.\,Р., Конушин~А.\,С.} Алгоритм распознавания людей 
в видео-\linebreak
\vspace*{-12pt}\\
\hspace*{23pt}последовательности по одежде$\dotfill$&1&74\\
\textbf{Корепанов~Э.\, Р.} см.~Синицын~И.\,Н.&&\\
\textbf{Королев~В.\,Ю.} см.~Соколов~И.\,А.&&\\
\textbf{Королев~Р.\,А.} см.~Бенинг~В.\,Е.&&\\
\textbf{Коротышева~А.\,В.} см.~Зейфман~А.\,И.&&\\
\hangindent=23pt\noindent\textbf{Кривенко~М.\,П.} Непараметрическое оценивание элементов байесовского 
клас\-си-\linebreak
\vspace*{-12pt}\\
\hspace*{23pt}фикатора$\dotfill$&2&13\\
\textbf{Кривовязь~Г.\,Р.} см.~Конушин~В.\,С.&&\\
\textbf{Крылов~А.\,С.} см.~Павельева~Е.\,А.&&\\
\hangindent=23pt\noindent\textbf{Крылов~В.\,А.} Моделирование и классификация многоканальных дистанционных\linebreak
\vspace*{-12pt}\\
\hspace*{23pt}изображений с использованием копул$\dotfill$&4&34\\
\hangindent=23pt\noindent\textbf{Крючин~О.\,В.} Разработка параллельных эвристических алгоритмов подбора 
весовых\linebreak
\vspace*{-12pt}\\
\hspace*{23pt}коэффициентов искусственной нейтронной сети$\dotfill$&2&53\\
\hangindent=23pt\noindent\textbf{Кудрявцев~А.\,А., Шоргин~С.\,Я.} Байесовские модели массового обслуживания и 
надеж-\linebreak
\vspace*{-12pt}\\
\hspace*{23pt}ности: характеристики среднего числа заявок в системе $M\vert M \vert 1\vert 
\infty$$\dotfill$&3&16\\
\hangindent=23pt\noindent\textbf{Кузнецов~А.\,А.} Связь между временными и структурно-топологическими 
характери-\linebreak
\vspace*{-12pt}\\
\hspace*{23pt}стиками диаграмм ритма сердца здоровых людей$\dotfill$&4&39\\
\textbf{Кузнецов~И.\,П.} см.~Козеренко~Е.\,Б.&&\\
\textbf{Ле~Пезан~Д.} см.~Бунтман~Н.\,В.&&\\
\hangindent=23pt\noindent\textbf{Лукьяненко~А.\,С., Морозов~Е.\,В., Гуртов~А.\,В.} Анализ сетевого протокола с общей 
функ-\linebreak
\vspace*{-12pt}\\
\hspace*{23pt}цией расширения окна передачи сообщения при конфликтах$\dotfill$&2&46\\
\hangindent=23pt\noindent\textbf{Лямин~О.\,О.} О предельном поведении мощностей критериев в случае обобщенного\linebreak
\vspace*{-12pt}\\
\hspace*{23pt}распределения Лапласа$\dotfill$&3&47\\
\hangindent=23pt\noindent\textbf{Маркин~А.\,В., Шестаков~О.\,В.} Асимптотики оценки риска при пороговой 
обработке\linebreak
\vspace*{-12pt}\\
\hspace*{23pt}вейвлет-вейглет коэффициентов в задаче томографии$\dotfill$&2&36\\
\hangindent=23pt\noindent\textbf{Матвеева~С.\,С., Захарова~Т.\,В.} Сети массового обслуживания с наименьшей 
длиной\linebreak
\vspace*{-12pt}\\
\hspace*{23pt}очереди$\dotfill$&3&22\\
\hangindent=23pt\noindent\textbf{Матюшенко~С.\,И.} Стационарные характеристики двухканальной системы 
обслужива-\linebreak
\vspace*{-12pt}\\
\hspace*{23pt}ния с переупорядочиванием заявок и распределениями фазового типа$\dotfill$&4&68\\
\textbf{Минель~Ж.-Л.} см.~Бунтман~Н.\,В.&&\\
\textbf{Морозов~Е.\,В.} см.~Бородина~А.\,В.&&\\
\textbf{Морозов~Е.\,В.} см.~Лукьяненко~А.\,С.&&\\
\textbf{Ососков~М.\,В.} см.~Каратеев~С.\,Л.&&\\
\hangindent=23pt\noindent\textbf{Павельева~Е.\,А., Крылов~А.\,С.} Поиск и анализ ключевых точек радужной 
оболочки\linebreak
\vspace*{-12pt}\\
\hspace*{23pt}глаза методом преобразования Эрмита$\dotfill$&1&79\\
\textbf{Печинкин~А.\,В.} см.~Френкель~С.\,Л.,&&\\
\hangindent=23pt\noindent\textbf{Протасов~В.\,И.} Составление субъективного портрета с использованием 
эволюционно-\linebreak
\vspace*{-12pt}\\
\hspace*{23pt}го морфинга и квалиметрия метода$\dotfill$&1&83\\
\hangindent=23pt\noindent\textbf{Рудаков~К.\,В., Торшин~И.\,Ю.} Вопросы разрешимости задачи распознавания 
вторичной\linebreak
\vspace*{-12pt}\\
\hspace*{23pt}структуры белка$\dotfill$&2&25\\
\textbf{Сатин~Я.\,А.} см.~Зейфман~А.\,И.&&\\
\hangindent=23pt\noindent\textbf{Сейфуль-Мулюков~Р.\,Б.} Нефть как носитель информации о своем 
происхождении,\linebreak
\vspace*{-12pt}\\
\hspace*{23pt}структуре и эволюции$\dotfill$&1&41\\
\end{tabular}
}

{\tabcolsep=3pt
\begin{tabular}{p{388pt}rr}
&\textbf{Выпуск} & \textbf{Стр.}\\[6pt]
\textbf{Семендяев~Н.\,Н.} см.~Синицын~И.\,Н.&&\\
\textbf{Серебряков~В.\,А.} см.~Захаров~А.\,А.&&\\
\textbf{Синицын~В.\,И.} см.~Синицын~И.\,Н.&&\\
\hangindent=23pt\noindent\textbf{Синицын~И.\,Н., Синицын~В.\,И., Корепанов~Э.\, Р., Белоусов~В.\,В., 
Семендяев~Н.\,Н.} Оперативное построение информационных моделей движения полюса 
Земли\linebreak
\vspace*{-12pt}\\
\hspace*{23pt}методами линейных и линеаризованных фильтров$\dotfill$&1&2\\
\textbf{Сипина~А.\,В.} см.~Бенинг~В.\,Е.&&\\
\hangindent=23pt\noindent\textbf{Соколов~И.\,А.} О работах заслуженного деятеля науки Российской Федерации 
И.\,Н.~Синицына в области информационных технологий и автоматизации (к 70-летию\linebreak
\vspace*{-12pt}\\
\hspace*{23pt}со дня рождения)$\dotfill$&3&84\\
\textbf{Соколов~И.\,А.} см.~Илюшин~Г.\,Я.&&\\
\hangindent=23pt\noindent\textbf{Соколов~И.\,А., Королев~В.\,Ю.} Предисловие$\dotfill$&2&2\\
\textbf{Солдатов~С.\,А.} см.~Колесников~А.\,В.&&\\
\hangindent=23pt\noindent\textbf{Степанов~С.\,Ю.} Использование координатного метода фрагментации 
коммутаторной\linebreak
\vspace*{-12pt}\\
\hspace*{23pt}нейронной сети для сокращения трафика$\dotfill$&2&57\\
\textbf{Тимонина~Е.\,Е.} см.~Грушо~А.\,А.&&\\
\textbf{Торшин~И.\,Ю.} см.~Рудаков~К.\,В.&&\\
\textbf{Ульянов~В.\,В.} см.~Кавагучи~Ю.&&\\
\textbf{Фазекаш~И.} см.~Чупрунов~А.\,Н.&&\\
\textbf{Френкель~С.\,Л.} см.~Баранов~С.\,И.&&\\
\hangindent=23pt\noindent\textbf{Френкель~С.\,Л., Печинкин~А.\,В.} Оценка времени самовосстановления в 
цифровых\linebreak
\vspace*{-12pt}\\
\hspace*{23pt}системах после сбоев, вызываемых переходными помехами$\dotfill$&3&2\\
\textbf{Фуджикоши~Я.} см.~Кавагучи~Ю.&&\\
\hangindent=23pt\noindent\textbf{Цискаридзе~А.\,К.} Математическая модель и метод восстановления позы человека 
по\linebreak
\vspace*{-12pt}\\
\hspace*{23pt}стереопаре силуэтных изображений$\dotfill$&4&27\\
\hangindent=23pt\noindent\textbf{Чупраков~К.\,Г.} К вопросу о размещении коллективных средств отображения в 
ситуа-\linebreak
\vspace*{-12pt}\\
\hspace*{23pt}ционном зале с заданными параметрами$\dotfill$&4&89\\
\textbf{Чупраков~К.\,Г.} см.~Зацаринный~А.\,А.&&\\
\hangindent=23pt\noindent\textbf{Чупрунов~А.\,Н., Фазекаш~И.} Законы повторного логарифма для числа 
безошибочных\linebreak
\vspace*{-12pt}\\
\hspace*{23pt}блоков при помехоустойчивом кодировании$\dotfill$&3&42\\
\textbf{Шевцова~И.\,Г.} см.~Григорьева~М.\,Е.&&\\
\hangindent=23pt\noindent\textbf{Шестаков~О.\,В.} Аппроксимация распределения оценки риска пороговой 
обработки вейвлет-коэффициентов нормальным распределением при использовании 
выбо-\linebreak
\vspace*{-12pt}\\
\hspace*{23pt}рочной дисперсии$\dotfill$&4&73\\
\textbf{Шестаков~О.\,В.} см.~Маркин~А.\,В.&&\\
\textbf{Шоргин~С.\,Я.} см.~Зейфман~А.\,И.&&\\
\textbf{Шоргин~С.\,Я.} см.~Кудрявцев~А.\,А.&&\\
\end{tabular}
}

%\thispagestyle{myheadings}
\def\leftfootline{\small{\textbf{\thepage}
\hfill ИНФОРМАТИКА И ЕЁ ПРИМЕНЕНИЯ\ \ \ том~4\ \ \ выпуск~4\ \ \ 2010}
}%
 \def\rightfootline{\small{ИНФОРМАТИКА И ЕЁ ПРИМЕНЕНИЯ\ \ \ том~4\ \ \ выпуск~4\ \ \ 2010
 \hfill \textbf{\thepage}}}
 \label{end\stat}
%
%Том 10 Выпуск 1-4 Год 2016

\def\stat{cont-e}
{%\hrule\par
%\vskip 7pt % 7pt
\raggedleft\Large \bf%\baselineskip=3.2ex
2\,0\,1\,6\ \ A\,U\,T\,H\,O\,R\ \ I\,N\,D\,E\,X \vskip 17pt
 \hrule
 \par
\vskip 21pt plus 6pt minus 3pt }

\label{st\stat}

\def\tit{\ }

\def\aut{\ }
\def\auf{\ }

\def\leftkol{\ } %2016 AUTHOR INDEX} % ENGLISH ABSTRACTS}

\def\rightkol{\ } %2016 AUTHOR INDEX} %ENGLISH ABSTRACTS}

\titele{\tit}{\aut}{\auf}{\leftkol}{\rightkol}

\def\leftfootline{\small{\textbf{\thepage}
\hfill INFORMATIKA I EE PRIMENENIYA~--- INFORMATICS AND APPLICATIONS\ \ \ 2016\
\ \ volume~10\ \ \ issue\ 4}
}%
 \def\rightfootline{\small{INFORMATIKA I EE PRIMENENIYA~--- INFORMATICS AND APPLICATIONS\ \ \ 2016\ \ \ volume~10\ \ \ issue\ 4
\hfill \textbf{\thepage}}}

\vspace*{-12pt}
\vspace*{-18pt}

{\tabcolsep=2.8pt
\begin{tabular}{p{382pt}cc}
&\textbf{Issue} & \textbf{Page}\\[6pt]
\Avtors{Agalarov~M.\,Ya.} see~Agalarov~Ya.\,M.&&\\
\Avtors{Agalarov~Ya.\,M., Agalarov~M.\,Ya., and
Shorgin~V.\,S.} About the optimal threshold of queue\linebreak
\\[-12pt]
\hspace*{23pt}length in a~particular problem of profit maximization
in the $M/G/1$ queuing system&2&70--79\\
\Avtors{Alexeyevsky~D.\,A.} BioNLP ontology extraction from 
a~restricted language corpus with\linebreak
\\[-12pt]
\hspace*{23pt}context-free grammars&1&119--128\\
\Avtors{Andreev~S.\,D.} see~Gaidamaka~Yu.\,V.&&\\
\Avtors{Andreev~S.\,D.} see~Ometov~A.\,Ya.&&\\
\Avtors{Arkhipov~O.\,P., Arkhipov~P.\,O., and Sidorkin~I.\,I.} The
option to create a~local coordinate\linebreak
\\[-12pt]
\hspace*{23pt}system for synchronization of selected images&3&91--97\\
\Avtors{Arkhipov~P.\,O.} see~Arkhipov~O.\,P.&&\\
\Avtors{Belousov~V.\,V.} see~Shnurkov~P.\,V.&&\\
\Avtors{Belousov~V.\,V.} see~Shnurkov~P.\,V.&&\\
\Avtors{Bening~V.\,E.} Calculation of~the~asymptotic deficiency
of~some statistical procedures based\linebreak
\\[-12pt]
\hspace*{23pt}on~samples with~random sizes&4&34--45\\
\Avtors{Borisov~A.\,V., Bosov~A.\,V., and Miller~G.\,B.} Modeling and
monitoring of VoIP connection&2&\hphantom{1}2--13\\
\Avtors{Bosov~A.\,V.} see~Borisov~A.\,V.&&\\
\Avtors{Briukhov~D.\,O.} see~Stupnikov~S.\,A.&&\\
\Avtors{Callaos~N.\,K.\ and Seyful-Mulyukov~R.\,B.} Complexity and
its information content&1&129--139\\
\Avtors{Chertok~A.\,V., Kadaner~A.\,I., Khazeeva~G.\,T., and
Sokolov~I.\,A.} Regime switching detection\linebreak
\\[-12pt]
\hspace*{23pt}for~the~Levy driven
Ornstein--Uhlenbeck process using CUSUM methods&4&46--56\\
\Avtors{Chichagov~V.\,V.} Asymptotic expansions of mean absolute
error of uniformly minimum variance unbiased and maximum likelihood
estimators on the one-parameter exponential\linebreak
\\[-12pt]
\hspace*{23pt}family model of lattice distributions&3&66--76\\
\Avtors{Danishevsky~V.\,I.} see~Kolesnikov A.\,V.&&\\
\Avtors{Fazliev~A.\,Z.} see~Kalinichenko~L.\,A.&&\\
\Avtors{Fedoseev~A.\,A.} What is behind the concept of ``knowledge in
small packages''&3&105--110\\
\Avtors{Gaidamaka~Yu.\,V., Andreev~S.\,D., Sopin~E.\,S.,
Samouylov~K.\,E., and Shorgin~S.\,Ya.} Interference analysis
of~the~device-to-device communications model with~regard to~a~signal\linebreak
\\[-12pt]
\hspace*{23pt}propagation environment&4&\hphantom{1}2--10\\
\Avtors{Gasilov~A.\,V.} see~Yakovlev~O.\,A.&&\\
\Avtors{Goncharov~A.\,V.\ and Strijov~V.\,V.} Metric time series
classification using weighted dynamic\linebreak
\\[-12pt]
\hspace*{23pt}warping relative to centroids of classes&2&36--47\\
\Avtors{Gordov~E.\,P.} see~Kalinichenko~L.\,A.&&\\
\Avtors{Gorshenin~A.\,K.} Concept of online service for stochastic
modeling of real processes&1&72--81\\
\Avtors{Gorshenin~A.\,K.} see~Shnurkov~P.\,V.&&\\
\Avtors{Gorshenin~A.\,K.} see~Shnurkov~P.\,V.&&\\
\Avtors{Grusho~A.\,A., Grusho~N.\,A., Zabezhailo~M.\,I., and
Timonina~E.\,E.} Integration of statistical and\linebreak
\\[-12pt]
\hspace*{23pt}deterministic methods for
analysis of information security&3&2--8\\
\Avtors{Grusho~A.\,A., Zabezhailo~M.\,I., and Zatsarinny~A.\,A.} On
the advanced procedure to reduce\linebreak
\\[-12pt]
\hspace*{23pt}calculation of Galois closures&4&\hphantom{1}96--104\\
\Avtors{Grusho~N.\,A.} see~Grusho~A.\,A.&&\\
\Avtors{Havanskov~V.\,A.} see~Minin~V.\,A.&&\\
\Avtors{Inkova~O.\,Yu.} see~Zatsman~I.\,M.&&\\
\Avtors{Isachenko~R.\,V.\ and Strijov~V.\,V.} Metric learning in
multiclass time series classification\linebreak
\\[-12pt]
\hspace*{23pt}problem&2&48--57\\
\end{tabular}
}
\pagebreak

\def\leftfootline{\small{\textbf{\thepage}
\hfill INFORMATIKA I EE PRIMENENIYA~--- INFORMATICS AND APPLICATIONS\ \ \ 2016\
\ \ volume~10\ \ \ issue\ 4}
}%
 \def\rightfootline{\small{INFORMATIKA I EE PRIMENENIYA~---
INFORMATICS AND APPLICATIONS\ \ \ 2016\ \ \ volume~10\ \ \ issue\ 4
\hfill \textbf{\thepage}}}

\def\leftkol{2016 AUTHOR INDEX} % ENGLISH ABSTRACTS}

\def\rightkol{2016 AUTHOR INDEX} %ENGLISH ABSTRACTS}


{\tabcolsep=2.83pt
\begin{tabular}{p{382pt}cc}
&\textbf{Issue} & \textbf{Page}\\[6pt]
\Avtors{Kadaner~A.\,I.} see~Chertok~A.\,V.&&\\[.255pt]
\Avtors{Kalinichenko~L.\,A., Volnova~A.\,A., Gordov~E.\,P.,
Kiselyova~N.\,N., Kovaleva~D.\,A., Malkov~O.\,Yu., Okladnikov~I.\,G.,
Podkolodnyy~N.\,L., Pozanenko~A.\,S., Ponomareva~N.\,V.,
Stupnikov~S.\,A.,} \textbf{and Fazliev~A.\,Z.} Data access challenges for data
intensive\linebreak
\\[-12pt]
\hspace*{23pt}research in Russia&1& 2--22\\[.255pt]
\Avtors{Karasikov~M.\,E.\ and Strijov~V.\,V.} Feature-based
time-series classification&4&121--131\\[.255pt]
\Avtors{Khazeeva~G.\,T.} see~Chertok~A.\,V.&&\\[.255pt]
\Avtors{Khokhlov~Yu.\,S.} Multivariate fractional Levy motion and its
applications&2&\hphantom{1}98--106\\[.255pt]
\Avtors{Kirikov~I.\,A., Kolesnikov~A.\,V., Listopad~S.\,V., and
Rumovskaya~S.\,B.} Fine-grained hybrid\linebreak
\\[-12pt]
\hspace*{23pt}intelligent systems. Part 2:
Bidirectional hybridization&1&\hphantom{1}96--105\\[.255pt]
\Avtors{Kirikov~I.\,A., Kolesnikov~A.\,V., Listopad~S.\,V., and
Rumovskaya~S.\,B.} ``Virtual council''~---\linebreak
\\[-12pt]
\hspace*{23pt}source environment
supporting complex diagnostic decision making&3&81--90\\[.255pt]
\Avtors{Kiselyova~N.\,N.} see~Kalinichenko~L.\,A.&&\\[.255pt]
\Avtors{Kolesnikov A.\,V., Listopad~S.\,V., Rumovskaya~S.\,B., and
Danishevsky~V.\,I.} Informal axiomatic\linebreak
\\[-12pt]
\hspace*{23pt}theory of~the~role visual models&4&114--120\\[.255pt]
\Avtors{Kolesnikov~A.\,V.} see~Kirikov~I.\,A.&&\\[.255pt]
\Avtors{Kolesnikov~A.\,V.} see~Kirikov~I.\,A.&&\\[.255pt]
\Avtors{Kolin~K.\,K.} Humanitarian aspects of information
security&3&111--121\\[.255pt]
\Avtors{Konovalov~M.\,G.\ and Razumchik~R.\,V.} Dispatching
to~two parallel nonobservable queues using\linebreak
\\[-12pt]
\hspace*{23pt}only static
information&4&57--67\\[.255pt]
\Avtors{Korchagin~A.\,Yu.} see~Korolev~V.\,Yu.&&\\[.255pt]
\Avtors{Korchagin~A.\,Yu.} see~Korolev~V.\,Yu.&&\\[.255pt]
\Avtors{Korepanov~E.\,R.} see~Sinitsyn~I.\,N.&&\\[.255pt]
\Avtors{Korepanov~E.\,R.} see~Sinitsyn~I.\,N.&&\\[.255pt]
\Avtors{Korolev~V.\,Yu., Korchagin~A.\,Yu., and Zeifman~A.\,I.} The
Poisson theorem for Bernoulli trials\linebreak
\\[-12pt]
\hspace*{23pt}with~a~random probability
of~success and~a~discrete analog of~the~Weibull distribution&4&11--20\\[.255pt]
\Avtors{Korolev~V.\,Yu., Zeifman~A.\,I., and Korchagin~A.\,Yu.}
Asymmetric Linnik distributions as~limit\linebreak
\\[-12pt]
\hspace*{23pt}laws for~random sums
of~independent random variables with~finite variances&4&21--33\\[.255pt]
\Avtors{Koucheryavy~E.\,A.} see~Ometov~A.\,Ya.&&\\[.255pt]
\Avtors{Kovaleva~D.\,A.} see~Kalinichenko~L.\,A.&&\\[.255pt]
\Avtors{Kovalyov~S.\,P.} Metaprogramming to increase
manufacturability of large-scale software-\linebreak
\\[-12pt]
\hspace*{23pt}intensive systems&1&56--66\\[.255pt]
\Avtors{Krivenko~M.\,P.} Significance tests of feature selection for
classification&3&32--40\\[.255pt]
\Avtors{Kruzhkov~M.\,G.} see~Zalizniak~Anna~A.&&\\[.255pt]
\Avtors{Kruzhkov~M.\,G.} see~Zatsman~I.\,M.&&\\[.255pt]
\Avtors{Kudryavtsev~A.\,A.} Bayesian queueing and reliability models:
\textit{A~priori} distributions with\linebreak
\\[-12pt]
\hspace*{23pt}compact support&1&67--71\\[.255pt]
\Avtors{Kudryavtsev~A.\,A.} Characteristics dependent on the balance
coefficient in Bayesian models\linebreak
\\[-12pt]
\hspace*{23pt}with compact support of \textit{a priori}
distributions&3&77--80\\[.255pt]
\Avtors{Kudryavtsev~A.\,A.\ and Palionnaia~S.\,I.} Bayesian recurrent
model of reliability growth:\linebreak
\\[-12pt]
\hspace*{23pt}Parabolic distribution of parameters&2&80--83\\[.255pt]
\Avtors{Kudryavtsev~A.\,A.\ and Titova~A.\,I.} Bayesian queuing
and~reliability models: Degenerate-\linebreak
\\[-12pt]
\hspace*{23pt}Weibull case&4&68--71\\[.255pt]
\Avtors{Leontyev~N.\,D.\ and Ushakov~V.\,G.} Analysis of a queueing
system with autoregressive arrivals\linebreak
\\[-12pt]
\hspace*{23pt}and nonpreemptive priority&3&15--22\\[.255pt]
\Avtors{Listopad~S.\,V.} see~Kirikov~I.\,A.&&\\[.255pt]
\Avtors{Listopad~S.\,V.} see~Kirikov~I.\,A.&&\\[.255pt]
\Avtors{Listopad~S.\,V.} see~Kolesnikov A.\,V.&&\\[.255pt]
\Avtors{Malkov~O.\,Yu.} see~Kalinichenko~L.\,A.&&\\[.255pt]
\Avtors{Markov~A.\,S., Monakhov~M.\,M., and
Ulyanov~V.\,V.} Generalized Cornish--Fisher expansions\linebreak
\\[-12pt]
\hspace*{23pt}for distributions of statistics based on samples
of random size&2&84--91\\[.255pt]
\Avtors{Melnikov~A.\,K.\ and Ronzhin~A.\,F.} Generalized statistical
method of~text analysis based\linebreak
\\[-12pt]
\hspace*{23pt}on~calculation of~probability distributions
of~statistical values&4&89--95\\
\end{tabular}
}
\pagebreak

\def\leftfootline{\small{\textbf{\thepage}
\hfill INFORMATIKA I EE PRIMENENIYA~--- INFORMATICS AND APPLICATIONS\ \ \ 2016\
\ \ volume~10\ \ \ issue\ 4}
}%
 \def\rightfootline{\small{INFORMATIKA I EE PRIMENENIYA~---
INFORMATICS AND APPLICATIONS\ \ \ 2016\ \ \ volume~10\ \ \ issue\ 4
\hfill \textbf{\thepage}}}

\def\leftkol{2016 AUTHOR INDEX} % ENGLISH ABSTRACTS}

\def\rightkol{2016 AUTHOR INDEX} %ENGLISH ABSTRACTS}


{\tabcolsep=3pt
\begin{tabular}{p{381pt}cc}
&\textbf{Issue} & \textbf{Page}\\[6pt]
\Avtors{Meykhanadzhyan~L.\,A.} Stationary characteristics of the finite
capacity queueing system with\linebreak
\\[-12pt]
\hspace*{23pt}inverse service order and generalized
probabilistic priority&2&123--131\\[.23pt]
\Avtors{Miller~G.\,B.} see~Borisov~A.\,V.&&\\[.23pt]
\Avtors{Minin~V.\,A., Zatsman~I.\,M., Havanskov~V.\,A., and
Shubnikov~S.\,K.} Intensity of citation of scientific publications in
inventions on information and computer technologies patented\linebreak
\\[-12pt]
\hspace*{23pt}in Russia by domestic and foreign applicants&2&107--122\\[.23pt]
\Avtors{Monakhov~M.\,M.} see~Markov~A.\,S.&&\\[.23pt]
\Avtors{Naumov~V.\,A.\ and Samouylov~K.\,E.} On relationship
between queuing systems with resources\linebreak
\\[-12pt]
\hspace*{23pt}and Erlang networks&3&\hphantom{1}9--14\\[.23pt]
\Avtors{Okladnikov~I.\,G.} see~Kalinichenko~L.\,A.&&\\[.23pt]
\Avtors{Ometov~A.\,Ya., Andreev~S.\,D., Turlikov~A.\,M., and
Koucheryavy~E.\,A.} Performance analysis of\linebreak
\\[-12pt]
\hspace*{23pt}a wireless data
aggregation system with contention for contemporary sensor
networks&3&23--31\\[.23pt]
\Avtors{Palionnaia~S.\,I.} see~Kudryavtsev~A.\,A.&&\\[.23pt]
\Avtors{Podkolodnyy~N.\,L.} see~Kalinichenko~L.\,A.&&\\[.23pt]
\Avtors{Ponomareva~N.\,V.} see~Kalinichenko~L.\,A.&&\\[.23pt]
\Avtors{Popkova~N.\,A.} see~Zatsman~I.\,M.&&\\[.23pt]
\Avtors{Pozanenko~A.\,S.} see~Kalinichenko~L.\,A.&&\\[.23pt]
\Avtors{Razumchik~R.\,V.} see~Konovalov~M.\,G.&&\\[.23pt]
\Avtors{Ronzhin~A.\,F.} see~Melnikov~A.\,K.&&\\[.23pt]
\Avtors{Rumovskaya~S.\,B.} see~Kirikov~I.\,A.&&\\[.23pt]
\Avtors{Rumovskaya~S.\,B.} see~Kirikov~I.\,A.&&\\[.23pt]
\Avtors{Rumovskaya~S.\,B.} see~Kolesnikov A.\,V.&&\\[.23pt]
\Avtors{Samouylov~K.\,E.} see~Gaidamaka~Yu.\,V.&&\\[.23pt]
\Avtors{Samouylov~K.\,E.} see~Naumov~V.\,A.&&\\[.23pt]
\Avtors{Serebryanskii~S.\,M.} see~Tyrsin~A.\,N.&&\\[.23pt]
\Avtors{Seyful-Mulyukov~R.\,B.} see~Callaos~N.\,K.&&\\[.23pt]
\Avtors{Shestakov~O.\,V.} Statistical properties of the denoising method
based on the stabilized hard\linebreak
\\[-12pt]
\hspace*{23pt}thresholding&2&65--69\\[.23pt]
\Avtors{Shestakov~O.\,V.} The strong law of large numbers for the risk
estimate in the problem of\linebreak
\\[-12pt]
\hspace*{23pt}tomographic image reconstruction from
projections with a correlated noise&3&41--45\\[.23pt]
\Avtors{Shestakov~O.\,V.} see~Zakharova~T.\,V.&&\\[.23pt]
\Avtors{Shnurkov~P.\,V., Gorshenin~A.\,K., and Belousov~V.\,V.}
Analytical solution of~the~optimal control\linebreak
\\[-12pt]
\hspace*{23pt}task of~a~semi-Markov
process with~finite set of~states&4&72--88\\[.23pt]
\Avtors{Shnurkov~P.\,V., Zasypko~V.\,V., Belousov~V.\,V., and
Gorshenin~A.\,K.} Development of the algorithm of numerical solution
of the optimal investment control problem\linebreak
\\[-12pt]
\hspace*{23pt}in the closed dynamical model of three-sector economy&1&82--95\\[.23pt]
\Avtors{Shorgin~S.\,Ya.} see~Gaidamaka~Yu.\,V.&&\\[.23pt]
\Avtors{Shorgin~V.\,S.} see~Agalarov~Ya.\,M.&&\\[.23pt]
\Avtors{Shubnikov~S.\,K.} see~Minin~V.\,A.&&\\[.23pt]
\Avtors{Sidorkin~I.\,I.} see~Arkhipov~O.\,P.&&\\[.23pt]
\Avtors{Sinitsyn~I.\,N.} Analytical modeling of processes in stochastic
systems with complex fractional\linebreak
\\[-12pt]
\hspace*{23pt}order Bessel nonlinearities&3&55--65\\[.23pt]
\Avtors{Sinitsyn~I.\,N.} Orthogonal supoptimal filters for nonlinear
stochastic systems on manifolds&1&34--44\\[.23pt]
\Avtors{Sinitsyn~I.\,N.\ and Korepanov~E.\,R.} Normal Pugachev
conditionally-optimal filters and extra-\linebreak
\\[-12pt]
\hspace*{23pt}polators for state linear stochastic systems&2&14--23\\[.23pt]
\Avtors{Sinitsyn~I.\,N.\ and Sinitsyn~V.\,I.} Analytical modeling of
distributions in stochastic systems on\linebreak
\\[-12pt]
\hspace*{23pt}manifolds based on ellipsoidal approximation&1&45--55\\[.23pt]
\Avtors{Sinitsyn~I.\,N., Sinitsyn~V.\,I., and
Korepanov~E.\,R.} Ellipsoidal suboptimal filters for nonlinear\linebreak
\\[-12pt]
\hspace*{23pt}stochastic systems on manifolds&2&24--35\\[.23pt]
\Avtors{Sinitsyn~V.\,I.} see~Sinitsyn~I.\,N.&&\\[.23pt]
\Avtors{Sinitsyn~V.\,I.} see~Sinitsyn~I.\,N.&&\\[.23pt]
\Avtors{Skvortsov~N.\,A.} see~Stupnikov~S.\,A.&&\\[.23pt]
\Avtors{Sokolov~I.\,A.} see~Chertok~A.\,V.&&\\
\end{tabular}
}
\pagebreak

\def\leftfootline{\small{\textbf{\thepage}
\hfill INFORMATIKA I EE PRIMENENIYA~--- INFORMATICS AND APPLICATIONS\ \ \ 2016\
\ \ volume~10\ \ \ issue\ 4}
}%
 \def\rightfootline{\small{INFORMATIKA I EE PRIMENENIYA~---
INFORMATICS AND APPLICATIONS\ \ \ 2016\ \ \ volume~10\ \ \ issue\ 4
\hfill \textbf{\thepage}}}

\def\leftkol{2016 AUTHOR INDEX} % ENGLISH ABSTRACTS}

\def\rightkol{2016 AUTHOR INDEX} %ENGLISH ABSTRACTS}


{\tabcolsep=3pt
\begin{tabular}{p{382pt}cc}
&\textbf{Issue} & \textbf{Page}\\[6pt]
\Avtors{Sopin~E.\,S.} see~Gaidamaka~Yu.\,V.&&\\
\Avtors{Strijov~V.\,V.} see~Goncharov~A.\,V.&&\\
\Avtors{Strijov~V.\,V.} see~Isachenko~R.\,V.&&\\
\Avtors{Strijov~V.\,V.} see~Karasikov~M.\,E.&&\\
\Avtors{Stupnikov~S.\,A., Briukhov~D.\,O., and Skvortsov~N.\,A.}
Co-lending systemic risk analysis over\linebreak
\\[-12pt]
\hspace*{23pt}heterogeneous data collections&1&23--33\\
\Avtors{Stupnikov~S.\,A.} see~Kalinichenko~L.\,A.&&\\
\Avtors{Suchkov~A.\,P.} see~Zatsarinny~A.\,A.&&\\
\Avtors{Timonina~E.\,E.} see~Grusho~A.\,A.&&\\
\Avtors{Titova~A.\,I.} see~Kudryavtsev~A.\,A.&&\\
\Avtors{Turlikov~A.\,M.} see~Ometov~A.\,Ya.&&\\
\Avtors{Tyrsin~A.\,N.\ and Serebryanskii~S.\,M.} Recognition of
dependences on the basis of inverse\linebreak
\\[-12pt]
\hspace*{23pt}mapping&2&58--64\\
\Avtors{Ulyanov~V.\,V.} see~Markov~A.\,S.&&\\
\Avtors{Ushakov~V.\,G.} Queueing system with working vacations and
hyperexponential input stream&2&92--97\\
\Avtors{Ushakov~V.\,G.} see~Leontyev~N.\,D.&&\\
\Avtors{Volnova~A.\,A.} see~Kalinichenko~L.\,A.&&\\
\Avtors{Yakovlev~O.\,A.\ and Gasilov~A.\,V.} Speeded-up stereo
matching using geodesic support weights&3&\hphantom{1}98--104\\
\Avtors{Zabezhailo~M.\,I.} see~Grusho~A.\,A.&&\\
\Avtors{Zabezhailo~M.\,I.} see~Grusho~A.\,A.&&\\
\Avtors{Zakharova~T.\,V.\ and Shestakov~O.\,V.} Precision analysis of
wavelet processing of aerodynamic\linebreak
\\[-12pt]
\hspace*{23pt}flow patterns&3&46--54\\
\Avtors{Zalizniak~Anna~A.\ and Kruzhkov~M.\,G.} Database
of~Russian impersonal verbal constructions&4&132--141\\
\Avtors{Zasypko~V.\,V.} see~Shnurkov~P.\,V.&&\\
\Avtors{Zatsarinny~A.\,A.\ and Suchkov~A.\,P.} Systems engineering
approaches to~the~establishment of\linebreak
\\[-12pt]
\hspace*{23pt}a~system for~decision support based
on~situational analysis&4&105--113\\
\Avtors{Zatsarinny~A.\,A.} see~Grusho~A.\,A.&&\\
\Avtors{Zatsman~I.\,M., Inkova~O.\,Yu., Kruzhkov~M.\,G., and
Popkova~N.\,A.} Representation of cross-\linebreak
\\[-12pt]
\hspace*{23pt}lingual knowledge about
connectors in supracorpora databases&1&106--118\\
\Avtors{Zatsman~I.\,M.} see~Minin~V.\,A.&&\\
\Avtors{Zeifman~A.\,I.} see~Korolev~V.\,Yu.&&\\
\Avtors{Zeifman~A.\,I.} see~Korolev~V.\,Yu.&&\\
\end{tabular}
}

%\thispagestyle{myheadings}
\def\leftfootline{\small{\textbf{\thepage}
\hfill INFORMATIKA I EE PRIMENENIYA~--- INFORMATICS AND APPLICATIONS\ \ \ 2016\
\ \ volume~10\ \ \ issue\ 4}
}%
 \def\rightfootline{\small{INFORMATIKA I EE PRIMENENIYA~---
INFORMATICS AND APPLICATIONS\ \ \ 2016\ \ \ volume~10\ \ \ issue\ 4
\hfill \textbf{\thepage}}}

 \label{end\stat}

\newpage

%\def\stat{rekl}
%\label{preobr}

%\def\tit{АКАДЕМИК ПУГАЧЁВ  ВЛАДИМИР СЕМЁНОВИЧ\\
%25.03.1911--25.03.1998}


%   \vspace*{-48pt}
%   \begin{center}\LARGE
%Академик Пугачёв  Владимир Семёнович\\ (25.03.1911--25.03.1998)
%   \end{center}
   
   %\vspace*{2.5mm}
   
   \begin{center}

{\prgsh\LARGE
ОБЪЯВЛЕНИЯ О КОНФЕРЕНЦИЯХ}

\end{center}
%\hrule

\vspace*{6pt}

   
   \vspace*{10mm}
   
   \thispagestyle{empty}

\noindent
\begin{tabular}{cc}
%\begin{center}
\multicolumn{1}{c}{\raisebox{-40pt}[0pt][0pt]{\mbox{%
\epsfxsize=33mm
\epsfbox{vspu.eps}
}}}
%\end{center}
&
\tabcolsep=0pt\begin{tabular}{c}
{\prg{\Large\textbf{XII Всероссийское совещание}}}\\[6pt]
{\prg{\Large\textbf{по проблемам управления}}}\\[12pt]
{\prg{\large 16--19 июня 2014~г.}}\\[6pt] 
{\prg{\large Институт проблем управления имени В.\,А.~Трапезникова РАН}}\\[6pt]
{\prg{\large Москва, Россия}}
\end{tabular}
\end{tabular}

\vspace*{60pt}

     
 { %\large    
 XII Всероссийское совещание по проблемам управления (ВСПУ XII), посвященное 75-летию 
Института проблем управления (ИПУ) имени В.\,А.~Трапезникова РАН, проводится 16--19~июня 
2014~г.\ 
в ИПУ РАН (г.~Москва, Россия). ВСПУ XII организуется ИПУ РАН при поддержке РФФИ, Отделения 
энергетики, машиностроения, механики и процессов управления Российской академии наук, 
Российского 
национального комитета по автоматическому управлению, Академии навигации и управ\-ле\-ния 
движением, 
Научного совета РАН по комплексным проблемам управления и автоматизации, Совета по 
мехатронике и робототехнике РАН. Официальный язык Совещания~--- русский.

\vspace*{24pt}
     
     \textbf{Направления работы}
     \begin{enumerate}[1.]
\item Теория систем управления
\item Управление подвижными объектами и навигация
\item Интеллектуальные системы управления
\item Управление в промышленности, транспортом и логистикой
\item Управление системами междисциплинарной природы
\item Средства измерения, вычислений и контроля в управлении
\item Системный анализ и принятие решений в задачах управления
\item Информационные технологии в управлении
\item Проблемы образования в области управления: современное содержание и технологии обучения
\end{enumerate}

\vspace*{24pt}

     Подробная информация о Совещании находится на сайте {\sf http://vspu2014.ipu.ru}. Срок 
окончательной подачи докладов через систему подачи докладов на сайте~--- \textbf{30~ноября} 
2013~г.
}

%\include{rekl-1}

%\end{document}

%\include{nekrolog-rb}


%\end{document}

%\include{IPPM-25}

\def\stat{cont-rus}
{%\hrule\par
%\vskip 7pt % 7pt
\vspace*{-24pt}
\raggedleft\Large \bf%\baselineskip=3.2ex
Правила подготовки рукописей  для публикации в журнале
<<Информатика~и~её~применения>> \vskip 8pt
    \hrule
    \par
\vskip 14pt plus 6pt minus 3pt }

\label{st\stat}

\def\tit{\ }

\def\aut{\ }
\def\auf{\ }

\def\leftkol{\ }
% Правила подготовки рукописей  для публикации в журнале
%<<Информатика и её применения>>

\def\rightkol{\ }
%Правила подготовки рукописей  для публикации в журнале
%<<Информатика и её применения>>}


\titele{\tit}{\aut}{\auf}{\leftkol}{\rightkol}


\vspace*{-60pt}
{ %\small

Журнал <<Информатика и её применения>>
публикует теоретические, обзорные и дискуссионные статьи,
посвященные научным исследованиям и разработкам в области
информатики и ее приложений.

Журнал издается на русском языке. По специальному решению
редколлегии отдельные статьи могут печататься на английском языке.

Тематика журнала охватывает следующие направления:
\begin{itemize}
\item теоретические основы информатики;\\[-15pt]
      \item
математические методы исследования сложных систем и процессов;\\[-15pt]
           \item
информационные системы и сети;\\[-15pt]
                \item
информационные технологии;\\[-15pt]
                     \item
архитектура и программное обеспечение вычислительных комплексов и сетей.\\[-15pt]
\end{itemize}


\noindent
\begin{enumerate}[1.]
\item В журнале печатаются статьи, содержащие результаты, ранее не опубликованные и
не предназначенные к одновременной публикации в других изданиях.

%Публикация не должна нарушать закон об авторских правах.
Публикация предоставленной автором(ами) рукописи не должна нарушать 
положений глав~69, 70 раздела~VII части~IV Гражданского кодекса, 
которые определяют права на результаты интеллектуальной деятельности 
и~средства индивидуализации, в~том числе авторские права, в~РФ.

Ответственность за нарушение авторских прав, в~случае предъявления претензий к~редакции журнала,  
несут авторы статей.



Направляя рукопись в редакцию, авторы сохраняют свои права на данную
рукопись и при этом передают учредителям и редколлегии журнала неисключительные права на
издание статьи на русском языке 
(или на языке статьи, если он отличен от рус\-ско\-го) и~на перевод ее на английский
язык, а~также на
ее распространение в России и за рубежом. 
Каждый автор должен представить в~редакцию подписанный 
с~его стороны <<Лицензионный договор о~передаче неисключительных прав 
на использование произведения>>, текст которого размещен по адресу 
{\sf http://www.ipiran.ru/publications/licence.doc}. 
Этот договор может быть пред\-став\-лен в~бумажном (в~2-х экз.)\ 
или в~электронном виде (отсканированная копия заполненного и~подписанного документа).




Редколлегия вправе запросить у авторов экспертное заключение о возможности
пуб\-ли\-ка\-ции пред\-став\-лен\-ной статьи в открытой печати.\\[-13.5pt]

\item К статье прилагаются данные автора (авторов) (см.\ п.~8). При наличии нескольких
авторов указывается фамилия автора, ответственного за переписку с редакцией.\\[-13.5pt]

\item Редакция журнала осуществляет экспертизу присланных статей в соответствии с
принятой в журнале процедурой рецензирования.

Возвращение рукописи на доработку не означает ее принятия к печати.

Доработанный вариант с ответом на замечания рецензента необходимо прислать в
редакцию.\\[-13.5pt]

\item Решение редколлегии о публикации статьи или ее отклонении сообщается авторам.

Редколлегия может также направить авторам текст рецензии на их статью. Дискуссия по
поводу отклоненных статей не ведется.\\[-13.5pt]

%\pagebreak

\item Редактура статей высылается авторам для просмотра. Замечания к редактуре должны
быть присланы авторами в кратчайшие сроки.\\[-13.5pt]

\item Рукопись предоставляется в электронном виде в форматах MS WORD (.doc или
.docx) или \LaTeX\  (.tex), дополнительно~--- в формате .pdf, на дискете, лазерном диске
или электронной почтой. Предоставление бумажной рукописи необязательно.\\[-13.5pt]

\item При подготовке рукописи в MS Word рекомендуется использовать следующие
настройки.

Параметры страницы:
формат~--- А4; ориентация~--- книжная; поля (см): внутри~--- 2,5, снаружи~--- 1,5,
сверху~--- 2, снизу~--- 2, от края до нижнего колонтитула~--- 1,3.

Основной текст: стиль~--- <<Обычный>>, шрифт~--- Times New Roman, размер~---
14~пунк\-тов, абзацный отступ~--- 0,5~см, 1,5~интервала, выравнивание~--- по ширине.

\pagebreak

\def\leftkol{Правила подготовки рукописей  для публикации в журнале
<<Информатика и её применения>>}

\def\rightkol{Правила подготовки рукописей  для публикации в журнале
<<Информатика и её применения>>}



Рекомендуемый объем рукописи~--- не свыше 10~страниц указанного формата.
При превышении указанного объема редколлегия вправе потребовать от 
автора сокращения объема рукописи.


Сокращения слов, помимо стандартных, не допускаются. Допускается минимальное
количество аббревиатур.


Все страницы рукописи нумеруются.

Шаблоны оформления представлены в интернете:

\noindent
 {\sf
http://www.ipiran.ru/journal/template\_iiep\_ssi\_2024.zip}\\[-14pt]

\item Статья должна содержать следующую информацию на {\bfseries\textit{русском и
английском языках}}:\\[-16pt]

\begin{itemize}
\item название статьи;\\[-15pt]
\item Ф.И.О.\ авторов, на английском можно только имя и фамилию;\\[-15pt]
\item место работы, с указанием почтового адреса организации и электронного адреса каждого
автора;\\[-15pt]
\item сведения об авторах, в соответствии с форматом, образцы которого
представлены на страницах:



\def\leftfootline{\small{\textbf{\thepage}
\hfill ИНФОРМАТИКА И ЕЁ ПРИМЕНЕНИЯ\ \ \ том\ 18\ \ \ выпуск\ 3\ \ \ 2024}
}%
 \def\rightfootline{\small{ИНФОРМАТИКА И ЕЁ ПРИМЕНЕНИЯ\ \ \ том\ 18\ \ \ выпуск\ 3\ \ \ 2024
\hfill \textbf{\thepage}}}



{\sf http://www.ipiran.ru/journal/issues/2013\_07\_01/authors.asp} и

{\sf http://www.ipiran.ru/journal/issues/2013\_07\_01\_eng/authors.asp};
\item аннотация (не менее 100~слов на каждом из языков). Аннотация~--- это краткое
резюме работы, которое может публиковаться отдельно. Она является основным
источником информации в~ин\-фор\-ма\-ци\-он\-ных системах и базах данных. Английская
аннотация должна быть оригинальной, может не быть дословным переводом русского
текста и должна быть написана хорошим английским языком. В~аннотации не должно
быть ссылок на литературу и, по возможности, формул;\\[-15pt]
\item ключевые слова~--- желательно из принятых в мировой
на\-уч\-но-тех\-ни\-че\-ской литературе тематических тезаурусов. Предложения не
могут быть ключевыми словами;\\[-15pt]
\item источники финансирования работы (ссылки на гранты, проекты,
поддерживающие организации и~т.\,п.).
\end{itemize}



%\pagebreak

\item  Требования к спискам литературы.\\[-14pt]

Ссылки на литературу в тексте статьи нумеруются (в квадратных скобках) и
располагаются в каждом из списков литературы в порядке  первых упоминаний. Если источник имеет DOI и/или EDN,
то их необходимо указывать.

Списки литературы представляются в двух вариантах:\\[-14pt]


\noindent
\begin{enumerate}[(1)]
\item \textbf{Список литературы к русскоязычной части}. Русские и английские
работы~---  на языке и в алфавите оригинала;\\[-14.5pt]
\item  \textbf{References}. Русские работы и работы на других языках~--- в латинской
транслитерации с переводом на английский язык; английские работы и работы на других
языках~--- на языке оригинала.
\end{enumerate}

Необходимо для составления списка ``References'' пользоваться размещенной на сайте
{\sf http://www. translit.net/ru/bgn/} бесплатной программой транслитерации русского
 текста в~латиницу. %, при этом в~за\-клад\-ке <<варианты\ldots>> следует выбратьопцию BGN.

Список литературы ``References'' приводится полностью отдельным блоком, повторяя все
позиции из списка литературы к русскоязычной части, независимо от того, имеются или
нет в нем иностранные источники. Если в списке литературы к русскоязычной части есть
ссылки на иностранные публикации, набранные латиницей, они полностью повторяются в
списке ``References''.

Ниже приведены примеры ссылок на различные виды публикаций в списке ``References''.

\def\leftfootline{\small{\textbf{\thepage}
\hfill ИНФОРМАТИКА И ЕЁ ПРИМЕНЕНИЯ\ \ \ том\ 18\ \ \ выпуск\ 3\ \ \ 2024}
}%
 \def\rightfootline{\small{ИНФОРМАТИКА И ЕЁ ПРИМЕНЕНИЯ\ \ \ том\ 18\ \ \ выпуск\ 3\ \ \ 2024
\hfill \textbf{\thepage}}}

{\small

\noindent
\textbf{Описание статьи из журнала:}

\Aue{Zagurenko, A.\,G., V.\,A.~Korotovskikh, A.\,A.~Kolesnikov, A.\,V.~Timonov, and D.\,V.~Kardymon}. 2008.
Tekhniko-ekonomicheskaya optimizatsiya dizayna gidrorazryva plasta [Technical and
economic optimization of the design
of hydraulic fracturing]. \textit{Neftyanoe hozyaystvo} [\textit{Oil Industry}] 11:54--57.

\Aue{Zhang, Z., and D.~Zhu}. 2008. Experimental research on the localized
electrochemical micromachining. \textit{Russ. J.~Electrochem.}  44(8):926--930.
{\sf doi:10.1134/S1023193508080077}.

\noindent
\textbf{Описание статьи из электронного журнала:}

\Aue{Swaminathan, V., E.~Lepkoswka-White, and B.\,P.~Rao}. 1999. Browsers or buyers in cyberspace? An
investigation of electronic factors influencing electronic exchange. \textit{JCMC}
5(2). Available at: {\sf http://www.ascusc.org/jcmc/vol5/issue2/} (accessed April~28, 2011).

\def\leftkol{Правила подготовки рукописей  для публикации в журнале
<<Информатика и её применения>>}

\def\rightkol{Правила подготовки рукописей  для публикации в журнале
<<Информатика и её применения>>}


\noindent
\textbf{Описание статьи из продолжающегося издания (сборника трудов):}

\Aue{Astakhov, M.\,V., and T.\,V.~Tagantsev}. 2006. Eksperimental'noe
issledovanie prochnosti soedineniy ``stal'--kompozit'' [Experimental study of
the strength of joints ``steel--composite'']. \textit{Trudy MGTU
``Matematicheskoe modelirovanie slozhnykh tekh\-ni\-che\-skikh sistem''}
[\textit{Bauman MSTU ``Mathematical Modeling of Complex Technical
Systems'' Proceedings}]. 593:125--130.


\pagebreak



\noindent
\textbf{Описание материалов конференций:}

\Aue{Usmanov, T.\,S., A.\,A.~Gusmanov, I.\,Z.~Mullagalin, R.\,Ju.~Muhametshina, A.\,N.~Chervyakova, and
A.\,V.~Sveshnikov}. 2007. Osobennosti proektirovaniya razrabotki mestorozhdeniy
s primeneniem gidrorazryva
plasta [Features of the design of field development with the use of hydraulic fracturing].
\textit{Trudy 6-go
Mezhdu\-na\-rod\-no\-go Simpoziuma ``Novye resursosberegayushchie tekhnologii nedropol'zovaniya i povysheniya
neftegazootdachi''} [\textit{6th  Symposium (International) ``New Energy Saving Subsoil Technologies and
the Increasing of the Oil and Gas Impact'' Proceedings}]. Moscow. 267--272.



\def\leftfootline{\small{\textbf{\thepage}
\hfill ИНФОРМАТИКА И ЕЁ ПРИМЕНЕНИЯ\ \ \ том\ 18\ \ \ выпуск\ 3\ \ \ 2024}
}%
 \def\rightfootline{\small{ИНФОРМАТИКА И ЕЁ ПРИМЕНЕНИЯ\ \ \ том\ 18\ \ \ выпуск\ 3\ \ \ 2024
\hfill \textbf{\thepage}}}



\noindent
\textbf{Описание книги (монографии, сборники):}



Lindorf, L.\,S., and L.\,G.~Mamikoniants, eds. 1972.
\textit{Ekspluatatsiya turbogeneratorov s neposredstvennym
okhlazhdeniem} [\textit{Operation of turbine generators with direct cooling}].
Moscow: Energy Publs. 352~p.


\Aue{Latyshev, V.\,N.} 2009. \textit{Tribologiya rezaniya. Kn.~1: Friktsionnye protsessy
pri rezanii metallov}
[\textit{Tribology of cutting. Vol.~1: Frictional processes in metal cutting}]. Ivanovo: Ivanovskii
State Univ. 108~p.

\def\leftkol{Правила подготовки рукописей  для публикации в журнале
<<Информатика и её применения>>}

\def\rightkol{Правила подготовки рукописей  для публикации в журнале
<<Информатика и её применения>>}

\noindent
\textbf{Описание переводной книги}
(в списке литературы к русскоязычной части необходимо указать:~/ Пер.\ с англ.~---
после названия книги, а в конце ссылки указать оригинал книги в круглых скобках):
\begin{enumerate}[1.]
\item  В русскоязычной части:

\def\leftfootline{\small{\textbf{\thepage}
\hfill ИНФОРМАТИКА И ЕЁ ПРИМЕНЕНИЯ\ \ \ том\ 18\ \ \ выпуск\ 3\ \ \ 2024}
}%
 \def\rightfootline{\small{ИНФОРМАТИКА И ЕЁ ПРИМЕНЕНИЯ\ \ \ том\ 18\ \ \ выпуск\ 3\ \ \ 2024
\hfill \textbf{\thepage}}}

\Au{Тимошенко С.\,П., Янг Д.\,Х., Уивер~У.}
Колебания в инженерном деле~/ Пер.\ с англ.~--- М.: Машиностроение, 1985. 472~с.
(\Au{Timoshenko~S.\,P., Young~D.\,H., Weaver~W.}
Vibration problems in engineering.~--- 4th ed.~--- New York, NY, USA: Wiley, 1974. 521~p.)\\[-13.5pt]
\item  В англоязычной части:

\Aue{Timoshenko, S.\,P., D.\,H.~Young, and W.~Weaver}.
1974. \textit{Vibration problems in engineering}. 4th ed. New York: 
Wiley. 521~p.
\end{enumerate}

\vspace*{-3pt}


\noindent
\textbf{Описание неопубликованного документа:}


\Aue{Latypov, A.\,R., M.\,M.~Khasanov, and V.\,A.~Baikov}.
2004 (unpubl.). Geologiya i~dobycha (NGT GiD) [Geology and production (NGT GiD)]. Certificate on official registration of the computer program
No.\,2004611198. 

\noindent
\textbf{Описание интернет-ресурса:}


Pravila tsitirovaniya istochnikov [Rules for the citing of sources]. Available at: {\sf
http://www.scribd.com/doc/1034528/} (accessed February~7, 2011).

%\pagebreak

\noindent
\textbf{Описание диссертации или автореферата диссертации:}

\Aue{Semenov, V.\,I.}
2003. Matematicheskoe modelirovanie plazmy v sisteme kompaktnyy tor [Mathematical
modeling of the plasma in the compact torus].  Moscow.  D.Sc.\ Diss. 272~p.

\Aue{Kozhunova, O.\,S.} 2009. Tekhnologiya razrabotki semanticheskogo
slovarya informatsionnogo monitoringa [Technology of development of
semantic dictionary of information monitoring system].  Moscow: IPI RAN. PhD Thesis. 23~p.


\noindent
\textbf{Описание ГОСТа:}

GOST 8.586.5-2005. 2007. Metodika vypolneniya izmereniy. Izmerenie raskhoda i~kolichestva zhidkostey i~gazov
s~pomoshch'yu standartnykh suzhayushchikh ustroystv [Method of measurement.
Measurement of flow rate and volume of liquids and gases by means of orifice devices]. Moscow:
Standardinform  Publs. 10~p.

\noindent
\textbf{Описание патента:}

\Aue{Bolshakov, M.\,V., A.\,V.~Kulakov, A.\,N.~Lavrenov, and M.\,V.~Palkin}.
2006. Sposob orientirovaniya po krenu letatel'nogo
apparata s opti\-che\-skoy golovkoy
samonavedeniya [The way to orient on the roll of aircraft with optical homing head].
Patent RF No.\,2280590.
}

\item Присланные в редакцию материалы авторам не возвращаются.\\[-13.5pt]

\item При отправке файлов по электронной почте просим придерживаться следующих
правил:
\begin{itemize}
\item указывать в поле subject (тема) название журнала и фамилию автора;\\[-13.5pt]
\item указывать в тексте письма название статьи, авторов и~журнал, в~который направляется статья;\\[-13.5pt]
\item использовать attach (присоединение);\\[-13.5pt]
\item в состав электронной версии статьи должны входить: файл, содержащий текст
статьи, и файл(ы), содержащий(е) иллюстрации.\\[-13.5pt]
\end{itemize}

\item Журнал <<Информатика и её применения>> является некоммерческим изданием.
Плата за публикацию не взимается, гонорар авторам не выплачивается.
\end{enumerate}



\def\leftfootline{\small{\textbf{\thepage}
\hfill ИНФОРМАТИКА И ЕЁ ПРИМЕНЕНИЯ\ \ \ том\ 18\ \ \ выпуск\ 3\ \ \ 2024}
}%
 \def\rightfootline{\small{ИНФОРМАТИКА И ЕЁ ПРИМЕНЕНИЯ\ \ \ том\ 18\ \ \ выпуск\ 3\ \ \ 2024
\hfill \textbf{\thepage}}}


\vspace*{-1mm}

\begin{center}

\textbf{Адрес редакции журнала <<Информатика и её применения>>:} \\




Москва 119333, ул.~Вавилова, д.~44, корп.~2, ФИЦ ИУ РАН\\[-10pt]

\

Тел.: +7\,(499)\,135-86-92\ \ Факс:  +7\,(495)\,930-45-05\\[-10pt]

 \

e-mail:   {\sf iiep@frccsc.ru} (Стригина Светлана Николаевна)\\[-10pt]

\

{\sf http://www.ipiran.ru/journal/issues/}
\end{center}
}


\def\leftkol{Правила подготовки рукописей  для публикации в журнале
<<Информатика и её применения>>}

\def\rightkol{Правила подготовки рукописей  для публикации в журнале
<<Информатика и её применения>>}


\def\leftfootline{\small{\textbf{\thepage}
\hfill ИНФОРМАТИКА И ЕЁ ПРИМЕНЕНИЯ\ \ \ том\ 18\ \ \ выпуск\ 3\ \ \ 2024}
}%
 \def\rightfootline{\small{ИНФОРМАТИКА И ЕЁ ПРИМЕНЕНИЯ\ \ \ том\ 18\ \ \ выпуск\ 3\ \ \ 2024
\hfill \textbf{\thepage}}} 
\def\stat{podg-e}
{%\hrule\par
%\vskip 7pt % 7pt
\vspace*{-24pt}
\raggedleft\Large \bf%\baselineskip=3.2ex
Requirements for manuscripts submitted to Journal
``Informatics~and~Applications'' \vskip 8pt
    \hrule
    \par
\vskip 21pt plus 6pt minus 3pt }

\label{st\stat}

\def\tit{\ }

\def\aut{\ }
\def\auf{\ }

\def\leftkol{\ }

\def\rightkol{\ }
%Requirements for manuscripts submitted to Journal
%``Informatics~and~Applications''}

\titele{\tit}{\aut}{\auf}{\leftkol}{\rightkol}

\def\leftfootline{\small{\textbf{\thepage}
\hfill INFORMATIKA I EE PRIMENENIYA~--- INFORMATICS AND APPLICATIONS\ \ \ 2019\
\ \ volume~13\ \ \ issue\ 4}
}%
 \def\rightfootline{\small{INFORMATIKA I EE PRIMENENIYA~--- INFORMATICS AND APPLICATIONS\ \ \ 2019\ \ \ volume~13\ \ \ issue\ 4
\hfill \textbf{\thepage}}}

\vspace*{-60pt}

{\small

\noindent
Journal ``Informatics and Applications'' (Inform.\ Appl.)
publishes theoretical, review, and discussion
articles on the research and development in the
field of informatics and its applications.

The journal is published in Russian.
By a special decision of the editorial
board, some articles can be published in English.


The topics covered include the following areas:
\begin{itemize}
               \item
     theoretical fundamentals of informatics; \\[-14pt]
\item
mathematical methods for studying complex systems and processes; \\[-14pt]
\item
information systems and networks;\\[-14pt]
\item
information technologies; and \\[-14pt]
\item
architecture and software of computational complexes and networks. \\[-14pt]
\end{itemize}

\noindent
\begin{enumerate}[1.]
\item The Journal publishes original articles which have not been published before and are not
intended for simultaneous publication in other editions. An article submitted to the Journal must not violate the
Copyright law. Sending the manuscript to the Editorial Board, the authors retain all rights of the
owners of the manuscript and transfer the nonexclusive rights to publish the article in Russian
(or the language of the article, if not Russian) and its distribution in Russia and abroad to the
Founders and the Editorial Board. Authors should submit a letter to the Editorial Board in the
following form:

{\bfseries\textit{Agreement on the transfer of rights to publish:}}

``\textit{We, the undersigned authors of the manuscript ``\ldots'', pass to the
Founder and the Editorial Board of the Journal ``Informatics and Applications''
the nonexclusive right to publish the manuscript of the article in Russian (or
in English) in both print and electronic versions of the Journal. We affirm
that this publication does not violate the Copyright of other persons or
organizations.}

\textit{Author(s) signature(s): (name(s), address(es), date).}

This agreement should be submitted in paper form or in the form of a scanned copy (signed by
the authors).


%The Editorial Board has the right to request from the authors an official expert conclusion that
%the submitted article has no secret data prohibited for publication. \\[-13.5pt]
\item
A submitted article should be attached with \textbf{the data on the author(s)} (see item~8). If
there are several authors, the contact person should be indicated who is responsible for
correspondence with the Editorial Board and other authors about revisions and final approval
of the proofs.\\[-13.5pt]

\item The Editorial Board of the Journal examines the article according to the established
reviewing procedure. If the authors receive their article for correction after reviewing, it does not
mean that the article is approved for publication. The corrected article should be sent to the
Editorial Board for the subsequent review and approval.\\[-13.5pt]

\item The decision on the article publication or its rejection is communicated to the authors. The
Editorial Board may also send the reviews on the submitted articles to the authors. Any
discussion upon the rejected articles is not possible.\\[-13.5pt]

\item The edited articles will be sent to the authors for proofread. The comments of the authors
to the edited text of the article should be sent to the Editorial Board as soon as possible.\\[-13.5pt]

\item The manuscript of the article should be presented electronically in the MS WORD (.doc or
.docx) or \LaTeX\ (.tex) formats, and additionally in the .pdf format. All documents
 may be sent
by e-mail or provided on a CD or diskette. A~hard copy submission is not necessary.\\[-13.5pt]

\item The recommended typesetting instructions for manuscript.

Pages parameters: format A4, portrait orientation, document margins (cm): left~--- 2.5, right~---
1.5, above~--- 2.0, below~--- 2.0, footer 1.3.

Text: font~---Times New Roman, font size~--- 14, paragraph indent~--- 0.5, line spacing~--- 1.5,
justified alignment.

The recommended manuscript size: not more than 15~pages of the specified format.
If the specified size exceeded, the editorial board is entitled to require the author
to reduce the manuscript.

Use only standard abbreviations. Avoid  abbreviations in the title and
abstract. The full term for which an abbreviation stands should precede
its first use in the text unless it is a standard unit of measurement.

All pages of the manuscript should be numbered.

The templates for the manuscript typesetting are presented on site: {\sf
http://www.ipiran.ru/journal/template.doc}.\\[-13.5pt]


%\def\leftkol{Requirements for manuscripts submitted to Journal
%``Informatics~and~Applications''}

\item The articles should enclose data both in \textbf{Russian and English}:
\begin{itemize}
\item title;\\[-13.5pt]
\item author's name and surname;\\[-13.5pt]
\item affiliation~--- organization, its address with ZIP code, city, country, and
official e-mail address;\\[-13.5pt]
\item data on authors according to the format: (see site)

{\sf http://www.ipiran.ru/journal/issues/2013\_07\_01/authors.asp}  and

{\sf  http://www.ipiran.ru/journal/issues/2013\_07\_01\_eng/authors.asp};\\[-13.5pt]

\pagebreak

\def\leftfootline{\small{\textbf{\thepage}
\hfill INFORMATIKA I EE PRIMENENIYA~--- INFORMATICS AND APPLICATIONS\ \ \ 2019\
\ \ volume~13\ \ \ issue\ 4}
}%
 \def\rightfootline{\small{INFORMATIKA I EE PRIMENENIYA~--- INFORMATICS AND APPLICATIONS\ \ \ 2019\ \ \ volume~13\ \ \ issue\ 4
\hfill \textbf{\thepage}}}


%\def\leftkol{Requirements for manuscripts submitted to Journal
%``Informatics~and~Applications''}

%\def\rightkol{Requirements for manuscripts submitted to Journal
%``Informatics~and~Applications''}



\item abstract (not less than 100 words) both in Russian and in English. Abstract is a short
summary of the article that can be published separately. The abstract is the
main source of information on the article and it could be included in leading information
systems and data bases. The abstract in English has to be an original text and should
not be an exact translation of the Russian one. Good English is required.
In abstracts, avoid references and formulae;\\[-13.5pt]
\item indexing is performed on the basis of keywords. The use of keywords from the
internationally accepted thematic Thesauri is recommended.

%\def\leftkol{Requirements for manuscripts submitted to Journal
%``Informatics~and~Applications''}

%\def\rightkol{Requirements for manuscripts submitted to Journal
%``Informatics~and~Applications''}

Important! Keywords must not be sentences;
\item Acknowledgments.
\end{itemize}

\item References. Russian references have to be presented both in English translation and Latin
transliteration (refer {\sf http://www.translit.net/ru/bgn/}).

Please take into account the following examples of Russian references appearance:

\noindent
\textbf{Article in journal:}

\Aue{Zhang, Z., and D.~Zhu}. 2008. Experimental research on the localized electrochemical
micromachining.
\textit{Rus. J.~Electrochem.}  44(8):926--930. {\sf doi:10.1134/S1023193508080077}.


\noindent
\textbf{Journal article in electronic format:}

\Aue{Swaminathan, V., E.~Lepkoswka-White, and B.\,P.~Rao}. 1999. Browsers or buyers in
cyberspace? An
investigation of electronic factors influencing electronic exchange. \textit{JCMC}
5(2). Available at: {\sf http://www.ascusc.org/jcmc/vol5/issue2/} (accessed April~28, 2011).




\noindent
\textbf{Article from the continuing publication (collection of works, proceedings):}

\Aue{Astakhov, M.\,V., and T.\,V.~Tagantsev}. 2006. Eksperimental'noe
issledovanie prochnosti soedineniy ``stal'--kompozit'' [Experimental study of
the strength of joints ``steel--composite'']. \textit{Trudy MGTU
``Matematicheskoe modelirovanie slozhnykh tekh\-ni\-che\-skikh sistem''}
[\textit{Bauman MSTU ``Mathematical Modeling of Complex Technical
Systems'' Proceedings}]. 593:125--130.

\def\leftfootline{\small{\textbf{\thepage}
\hfill INFORMATIKA I EE PRIMENENIYA~--- INFORMATICS AND APPLICATIONS\ \ \ 2019\
\ \ volume~13\ \ \ issue\ 4}
}%
 \def\rightfootline{\small{INFORMATIKA I EE PRIMENENIYA~--- INFORMATICS AND APPLICATIONS\ \ \ 2019\ \ \ volume~13\ \ \ issue\ 4
\hfill \textbf{\thepage}}}

\def\leftkol{Requirements for manuscripts submitted to Journal
``Informatics~and~Applications''}

\def\rightkol{Requirements for manuscripts submitted to Journal
``Informatics~and~Applications''}

\noindent
\textbf{Conference proceedings:}

\Aue{Usmanov, T.\,S., A.\,A.~Gusmanov, I.\,Z.~Mullagalin, R.\,Ju.~Muhametshina,
A.\,N.~Chervyakova, and
A.\,V.~Sveshnikov}. 2007. Osobennosti proektirovaniya razrabotki mestorozhdeniy
s primeneniem gidrorazryva
plasta [Features of the design of field development with the use of hydraulic fracturing].
\textit{Trudy 6-go
Mezhdu\-na\-rod\-no\-go Simpoziuma ``Novye resursosberegayushchie tekhnologii
nedropol'zovaniya i povysheniya
neftegazootdachi''} [\textit{6th  Symposium (International) ``New Energy Saving Subsoil
Technologies and
the Increasing of the Oil and Gas Impact'' Proceedings}]. Moscow. 267--272.


\noindent
\textbf{Books and other monographs:}




Lindorf, L.\,S., and L.\,G.~Mamikoniants, eds. 1972.
\textit{Ekspluatatsiya turbogeneratorov s neposredstvennym
okhlazhdeniem} [\textit{Operation of turbine generators with direct cooling}].
Moscow: Energy Publs. 352~p.


%\Aue{Latyshev, V.\,N.} 2009. \textit{Tribologiya rezaniya. Kn.~1: Frikcionnye prosessy
%pri rezanii metallov}
%[\textit{Tribology of cutting. Vol.~1: Frictional processes in metal cutting}]. Ivanovo: Ivanovskii
%State Univ. 108~p.


%\noindent
%\textbf{Unpublished material:}

%\Aue{Latypov, A.\,R., M.\,M.~Khasanov, and V.\,A.~Baikov}.
%2004. Geology and production (NGT GiD). Certificate on official registration of the computer
%program
%No.\,2004611198. (In Russian, unpubl.)

%\noindent
%\textbf{Internet-source:}

%APA Style. 2011. Available at: {\sf http://www.apastyle.org/apa-style-help.aspx} (accessed
%February~5, 2011).

%Pravila citirovaniya istochnikov [Rules for the citing of sources]. Available at: {\sf
%http://www.scribd.com/doc/1034528/} (accessed February~7, 2011).


\noindent
\textbf{Dissertation and Thesis:}

%\Aue{Semenov, V.\,I.}
%2003. Matematicheskoe modelirovanie plazmy v sisteme kompaktnyy tor. [Mathematical
%modeling of the plasma in the compact torus]. D.Sc.\ Diss. Moscow. 272~p.

\Aue{Kozhunova, O.\,S.} 2009. Tekhnologiya razrabotki semanticheskogo
slovarya informatsionnogo monitoringa [Technology of development of
semantic dictionary of information monitoring system]. PhD Thesis. Moscow: IPI RAN. 23~p.


\noindent
\textbf{State standards and patents:}

GOST 8.586.5-2005. 2007. Metodika vypolneniya izmereniy. Izmerenie raskhoda i~kolichestva
zhidkostey i gazov 
s~pomoshch'yu standartnykh suzhayushchikh ustroystv [Method of measurement.
Measurement of flow rate and volume of liquids and gases by means of orifice devices]. M.:
Standardinform
Publs. 10~p.

%\noindent
%\textbf{Patent:}

\Aue{Bolshakov, M.\,V., A.\,V.~Kulakov, A.\,N.~Lavrenov, and M.\,V.~Palkin}.
2006. Sposob orientirovaniya po krenu letatel'nogo
apparata s opti\-che\-skoy golovkoy
samonavedeniya [The way to orient on the roll of aircraft with optical homing head].
Patent RF No.\,2280590.

References in Latin transcription are presented in the original language.

References in the text are numbered according to the order of their
first appearance; the number is
placed in square brackets. All items from the reference list should be
cited.\\[-13.5pt]

\item Manuscripts and additional materials are not returned to Authors by the Editorial Board.\\[-13.5pt]

\item Submissions of files by e-mail must include:\\[-13.5pt]
\begin{itemize}
\item   the journal title and author's name in the ``Subject'' field; \\[-13.5pt]
\item   an article and additional materials have to be attached using the ``attach'' function;\\[-13.5pt]
\item   an electronic version of the article should contain the file with the text and a separate file
with figures.\\[-13.5pt]
\end{itemize}

\item ``Informatics and Applications'' journal is not a profit publication. There are no
charges for the authors as well as there are no royalties.\\[-13.5pt]
\end{enumerate}

\def\leftfootline{\small{\textbf{\thepage}
\hfill INFORMATIKA I EE PRIMENENIYA~--- INFORMATICS AND APPLICATIONS\ \ \ 2019\
\ \ volume~13\ \ \ issue\ 4}
}%
 \def\rightfootline{\small{INFORMATIKA I EE PRIMENENIYA~--- INFORMATICS AND APPLICATIONS\ \ \ 2019\ \ \ volume~13\ \ \ issue\ 4
\hfill \textbf{\thepage}}}

\def\leftkol{Requirements for manuscripts submitted to Journal
``Informatics~and~Applications''}

\def\rightkol{Requirements for manuscripts submitted to Journal
``Informatics~and~Applications''}


%\vspace*{5mm}


\begin{center}
\textbf{Editorial Board address:} \\

%ABOUT AUTHORS



FRC CSC RAS, 44, block~2, Vavilov Str., Moscow 119333, Russia\\[-10pt]

\

Ph.: +7\,(499)\,135\,86\,92,\ \ Fax: +7\,(495)\,930\,45\,05\\[-10pt]

\

 e-mail: {\sf rust@ipiran.ru} (to Prof.\ Rustem Seyful-Mulyukov)\\[-10pt]

\

 {\sf http://www.ipiran.ru/english/journal.asp}
\end{center}
 }
%\thispagestyle{myheadings}

\def\leftkol{Requirements for manuscripts submitted to Journal
``Informatics~and~Applications''}

\def\rightkol{Requirements for manuscripts submitted to Journal
``Informatics~and~Applications''}

\def\leftfootline{\small{\textbf{\thepage}
\hfill INFORMATIKA I EE PRIMENENIYA~--- INFORMATICS AND APPLICATIONS\ \ \ 2019\
\ \ volume~13\ \ \ issue\ 4}
}%
 \def\rightfootline{\small{INFORMATIKA I EE PRIMENENIYA~--- INFORMATICS AND APPLICATIONS\ \ \ 2019\ \ \ volume~13\ \ \ issue\ 4
\hfill \textbf{\thepage}}}

 \label{end\stat}

\newpage

%\vspace*{-60pt} {\small
{\baselineskip=9.1pt
\section*{Правила подготовки рукописей статей для публикации в журнале
<<Информатика и её применения>>}

\thispagestyle{empty}

 Журнал <<Информатика и её применения>> публикует
теоретические, обзорные и дискуссионные статьи, посвященные научным
исследованиям и разработкам в области информатики и ее приложений. Журнал
издается на русском языке. По специальному решению редколлегии отдельные статьи,
в виде исключения, могут печататься на английском языке.
Тематика журнала охватывает следующие направления:
\begin{itemize}
\item теоретические основы информатики; %\\[-13.5pt]
\item математические методы исследования сложных систем и процессов; %\\[-13.5pt]
\item информационные системы и сети; %\\[-13.5pt]
\item информационные технологии; %\\[-13.5pt]
\item архитектура и программное
обеспечение вычислительных комплексов и сетей.
\end{itemize}
\begin{enumerate}
\item В журнале печатаются результаты, ранее не
опубликованные и не предназначенные к одновременной публикации в других
изданиях. Публикация не должна нарушать закон об авторских правах. Направляя
свою рукопись в редакцию, авторы автоматически передают учредителям и
редколлегии неисключительные права на издание данной статьи на русском языке и
на ее распространение в России и за рубежом. При этом за авторами сохраняются
все права как собственников данной рукописи. В связи с этим авторами должно
быть представлено в редакцию письмо в следующей форме:
Соглашение о передаче права на публикацию:

\textit{<<Мы, нижеподписавшиеся, авторы рукописи <<$\qquad\qquad$>>, передаем
учредителям и редколлегии журнала <<Информатика и её применения>>
неисключительное право опубликовать данную рукопись статьи на русском языке как
в печатной, так и в электронной версиях журнала. Мы подтверждаем, что данная
публикация не нарушает авторского права других лиц или организаций. Подписи
авторов: (ф.\,и.\,о., дата, адрес)>>.}

Указанное соглашение может быть представлено 
как в бумажном виде, так и в виде отсканированной копии (с подписями авторов).


Редколлегия вправе запросить у авторов экспертное заключение о возможности
опубликования представленной статьи в открытой печати. %\\[-13.5pt]
\item Статья
подписывается всеми авторами. На отдельном листе представляются данные автора
(или всех авторов): фамилия, полные имя и отчество, телефон, факс, e-mail,
почтовый адрес. Если работа выполнена несколькими авторами, указывается фамилия
одного из них, ответственного за переписку с редакцией. %\\[-13.5pt]
\item Редакция журнала
осуществляет самостоятельную экспертизу присланных статей. Возвращение рукописи
на доработку не означает, что статья уже принята к печати. Доработанный вариант
с ответом на замечания рецензента необходимо прислать в редакцию. %\\[-13.5pt]
\item Решение
редакционной коллегии о принятии статьи к печати или ее отклонении сообщается
авторам. Редколлегия не обязуется направлять рецензию авторам отклоненной
статьи. %\\[-13.5pt]
\item Корректура статей высылается авторам для просмотра. Редакция
просит авторов присылать свои замечания в кратчайшие сроки. %\\[-13.5pt]
\item При
подготовке рукописи в MS Word рекомендуется использовать следующие настройки.
Параметры страницы: формат~--- А4; ориентация~--- книжная; поля (см): внутри~---
2,5, снаружи~--- 1,5, сверху~--- 2, снизу~--- 2, от края до нижнего
колонтитула~--- 1,3. Основной текст: стиль~--- <<Обычный>>: шрифт Times New
Roman, размер 14~пунктов, абзацный отступ~--- 0,5~см, 1,5 интервала,
выравнивание~--- по ширине. Рекомендуемый объем рукописи~--- не свыше
25~страниц указанного формата. Ознакомиться с шаблонами, содержащими примеры
оформления, можно по адресу в Интернете:
\textsf{http://www.ipiran.ru/journal/template.doc}.
\item К рукописи, предоставляемой в 2-х
экземплярах, обязательно прилагается электронная версия статьи (как правило, в
форматах MS WORD (.doc) или \LaTeX\ (.tex), а также~--- дополнительно~--- в
формате .pdf) на дискете, лазерном диске или по электронной почте. Сокращения
слов, кроме стандартных, не применяются. Все страницы рукописи должны быть
пронумерованы. %\\[-13.5pt]
\item Статья должна содержать следующую информацию на русском и
английском языках: название, Ф.И.О. авторов, места работы авторов и их
электронные адреса, подробные сведения об авторах, оформленные в соответствии с форматом, 
определяемым файлами {\sf http://www.ipiran.ru/journal/issues/2011\_05\_01/authors.asp} и 
{\sf http://www.ipiran.ru/journal/issues/2011\_01\_eng/authors.asp},
аннотация (не более 100~слов), ключевые слова. Ссылки на
литературу в тексте статьи нумеруются (в квадратных скобках) и располагаются в
порядке их первого упоминания. В~списке литературы не должно быть позиций, на которые нет ссылки в тексте статьи.
Все фамилии авторов, заглавия статей, названия
книг, конференций и~т.\,п.\ даются на языке оригинала, если этот язык
использует кириллический или латинский алфавит. %\\[-13.5pt]
\item Присланные в редакцию материалы авторам не возвращаются.
\item При отправке файлов по электронной
почте просим придерживаться следующих правил:
\begin{itemize}
\item указывать в поле subject (тема) название журнала и фамилию автора; %\\[-13.5pt]
\item использовать attach (присоединение); %\\[-13.5pt]
\item в случае больших объемов информации возможно
использование общеизвестных архиваторов (ZIP, RAR); %\\[-13.5pt]
\item в состав электронной версии статьи должны входить: файл, содержащий текст статьи, и файл(ы),
содержащий(е) иллюстрации. %\\[-13.5pt]
\end{itemize}
\item Журнал <<Информатика и её применения>> является некоммерческим изданием. 
Плата за публикацию с авторов не взимается, гонорар авторам не выплачивается.
\end{enumerate}
\thispagestyle{empty}
\textbf{Адрес редакции:} Москва 119333,
ул.~Вавилова, д.~44, корп.~2, ИПИ РАН\\
\hphantom{\textbf{Адрес редакции:} }Тел.: +7 (499) 135-86-92\ \
Факс:  +7 (495) 930-45-05\ \  E-mail:   rust@ipiran.ru }
}

%\include{ipi-ind}

%\tableofcontents

\end{document}

%\tableofcontents

%\end{document}

%\tableofcontents


\end{document}

\newcommand{\Ack}{\subsection*{\protect\large\bf Acknowledgments}}

\vphantom{\int\limits_0^T }

{ \begin{center}  %fig1
 \vspace*{3pt}
    \mbox{%
 \epsfxsize=79mm 
 \epsfbox{gru-1.eps}
 }

\end{center}

\noindent
{{\figurename~1}\ \ \small{
Временные зависимости данные 
}}}

\vspace*{6pt}

\setcounter{figure}{1}

$\acute{\mbox{о}}$

\linebreak