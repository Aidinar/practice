\def\stat{naumov}

\def\tit{О МАРКОВСКИХ И~РАЦИОНАЛЬНЫХ ПОТОКАХ 
СЛУЧАЙНЫХ СОБЫТИЙ.~II$^*$} % Часть~2$^*$}

\def\titkol{О марковских и рациональных потоках случайных 
событий. II} %Часть 2}

\def\aut{В.\,А.~Наумов$^1$, К.\,Е.~Самуйлов$^2$}

\def\autkol{В.\,А.~Наумов, К.\,Е.~Самуйлов}

\titel{\tit}{\aut}{\autkol}{\titkol}

\index{Наумов В.\,А.}
\index{Самуйлов К.\,Е.}
\index{Naumov V.\,A.}
\index{Samouylov К.\,Е.}


{\renewcommand{\thefootnote}{\fnsymbol{footnote}} \footnotetext[1]
{Исследование выполнено при финансовой поддержке РФФИ в рамках научного проекта №\,19-17-50126.}}


\renewcommand{\thefootnote}{\arabic{footnote}}
\footnotetext[1]{Исследовательский институт инноваций, г.~Хельсинки, Финляндия, 
\mbox{valeriy.naumov@pfu.fi}}
\footnotetext[2]{Российский университет дружбы народов; Институт проблем информатики Федерального 
исследовательского центра <<Информатика и~управ\-ле\-ние>> Российской академии наук, \mbox{samouylov-ke@rudn.ru}}

%\vspace*{6pt}

  \Abst{Статья представляет собой вторую часть обзора, выполненного в рамках проекта 
РФФИ 
  №\,19-17-50126. Цель обзора~--- ознакомление заинтересованных читателей с основами 
теории марковских потоков событий для более подробного изучения и облегчения 
применения этих моделей на практике. В~первой части приведены свойства общих 
марковских потоков событий и показана их связь с марковскими аддитивными процессами и 
процессами марковского восстановления. Во второй части обзора рассмотрены важные для 
приложений частные случаи таких потоков~--- подклассы марковских потоков событий, а~именно:
 простые и групповые потоки однородных и неоднородных событий. Показано, 
как свойства марковских потоков событий связаны с мультипликативностью стационарных 
распределений марковских систем. Обсуждаются  
мат\-рич\-но-экс\-по\-нен\-ци\-аль\-ные распределения и рациональные потоки событий, 
расширяющие возможности марковских потоков для моделирования сложных систем, при 
этом сохраняющие удобство их анализа с помощью вычислительной техники.}
  
  \KW{марковские процессы; марковские аддитивные процессы; потоки без последействия; 
  МС-по\-то\-ки}
  
\DOI{10.14357/19922264200406} 
  
\vspace*{6pt}


\vskip 10pt plus 9pt minus 6pt

\thispagestyle{headings}

\begin{multicols}{2}

\label{st\stat}


\section{Введение}

Настоящий обзор, состоящий из двух частей, включает изложение основ 
теории марковских потоков и снабжен ссылками на большое число работ, 
посвященных марковским и~рациональным потокам событий. Он начался с 
рассмотрения в первой части случайных величин фазового типа, определения 
марковских потоков общего вида и их связи с~марковскими аддитивными 
процессами и процессами марковского восстановления. Во второй части 
обзора  перейдем к важным для приложений подклассам марковских потоков 
однородных и неоднородных событий в разд.~2, а~в~завершение в~разд.~3 
обсудим  
мат\-рич\-но-экс\-по\-нен\-ци\-аль\-ные распределения и~в~разд.~4 
рациональные потоки событий, которые расширяют возможности марковских 
потоков для моделирования сложных систем и~при этом сохраняют удобство 
их анализа. 

Как и в первой части обзора, далее в работе жирные строчные буквы 
обозначают векторы, а~жирные прописные буквы обозначают матрицы. 
Кроме того, используются следующие обозначения: 
$$
\delta(i,j)= \begin{cases}
1, &\mbox{если } i=j\,;\\
0 & \mbox{в~противном\ случае};
\end{cases}
$$
 у~вектора~$\mathbf{e}_i$ 
$i$-я координата равна единице, а остальные равны нулю; $\mathbf{I}\hm= \left[ 
\delta(i,j)\right]$~--- единичная матрица; $\mathbf{u}$~---  
век\-тор-стол\-бец из единиц; $\boldsymbol{\mathcal{N}}^K$~--- множество 
неотрицательных целочисленных векторов длины~$K$, 
$\boldsymbol{\mathcal{N}}^K _0\hm= \boldsymbol{\mathcal{N}}^K \backslash 
\{\mathbf{0}\}$. Для краткости вмес\-то <<наступило $n_1$ событий типа~1, 
$n_2$ событий ти-\linebreak па~2,~\ldots , $n_K$ событий типа~$K$>> будем писать 
<<наступило $\mathbf{n}$ событий>>, где $\mathbf{n}\hm= \left( n_1, n_2, 
\ldots , n_K\right)$.

\section{Важные для~приложений частные случаи марковских потоков 
событий}

\subsection{Простой марковский поток однородных событий}

  Рассмотрим некоторый поток случайных неоднородных событий и 
обозначим через $N_k(t)$ чис\-ло событий типа~$k$, наступивших за время~$t$, 
$\mathbf{N}(t)\hm= (N_1(t), N_2(t), \ldots, N_K(t))$. Поток случайных событий 
называется марковским, если для некоторого случайного процесса~$X(t)$ с 
конечным \mbox{множеством} состояний $\boldsymbol{\mathcal{X}}\hm= \{1,2,\ldots , 
L\}$ процесс $\xi(t)\hm= (X(t), \mathbf{N}(t))$ является марковским процессом, 
однородным во времени и по второй компоненте, т.\,е.\ если для любых~$t, 
h\hm>0$ справедливы равенства
  \begin{multline*}
  {\sf P}\left(X(h+t)=j, \mathbf{N}(h+t)=\mathbf{k}+\mathbf{n}\vert X(h)=i, \right.\\
\left.\mathbf{N}(h)=\mathbf{k}\right)=p_{\mathbf{n}}(i,j,t)\,,\enskip
  \mathbf{k}, \mathbf{n} \in \boldsymbol{\mathcal{N}}^K,\enskip i,j\in 
\boldsymbol{\mathcal{X}}\,.
  \end{multline*}
Матрицы вероятностей переходов $\mathbf{P}_{\mathbf{n}}(t)\hm= 
[p_{\mathbf{n}}(i,j,t)]$ однозначно определяются матрицами интенсивностей 
переходов $\mathbf{A}_{\mathbf{n}}\hm= \left[ a_{\mathbf{n}}(i,j)\right]$, 
$\mathbf{n}\hm\geq \mathbf{0}$, где
\begin{align*}
a_{\mathbf{0}}(i,j) &=\lim\limits_{t\to0} \fr{1}{t}\left( p_{\mathbf{0}}(i,j,t)-\delta(i,j)\right)\,,\enskip
 i,j\in  \boldsymbol{\mathcal{X}}\,;\\
a_{\mathbf{n}}(i,j) &=\lim\limits_{t\to0} \fr{1}{t}\, p_{\mathbf{n}}(i,j,t)\,,\enskip i,j\in 
\boldsymbol{\mathcal{X}}\,,\enskip \mathbf{n}\in \boldsymbol{\mathcal{N}}^K_0,
\end{align*}
при этом фазовый процесс~$X(t)$ является однородным марковским 
процессом с матрицей интенсивностей переходов $\mathbf{A}\hm= 
\sum\nolimits_{\mathbf{n}\in \boldsymbol{\mathcal{N}}^K} 
\mathbf{A}_{\mathbf{n}}$.
  
  В первой части обзора определен процесс марковского восстановления 
$(X_l,\boldsymbol{\sigma}_l, \tau_l)$, где $X_l\hm= X(t_l)$~--- состояния 
фазового процесса~$X(t)$ марковского потока в моменты после наступления\linebreak 
событий потока, $X(t)\hm\in \boldsymbol{\mathcal{X}} \hm= \{1,2,\ldots ,L\}$, 
$0\hm< t_1\hm< t_2
  <\cdots$~--- моменты наступления событий, также называемые 
вызывающими моментами; $\tau_l\hm= t_l\hm- t_{l-1}$~--- длины интервалов 
между \mbox{моментами} наступления событий; $\boldsymbol{\sigma}_l$~--- вектор, 
$\boldsymbol{\sigma}_l\hm= (\sigma_{l,1}, \ldots , \sigma_{l,K})$, 
в~котором~$\sigma_{l,k}$ есть размер группы событий типа~$k$, наступивших 
в~момент~$t_l$, $l\hm=1, 2, \ldots$ Матрицы $\mathbf{G}_{\mathbf{n}}(x)\hm= 
[G_{\mathbf{n}}(i,j,x)]$, описывающие связанный с марковским потоком 
процесс марковского восстановления $(X_l, \boldsymbol{\sigma}_l, \tau_l)$, и 
их преобразования Лап\-ла\-са--Стилть\-еса имеют следующий вид:

\noindent
  \begin{align}
  \mathbf{G}_{\mathbf{n}}(x)&=\int\limits_0^x \exp 
(z\mathbf{A}_0)\mathbf{A}_{\mathbf{n}}\,dz={}\notag\\
&\hspace*{-10mm}{}=\left( \exp 
(x\mathbf{A}_{\mathbf{0}}))-\mathbf{I}\right)\mathbf{A}_0^{-1} \mathbf{A}_{\mathbf{n}}\,,\ \mathbf{n}\in 
\boldsymbol{\mathcal{N}}_0^K\,;
  \label{e1-nau}\\
  \int\limits_0^x e^{-\nu x}d\mathbf{G}_{\mathbf{n}}(x)&= (\nu\mathbf{I}-
\mathbf{A}_{\mathbf{0}})^{-1}\mathbf{A}_{\mathbf{n}}\,,\ \mathbf{n}\in 
\boldsymbol{\mathcal{N}}_0^K\,.
  \label{e2-nau}
  \end{align}
Используя матрицы $\mathbf{G}_{\mathbf{n}}(x)$, можно найти совместное 
распределение числа~$\boldsymbol{\sigma}_l$ наступивших событий и 
длин~$\tau_l$ интервалов между вызывающими моментами 
\begin{multline}
F_{\mathbf{k}_1, \mathbf{k}_2, \ldots , \mathbf{k}_m} \left(x_1, x_2, \ldots , 
x_m\right)={}\\
{}={\sf P}\left(
\boldsymbol{\sigma}_l=\mathbf{k}_l\,, \tau_l<x_l\,, l=1,2,\ldots, m\right)={}\\
{}=\bm{\alpha}\mathbf{G}_{\mathbf{k}_1}(x_1) \mathbf{G}_{\mathbf{k}_2}(x_2)\cdots 
\mathbf{G}_{\mathbf{k}_m}(x_m)\mathbf{u}\,,
\label{e3-nau}
\end{multline}
а также плотность этого распределения

\columnbreak

\noindent
\begin{multline}
f_{\mathbf{k}_1, \mathbf{k}_2, \ldots , \mathbf{k}_m} (x_1, x_2, \ldots , 
x_m)={}\\
{}=\bm{\alpha}\exp \left( x_1\mathbf{A}_{\mathbf{0}}\right) 
\mathbf{A}_{\mathbf{k}_1}\exp \left( x_2\mathbf{A}_{\mathbf{0}}\right) 
\mathbf{A}_{\mathbf{k}_2}\cdots\\
\cdots \exp \left( x_m\mathbf{A}_{\mathbf{0}}\right) 
\mathbf{A}_{\mathbf{k}_m}\mathbf{u}\,,\quad
\mathbf{k}_1, \mathbf{k}_2, \ldots , \mathbf{k}_m\in 
\boldsymbol{\mathcal{N}}_0^K\,,\\
 x_0, x_1, \ldots , x_m>0\,,\enskip m=1,2,\ldots
\label{e4-nau}
\end{multline}

\vspace*{-6pt}

\noindent
где $\bm{\alpha}$~--- начальное распределение фазового про\-цесса.


  
  Простой марковский поток однородных событий~--- это марковский поток 
событий одного типа, причем в каждый вызывающий момент наступает ровно 
одно событие. Он характеризуется двумя мат\-ри\-ца\-ми интенсивностей 
переходов $\mathbf{S}\hm= \mathbf{A}_0$ и~$\mathbf{R}\hm= \mathbf{A}_1$, 
а~остальные матрицы~$\mathbf{A}_k$, $k\hm\geq 2$, для такого потока~--- 
нулевые. Первыми работами, посвященными простым марковским потокам 
однородных событий, стали~[1--5]. Их применение к~решению задач теории 
телетрафика рассматривается  
в~\cite{6-nau, 7-nau}. Поток вызывающих моментов любого марковского 
потока~--- это простой марковский поток, характеризуемый матрицами 
$\mathbf{S}\hm= \mathbf{A}\hm-\mathbf{R}$ и~$\mathbf{R}\hm= 
\sum\nolimits_{\mathbf{n}\in \boldsymbol{\mathcal{N}}_0^K} 
\mathbf{A}_{\mathbf{n}}$. К~простым марковским потокам относятся также 
процессы восстановления фазового типа~\cite{8-nau}. Для таких потоков ранг 
матрицы~$\mathbf{R}$ равен единице и~она имеет вид $\mathbf{R}\hm= 
\mathbf{sq}$, где $\mathbf{s}\hm= -\mathbf{Su}$. Верно и~обратное~\cite{7-nau}. 
В~англоязычной литературе простые марковские потоки называют 
Markovian arrival process и~используют для их обозначения сокращение МАР 
или MArP.
  
  Простой марковский поток однородных событий является 
полумарковским, поскольку последовательность $(X_l, \tau_l)$, $l\hm=1, 
2,\ldots,$~--- процесс марковского восстановления. Из~(1) и~(2) вытекают 
следующие формулы для полумарковской матрицы $\mathbf{G}(x)\hm= \left[ 
G(i,j,x)\right]$ процесса $(X_l,\tau_l)$ марковского восстановления с 
элементами 

\vspace*{3pt}

\noindent
  $$
  G(i,j,x)={\sf P} \left( X_l=j,\ \tau_l<x\vert X_{l-1}=i\right)
  $$
  
  \vspace*{-1pt}
  
  \noindent
 и для ее преобразования Лап\-ла\-са--Стилть\-еса:
 
 \vspace*{2pt}
 
 \noindent
\begin{equation}
\left.
\begin{array}{rl}
\mathbf{G}(x)&=\left( \exp (x\mathbf{S})-\mathbf{I}\right) \mathbf{S}^{-
1}\mathbf{R}\,;\\
\displaystyle\int\limits_0^x e^{-\nu x}d\mathbf{G}(x)&=(\nu\mathbf{I}-\mathbf{S})^{-1}\mathbf{R}\,.
\end{array}
\right\}
\label{e5-nau}
\end{equation}

\vspace*{-2pt}
  
  Из~(\ref{e4-nau}) вытекает следующее выражение для плотности функции 
распределения длин интервалов~$\tau_l$ между моментами наступления 
событий простого марковского потока однородных событий:

\vspace*{-8pt}

\noindent
  \begin{multline}
  f\left( x_1, x_2, \ldots, x_m\right)={}\\
  {}=\bm{\alpha}\exp \left(x_1\mathbf{S}\right) 
\mathbf{R}\exp \left( x_2\mathbf{S}\right)\mathbf{R}\cdots \exp \left( 
x_m\mathbf{S}\right) \mathbf{Ru}\,,\\
  x_0, x_1,\ldots , x_m>0\,,\enskip m=1,2,\ldots
  \label{e6-nau}
  \end{multline}
  
  \vspace*{-2pt}
  
  Поскольку простой марковский поток является полумарковским, при 
анализе систем массового обслуживания с такими поступающими потоками 
можно использовать результаты, полученные для систем с полумарковским 
входящим потоком,  
например~[9--12].
   
  В первом разделе обзора указано, что стационарные распределения 
$\mathbf{q}\hm=[q(i)]$ и $\mathbf{q}_{\mathbf{n}}\hm= [q_{\mathbf{n}}(i)]$, 
$\mathbf{n}\hm\in \boldsymbol{\mathcal{N}}_0^K$, вложенных цепей 
Маркова~$X_l$ и~$(X_l, \boldsymbol{\sigma}_l)$ связаны со стационарным 
распределением~$\mathbf{p}$ фазового процесса~$X(t)$ следующими 
равенствами:
\begin{multline*}
  \mathbf{q}=\fr{1}{\lambda}\,\mathbf{p}\boldsymbol{\Lambda}\,,\
  \mathbf{p}=-\lambda \mathbf{q}\mathbf{A}_0^{-1}\,,\ 
  \mathbf{q}=\sum\limits_{\mathbf{n}\in \boldsymbol{\mathcal{N}}_0^K} 
\mathbf{q}_{\mathbf{n}}\,,\\
 \mathbf{q}_{\mathbf{n}}=\fr{1}{\lambda}\,\mathbf{p}
  \mathbf{A}_{\mathbf{n}}\,,\enskip \mathbf{n}\in \boldsymbol{\mathcal{N}}_0^K\,.
  \end{multline*}

  Если вектор из единиц~$\mathbf{u}$ является правым собственным 
вектором каждой из матриц~$\mathbf{A}_{\mathbf{n}}$ и выполняются 
равенства 
  \begin{equation}
  \mathbf{A}_{\mathbf{n}}\mathbf{u}=\lambda_{\mathbf{u}}\mathbf{u}\,,\quad
  \mathbf{n}\in \boldsymbol{\mathcal{N}}_0^K\,,
  \label{e7-nau}
  \end{equation}
то из~(\ref{e3-nau}) следует, что при любом начальном 
распределении~$\mathbf{s}$ марковский поток будет стационарным потоком 
без последействия. Аналогично, если вектор стационарных 
вероятностей~$\mathbf{p}$ является левым собственным вектором 
матриц~$\mathbf{A}_{\mathbf{n}}$ и выполняются равенства 
\begin{equation}
\mathbf{pA}_{\mathbf{n}}=\lambda_{\mathbf{n}}\mathbf{p}\,,\quad
\mathbf{n}\in \boldsymbol{\mathcal{N}}_0^K\,.
\label{e8-nau}
\end{equation}
    
Условия~(\ref{e7-nau}) и~(\ref{e8-nau}), достаточные для того чтобы 
марковский поток был пуассоновским, для простого марковского потока 
приобретают вид $\mathbf{Ru}\hm= \lambda\mathbf{u}$ и~$\mathbf{pR}\hm= 
\lambda\mathbf{p}$ соответственно, где $\lambda\hm= \mathbf{pRu}$~--- 
интенсивность потока. Проверка необходимых и~достаточных условий 
пуассоновости простого марковского потока более сложна и~требует знания 
собственных векторов матрицы~$\mathbf{S}$~\cite{13-nau}.
  
  Считающий процесс $N(t)$ стационарной версии простого марковского 
потока является асимптотически нормальным с~математическим ожиданием 
${\sf M}(t)\hm=\lambda t$ и дисперсией
  $$
  {\sf D}(t)=\left( 2\mathbf{d}_1\mathbf{s}-\lambda\right) t +2\left( 
\mathbf{d}_2\mathbf{s}-\lambda\right) +o(1)\,,
  $$
где векторы-столб\-цы~$\mathbf{d}_1$ и~$\mathbf{d}_2$~--- единственные 
решения систем линейных уравнений~\cite{2-nau}:
\begin{alignat*}{2}
\mathbf{d}_1\mathbf{A} &=\mathbf{p}(\lambda \mathbf{I}-\mathbf{R})\,,&\quad
\mathbf{d}_1\mathbf{u}&=1\,;\\
\mathbf{d}_2\mathbf{A}&=\mathbf{d}_1 -\mathbf{p}\,, &\quad
\mathbf{d}_2\mathbf{u}&=1\,.
\end{alignat*}
    
\subsection{Простой марковский поток неоднородных событий}

  Простой марковский поток неоднородных событий~--- это марковский 
поток событий нескольких типов, в каждый вызывающий момент которого 
наступает ровно одно событие. Такой поток характеризуется $K\hm+1$ 
матрицами интенсивностей переходов $\mathbf{S}\hm= \mathbf{A}_0$ и 
$\mathbf{R}_k\hm= \mathbf{A}_{\mathbf{e}_k}$, $k\hm=1,2,\ldots ,K$, 
а~остальные матрицы~$\mathbf{A}_{\mathbf{n}}$~--- нулевые. При этом поток 
событий одного типа, например типа~$i$, является простым марковским 
потоком однородных событий, описываемым матрицами 
$\mathbf{S}_i\hm=\mathbf{A}\hm- \mathbf{A}_{\mathbf{e}_i}$ 
и~$\mathbf{R}_i$. Первыми работами, посвященными прос\-тым марковским 
потокам неоднородных событий, считаются~[14--16]. В~англоязычной 
литературе такой поток называют Markovian Arrival Process with marked arrivals 
и~используют для его обозначения сокращение ММАР.  
Из~(\ref{e5-nau}) вытекает следующее выражение для плотности совместного 
распределения ${\sf P}(\omega_l=k_l,\tau_l<x_l, l\hm=1,2,\ldots ,m)$ типов 
$\omega_l$ событий, наступивших в~момент~$t_l$, и~длин~$\tau_l$ интервалов 
между вызывающими моментами: 
  \begin{multline}
  f_{{k}_1, {k}_2, \ldots , {k}_m}\left( x_1, x_2, \ldots , x_m\right)={}\\
  {}=\bm{\alpha}\exp \left( x_1\mathbf{S}\right)\mathbf{R}_{k_1}\exp\left( 
x_2\mathbf{S}\right) \mathbf{R}_{k_2}\cdots\\
\cdots \exp \left( x_m\mathbf{S}\right) 
\mathbf{R}_{k_m}\mathbf{u}\,,\quad
 1\leq k_1, k_2, \ldots , k_m\leq K\,,\\
x_0, x_1, \ldots , x_m>0\,,\quad  m=1,2,\ldots
 \label{e9-nau}
\end{multline}

\subsection{Марковский поток групп однородных событий}

  Марковский поток групп однородных событий~--- это марковский поток 
событий одного типа, в каждый вызывающий момент которого \mbox{может} 
наступить несколько событий. Такие марковские потоки впервые 
исследовались в~\cite{8-nau, 17-nau, 18-nau}, а их описание с помощью 
матриц~$\mathbf{A}_{\mathbf{n}}$ впервые появилось в~\cite{19-nau}. 
В~англоязычной литературе такой поток сейчас называют batch Markovian 
arrival process и используют для его обозначения сокращение BMAP. 
В~\cite{20-nau} получены формулы и асимптотики для первых двух моментов 
считающего процесса~$N(t)$, а~в~\cite{21-nau}~--- для старших моментов~$N(t)$.

\section{Матрично-экспоненциальные распределения}

  Функция распределения $F(t)$ неотрицательной случайной величины 
называется мат\-рич\-но-экс\-по\-нен\-ци\-аль\-ной, если $F(0)\hm<1$ и она 
представима в~виде 
  \begin{equation}
  F(t)=1-\mathbf{q}\exp (t\mathbf{S})\mathbf{u}
  \label{e10-nau}
  \end{equation}
с некоторым вектором~$\mathbf{q}$ и матрицей~$\mathbf{S}$, име\-ющей 
собственные числа лишь с отрицательными действительными частями. Для 
того чтобы функция распределения~$F(t)$ неотрицательной случайной 
величины была  
мат\-рич\-но-экс\-по\-нен\-ци\-аль\-ной, необходимо и достаточно, чтобы она 
имела рациональное преобразование 
Лап\-ла\-са--Стилть\-еса $\tilde{F}(\nu)$. Минимальный порядок 
матрицы~$\mathbf{S}$  
в~мат\-рич\-но-экс\-по\-нен\-ци\-аль\-ном представлении~(\ref{e10-nau}) равен 
чис\-лу полюсов функции $\tilde{F}(\nu)$ с учетом их кратности. Представление 
с~матрицей~$\mathbf{S}$ минимального порядка называется минимальным. 

  В некоторых работах по мат\-рич\-но-экс\-по\-нен\-ци\-аль\-ным функциям  
распределения~\cite{22-nau, 23-nau, 24-nau}, а~также в книгах~\cite{25-nau, 26-nau}, 
чтобы подчеркнуть аналогию с экспоненциальными функциями 
\mbox{распределения},  
вмес\-то~(\ref{e10-nau}) использовалось пред\-став\-ле\-ние $F(t)\hm= 1\hm - 
\mathbf{q}\exp (-t\mathbf{B})\mathbf{u}$ со знаком минус перед~$t$ 
и~мат\-ри\-цей~$\mathbf{B}$, име\-ющей собственные чис\-ла с~положительными 
действительными частями. В~настоящее\linebreak время используются только 
представления вида~(\ref{e10-nau}). Иногда допускается, что 
вектор~$\mathbf{u}$ в~(\ref{e10-nau}) может быть любым, а~не состоящим из 
единиц, как в~рас\-смат\-ри\-ва\-емом случае. Однако в~\cite{24-nau, 27-nau} 
было показано, что всегда можно подобрать мат\-рич\-но-экс\-по\-нен\-ци\-аль\-ное 
пред\-став\-ле\-ние с~$\mathbf{u}\hm=(1,1,\ldots , 1)$. 
  
  Идея матрично-экс\-по\-нен\-ци\-аль\-ных функций распределения восходит 
к работе~\cite{28-nau}, в которой показано, что рациональные преобразования  
Лап\-ла\-са--Стилть\-еса неотрицательных функций распределения 
представимы в виде:
  $$
  \tilde{F}(s)=p_0+\sum\limits^L_{l=1} q_0\cdots q_{l-1} p_l \prod\limits^l_{i=1} 
\fr{\lambda_i}{\lambda_{i}+s}\,,
  $$
где $p_i+q_i\hm=1$, $i\hm=1, \ldots , L$, $p_L\hm=1$, и~$-\lambda_i$, $i\hm=1, 
\ldots , L$,~--- полюсы~$\tilde{F}(s)$. Такое представление можно записать в 
мат\-рич\-но-экс\-по\-нен\-ци\-аль\-ном виде~(\ref{e10-nau}), полагая 
\begin{align*}
\mathbf{q}&=(1,0,\ldots ,0)\,;\\
\mathbf{S}&=\begin{bmatrix}
-\lambda_1&q_1\lambda_1&0&\cdots&0\\
0&-\lambda_2&q_2\lambda_2&\ddots &\vdots\\
0&0&\ddots& \ddots& 0\\
\vdots& \ddots& \ddots& -\lambda_{L-1}&q_{L-1}\lambda_{L-1}\\
0&\cdots & 0&0&-\lambda_L
\end{bmatrix}\,,
\end{align*}
%
  при этом элементы матрицы~$\mathbf{S}$ могут быть комплексными. 
В~\cite{22-nau} показано, что вектор~$\mathbf{q}$ и~мат\-ри\-ца~$\mathbf{S}$  
в~мат\-рич\-но-экс\-по\-нен\-ци\-аль\-ном  
пред\-ставлении~(\ref{e10-nau}) всегда могут быть выбраны действительными. 
  
  Из~(\ref{e10-nau}) вытекают формулы для начальных моментов
  \begin{equation*}
  \int\limits_0^\infty t^n dF(t)=n! \mathbf{q}(-\mathbf{S})^{-n}\mathbf{u}\,,\enskip 
n=1,2,\ldots
  %\label{e11-nau}
  \end{equation*}
и для преобразования Лап\-ла\-са--Стилть\-еса функции распределения~$F(t)$ 
\begin{multline*}
\tilde{F}(\nu)=\int\limits_0^\infty e^{-\nu t}dF(t)={}\\
{}=1-
\mathbf{q}\mathbf{u}+\mathbf{q}(\nu\mathbf{I}-\mathbf{S})^{-1} \mathbf{s}=1-
\nu\mathbf{q}(\nu\mathbf{I}-\mathbf{S})^{-1}\mathbf{u}\,,
%\label{e12-nau}
\end{multline*}
где $\mathbf{s}=-\mathbf{Su}$. Кроме того,  
мат\-рич\-но-экс\-по\-нен\-ци\-аль\-ные функции распределения обладают 
сле\-ду\-ющи\-ми свойствами~\cite{24-nau}.
\begin{enumerate}[1.]
\item Пусть $F_i(t)=1\hm- \mathbf{q}_i\exp(t\mathbf{S}_i) \mathbf{u}$, 
$i\hm=1,2$,~--- мат\-рич\-но-экс\-по\-нен\-ци\-аль\-ные функции 
распределения и $p_1\hm+p_2\hm=1$. Тогда
\begin{align*}
p_1F_1(t)+p_2F_2(t)&={}\\
&\hspace*{-15mm}{}=1-(p_1\mathbf{q}_1, p_2\mathbf{q}_2)\exp \left( 
t\begin{bmatrix} \mathbf{S}_1 & \mathbf{0}\\
\mathbf{0}& \mathbf{S}_2\end{bmatrix}
\right) \mathbf{u}\,; %\label{e13-naum}
\\
\left( F_1*F_2\right) (t) &={}\\
&\hspace*{-25mm}{}= 1-\left(\mathbf{q}_1,F_1(0) \mathbf{q}_2\right) \exp 
\left( t \begin{bmatrix}
\mathbf{S}_1 & -\mathbf{S}_1\mathbf{uq}_2\\
\mathbf{0} & \mathbf{S}_2\end{bmatrix} \right) \mathbf{u}\,.
%\label{e14-nau}
\end{align*}
\item Пусть $\tau$ и~$\gamma$~--- независимые неотрицательные случайные 
величины с функциями распределения~$F(t)$ и~$G(t)$ соответственно, 
причем~$F(t)$ имеет  
мат\-рич\-но-экс\-по\-нен\-ци\-аль\-ное представление~(\ref{e10-nau}). Тогда 
функция распределения~$H(t)$ случайной величины  $(\tau\hm-\gamma)^+$ 
имеет  
мат\-рич\-но-экс\-по\-нен\-ци\-аль\-ное пред\-став\-ление 
\begin{equation*}
H(t)=1-\mathbf{qU}\exp (t\mathbf{S})\mathbf{u}\,,
%\label{e15-nau}
\end{equation*}
где 
\begin{equation}
\mathbf{U}=\int\limits_0^\infty e^{t\mathbf{S}}dG(t)\,.
\label{e16-nau}
\end{equation}

\item Пусть $F(t)$ имеет мат\-рич\-но-экс\-по\-нен\-ци\-аль\-ное  
представление~(\ref{e10-nau}), а~у~квад\-рат\-ной мат\-ри\-цы~$\mathbf{V}$ 
все собственные числа имеют неотрицательные вещественные части. Тогда
\begin{multline*}
\int\limits_0^\infty e^{-t\mathbf{V}}dF(t)=(1-\mathbf{qu}) \mathbf{I}+
(\mathbf{q}\otimes \mathbf{I})\boldsymbol{\Psi}(\mathbf{Su}\otimes 
\mathbf{I})={}\\
{}=\mathbf{I}-(\mathbf{q}\otimes 
\mathbf{I})\boldsymbol{\Psi}(\mathbf{u}\otimes \mathbf{V})\,,
%\label{e17-nau}
\end{multline*}
где $\boldsymbol{\Psi}=(\mathbf{I}\otimes \mathbf{V}\hm- \mathbf{S}\otimes 
\mathbf{I})^{-1}$. 
  \end{enumerate}
  
  Последнее свойство можно использовать для вычисления 
матриц~$\mathbf{U}$ в~(\ref{e16-nau}) для мат\-рич\-но-экс\-по\-нен\-ци\-аль\-ных 
функций распределения~$G(t)$.
  
  Ясно, что функции распределения фазового типа являются  
мат\-рич\-но-экс\-по\-нен\-ци\-аль\-ны\-ми. Однако их  
мат\-рич\-но-экс\-по\-нен\-ци\-аль\-ные представления 
  \begin{multline*}
  F(t)-1-\mathbf{q}\exp (t\mathbf{S})\mathbf{u}\,,\enskip
  %\label{e18-nau}
    F(0)=1-\mathbf{qu}\,,\\
     \fr{d}{dt}\,F(t)= \mathbf{q}\exp 
(t\mathbf{S})\mathbf{s}\,,\ t>0\,,
  \end{multline*}
с ограничениями
\begin{equation}
\hspace*{-2mm}\left.
\begin{array}{rlrlrl}
\!\!\displaystyle 0<\sum\limits_{j\in \boldsymbol{\mathcal{X}}} q(j)&\leq 1\,,&\ q(i)&\geq0\,,&\ i&\in 
\boldsymbol{\mathcal{X}};
\\[9pt]
\!\!\displaystyle \sum\limits_{j\in \boldsymbol{\mathcal{X}}} \!\!s(i,j)&\leq 0\,,&\ s(i,j)&\geq 0\,,&\ 
i&\not= j\,,\ i, j\in \boldsymbol{\mathcal{X}},
\end{array}\!
\right\}\!\!\!\!
\label{e19-nau}
\end{equation}
где $\mathbf{S}=[s(i,j)]$, следует отличать от мат\-рич\-но-экс\-по\-нен\-ци\-аль\-ных 
представлений этих же функций, но без ограничений~(\ref{e19-nau}). 
Порядок  
мат\-рич\-но-экс\-по\-нен\-ци\-аль\-но\-го представления, удовлетворяющего 
ограничениям~(\ref{e19-nau}), будем называть числом этапов этого 
представления, а~порядок мат\-рич\-но-экс\-по\-нен\-ци\-аль\-но\-го 
представления, не удовлетворяющего этим ограничениям, следуя~\cite{28-nau}, 
будем называть\linebreak
 числом \textit{фиктивных} этапов. Необходимые и 
достаточные условия того, чтобы для функции распределения 
с~рациональным преобразованием Лап\-ла\-са--Стилть\-еса существовало 
представление, \mbox{удовлетворяющее} ограничениям~(\ref{e19-nau}), получены 
в~\cite{29-nau}. Для этого надо, чтобы (а)~функция распределения имела 
непрерывную положительную плотность на правой полуоси и~(б)~ее 
преобразование Лап\-ла\-са--Стилть\-еса имело единственный полюс 
с~максимальной вещественной частью. 

\section{Рациональные потоки событий}

  Рациональный поток групп неоднородных событий 
$(t_l,\boldsymbol{\sigma}_l)$, $l\hm=1,2,\ldots$, можно определить как поток, 
для которого совместное распределение чис\-ла~$\boldsymbol{\sigma}_l$ 
наступивших событий и~длин~$\tau_l$ интервалов между моментами~$t_l$ 
наступления событий дается формулами~(\ref{e1-nau}) и~(\ref{e3-nau}) 
с~матрицами~$\mathbf{A}_{\mathbf{n}}$, $\mathbf{n}\hm\in 
\boldsymbol{\mathcal{N}}^K $, обладающими следующими свойствами:
  \begin{enumerate}[(1)]
\item действительные части собственных чисел мат\-ри\-цы~$\mathbf{A}_{\mathbf{0}}$ 
отрицательны;
\item действительные части собственных чисел мат\-ри\-цы 
$\mathbf{A}\hm= \sum\nolimits_{\mathbf{n}\in 
\boldsymbol{\mathcal{N}}^K} \mathbf{A}_{\mathbf{n}}$ 
неположительны;
\item $\mathbf{Au}=\mathbf{0}$.
\end{enumerate}
  
  Для стационарных версий рациональных потоков дополнительно требуется, 
чтобы начальный вектор~$\bm{\alpha}$ совпадал с решением~$\mathbf{p}$ 
системы линейных уравнений $\mathbf{pA}\hm=0$, $\mathbf{pu}\hm=1$.
  
  Простой рациональный поток однородных событий, также называемый  
мат\-рич\-но-экс\-по\-нен\-циальным потоком~\cite{30-nau},~--- это поток 
событий одного типа, в каждый вызывающий момент которого наступает 
ровно одно событие и для которого плотность совместного распределения 
длин~$\tau_l$ интервалов между моментами наступления событий дается 
формулой~(\ref{e6-nau}) с матрицами~$\mathbf{S}$ и~$\mathbf{R}$, 
обладающими следующими свойствами~\cite{31-nau}:
\begin{itemize}
\item[(а)] вещественные части собственных чисел матрицы~$\mathbf{S}$ 
отрицательны;
\item[(б)] вещественные части собственных чисел матрицы 
$\mathbf{S}\hm+\mathbf{R}$ неположительны; 
\item[(в)] $(\mathbf{S}+\mathbf{R})\mathbf{u}=\mathbf{0}$. 
  \end{itemize}
  
  Примерами простых рациональных потоков однородных событий могут 
служить полумарковские потоки~\cite{22-nau} и процессы 
восстановления~\cite{27-nau}  
с~мат\-рич\-но-экс\-по\-нен\-ци\-аль\-ны\-ми функциями распределения длин 
интервалов между наступлениями событий.
  
  Рациональный поток неоднородных событий~--- это поток событий 
нескольких типов, в каждый вызывающий момент которого наступает ровно 
одно событие. Для такого потока совместное распределение типов 
наступивших событий~$\omega_l$ и длин~$\tau_l$ интервалов между 
моментами наступления событий дается формулой~(\ref{e9-nau}), а на 
матрицы~$\mathbf{S}$ 
и~$\mathbf{R}\hm=\mathbf{R}_1\hm+\mathbf{R}_2+\cdots  + \mathbf{R}_K$ 
накладываются перечисленные выше ограничения~(a)--(в)~\cite{32-nau}. 

\section{Заключение}

  Метод этапов Эрланга~\cite{33-nau} более 100~лет применяется при анализе 
стохастических систем. К~его широкому распространению привело открытие  
мат\-рич\-но-экс\-по\-нен\-ци\-аль\-но\-го представления для функций 
распределения фазового типа~\cite{34-nau} и моделей марковских потоков 
событий~\cite{1-nau, 17-nau}. Эти модели хорошо подходят для анализа 
стохастических систем с~по\-мощью вычислительной техники, 
приспособленной к обработке векторов и матриц, что привело к развитию 
специальных матричных методов анализа стохастических систем.
  
  Метод фиктивных этапов, предложенный в~\cite{28-nau}, позволил 
распространить метод Эрланга на любые распределения с рациональным 
преобразованием 
  Лап\-ла\-са--Стилть\-еса. Использование мат\-рич\-но-экс\-по\-нен\-ци\-аль\-ных 
  представлений для функций распределения~\cite{22-nau, 23-nau, 25-nau} 
  и~потоков случайных событий~\cite{31-nau} с произвольными рациональными 
преобразованиями 
  Лап\-ла\-са--Стилть\-еса упрощает применение метода фиктивных этапов. 
Формальное применение метода фиктивных этапов приводит\linebreak 
к~решению, 
в~котором вероятности, со\-от\-вет\-ст\-ву\-ющие фиктивным этапам, могут оказаться 
отрицательными, б$\acute{\mbox{о}}$льшими единицы или даже 
комплекс\-ны\-ми. Однако вероятности, соответствующие \mbox{не\-фик\-тив\-ным} 
состояниям, будут неотрицательными числами, не превосходящими единицы. 
Существуют различные интерпретации понятий отрицательных вероятностей 
и интенсивностей \mbox{переходов} \cite{35-nau, 36-nau, 37-nau, 38-nau}. Более 
детально ознакомиться с~{марковским} и~рациональным потоками событий, 
а~также с~матричными методами анализа стохастических систем можно 
 в~обзорах~\cite{39-nau, 40-nau, 41-nau, 42-nau, 43-nau, 44-nau}  
и~монографиях~[18, 25, 26, 45--57]. 

{\small\frenchspacing
 {%\baselineskip=10.8pt
 %\addcontentsline{toc}{section}{References}
 \begin{thebibliography}{99}

\bibitem{1-nau}
\Au{Наумов В.\,А.} О~независимой работе подсистем сложной системы~// Тр.~III 
Всесоюзной  
шко\-лы-се\-ми\-на\-ра по теории массового обслуживания.~--- 
М.: МГУ, 1976. №\,2. С.~169--177.
\bibitem{2-nau}
\Au{Бочаров П.\,П., Наумов В.\,А.} Анализ гиперэкспоненциальной двухфазной системы с 
ограниченным накопителем~// Информационные сети и их структура.~--- М.: Наука, 
1976.  
С.~168--180.
\bibitem{3-nau}
\Au{Наумов В.\,А.} Об обслуженной и избыточной нагрузках полнодоступного пучка с 
ограниченной очередью~// Численные методы решения задач математической физики и 
теории систем.~--- М.: УДН, 1977. С.~51--55.
\bibitem{4-nau}
\Au{Наумов В.\,А.} Исследование некоторых многофазных систем массового 
обслуживания: Дис.\ \ldots\ канд. физ.-мат. наук.~--- М.: УДН, 1978.  98~с.
\bibitem{5-nau}
\Au{Lucantoni D.\,M., Meier-Hellstern~K., Neuts M.\,F.} A~single-server queue with server 
vacations and a class of non-renewal arrival processes~// Adv. Appl. Probab., 1990. 
Vol.~22. Iss.~3. P.~676--705.
\bibitem{6-nau}
\Au{Башарин Г.\,П., Кокотушкин~В.\,А., Наумов~В.\,А.} О~методе эквивалентных замен 
расчета фрагментов сетей связи для ЦВМ~// Изв. АН \mbox{СССР}. Техническая кибернетика, 1979. №\,6. С.~92--99.
\bibitem{7-nau}
\Au{Basharin G., Naumov V.} Simple matrix description of peaked and smooth traffic and 
its applications~// 3rd ITC Specialist Seminar on Fundamentals of Teletraffic Theory.~--- M.: 
VINITI, 1984. P.~38--44. 
\bibitem{8-nau}
\Au{Neuts M.\,F.} Renewal processes of phase type~// Nav. Res. Logist.~Q., 1978. 
Vol.~25. Iss.~3. P.~445--454.
\bibitem{9-nau}
\Au{Cinlar E.} Queues with semi-Markovian arrivals~// J.~Appl. Probab., 1967. Vol.~4. Iss.~2.  
P.~365--379.
\bibitem{10-nau}
\Au{Franken P.} Erlangsche Formeln f$\ddot{\mbox{u}}$r semimarkowschen Eingang // 
Elektronische Informationsverarbeitung Kybernetik, 1968. Vol.~4. Iss.~3. P.~197--204.
\bibitem{11-nau}
\Au{Franken P., Kerstan~J.} Bedienungssysteme mit unendlich vielen Bedienungsapparaten~// 
Operationsforschung Mathematische Statistik.~--- Berlin: Akademie-Verlag, 1968. Vol.~I. 
P.~67--76.
\bibitem{12-nau}
\Au{Neuts M.\,F., Chen~S.-Z.} The infinite server queue with semi-Markovian arrivals and negative 
exponential services~// J.~Appl. Probab., 1972. Vol.~9. Iss.~1. P.~178--184.
\bibitem{13-nau}
\Au{Bean N.\,G., Green D.\,A., Taylor~P.\,G.} When is a MAP poisson?~// 2nd Australia--Japan 
Workshop on Stochastic Models in Engineering,
Technology 
and Management Proceedings~/ Eds. J.~Wilson, D.\,N.\,P.~Murthy, S.~Osaki.~--- 
Brisbane: Technology Management Center, University of Queensland, 1996. P.~34--43.
\bibitem{14-nau}
\Au{Наумов В.\,А.} Матричный аналог формулы Эрланга~// Модели распределения 
информации и методы их анализа.~--- М.: ВИНИТИ, 1988. C.~39--43.
\bibitem{15-nau}
\Au{He Q.-M.} Queues with marked customers~// Adv. Appl. Probab., 1996. Vol.~28. 
Iss.~2. P.~567--587.
\bibitem{16-nau}
\Au{He Q.-M., Neuts M.\,F.} Markov chains with marked transitions~// Stoch. Proc. 
Appl., 1998. Vol.~74. P.~37--52.
\bibitem{17-nau}
\Au{Neuts M.\,F.} A versatile Markovian point process.~--- Newark, DE: 
University of Delaware, Department of Statistics and Computer Science, 1977.
 Technical Report 77/13. 29~p.
\bibitem{18-nau}
\Au{Neuts M.\,F.} Structured stochastic matrices of $M/G/1$ type and their applications.~--- New 
York, NY, USA: Marcel Dekker, 1989. 512~p.
\bibitem{19-nau}
\Au{Lucantoni D.\,M.} New results on the single server queue with a batch Markovian arrival 
process~// Communications Statistics. Stochastic Models, 1991. Vol.~7. Iss.~1. P.~1--46. 
\bibitem{20-nau}
\Au{Narayana S., Neuts M.\,F.} The first two moment matrices of the counts for the Markovian 
arrival processes~// Communications Statistics. Stochastic Models, 1992. Vol.~8. Iss.~3. P.~459--477. 
\bibitem{21-nau}
\Au{Nielsen B.\,F., Nilsson L.\,A.\,F., Thygesen~U.\,H., Beyer~J.\,E.} Higher order moments and 
conditional asymptotics of the batch Markovian arrival process~// Stoch. Models, 2007. Vol.~23. 
Iss.~1. P.~1--26.
\bibitem{22-nau}
\Au{Бочаров П.\,П., Наумов В.\,А.} O~некоторых системах массового обслуживания 
конечной емкости~// Проб\-ле\-мы передачи информации, 1977. Т.~13. №\,4. С.~96--104.
\bibitem{23-nau}
\Au{Наумов В.\,А.} Об однолинейной системе с ограниченным накопителем и заявками 
нескольких видов~// Модели систем распределения информации и их анализ.~--- М.: 
Наука, 1982. C.~77--82.
\bibitem{24-nau}
\Au{Наумов В.\,А.} О~функциях распределения с рациональным преобразованием  
Лап\-ла\-са--Стилть\-еса~// Анализ информационно-вычислительных систем.~--- М.: 
УДН, 1986. С.~47--56.
\bibitem{25-nau}
\Au{Бочаров П.\,П., Печинкин~А.\,В.} Теория массового обслуживания.~--- М.: РУДН, 
1995. 528~с.
\bibitem{26-nau}
\Au{Bocharov P.\,P., D'Apice~C., Pechinkin~A.\,V., Salerno~S.} Queueing theory.~--- Utrecht--Boston: 
VSP, 2004. 446~p.
\bibitem{27-nau}
\Au{Asmussen S., Bladt M.} Renewal theory and queueing algorithms for matrix-exponential 
distributions~// Matrix-analytic methods in stochastic models~/
Eds. A.\,S.~Alfa, S.~Chakravarty.~--- New York, NY, USA: Marcel 
Dekker, 1996. P.~313--341.
\bibitem{28-nau}
\Au{Cox D.\,R.} A use of complex probabilities in the theory of stochastic processes~// Math. 
Proc. Cambridge, 1955. Vol.~51. Iss.~2. P.~313--319. 
\bibitem{29-nau}
\Au{O'Cinneide C.\,A.} Characterization of phase-type distributions~// Communications Statistics. 
Stochastic Models, 1990. Vol.~6. Iss.~1. P.~1--57.
\bibitem{30-nau}
\Au{Bodrog L., Horv$\acute{\mbox{a}}$th~A., Telek~M.} On the properties of moments of matrix 
exponential distributions and matrix exponential processes~// Dagstuhl Seminar Proceedings, 2008. 
Vol.~07461. Paper~1394.
\bibitem{31-nau}
\Au{Asmussen S., Bladt M.} Point processes with finite-dimensional conditional probabilities~// 
Stoch. Proc. \mbox{Appl.}, 1999. Vol.~82. Iss.~1. P.~127--142.
\bibitem{32-nau}
\Au{Horvath G., Telek M.} Acceptance-rejection methods for generating random variables from 
matrix exponential distribution and rational arrival processes~// Matrix-analytic methods in stochastic 
models~/ Eds. G.~Latouche, V.~Ramaswami, J.~Sethuraman, \textit{et al.}~--- 
New York, NY, USA: Springer, 2012. P.~123--144.
\bibitem{33-nau}
\Au{Erlang A.\,K.} \mbox{L{\!\ptb{\o}}sning} af nogle Problemer fra Sandsynlighedsregningen af 
Betydning for de automatiske Telefoncentraler~// Elektroteknikeren, 1917. Iss.~13. P.~5--13.
\bibitem{34-nau}
\Au{Neuts M.\,F.} Probability distribution of phase type~// Liber Amicorum Professor Emeritus 
H.~Florin.~--- Ottignies-Louvain-la-Neuve, Belgium: University of Louvain, Department of Mathematics, 
1975. P.~173--206.
\bibitem{35-nau}
\Au{Bartlett M.\,S.} Negative probability~// Math. Proc. Cambridge, 1945. Vol.~41. Iss.~1. P.~71--73.
\bibitem{36-nau}
\Au{Cox D.\,R.} The analysis of non-Markovian stochastic processes by the inclusion of 
supplementary variables~// Math. Proc. Cambridge, 1955. Vol.~51. Iss.~3. P.~433--441. 
\bibitem{37-nau}
\Au{Bladt M., Neuts M.\,F.} Matrix-exponential distributions: Calculus and interpretations via flows~// 
Stoch. Models, 2003. Vol.~19. Iss.~1. P.~113--124.
\bibitem{38-nau}
\Au{Khrennikov A.} Interpretations of probability.~--- 2nd ed.~--- Berlin: Walter de Gruyter, 2009. 
237~p.
\bibitem{39-nau}
\Au{Наумов В.\,А.} Марковские модели потоков требований~// Системы массового 
обслуживания и информатика.~--- М.: УДН, 1987. С.~67--73.
\bibitem{40-nau}
\Au{Asmussen S.} Matrix-analytic models and their analysis~// Scand. J.~Stat., 2000. 
Vol.~27. Iss.~2. P.~193--226.
\bibitem{41-nau}
\Au{Bladt M.} A~review on phase-type distributions and their use in risk theory~// ASTIN Bull., 
2005. Vol.~35. No.\,1. P.~145--161.
\bibitem{42-nau}
\Au{Artalejo J.\,R., G$\acute{\mbox{o}}$mez-Corral~A.} Markovian arrivals in stochastic 
modelling: A~survey and some new results~// SORT~--- Stat. Oper. Res.~T., 2010. 
Vol.~34. Iss.~2. P.~101--144.
\bibitem{43-nau}
\Au{Вишневский В.\,М., Дудин~А.\,Н.} Системы массового обслуживания с 
коррелированными входными потоками и их применение для моделирования 
телекоммуникационных сетей~// Автоматика и телемеханика, 2017. №\,8. С.~3--59.
\bibitem{44-nau}
\Au{Basharin G., Naumov~V., Samouylov~K.} On Markovian modelling of arrival processes~// 
Stat. Pap., 2018. Vol.~59. Iss.~4. P.~1533--1540. 
\bibitem{45-nau}
\Au{Neuts M.\,F.} Matrix-geometric solutions in stochastic models: An algorithmic approach.~--- 
Baltimore, MA, USA: The John Hopkins University Press, 1981. 332~p.
\bibitem{46-nau}
\Au{Latouche G., Ramaswami~V.} Introduction to matrix analytic methods in stochastic modeling.~--- 
Philadelphia, PA, USA: ASA \& SIAM, 1999. 334~p.
\bibitem{47-nau}
\Au{Asmussen S.} Applied probability and queues.~--- New York, NY, USA: Springer, 2003. 
438~p.
\bibitem{48-nau}
\Au{Breuer L., Baum D.} An introduction to queueing theory and matrix-analytic methods.~--- 
Dordrecht: Springer, 2005. 272~p.
\bibitem{49-nau}
\Au{Bini D.\,A., Latouche~G., Meini~B.} Numerical methods for structured Markov chains.~--- 
New York, NY, USA: Oxford University Press, 2005. 336~p.
\bibitem{50-nau}
\Au{Asmussen S., O'Cinneide~C.\,A.} Matrix-exponential distributions~// Encyclopedia of statistical 
sciences~/ Eds. S.~Kotz, C.\,B.~Read, N.~Balakrishnan, 
B.~Vidakovic, N.\,L.~Johnson.~--- Hoboken, NJ, USA: John Wiley \& Sons, 2006. Vol.~3. P.~1--5.
doi: 10.1002/0471667196.ess1092.
\bibitem{51-nau}
\Au{Li Q.-L.} Constructive computation in stochastic models with applications.~--- Berlin: Springer-Verlag, 2009. 650~p.
\bibitem{52-nau}
\Au{Lipsky L.} Queueing theory: A~linear algebraic approach.~--- 2nd ed.~--- New York, NY, 
USA: Springer, 2009. 548~p.
\bibitem{53-nau}
\Au{Alfa A.\,S.} Queueing theory for telecommunications.~--- New York, NY, USA: Springer, 2010. 
238~p.
\bibitem{54-nau}
\Au{He Q.-M.} Fundamentals of matrix-analytic methods.~--- New York, NY, USA: Springer, 2014. 
349~p.
\bibitem{55-nau}
\Au{Buchholz P., Kriege~J., Felko~I.} Input modeling with phase-type distributions and Markov 
models. Theory and applications.~--- New York, NY, USA: Springer, 2014. 127~p.
\bibitem{56-nau}
\Au{Наумов В.\,А., Самуйлов~В.\,А., Гайдамака~Ю.\,В.} Мультипликативные решения 
конечных цепей Маркова.~--- М.: РУДН, 2015. 159~с.
\bibitem{57-nau}
\Au{Bladt M., Nielsen B.\,F.} Matrix-exponential distributions in applied probability.~--- Boston, MA, USA: 
Springer, 2017. 736~p.
\end{thebibliography}

 }
 }

\end{multicols}

\vspace*{-12pt}

\hfill{\small\textit{Поступила в~редакцию 02.07.20}}

\vspace*{8pt}

%\pagebreak

\newpage

\vspace*{-28pt}

%\hrule

%\vspace*{2pt}

%\hrule

%\vspace*{-2pt}

\def\tit{ON MARKOVIAN AND RATIONAL ARRIVAL PROCESSES.~II}


\def\titkol{On Markovian and rational arrival processes.~II}


\def\aut{V.\,A.~Naumov$^1$ and~К.\,Е.~Samouylov$^{2,3}$}

\def\autkol{V.\,A.~Naumov and~К.\,Е.~Samouylov}

\titel{\tit}{\aut}{\autkol}{\titkol}

\vspace*{-11pt}


   \noindent
   $^1$Service Innovation Research Institute, 8A Annankatu, Helsinki 00120, Finland

\noindent
$^2$Peoples' Friendship University of Russia (RUDN University), 6~Miklukho-Maklaya Str., Moscow 
117198, Russian\linebreak
$\hphantom{^1}$Federation

\noindent
$^3$Institute of Informatics Problems, Federal Research Center ``Computer Science and Control'' 
of the Russian\linebreak
$\hphantom{^1}$Academy of Sciences, 44-2~Vavilov Str., Moscow 119333, Russian Federation

  

\def\leftfootline{\small{\textbf{\thepage}
\hfill INFORMATIKA I EE PRIMENENIYA~--- INFORMATICS AND
APPLICATIONS\ \ \ 2020\ \ \ volume~14\ \ \ issue\ 4}
}%
 \def\rightfootline{\small{INFORMATIKA I EE PRIMENENIYA~---
INFORMATICS AND APPLICATIONS\ \ \ 2020\ \ \ volume~14\ \ \ issue\ 4
\hfill \textbf{\thepage}}}

\vspace*{3pt} 
  
  
   
   
  \Abste{This article is the second part of the review carried out within the framework of the RFBR 
project No.\,19-17-50126. The purpose of this review is to get the interested readers familiar with the 
basics of the theory of Markovian arrival processes to facilitate the application of these models in practice 
and, if necessary, to study them in detail. In the first part of the review, the properties of the general 
Markovian arrival processes are presented and their relationship with Markov additive processes and 
Markov renewal processes is shown. In the second part of the review, the important for applications 
subclasses of Markovian arrival processes, i.\,e., simple and batch arrival processes of homogeneous and 
heterogeneous arrivals, are considered. It is shown how the properties of Markovian arrival processes are 
associated with the product form of stationary distributions of Markov systems. In conclusion, 
matrix-exponential distributions and rational arrival processes are discussed that expand the capabilities of 
Markovian arrival processes for modeling complex systems, while preserving the convenience of analyzing 
them using computations.}
  
  \KWE{Markov chain; Markovian arrival process; Markov additive process; MAP; MArP}
  
  
\DOI{10.14357/19922264200406} 

%\vspace*{-20pt}

  \Ack
  \noindent
  The reported study was funded by RFBR, project No.\,19-17-50126. 
  

%\vspace*{6pt}

  \begin{multicols}{2}

\renewcommand{\bibname}{\protect\rmfamily References}
%\renewcommand{\bibname}{\large\protect\rm References}

{\small\frenchspacing
 {%\baselineskip=10.8pt
 \addcontentsline{toc}{section}{References}
 \begin{thebibliography}{99}
  
  \bibitem{1-nau-1}
  \Aue{Naumov, V.\,A.} 1976. O~nezavisimoy rabote podsistem slozhnoy sistemy [About independent 
operation of subsystems of a complex system]. \textit{Tr. III Vsesoyuznoy shkoly-seminara po teorii 
massovogo obsluzhivaniya} [3th All-Russian School-Seminar of Queuing Theory Proceedings]. 
Moscow. 2:169--177.
  \bibitem{2-nau-1}
  \Aue{Bocharov, P.\,P., and V.\,A.~Naumov.} 1976. Analiz gipereksponentsial'noy dvukhfaznoy sistemy 
s~ogranichennym nakopitelem [Analysis of a hyperexponential two-phase system with a limited storage]. 
\textit{Informatsionnye seti i~ikh struktura} [Information networks and their structure]. Moscow: 
Nauka. 
  168--180.
  \bibitem{3-nau-1}
  \Aue{Naumov, V.\,A.} 1977. Ob obsluzhennoy i~izbytochnoy nagruzkakh polnodostupnogo puchka 
s~ogranichennoy ochered'yu [About serviced and excessive loads of a fully accessible bundle with a limited 
queue]. \textit{Chislennye metody resheniya zadach matematicheskoy fiziki i~teorii system} 
[Numerical methods for solving problems of mathematical physics and systems theory]. Moscow: UDN. 
51--55.
  \bibitem{4-nau-1}
  \Aue{Naumov, V.\,A.} 1978. Issledovanie nekotorykh mnogofaznykh sistem massovogo obsluzhivaniya 
[Research of some multiphase queuing systems].  Moscow: UDN.  PhD Thesis. 98~p.
  \bibitem{5-nau-1}
  \Aue{Lucantoni, D.\,M., K.~Meier-Hellstern, and M.\,F.~Neuts.} 1990. A single-server queue with 
server vacations and a~class of non-renewal arrival processes. \textit{Adv. Appl. Probab.} 
22(3):676--705.
  \bibitem{6-nau-1}
  \Aue{Basharin, G.\,P., V.\,A.~Kokotushkin, and V.\,A.~Naumov.} 1979. O~metode ekvivalentnykh 
zamen rascheta fragmentov setey svyazi dlya TsVM [On the method of equivalent substitutions for 
calculating fragments of communication networks for a central computer]. \textit{Engineering Cybernetics}
 6:92--99.
  \bibitem{7-nau-1}
  \Aue{Basharin, G.\,P., and V.\,A.~Naumov.} 1984. Simple matrix description of peaked and smooth 
traffic and its applications. \textit{3rd ITC Specialist Seminar on Fundamentals of Teletraffic Theory}. 
Moscow: VINITI. 
  38--44. 
  \bibitem{8-nau-1}
  \Aue{Neuts, M.\,F.} 1978. Renewal processes of phase type. \textit{Nav. Res. Logist.~Q.} 25(3):445--454.
  \bibitem{9-nau-1}
  \Aue{Cinlar, E.} 1967. Queues with semi-Markovian arrivals. \textit{J.~Appl. Probab.} 4(2):365--379.
  \bibitem{10-nau-1}
  \Aue{Franken, P.} 1968. Erlangsche Formeln f$\ddot{\mbox{u}}$r semimarkowschen Eingang. 
\textit{Elektronische Informationsverarbeitung  Kybernetik} 4(3):197--204.
  \bibitem{11-nau-1}
  \Aue{Franken, P., and J.~Kerstan.} 1968. Bedienungssysteme mit unendlich vielen 
Bedienungsapparaten. \textit{Operationsforschung Mathematische Statistik} 1:67--76.
  \bibitem{12-nau-1}
  \Aue{Neuts, M.\,F., and S.-Z.~Chen.} 1972. The infinite server queue with semi-Markovian arrivals 
and negative exponential services. \textit{J.~Appl. Probab.} 9(1):178--184.
  \bibitem{13-nau-1}
  \Aue{Bean, N.\,G., D.\,A.~Green, and P.\,G.~Taylor.} 1996. When is a MAP poisson? \textit{2nd 
  Australia--Japan Workshop on Stochastic Models in Engineering, Technology and Management 
Proceedings}. Eds. J.~Wilson, D.\,N.\,P.~Murthy, and S.~Osaki. 
Brisbane: Technology Management Center, University of Queensland. 34--43.
  \bibitem{14-nau-1}
  \Aue{Naumov, V.\,A.} 1988. Matrichnyy analog formuly Erlanga [The matrix analogue of a formula of 
Erlang]. \textit{Modeli raspredeleniya informatsii i~metody ikh analiza} [Information distribution 
models and methods for their analysis]. Moscow: VINITI. 39--43.
  \bibitem{15-nau-1}
  \Aue{He, Q.-M.} 1996. Queues with marked customers. \textit{Adv. Appl. Probab.} 
28(2):567--587.
  \bibitem{16-nau-1}
  \Aue{He, Q.-M., and M.\,F. Neuts.} 1998. Markov chains with marked transitions. \textit{Stoch. 
Proc. Appl.} 74:37--52.
  \bibitem{17-nau-1}
  \Aue{Neuts, M.\,F.} 1977. A~versatile Markovian point process.  
Newark, DE: University of Delaware, Department of Statistics and Computer Science.
Technical Report 77/13. 29~p.
  \bibitem{18-nau-1}
  \Aue{Neuts, M.\,F.} 1989. \textit{Structured stochastic matrices of $M/G/1$ type and their 
applications}. New York, NY: Marcel Dekker. 512~p.
  \bibitem{19-nau-1}
  \Aue{Lucantoni, D.\,M.} 1991. New results on the single server queue with a batch Markovian arrival 
process. \textit{Communications Statistics. Stochastic Models} 7(1):1--46. 
  \bibitem{20-nau-1}
  \Aue{Narayana, S., and M.\,F.~Neuts.} 1992. The first two moment matrices of the counts for the 
Markovian arrival processes. \textit{Communications Statistics. Stochastic Models} 8(3):459--477. 
  \bibitem{21-nau-1}
  \Aue{Nielsen, B.\,F., L.\,A.\,F.~Nilsson, U.\,H.~Thygesen, and J.\,E.~Beyer}. 2007. Higher order 
moments and conditional asymptotics of the batch Markovian arrival process. \textit{Stoch. Models} 
23(1):1--26.
  \bibitem{22-nau-1}
  \Aue{Bocharov, P.\,P., and V.\,A.~Naumov.} 1977. O~nekotorykh sistemakh massovogo 
obsluzhivaniya konechnoy emkosti [On some queueing systems of finite capacity]. \textit{Problemy 
peredachi informatsii} [Problems of Information Transmission] 13(4):96--104.
  \bibitem{23-nau-1}
  \Aue{Naumov, V.\,A.} 1982. Ob odnolineynoy sisteme s~ogranichennym nakopitelem i~zayavkami 
neskol'kikh vidov [About a single-line system with limited storage and multiple types of requests]. 
\textit{Modeli sistem raspredeleniya informatsii i~ikh analiz} [Models of information distribution 
systems and methods for their analysis]. Moscow: Nauka. 77--82.
  \bibitem{24-nau-1}
  \Aue{Naumov, V.\,A.} 1986. O~funktsiyakh raspredeleniya s~ratsio\-nal'nym preobrazovaniem  
Laplasa--Stilt'esa [On distribu\-tion functions with rational Laplace--Stiltjes transformation]. \textit{Analiz 
  informatsionno-vychislitel'nykh \mbox{system}}
   [\mbox{Analysis} of information and computing systems]. Moscow: 
UDN. 47--56.
  \bibitem{25-nau-1}
  \Aue{Bocharov, P.\,P., and A.\,V.~Pechinkin.} 1995. \textit{Teoriya massovogo obsluzhivaniya} 
[Queueing theory]. Moscow: RUDN. 528~p.
  \bibitem{26-nau-1}
  \Aue{Bocharov, P.\,P., C.~D'Apice, A.\,V.~Pechinkin, and S.~Salerno.} 2004. \textit{Queueing 
theory}. Utrecht--Boston: VSP. 446~p.
  \bibitem{27-nau-1}
  \Aue{Asmussen, S., and M.~Bladt}. 1996. Renewal theory and queueing algorithms for 
  matrix-exponential distributions. \textit{Matrix-analytic methods in stochastic models}. 
  Eds. A.\,S.~Alfa and 
S.~Chakravarty. New York, NY: Marcel Dekker. 313--341.
  \bibitem{28-nau-1}
  \Aue{Cox, D.\,R.} 1955. A~use of complex probabilities in the theory of stochastic processes. 
\textit{Math. Proc. Cambridge} 51(2):313--319.
  \bibitem{29-nau-1}
  \Aue{O'Cinneide, C.\,A.} 1990. Characterization of phase-type distributions. \textit{Communications 
Statistics. Stochastic Models} 6(1):1--57.
  \bibitem{30-nau-1}
  \Aue{Bodrog, L., A.~Horv$\acute{\mbox{a}}$th, and M.~Telek.} 2008. On the properties of 
moments of matrix exponential distributions and matrix exponential processes. 
\textit{Dagstuhl Seminar Proceedings} 07461:1394.
  \bibitem{31-nau-1}
  \Aue{Asmussen, S., and M.~Bladt.} 1999. Point processes with finite-dimensional conditional 
probabilities. \textit{Stoch. Proc. Appl.} 82(1):127--142.
  \bibitem{32-nau-1}
  \Aue{Horvath, G., and M.~Telek.} 2012. Acceptance-rejection methods for generating random 
variables from matrix exponential distribution and rational arrival processes. \textit{Matrix-analytic 
methods in stochastic models.} Eds. G.~Latouche, V.~Ramaswami, J.~Sethuraman, \textit{et al.} New York, NY: Springer. 123--144.
  \bibitem{33-nau-1}
  \Aue{Erlang, A.\,K.} 1917. \mbox{L{\!\ptb{\o}}sning} af nogle Problemer fra 
Sandsynlighedsregningen af Betydning for de automatiske Telefoncentraler. \textit{Elektroteknikeren} 
13:5--13.
  \bibitem{34-nau-1}
  \Aue{Neuts, M.\,F.} 1975. Probability distribution of phase type. \textit{Liber Amicorum Professor 
Emeritus H.~Florin}. Ottignies-Louvain-la-Neuve, Belgium: University of Louvain, Department of 
Mathematics.  
173--206.
  \bibitem{35-nau-1}
  \Aue{Bartlett, M.\,S.} 1945. Negative probability. \textit{Math. Proc. 
Cambridge} 41(1):71--73.
  \bibitem{36-nau-1}
  \Aue{Cox, D.\,R.} 1955. The analysis of non-Markovian stochastic processes by the inclusion of 
supplementary variables. \textit{Math. Proc. Cambridge} 
51(3):433--441.
  \bibitem{37-nau-1}
  \Aue{Bladt, M., and M.\,F.~Neuts.} 2003. Matrix-exponential distributions: Calculus and 
interpretations via flows. \textit{Stoch. Models} 19(1):113--124.
  \bibitem{38-nau-1}
  \Aue{Khrennikov, A.} 2009. \textit{Interpretations of probability}. 2nd ed. Berlin: Walter de 
Gruyter. 237~p.
  \bibitem{39-nau-1}
  \Aue{Naumov, V.\,A.} 1987. Markovskie modeli potokov trebovaniy [Markov models of demand 
flows]. \textit{Sistemy massovogo obsluzhivaniya i~informatika} [Queuing systems and computer 
science]. Moscow: UDN. 67--73.
  \bibitem{40-nau-1}
  \Aue{Asmussen, S.} 2000. Matrix-analytic models and their analysis. \textit{Scand. 
J.~Stat.} 27(2):193--226.
  \bibitem{41-nau-1}
  \Aue{Bladt, M.} 2005. A~review on phase-type distributions and their use in risk theory. \textit{ASTIN 
Bull.} 35(1):145--161.
  \bibitem{42-nau-1}
  \Aue{Artalejo, J.\,R., and A.~G$\acute{\mbox{o}}$mez-Corral.} 2010. Markovian arrivals in 
stochastic modelling: A~survey and some new results. \textit{SORT~--- Stat. Oper. Res.~T.}  
34(2):101--144.
  \bibitem{43-nau-1}
  \Aue{Vishnevskiy, V.\,M., and A.\,N.~Dudin.} 2017. Queueing systems with correlated arrival flows 
and their applications to modeling telecommunication networks. \textit{Automat. Rem. Contr.} 
78(8):1361--1403.
  \bibitem{44-nau-1}
  \Aue{Basharin, G., V.~Naumov, and K.~Samouylov.} 2018. On Markovian modelling of arrival 
processes. \textit{Stat. Pap.} 59(4):1533--1540.
  \bibitem{45-nau-1}
  \Aue{Neuts, M.\,F.} 1981. \textit{Matrix-geometric solutions in stochastic models: An algorithmic 
approach.} Baltimore, MA: The John Hopkins University Press. 332~p.
  \bibitem{46-nau-1}
  \Aue{Latouche, G., and V.~Ramaswami.} 1999. \textit{Introduction to matrix analytic methods in 
stochastic modeling}. Philadelphia, PA: ASA \& SIAM. 334~p.
  \bibitem{47-nau-1}
  \Aue{Asmussen, S.} 2003. \textit{Applied probability and queues}. New  York, NY: Springer. 
438~p.
  \bibitem{48-nau-1}
  \Aue{Breuer, L., and D.~Baum.} 2005. \textit{An introduction to queueing theory and 
  matrix-analytic methods.} Dordrecht: Springer. 272~p.
  \bibitem{49-nau-1}
  \Aue{Bini, D.\,A., G.~Latouche, and B.~Meini.} 2005. \textit{Numerical methods for structured 
Markov chains}. New  York, NY: Oxford University Press. 336~p.
  \bibitem{50-nau-1}
  \Aue{Asmussen, S., and C.\,A.~O'Cinneide}. 2006. Matrix-exponential distributions. 
\textit{Encyclopedia of statistical sciences.} Eds. S.~Kotz, C.\,B.~Read, N.~Balakrishnan, 
B.~Vidakovic, and N.\,L.~Johnson. Hoboken, NJ: John Wiley \&~Sons. 3:1--5. doi: 10.1002/0471667196.ess1092.pub2.
  \bibitem{51-nau-1}
  \Aue{Li, Q.-L.} 2009. \textit{Constructive computation in stochastic models with applications}. 
Berlin: 
  Springer-Verlag. 650~p.
  \bibitem{52-nau-1}
  \Aue{Lipsky, L.} 2009. \textit{Queueing theory: A~linear algebraic approach}. 2nd ed. New York, 
NY: Springer. 548~p.
  \bibitem{53-nau-1}
  \Aue{Alfa, A.\,S.} 2010. \textit{Queueing theory for telecommunications}. New York, NY: 
Springer. 238 p.
  \bibitem{54-nau-1}
  \Aue{He, Q.-M.} 2014. \textit{Fundamentals of matrix-analytic methods.} New York, NY: 
Springer. 349 p.
  \bibitem{55-nau-1}
  \Aue{Buchholz, P., J.~Kriege, and I.~Felko.} 2014. \textit{Input modeling with phase-type 
distributions and Markov models. Theory and applications.} New York, NY: Springer. 127~p.
  \bibitem{56-nau-1}
  \Aue{Naumov, V.\,A., K.\,E.~Samuylov, and Yu.\,V.~Gaidamaka.} 2015. \textit{Mul'tiplikativnye 
resheniya konechnykh tsepey Markova} [Multiplicative solutions of finite Markov chains]. Moscow: 
RUDN. 159~p.
  \bibitem{57-nau-1}
  \Aue{Bladt, M., and B.\,F.~Nielsen.} 2017. \textit{Matrix-exponential distributions in applied 
probability}. Boston, MA: Springer. 736~p.
\end{thebibliography}

 }
 }

\end{multicols}

\vspace*{-3pt}

\hfill{\small\textit{Received July 2, 2020}}

%\pagebreak

  %\vspace*{-24pt}
  
  \Contr
  
  \noindent
  \textbf{Naumov Valeriy A.} (b.\ 1950)~--- Candidate of Science (PhD) in physics and mathematics, 
scientific director, Service Innovation Research Institute, 8A~Annankatu, Helsinki 00120, Finland; 
\mbox{valeriy.naumov@pfu.fi}
  
  \vspace*{3pt}
  
  \noindent
  \textbf{Samouylov Konstantin E.} (b.\ 1955)~--- Doctor of Science in technology, professor, Head of 
Department,  Peoples' Friendship 
University of Russia (RUDN University), 6~Miklukho-Maklaya Str., Moscow 117198, Russian 
Federation; senior scientist, Institute of Informatics Problems, Federal Research Center ``Computer 
Science and Control'' of the Russian Academy of Sciences, 44-2~Vavilov Str., Moscow 119333, Russian 
Federation; 
  \mbox{samuylov-ke@rudn.university}
  
\label{end\stat}

\renewcommand{\bibname}{\protect\rm Литература} 
  
  