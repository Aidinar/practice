\def\stat{budzko-sochenkov}

\def\tit{ОБ ОДНОМ ПОДХОДЕ К ФОРМИРОВАНИЮ\\ В~УСЛОВИЯХ 
ВЫСОКОЙ НЕОПРЕДЕЛЕННОСТИ\\ МАРКЕРОВ КОНФИДЕНЦИАЛЬНОСТИ 
В~СИСТЕМАХ\\ ИНТЕНСИВНОГО ИСПОЛЬЗОВАНИЯ ДАННЫХ$^*$}

\def\titkol{Об одном подходе к~формированию в~условиях высокой неопределенности 
маркеров конфиденциальности в~СИИД} %системах интенсивного использования данных}

\def\aut{В.\,И.~Будзко$^1$, В.\,В.~Ядринцев$^2$, 
И.\,В.~Соченков$^3$, В.\,И.~Королёв$^4$, 
В.\,Г.~Беленков$^5$}

\def\autkol{В.\,И.~Будзко, В.\,В.~Ядринцев, 
И.\,В.~Соченков и~др.}
%$^3$, В.\,И.~Королёв, В.\,Г.~Беленков}

\titel{\tit}{\aut}{\autkol}{\titkol}

\index{Будзко В.\,И.}
\index{Ядринцев В.\,В.}
\index{Соченков И.\,В.}
\index{Королёв В.\,И.}
\index{Беленков В.\,Г.}
\index{Budzko V.\,I.}
\index{Yadrintsev V.\,V.}
\index{Sochenkov I.\,V.}
\index{Korolev V.\,I.}
\index{Belenkov V.\,G.}


{\renewcommand{\thefootnote}{\fnsymbol{footnote}} \footnotetext[1]
{Работа выполнена при частичной финансовой поддержке РФФИ (проект 18-29-03215), 
экспериментальные исследования выполнены при поддержке Программы РУДН <<5-100>>.}}


\renewcommand{\thefootnote}{\arabic{footnote}}
\footnotetext[1]{Федеральный исследовательский центр <<Информатика 
и~управление>> Российской академии наук, \mbox{vbudzko@ipiran.ru}}
\footnotetext[2]{Федеральный исследовательский центр <<Информатика 
и~управление>> Российской академии наук; Российский университет дружбы народов, 
\mbox{vvyadrincev@gmail.com}}
\footnotetext[3]{Федеральный исследовательский центр <<Информатика 
и~управление>> Российской академии наук, \mbox{sochenkov@isa.ru}}
\footnotetext[4]{Федеральный исследовательский центр <<Информатика 
и~управление>> Российской академии наук, \mbox{vkorolev@ipiran.ru}}
\footnotetext[5]{Федеральный исследовательский центр <<Информатика 
и~управление>> Российской академии наук, \mbox{vbelenkov@ipiran.ru}}

\vspace*{-10pt}

\Abst{Основные задачи, результаты решения которых отражены в~статье, связаны 
с~формированием маркеров конфиденциальности (МК) при их применении в~системах 
интенсивного использования данных (СИИД) в~условиях, когда состав и~структура охраняемых 
сведений не может быть заранее определена в~связи с~отсутствием данных или с~высокой 
динамикой их изменения либо их определение нецелесообразно в~связи с~большим (или 
неограниченным) количеством сущностей, сведения о которых подлежат охране. В~данной 
работе предложен подход к~формированию в~указанных условиях МК
 текстовых материалов. Сформулирована логика процесса 
семантической обработки текста, позволяющего в~условиях высокой неопределенности 
состава и~структуры охраняемых сведений формировать МК при 
их применении для обеспечения информационной безопас\-ности в~СИИД. 
Полученные экспериментальные результаты позволяют перейти 
к~практической реализации рассмотренного подхода в~СИИД.}

\KW{маркер конфиденциальности; информационная безопасность; интенсивное 
использование данных; кластер; семантика; технические средства обеспечения безопас\-ности; 
интеллектуальные задачи безопас\-ности; текстовая классификация; обнаружение 
текстовых заимствований}

\DOI{10.14357/19922264200410} 
  
%\vspace*{9pt}


\vskip 10pt plus 9pt minus 6pt

\thispagestyle{headings}

\begin{multicols}{2}

\label{st\stat}

\section{Введение}

\vspace*{-2pt}

  Рассмотрение проблематики формирования МК
текстовых материалов при их применении для обеспечения информационной 
безопас\-ности (далее~--- формирование МК) 
в~СИИД (Data Intensive Domains~--- DID) 
продолжает прикладные исследования, результаты которых отражены 
авторами  
в~работах~[1--7].
   
  Рассматриваемый в~работе возможный подход к~формированию МК
   основан на использовании методов анализа содержания 
текстов и~не зависит от конфиденциальности материала. Также выполнено 
экспериментальное исследование, в~рамках которого промоделирована 
применимость методов анализа текстов к~решению задач предотвращения 
утечек информации ограниченного доступа.
  
  При рассмотрении подхода используются следующие основные 
определения и~понятия: 
  
  \textit{Системы интенсивного использования данных}~--- 
автоматизированные информационные системы, обеспечивающие анализ  
и~управ\-ле\-ние данными, обработку информации и~решение 
функциональных исследовательских и~прикладных задач в~различных 
областях с~DID. 
  
  \textit{Артефакт}~--- обособленный каким-либо образом контент или 
фрагмент контента, рас\-смат\-ри\-ва\-емый как единое целое совместно с~его 
атрибутикой: учетными и~иными сопровождающими данными. 
  
  \textit{Сущность}~--- именование конкретной области деятельности, либо 
принадлежащего ей объекта, либо по\-ка\-за\-те\-ля/ха\-рак\-те\-ри\-сти\-ки/ви\-да 
сведений, упоминание или конкретная информация о~которых подлежат 
защите.
  
  \textit{Маркер конфиденциальности}~--- метка ка\-ко\-го-либо рода 
(например, слово), в~явном виде определяющая необходимость защиты 
и~уровень конфиденциальности сведений фрагмента или \mbox{конкретного} артефакта в~целом.
  
  \textit{Частный маркер конфиденциальности}~--- метка какого-ли\-бо 
рода (например, слово, текст, агрегат текстов и/или данных), в~явном виде 
опре\-деляющая необходимость защиты и~уровень конфиденциальности 
сведений, относящихся к~{конкретному} упоминанию сущности в~контенте 
артефакта.
  
  Маркер конфиденциальности  относятся к~результату обработки запроса 
  в~целом или к~его разделам~--- объектам доступа для пользователя, выдавшего 
запрос (далее~--- объекты доступа). Они отражают наличие в~результате или 
в~его разделах следующих сведений, которые относятся или могут относиться 
к~охраняемым сведениям в~этих областях деятельности:
  \begin{itemize}
\item информация о конкретных областях деятельности;
\item информация о конкретном объекте;
\item информация о показателях/характеристиках конкретного объекта или 
конкретной области деятельности.
  \end{itemize}
  
  Маркер конфиденциальности также отражает уровни доступности 
сведений, содержащихся в~результате обработки запроса или в~его разделах, 
относящихся:
  \begin{itemize}
  \item к~конкретным областям деятельности или к~конкретным объектам;
\item в~целом к~результату обработки запроса или к~его разделу.
\end{itemize}

  Далее в~статье будет представлен процесс семантической обработки текста, 
позволяющий в~условиях высокой неопределенности состава и~структуры 
охраняемых сведений формировать МК при их 
применении для обеспечения информационной\linebreak безопас\-ности в~СИИД.
 Для решения задач выявления 
конфиденциального содержимого пред\-ла\-га\-ется использовать методы поиска 
тематически похожих документов на основе автоматически формируемых 
наборов МК, а также методов поиска семантически сходных фрагментов 
текстов. Оба предложенных подхода экспериментально проверены в~ходе 
настоящего исследования. 

\section{Формирование маркеров и~определение уровня 
конфиденциальности материала на основе анализа его текста} %2
     
  Последовательность действий по формированию маркеров и~определению 
уровня конфиденциальности материала на основе анализа его текста 
приведена в~\cite{1-bs}. Она включает следующие основные шаги: 
  \begin{enumerate}[(1)]
  \item приведение артефакта к~единому формату представления (например, 
txt);
\item определение в~артефакте или его фрагменте позиций именований 
сущностей;
\item приведение артефакта или его фрагмента к~единому именованию 
сущностей;
\item замена личных местоимений на именования сущностей;
\item проверка принадлежности и~формирование перечня сущностей, 
упоминания или конкретные сведения о которых подлежат защите; 
\item определение отнесения фрагментов текстов  
к~име\-но\-ва\-ни\-ям/упо\-ми\-на\-ни\-ям конкретных 
показателей/характеристик/видов сведений; 
\item определение уровней конфиденциальности сведений об объектах 
и~областях де\-я\-тель\-ности;
\item определение уровня конфиденциальности артефакта/его фрагмента;
\item формирование МК ар\-те\-фак\-та/его фрагмента.
\end{enumerate}

  Указанный процесс не позволяет определить МК
текста в~условиях, когда состав и~структура охраняемых сведений:
  \begin{itemize}
  \item не может быть заранее определена в~связи с~отсутствием данных;
\item их определение нецелесообразно в~связи с~высокой динамикой их 
изменения либо в~связи с~большим (или неограниченным) количеством 
сущностей, сведения о которых подлежат \mbox{охране}.
  \end{itemize}
  
  В указанных условиях для формирования маркеров конфиденциальности 
текстовых материалов должен быть использован иной подход, например 
основанный на смысловом сопоставлении конфиденциальных материалов 
с~текстами, передаваемыми в~сообщениях. Рассмотрим далее эти подходы более 
подробно.

\section{Формирование маркеров конфиденциальности текстов 
на~основе определения их~близости к~одному из~кластеров 
конфиденциальных материалов} %3
  
  Суть предлагаемого подхода состоит в~прямом (не пошаговом) 
формировании МК текста в~целом без разрешения в~явном виде 
неопределенности состава и~структуры сведений, охраняемых в~конкретной 
области (областях) деятельности.
  
  Для этого заблаговременно в~каждой рассматриваемой области 
деятельности:
  \begin{itemize}
  \item  по каждой коллекции, содержащей охра\-ня\-емые сведения одного 
уровня кон\-фи\-ден\-ци\-аль\-ности, осуществляется формирование клас\-те\-ров. 
С~каж\-дым  
клас\-те\-ром связывается \mbox{набор} ключевых слов и~словосочетаний (НКСС) и~МК, соответствующий уровню конфиденциальности;
  \item при определении МК текста осуществляется проверка его бли\-зости 
или принадлежности к~одному из клас\-те\-ров (близости к~его НКСС); текс\-ту 
присваивается МК, определенный для кластера, к~которому  
при\-над\-ле\-жит\,/\,на\-и\-бо\-лее близок этот текст; при отсутствии 
однозначности в~определении принадлежности текс\-та к~одному из клас\-те\-ров 
текс\-ту присваивается МК, отвечающий НКСС с~наивысшим уровнем 
конфиденциальности.
  \end{itemize}
   
  Подход хорошо согласуется с~условиями обработки сведений, не 
относящихся к~высшим уровням конфиденциальности (например, 
с~обработкой сведений, не составляющих государственную тайну).
  
  В работе~\cite{8-bs} описан способ обучения классификатора, 
предназначенного для определения категории документа в~задаче 
предотвращения утечек информации (\textit{англ}.\ DLP~--- Data Leak 
Prevention);\linebreak
 DLP-сис\-те\-ма должна быть способна определить 
принадлежность документа к~классу конфиденциальных документов (или 
других, в~зависимости от об\-ласти применения сис\-те\-мы). В~патенте на 
изоб\-ре\-те\-ние~\cite{9-bs} описаны система и~способ определения текста, 
содержащего конфиденциальные данные. В~основе способа лежит поиск 
ключевых слов и~вычисление их плотности. Способ предназначен для 
выявления шаблонных конфиденциальных документов.
  
  Также отметим, что DLP-системам важно уметь предотвращать утечку 
дубликатов документов. Для выявления таких документов предлагается 
использовать методы поиска текстовых заимствований~[10, 11], дополненные 
методами машинного обучения для тематической классификации текс\-тов.
   
  Основная идея применения методов ма\-шин\-ного обучения и~анализа 
текстовой информации заключается  
в~сле\-ду\-ющем. Предполагается, что организация имеет определенное 
множество конфиденциальных документов. Это множество может быть 
динамически пополняемым, при этом DLP-сис\-те\-ма дополнительно 
индексирует вновь регистрируемые конфиденциальные документы.
  
  На этапе фильтрации система сопоставляет с~помощью эффективных 
методов анализа текстов те материалы, которые передаются сотрудниками 
организации по каналам связи, конт\-ро\-ли\-ру\-емых DLP-сис\-те\-мой. При 
выявлении конфиденциальной по тематике и~содержанию информации 
в~исходящем сообщении регистрируется соответствующее нарушение политики 
конфиденциальности и~создается оповещение. Само сообщение в~таком 
случае может быть заблокировано (не передано). 
  
  Для \textit{определения близости или принадлежности текста} к~одному 
или  нескольким клас\-те\-рам целесообразно использовать подход машинного 
обучения, описанный в~разд.~4. В~разд.~5 будут приведены результаты 
экспериментов по применению методов выявления текстовых заимствований 
(в~том чис\-ле перефразированных) для предотвращения утечек 
конфиденциальных документов. В~модельных экспериментах будет 
предполагаться, что конфиденциальная информация передается фрагментами 
(не обязательно в~форме целых документов без модификаций).

\begin{table*}\small %tabl1
\begin{center}
     \Caption{Результаты классификации}
     \vspace*{3pt}
     
     \begin{tabular}{|l|p{100mm}|c|c|}
     \hline
\multicolumn{1}{|c|}{Метод} &\multicolumn{1}{c|}{Наиболее важные признаки 
(в порядке убывания)}&\textbf{NB}&\textbf{LR}\\
\hline
\multicolumn{1}{|l|}{\raisebox{-12pt}[0pt][0pt]{\textbf{Count Vectors}}}&\textit{исследования, г, фиг, м, н, л, работы, развития, диссертации, 
е, т, р, деятельности, п, изобретения, д, наук, устройство, анализ, способ, проблемы, 
научно, конференции, основные, результаты}&0,861& 0,975\\
\hline
\multicolumn{1}{|l|}{\raisebox{-18pt}[0pt][0pt]{\textbf{WordLevel  
TF-IDF}}}&\textit{фиг, новости, news, компании, компания, адрес, исследования, диссертации, 
изобретения, способ, устройство, изобретение, р, н, м, л, е, г, отличающийся, словам, масс, 
т, рис, v, диссертационного, наук, ит, году, корпуса, д}&0,908&\textbf{0,976}\\
\hline
\multicolumn{1}{|l|}{\raisebox{-18pt}[0pt][0pt]{\textbf{N-Gram Vectors}}}&\textit{адрес новости, адрес новости news, новости news, 
изобретение относится, п л, п отличающийся, читайте также, меньшей мере, ключевые 
слова, технический результат, диссертационного совета, изобретения 
является}&0,926&0,963\\
\hline
\multicolumn{1}{|l|}{\raisebox{-6pt}[0pt][0pt]{\textbf{CharLevel Vectors}}}&\textit{ti, on, s, at, мпа, es, en, in, n, ion, io, e, al, tio, re, исс, фиг, 
зоб, er, пан, y, nt, ati, st, l, p, te, иг, ed, an, s, ic, мс, al}&0,809&0,948\\
\hline
\end{tabular}
\end{center}
\end{table*}

\section{Методы обработки естественного языка 
для~формирования маркеров конфиденциальности} %4
  
  В данном разделе рассматриваются методы обработки естественного языка 
для формирования МК, применимые для выявления кон\-фи\-ден\-ци\-альной 
информации по тематике. Информация, использованная в~проведенных 
модельных экспериментах, не относится к~конфиденциальной. 
Соответственно, маркеры или признаки документов, выделяемых для 
коллекций, индицируют неконфиденциальные охраняемые сведения. Но 
отметим, что рассматриваемые методы применимы и~к~текстовым 
документам, содержащим конфиденциальную информацию.
  
  Случайным образом были выбраны по 10~тыс.\ случайных текстов для  
сле\-ду\-ющих~4~<<жанров>> (классов): патентные документы; авторефераты; 
российские журналы; российские средства массовой информации. Для проведения экспериментов 
применены известные подходы на основе сле\-ду\-ющих открытых библиотек 
языка программирования Python: NumPy, Pandas, sklearn, NLTK и~др. 
Использованы следующие подходы к~формированию обучающей матрицы:
  \begin{itemize}
\item подсчет слов (\textbf{Count Vectors}), число признаков не 
ограничено;
\item TF-IDF (максимальное число признаков~--- 5\,000) на уровнях: 
\begin{itemize}
\item Слов~--- \textbf{WordLevel TF-IDF};
\item N-грамм (от 2 до 3)~--- \textbf{N-Gram Vectors};
\item символьных N-грамм (от~2 до~3)~--- \textbf{CharLevel 
Vectors}.
  \end{itemize}
  \end{itemize}
  
  Для оценки моделей использована мера~$f_1$ с~мак\-ро\-усред\-не\-ни\-ем 
(скользящая средняя при $K\hm=5$). В~табл.~1 приведены результаты 
классификации: в~первом столбце приведен метод формирования матрицы 
признаков, во втором~--- наиболее важные признаки (выделенные с~помощью 
метода главных компонент) в~порядке убывания, в~третьем и~четвертом 
столбцах~--- оценки для следующих обучаемых моделей соответственно: 
\mbox{Na$\ddot{\mbox{\ptb{\!\!\i}}}$ve} Bayes (\textbf{NB}) и~логистическая регрессия (\textbf{LR}).
  

  
  Таким образом, для первичного выявления попыток нарушения 
конфиденциальности применимы методы текстовой классификации, которые 
обеспечивают высокую полноту и~точность на \mbox{уровне} выше~97\% (такой 
результат показал подход \textbf{WordLevel TF-IDF} с~обуча\-емой моделью 
\textbf{LR}).
  
  Завершая данный раздел, отметим следующее:
  \begin{itemize}
\item методы текстовой классификации успешно решают задачу 
\textit{определения бли\-зости или принадлежности текста};
\item при помощи анализа главных компонент можно получить неявные 
маркеры (слова, словосочетания, последовательность символов и~др.\ 
в~зависимости от метода), которые, в~том чис\-ле, могут быть 
\textit{маркерами конфиденциальных данных}.
\end{itemize}

\section{Смысловой поиск заимствований для~предотвращения 
утечки конфиденциальной информации} %5

\vspace*{-6pt}
     
  Рассмотренный в~данном разделе эксперимент на практике соответствует 
ситуации, когда сущность 
 ка\-ко\-го-ли\-бо изобретения передается сотрудником организации не 
непосредственным копированием материалов (выявление этого случая не 
представляет сложности), а~излагается в~свободной форме путем 
переписывания или изложения собственными словами.
  
  В качестве набора, содержащего <<конфиденциальную>> информацию, 
используется коллекция <<Российские журналы>> (далее~--- тестируемая 
коллекция) объемом 1,85~млн документов; в~качестве <<исходного>> набора 
документов (содержимое которых раскрывается в~неизменном или 
перефразированном виде)~--- коллекция патентных документов Федерального института промышленной собственности 
(далее~--- исходная коллекция) объемом около 800~тыс.\ документов. 
  
  Анализ выполнен в~режиме сплошной фильтрации тестируемой коллекции, 
т.\,е.\ для каждого документа из нее проводился поиск по предварительно 
проиндексированной исходной коллекции. Далее описывается метод 
измерения сходства предложений, который лежит в~основе алгоритма 
смыслового сопоставления текстов. Предложения~$s_e$ и~$s_t$ из 
документов~$d_e$ и~$d_t$ соответственно представляются в~виде множества 
пар слов $N(s_e,s_t)$, где первый элемент взят из предложения~$s_e$, а 
второй~--- из~$s_t$. Сопоставление двух предложений основано на парах слов из 
$N(s_e,s_t)$. Оценка сходства вычисляется с~учетом сле\-ду\-ющих 
показателей: $I_1$ (IDF-ме\-ра), $I_2$ (показатель на основе TF-IDF) и~$I_3$ 
(сопоставление синтаксических структур предложений): 
  \begin{itemize}
  \item[$\bullet$]
  $I_1\left(s_e,s_t\right)= \displaystyle \sum\limits_{(w_e,w_t)\in N(s_e,s_t)} 
\mathrm{IDF}\left( w_e\right).$
  \begin{itemize}
  \item[$\circ$]
Здесь $\mathrm{IDF}\,(w_e)\hm= \log_{\vert D\vert} \left( \vert 
D\vert/m(w_e,D)\right)$, где $D$~---  набор документов; 
$m(w_e,D)$~--- количество документов, содержащих 
слово~$w_e$; 
\end{itemize}
\item[$\bullet$]
$I_2\left( s_e,s_t\right)=\displaystyle\hspace*{-6mm} \sum\limits_{(w_e,w_t)\in N(s_e,s_t)}\hspace*{-8mm} 
f\left( w_e,w_t\right)\mathrm{IDF}(w_e)\mathrm{TF}\left(w_t,d_t\right).$
  \begin{itemize}
  \item[$\circ$]
Здесь $\mathrm{TF}(w_t,d_t)=\log_{\vert d_t\vert}\left( k(w_t,d_t)\right)$, где $\vert 
d_t\vert$~--- число слов в~документе~$d_t$; $k(w_t,d_t)$~---  
число встречаемых слов~$w_t$ в~$d_t$; $f(w_e, w_t)$ служит 
штрафом за несоответствие~$w_e$ и~$w_t$;
\end{itemize}
\item[$\bullet$]
$I_3\left(s_e, s_t\right)=$\\
$= \displaystyle
\hspace*{-16pt}{\sum\limits_{(w_h,\sigma,w_d)\in 
(\mathrm{Syn}(s_e)\cap \mathrm{Syn}(s_t))} \hspace*{-17pt}\mathrm{IDF}(w_h)}\Big /
\displaystyle\hspace*{-16pt}{\sum\limits_{(w_h,\sigma,w_d)\in 
\mathrm{Syn}(s_e)} \hspace*{-17pt}\mathrm{IDF}(w_h)}\,.$
  \begin{itemize}
  \item[$\circ$]
Каждое предложение представляется в~виде набора $\mathrm{Syn}(s_e)$ 
триплетов $(w_h,\sigma, w_d)$, где~$w_h$ и~$w_d$~--- 
нормализованные главное и~зависимое слова соответственно; 
$\sigma$~--- тип синтаксического отношения (например, именная 
группа).
  \end{itemize}
  \end{itemize}
  
  Общее сходство предложений определяется как линейная комбинация 
вышеприведенных мер: 
  \begin{multline*}
  \mathrm{Sim}\left(s_e, s_t\right) =\mathrm{WIdf}\cdot
  I_1\left( s_e,s_t\right)+{}\\
  {}+\mathrm{WTfIdf}\cdot I_2\left( 
s_e,s_t\right)+\mathrm{WSynt}\cdot I_3\left( s_e, s_t\right)\,,
  \end{multline*}
    где $\mathrm{WIdf}$, $\mathrm{WTfIdf}$  и~$\mathrm{WSynt}$~--- относительный вклад 
каждой меры. Полное описание метода индексации и~смыслового 
сопоставления текстов приведено в~работах~[10, 11]. Там же приведены 
метрики качества реализованного метода.
  
  В табл.~2 представлен результат в~виде кумулятивной статистики поиска 
заимствований. Во втором столбце отражено распределение количе-\linebreak\vspace*{-12pt}

\vspace*{6pt}

\begin{center}
\noindent
\parbox{71mm}{{{\tablename~2}\ \ \small{
Распределение документов по уровню заимствований 
}}}

\vspace*{6pt}


{\small \begin{tabular}{|c|r|r|}
\hline
\multicolumn{1}{|c|}{Заимствования}&  \textbf{в} документе &  \textbf{из} документа\\
\hline
$>0{,}1$ & 17\,279\hspace*{5mm}& 18\,990\hspace*{5mm}\\
$>0{,}2$& 4\,385\hspace*{5mm}& 5\,863\hspace*{5mm}\\
$>0{,}3$& 2\,066\hspace*{5mm}& 2\,647\hspace*{5mm}\\
$>0{,}4$& 1\,109\hspace*{5mm}& 1\,011\hspace*{5mm}\\
$>0{,}5$& 612\hspace*{5mm} & 512\hspace*{5mm}\\
$>0{,}6$& 301\hspace*{5mm}& 279\hspace*{5mm}\\
$>0{,}7$& 122\hspace*{5mm}& 147\hspace*{5mm}\\
$>0{,}8$& 54\hspace*{5mm}& 99\hspace*{5mm}\\
$>0{,}9$& 27\hspace*{5mm}& 81\hspace*{5mm}\\
\hline
\end{tabular}
}
\vspace*{3pt}
\end{center}
%\end{table*}

%\vspace*{9pt}
  
  
  \noindent
  ства 
документов по общему уровню заимствований \textbf{в}~документе (т.\,е.\ по 
общей доле заимствованного текста), а~в~третьем столбце~--- по уровню 
заимствований \textbf{из} документа (т.\,е.\ по доле текста, которая 
заимствована \textbf{из} одного документа).
   

  
  Анализ данных, приведенных в~табл.~2, показывает, что:
  \begin{itemize}
  \item 27 документов имеют общий уровень заимствований более 90\%; 
свыше 600~документов~--- более 50\%. 
\item 81 раз встречаются документы, \textbf{из} которых заимствовано 
более~90\%; свыше 500~раз~--- более 50\%.
\end{itemize} 

%\begin{table*}
%\small %tabl2



  Отметим, что задача анализа характера заимствований с~точки зрения 
академической этики или с~юридической точки зрения не ставилась. Отметим 
лишь, что у большинства пар документов (заимствующий документ  
и~до\-ку\-мент-ис\-точ\-ник) с~большой долей заимствования наблюдаются 
пересечения авторов. Это говорит о~том, что один и~тот же текст одного 
автора (или научной группы) использовался как для патентного документа, 
так и~для публикации статьи, возможно с~некоторыми изменениями. 

\section{Заключение}
     
  В статье рассмотрен возможный подход к~формированию 
МК при их использовании в~СИИД в~условиях, когда состав 
и~структура охраняемых сведений не может быть заранее определена в~связи 
с~отсутствием данных или высокой динамикой их изменения, либо их 
определение нецелесообразно в~связи большим (или неограниченным) 
количеством сущностей, сведения о которых подлежат охране. В~основе 
подхода к~формированию МК лежит:
  \begin{itemize}
  \item  предварительное формирование кластеров, к~каждому из которых 
относятся документы, содержащие охраняемые сведения, и~НКСС этих 
кластеров;
  \item определение близости или принадлежности конкретного текста 
  к~одному из таких кластеров;
  \item  реализация на регулярной основе процедуры актуализации состава и~структуры кластеров и~их НКСС.
  \end{itemize}
  
  Предлагаемый подход не зависит от конфиденциальности обрабатываемых 
материалов и~позволяет осуществить формирование МК в~указанных 
условиях. Инструменты, использованные для получения результатов, 
описанных в~разд.~4 и~5, применимы, например, для решения следующих 
задач:
  \begin{itemize} 
  \item классификация документов для определения потенциальных 
документов, подлежащих дальнейшему анализу;
\item выявление важных характеристик текстовых документов;
\item поиск возможной утечки информации (при условии наличия 
заранее подготовленной коллекции конфиденциальных данных);
\item поиск похожих или дублирующих документов для использования 
в~автоматизированных сис\-те\-мах документооборота.
  \end{itemize}
  
{\small\frenchspacing
{%\baselineskip=10.8pt
%\addcontentsline{toc}{section}{References}
\begin{thebibliography}{99}

\bibitem{6-bs} %1
\Au{Будзко В.\.И., Беленков В.\,Г., Борохов~С.\,В.} Проб\-ле\-мы обеспечения 
информационной безопас\-ности при интенсивном использовании данных~// 
Информационные технологии и~математическое моделирование сис\-тем: Тр. 
Междунар. 
на\-учн.-тех\-нич. конф.~--- Одинцово: ЦИТП РАН, 2017. С.~122--124.
\bibitem{7-bs} %2
\Au{Беленков В.\,Г., Борохов С.\,В., Будзко~В.\,И., Кейер~П.\,А., Королев~В.\,И.} Вопросы 
обеспечения информационной безопас\-ности информационных систем, реализующих 
интенсивное использование данных~//\linebreak Аналитика и~управ\-ле\-ние данными в~областях 
с~интенсивным использованием данных: Сб. научных тр. XIX Междунар. конф. 
DAMDID/RCDL.~--- М: ФИЦ ИУ РАН, 2017. С.~155--158.
  

\bibitem{3-bs}
\Au{Будзко В.\,И., Беленков В.\,Г., Королев~В.\,И.} Элементы конфиденциальности и~перспективы их использования  
в~сис\-те\-мах, реализующих интенсивное использование данных~// Сис\-те\-мы высокой 
до\-ступ\-ности, 2018. Т.~14. №\,4. С.~55--60.
\bibitem{4-bs}
\Au{Будзко В.\,И., Беленков В.\,Г., Королев~В.\,И}. Об одном концептуальном подходе к~защите информации 
в~сис\-те\-мах, реализующих DID~// Информационные технологии и~
математическое моделирование сис\-тем: Тр. Междунар. на\-учн.-тех\-нич. конф.~--- 
Одинцово: ЦИТП РАН, 2018. С.~43--46.
\bibitem{5-bs}
\Au{Будзко В.\,И., Беленков В.\,Г., Королев~В.\,И.} Об особенностях использования средств и~методов ОИБ в~сис\-те\-мах, реализующих DID~// Информационные технологии 
и~математическое моделирование сис\-тем: Тр. Междунар. на\-учн.-тех\-нич. конф.~--- 
Одинцово: ЦИТП РАН, 2018. С.~47--50.

\bibitem{1-bs} %6
\Au{Будзко В.\,И., Королев~В.\,И., Беленков~В.\,Г.} Особенности использования 
маркеров конфиден\-ци\-аль\-ности в~системах интенсивного использования данных~// 
Сис\-те\-мы высокой до\-ступ\-ности, 2019. Т.~15. №\,2. С.~57--65.
\bibitem{2-bs} %7
\Au{Будзко В.\,И., Королев В.\,И., Беленков~В.\,Г.} Архитектура инструментального 
комплекса извлечения информации с~учетом встроенных экстрактов конфиденциальности и~интеграции извлеченных данных~// Сис\-те\-мы высокой до\-ступ\-ности, 2020. Т.~16. 
№\,2. С.~5--21.

\bibitem{8-bs}
\Au{Дорогой Д.\,С., Шаров А.\,В., Тузовский~А.\,А., Терещенко~И.\,А.} Способ обучения 
классификатора, предназначенного для определения категории документа. Патент на 
изобретение №\,2672395 с~приоритетом от 29.09.2017. Опубл. 14.11.2018, бюл. №\,32.
\bibitem{9-bs}
\Au{Дорогой Д.\,С.} Система и~способ определения текста, содержащего 
конфиденциальные данные. Патент на изобретение №\,2665915 с~приоритетом от 
16.06.2017. Опубл. 04.09.2018, бюл. №\,25.
\bibitem{11-bs} %10
\Au{Zubarev D.\,V., Sochenkov I.\,V.} Using sentence similarity measure for plagiarism source 
retrieval~// CEUR Workshop Proceedings: Cross Language Evaluation Forum, 2014. Vol.~1180.
P.~1027--1034.
\bibitem{10-bs} %11
\Au{Zubarev D.\,V., Sochenkov I.\,V., Tikhomirov~I.\,A., Grigoriev~O.\,G.} Double funding of 
scientific projects: Similarity and plagiarism detection~// Аналитика и~управ\-ле\-ние данными 
в~областях с~интенсивным использованием данных: Сб. научных тр. XIX Междунар. конф. 
\mbox{DAMDID}/RCDL.~--- М.: ФИЦ ИУ РАН, 2017. С.~282--285.

\end{thebibliography}

 }
 }

\end{multicols}

\vspace*{-6pt}

\hfill{\small\textit{Поступила в~редакцию 23.06.20}}

\vspace*{7pt}

%\pagebreak

%\newpage

%\vspace*{-28pt}

\hrule

\vspace*{2pt}

\hrule

\vspace*{8pt}

\def\tit{EXTRACTION OF CONFIDENTIALITY MARKERS\\ FROM TEXTS 
UNDER CONDITIONS OF HIGH UNCERTAINTY\\ IN SYSTEMS WITH DATA 
INTENSIVE USAGE}


\def\titkol{Extraction of confidentiality markers from texts under conditions of 
high uncertainty in systems with data intensive usage}


\def\aut{V.\,I.~Budzko$^1$, V.\,V.~Yadrintsev$^{1,2}$, I.\,V.~Sochenkov$^1$, 
V.\,I.~Korolev$^1$, and~V.\,G.~Belenkov$^1$}

\def\autkol{V.\,I.~Budzko, V.\,V.~Yadrintsev, I.\,V.~Sochenkov, et al.}
%V.\,I.~Korolev$^1$, and~V.\,G.~Belenkov}

\titel{\tit}{\aut}{\autkol}{\titkol}

\vspace*{-15pt}

\noindent
$^1$Federal Research Center ``Computer Science and Control'' of the Russian Academy of Sciences, 
44-2~Vavilov\linebreak
$\hphantom{^1}$Str., Moscow 119333, Russian Federation

\noindent
$^2$Peoples' Friendship University of Russia (RUDN University), 6~Miklukho-Maklaya Str., Moscow 117198,\linebreak
$\hphantom{^1}$Russian Federation

\def\leftfootline{\small{\textbf{\thepage}
\hfill INFORMATIKA I EE PRIMENENIYA~--- INFORMATICS AND
APPLICATIONS\ \ \ 2020\ \ \ volume~14\ \ \ issue\ 4}
}%
 \def\rightfootline{\small{INFORMATIKA I EE PRIMENENIYA~---
INFORMATICS AND APPLICATIONS\ \ \ 2020\ \ \ volume~14\ \ \ issue\ 4
\hfill \textbf{\thepage}}}

\vspace*{3pt} 

\Abste{The main tasks, the results of the solution of which are reflected in the article, 
are associated with the formation of confidentiality markers when they
are used in 
data-intensive systems under conditions when the
composition and structure of the 
protected information cannot be determined in advance due to the lack of data\linebreak\vspace*{-12pt}}

\Abstend{ or the 
high dynamics of their change, or their definition is not advisable due to the large 
number of entities whose
information is subject to protection. In this paper, an 
approach is proposed for the formation of confidentiality markers for text materials in 
the indicated conditions. The article presents the semantic text analysis, which forms 
confidentiality markers when used to ensure information security in data-intensive 
systems under high uncertainty in the composition and structure of protected 
information. The obtained experimental results show that practical implementation of 
the considered approach in data-intensive systems is promising.}
     
     \KWE{confidentiality marker; information security; data-intensive domains; topical 
cluster; semantics; data leak prevention; intelligent security tasks; text classification; 
detection of text reuse}
     
\DOI{10.14357/19922264200410} 

\vspace*{-12pt}

\Ack
\noindent
The research was supported in part by the Russian Foundation for Basic Research in the framework of 
scientific project No.\,18-29-03215, and experimental research was supported by the ``RUDN University 
Program 5-100.''


%\vspace*{6pt}

  \begin{multicols}{2}

\renewcommand{\bibname}{\protect\rmfamily References}
%\renewcommand{\bibname}{\large\protect\rm References}

{\small\frenchspacing
 {%\baselineskip=10.8pt
 \addcontentsline{toc}{section}{References}
 \begin{thebibliography}{99}
 
 \bibitem{6-bs-1} %1
\Aue{Budzko, V.\,I., V.\,G.~Belenkov, and S.\,V.~Borokhov.} 2017. Problemy obespecheniya 
informatsionnoy bezopas\-nosti pri intensivnom ispol'zovanii dannykh [Problems of ensuring information 
security with intensive-data use]. \textit{Scientific and Technical Conference 
(International) ``Information Technologies and Mathematical Modeling of Systems'' Proceedings}. 
Odintsovo. 122--124.
\bibitem{7-bs-1} %2
\Aue{Belenkov, V.\,G., S.\,V. Borokhov, V.\,I.~Budzko, P.\,A.~Keyer, and V.\,I.~Korolev.} 2017. 
Voprosy obespecheniya informatsionnoy bezopasnosti informatsionnykh sistem, re\-a\-li\-zu\-yushchikh 
intensivnoe ispol'zovanie dannykh [Issues of ensuring information security of information systems that 
implement intensive data use]. \textit{19th Conference (International) ``Data 
Analytics and Management in Data Intensive Domains'' Proceedings}. Moscow: FRC CSC RAS. 155--158.


\bibitem{3-bs-1}
\Aue{Budzko, V.\,I., V.\,I.~Korolev, and V.\,G.~Belenkov.} 2018. Elementy konfidentsial'nosti 
i~perspektivy ikh ispol'zovaniya v~sistemakh, realizuyushchikh intensivnoe ispol'zovanie dannykh 
[Privacy elements and prospects of their use in data intensive
systems]. 
\textit{Highly Available Systems} 14(4):55--60.
\bibitem{4-bs-1}
\Aue{Budzko, V.\,I., V.\,G.~Belenkov, and V.\,I.~Korolev.} 2018. Ob odnom kontseptual'nom 
podkhode k~zashchite in\-for\-ma\-tsii v~sistemakh, realizuyushchikh DID [About one conceptual approach to 
information security in DID-systems]. \textit{Scientific and Technical Conference 
(International) ``Information Technologies and Mathematical Modeling of Systems'' Proceedings}. 
Odintsovo. 43--46. 
\bibitem{5-bs-1}
\Aue{Budzko, V.\,I., V.\,G.~Belenkov, and V.\,I.~Korolev.} 2018. Ob osobennostyakh ispol'zovaniya 
sredstv i~metodov OIB v~sistemakh, realizuyushchikh DID [On the features of the use of tools and 
methods of information security in DID-systems]. \textit{Scientific and Technical Conference (International) ``Information Technologies and Mathematical 
Modeling of Systems'' Proceedings}. Odintsovo. 47--50.
\bibitem{1-bs-1} %6
\Aue{Budzko, V.\,I., V.\,I. Korolev, and V.\,G.~Belenkov.} 2019. Osobennosti ispol'zovaniya markerov 
konfidentsial'nosti v~sistemakh intensivnogo ispol'zovaniya dannykh [Features of use privacy 
tokens in systems that implement intensive use of data]. \textit{Highly Available Systems} 
15(2):57--65.
\bibitem{2-bs-1} %7
\Aue{Budzko, V.\,I., V.\,I.~Korolev, and V.\,G.~Belenkov.} 2020. Arkhitektura instrumental'nogo 
kompleksa izvlecheniya informatsii s~uchetom vstroennykh ekstraktov konfidentsial'nosti i~integratsii 
izvlechennykh dannykh [Architecture of the information extraction tool complex with built-in privacy 
elements and integration of extracted data]. \textit{Highly Available Systems}  
16(2):5--21.

\bibitem{8-bs-1}
\Aue{Dorogoy, D.\,S., A.\,V. Sharov, A.\,A.~Tuzovskiy, and I.\,A.~Tereshchenko.} 2018. Sposob 
obucheniya klassifikatora, prednaznachennogo dlya opredeleniya kategorii dokumenta [A~method of 
training a~classifier designed to determine the category of a document]. Patent RF No.\,2672395.
\bibitem{9-bs-1}
\Aue{Dorogoy, D.\,S.} 2017. Sistema i~sposob opredeleniya teks\-ta, soderzhashchego konfidentsial'nye 
dannye [System and method for determining text containing confidential data]. Patent RF No.\,2665915.
\bibitem{11-bs-1} %10
\Aue{Zubarev, D.\,V., and I.\,V. Sochenkov.} 2014. Using sentence similarity measure for plagiarism 
source retrieval. \textit{CEUR Workshop Proceedings: Cross Language Evaluation Forum}. 
1180:1027--1034. 
\bibitem{10-bs-1} %11
\Aue{Zubarev, D.\,V., I.\,V. Sochenkov, I.\,A.~Tikhomirov, and O.\,G.~Grigoriev.} 2017. Double 
funding of scientific projects: Similarity and plagiarism detection. \textit{19th  
Conference (International) ``Data Analytics and Management in Data Intensive Domains'' Proceedings}. 
Moscow: FRC CSC RAS. 282--285.

\end{thebibliography}

 }
 }

\end{multicols}

\vspace*{-6pt}

\hfill{\small\textit{Received June 23, 2020}}

%\pagebreak

\vspace*{-16pt}


\Contr

\vspace*{-3pt}

\noindent
\textbf{Budzko Vladimir I.} (b.\ 1944)~--- Doctor of Science in technology, Academician of the 
Academy of Cryptography of the Russian Federation, principal scientist, Federal Research Center 
``Computer Science and Control'' of the Russian Academy of Sciences, 44-2~Vavilov Str., Moscow 
119333, Russian Federation; \mbox{vbudzko@ipiran.ru}
     
    \pagebreak
     
     \noindent
     \textbf{Yadrintsev Vasiliy V.} (b.\ 1993)~--- PhD student, Peoples' Friendship University of Russia 
(RUDN University), 6~Miklukho-Maklaya Str., Moscow 117198, Russian Federation; 
engineer-researcher, Federal Research Center ``Computer Science and Control'' of the Russian Academy of 
Sciences, 44-2~Vavilov Str., Moscow 119333, Russian Federation; \mbox{vvyadrincev@gmail.com}
     
     \vspace*{3pt}
     
     \noindent
\textbf{Sochenkov Ilya V.} (b.\ 1985)~--- Candidate of Science (PhD) in physics and mathematics, Head 
of Department, Federal Research Center ``Computer Science and Control'' of the Russian Academy of 
Sciences, 44-2~Vavilov Str., Moscow 119333, Russian Federation; \mbox{sochenkov@isa.ru}
     
     \vspace*{3pt}
     
     \noindent
     \textbf{Korolev Vadim I.} (b.\ 1943)~--- Doctor of Science in technology, professor, leading 
scientist, Federal Research Center ``Computer Science and Control'' of the Russian Academy of 
Sciences, 44-2~Vavilov Str., Moscow 119333, Russian Federation; professor, Financial University under 
the Government of the Russian Federation, 49~Leningradskiy Prosp., Moscow 125993, Russian 
Federation; \mbox{vkorolev@ipiran.ru}
     
     \vspace*{3pt}
     
     \noindent
     \textbf{Belenkov Viktor G.} (b.\ 1952)~--- Candidate of Science (PhD) in technology, leading 
scientist, Federal Research Center ``Computer Science and Control'' of the Russian Academy of 
Sciences, 44-2~Vavilov Str., Moscow 119333, Russian Federation; \mbox{vbelenkov@ipiran.ru}
\label{end\stat}

\renewcommand{\bibname}{\protect\rm Литература} 
  