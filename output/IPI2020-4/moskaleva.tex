\def\stat{moskaleva}

\def\tit{ВЛИЯНИЕ ПАРАМЕТРОВ ИЗОЛЯЦИИ НА~РАЗДЕЛЕНИЕ РЕСУРСОВ ПРИ~НАРЕЗКЕ 
СЕТИ$^*$}

\def\titkol{Влияние параметров изоляции на~разделение ресурсов при~нарезке 
сети}

\def\aut{Ф.\,А.~Москалева$^1$, Ю.\,В.~Гайдамака$^2$, В.\,С.~Шоргин$^3$}

\def\autkol{Ф.\,А.~Москалева, Ю.\,В.~Гайдамака, В.\,С.~Шоргин}

\titel{\tit}{\aut}{\autkol}{\titkol}

\index{Москалева Ф.\,А.}
\index{Гайдамака Ю.\,В.}
\index{Шоргин В.\,С.}
\index{Moskaleva F.\,A.}
\index{Gaidamaka Yu.\,V.}
\index{Shorgin V.\,S.}


{\renewcommand{\thefootnote}{\fnsymbol{footnote}} \footnotetext[1]
{Публикация выполнена при поддержке Программы стратегического академического лидерства РУДН 
и при финансовой поддержке РФФИ (проекты 19-07-00933 и 20-07-01064).}}


\renewcommand{\thefootnote}{\arabic{footnote}}
\footnotetext[1]{Российский университет дружбы народов, moskaleva-fa@rudn.ru}
\footnotetext[2]{Российский университет дружбы народов; Институт проблем информатики 
Федерального исследовательского центра <<Информатика и~управ\-ле\-ние>> Российской академии 
наук, \mbox{gaidamaka-yuv@rudn.ru}}
\footnotetext[3]{Институт проблем информатики Федерального исследовательского центра 
<<Информатика и~ управ\-ле\-ние>> Российской академии наук, \mbox{vshorgin@ipiran.ru}}
%
\vspace*{-15pt}

  
  \Abst{Технология нарезки радиоресурсов сети определяется как один из основных 
компонентов пятого поколения мобильных коммуникаций, способных решить проблему 
колоссального роста объема трафика данных в сотовых сетях. Ключевая особенность 
нарезки радиоресурсов сети, или сетевого слайсинга, позволяющая ограничить влияние 
одного слайса на другой, заключается в обеспечении изолированных гарантий 
производительности для предоставления высокого качества обслуживания (QoS, Quality of Service). 
В~статье  
с~по\-мощью аппарата тео\-рии массового обслуживания построена модель разделения 
ресурсов при нарезке сети, позволяющая исследовать разделение ресурсов в соответствии 
с различными стратегиями справедливости. Задача разделения ресурсов сформулирована 
в~форме задачи оптимизации относительно зависящей от параметра изоляции весовой 
функции ресурса системы, занятого заявками каждого слайса. Проведенный численный 
анализ показал существенное влияние параметра изоляции на изменение характеристик 
производительности сис\-темы.}
  
  \KW{нарезка сети; справедливое разделение ресурсов; изоляция слайсов; параметр 
изоляции}
\DOI{10.14357/19922264200402} 
  
\vspace*{-3pt}


\vskip 10pt plus 9pt minus 6pt

\thispagestyle{headings}

\begin{multicols}{2}

\label{st\stat}

\section{Введение}

  Нарезка радиоресурсов сети (\textit{англ}.\ network slicing)~--- ключевая 
технология, позволяющая операторам сети предоставлять свою физическую 
инфраструктуру для поддержки услуг с различными\linebreak требованиями~[1]. 
Определенный набор услуг\linebreak может быть связан с логически независимой 
сквоз\-ной сетью, т.\,е.\ слайсом. Слайс (\textit{англ}.\ slice)~--- логи-\linebreak ческая 
сеть, обеспечивающая определенные\linebreak функциональные возможности и 
сетевые харак\-те\-ри\-сти\-ки~[2]. Слайсы настраиваются и управ\-ля\-ют\-ся 
арендаторами (\textit{англ}.\ tenants), например виртуальными операторами 
(Virtual Network Operator, VNO), которым мобильный оператор, владеющий 
инфраструктурой, делегирует контроль над использованием ресурсов и 
качеством предоставления услуг внутри слайса. 

Концепция нарезки сети 
подразумевает автоматизацию создания и настройки слайса, изоляцию 
слайсов (независимость показателей качества обслуживания в слайсе от 
трафика в других слайсах, а~также безопас\-ность и~т.\,п.), эластичность 
нарезки (справедливое~\cite{4-mos} и эффективное использование ресурсов, 
адаптация к~условиям), иерархию управ\-ления (самоуправление в слайсе), 
возможность назначать приоритетные слайсы~\cite{3-mos}.
  
  Гарантии качества обслуживания в логической сети обеспечиваются 
изоляцией слайса, так что никакие изменения в других слайсах не могут 
повлиять на показатели качества обслуживания (QoS). При 
этом выбор стратегии изоляции и разделения ресурсов на 
радиоинтерфейсе~--- достаточно сложная  
задача~\cite{5-mos}. 

Из-за стохастической природы беспроводной среды и 
высокой изменчивости трафика во времени и пространстве идеальную 
изоляцию можно обеспечить лишь в случае резервирования ресурсов 
в~соответствии с наихудшими ожидаемыми условиями, что ведет к 
неэффективному использованию ресурсов в~большинстве случаев. 
Попытка 
учесть стохастическую природу беспроводной среды и ее особенности 
с~использованием аппарата цепей Маркова сделана в~\cite{6-mos}. 

Как показано в~\cite{7-mos}, устанавливая взаимосвязь параметров сети, 
можно достичь баланса между изоляцией и эф\-фек\-тив\-ностью, что позволяет 
прио\-ри\-тизировать и~настраивать каждый слайс в со\-от\-вет\-ствии с 
конкретными задачами, для которого он используется. При этом управ\-ле\-ние 
ресурсами как внутри слайса, так и~меж\-ду разными слайсами должно 
гарантировать не только изоляцию слайса, но и справедливость разделения 
ресурсов между пользователями~\cite{8-mos}. 

В~\cite{9-mos} механизмы 
нарезки сети с учетом гарантий для различных типов трафика исследованы 
при фиксированных значениях параметров изоляции.
  
  В статье построена математическая модель разделения радиоресурсов 
соты между двумя виртуальными операторами (далее~--- операторами), 
которая иллюстрирует влияние параметров изоляции на мет\-рики 
производительности сети. Положим, что на базовой станции сети связи 
пятого поколения с~технологией радиодоступа New Radio активированы два 
слайса, принадлежащие разным операторам, и модуль нарезки делит между 
ними общий ресурс ем\-костью~$C$  единиц ресурса. Примером единицы 
ресурса может быть герц для полосы радиочастот, бит в секунду для 
ско\-рости или ресурсный блок LTE/NR (long-term evolution\,/\,new radio). 
Каждый оператор осуществляет  
предостав\-ле\-ние некоторой услуги связи, пред\-по\-ла\-га\-ющей 
непрерывную передачу пользователю потокового трафика на выделенном 
ресурсе не менее $b_{\min}, d_{\min}\hm >0$ и~не более $b_{\max}, 
d_{\max}\hm>0$ единиц ресурса для первого и второго слайсов 
соответственно. Длительность предостав\-ле\-ния услуги пользователю 
(длительность пользовательской сессии) определяется объемом 
выделенного для обслуживания сессии ресурса, который зависит от числа 
активных сессий в каж\-дом слайсе. Считаем, что ресурсы каждого слайса 
делятся между его пользователями поровну. Для обеспечения 
справедливого разделения ресурсов введены параметры изоляции, 
определяющие чис\-ло пользователей в каждом слайсе, которым оператор 
обязуется предоставить услугу с~минимальным качеством. При низ\-кой 
за\-гру\-жен\-ности соты каждой сессии\linebreak выделяется ресурс, достаточный 
для получения пользователем услуги на максимальной ско\-рости. 
С~увеличением нагрузки, создаваемой запросами пользователей обоих 
операторов, вы\-де\-ля\-емый каж\-дой сессии ресурс снижается, пока не 
достигнет уровня минимальной ско\-рости, тре\-бу\-емой для 
предоставления услуги. При дальнейшем росте нагрузки начинает работать 
концепция нарезки сети, согласно которой для приема в сис\-те\-му заявки 
слайса, не достигшего заданного па\-ра\-мет\-ром изоляции предела, долж\-но 
быть прервано обслуживание одной или нескольких заявок второго 
слайса, т.\,е.\ <<нарушителя>> (\textit{англ}.\ violator), который пользовался 
простаивающим ресурсом недогруженного слайса. 

В~сле\-ду\-ющих разделах 
построена модель в виде\linebreak сис\-те\-мы массового обслуживания (СМО), 
позво\-ля\-ющая исследовать за\-ви\-си\-мость показателей качества 
обслуживания от параметров изоляции,\linebreak предложен алгоритм разделения 
радиоресурсов с~по\-мощью методов теории оптимизации, приведен 
пример  
чис\-лен\-но\-го анализа полученных результатов.
  
  \section{Математическая модель системы с~двумя слайсами} %2
  
  Пусть в многолинейную СМО поступают два пуассоновских потока 
заявок с интенсивностями~$\lambda_1$ и~$\lambda_2$ (рис.~1), 
соответствующие запросам на установление сессии от пользователей двух 
операторов. Длительности обслуживания заявок 1-го и~2-го потоков 
распределены по экспоненциальному закону с~параметрами~$\mu_1$ 
и~$\mu_2$ соответственно. Чис\-ло единиц ресурса, выделяемых заявке, 
принятой на обслуживание, зависит от общей за\-гру\-жен\-ности сис\-те\-мы 
и~варьируется в~диапазонах $[b_{\min}, b_{\max}]$ и~$[d_{\min}, d_{\max}]$ 
для 1-го и~2-го потоков со\-от\-вет\-ст\-венно.
  
  В системе предусмотрены параметры изоляции~$\overline{M}$ и 
$\overline{N}$, характеризующие максимальные значения числа заявок 1-го 
и~2-го потока, для которых обслуживание гарантировано.

  Состояние системы описывает случайный процесс 
$\boldsymbol{X}(t)\hm= \left(M(t), N(t)\right)$, где $M(t)$~--- число заявок 1-го потока; 
$N(t)$~--- число заявок 2-го потока в~момент~$t$, с пространством 
состояний
\begin{multline*}
  \mathbb{X}=\left\{ (m,n): m=0,\ldots , \left\lfloor \fr{C}{b_{\min}}\right\rfloor\,,\right.\\ 
n=0,\ldots ,\enskip
\left. \left\lfloor \fr{C}{d_{\min}}\right\rfloor \,,\ mb_{\min}+nd_{\min}\leq 
C\right\}\,.
  \end{multline*}
  
  Введем величины $b(m,n)$ и $d(m,n)$, обо\-зна\-ча\-ющие число единиц 
ресурса, выделенных для обслуживания одной заявки 1-го и 2-го потоков 
соответственно в состоянии $(m,n)\hm\in\mathbb{X}$. Заметим, что 
$b_{\min}\hm\leq b(m,n)\hm\leq b_{\max}$, $d_{\min}\hm\leq d(m,n)\hm\leq 
d_{\max}$, при этом значения~$b(m,n)$ и~$d(m,n)$ являются решением 
задачи оптимизации, которая сформулирована в разд.~3 статьи. 
  
  Интенсивности обслуживания заявок 1-го и 2-го потока определяются как 
$mb(m,n)\mu_1$ и~$nd(m,n)\mu_2$ соответственно.
  

  Параметры изоляции $\overline{M}$ и~$\overline{N}$ управляют при\-емом 
в систему поступающих заявок и~прерывани-\linebreak
\vspace*{-12pt}

{ \begin{center}  %fig1
 \vspace*{3pt}
    \mbox{%
 \epsfxsize=79mm 
 \epsfbox{mos-1.eps}
 }
\vspace*{3pt}

\noindent
{{\figurename~1}\ \ \small{Модель СМО}}
\end{center}
}

%\vspace*{6pt}


\setcounter{figure}{1}
\begin{figure*} %fig2
\vspace*{1pt}
    \begin{center}  
  \mbox{%
 \epsfxsize=161.667mm 
\epsfbox{mos-2.eps}
 }
\end{center}
\vspace*{-11pt}
\Caption{Центральное состояние диаграммы интенсивностей переходов случайного 
процесса~$\mathbb{X}(t)$}
\end{figure*}



\pagebreak

\noindent
ем обслуживания ранее 
принятых заявок следу\-ющим образом. Если при поступлении заявки 1-го 
потока в состоянии $(m,n)\hm\in \mathbb{X}$, где $m\hm<\overline{M}$, 
$n\hm>\overline{N}$,  
в~сис\-т\-еме недостаточно свободного ресурса для обслуживания $(m+1)$ 
заявок 1-го потока с выделением каждой минимального числа~$b_{\min}$ 
единиц ресурса, поступающая заявка 1-го потока вытесняет одну или 
несколько заявок 2-го потока, чтобы встать на обслуживание (рис.~2). 
Число $k(m,n)$ таких заявок, которые после прерывания обслуживания 
покинут систему, не оказывая влияния на ее дальнейшее функционирование, 
вычисляется следующим образом:
  \begin{multline*}
  k(m,n)=\left\lceil \fr{(m+1)b_{\min}+nd_{\min}-C}{d_{\min}}\right\rceil\,,\\ 
(m,n)\in\mathbb{X}\,,\enskip (m+1,n)\not\in\mathbb{X}\,,\enskip k(m,n)\leq n\,.
\end{multline*}
  
  Аналогично для случая $m\hm>\overline{M}$, $n\hm> \overline{N}$ для 
при\-ема заявки 2-го потока будет прервано обслуживание $s(m,n)$ заявок 1-го потока:
  \begin{multline*}
  s(m,n)= \left\lceil \fr{mb_{\min}+(n+1)d_{\min}-C}{b_{\min}}\right\rceil,\\ 
(m,n)\in\mathbb{X}\,,\enskip (m,n+1)\not\in\mathbb{X}\,,\enskip s(m,n)\leq m\,.
 \end{multline*}
  
  Для случая $m<\overline{M}$, $n<\overline{N}$ и для случая 
$m\hm>\overline{M}$, $n\hm>\overline{N}$ поступающая заявка теряется, 
если в системе недостаточно свободного ресурса для ее приема в систему.
  
  Таким образом, множества потери $\mathbb{B}_1^{\mathrm{arr}}$ заявок \mbox{1-го}
потока и~$\mathbb{B}_2^{\mathrm{arr}}$ заявок 2-го потока при поступлении 
(\textit{англ}.\ arrival) и~множества прерывания 
обслуживания~$\mathbb{B}_1^{\mathrm{pr}}$ заявок 1-го потока 
и~$\mathbb{B}_2^{\mathrm{pr}}$ заявок 2-го потока при вытеснении (\textit{англ}.\ 
preemption)~\cite{10-mos} имеют сле\-ду\-ющий вид:
  \begin{align*}
&  \mathbb{B}_1^{\mathrm{arr}}\!=\!\left\{\! (m,n)\in \mathbb{X}: 
(m+1,n)\not\in\mathbb{X}\,,\,m\geq \overline{M}\right\}\!;\\[3pt]
&  \mathbb{B}_2^{\mathrm{arr}}\!=\!\left\{\! (m,n)\in \mathbb{X}: 
(m, n+1)\not\in\mathbb{X}\,,\,n\geq \overline{N}\right\}\!;\\[3pt]
&  \mathbb{B}_1^{\mathrm{pr}}\!=\!\left\{\! (m,n)\in \mathbb{X}: 
(m,n+1)\not\in\mathbb{X}\,,\,m>\overline{M},\,n<\overline{N}\right\}\!;\\[3pt] \hspace*{-.3pt}
&\mathbb{B}_2^{\mathrm{pr}}\!=\!\left\{\! (m,n)\in \mathbb{X}: 
(m+1,n)\not\in\mathbb{X}\,,\,m<\overline{M},\,n>\overline{N}\right\}\!. \hspace*{-.3pt}
\end{align*}

Распределение стационарных вероятностей~$\mathbf{p}$ получаем путем 
решения системы линейных уравнений:
\begin{align*}
\mathbf{pA} &=\mathbf{0}\,;\\
\mathbf{p1} &=\mathbf{1}\,,
\end{align*}
где $\mathbf{A}$~--- инфинитезимальная матрица, элементы которой $a\left( 
(m,n),(m^\prime n^\prime)\right)$ записаны ниже:
\begin{multline*}
 a\left( (m,n), \left( m^\prime,n^\prime\right)\right) ={}\\
{}=\! 
\begin{cases}
\lambda_1\,,  & m+1,\ n^\prime=n\left( m^\prime, n^\prime\right)\in \mathbb{X}\,;\\[3pt]
\lambda_1\,, & m^\prime=m+1\,,\ n^\prime=n-k\,,\\
   & (m+1,n)\notin\mathbb{X}\,,\ 
m<\overline{M},\ n\geq \overline{N}\,;\\[3pt]
\lambda_2\,, &m^\prime=m,\ n^\prime=n+1,\
   \left(m^\prime,n^\prime\right)\in\mathbb{X}\,;\\[3pt]
\lambda_2\,, & m^\prime=m-s\,,\ n^\prime=n+1\,,\\
   & (m,n+1)\notin \mathbb{X}\,,\ 
m\geq\overline{M}\,,\ n< \overline{N}\,;\\[3pt]
mb(m,n)\mu_1\,,\hspace*{-6pt}& m^\prime=m-1\,,\ n^\prime=n\,,\ m^\prime>0\,;\\[3pt]
nd(m,n)\mu_2\,,\hspace*{-6pt}& m^\prime=m\,,\ n^\prime=n-1\,,\ n^\prime>0 \,;\\[3pt]
A\,, & m^\prime=m\,,\ n^\prime=n\,;\\[3pt]
0 &\mbox{иначе}.
\end{cases}\hspace*{-8.8pt}
\end{multline*}
Здесь диагональные элементы $a\left( (m,n), m,n)\right)\hm=A$ имеют 
следующий вид:
\begin{multline*}
A=-\lambda_1 I\left((m+1,n)\in\mathbb{X},n\geq N\right) -{}\\
{}-\lambda_2 
I\left( (m,n+1)\in\mathbb{X},m\geq M\right)-{}\\
{}- mb(m,n)\mu_1 I(m>0)-nd(m,n)\mu_2 I(n>0)\,.
%\label{e3-mos}
\end{multline*}

  Получив распределение вероятностей, можно найти некоторые 
характеризующие производительность системы метрики для заявок 1-го и 
2-го потока:
  \begin{itemize}
  \item вероятность потери заявки при поступлении и~вероятность 
прерывания обслуживания заявки при вытеснении ($s=1, 2$)
  \begin{align*}
  B_s^{\mathrm{arr}}&=\sum\limits_{(m,n)\in \mathbb{B}_s^{\mathrm{arr}}} p(m,n)\,;\\
  B_s^{\mathrm{pr}}&=\sum\limits_{(m,n)\in \mathbb{B}_s^{\mathrm{pr}}} p(m,n)\,;
  \end{align*}
\item среднее время обслуживания заявки
\begin{multline*}
\hspace*{-15pt}S_1={}\\
\hspace*{-15pt}{}={N_1} \!\left(\!\lambda_1(1-B_1^{\mathrm{arr}})- %\vphantom{\sum\limits_{(m,n)}}{}\right.\\[-9pt]
\lambda_2\!\!\! \sum\limits_{(m,n)\in \mathbb{B}_1^{\mathrm{arr}}}\!\!\! 
s(m,n)p(m,n)\!\right)^{\!-1}\!;\hspace*{-9pt}
\end{multline*}

\vspace*{-12pt}

\noindent
\begin{multline*}
\hspace*{-15pt}S_2={}\\
\hspace*{-15pt}{}={N_2} \!\left(\!\lambda_2(1-B_2^{\mathrm{arr}})- %\vphantom{\sum\limits_{(m,n)}}{}\right.\\[-9pt]
\lambda_1 \!\!\!\sum\limits_{(m,n)\in 
\mathbb{B}_2^{\mathrm{arr}}} \!\!\!k(m,n)p(m,n)\!\right)^{\!-1}\!;\hspace*{-9pt}
\end{multline*}
\item вероятность нарушения
$$
V_1=\sum\limits_{m>\overline{M}} p(m,n)\,;\quad 
V_2=\sum\limits_{n>\overline{N}} p(m,n)\,.
$$
  \end{itemize}
  
  В следующем разделе сформулирована задача оптимизации для 
вычисления значений  
величин $b(m,n)$ и~$d(m,n)$, обеспечивающая эффективное использование 
ресурса системы.
  
  \section{Решение задачи разделения ресурсов} %3
  
  Для состояний $(m,n)\in \mathbb{X}$, в которых $mb_{\max}\hm+ 
nd_{\max}\hm> C$, необходимо определить значения величин $b(m,n)$ 
и~$d(m,n)$, позволяющие не только полностью использовать ресурс 
системы, но и максимизировать некоторую функцию по\-лез\-ности. Для 
примера выберем стратегию max-min-справедливости при разделении 
ресурсов (\textit{англ.}\  
max-min fairness), функция полезности при кото-\linebreak
\vspace*{-12pt}
\columnbreak

\noindent
рой~\cite{11-mos} относится 
к~логарифмическому типу, является возрастающей, строго вогнутой и 
непрерывно дифференцируемой и определяется как $U(x)\hm= \ln x$, $\min 
\left( b_{\min}, d_{\min}\right)\hm\leq x\hm\leq \max\left( b_{\max}, 
d_{\max}\right)$. 
  
  Разделение ресурса между заявками в системе соответствует решению 
следующей задачи оптимизации:
  \begin{equation}
  \left.
  \begin{array}{c}
  w_1(m,n) mU(b(m,n))+{}\hspace*{60pt}\\[6pt]
  \hspace*{20pt}{}+w_2(m,n) nU(d(m,n))\to \max\\[6pt]
  \mbox{s.t.}\ mb(m,n)+nd(m,n)=C\,;\\[6pt]
  b_{\min} \leq b(m,n)\leq b_{\max}\,;\\[6pt]
  d_{\min}\leq d(m,n)\leq d_{\max}\,.
  \end{array}
  \right\}
  \label{e4-mos}
  \end{equation}

Весовые функции $w_1(m,n)$ и~$w_2(m,n)$ будем вычислять по формулам: 
\begin{align*}
w_1(m,n) &=\begin{cases}
1\,, & m\leq \overline{M}\,;\\
\fr{1}{m-\overline{M}+1}\,,& m>\overline{M}\,;
\end{cases}\\
w_2(m,n) &=\begin{cases}
1\,, & n\leq \overline{N}\,;\\
\fr{1}{n-\overline{N}+1}\,, & n>\overline{N}\,.
\end{cases}
\end{align*}
  
  Такой выбор весовых функций обеспечивает max-min-спра\-вед\-ли\-вое 
распределение ресурсов для пользователей до тех пор, пока их количество 
в~соответствующих слайсах не превышает зарезервированное, 
и~<<штрафует>>  
сре\-зы-на\-ру\-ши\-те\-ли уменьшением их веса. 
  
  Таким образом, стационарная точка задачи оптимизации имеет 
координаты
  \begin{equation}
  b(m,n)=\fr{w_1C}{w_1m+w_2n}\,;\enskip
  d(m,n)=\fr{w_2C}{w_1m+w_2n}
  \label{e5-mos}
  \end{equation}
и расположена на пересечении прямых $C\hm= mb\hm+ nd$ 
и~$w_1/b(m,n)\hm= w_2/d(m,n)$.
  
\begin{figure*}[b] %fig3
\vspace*{6pt}
    \begin{center}  
  \mbox{%
 \epsfxsize=162.998mm 
\epsfbox{mos-3.eps}
 }
\end{center}
\vspace*{-12pt}
\Caption{Разделение ресурсов между заявками: (\textit{а})~$X(t)\hm=(4,4)$;
(\textit{б})~$X(t)\hm=(5,1)$; (\textit{в})~$X(t)\hm= (9,5)$}
%\end{figure*}
%\begin{figure*} %fig4
\vspace*{15pt}
    \begin{center}  
  \mbox{%
 \epsfxsize=163mm 
 \epsfbox{mos-4.eps}
 }
\end{center}
\vspace*{-12pt}
\Caption{Вероятности блокировки~$B_s$~(\textit{а}) и~потери при 
поступлении~$B_s^{\mathrm{arr}}$ (залитые значки) и~прерывания обслуживания при 
вытеснении~$B_s^{\mathrm{pr}}$~(пустые значки)~(\textit{б}), $s\hm=1$ (сплошные кривые) 
и~2 (штриховые кривые)}
\end{figure*}
  
  Решение~(\ref{e5-mos}) задачи оптимизации~(\ref{e4-mos}) обеспечивает 
одновременно как изоляцию слайсов  
с~по\-мощью па\-ра\-мет\-ров~$\overline{M}$ и~$\overline{N}$, так и 
эластичность нарезки для эффективного использования ресурса 
системы~\cite{3-mos}.
  
  \section{Пример численного анализа}  %4
  
  Проиллюстрируем зависимость метрик, характеризующих 
производительность системы, от пара\-метров изоляции при нарезке сети. 
Предполагаем, что два оператора делят между собой 50~Мбит/с 
($C\hm=50$) согласно решению задачи оптимизации~(\ref{e4-mos}). 
Минимальные скорости передачи данных равны 5~Мбит/с 
($b_{\min}\hm=5$) и 1~Мбит/с ($d_{\min}\hm=1$),\linebreak
\vspace*{-12pt}
\pagebreak

\noindent
 максимальные~--- 
8~Мбит/с ($b_{\max}\hm=8$) и~50~Мбит/с ($d_{\max}\hm=50$), эти 
диапазоны показаны на рис.~3 тем\-но-се\-рым цветом. Параметры 
изоляции: $\overline{M}\hm=5$ и~$\overline{N}\hm=25$ заявок, 
интенсивности поступления: $\lambda_1\hm=1/150$ и~$\lambda_2\hm=1/120$, 
интенсивности обслуживания: $\mu_1\hm= 1/1920$ и~$\mu_2\hm=1/4000$, 
средние размеры файлов: 1,2~ГБ и~500~МБ для услуг 1-го и 2-го операторов 
соответственно. Указанные значения параметров сис\-те\-мы близ\-ки 
к~реальным и~соответствуют услуге буферизуемого потокового видео 
в~высоком разрешении для 1-го оператора, и услуге загрузки файлов, 
например
%  \begin{align*}
%&  \mathbb{B}_1^{\mathrm{arr}}=\left\{\! (m,n)\in\mathbb{X}: (m+1,n)\not=\mathbb{X}, 
%m\geq \overline{M}\right\}\!;\\
%&  \mathbb{B}_2^{\mathrm{arr}}=\left\{\! (m,n)\in\mathbb{X}: (m,n+1)\not=\mathbb{X}, 
%n\geq \overline{N}\right\}\!;\\
%&  \mathbb{B}_1^{\mathrm{pr}} = \left\{\! (m,n)\in \mathbb{X}: (m+1,n)\not\in \mathbb{X}, 
%m>\overline{M}, n<\overline{N}\right\}\!;\\
%&  \mathbb{B}_2^{\mathrm{pr}} = \left\{\! (m,n)\in \mathbb{X}: (m,n+1)\not\in \mathbb{X}, 
%m<\overline{M}, n>\overline{N}\right\}
%  \end{align*}
при обновлении программного обеспечения, для 2-го оператора.
  
  На рис.~3 штриховыми линиями показано решение~(\ref{e5-mos}) задачи 
оптимизации~(\ref{e4-mos}) для  
max-min-спра\-вед\-ли\-во\-го разделения ресурсов: на рис.~3,\,\textit{а} всем 
пользователям обоих операторов выделен одинаковый ресурс~6,2~Мбит/с; 
на рис.~3,\,\textit{б} все 5~пользователей 1-го оператора получили по 
$b_{\max}\hm=8$~Мбит/с, а~единственный пользователь 2-го оператора~--- 
оставшиеся~10~Мбит/с; на рис.~3,\,\textit{в} все пользователи обоих 
операторов получили минимальный требуемый ресурс.


\begin{figure*} %fig5
\vspace*{1pt}
    \begin{center}  
  \mbox{%
 \epsfxsize=160.393mm 
 \epsfbox{mos-5.eps}
 }
\end{center}
\vspace*{-11pt}
\Caption{Вероятность нарушения $V_s$~(\textit{а}) и среднее время 
обслуживания~$S_s$~(\textit{б}), $s\hm=1$ (сплошные кривые) и~2 (штриховые кривые)}
\end{figure*}
  
  На рис.~4 показаны графики вероятностей блокировки, потери при 
поступлении и~прерывания обслуживания заявки при вытеснении для 
обоих операторов. Вероятность блокировки заявки вы\-чис\-ля\-ет\-ся по 
формуле: $B_s\hm= B_s^{\mathrm{arr}}\hm+ B_s^{\mathrm{pr}}$, $s\hm=1,2$.
%o
 Сравнение рис.~4,\,\textit{а} и рис.~4,\,\textit{б} показывает, что потери 
при поступлении вносят основной вклад в~блокировку заявки, которая 
соответствует отказу\linebreak пользователю в получении услуги. При этом с~рос\-том 
параметра изоляции первого слайса, т.\,е.\ гарантированного 
числа~$\overline{M}$ принимаемых заявок \mbox{1-го} потока, вероятность 
прерывания обслуживания заявки 1-го потока~$B_1^{\mathrm{pr}}$ падает, 
а~вероятность потери заявки 2-го потока при поступлении~$B_2^{\mathrm{arr}}$ растет 
попарно симметрично. Переломным значением оказывается 
$\overline{M}\hm=5$, при котором ресурса системы становится 
недостаточно, чтобы удовлетворять одновременно гарантиям 1-го и 2-го 
слайса, т.\,е.\ $\overline{M} b_{\min} \hm+ \overline{N} d_{\min}\hm>C$.
  
  На рис.~5,\,\textit{а} показано, что существенное влияние параметр 
изоляции оказывает на вероятность~$V_s$ пребывания слайса в состоянии 
нарушителя ($m\hm> \overline{M}$ для первого слайса, $n\hm> \overline{N}$ 
для второго). С~ростом параметра изоляции~$\overline{M}$ 
вероятность~$V_1$ для 1-го слайса стремится к нулю, а вероятность~$V_2$ 
для 2-го слайса меняется незначительно, что свидетельствует об 
обеспечении изоляции. С~увеличением гарантии для 1-го слайса до 
значения $\overline{M}\hm=5$ среднее время~$S_1$ обслуживания заявок 
1-го потока (рис.~5,\,\textit{б}) падает, после чего стабилизируется, поскольку 
при переходе через значение $\overline{M}\hm=5$ параметр изоляции 
перестает оказывать влияние на вероятность приема заявок, 
а~следовательно, и на среднее время обслуживания. Противоположный 
характер поведения кривой наблюдается для второго слайса. 
  
  Таким образом, можно сделать вывод, что изменение параметра 
изоляции оказывает существенное влияние на характеристики 
про\-из\-во\-ди\-тель\-ности сис\-те\-мы, при этом построенная модель способна 
обеспечить изоляцию слайсов.

\vspace*{-10pt}
  
  \section{Заключение}
  
  Построенная в работе модель разделения ресурсов при нарезке 
радиоресурса сети с обеспечением изоляции слайсов и разработанный 
алгоритм разделения ресурсов, включающий решение задачи оптимизации, 
позволяют справедливо разделить ресурсы сети для их эффективного 
использования\linebreak двумя виртуальными сетевыми операторами, арендующими 
радиоресурс у~мобильного оператора.\linebreak Заметим, что замена  
max-min-стра\-те\-гии справедливости на стратегию пропорциональной 
справедливости или более общий случай  
$\alpha$-спра\-вед\-ли\-вости повлияет лишь на вид целевой функции 
в~задаче оптимизации. Модель может быть применена при поиске 
комбинации па\-ра\-мет\-ров изоляции, удерживающих сис\-те\-му в~целевой  
об\-ласти значений мет\-рик ее производительности. Целью дальнейших 
исследований модели разделения ресурсов при нарезке сети может стать 
развитие построенной модели на случай отсутствия минимальных 
требований к~ресурсам, что позволит исследовать элас\-тич\-ный трафик, 
а~также расширение модели до произвольного чис\-ла виртуальных сетевых 
операторов.
{\looseness=1

}
  
  \bigskip
  
  Авторы благодарят Н.\,В.~Яркину и Е.\,Ю.~Лисовскую за полезные 
обсуждения в ходе работы над статьей.
  
{\small\frenchspacing
 {%\baselineskip=10.8pt
 %\addcontentsline{toc}{section}{References}
 \begin{thebibliography}{99}
\vspace*{3pt}

\bibitem{1-mos}
NGMN 5G White Paper. 2015. {\sf http://www.ngmn.de/\linebreak 5gwhite-paper.html}.

\bibitem{2-mos}
5G: System Architecture for the 5G System (Release~15). Version 15.2.0. ETSI 3GPP TS 23.501, 2018. 
{\sf https://\linebreak
www.etsi.org/deliver/etsi\_ts/123500\_123599/123501/ 15.02.00\_60/ts\_123501v150200p.pdf}. 

\bibitem{4-mos} %3
\Au{Jain R., Chiu D.-M., Hawe~W.\,R.} A quantitative measure of fairness and discrimination 
for resource allocation in shared computer system~// arXiv.org, 1998. 
\mbox{arXiv}: cs/9809099 [cs.NI].

\bibitem{3-mos} %4
Network slicing~--- use case requirements.~--- \mbox{GSMA},\linebreak 2018. {\sf  
https://www.gsma.com/futurenetworks/wp-content/uploads/2018/07/Network-Slicing-Use-Case-Requirements-fixed.pdf}.

\bibitem{5-mos}
\Au{Richart M., Baliosian~J., Serrat~J., Gorricho~J.-L.} Resource slicing in virtual wireless 
networks: A~survey~// IEEE~T. Netw. Serv. Man., 2016. Vol.~13. Iss.~3.  
P.~462--476.
\bibitem{6-mos}
\Au{Vila I., P$\acute{\mbox{e}}$rez-Romero J., Sallent~O., Umbert~A.} Characterisation 
of radio access network slicing scenarios with 5G\linebreak
\vspace*{-12pt}
\pagebreak

\noindent
QoS provisioning~// IEEE Access, 2020. 
Vol.~8. P.~51414--51430. doi: 10.1109/access.2020.2980685.
\bibitem{7-mos}
\Au{Marabissi D., Fantacci~R.} Highly flexible RAN slicing approach to manage isolation, 
priority, efficiency~// IEEE Access, 2019. Vol.~7. P.~97130--97142. doi: 
10.1109/ \mbox{access}.2019.2929732.

\bibitem{8-mos}
\Au{Lieto A., Malanchini~I., Capone~A.} Enabling dynamic resource sharing for slice 
customization in 5G networks~// Conference and Exhibition on Global Telecommunications 
Proceedings.~--- IEEE, 2018. P. 1--7. doi: 
10.1109/\mbox{GLOCOM}.2018.8647249. 
\bibitem{9-mos}
\Au{Агеев К.\,А., Сопин Э.\,С., Яркина~Н.\,В., Самуйлов~К.\,Е., Шоргин~С.\,Я.} 
Анализ механизмов нарезки сети с учетом гарантий для различных типов 
трафика~// Информатика и~её применения, 2020. Т.~14. Вып.~3. С.~94--100.
\bibitem{10-mos}
\Au{Yarkina N., Gaidamaka Y., Correia~L.\,M., Samouylov~K.} An analytical model for 
5G network resource sharing with flexible SLA-oriented slice isolation~// Mathematics, 2020. 
Vol.~8. Iss.~7. Art. No.\,1177.
\bibitem{11-mos}
\Au{Kelly F.} Charging and rate control for elastic traffic~// Eur.~T. Telecommun., 1997. 
Vol.~8. P.~33--37.
\end{thebibliography}

 }
 }

\end{multicols}

\vspace*{-5pt}

\hfill{\small\textit{Поступила в~редакцию 11.10.20}}

\vspace*{8pt}

%\pagebreak

%\newpage

%\vspace*{-28pt}

\hrule

\vspace*{2pt}

\hrule

%\vspace*{-2pt}

\def\tit{IMPACT OF THE ISOLATION PARAMETERS ON~RESOURCE ALLOCATION 
IN~THE~NETWORK SLICING MODEL}
                  
\def\titkol{Impact of the isolation parameters on resource allocation in the 
network slicing model}


\def\aut{F.\,A.~Moskaleva$^1$, Yu.\,V.~Gaidamaka$^{1,2}$, and~V.\,S.~Shorgin$^2$}

\def\autkol{F.\,A.~Moskaleva, Yu.\,V.~Gaidamaka, and~V.\,S.~Shorgin}

\titel{\tit}{\aut}{\autkol}{\titkol}

\vspace*{-9pt}


\noindent
$^1$Peoples' Friendship University of Russia (RUDN University), 6~Miklukho-Maklaya Str., Moscow 
117198, Russian\linebreak
$\hphantom{^1}$Federation


\noindent
$^2$Institute of Informatics Problems, Federal Research Center ``Computer Science and Control'' of the 
Russian\linebreak
$\hphantom{^1}$Academy of Sciences, 44-2 Vavilov Str., Moscow 119333, Russian Federation


\def\leftfootline{\small{\textbf{\thepage}
\hfill INFORMATIKA I EE PRIMENENIYA~--- INFORMATICS AND
APPLICATIONS\ \ \ 2020\ \ \ volume~14\ \ \ issue\ 4}
}%
 \def\rightfootline{\small{INFORMATIKA I EE PRIMENENIYA~---
INFORMATICS AND APPLICATIONS\ \ \ 2020\ \ \ volume~14\ \ \ issue\ 4
\hfill \textbf{\thepage}}}

\vspace*{6pt} 


\Abste{Network slicing technology is defined as one of the main components of the fifth generation of 
mobile communications, capable of solving the problem of the colossal growth of data traffic in cellular 
networks. A~key feature of slicing that limits the impact of one slice on another is to provide isolated 
performance guarantees to deliver high quality of service. In this article, a model of resource allocation 
during slicing is developed using the queuing theory. The main task of the work is to determine how 
network resources should be fairly shared between two slices in the system. The resource allocation 
problem is formulated as an optimization problem. For the constructed model, a numerical analysis was 
carried out showing the significant effect of the isolation parameters on the performance characteristics 
of the system.}

\KWE{network slicing; fairness resource allocation; isolation of slices; isolation parameter}


\DOI{10.14357/19922264200402} 

\vspace*{-12pt}

\Ack
\noindent
The paper has been supported by the RUDN University Strategic Academic Leadership Program and 
funded by the Russian Foundation for Basic Research according to the research projects 
No.\,19-07-00933 and No.\,20-07-01064.
\vspace*{3pt}

  \begin{multicols}{2}

\renewcommand{\bibname}{\protect\rmfamily References}
%\renewcommand{\bibname}{\large\protect\rm References}

{\small\frenchspacing
 {%\baselineskip=10.8pt
 \addcontentsline{toc}{section}{References}
 \begin{thebibliography}{99}

\bibitem{1-mos-1}
NGMN 5G white paper. 2015. Available at: {\sf http:// www.ngmn.de/5gwhite-paper.html} (accessed 
Octo-\linebreak ber~22, 2020).
\bibitem{2-mos-1}
TS 23.501. 2018. Technical Specification: System architecture for the 5G system (Release~15).
Version 15.2.0.\linebreak Available at: {\sf 
https://www.etsi.org/deliver/etsi\_ts/\linebreak
 123500\_123599/123501/15.02.00\_60/ts\_123501v
 150200p.pdf} 
(accessed October~22, 2020).

\bibitem{4-mos-1}
\Aue{Jain, R., D.-M. Chiu, and W.\,R.~Hawe.} 1998. A~quantitative measure of fairness and 
discrimination for resource allocation in shared computer system. arXiv:cs/9809099 [cs.NI]. Available 
at: {\sf https://arxiv.org/abs/cs/9809099} (accessed October~22, 2020).


\bibitem{3-mos-1} %4
GSMA. 2018. Network slicing~--- use case requirements.  Available at: {\sf  
https://www.gsma.com/futurenetworks/
wp-content/uploads/2018/07/Network-Slicing-Use-Case-Requirements-fixed.pdf}
(accessed October~22, 2020).

\bibitem{5-mos-1}
\Aue{Richart, M., J. Baliosian, J.~Serrat, and J.-L.~Gorricho.} 2016. Resource slicing in virtual 
wireless networks: A~survey. \textit{IEEE~T. Netw. Serv. Man.} 13(3):462--476.
\bibitem{6-mos-1}
\Aue{Vila, I., J.~P$\acute{\mbox{e}}$rez-Romero, O.~Sallent, and A.~Umbert.} 2020. 
Characterisation of radio access network slicing scenarios with 5G QoS provisioning. \textit{IEEE Access} 
8:51414--51430. doi: 10.1109/\mbox{access}.2020.2980685.
\bibitem{7-mos-1}
\Aue{Marabissi, D., and R.~Fantacci.} 2019. Highly flexible RAN slicing approach to manage 
isolation, priority, efficiency. \textit{IEEE Access} 7:97130--97142. doi: 
10.1109/\linebreak access.2019.2929732.
\bibitem{8-mos-1}
\Aue{Lieto, A., I. Malanchini, and A.~Capone.} 2018. Enabling dynamic resource sharing for slice 
customization in 5G\linebreak networks. \textit{Conference and Exhibition on Global Tele\-communications 
Proceedings}. IEEE. 1--7. doi: 10.1109/\linebreak GLOCOM.2018.8647249.
\bibitem{9-mos-1}
\Aue{Ageev, K.\,A., E.\,S.~Sopin, N.\,V.~Yarkina, K.\,E.~Samuylov, and S.\,Ya.~Shorgin.} 2020. 
Analiz mekhanizmov narezki seti s~uchetom garantiy dlya razlichnykh tipov trafika [Analysis of network 
slicing mechanisms with guarantees for various types of traffic]. \textit{Informatika i~ee 
Primeneniya~--- Inform. Appl.} 14(3):94--100.
\bibitem{10-mos-1}
\Aue{Yarkina, N., Yu. Gaidamaka, L.\,M.~Correia, and K.~Samouylov.} 2020. An analytical model 
for 5G network resource sharing with flexible SLA-oriented slice isolation. \textit{Mathematics} 
8(7):1177.
\bibitem{11-mos-1}
\Aue{Kelly, F.} 1997. Charging and rate control for elastic traffic. \textit{Eur.~T. Telecommun.} 
8:33--37.
\end{thebibliography}

 }
 }

\end{multicols}

\vspace*{-3pt}

\hfill{\small\textit{Received October 11, 2020}}

%\pagebreak

%\vspace*{-24pt}


\Contr

\noindent
\textbf{Moskaleva Faina A.} (b.\ 1996)~--- PhD student, Department of Applied Probability and 
Informatics, Peoples' Friendship University of Russia (RUDN University), 6~Miklukho-Maklaya Str., 
Moscow 117198, Russian Federation; \mbox{moskaleva-fa@rudn.ru}

\vspace*{3pt}

\noindent
\textbf{Gaidamaka Yuliya V.} (b.\ 1971)~--- Doctor of Science in physics and mathematics, 
professor, Department of Applied Probability and Informatics, Peoples' Friendship University of Russia 
(RUDN University), 6~Miklukho-Maklaya Str., Moscow 117198, Russian Federation; senior scientist, 
Institute of Informatics Problems, Federal Research Center ``Computer Science and Control'' of the 
Russian Academy of Sciences, 44-2~Vavilov Str., Moscow 119333, Russian Federation;  
\mbox{gaydamaka-yuv@rudn.ru}

\vspace*{3pt}

\noindent
\textbf{Shorgin Vsevolod S.} (b.\ 1978)~--- Candidate of Science (PhD) in technology, senior 
scientist, Institute of Informatics Problems, Federal Research Center ``Computer Science and 
Control'' of the Russian Academy of Sciences, 44-2~Vavilov Str., Moscow 119333, Russian 
Federation; \mbox{vshorgin@ipiran.ru}
\label{end\stat}

\renewcommand{\bibname}{\protect\rm Литература} 