%\newcommand{\wX}{\widetilde X}
%\newcommand{\N}{\mathbb N}
%\newcommand{\tod}{\stackrel{d}{\longrightarrow}}

\def\stat{korolev}

\def\tit{О РАСПРЕДЕЛЕНИИ ОТНОШЕНИЯ СУММЫ ЭЛЕМЕНТОВ ВЫБОРКИ,
ПРЕВОСХОДЯЩИХ НЕКОТОРЫЙ ПОРОГ,\\ К~СУММЕ ВСЕХ ЭЛЕМЕНТОВ
ВЫБОРКИ.~II$^*$}

\def\titkol{О распределении отношения суммы элементов выборки,
превосходящих %некоторый 
порог, к сумме всех элементов
выборки.~II}

\def\aut{В.\,Ю.~Королев$^1$}

\def\autkol{В.\,Ю.~Королев}

\titel{\tit}{\aut}{\autkol}{\titkol}

\index{ В.\,Ю.~Королев$^1$}
\index{ V.\,Yu.~Korolev }

{\renewcommand{\thefootnote}{\fnsymbol{footnote}} \footnotetext[1]
{Исследования выполнены при поддержке РФФИ (проект 19-07-00914) 
и~в~соответствии с программой Московского центра фундаментальной и прикладной 
математики.}}


\renewcommand{\thefootnote}{\arabic{footnote}}
\footnotetext[1]{ Факультет вычислительной математики и кибернетики 
Московского государственного университета имени  
М.\,В.~Ломоносова; Институт проб\-лем информатики Федерального 
исследовательского  
цент\-ра <<Информатика и управление>> Российской академии наук, 
\mbox{vkorolev@cs.msu.ru}}

\vspace*{-4pt}

\Abst{Рассматривается задача описания распределения доли суммы независимых 
случайных величин, которая составлена из слагаемых, превосходящих некоторый 
заданный порог. В~отличие от известных вариантов такой задачи, в которых 
фиксируется число суммируемых крайних порядковых статистик, особенность 
рассматриваемой здесь задачи заключается в том, что заданный порог может быть 
превзойден не предсказуемым заранее числом элементов выборки. Для случая, 
когда порог неограниченно возрастает с увеличением объема выборки, показано, 
что распределение указанного отношения может быть приближено обобщенным 
пуассоновским распределением, в котором обобщающим служит обобщенное 
распределение Парето.}


\KW{сумма независимых случайных величин; случайная сумма; биномиальное 
распределение; смесь распределений вероятностей;  экстремальная порядковая 
статистика; теорема Бал\-ке\-ма\,--\,Де  
Ха\-ана\,--\,Пи\-канд\-са; обобщенное распределение Парето; обобщенное 
пуассоновское распределение}

\DOI{10.14357/19922264200405} 
  
\vspace*{-4pt}


\vskip 10pt plus 9pt minus 6pt

\thispagestyle{headings}

\begin{multicols}{2}

\label{st\stat}


В 1897~г.\ итальянский экономист и социолог Вильфридо Парето выявил 
эмпирическую закономерность, заключающуюся в том, что 80\%  дохода страны 
аккумулируются в 20\% семей~\cite{Koch1998}. Эту закономерность многие 
пытались обосновать или опровергнуть. В~данной заметке предпринята попытка 
рассмотреть связанную с этим принципом  
ве\-ро\-ят\-ност\-но-ста\-ти\-сти\-че\-скую задачу о том, какую долю суммы 
наблюдений составляют наблюдения, превосходящие заданный порог.

Эта задача имеет большое значение не только для экономики, но и для других 
областей знания. Например, прослеживая изменение во времени параметров 
распределения отношения суммы элементов выборки, превосходящих некоторый 
порог, к~сумме всех элементов выборки при исследовании метеорологических 
данных (температура, осадки, теплообмен между атмосферой и океаном) во 
времени (например, в скользящем режиме, когда выборка~--- это <<окно>>, 
сдвигающееся в направлении астрономического времени при исследовании 
соответствующего временн$\acute{\mbox{о}}$го ряда), можно получить информацию 
об особенностях проявления процесса изменения климата, в просторечии 
называемого <<глобальным потеплением>>.

Данная статья продолжает исследования, начатые в работе~\cite{Korolev2020}. 
C~математической точки зрения эта задача тесно связана со статистикой 
цензурированных выборок. Известны разные варианты этой задачи. Некоторые из 
них упомянуты в~\cite{Korolev2020}, где рассмотрена задача описания 
распределения доли суммы независимых случайных величин, которая составлена из 
слагаемых, превосходящих некоторый заданный \textit{фиксированный} порог. 
В~отличие от известных вариантов такой задачи, в которых фиксируется число 
суммируемых крайних порядковых статистик, особенность рассмотренной 
в~\cite{Korolev2020} задачи заключается в том, что заданный порог может быть 
превзойден не предсказуемым заранее числом элементов выборки. 

В~указанной 
статье в терминах функции распределения отдельного слагаемого формально 
пред\-став\-лен явный вид распределения отношения суммы элементов выборки, 
превосходящих заданный порог, к сумме всех наблюдений. На эвристическом 
уровне выведены асимптотические и предельные\linebreak распределения этого отношения 
при фиксированном пороге, удобные для использования в качестве 
асимптотических аппроксимаций в~практических вычислениях. Рассмотрены 
ситуации, в~\mbox{которых} распределение слагаемых имеет легкий хвост (конечны 
вторые моменты), и~ситуации, в~которых\linebreak распределение слагаемых имеет тяжелый 
хвост (принадлежит к~об\-ласти притяжения устойчивого закона). Во всех случаях 
описана нормировка отношения, гарантирующая невырожденность предельного (при 
неограниченном увеличении числа слагаемых) распределения, и~сами предельные 
распределения (нормальное в~случае легких хвостов и~устойчивое в~случае 
тяжелых хвостов). В~на\-сто\-ящей работе рас\-смот\-рен случай, когда порог 
неограниченно возрастает с~увеличением объема выборки. Показано, что 
распределение указанного отношения может быть приближено обобщенным 
пуассоновским распределением, в~котором обобщающим служит обобщенное 
распределение Па\-рето.

Обозначим $S_n=X_1+\cdots+X_n$. Индикатор множества (события)~$A$
обозначим~$\mathbb{I}(A)$.

Пусть $u>0$ таково, что $0\hm<F(u)\hm<1$. Очевидно, 
$X_j\hm=X_j\mathbb{I}(X_j\hm<u)\hm+X_j\mathbb{I}(X_j\hm\ge u)$. Тогда
\begin{multline*}
S_n=\sum\limits_{j=1}^n X_j\mathbb{I}(X_j<u)+\sum\limits_{j=1}^n 
X_j\mathbb{I}(X_j\ge u)\equiv{}\\
{}\equiv
S_n^{(<u)}+S_n^{(\ge u)}.
\end{multline*}

Основным объектом изучения будет распределение отношения $R(u)\hm=S_n^{(\ge 
u)}/S_n$, а~в~первую очередь~--- распределение случайной величины $S_n^{(\ge 
u)}$ при условии, что порог возрастает при увеличении объема выборки.

Как было показано в статье~\cite{Korolev2020}, если порог $u$ фиксирован, а 
объем $n$ доступной выборки настолько большой (неограниченно возрастает), что 
обе суммы~$S_n^{(< u)}$ и~$S_n^{(\ge u)}$ содержат много слагаемых, то для
распределения случайной величины~$S_n^{(< u)}$ можно применять
нормальную аппроксимацию. Если хвосты распределения случайных
величин~$X_j$ убывают достаточно быстро (так, что конечен второй
момент), то нормальная аппроксимация также справедлива для распределения
случайных величин~$S_n^{(\ge u)}$ и $R_u$. Если же функция~$F(u)$ имеет столь
тяжелый хвост, что у случайной величины~$X_j$ отсутствует дисперсия, то для
распределений случайных величин~$S_n^{(\ge u)}$ и $R_u$ справедлива
аппроксимация устойчивым законом.

Однако иногда интерес представляет ситуация, в~которой порог~$u$
столь велик, что сумма $S_n^{(\ge u)}$ содержит лишь умеренное число
слагаемых. В~таком случае для распределения случайной величины
$S_n^{(\ge u)}$ оказывается возможным применить обобщенную
пуассоновскую аппроксимацию.

Формально предположим, что порог~$u$ зависит от~$n$: $u\hm=u_n$. При
этом будем считать, что существует число $\lambda\hm\in(0,\infty)$
такое, что
$$
n\left(1-F(u_n)\right)\longrightarrow \lambda
$$
при $n\to\infty$. Например, если для простоты предположить, что $n\left(1\hm-
F(u_n)\right)\hm=\lambda$, то $u_n\hm=F^{-1}(1-{\lambda}/{n})$. При этом 
порог~$u_n$ не обязан неограниченно возрастать при $n\hm\to\infty$, например, 
если носитель распределения~$F$ конечен, т.\,е.\
 $\mathrm{rext}\,F\hm\equiv\sup\{x:\,F(x)\hm<1\}\hm<\infty$.

В сделанных предположениях
$$
\lim\limits_{n\to\infty}\sum\limits_{k=0}^{\infty}\left\vert {\sf
P}\left(N_n(u_n)=k\right)-e^{\lambda}\fr{\lambda^k}{k!}\right\vert =0\,.
$$
Пусть $N_{\lambda}$~--- случайная величина, имеющая распределение
Пуассона с параметром~$\lambda$. Предположим, что при каждом $n\hm\ge1$
случайная величина~$N_{\lambda}$ независима от последовательности
$X_1^{(\ge u_n)},X_2^{(\ge u_n)}$.

Имеем
\begin{multline*}
\sup\limits_x \left\vert {\sf P}(S_n^{(\ge u)}<x)-{\sf P}
\left(\sum\limits_{k=0}^{N_{\lambda}}X_j^{(\ge u_n)}<x\right)\right\vert =
{}\\
{}
=\sup\limits_x\left\vert \sum\limits_{k=0}^{\infty}{\sf P}(N_n(u_n)=k){\sf P}
\left(\sum\nolimits_{j=0}^kX_j^{(\ge u_n)}<x\right)-{}\right.\hspace*{-1.26616pt}
\\
\left.{}
-\sum\nolimits_{k=0}^{\infty}{\sf P}
\left(N_{\lambda}=k\right){\sf P}\left(\sum\limits_{j=0}^kX_j^{(\ge
u_n)}<x\right)\right\vert \le
{}\\
{}
\le\sum\nolimits_{k=0}^{\infty}\left\vert {\sf P}
\left(N_n(u_n)=k\right)-{\sf P}
(N_{\lambda}=k)\right\vert \le{}\\
{}\le
 2\left(1-F(u_n)\right)
\min\{1,\lambda\}
\end{multline*}
(см., например,~\cite{BarbourHall1984}).

Характеристическую функцию случайной величины~$X_j^{(\ge u_n)}$
обозначим~$f_{u_n}(t)$:
$$
f_{u_n}(t)=\fr{1}{1-F(u_n)}\int\limits_{u_n}^{\infty}e^{itx}dF(x),\enskip
t\in\mathbb{R}\,.
$$
Из сказанного выше вытекает, что в сделанных предположениях
распределение случайной величины~$S_n^{(\ge u_n)}$ может быть
аппроксимировано обобщенным пуассоновским распределением, задаваемым
характеристической функцией
$$
g_n(t;\lambda)=\exp\left\{\lambda\left[f_{u_n}(t)-1\right]\right\},\enskip
t\in\mathbb{R}.
$$

Более того, если $\mathrm{rext}F\hm=\infty$, то в сделанных
предположениях $u_n\hm\to\infty$. Это позволяет для распределения
случайной величины~$S_n^{(\ge u_n)}$ использовать аппроксимацию, не зависящую 
от вида функции распределения~$F$, а~именно справедлива
\mbox{теорема} Бал\-ке\-мы\,--\,Де Ха\-ана\,--\,Пи\-канд\-са~\cite{BalkemaDeHaan1974, Pickands1975}, 
согласно которой, если при
некоторой линейной нормировке с~по\-мощью числовых последовательностей~$c_n$ 
и~$b_n\hm>0$ распределения случайных величин $(X_{(n)}\hm-c_n)/b_n$
имеют слабым пределом при $n\hm\to\infty$ невырожденную функцию
распределения (обязательно при этом принадлежащую к~одному из
трех возможных типов предельных распределений экстремальных
значений, см., например,~\cite{Galambos1978}), то существуют такие
числа $\alpha$, $\beta\hm>0$ и $\gamma$, что
\begin{multline*}
\lim\limits_{u\to\infty}{\sf P}
\left(X_j-u<x|X_j>u\right)={}\\
{}=
\lim\limits_{u\to\infty}\fr{F(x+u)-F(u)}{1-F(u)}=
{}\\
{}
=H_{\alpha,\beta,\gamma}(x)\equiv
1-\left(1+\gamma\cdot\fr{x-\alpha}{\beta}\right)^{-1/\gamma},\enskip
x>0\,.
\end{multline*}
Распределение, соответствующее функции распределения
$H_{\alpha,\beta,\gamma}(x)$, называется \textit{обобщенным
распределением Парето}, при этом $\gamma$~--- параметр формы,
$\alpha$~--- параметр положения (сдвига), $\beta$~--- параметр
масштаба. При $\gamma\hm>0$ $H_{\alpha,\beta,\gamma}(x)$~--- это
распределение Парето, при $\gamma\hm=0$ $H_{\alpha,\beta,\gamma}(x)$~---
это показательное распределение, при $\gamma\hm<0$
$H_{\alpha,\beta,\gamma}(x)$~--- это бе\-та-рас\-пре\-де\-ление.

Характеристическую функцию обобщенного распределения Парето
$H_{\alpha,\beta,\gamma}(x)$ обозначим $h_{\alpha,\beta,\gamma}(t)$:
\begin{multline*}
h_{\alpha,\beta,\gamma}(t)=\int\limits_{0}^{\infty}e^{itx}dH_{\alpha,\beta,\gamma}(x)={}\\
{}=
\fr{1}{\beta}\int\limits_{0}^{\infty}e^{itx}
\left(1+\gamma\cdot\fr{x-\alpha}{\beta}\right)^{-
(\gamma+1)/\gamma}dx\,,\enskip
t\in\mathbb{R}\,.
\end{multline*}

Пусть $Y_1,Y_2,\ldots$~--- независимые случайные величины, имеющие
одно и то же обобщенное распределение Парето
$H_{\alpha,\beta,\gamma}(x)$. Тогда из теоремы Бал\-ке\-мы\,--\,Де
Ха\-ана\,--\,Пи\-канд\-са и вида распределения случайной величины $X_j^{(\ge u)}$ 
вытекает, что при большом~$u$ справедливо представление
$$
X_j^{(\ge u)}\approx u+Y_j,
$$
в котором параметры $\alpha$, $\beta$ и $\gamma$ обобщенного
распределения Парето случайных величин~$Y_j$ зависят от вида функции
распределения~$F$.

Тогда из сказанного выше вытекает, что если $u_n\hm\to\infty$ и
$n\left(1\hm-F(u_n)\right)\hm\to\lambda$ при $n\hm\to\infty$, то
$$
{\sf P}(S_n^{(\ge u)}<x)\approx{\sf P}
\left(\sum\limits_{j=0}^{N_{\lambda}}(u_n+Y_j)<x\right),
$$
т.\,е.\ в таком случае для аппроксимации распределения случайной
величины $S_n^{(\ge u)}$ можно использовать обобщенное пуассоновское
распределение, соответствующее характеристической функции
\begin{multline}
g_n(t;\alpha,\beta,\gamma,\lambda)=
\exp\left\{\lambda\left[e^{itu_n}h_{\alpha,\beta,\gamma}(t)-
1\right]\right\},\\
t\in\mathbb{R}\,,
\label{e18-kr}
\end{multline}
при этом параметры $\alpha$, $\beta$, $\gamma$ и~$\lambda$
оцениваются статистически.

Что же касается распределения отношения $S_n^{(\ge u_n)}/S_n^{(< u_n)}$ при 
пороге, растущем описанным выше образом, то можно заметить, что, тогда как 
при $n\hm\to\infty$  
сумма~$S_n^{(\ge u_n)}$ содержит умеренное число слагаемых, число слагаемых  
в~сумме~$S_n^{(< u_n)}$ неограниченно возрастает, и,~поскольку эти слагаемые 
ограничены, можно считать, что
$$
\fr{S_n^{(< u_n)}}{n}\approx {\sf E}X_1^{(<u_n)}\equiv \underline{a}_{u_n}\,.
$$
Тогда при указанных выше условиях
$$
n \,\fr{S_n^{(\ge u_n)}}{S_n^{(< u_n)}}\approx \fr{Z_n(\alpha, \beta, 
\gamma, \lambda)}{\underline{a}_{u_n}}\,,
$$
где $Z_n(\alpha, \beta, \gamma, \lambda)$~--- случайная величина с обобщенной 
пуассоновской характеристической функцией 
$g_n(t;\alpha,\beta,\gamma,\lambda)$ (см.~(\ref{e18-kr})).

{\small\frenchspacing
 {%\baselineskip=10.8pt
 %\addcontentsline{toc}{section}{References}
 \begin{thebibliography}{9}

\bibitem{Koch1998}
\Au{Kox Р.} Принцип 80/20~/ Пер. с  англ.~--- М.: Эксмо, 2012. 352~с.
(\Au{Koch~R.} {The 80/20 principle: The secret of achieving more with less}.~--- London: 
Nicholas Brealey Publishing, 1998. 302~p.)

\bibitem{Korolev2020}
\Au{Королев В.\,Ю.} О распределении отношения суммы элементов выборки, 
превосходящих некоторый порог, к~сумме всех элементов выборки.~I~// 
Информатика и~её применения, 2020. Т.~14. Вып.~3. С.~26--34.

\bibitem{BarbourHall1984}
\Au{Barbour A.\,D., Hall P.} On the rate of Poisson convergence~// 
P.~Camb. Philos.
Soc., 1984. Vol.~95. P.~473--480.

\bibitem{BalkemaDeHaan1974}
\Au{Balkema A., de Haan~L.}
Residual life time at great age~// Ann. Probab., 1974. Vol.~2. P.~792--804.

\bibitem{Pickands1975}
\Au{Pickands J.} Statistical inference using extreme order statistics~// 
Ann. Stat., 1975. Vol.~3.  
P.~119--131.

\bibitem{Galambos1978}
\Au{Галамбош Я.} Асимптотическая теория экстремальных порядковых статистик~/
Пер. с англ.~---
М.: Наука, 1984. 314~с.
(\Au{Galambos~J.} {The asymptotic theory of extreme order statistics}.~--- New York, NY, USA: 
Wiley, 1978. 352~p.)
\end{thebibliography}

 }
 }

\end{multicols}

\vspace*{-6pt}

\hfill{\small\textit{Поступила в~редакцию 28.11.19}}

%\vspace*{8pt}

%\pagebreak

\newpage

\vspace*{-28pt}

%\hrule

%\vspace*{2pt}

%\hrule

%\vspace*{-2pt}

\def\tit{ON THE DISTRIBUTION OF~THE~RATIO OF~THE~SUM\\ 
OF~SAMPLE ELEMENTS EXCEEDING A~THRESHOLD\\ TO~THE~TOTAL 
SUM OF~SAMPLE ELEMENTS.~II}


\def\titkol{On the distribution of the ratio of the sum of sample elements 
exceeding a threshold to the total sum of sample elements.~II}


\def\aut{V.\,Yu.~Korolev$^{1,2}$}

\def\autkol{V.\,Yu.~Korolev}

\titel{\tit}{\aut}{\autkol}{\titkol}

\vspace*{-11pt}


\noindent
$^1$Faculty of Computational Mathematics and Cybernetics, Lomonosov Moscow State University, 
GSP-1, Leninskie\linebreak
$\hphantom{^1}$Gory, Moscow 119991, Russian Federation

\noindent
$^2$Institute of Informatics Problems, Federal Research Center 
``Computer Sciences and Control'' of the 
Russian\linebreak
$\hphantom{^1}$Academy of Sciences; 44-2~Vavilov Str., Moscow 119133, Russian Federation


\def\leftfootline{\small{\textbf{\thepage}
\hfill INFORMATIKA I EE PRIMENENIYA~--- INFORMATICS AND
APPLICATIONS\ \ \ 2020\ \ \ volume~14\ \ \ issue\ 4}
}%
 \def\rightfootline{\small{INFORMATIKA I EE PRIMENENIYA~---
INFORMATICS AND APPLICATIONS\ \ \ 2020\ \ \ volume~14\ \ \ issue\ 4
\hfill \textbf{\thepage}}}

\vspace*{3pt} 


\Abste{The problem of description of the distribution of the ratio of the sum of 
sample elements exceeding a~threshold to the total sum of sample elements is 
considered. Unlike other versions of this problem in which the number of summed 
extreme order statistics and the threshold are fixed, here the specified threshold 
can be exceeded by an unpredictable number of sample elements. The situation is 
considered where the threshold infinitely increases as the sample size grows. It is 
demonstrated that in this case, the distribution of the ratio mentioned above can be 
approximated by the compound Poisson distribution in which the compounding 
law is the generalized Pareto distribution.}

\KWE{sum of independent random variables; random sum; binomial distribution; 
Poisson approximation; extreme order statistic; Balkema\,--\,De~Haan\,--\,Pickands 
theorem; generalized Pareto distribution; compound Poisson distribution}


\DOI{10.14357/19922264200405} 

%\vspace*{-20pt}

\Ack
\noindent
The research was supported by the Russian Foundation for Basic Research 
(project 19-07-00914). The research was conducted in accordance with the 
program of the Moscow Center for Fundamental and Applied Mathematics.

%\vspace*{6pt}

  \begin{multicols}{2}

\renewcommand{\bibname}{\protect\rmfamily References}
%\renewcommand{\bibname}{\large\protect\rm References}

{\small\frenchspacing
 {%\baselineskip=10.8pt
 \addcontentsline{toc}{section}{References}
 \begin{thebibliography}{9}

\bibitem{1-kr}
\Aue{Koch, R.} 1998. \textit{The 80/20 principle: The secret of achieving more with less}. London: 
Nicholas Brealey Publishing. 302~p.
\bibitem{2-kr}
\Aue{Korolev, V.\,Yu.} 2020. O~raspredelenii otnosheniya summy elementov vyborki, 
prevoskhodyashchikh nekotoryy porog, k~summe vsekh elementov vyborki.~I [On the distribution of the 
ratio of the sum of sample elements exceeding a~threshold to the total sum of sample elements.~I]. 
\textit{Informatika i~ee Primeneniya~--- Inform.Appl.} 14(3):26--34.
\bibitem{3-kr}
\Aue{Barbour, A.\,D., and P.~Hall.} 1984. On the rate of Poisson convergence. \textit{P.~Camb. Philos. Soc.} 95:473--480.
\bibitem{4-kr}
\Aue{Balkema, A., and L.~de Haan.} 1974. Residual life time at great age. \textit{Ann. Probab.} 
2:792--804.
\bibitem{5-kr}
\Aue{Pickands, J.} 1975. Statistical inference using extreme order statistics. \textit{Ann.  Stat.} 
3:119--131.
\bibitem{6-kr}
\Aue{Galambos, J.} 1978. \textit{The asymptotic theory of extreme order statistics}. New York, NY: 
Wiley. 352~p.
\end{thebibliography}

 }
 }

\end{multicols}

\vspace*{-3pt}

\hfill{\small\textit{Received November 28, 2019}}

%\pagebreak

%\vspace*{-24pt}


\Contrl

\noindent
\textbf{Korolev Victor Yu.} (b.\ 1954)~--- Doctor of Science in physics and mathematics, professor, 
Head of Department, Faculty of Computational Mathematics and Cybernetics, and principal scientist, 
Moscow Center for Fundamental and Applied Mathematics, Lomonosov Moscow State University, 
GSP-1, Leninskie Gory, Moscow 119991, Russian Federation; leading scientist, Federal Research Center 
``Computer Science and Control'' of the Russian Academy of Sciences, 44-2~Vavilov Str.,Moscow 
119333, Russian Federation; \mbox{vkorolev@cs.msu.ru}

\label{end\stat}

\renewcommand{\bibname}{\protect\rm Литература} 