\def\stat{grusho-zab}

\def\tit{О ВЕРОЯТНОСТНЫХ ОЦЕНКАХ ДОСТОВЕРНОСТИ 
ЭМПИРИЧЕСКИХ ВЫВОДОВ$^*$}

\def\titkol{О вероятностных оценках достоверности 
эмпирических выводов}

\def\aut{А.\,А.~Грушо$^1$, М.\,И.~Забежайло$^2$, 
Д.\,В.~Смирнов$^3$, Е.\,Е.~Тимонина$^4$}

\def\autkol{А.\,А.~Грушо, М.\,И.~Забежайло, Д.\,В.~Смирнов, 
Е.\,Е.~Тимонина}

\titel{\tit}{\aut}{\autkol}{\titkol}

\index{Грушо А.\,А.}
\index{Забежайло М.\,И.}
\index{Смирнов Д.\,В.}
\index{Тимонина Е.\,Е.}
\index{Grusho A.\,A.}
\index{Zabezhailo M.\,I.}
\index{Smirnov D.\,V.}
\index{Timonina E.\,E.}


{\renewcommand{\thefootnote}{\fnsymbol{footnote}} \footnotetext[1]
{Работа выполнена при частичной финансовой поддержке РФФИ (проект 18-29-03081).}}


\renewcommand{\thefootnote}{\arabic{footnote}}
\footnotetext[1]{Институт проблем информатики Федерального исследовательского центра 
<<Информатика и~управ\-ле\-ние>> Российской академии наук, \mbox{grusho@yandex.ru}}
\footnotetext[2]{Вычислительный центр Федерального исследовательского центра <<Информатика 
и~управ\-ле\-ние>> Российской академии наук, \mbox{m.zabezhailo@yandex.ru}}
\footnotetext[3]{ПАО Сбербанк России, dvlsmirnov@sberbank.ru}
\footnotetext[4]{Институт проблем информатики Федерального исследовательского центра 
<<Информатика и~управ\-ле\-ние>> Российской академии наук, \mbox{eltimon@yandex.ru}}

%\vspace*{-12pt}


  \Abst{Работа посвящена некоторым особенностям анализа данных в~задачах поиска 
инсайдеров. Обсуждаются возможности использования различных подходов к~описанию 
диагностики действий инсайдеров при анализе больших эмпирических данных. В~задачах этого 
типа необходимо установить (спрогнозировать, диагностировать и~др.) наличие или отсутствие 
целевых свойств у каких-либо пользователей из заданного множества.
  Оценка правильности правдоподобных рассуждений проверяется на основе оценок 
вероятностей случайного появления найденных закономерностей в~простейших вероятностных 
моделях. 
  Рассмотренные примеры показывают, при каких соотношениях параметров 
возможно эффективное выявление корреляционных связей между событиями, с~помощью 
которых можно выявлять инсайдеров. 
  Указаны два способа управления соотношениями между параметрами, позволяющие 
получать содержательную информацию. Первый способ основан на разделении периода 
наблюдений на промежутки, в~течение которых искомая корреляция может проявиться. Второй 
способ связан со способами сокращения множества пользователей, которые потенциально 
могут стать инсайдерами, т.\,е.\ речь идет о~формировании кластеров, в~которых 
вероятностные оценки становятся работоспособными. Искомые соотношения между 
параметрами для поиска корреляций можно определять с~по\-мощью предельных тео\-рем в~схеме 
серий.}
  
  \KW{враждебный инсайдер; каузальный анализ; вероятностные оценки случайного 
появления свойств}
  
\DOI{10.14357/19922264200401} 
  
\vspace*{3pt}


\vskip 10pt plus 9pt minus 6pt

\thispagestyle{headings}

\begin{multicols}{2}

\label{st\stat}

  

\section{Введение} 
  
  Угроза информационной безопас\-ности со стороны враждебного инсайдера 
(далее~--- инсайдера) считается второй в~мире по значимости после хакеров. 
Обнаружить враждебные действия инсайдера намного сложнее, чем 
вредоносное воздействие хакера, потому что инсайдеры знакомы с~системой 
безопас\-ности организации и~имеют доступ к~ее час\-ти, при этом объем 
вредоносных действий, которые можно зафиксировать, мал по сравнению 
с~объемом легальных действий, совершаемых пользователями в~процессе работы
 с~информационной системой (ИС) организации. Разработанные алгоритмы 
поиска инсайдеров часто не способны выявить новые методы атак инсайдера, 
а~лучшие результаты показывают на шаблонных входных данных.
  
  Необходимо отметить, что, как правило, поиск инсайдеров связан с~поиском 
аномалий в~различных информационных пространствах, где могут появляться 
признаки инсайдеров. Например, Министерство обороны США провело 
исследования под названием Anomaly Detection at Multiple Scales (ADAMS)~[1], 
основная цель которого~--- разработать и~внедрить технологию классификации 
и~обнаружения аномалий в~больших объемах данных с~целью поиска следов 
инсайдеров.
  
  В настоящее время множество статей посвящено проблеме поиска 
инсайдеров. Отметим некоторые работы, имеющие связь с~исследованиями 
данной статьи.
  
  Пример корреляционного подхода в~поиске инсайдера можно найти 
  в~работе~[2], в~которой рас\-смат\-ри\-ва\-ют\-ся описания связей между событиями 
  и~инсайдерами. В~работе~[3] прямо ставится проблема верификации вывода о 
результатах поиска инсайдера в~нескольких информационных пространствах, но 
методы верификации не рассматриваются. Вероятностные методы рассмотрены 
в~[4], где показано, как использовать скрытые марковские модели для оценок 
нормального поведения пользователей ИС. Обзор~[5] рассматривает различные 
подходы к~анализу угроз, порождаемых инсайдерами. 
  
  Приведем несколько российских работ, посвященных проблеме поиска 
инсайдеров. В~статьях~[6--8] построены методы работы с~несколькими 
информационными пространствами в~задачах поиска инсайдера и~поиска 
аномалий, связанных с~инсайдером.
  
   В данной работе рассматривается задача верификации найденных признаков 
инсайдеров вероятностными методами.
  
  Пусть задано множество пользователей ИС, содержащей профессиональную 
информацию и~ценную информацию (в том числе дорогостоящую). Допустим, 
что среди пользователей есть один инсайдер. Относительно его выявления 
возможны различные частные постановки задачи.
  \begin{itemize}
  \item[A.] 
Если произошло компьютерное преступление, в~котором виновен 
инсайдер, то надо провести расследование и~определить его личность 
среди множества пользователей, собрать доказательную базу.
\item [B.] Если есть множество конкретных должностных лиц, то надо 
определить, есть ли в~этом множестве инсайдер.
\end{itemize}
 
  Признаком возможных действий инсайдера могут служить аномалии 
  в~технологических информационных процессах. Тогда надо выявлять аномалии 
различных типов и~на основе найденных аномалий определить, существует ли 
инсайдер.
  
  Перечислим некоторые подходы к~выявлению искомой информации.
  \begin{enumerate}[1.]
  \item Поиск связи ка\-ко\-го-ни\-будь пользователя~$u$ и~свойства, которое 
наблюдается в~некоторых событиях, связанных с~возможным появлением 
инсайдера. Например, появление у клиентов банка больших сумм на счету 
сопровождается дорогими покупками пользователя~--- служащего банка. 
Гипотеза состоит в~том, что этот пользователь продает информацию о 
появлениях указанного свойства и,~получив деньги, тратит на заранее 
запланированные покупки или участие в~азартных играх. Связь пользователя 
с~появлением данного свойства связана с~задачами~A и~B.
\item В ходе расследования компьютерного преступления в~банке 
выяснилось, что преступления подобного типа уже совершались 
и~проходили по определенной схеме. Если пользователь участвует в~схеме, 
то должны появляться другие элементы схемы при проявлении действия 
из этой схемы. Тогда подход состоит в~выявлении действия схемы.
\item На фоне стандартной деятельности пользователей появились 
вкрапления нестандартных действий пользователя, которые следует 
рас\-смат\-ри\-вать как аномалию, если эти действия удается заметить.
\item Одним из важных свойств, используемых для выявления инсайдеров, 
являются противоречия, например размер зарплаты и~размеры расходов, 
круг профессиональных задач и~широта интересов и~др.
\item Подходом, близким по эффективности, но отличающимся от метода 
поиска противоречий, является поиск некоторых характеристик 
пользователя, порождающих желание стать инсайдером. К~таким 
характеристикам относятся, например, знакомства с~криминальными 
элементами, жадность, игромания и~др.
\end{enumerate}

  Выше перечислены некоторые свойства, но этот набор свойств далеко не 
полный. Во всех случаях возможно построение вероятностной модели 
и~проверки согласия с~ней. Но надо заметить, что выявленные свойства, которые 
назовем эмпирическими закономерности, могут проистекать от случайного 
стечения обстоятельств, поэтому надо иметь методы, позволяющие фильтровать 
возможности случайных сочетаний свойств, которые можно принять за 
искомые закономерности. Отметим, что чем сложнее эти методы, тем меньше 
они применимы на практике. С~другой стороны, упрощение моделей и~методов 
может привести к~потере реальных закономерностей. Выскажем гипотезу, что 
пропущенные или отброшенные кандидаты на закономерности в~случае 
реального инсайдера должны повторяться и~при запоминании отброшенных 
кандидатов могут получить уверенное подтверждение при вторичном анализе. 
  
  Работа посвящена построению простых методов фильтрации кандидатов на 
закономерности в~подходах 1-го типа или фильтрации аномалий.
  
\section{Модель корреляции важных событий и~действий пользователей}

  Пусть $U=\{ u_1, u_2, \ldots, u_m\}$~--- пользователи ИС (сотрудники 
организации), $V\hm= \{ v_1, v_2, \ldots , v_n\}$~--- клиенты, на которых 
работают пользователи, и~эти множества не пересекаются. 
  
  Простейшие модели и~логику работы с~ними построим для подходов 1-го 
типа.
  
  Пусть $C$~--- события, связанные с~появлением и~доступностью ценной 
информации. Предположим, что события~$C$ возникают в~соответствии 
с~пуассоновским процессом с~параметром~$\lambda$. Траты (большие) 
произвольного пользователя~$u$ из множества~$U$ также происходят 
в~соответствии с~пуассоновским процессом с~параметром~$\mu$. Все процессы 
независимы. Введем параметр~$T$~--- время\linebreak возможного появления 
зависимости поль\-зо\-ва\-ельских трат и~появления ценной информации.\linebreak 
Вероятность того, что данный пользователь в~промежуток времени~$T$ не 
проводил больших трат,\linebreak рав\-на~$e^{-\mu T}$.
{ %\looseness=1

}
  
  Рассмотрим возможность появления трат в~фиксированный промежуток 
времени~$T$, связанный с~фиксированным событием~$C$. Вероятность того, 
что все пользователи в~промежуток~$T$ не проводили больших трат, 
равна~$e^{-\mu Tm}$, т.\,е.\ можно окружить каждое появление события~$C$ 
отрезком длины~$T$, начинающимся с~момента появления события~$C$. По 
предположению, события~$C$ появляются в~соответствии с~пуассоновским 
процессом, и~можно выбрать промежуток времени~$T$ таким образом, чтобы 
отрезки длины~$T$ не пересекались с~вероятностью, близкой к~1 (для этого 
надо рассматривать события~$C$ в~процессе с~$\lambda\hm\ll \mu$). 
Вероятность того, что ни в~одном промежутке времени при 
появлении~$k$~событий~$C$ не было больших трат, не превосходит $ke^{-\mu Tm}$.
  
  Если $\mu Tm$~--- большая величина, то вероятность\linebreak непопадания больших 
трат на ка\-кой-ни\-будь отрезок длины~$T$ вокруг событий~$C$ мала. Тогда 
появление таких событий указывает на множество подозрительных 
пользователей. Можно выписать распределе\-ние числа таких пользователей, но 
вопрос, когда их будет мало или не будет вовсе, с~большой вероятностью 
в~данном случае более интересен, т.\,е.\ поставим вопрос о выборе параметров 
целесообразного поиска. При каких соотношениях параметров асимптотически 
в~результате корреляционной проверки проявляется только инсайдер? Для 
решения этой задачи используем параметр~$k$~--- число появлений 
события~$C$.
  
  Предположим, что все пользователи честные. Пусть $\varphi$~--- случайная 
величина, равная числу появлений события~$C$ за время наблюдения~$t$ за 
поведением и~тратами пользователей из множества $U\hm= \{ u_1, u_2, \ldots , 
u_m\}$. События~$C$ появляются в~соответствии с~законом простого 
пуассоновского процесса:
  $$
  P(\varphi=k)=\fr{(\lambda t)^k}{k!}\,e^{-\lambda t}\,.
  $$
  
  Предположим, что выброшены все пересечения промежутков~$T$ после 
событий~$C$. Распределение Пуассона при больших~$k$ можно 
аппроксимировать нормальным распределением. Отсюда получим
  $$
  P(\varphi=\lambda t(1+o(1)))=1+o(1)\,.
  $$
    Тогда время~$t$ наблюдения для получения~$k$~событий~$C$ приближенно 
равно~$k/\lambda$. 
  
  Для определения необходимых значений~$k$ и~определения невозможного 
события для всех честных пользователей воспользуемся следующими оценками:
    $p=1-e^{-\mu T}$~--- вероятность появления большой траты в~данный 
промежуток~$T$ для фиксированного пользователя~$u$. По предположению 
промежутки~$T$ не пересекаются. Тогда вероятность появления 
большой траты в~каждый промежуток~$T$ для данного пользователя~$u$ равна~$p^k$. Отсюда 
среднее число таких пользователей равно~$mp^k$. При $T\hm= const$ и~$k\hm= 
(\ln m+\ln \ln m)e^{\mu T}$ получим оценку среднего числа таких 
пользователей~$1/(\ln m)$.
  
  В этих условиях проявление инсайдера с~вероятностью, близкой к~1, 
возможно при появлении пользователя, который делает большие траты после 
каждого появления события~$C$ не менее~$k$~раз. Отметим, что время 
наблюдения оценивается величиной 
  $$
  t=\fr{k}{\lambda}\left( 1+o(1)\right)\,.
  $$
Такие параметры поиска значительно лучше, чем при ожидании полного 
отсутствия трат.
  
\section{Модель машинного обучения}
  
  В предыдущем разделе показано, что предложенный метод работает, но 
требует выполнения двух условий: 
  \begin{enumerate}[(1)]
  \item наблюдения за всеми тратами; 
\item появления трат после наступления событий~$C$.
\end{enumerate}
  
  Наблюдения за всеми большими тратами возможно, например, когда 
  в~игровом клубе находится инсайдер, который докладывает в~службу 
безопасности обо всех сотрудниках организации, иг\-ра\-ющих на крупные суммы. 
В~таком случае в~предыду\-щем разделе найдены условия однозначного 
выявления инсайдера.
  
  Рассмотрим схожую задачу поиска за\-ви\-си\-мости больших трат и~появления 
ценной информации в~базе данных организации в~несколько других 
предположениях. В~рассматриваемом случае возможность выявления инсайдера 
основана на сокращении числа подозреваемых пользователей (2-й подход).
  
  Пусть за время~$t$ в~множестве~$U$ выделено подмножество~$U_1$ 
пользователей с~очень большими единовременным тратами (покупка машины, 
квартиры, дачи и~др.). Под эти покупки, возможно, взяты кредиты или займы у 
родственников или знакомых. Ясно, что объем множества~$U_1$ значительно 
меньше, чем объем множества~$U$. Воспользуемся этим обстоятельством. 
Каждый участник множества~$U_1$ имеет основания использовать свое 
служебное положение для дополнительного заработка. Соберем информацию о 
доступах пользователей из~$U_1$ за время~$t$ к~ценной информации. Пусть 
$C(1,\varphi), C(2,\varphi),\ldots , C(k(\varphi),\varphi)$~--- случаи получения 
ценной информации пользователем~$\varphi$ из~$U_1$ за время~$t$ (за счет 
расширения множества запросов в~базе данных и~других способов).
  
  Сравним полученные данные с~доступами к~ценной информации у каждого из 
пользователей в~множестве $U\backslash U_1$ (машинное обучение). Пусть 
$v(i)$~--- число пользователей, получившие $i$~раз за время~$t$ данные, 
которые можно считать ценной информацией. Рассмотрим относительные 
частоты чисел~$v(i)$ к~мощности множества $U\backslash U_1$. Эти числа 
можно рассматривать как оценки вероятностей~$P$ того, что данный (честный) 
пользователь~$i$~раз получит ценную информацию за время~$t$. Тогда при 
маленьких значениях~$P(k(\varphi))$ для~$\varphi$ из~$U_1$ и~небольших 
множествах~$U_1$ получаем оценку вероятности того, что 
пользователь~$\varphi$ злоупотреблял своим служебным положением и~может 
быть инсайдером, который продает ценную информацию.
  
  \section{Заключение}
   
  Рассмотренные в~работе методы выявления инсайдеров служат простейшими 
примерами задач поиска и~анализа причинности (идентификации каузальных 
оснований)~[9]. В~самом деле, предполагается, что наступление события~$C$ 
служит причиной появления больших трат пользователя, торговца ценной 
информацией, т.\,е.\ предполагается, что продажа информации о 
событиях~$C$ влечет возможность осуществления больших трат. Поскольку 
сделанное предположение может оказаться ложным, то необходимо найти 
критерии фильтрации ложных каузальных оснований. Наоборот, должны быть 
подтверждения правильности предполагаемых каузальных оснований. Для 
решения этих задач подходят ве\-ро\-ят\-ност\-но-ста\-ти\-сти\-че\-ские 
методы~[10]. Однако применения этих методов явно отличаются от 
традиционных методов математической статистики~[11]. Основная идея 
применения вероятностных методов~--- это построение событий, вероятности 
которых либо равны~0, либо при выбранных соотношениях параметров 
асимптотически стремятся к~0. При равенстве~0 мы имеем дело с~запретами 
вероятностных мер в~конечных вероятностных пространствах~[12], при 
стремлении вероятностей к~0 имеем дело с~состоятельными процедурами 
статистических решений~[13] (чаще всего в~схеме серий, что отличает эти 
подходы от классической математической статистики). Но такие подходы 
соответствуют задаче фильтрации кандидатов на каузальные основания. 
Возникает вопрос об истинности предположений  
о~рас\-смат\-ри\-ва\-емых вероятностных мерах и~точности решений. Здесь 
также видны отличия традиционных задач математической статистики от 
фильтрации каузальных оснований~[9, 14]. Эти отличия состоят в~избыточности 
поиска каузальных оснований, которая заключается в~том, что, как правило, 
можно рассматривать несколько информационных пространств~\cite{8-gr} 
и~использовать для анализа различные каузальные основания различных исходных 
характеристик. 
  
{\small\frenchspacing
 {%\baselineskip=10.8pt
 %\addcontentsline{toc}{section}{References}
 \begin{thebibliography}{99}
  
\bibitem{1-gr}
Anomaly Detection at Multiple Scales (ADAMS). 2011. {\sf 
https://info.publicintelligence.net/DARPAADAMS.pdf}.
\bibitem{2-gr}
\Au{Memory A., Goldberg~H.\,G., Senator~T.\,E.} Context-aware insider threat detection~// Activity 
Context-Aware System Architectures: Papers from the AAAI 2013 Workshop, 2013. P.~44--47.
{\sf 
https://pdfs.semanticscholar.org/\linebreak 04aa/e6d97900ba62e90b07ac682fb7bd8c2e1029.pdf}.
\bibitem{3-gr}
\Au{Ruttenberg B.\,E., Blumstein~D., Druce~J., \textit{et al.}} 
Probabilistic modeling of insider threat detection systems~// Graphical 
models for security~/
Eds. P.~Liu, S.~Mauw, K.~\mbox{St{\!\ptb{\o}}len}.~--- Lecture notes in computer science  ser.~--- Springer, 2018. 
Vol.~10744. P.~91--98. doi:  
1007/978-3-319-74860-3\_6.

\bibitem{4-gr}
\Au{Rashid T., Agrafiotis~I., Nurse~J.\,R.\,C.} A~new take on detecting insider threats: Exploring the 
use of hidden Markov models~//  8th ACM CCS  Workshop (International) on Managing Insider Security 
Threats Proceedings.~--- ACM, 2016. P.~47--56.
\bibitem{5-gr}
\Au{Gheyas I.\,A., Abdallah~A.\,E.} Detection and prediction of insider threats to cyber security: 
A~systematic literature review and meta-analysis~// Big Data Anal., 2016. Vol.~1. No.\,6. P.~1--29. doi: 
10.1186/s41044-016-0006-0.

\bibitem{8-gr} %6
\Au{Грушо А.\,А., Забежайло~М.\,И., Смирнов~Д.\,В., Тимонина~Е.\,Е.} Модель множества 
информационных пространств в~задаче поиска инсайдера~// Информатика и~её применения, 
2017. Т.~11. Вып.~4. С.~65--69.


\bibitem{7-gr} %7
\Au{Грушо А.\,А., Грушо Н.\,А., Забежайло~М.\,И., Смирнов~Д.\,В., Тимонина~Е.\,Е.} 
Параметризация в~прикладных задачах поиска эмпирических причин~// Информатика и~её 
применения, 2018. Т.~12. Вып.~3. С.~62--66.

\bibitem{6-gr} %8
\Au{Grusho A., Grusho~N., Timonina~E.} Method of several information spaces for identification of 
anomalies~// 
Intelligent distributed computing XIII~/ Eds. I.~Kotenko, C.~Badica,V.~Desnitsky, D.~ElBaz, and 
M.~Ivanovic.~--- 
Studies in computational intelligence ser.~--- Springer, 2020. Vol.~868. P.~515--520.

\bibitem{9-gr}
\Au{Грушо А.\,А., Забежайло~М.\,И., Тимонина~Е.\,Е.} О~каузальной репрезентативности 
обучающих выборок прецедентов в~задачах диагностического типа~// Информатика и~ёе 
применения, 2020. Т.~14. Вып.~1. С.~80--86.
\bibitem{10-gr}
\Au{Грушо А.\,А., Грушо~Н.\,А., Тимонина~Е.\,Е.} Методы выявления <<слабых>> признаков 
нарушений информационной  
безопас\-ности~// Информатика и~её применения, 2019. Т.~13. Вып.~3. С.~3--8.
\bibitem{11-gr}
\Au{Ширяев А.\,Н.} Вероятность: В~2 кн.~--- 3-е изд.~--- М.: МЦНМО, 2004. 
521~с.
\bibitem{12-gr}
\Au{Грушо А.\,А., Тимонина~Е.\,Е.} Запреты в~дискретных ве\-ро\-ят\-ност\-но-ста\-ти\-сти\-че\-ских 
задачах~// Дискретная математика, 2011. Т.~23. №\,2. С.~53--58.
\bibitem{13-gr}
\Au{Grusho A., Kniazev~A., Timonina~E.} Detection of illegal information flow~// Computer network 
security~/ Eds. V.~Gorodetsky, I.~Kotenko, V.~Skormin.~--- Lecture notes in computer 
science ser.~--- Springer, 2005. Vol.~3685.  P.~235--244.
\bibitem{14-gr}
\Au{Забежайло М.\,И., Трунин~Ю.\,Ю.} К~проб\-ле\-ме доказательности медицинского 
диагноза: интеллектуальный анализ данных о~пациентах в~выборках ограниченного 
размера~// На\-уч\-но-тех\-ни\-че\-ская информация. Сер.~2: Информационные процессы 
и~системы, 2019. №\,12. C.~12--18. 
\end{thebibliography}

 }
 }

\end{multicols}

\vspace*{-12pt}

\hfill{\small\textit{Поступила в~редакцию 12.10.20}}

\vspace*{8pt}

%\pagebreak

%\newpage

%\vspace*{-28pt}

\hrule

\vspace*{2pt}

\hrule

%\vspace*{-2pt}

\def\tit{ON PROBABILISTIC ESTIMATES OF~THE~VALIDITY\\ OF~EMPIRICAL CONCLUSIONS}


\def\titkol{On probabilistic estimates of the validity of empirical conclusions}


\def\aut{A.\,A.~Grusho$^1$, M.\,I.~Zabezhailo$^2$, D.\,V.~Smirnov$^3$, 
and~E.\,E.~Timonina$^1$}

\def\autkol{A.\,A.~Grusho, M.\,I.~Zabezhailo, D.\,V.~Smirnov, and~E.\,E.~Timonina}

\titel{\tit}{\aut}{\autkol}{\titkol}

\vspace*{-11pt}


\noindent
$^1$Institute of Informatics Problems, Federal Research Center 
``Computer Sciences and Control'' of the 
Russian\linebreak
$\hphantom{^1}$Academy of Sciences;  
44-2~Vavilov Str., Moscow 119133, Russian Federation


\noindent
$^2$A.\,A.~Dorodnicyn Computing Center, Federal Research Center ``Computer Science and Control'' of 
the Russian\linebreak
$\hphantom{^1}$Academy of Sciences, 40~Vavilov Str., Moscow 119333, Russian Federation

\noindent
$^3$Sberbank of Russia, 19 Vavilov Str., Moscow 117999, Russian Federation


\def\leftfootline{\small{\textbf{\thepage}
\hfill INFORMATIKA I EE PRIMENENIYA~--- INFORMATICS AND
APPLICATIONS\ \ \ 2020\ \ \ volume~14\ \ \ issue\ 4}
}%
 \def\rightfootline{\small{INFORMATIKA I EE PRIMENENIYA~---
INFORMATICS AND APPLICATIONS\ \ \ 2020\ \ \ volume~14\ \ \ issue\ 4
\hfill \textbf{\thepage}}}

\vspace*{3pt} 



\Abste{The work focuses on some features of data analysis in insider search problems. The possibilities of 
using different approaches to describe the diagnosis of insider actions in the analysis of large empirical data are 
discussed. In tasks of this type, it is necessary to establish (predict, diagnose, etc.) the presence or the 
absence of target properties in any users from a given set. The assessment of the correctness of plausible 
reasoning is checked on the basis of estimates of the probabilities of the random appearance of the found laws 
in the simplest probabilistic models. The examples discussed show at what ratios of parameters it 
is possible to effectively identify correlations between events with which insiders can be identified. Two 
methods of controlling relations between parameters are indicated, allowing to obtain content information. The 
first method is based on dividing the observation period at the intervals during which the desired correlation 
may appear. The second method relates to the ways to reduce the set of users that could potentially become 
insiders, i.\,e., the authors are talking about the formation of clusters in which probabilistic estimates become 
operational. The desired relationships between the parameters for finding correlations can be determined using 
limit theorems in the series scheme.}

\KWE{hostile insider; causal analysis; probabilistic estimates of random appearance of properties}


\DOI{10.14357/19922264200401} 

%\vspace*{-20pt}

\Ack
\noindent
The paper was partially supported by RFBR, project No.\,18-29-03081.

%\vspace*{6pt}

  \begin{multicols}{2}

\renewcommand{\bibname}{\protect\rmfamily References}
%\renewcommand{\bibname}{\large\protect\rm References}

{\small\frenchspacing
 {%\baselineskip=10.8pt
 \addcontentsline{toc}{section}{References}
 \begin{thebibliography}{99}

\bibitem{1-gr-1}
Anomaly Detection at Multiple Scales. 2011. Available at: 
{\sf https://info.publicintelligence.net/DARPAADAMS.pdf} (accessed October~11, 2020).
\bibitem{2-gr-1}
\Aue{Memory, A., H.\,G.~Goldberg, and T.\,E.~Senator.} 2013. Context-aware insider threat detection. 
\textit{Activity 
Context-Aware System Architectures: Papers from the AAAI 2013 Workshop}. 44--47. Available at: {\sf 
https://pdfs.\linebreak  semanticscholar.org/04aa/e6d97900ba62e90b07ac682fb\linebreak 7bd8c2e1029.pdf} (accessed 
August~13, 2020).
\bibitem{3-gr-1}
\Aue{Ruttenberg, B.\,E., D.~Blumstein, J.~Druce, \textit{et al.}} 2018. Probabilistic modeling of insider threat detection systems. 
 \textit{Graphical models for security.} Eds. P.~Liu, S.~Mauw, and 
K.~\mbox{St{\!\ptb{\o}}len}. Lecture notes in computer science ser. Springer. 10744:91--98. doi: 1007/978-3-319-74860-
3\_6.
\bibitem{4-gr-1}
\Aue{Rashid, T., I.~Agrafiotis, and J.\,R.\,C.~Nurse.} 2016. A~new take on detecting insider threats: 
Exploring the use of hidden Markov models. \textit{8th ACM CCS Workshop (International) on Managing 
Insider Security Threats Proceedings}. ACM. 47--56.
\bibitem{5-gr-1}
\Aue{Gheyas, I., and A. Abdallah.} 2016. Detection and prediction of insider threats to cyber security: 
A~systematic literature review and  
meta-analysis. \textit{Big Data Anal.} 1(6):1--29. doi: 10.1186/s41044-016-0006-0.
\bibitem{8-gr-1} %6
\Aue{Grusho, A.\,A., M.\,I. Zabezhailo, D.\,V.~Smirnov, and E.\,E.~Timonina.} 2017. Model' mnozhestva 
informatsionnykh prostranstv v~zadache poiska insaydera [The model of the set of information spaces in the 
problem of insider detection]. \textit{Informatika i~ee Primeneniya~--- Inform. Appl.} 11(4):65--69.

\bibitem{7-gr-1}
\Aue{Grusho, A.\,A., N.\,A. Grusho, M.\,I.~Zabezhailo, D.\,V.~Smirnov, and E.\,E.~Timonina.} 2018. 
Parametrizatsiya v~prikladnykh zadachakh poiska empiricheskikh prichin [Parametrization in applied 
problems of search of the empirical reasons]. \textit{Informatika i~ee Primeneniya~--- Inform. Appl.} 
12(3):62--66.

\bibitem{6-gr-1} %8
\Aue{Grusho, A., N. Grusho, and E.~Timonina.} 2020. Method of several information spaces for 
identification of anomalies. \textit{Intelligent distributed computing XIII.} Eds. I.~Kotenko, 
C.~Badica,V.~Desnitsky, D.~ElBaz, and M.~Ivanovic.
Studies in computational intelligence ser. Springer. 868:515--520.

\bibitem{9-gr-1}
\Aue{Grusho, A.\,A., M.\,I.~Zabezhailo, and E.\,E.~Timonina.} 2020. O~kauzal'noy reprezentativnosti 
obuchayushchikh vyborok pretsedentov v~zadachakh diagnosticheskogo tipa [On causal representativeness 
of training samples of precedents in diagnostic type tasks]. \textit{Informatika i~ee Primeneniya~--- 
Inform. Appl.} 14(1):80--86.
\bibitem{10-gr-1}
\Aue{Grusho, A.\,A., N.\,A. Grusho, and E.\,E.~Timonina.} 2019. Metody vyyavleniya ``slabykh'' 
priznakov narusheniy informatsionnoy 
bezopas\-nosti [Methods of identification of ``weak'' signs of violations of information security]. 
\textit{Informatika i~ee Primeneniya~--- Inform. Appl.} 13(3):3--8.
\bibitem{11-gr-1}
\Aue{Shiryaev, A.\,N.} 2004. \textit{Veroyatnost'} [Probability]. Moscow: MTsNMO. 521~p.
\bibitem{12-gr-1}
\Aue{Grusho, A.\,A., and E.\,E.~Timonina.} 2011. Prohibitions in discrete probabilistic statistical problems. 
\textit{Discrete Mathematics Applications}  
21(3):275--281.
\bibitem{13-gr-1}
\Aue{Grusho, A., A.~Kniazev,  and E.~Timonina.} 2005. Detection of illegal information flow. 
\textit{Computer network security}. Eds. V.~Gorodetsky, I.~Kotenko, and V.~Skormin. 
Lecture notes in computer science ser. Springer. 3685:235--244.
\bibitem{14-gr-1}
\Aue{Zabezhailo, M.\,I., and Y.\,Y.~Trunin.} 2019. On the problem of medical diagnostic evidence: Intelligent 
analysis of empirical data on patients in samples of limited size. \textit{Automatic Documentation Mathematical
Linguistics} 53:322--328. doi: 10.3103/S0005105519060086.
\end{thebibliography}

 }
 }

\end{multicols}

\vspace*{-3pt}

\hfill{\small\textit{Received October 12, 2020}}

%\pagebreak

%\vspace*{-24pt}

\Contr

\noindent
\textbf{Grusho Alexander A.} (b.\ 1946)~--- Doctor of Science in physics and mathematics, professor, 
principal scientist, Institute of Informatics Problems, Federal Research Center ``Computer Sciences and 
Control'' of the Russian Academy of Sciences; 44-2~Vavilov Str., Moscow 119133, Russian Federation;  
\mbox{grusho@yandex.ru}

\vspace*{3pt}

\noindent
\textbf{Zabezhailo Michael I.} (b.\ 1956)~---Doctor of Science in physics and mathematics, principal 
scientist, A.\,A.~Dorodnicyn Computing Center, Federal Research Center ``Computer Science and Control'' 
of the Russian Academy of Sciences, 40~Vavilov Str., Moscow 119333, Russian Federation; 
\mbox{m.zabezhailo@yandex.ru}


\vspace*{3pt}

\noindent
\textbf{Smirnov Dmitry V.} (b.\ 1984)~--- business partner for IT Security Department of Sberbank of 
Russia, 19~Vavilov Str., Moscow 117999, Russian Federation; \mbox{dvlsmirnov@sberbank.ru}

\vspace*{3pt}

\noindent
\textbf{Timonina Elena E.} (b.\ 1952)~--- Doctor of Science in technology, professor, leading scientist, 
Institute of Informatics Problems, Federal Research Center ``Computer Science and Control'' of the Russian 
Academy of Sciences; 44-2~Vavilov Str., Moscow 119133, Russian Federation; \mbox{eltimon@yandex.ru}

%Shorgin Sergey Ya. (b.\ 1952)~---
% Doctor of Science in physics and mathematics, professor, principal 
%scientist, Institute of Informatics Problems,  
%Federal Research Center ?Computer Sciences and Control? of the Russian Academy of Sciences, 44-2 
%Vavilov Str., Moscow 119133, Russian  
%Federation; sshorgin@ipiran.ru


\label{end\stat}

\renewcommand{\bibname}{\protect\rm Литература} 