%без проверки Юлей
\renewcommand{\d}{\mathrm{d}}
%\renewcommand{\i}{\mathrm{i}}
%\def\ld{\ldots}
%\def\cd{\cdots}
%\def\b{\overline b}

\def\stat{razumcik}

\def\tit{СТАЦИОНАРНЫЕ ХАРАКТЕРИСТИКИ СИСТЕМЫ Geo$/G/1/\infty$ 
С~НЕОРДИНАРНЫМ ВХОДЯЩИМ ПОТОКОМ, УПРАВЛЯЮЩИМ~РАЗМЕРОМ ОЧЕРЕДИ$^*$}

\def\titkol{Стационарные характеристики системы Geo$/G/1/\infty$ 
с~неординарным входящим потоком}
%, управляющим размером очереди}

\def\aut{С.\,И.~Матюшенко$^1$, Р.\,В.~Разумчик$^2$}

\def\autkol{С.\,И.~Матюшенко, Р.\,В.~Разумчик}

\titel{\tit}{\aut}{\autkol}{\titkol}

\index{Матюшенко С.\,И.}
\index{Разумчик Р.\,В.}
\index{Matyushenko S.\,I.}
\index{Razumchik R.\,V.}

{\renewcommand{\thefootnote}{\fnsymbol{footnote}} \footnotetext[1]
{Исследование выполнено при финансовой поддержке РФФИ  
(проект 20-07-00804) и~в~соответствии с~программой Московского центра 
фундаментальной и~прикладной математики.}}


\renewcommand{\thefootnote}{\arabic{footnote}}
\footnotetext[1]{Российский
университет дружбы народов, matyushenko\_si@pfur.ru}
\footnotetext[2]{Институт проблем информатики Федерального 
исследовательского
центра <<Информатика и~управ\-ле\-ние>> Российской академии наук,
\mbox{rrazumchik@ipiran.ru}}

%\vspace*{-10pt}


\Abst{Рассматривается функционирующая в~дискретном времени
система массового обслуживания (СМО) с~одним прибором,
очередью неограниченной емкости и~неординарным геометрическим потоком 
заявок.
В~системе реализован специальный механизм управления очередью:
в~момент поступления в~систему новой группы заявок ее размер сравнивается 
с~текущим общим числом заявок в~системе
и,~если число заявок в~новой группе превышает общее число заявок 
в~системе, новая группа целиком принимается в~систему,
вытесняя при этом все прежде находившиеся в~ней заявки;
в~противном случае новая группа покидает систему, не оказывая на нее 
никакого воздействия.
Заявки обслуживаются прибором по одной. В предположении, что заявки 
в~группе независимы, а
распределения чисел заявок в~группе и~времени обслуживания являются 
произвольными дискретными,
найдены основные стационарные характеристики функционирования.}

\KW{дискретное время; неординарный поток; управление очередью; выходящий 
поток}

\DOI{10.14357/19922264200404} 
 
\vspace*{2pt}


\vskip 10pt plus 9pt minus 6pt

\thispagestyle{headings}

\begin{multicols}{2}

\label{st\stat}


\section{Введение}

\vspace*{-5pt}

Дисциплины обслуживания очередей, которые позволяют
повысить эффективность работы %систем массового обслуживания 
СМО
путем использования доступной информации (известной либо точно, либо 
приближенно)
о~размерах (временах обслуживания и~т.\,п.)\
поступающих в~них заявок, продолжают оставаться предметом активных
научных исследований~\cite{i2, i3, i4, i5, i6}.
При этом внимание часто сосредоточено вокруг наиболее известной
из всех специальных дисциплин обслуживания~---
дисциплины преимущественного обслуживания заявки минимальной
остаточной длины (shortest remaining time first, SRPT).

В~недавней работе~\cite{i1} авторами рассмотрена новая однолинейная СМО 
с~групповым потоком и~специальным механизмом обработки очереди,
по которому поступающая группа всегда вытесняет из системы
все находящиеся в~ней заявки, если размер группы достаточно велик.
Побудительным мотивом\footnote[3]{Появление описанного механизма, по-видимому,
также связано с~результатами работ~\cite{i7,i8}, в~которых показано, что
одна его более простая разновидность приводит к~мультипликативному виду
совместного стационарного распределения в~соответствующим образом
образованных сетях массового обслуживания. Отметим, что сети из СМО, 
рассмотренных
в~\cite{i1} и~в~этой работе, таким свойством уже не обладают.}
к~изучению этого механизма (в отличие от SRPT) послужил
поиск путей максимизации загрузки системы.

В предположениях, что поток групп пуассоновский, заявки в~группе
независимы, число заявок в~группе имеет произвольное,
а~времена обслуживания~--- экспоненциальное распределение,
в~\cite{i1} найде\-ны с~помощью метода обращения времени основные 
стационарные характеристики сис\-те\-мы.
В~данной статье рассматривается аналогичная~\cite{i1} СМО, но 
функционирующая в~дискретном времени и~при этом
в~более общих предположениях о~распределении времени обслуживания (оно 
допускается произвольным дискретным).
Предложен основанный на вероятностных соображениях из~\cite{n4}
метод\footnote[4]{Здесь же необходимо отметить,
что в~ряде частных случаев (например, для геометрического
распределения времени обслуживания) для вывода стационарного 
распределения,
как и~в~непрерывном времени, может быть применен метод обращения 
времени.}
нахождения совместного стационарного распределения числа заявок в~системе 
и~остаточных длин заявок в~очереди.
Стандартными методами изучен и~ряд других стационарных характеристик: 
период занятости,
время пребывания заявки в~системе, выходящий поток потерянных заявок.

Прежде чем переходить к~подробному описанию системы, отметим, что 
отличительная особенность рассмотренной СМО заключается в~том, что 
управ\-ля\-ющие размером очереди решения принимаются по результатам
сравнения не остаточных времен обслуживания (см., например,~\cite{tata,i10,i11}), 
а~остаточных размеров групп заявок. Несмотря на то что в~такой СМО 
поступление заявок
может привести к~потере уже находящихся в~системе заявок,
она не относится к~типу СМО с~отрицательными за\-яв\-ка\-ми/сиг\-на\-ла\-ми 
(см., например,~\cite{i12}).



\section{Описание системы}

Рассматривается функционирующая в~дискретном времени\footnote{Дискретное 
время вводится обычным образом
(см., например, \cite{distime}).} однолинейная СМО
%система массового обслуживания 
с~очередью неограни\-чен\-ной емкости,
в~которую поступает неординарный геометрический поток заявок,
определяемый следующим образом. На каждом такте
(далее будем называть тактом как интервал времени
между соседними изменениями состояния сис\-те\-мы, так
и~сами моменты, в~которые происходят эти изменения)
с~вероятностью~$a$ приходит
группа заявок случайного размера, не зависящего
от всего процесса функционирования сис\-те\-мы. При
этом в~каждой поступившей группе имеется $i\hm\ge 1$ заявок 
с~вероятностью~$l_i$. 
Заявки обслуживаются прибором по одной,
причем время обслуживания заявки становится известным
в~момент ее поступления на прибор.
Распределение времени обслуживания заявки является произвольным
дискретным с~вероятностью~$b_i$, $i\hm\ge0$, того, что
обслуживание заявки продлится~$i$~так\-тов (предполагается, что
$b_0\hm=0$).

% (предполагается, что $l_i>0$ при всех $i \ge 1$) и~$b_i>0$ при всех $i \ge 1$

Будем использовать следующие обозначения:
\begin{description}
\item[\,]
$\overline{a}=1-a$ --- вероятность непоступления заявки на такте;
\item[\,]
$B_i= \sum\nolimits_{j=i}^\infty b_j$, $i \hm\ge 0$,~--- вероятность 
того,
что обслуживание заявки продлится не менее~$i$~тактов;
\item[\,]
$\mathsf{E}b^k=\sum\nolimits_{i=1}^\infty i^k b_i$~--- $k$-й момент 
времени обслуживания;
\item[\,]
$L_i =\sum\nolimits_{j=i}^\infty l_j$, $i\hm\ge 1$,~---
вероятность того, что в~поступившей группе окажется не менее~$i$~заявок.
Очевидно, $L_1=1$;  
\item[\,]

$\mathsf{E}l^k=\sum\nolimits_{i=1}^\infty i^k l_i$~--- $k$-й момент 
размера группы;
\item[\,]
$\beta(z)=\sum\nolimits_{j=1}^\infty z^j b_j$~--- производящая функция 
(ПФ)
времени обслуживания заявки.
\end{description}


В системе реализован следующий механизм управления очередью. В~момент
поступления в~сис\-те\-му новой группы заявок ее размер $x$ сравнивается
с текущим общим числом заявок в~системе~$y$. Та из групп
заявок, длина которой больше, остается в~системе, а~другая
покидает систему. Другими словами, если $x\hm>y$, то все~$y$~заявок 
мгновенно
уходят из системы; новая группа заявок размера~$x$
целиком помещается в~очередь, и~одна заявка из группы немедленно занимает 
прибор.
Если же $x \hm\le y$, то поступающая группа заявок теряется,
не оказывая на систему никакого воздействия.

Примем, что все изменения состояния СМО происходят в~конце такта 
в~следующем порядке\footnote{В зарубежной литературе это схема EAS-IA 
(см., например, \cite[с.~2--3]{nobel}).}:
\begin{itemize}
\item если на этом такте завершилось обслуживание заявки на приборе, то 
она покидает систему
и~на прибор сразу же поступает следующая заявка из очереди;
\item затем с~вероятностью~$a$ в~систему поступает группа заявок и, если 
система оказалась непустой, происходит потеря либо поступившей группы, 
либо всех тех заявок, которые находились в~системе (до момента 
поступления).
\end{itemize}

Далее будем предполагать, что $L_i\hm>0$ и~$B_i\hm>0$ при всех~$i$ 
и~выполнено условие~\eqref{stab} (см.\ разд.~4),
необходимое и~достаточное для существования стационарного режима.
\vspace*{-.5pt}

\section{Период занятости}

Рассмотрим случайный процесс $\{ \eta(t)\hm=(\nu(t),\xi(t)), \ t \hm\ge 
0\}$,
где $\nu(t)$~--- общее число заявок в~системе,
а $\xi(t)$~--- остаточное время обслуживания (далее~--- длина) заявки на 
приборе
непосредственно после такта~$t$. При $\nu(t)\hm=0$
координата~$\xi(t)$ не определяется.
Процесс ${\{\eta(t), \ t\ge0\}}$ является \mbox{цепью} Маркова, причем множество 
ее состояний~$\mathcal{X}$ имеет вид:
$$\mathcal{X}\hm=\{0\} \bigcup \{ (n,i), \ n \hm\ge 1, i \hm\ge 1 \},
$$
где $n$~--- число заявок в~системе; $i$~--- остаточная длина заявки
на приборе.

Пусть после очередного такта в~системе оказалось $n \hm\ge 1$ заявок 
и~обслуживание заявки только началось.
Обозначим через~$\mathcal{U}_{n}(z)$, $0\hm<z\hm\le 1$, ПФ числа тактов 
до того момента, когда в~системе впервые окажется $n\hm-1$ заявок.
Если в~момент начала функционирования в~системе находится $n \hm\ge 1$ 
заявок, то
распределение ее периода занятости (ПЗ) в~терминах ПФ имеет вид 
$\prod\nolimits_{j=1}^n \mathcal{U}_{j}(z)$;
иначе~--- $\sum\nolimits_{n=1}^\infty l_n \mathcal{U}_{n}(z)$.
Воспользовавшись формулой полной вероятности,
получаем систему уравнений для~$\mathcal{U}_{n}(z)$:
%\vspace*{-.5pt}

%\pagebreak

\noindent
\begin{multline}
\label{eq1}
\mathcal{U}_{n}(z)
=\mathcal{D}_{n}(z)
\mathcal{U}_{n-1}(z)
+\mathcal{E}_{n}(z)
\mathcal{U}_{n}(z)+{}\\
{}+ \mathcal{F}_{n}(z)
\sum\limits_{j=n+1}^\infty
al_j \mathcal{U}_{j}(z),
\enskip n \ge 1.
\end{multline}
Здесь используется соглашение $\mathcal{U}_{0}(z) \hm\equiv 1$ и
обозначения:
$$
\mathcal{D}_{n}(z)
=
\fr{z_{n-1}}{z_{n}}\,
\beta\left( z z_n \right);
\quad
\mathcal{E}_{n}(z)
=
\fr{al_{n}}{z_{n}}\,
\beta\left( z z_n \right);
$$$$
\mathcal{F}_{n}(z)
=
z\fr{ 1- \beta\left( z z_n \right)}{1-z z_n}\,,
$$
где $z_n=1-aL_{n+1}$.

Заметим, что 
$$\mathcal{D}_{n}(z) \hm+ \mathcal{E}_{n}(z)\hm + 
\mathcal{F}_{n}(z) \fr{1\hm-z z_n}{z}\hm=1\,.$$
Системе~\eqref{eq1} можно придать следующий вид:
$$
\mathcal{U}_{n}(z)=
\sum\limits_{j=1}^\infty \mathcal{T}_{nj}(z) \mathcal{U}_{j}(z)
+\mathcal{B}_n(z), \enskip n \ge 1\,,
$$

\noindent 
где $\mathcal{B}_1(z)\hm=\mathcal{D}_{1}(z)$, $\mathcal{B}_n(z)\hm=0$, 
$n\hm\ge 2$;
$\mathcal{T}_{nj}(z)$~--- соответствующим образом подобранные 
по~\eqref{eq1} коэффициенты.
Поскольку при $0<z<1$ в~каждой строке $\sum\nolimits_{j=1}^\infty 
\mathcal{T}_{nj}(z)\hm<1$
и, очевидно, свободные члены удовлетворяют условию
$\mathcal{B}_n(z) \hm\le K (1\hm-\sum\nolimits_{j=1}^\infty 
\mathcal{T}_{nj}(z))$
при некоторой постоянной $K\hm>0$,
то система~\eqref{eq1} имеет ограниченное решение, которое
может быть найдено методом последовательных приближений
(см., например,~\cite[теоремы~Ia,~IVa]{kry} или~\cite[теорема~1]{wil}).
При $z\hm=1$ единственное решение\footnote{Ввиду громоздкости выкладок 
лишь заметим, что это можно показать и~прямыми вычислениями,
если воспользоваться представлением решения~\eqref{eq1}, данным 
в~\cite[соотн.~(7)]{Car},
и выписать явный вид входящих в~него слагаемых.}~\eqref{eq1}~--- это 
$\mathcal{U}_{n}(1)\hm=1$ при всех~$n$, если параметры системы таковы, 
что
стационарный режим функционирования существует (т.\,е. 
выполняется~\eqref{stab}, см.\ разд.~4).


\section{Стационарное распределение очереди}

Введем обозначения:
\begin{description}
\item[\,]
$P_0=\lim\nolimits_{t \rightarrow \infty } {\sf P} (\nu(t)\hm=0)$~---
стационарная вероятность того, что непосредственно после очередного такта 
система будет
пуста;
\item[\,]
$p_{ni}=\lim\nolimits_{t \rightarrow \infty } {\sf P} (\nu(t)=n, 
\xi(t)\hm=i)$, $n \hm\ge 1$, $i \hm\ge 1$,~---
стационарная вероятность того, что непосредственно после очередного такта 
в~системе будет $n$
заявок и~до окончания обслуживания заявки на приборе останется $i$ 
тактов.
\end{description}

Положим
$$
P_n=\sum\limits_{i=1}^\infty p_{ni}, \enskip n \ge 1; \enskip {\overline 
P}_n=\sum\limits_{i=0}^n P_{i}, \enskip
 n \ge 0\,.
$$

Из системы уравнений равновесия (СУР) стандартным образом находится
двойная ПФ $\mathcal{P}(u,v)\hm=P_0+\sum\nolimits_{n=1}^\infty 
\sum\nolimits_{i=1}^\infty u^n v^i p_{ni}$,
$0\hm<u,v\hm\le 1$:
\begin{multline*}
%\label{eq2}
\mathcal{P}(u,v)=
\sum\limits_{n=1}^\infty \sum\limits_{i=1}^\infty p_{ni}\times{}\\
{}\times
\left( (1-aL_{n+1}) u^n v^{i-1} + \beta(v) \sum\limits_{j=n+1}^\infty a 
l_j u^j \right)+{}\\
{}+\sum\limits_{n=1}^\infty p_{n1}
\left( (1-aL_{n}) u^{n-1} \left(\beta(v)-u \right) +{}\right.\\
\!\!\left.{}+ a l_n u^n (\beta(v)\!-\!1) \right)+
P_0\! \left(\!\beta(v) \sum\limits_{j=1}^\infty a l_j u^j+\!1\!-a \beta(v)\!\!\right)\!\!.\!\!
\end{multline*}

\noindent 
Однако ее вид малопригоден для проведения анализа.
Поэтому поступим следующим образом.
Введем новую СМО с~конечным числом $n$ мест для ожидания,
отличающуюся от исходной только тем, что
если в~очереди находится $n$ заявок и~поступает
новая группа заявок размера больше $n$,
то все находящиеся в~системе заявки покидают ее,
а~(любые)~$n$~заявок из новой группы принимаются в~систему.
Воспользовавшись приемом, введенном в~\cite{n4} и~подробно изложенном 
в~\cite{distime},
можно показать, что стационарные вероятности состояний в~исходной и~новой 
СМО
отличаются лишь на постоянный множитель.
Это дает возможность записать следующую СУР:
\begin{multline}
\label{eq3r}
p_{ni}=(1-aL_{n+1})
p_{n,i+1}
+a l_n b_i p_{n1}
+a l_n b_i
{\overline P}_{n-1}+{}\\
{}+ {\overline P}_n
\sum\limits_{j=n+1}^{\infty} a l_j q_{n,i,j},
\enskip n \ge 1\,, \enskip i \ge 1\,,
\end{multline}

\noindent  
где $q_{n,i,j}$~--- условная вероятность того, что, когда в~системе 
впервые
окажется $n$ заявок, остаточное время обслуживания заявки
на приборе будет равно~$i$ при условии, что
после очередного такта в~сис\-те\-ме оказалось $j \hm\ge n\hm+1$ заявок
и~обслуживание заявки на приборе только началось.
Если система находится в~стационарном режиме, то $q_{n,i,j}\hm=b_i$,
$n\hm\ge 1$, $j \hm\ge n\hm+1$.

Переходя к~ПФ $\mathcal{P}_n(z)\hm=\sum\nolimits_{i=1}^\infty z^i 
p_{ni}$, $0 \hm< z \hm\le 1$,
из~\eqref{eq3r} получаем:
\begin{multline}
\label{eq5}
P_n(z) \fr{z - z_n }{z}=
\left(a l_n \beta(z) - z_n \right) p_{n1}+{}\\
{}+a L_n \beta(z) {\overline P}_{n-1}
+a L_{n+1} \beta(z) P_n, \enskip n \ge 1\,.
\end{multline}

\noindent  
Подставляя $z\hm=1$, находим 
$${p_{n1}\hm={\overline P}_{n-1}\,\fr{1\hm-z_{n-1}}{z_{n-1}}}.
$$
Теперь, воспользовавшись теоремой Руше, с~учетом найденного вида~$p_{n1}$ 
из~\eqref{eq5} имеем:
%\begin{equation}
\begin{multline*}
%\label{eq6}
P_n
=c_n {\overline P}_{n-1}\,,\\
c_n=\fr{z_{n}}{1-z_{n}}\,
\fr{1-z_{n-1}}{z_{n-1}}\,
\fr{1- \beta(z_n)}{\beta(z_n)}\,,
\enskip n \ge 1\,.
\end{multline*}
%\end{equation}

\noindent 
Отсюда, с~учетом соотношения ${\overline P}_n\hm={\overline P}_{n-
1}\hm+P_n$, следует, что 
$${{\overline P}_{n} \hm= P_0 \prod\limits_{i=1}^n
(1+c_i)},\enskip {n \hm\ge 1}\,.
$$
Используя теперь условие нормировки
$\lim\limits_{n \rightarrow \infty } {\overline P}_n=1$,
окончательно получаем, что
\begin{equation}
\label{sn4-1}
\left.
\begin{array}{rl}
P_0 &=\left( \prod\limits_{i=1}^\infty (1+c_i) \right)^{-1};\\
P_n &= \fr{c_n}{\prod\nolimits_{i=n}^\infty
(1+c_i)}\,, \enskip
 n \ge 1\,.
\end{array}
\right\} 
\end{equation}

Из эргодической теоремы Фостера следует, что необходимым и~достаточным
условием существования стационарного режима является сходимость 
произведения
$\prod\nolimits_{i=1}^\infty (1+c_i)$, которая эквивалентна условию
\begin{equation}
\label{stab}
\sum\limits_{i=1}^\infty
\fr{z_{i}}{1-z_{i}}\,
\fr{1-z_{i-1}}{z_{i-1}}\,
\fr{1- \beta(z_i)}{\beta(z_i)}
< \infty\,.
\end{equation}

\noindent 
Для выполнения \eqref{stab} достаточно\footnote{Действительно,
так как $\beta(z_n) \ge 1- (1-z_n) \mathsf{E}b$ 
и~$\beta(z_n)\hm<\beta(z_{n+1})$,
то $c_n\hm\le a L_n \mathsf{E}b/(\beta(z_1)(1\hm- a L_n))$,
а ряд $\sum\nolimits_{i=1}^\infty aL_i /(1\hm- a L_i)$
сходится, если $\sum\nolimits_{i=1}^\infty aL_i \hm= a \mathsf{E}l 
\hm<\infty$.}, чтобы $\mathsf{E}b \, \mathsf{E}l \hm< \infty$.
При расчете моментов стационарного распределения по~\eqref{sn4-1}
необходимо быть уверенным, что соответствующие ряды сходятся;
достаточным условием существования $\mathsf{E}\nu^k$
является существование соответствующего момента размера группы.
Для расчета же совместного
стационарного распределения числа заявок в~системе и~остаточного
времени обслуживания заявки на приборе можно воспользоваться
формулой\footnote{Здесь и~далее используется соглашение
$\sum\nolimits_{j=1}^0\hm=0$.}:
\begin{multline*}
p_{ni}={\overline P}_{n-1}
\left(\fr{1-z_{n-1}}{z_{n-1} z_n^{i-1}}-
\vphantom{\sum\limits_{j=1}^{i-1}} 
{}\right.\\
\left.{}-\left(
\fr{a l_n}{z_{n-1}}
+aL_{n+1} (1+c_n)
\right)
\sum\limits_{j=1}^{i-1}
\fr{b_{i-j}}{z_n^{j}}
\right)\!, \enskip n,i \ge 1\,,
\end{multline*}

\noindent  
которая получается путем обращения ПФ~\eqref{eq5}. Отсюда, поскольку 
времена обслуживания заявок в~группе
предполагаются независимыми, немедленно
следует совместное стационарное распределение общего числа
заявок в~системе, остаточного времени обслуживания
заявки на приборе и~каждой заявки в~очереди.

\section{Некоторые характеристики производительности}

Остановимся на выводе формул для вероят\-ностей потери заявки.
Обозначим через~$\pi_1$ и~$\pi_2$ соответственно вероятность
потери произвольной заявки при поступлении и~во время пребывания 
в~системе.
Для этого необходимо перейти от стационарных вероятностей~$P_n$
по тактам к~стационарным вероятностям по моментам
поступления заявок в~систему (которые будем обозначать~$P^*_n$),
а~также зафиксировать порядок выбора заявок на обслуживание из очереди.

Нетрудно видеть, что 
$$P^*_0\hm=P_0\hm+p_{11}\,;\quad P^*_n\hm=P_n\hm-
p_{n1}\hm+p_{n+1,1}\,,\enskip n \hm\ge 1\,.
$$
Поскольку случайно выбранная заявка с~вероятностью $kl_k/\mathsf{E}l$
принадлежит группе размера~$k$, по формуле полной вероятности
получаем\footnote{А вероятность потери
поступающей группы равна $\sum\nolimits_{n=1}^\infty P^*_n (1-L_{n+1})$.}: 
$$\pi_1=\sum\limits_{n=1}^\infty P^*_n \sum\limits_{j=1}^n \fr{j l_j }{\mathsf{E}l}\,.
$$
Предположим, что заявки обслуживаются из очереди в~порядке поступления.
Обозначим через $\pi_{2,k,j}$, $k\hm\ge 1$, $1\hm \le j \hm\le k$, 
условную вероятность того, что заявка будет потеряна, при условии что 
она принята в~сис\-те\-му в~группе размера~$k$ и
оказалась в~группе на $j$-м месте. Величины~$\pi_{2,k,j}$ могут быть
вычислены рекуррентно по формулам:
\begin{align*}
\pi_{2,k,1} &= 1- \fr{\beta(z_k)}{z_k}, \enskip k \ge 1\,;\\
\pi_{2,k,j} &=\pi_{2,k,1}+ (1-\pi_{2,k,1})(aL_k + z_{k-1} \pi_{2,k-1,j-1}),\\ 
&\hspace*{42mm}
 1 \le j \le k\,, \enskip k \ge 2\,.
\end{align*}

\noindent 
Усредняя $\pi_{2,k,j}$ по распределению
размера принятой в~систему группы, содержащей случайно выбранную заявку,
и~предполагая, что, поступая в~группе размера~$k$, заявка
может равновероятно оказаться на любом из $k$ мест, находим:
$$
\pi_2= \sum\limits_{n=1}^\infty 
\fr{l_n {\overline P}^*_{n-1} }{\sum\nolimits_{k=1}^\infty k l_k 
{\overline P}^*_{k-1} }
\sum\limits_{j=1}^n \pi_{2,n,j},
\ \ {\overline P}^*_n=\sum\limits_{i=0}^n P^*_{i}.
$$

Остановимся теперь на стационарных распределениях
времен пребывания в~системе
обслуженной и~потерянной заявки.
Обозначим через~$V_{1,j,k}(z)$ ПФ условных вероятностей того, что
заявка будет обслужена и~время ее пребывания в~системе равно~$i$~тактам,
при условии что она принята в~систему в~группе размера~$k$ и
оказалась в~группе на $j$-м месте. При $j\hm=1$ заявка будет обслужена,
если за время ее пребывания на приборе в~систему не поступит
группа размера больше~$k$. Поэтому 
$$V_{1,1,k}(z)\hm=\sum\limits_{j=1}^\infty z^j b_j z_k^{j-1}\hm=
\fr{\beta(z z_k)}{z_k}\,.
$$
Если выделенная заявка оказалась не на первом месте в~группе, то
ее время пребывания зависит от времени обслуживания находящихся перед ней 
заявок.
Так как эти времена по предположению независимы, то в~терминах ПФ имеем: 
$$V_{1,j,k}(z)\hm= V_{1,1,k}(z)z_{k-1}V_{1,j-1,k-1}(z)\,,\enskip
2 \hm\le j \hm\le k\,.
$$
Воспользовавшись теперь формулой полной вероятности, получаем следующее
выражение для ПФ~$V_1(z)$ стационарного распределения времени пребывания
обслуженной заявки в~системе:
\begin{equation}
\label{v1}
V_1(z)=
\sum\limits_{n=1}^\infty \fr {l_n {\overline P}^*_{n-1} 
}{\sum\nolimits_{k=1}^\infty k l_k {\overline P}^*_{k-1} }
\sum\limits_{j=1}^n V_{1,n,j}(z).
\end{equation}

\noindent 
Вводя $V_{2,j,k}(z)$~--- ПФ условных вероятностей того, что
заявка не будет обслужена и~время ее пребывания в~системе равно $i$ 
тактам,
при условии что она принята в~систему в~группе размера~$k$ и
оказалась в~группе на $j$-м месте,~--- 
и~рассуждая аналогичным образом, нетрудно по формуле полной вероятности
получить следующие соотношения:
\begin{align*}
V_{2,1,k}(z)&=\fr{1-z_k}{z_k}\,\fr { z z_k - \beta(z z_k)}{1- z z_k}\,;\\
V_{2,j,k}(z)&=V_{2,1,k}(z)+V_{1,1,k}(z)
\left(1-z_{k-1} +{}\right.\\
&\hspace*{12mm}\left.{}+ z_{k-1} V_{2,j-1,k-1}(z)\right), \enskip 2 \le j \le k\,.
\end{align*}

\noindent Безусловная ПФ~$V_2(z)$ стационарного распределения времени 
пребывания в~системе
принятой, но в~итоге потерянной заявки рассчитывается по 
формуле~\eqref{v1}
с~заменой~$V_{1,n,j}(z)$ на $V_{2,n,j}(z)$.


\section{Выходящий поток потерянных заявок}

При изучении рассмотренной системы в~связке с~другими СМО
важны характеристики выходящего из нее потока потерянных заявок.
В~общем случае, очевидно, он не является ординарным.
Не является он и~геометрическим:
числа заявок, покидающих систему на соседних интервалах,
представляют собой зависимые случайные величины.


Рассмотрим последовательные моменты~$\tau^-_n$,\linebreak $n\hm\ge 1$,
потерь заявок и~введем вложенную цепь Маркова
$\nu_n\hm=\nu(\tau^-_n)$~--- общее число заявок в~системе
непосредственно после момента~$\tau^-_n$.
Положим $l_n\hm=\tau^-_{n+1}-\tau^-_n$
и~обозначим через ${h_{i,t_1,t_2}(n)\hm={\sf P} (\nu_n\hm=i, l_n\hm=t_1, 
l_{n+1}\hm=t_2)}$, $i, t_1,t_2 \hm\ge 1$,
вероятность того, что после $n$-й потери общее число заявок в~системе 
будет равно~$i$ и~длины интервалов между последующими двумя потерями 
равны~$t_1$ и~$t_2$ тактам соответственно. 
Положим
\begin{align*}
p_i^-(n)&=\sum\limits_{t_1=1}^\infty \sum\limits_{t_2=1}^\infty h_{i,t_1,t_2}(n)\,;\\
h_{t}(n)&=\sum\limits_{i=1}^\infty\sum\limits_{t_2=1}^\infty 
h_{i,t,t_2}(n)\,;\\
h_{t_1,t_2}(n)&=\sum\limits_{i=1}^\infty h_{i,t_1,t_2}(n)\,.
\end{align*}
При выполнении условия существования стационарного режима
существуют и~стационарные вероятности
$p_i^-\hm=\lim\nolimits_{n \rightarrow \infty } p_i^-(n)$, $i\hm\ge 1$, 
того,
что непосредственно после момента потери в~системе будет~$i$~заявок,
а также стационарные вероятности
\begin{align*}
h_{t}&=\lim\limits_{n \rightarrow \infty } h_{t}(n)\,,\enskip t \hm\ge 1\,;\\
h_{t_1,t_2}&=\lim\limits_{n \rightarrow \infty } h_{t_1,t_2}(n)\,,\enskip 
t_1,t_2 \hm\ge 1\,.
\end{align*}

Cистема уравнений Колмогорова--Чепмена для вероятностей $p_i^-(n)$, $n 
\hm\ge 2$, $i \hm\ge 1$, имеет вид:
\begin{multline}
p_i^-(n+1)=p_i^-(1)\sum\limits_{j=1}^\infty
p_j^-(n) \fr{\beta(\overline{a})^j}{\overline{a}}+{}\\
{}+\left(al_i \sum\limits_{j=1}^{i-1}
p_j^-(n) +p_i^-(n)\sum\limits_{j=1}^{i}al_j\right)
\left( 1-\fr{\beta(\overline{a}) }{\overline{a}}\right)+{}
\\
{}+
\sum\limits_{k=1}^\infty
\left(
al_i \sum\limits_{j=k+1}^{k+i-1}
p_j^-(n) +
p_{k+i}^-(n) \sum\limits_{j=1}^{i}
al_j \right)\times{}\\
{}\times
\fr{\beta(\overline{a})^k (1-\beta(\overline{a})) }{\overline{a}}, 
\enskip n \ge 1\,, \enskip i \ge 1\,,
\label{loss1}
\end{multline}

\noindent к~которому необходимо добавить условие нормировки 
$\sum\nolimits_{i=1}^{\infty} p_i^-(n+1)\hm=1$.
Вероятности $p_i^-(1)$, $i \hm\ge 1$, того, что после первой потери 
в~системе окажется
$i$ заявок, также могут быть найдены из~\eqref{loss1}, если зафиксировать
общее число заявок в~системе в~начальный момент функционирования.
Так, если изначально система пуста,
для нахождения $\{ p_i^-(1), \ i \hm\ge 1\}$ достаточно положить 
в~\eqref{loss1}
$n\hm=0$ и~$p_i^-(0)\hm=a \overline{a}^{i-1}$, $i \hm\ge 1$. Устремляя 
в~\eqref{loss1} $n \hm\rightarrow \infty$,
получаем систему уравнений для стационарных вероятностей~$p_i^-$, $i 
\hm\ge 1$,
решение которой при сделанных предположениях
о~распределениях $\{ b_i, i \hm\ge 0\}$ и~$\{ l_i, i \hm\ge 1\}$
(см.\ разд.~2) может быть найдено чис\-лен\-но.

Перейдем к~нахождению распределений $\{h_{t},\linebreak t \hm\ge 1\}$ 
и~$\{h_{t_1,t_2}, t_1,t_2 \hm\ge 1\}$.
Обозначим через $h^{(1)}_{t,i,j}$, $t,i,j \hm\ge 1$, условную вероятность 
того, что
очередная потеря произойдет через~$t$~тактов, при условии что изначально
в~системе находится~$i$~заявок и~остаточное время обслуживания
заявки на приборе равно~$j$~тактам. Положим 

\noindent
$$
h^{(1)}_{t,i}\hm=\sum\limits_{j=1}^\infty b_j h^{(1)}_{t,i,j}\,.
$$
Воспользовавшись формулой полной вероятности, находим

\vspace*{-2pt}

\noindent
\begin{align}
\label{loss2}
h^{(1)}_{t,1,j}&=
{\mathbf{1}_{(1 \le j \le t-1)}}
\sum\limits_{k=0}^{t-j-1} \overline{a}^{k+j-1}
\sum\limits_{m=1}^{\infty} a l_m h^{(1)}_{t-j-k,m}
+ {}\notag\\[6pt]
&\hspace*{25mm}{}+{\mathbf{1}_{(j \ge t+1)}}
\overline{a}^{t-1} a\,, \enskip j\ge 1\,;\\[9pt]
\label{loss3}
h^{(1)}_{t,i,j}&=
{\mathbf{1}_{(1 \le j \le t-1)}}
\overline{a}^{j} h^{(1)}_{t-j,i-1}
+
{\mathbf{1}_{(j \ge t)}}
\overline{a}^{t-1} a\,,\notag\\[6pt]
&\hspace*{41mm}\enskip i \ge 2\,, \enskip j\ge 1\,,
\end{align}

\vspace*{-2pt}

\noindent где ${\mathbf{1}_{(A)}}$~--- индикатор множества~$A$.
Соотношения~\eqref{loss2} и~\eqref{loss3} позволяют
последовательно по~$t$, начиная с~$t\hm=1$, определять\footnote{Расчет 
необходимо
вести в~следующем порядке: $h^{(1)}_{t,1,1}, h^{(1)}_{t,1,2},\dots$,
$h^{(1)}_{t,1},h^{(1)}_{t,2},\dots$} вероятности $h^{(1)}_{t,i,j}$
и~$h^{(1)}_{t,i}$. Усредняя $h^{(1)}_{t,i}$ по распределению $\{ p_i^-, i 
\hm\ge 1\}$, получаем
выражение для стационарного распределения длины
интервала между последовательными потерями:

\noindent
\begin{equation}
\label{loss4}
h_{t}
=\sum\limits_{i=1}^{\infty} h^{(1)}_{t,i} p_i^-, \enskip t \ge 1\,.
\end{equation}

\vspace*{-2pt}

\noindent Формулы для совместного стационарного распределения длин~$k$, 
$k\hm\ge2$, последовательных
интервалов между потерями могут быть получены аналогичным образом. 
Остановимся на случае $k\hm=2$. Обозначим через $h^{(2)}_{t_1,t_2,i}$, 
$t_1,t_2,i \hm\ge 1$, условную вероятность того, что
очередная потеря произойдет через~$t_1$ тактов, а~последующая~--- 
через~$t_2$ так-\linebreak\vspace*{-12pt}

\columnbreak

\noindent
тов, при условии что сразу после очередной потери 
в~сис\-те\-ме находится~$i$~заявок. Вводя обозначение

\vspace*{2pt}

\noindent
$$
\Delta_{t,i,j}=
h^{(1)}_{t,i,j-1} \sum\limits_{m=1}^{i} a l_m 
+\sum\limits_{m=i+1}^{\infty} a l_m
h^{(1)}_{t,m}
$$

\noindent  
и применяя формулу полной вероятности, получаем, что
$h^{(2)}_{t_1,t_2,i}$ могут быть рассчитаны на основе
$h^{(1)}_{t,i}$ рекуррентно по следующим формулам:

\vspace*{-3pt}

\noindent
\begin{align*}
%\label{loss5}
h^{(2)}_{t_1,t_2,1} &=
\sum\limits_{j=1}^{t_1-1}b_j
\!\sum\limits_{k=0}^{t_1-j-1}\!
\overline{a}^{k+j-1}
\sum\limits_{m=1}^{\infty} a l_m
h^{(2)}_{t_1-j-k,t_2,m}+{}\\
&\hspace*{24mm}{}+\sum\limits_{j=t_1+1}^{\infty}
b_j \overline{a}^{t_1-1}
\Delta_{t_2,1,j-t_1+1}\,;\\
%\end{multline*}
%\begin{multline*}
%\label{loss6}
h^{(2)}_{t_1,t_2,i} &=
\sum\limits_{j=1}^{t_1-1} b_j
\overline{a}^{j-1} h^{(2)}_{t_1-j,t_2,i-1}+{}\\
&{} + b_{t_1}
\overline{a}^{t_1-1}
\sum\limits_{m=1}^{\infty} a l_m
h^{(1)}_{t_2,\max(m,i-1)}+{}\\
&\hspace*{24mm}{}+ \sum\limits_{j=t_1+1}^{\infty}
b_j \overline{a}^{t_1-1}
\Delta_{t_2,i,j-t_1+1}\,.
\end{align*}

\vspace*{-2pt}

\noindent Усредняя $h^{(2)}_{t_1,t_2,i}$, как в~\eqref{loss4}, получаем
безусловное распределение длин последовательных двух
интервалов между потерями в~стационарном режиме.
%Справедливости ради необходимо отметить, что вычисления по полученным формулам
%явлются очень трудоемкими.

\vspace*{-6pt}

\section{Заключение}

В связи с~найденным видом стационарного распределения встает вопрос 
о~точности вычислений.
При расчете~$P_0$ по~\eqref{sn4-1} нельзя сказать,
когда нужно оборвать вычисления, чтобы гарантировать заданную точность. 
Этот вопрос,
встающий особенно остро, когда распределение
$\{b_i, i \hm\ge 0\}$ имеет тяжелый хвост, требует дополнительных 
исследований.
Некоторым ориентиром на практике могут служить двусторонние
оценки для~$P_0$, например $e^{-\sum\nolimits_{i=1}^\infty c_i} \hm\le 
P_0 \hm\le 
(1+\sum\nolimits_{i=1}^\infty c_i)^{-1}$ (см.\ далее, 
например,~\cite{klam}). Полезными могут оказаться и~приближенные формулы 
(см., например,~(8) в~\cite{i1}). В~плане дальнейших исследований 
несомненный интерес
представляет обобщение использованного метода на несколько СМО, связанных
рассмотренной дисциплиной обслуживания, а~также снятие наложенных
на входящий поток ограничений.

\vspace*{-6pt}


{\small\frenchspacing
 {%\baselineskip=10.8pt
 %\addcontentsline{toc}{section}{References}
 \begin{thebibliography}{99}

\bibitem{i5} %1
\Au{Schroeder B., Harchol-Balter~M.}
Web servers under overload: How scheduling can help~//
ACM~T. Internet Techn., 2006. Vol.~6. Iss.~1. P.~20--52.

\bibitem{i4} %2
\Au{Pradhan S., Gupta~U.\,C.}
Modeling and analysis of an infinite-buffer batch-arrival
queue with batch-size-dependent service~// Perform. Evaluation, 2017. 
Vol.~108. P.~16--31.

\bibitem{i2} %3
\Au{Grosof I., Scully~Z., Harchol-Balter~M.}
SRPT for multiserver systems~// Perform. Evaluation, 2018. Vol.~127-128. 
P.~154--175.

\bibitem{i3} %4
\Au{Marin A., Mitrani~I., Elahi~B.\,M., Williamson~C.}
Control and optimization of the SRPT service policy by frequency 
scaling~// 
Conference (International) on Quantitative Evaluation of Systems~/ Eds. 
A.~McIver, A.~Horvath.~--- Lecture notes in computer science ser.~--- 
Springer, 2018. Vol.~11024. P.~257--272.

\bibitem{i6} %5
\Au{Scully Z., Harchol-Balter~M., Scheller-Wolf~A.}
Simple near-optimal scheduling for the $M/G/1$~//
SIGMETRICS Perform. Eval. Rev., 2019. Vol.~47. Iss.~2. P.~24--26.


\bibitem{i1} %6
\Au{Marin A., Rossi~S.} A~queueing model that works
only on biggest jobs~// European Workshop on Performance Engineering~/ 
Eds. M.~Gribaudo, M.~Iacono, T.~Phung-Duc, R.~Razumchik.~--- Lecture 
notes in computer science ser.~--- Springer, 2020. Vol.~12039. P.~118--132.

%Marin A., Rossi S. (2020) A Queueing Model that Works Only on the 
%Biggest Jobs. In: Gribaudo M., Iacono M., Phung-Duc  
%T., Razumchik R. (eds) Computer Performance Engineering. EPEW 2019. 
%Lecture Notes in Computer Science, vol 12039.  
%Springer, Cham. https://doi.org/10.1007/978-3-030-44411-2_8

\bibitem{i8} %7
\Au{Pittel~B.\,G.}
Closed exponential networks of queues with saturation: The Jackson-type 
stationary distribution and its asymptotic analysis~//
Math. Oper. Res., 1979. Vol. 4. Iss.~4. P.~357--378.

\bibitem{i7} %8
\Au{Balsamo~S., Harrison~P., Marin~A.}
A~unifying approach to product-forms in
networks with finite capacity constraints~//
SIGMETRICS Perform. Eval. Rev., 2010. Vol.~38. Iss.~1. P.~25--36.

\bibitem{n4} %9
\Au{Печинкин А.\,В.} Об одной
инвариантной системе массового обслуживания~//
Math.\ Operationsforsch.\ Statist.
Ser.\ Optimization, 1983. Vol.~14. No.\,3. P.~433--444.

%Pechinkin, A.V. 1983.
%Ob odnoy invariantnoy sisteme massovogo
%obsluzhivaniya
%[On an Invariant Queuing System].
%{\it Math.\ Operationsforsch.\ Statist.
%Ser.\ Optimization} 14(3): 433--444.


\bibitem{tata} %10
\Au{Таташев~А.\,Г.}
Многоканальная система массового обслуживания с~потерями кратчайших 
требований~// Автоматика и~телемеханика, 1991. №\,7. С.~187--189.


\bibitem{i10} %11
\Au{Милованова Т.\,А.} Система BMAP${/G/1}$ с~инверсионным порядком 
обслуживания и~вероятностным приоритетом~// Автоматика и~телемеханика, 2009. 
№\,5. С.~155--168.

%Milovanova, T. A. 2009. ${BMAP/G/1/\infty}$ system with last
%come first served probabilistic priority. \textit{Automat. Rem.
%Contr.} 70(5): 885--896.

\bibitem{i11}
\Au{Мейханаджян Л.\,А.}
Стационарные вероятности состояний в~системе обслуживания конечной 
емкости с~инверсионным порядком обслуживания и~обобщенным вероятностным 
приоритетом~// Информатика и~её применения, 2016. Т.~10. Вып.~2. С.~123--131.
%Meykhanadzhyan, L. A. 2016. Statsionarnyye veroyatnosti sostoyaniy v sisteme 
%obsluzhivaniya konechnoy emkosti s inversionnym poryadkom 
%obsluzhivaniya i obobshchennym veroyatnostnym prioritetom
%[Stationary Characteristics of the Finite Capacity Queueing System with 
%Inverse Service Order and Generalized Probabilistic Priority].
%\textit{Informatika i ee Primeneniya --- Inform. Appl.} 10(62): 123--131.

\bibitem{i12}
\Au{Бочаров П.\,П., Гаврилов~Е.\,В., Печинкин~А.\,В.}
Экспоненциальная сеть массового обслуживания с~зависимым обслуживанием, 
отрицательными заявками и~изменением типа заявок~// Автоматика и~телемеханика, 
2004. №\,7. С.~35--59.

%P. P. Bocharov, E. V. Gavrilov, A. V. Pechinkin, ``Exponential queuing 
%network with dependent servicing, negative  
%customers, and modification of the customer type'', Autom. Remote 
%Control, 65:7 (2004), 1066--1088.

\bibitem{distime}
\Au{Печинкин А.\,В., Разумчик~Р.\,В.}
Системы массового обслуживания в~дискретном времени.~--- M.: Физматлит, 
2018. 432~с.

%Pechinkin, A. V., and R. V. Razumchik. 2018.
%\textit{Sistemy massovogo obsluzhivaniya v diskretnom vremeni}
%[Discrete Time Queuing Systems]. Moscow: Fizmatlit. 432 p.

\bibitem{nobel}
\Au{Nobel R.}
Retrial queueing models in discrete time: A~short survey of some late 
arrival models~//
Ann. Oper. Res., 2015. Vol. 247. Iss.~1. P.~37--63.

\bibitem{kry}
\Au{Канторович Л.\,В., Крылов~В.\,И.} Приближенные методы высшего 
анализа.~--- 5-е изд.~--- М.--Л.: Физматлит, 1962. 708~с.

\bibitem{wil}
\Au{Shivakumar P.\,N., Williams~J.\,J.}
An iterative method with trunction for infinite linear systems~// J.~Comput. Appl. Math., 1988.
Vol.~24. P.~199--207.

\bibitem{Car}
\Au{Carmichael R.\,D.} On non-homogeneous equations with an infinite 
number of variables~// Am. J.~Math., 1914. Vol.~36. Iss.~1. P.~13--20.

\bibitem{klam}
\Au{Klamkin M.\,S., Newman~D.\,J.}
Extensions of the Weierstrass product inequalities~//
Math. Mag., 1970. Vol.~43. Iss.~3. P.~137--141.
\end{thebibliography}

 }
 }

\end{multicols}

\vspace*{-6pt}

\hfill{\small\textit{Поступила в~редакцию 15.10.20}}

\vspace*{8pt}

%\pagebreak

%\newpage

%\vspace*{-28pt}

\hrule

\vspace*{2pt}

\hrule

%\vspace*{-2pt}

\def\tit{STATIONARY CHARACTERISTICS\\ 
OF~DISCRETE-TIME~Geo$/G/1/\infty$ QUEUE\\ 
WITH~BATCH ARRIVALS AND~ONE QUEUE SKIPPING POLICY}


\def\titkol{Stationary characteristics of discrete-time Geo$/G/1/\infty$ 
queue with~batch arrivals and~one queue skipping policy}


\def\aut{S.\,I.~Matyushenko$^1$ and R.\,V.~Razumchik$^2$}

\def\autkol{S.\,I.~Matyushenko and R.\,V.~Razumchik}

\titel{\tit}{\aut}{\autkol}{\titkol}

\vspace*{-11pt}


\noindent
$^1$Peoples' Friendship University of Russia (RUDN University), 
6~Miklukho-Maklaya Str., Moscow 117198,\\
$\hphantom{^1}$Russian Federation

\noindent
$^2$Institute of Informatics Problems, Federal Research Center 
``Computer Science and Control'' of the Russian\\
$\hphantom{^1}$Academy of Sciences, 
44-2~Vavilov Str., Moscow 119333, Russian Federation

\def\leftfootline{\small{\textbf{\thepage}
\hfill INFORMATIKA I EE PRIMENENIYA~--- INFORMATICS AND
APPLICATIONS\ \ \ 2020\ \ \ volume~14\ \ \ issue\ 4}
}%
 \def\rightfootline{\small{INFORMATIKA I EE PRIMENENIYA~---
INFORMATICS AND APPLICATIONS\ \ \ 2020\ \ \ volume~14\ \ \ issue\ 4
\hfill \textbf{\thepage}}}

\vspace*{3pt} 

\Abste{
Consideration is given to the discrete-time single-server system 
with one queue of infinite capacity and the geometric (Bernoulli) 
input flow. Customers are homogeneous, arrive in batches, 
and are served one by one in FIFO (first in, first out) manner. 
The sizes of arriving batches as well as the service times are assumed 
to be independent and identically distributed random variables with arbitrary discrete distributions.
The queue skipping policy is implemented in the system:
upon arrival of a~batch, its size is compared with the current
total number of customers in the system. If the size of the batch 
is larger than the system content, all customers residing in the\linebreak\vspace*{-12pt}}

\Abstend{system
(including the one in server) are lost and the arrived batch enters the 
system; otherwise, the new batch leaves the system having no effect on it. 
Main stationary system performance characteristics, including 
those of the flow of lost customers, are obtained.}


\KWE{discrete-time; queueing system; batch arrivals; queue skipping 
policy}


\DOI{10.14357/19922264200404} 

\vspace*{-16pt}

\Ack
\noindent
The reported 
study was funded by RFBR (project number 20-07-00804) and conducted
in accordance with the Program of Moscow Center for Fundamental and 
Applied Mathematics.


%\vspace*{6pt}

 \begin{multicols}{2}

\renewcommand{\bibname}{\protect\rmfamily References}
%\renewcommand{\bibname}{\large\protect\rm References}

{\small\frenchspacing
 {%\baselineskip=10.8pt
 \addcontentsline{toc}{section}{References}
 \begin{thebibliography}{99}

\bibitem{i5-1} %1
\Aue{Schroeder, B., and M.~Harchol-Balter.} 2006. 
Web servers under overload: How scheduling can help.
\textit{ACM~T. Internet Techn.} 6(1):20--52.

\bibitem{i4-1}%2
\Aue{Pradhan, S., and U.\,C.~Gupta.} 2017.
Modeling and analysis of an infinite-buffer batch-arrival
queue with batch-size-dependent service.
\textit{Perform. Evaluation} 108:16--31.

\bibitem{i2-1} %3
\Aue{Grosof, I., Z.~Scully, and M.~Harchol-Balter.} 2018.
SRPT for multiserver systems. \textit{Perform. Evaluation} 127-128:154--175.

\bibitem{i3-1} %4
\Aue{Marin, A., I. Mitrani, B.\,M.~Elahi, and C.~Williamson}. 2018.
Control and optimization of the SRPT service policy by frequency scaling.
\textit{Conference (International) on Quantitative Evaluation of 
Systems}. 
Eds. A.~McIver, and A.~Horvath. Lecture notes in computer science ser. 
Springer. 11024:257--272.

\bibitem{i6-1}
\Aue{Scully, Z., M.~Harchol-Balter, and A.~Scheller-Wolf}. 2019. 
Simple near-optimal scheduling for the $M/G/1$.
\textit{\mbox{SIGMETRICS} Perform. Eval. Rev.} 47(2):24--26.

\bibitem{i1-1}
\Aue{Marin, A., and S.~Rossi.} 2020. A~queueing model that works
only on biggest jobs. \textit{European Workshop on Performance 
Engineering}.
Eds. M.~Gribaudo, M.~Iacono, T.~Phung-Duc, and R.~Razumchik.
Lecture notes in computer science ser. Springer. 12039:118--132.

\bibitem{i8-1} %7
\Aue{Pittel, B.\,G.} 1979.
Closed exponential networks of queues with saturation: The Jackson-type 
stationary distribution and its asymptotic analysis. 
\textit{Math. Oper. Res.} 4(4):357--378.

\bibitem{i7-1} %8
\Aue{Balsamo, S., P.~Harrison, and A.~Marin.} 2010. 
A~unifying approach to product-forms in
networks with finite capacity constraints.
\textit{SIGMETRICS Perform. Eval. Rev.} 38(1):25--36.

\bibitem{n4-1} 
\Aue{Pechinkin, A.\,V.} 1983.
Ob odnoy invariantnoy sisteme massovogo
obsluzhivaniya [On an invariant queuing system]. \textit{Math.\ 
Operationsforsch.\ Statist. Ser.\ Optimization} 14(3):433--444.

\bibitem{tata-1}
\Aue{Tatashev, A.\,G.} 1991.
A~queueing system with invariant discipline.
\textit{Automat. Rem. Contr.} 52(7):1034--1037.

\bibitem{i10-1}
\Aue{Milovanova, T.\,A.} 2009. BMAP${/G/1/\infty}$ system with last
come first served probabilistic priority. \textit{Automat. Rem.
Contr.} 70(5):885--896.

\bibitem{i11-1}
\Aue{Meykhanadzhyan, L.\,A.} 2016. Statsionarnye veroyatno\-sti sostoyaniy 
v~sisteme obsluzhivaniya konechnoy emkosti s~inversionnym poryadkom 
obsluzhivaniya i~obobshchennym veroyatnostnym prioritetom
[Stationary characteristics of the finite capacity queueing system with 
inverse service order and generalized probabilistic priority]. 
\textit{Informatika i~ee Primeneniya --- Inform. Appl.} 10(62):123--131.

\bibitem{i12-1}
\Aue{Bocharov, P. P., E.\,V.~Gavrilov, and A.\,V.~Pechinkin.} 2004. 
Exponential queuing network with dependent servicing, negative customers, 
and modification of the customer type.
\textit{Automat. Rem. Contr.} 65(7):1066--1088.

\bibitem{distime-1}
\Aue{Pechinkin, A.\,V., and R.\,V.~Razumchik.} 2018.
\textit{Sistemy massovogo obsluzhivaniya v~diskretnom vremeni}
[Discrete time queuing systems]. Moscow: Fizmatlit. 432~p.

\bibitem{nobel-1}
\Aue{Nobel, R.} 2015.
Retrial queueing models in discrete time: A~short survey of some late 
arrival models.
\textit{Ann. Oper. Res}. 247(1):37--63.

\bibitem{kry-1}
\Aue{Kantorovich, L.\,V., and V.\,I.~Krylov}. 1962.
\textit{Priblizhennye metody vysshego analiza}.
Moscow--Saint-Petersburg: Fizmatlit. 708~p. 

\bibitem{wil-1}
\Aue{Shivakumar, P.\,N., and J.\,J.~Williams}. 1988.
An iterative method with trunction for infinite linear systems. 
\textit{J.~Comput. Appl. Math.}
24:199--207.

\bibitem{Car-1}
\Aue{Carmichael, R.\,D.} 1914.
On non-homogeneous equations with an infinite number of variables.
\textit{Am. J.~Math}. 36(1):13--20.

\bibitem{klam-1}
\Aue{Klamkin, M.\,S., and D.\,J.~Newman.} 1970. 
Extensions of the Weierstrass product inequalities.
\textit{Math. Mag.} 43(3):137--141.
\end{thebibliography}

 }
 }

\end{multicols}

\vspace*{-6pt}

\hfill{\small\textit{Received October 15, 2020}}

%\pagebreak

\vspace*{-20pt}


\Contr

\vspace*{-4pt}

\noindent
\textbf{Matyushenko Sergey I.} (b.\ 1963)~---
Candidate of Science (PhD) in physics and mathematics, associate 
professor,
Department of Applied Informatics and Probability Theory,
Peoples' Friendship University of Russia (RUDN University), 
6~Miklukho-Maklaya Str., Moscow 117198, Russian Federation; 
\mbox{matyushenko\_si@pfur.ru}

\vspace*{3pt}

\noindent
\textbf{Razumchik Rostislav V.} (b.\ 1984)~---
Candidate of Science (PhD) in physics and mathematics, leading scientist,
Institute of Informatics Problems, Federal Research Center ``Computer 
Science and Control'' of the Russian Academy of Sciences, 44-2~Vavilov 
Str., Moscow 119333, Russian Federation; \mbox{rrazumchik@ipiran.ru}

\label{end\stat}

\renewcommand{\bibname}{\protect\rm Литература} 