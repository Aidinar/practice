%Переписка с~не с~авторами, а с~Татьяной Александровной Рудченко 
%(rudchenko1@yandex.ru).  
\def\stat{betelin}


\def\tit{О ЦИФРОВОЙ ГРАМОТНОСТИ И~СРЕДАХ 
ЕЕ~ФОРМИРОВАНИЯ$^*$}

\def\titkol{О цифровой грамотности и~средах 
ее~формирования}

\def\aut{В.\,Б.~Бетелин$^1$, А.\,Г.~Кушниренко$^2$, 
А.\,Л.~Семенов$^3$, С.\,Ф.~Сопрунов$^4$}

\def\autkol{В.\,Б.~Бетелин, А.\,Г.~Кушниренко, А.\,Л.~Семенов, 
С.\,Ф.~Сопрунов}

\titel{\tit}{\aut}{\autkol}{\titkol}

\index{Бетелин В.\,Б.}
\index{Кушниренко А.\,Г.}
\index{Семенов А.\,Л.}
\index{Сопрунов С.\,Ф.}
\index{Betelin V.\,B.}
\index{Kushnirenko A.\,G.}
\index{Semenov A.\,L.}
\index{Soprunov S.\,F.}


{\renewcommand{\thefootnote}{\fnsymbol{footnote}} \footnotetext[1]
{Работа выполнена при поддержке РФФИ (проект №\,19-29-14199~--- С.\,Ф.~Сопрунов), Российского 
научного фонда (проект №\,17-11-01377~--- А.\,Л.~Семенов, разделы~3 и~5), госзадания 2020~года 
в~ФГУ ФНЦ НИИСИ РАН по теме 0065-2019-0010 (В.\,Б.~Бетелин и~А.\,Г.~Кушниренко).}}


\renewcommand{\thefootnote}{\arabic{footnote}}
\footnotetext[1]{Федеральный научный центр Научно-исследовательский институт 
системных исследований Российской академии наук, \mbox{betelin@niisi.msk.ru}}
\footnotetext[2]{Федеральный научный центр Научно-исследовательский институт 
системных исследований Российской академии наук, \mbox{agk\_@mail.ru}}
\footnotetext[3]{Московский государственный университет имени М.\,В.~Ломоносова; 
Институт кибернетики и~образовательной информатики им.\ А.\,И.~Берга Федерального 
исследовательского центра <<Информатика  
и~управ\-ле\-ние>> Российской академии наук; НИУ Московский фи\-зи\-ко-тех\-ни\-че\-ский институт, 
\mbox{alsemno@ya.ru}}
\footnotetext[4]{Центр педагогического мастерства, soprunov@mail.ru}


\vspace*{-12pt}

  
  \Abst{Цифровая грамотность становится ключевой характеристикой личности человека 
XXI~в. Ее можно и~нужно формировать с~раннего возраста. Цифровые среды могут быть 
средами достижения предметных, метапредметных и~личностных образовательных 
результатов.
  В~работе анализируется отечественный и~международный опыт в~данном направлении 
начиная с~1960-х~гг., приводится система принципов, обеспечивших эффективность для 
образовательных целей цифровых сред и~систем, разработанных под руководством и~при 
участии авторов. Эти среды и~системы эффективно используются сегодня в~РФ 
в~формировании у детей раннего возраста цифровой грамотности и~основы для 
\textit{computational thinking}~--- системного мышления XXI~в. В~статье приводится обзор 
результатов, полученных в~данном направлении. Специальное внимание уделяется проблеме 
визуализации и~представления в~реальном мире алгоритмических процессов и~задающих их 
программ.}
  
  \KW{информатика; цифровая грамотность; computational thinking; робот; язык 
программирования; Лого; ПервоЛого; ПиктоМир; визуализация}
  
\DOI{10.14357/19922264200414} 
  
\vspace*{-2pt}


\vskip 10pt plus 9pt minus 6pt

\thispagestyle{headings}

\begin{multicols}{2}

\label{st\stat}
  
\section{Формирование идеи цифровой грамотности 
и~конструирование сред ее достижения}

\vspace*{-6pt}

  Успешные попытки обучения школьников программированию начались 
  в~СССР в~начале 1960-х~гг.\ Это обучение велось в~специально набранных 
старших классах, которые позже назвали <<математическими>>, 
декларировалось как профессиональная подготовка и~сочеталось 
с~углубленной подготовкой по математике. Вели занятия не школьные учителя, 
а внешние специалисты~[1]. О~переносе подобной методики в~начальную 
школу не могло быть и~речи. 
  
  Другие подходы к~освоению детьми цифрового мира, о развитии которых 
пойдет речь в~настоящей статье, возникли в~конце 1960-х~гг. Тогда Симор 
Паперт, математик, программист, психолог (ученик Ж.~Пиаже) и~педагог, 
работал в~Массачусетском технологическом институте над проблемами 
искусственного интеллекта. Вместе с~Марвином Минским он написал книгу 
<<Персептроны>> о~машинном обучении, когда для этой технологии еще не 
было материальной базы. В~1967~г.\ коллеги Паперта (при его участии) из 
большой корпорации BBN (живущей, в~основном, оборонными заказами) 
создали для развития детей язык программирования Лого (на базе языка 
Лисп). Паперт включил в~язык команды управления рисующим на полу 
роботом, <<экранный аватар>>, который потом прославился как Черепашка 
Лого. Это задало подход к~преподаванию программирования детям как 
элемента их общего развития. В~мемориальной статье~[2] говорится: 
<<Добавленная $\langle$Папертом$\rangle$ к~Лого черепашка, управляемая 
компьютером на экране и~на полу, обеспечила визуализацию и~овеществление 
процессов программирования и~осмыс\-лен\-ность их результатов. Тем самым 
Лого превратился в~уникальную среду для освоения алгоритмического 
мышления, которой пользуются миллионы детей в~десятках стран мира>>.
  
  С середины 1970-х~гг.\ А.\,П.~Ершов с~коллегами в~ВЦ СО АН начал 
обучение школьников, в~том чис\-ле~---  младших классов, программированию 
в~мик\-ро\-ми\-рах <<Дежурик>> и~<<Маляр>>~[3]. К~1981~г.\ его представления 
о~роли программирования (сегодня  бы сказали~--- цифровой грамотности) 
оформились в~виде лозунга <<Программирование~--- вторая грамотность>>, 
ставшего названием его исторического доклада в~Лозанне~[4, 5].
  
  С.~Паперт был убежден, что использование компьютеров совершит 
переворот в~образовании. Его книга Mindstorms (1980~г.)\ оказала огромное 
влияние на образовательное сообщество всего мира, не исключая и~СССР 
(приблизительный русский перевод издан в~1988~г.)~[6]. В~нашу страну по 
приглашению А.\,Л.~Семенова Паперт приехал в~конце 1980-х~гг.; с~ним 
авторов настоящей публикации связала многолетняя дружба.
  
  За истекшие годы в~мире были разработаны сотни реализаций Черепашки. 
В~России распространение Лого началось с~работы Ю.\,А.~Первина 
в~Новосибирске. Большую популярность получили \mbox{системы} программирования 
LogoWriter и~\mbox{ЛогоМиры}~--- российские адаптации англоязычных продуктов 
компании Logo Computer Systems Inc., разработанные под руководством 
А.~Семенова и~С.~Сопрунова в~Институте новых технологий (ИНТ). 
\mbox{ЛогоМиры} продолжают активно использоваться и~сегодня~[7, 8]. Одной из 
самых распространенных сред для раннего изучения программирования стал 
Scratch, разработанный учениками и~коллегами Паперта~[9].
  
  Черепашка Паперта <<живет>> в~своем <<микромире>>, где ученик 
управляет ей и~ее учит. 
 В~1970--\linebreak 1980-х~гг.\ в~нескольких образовательных центрах 
мира началось использование разнообразных <<мик\-ро\-миров>> учения. 
В~част\-ности, в~Стенфорде Ричард Паттис изобрел исполнителя \textit{Karel 
The Robot} и~назвал его в~честь Карела Чапека~[10]. Роботы, <<живущие>> в~клетчатой среде,  
c~1980~г.\ использовались в~курсе программирования на мехмате МГУ~[11] 
(первая версия~--- <<Путник>>). 
  
  Новый этап в~развитии цифровой грамотности в~нашей стране начался 
историческим Постановлением ЦК КПСС и~Совета Министров СССР от 
28~марта 1985~г.\ №\,271 <<О~мерах по обеспечению компьютерной 
грамотности учащихся средних учебных заведений и~широкого внедрения 
элект\-рон\-но-вы\-чис\-ли\-тель\-ной техники в~учебный процесс>>~[12],  
позволившим нашей стране опередить весь мир в~тотальной цифровизации 
школы. В~подготовке Постановления А.\,П.~Ершов сыграл ключевую роль. 
Во всех старших классах СССР началось изучение обязательного предмета 
<<Основы информатики и~элект\-рон\-но-вы\-чис\-ли\-тель\-ной техники>>. 
Система программирования <<КуМир>>, основанная на микромире <<Робот в~лабиринте>> 
и~др., была реализована усилиями МГУ и~Академии наук \mbox{СССР} 
(ВНТК <<Школа-1>>, руководитель Е.\,П.~Велихов, зам.\ руководителя 
А.\,Л.~Семенов) и~стала цент\-ром основополагающего учебника информатики 
для старших классов СССР~[13], затем использовалась в~учебниках для 
учащихся 5--11~классов России~[14--16]. В~начале 2010-х~гг.\ по инициативе 
членов На\-уч\-но-ме\-то\-ди\-че\-ско\-го совета ФИПИ по информатике 
(А.\,Л.~Семенов, А.\,Ю.~Уваров, С.\,Г.~Анд\-ре\-ев, В.\,А.~Галкин, Г.\,И.~Савин, 
председатель НМС В.\,Б.~Бетелин) в~варианты государственной итоговой аттестации по информатике стали 
включаться задачи, использующие исполнителя <<Робот>>, доступные 
школьникам с~минимальной подготовкой по программированию. Реализация в~НИИСИ РАН 
свободно распространяемой версии проекта <<\mbox{КуМир}>>~[17] 
сделала возможным решение подобных задач на любых школьных и~домашних компьютерах.
  
\section{Естественный возраст формирования цифровой 
грамотности}

  Развитие информационных технологий, цифровизация быта, образования, 
науки и~экономики привели к~общемировой тенденции дальнейшего 
понижения возраста знакомства детей с~информатикой и~программированием 
вплоть до дошкольного возраста~[18]. Именно об этом говорили С.~Паперт 
и~А.\,П.~Ершов более 40~лет назад.
  
  В России эта тенденция проявляется как раннее знакомство детей с~обрамляющими 
  программирование информатикой и~математикой. 
Цент\-раль\-ное место при этом занимают не числовые,\linebreak а~символические 
и~комбинаторные аспекты объектов и~процессов окружающего мира. 
В~част\-ности,\linebreak в~действующем федеральном государственном 
образовательном стандарте начального {общего} образования~[19] 
описываются <<предметные результаты освоения основной образовательной 
\mbox{программы} начального общего образования по математике и~информатике: 
овладение основами {логического} и~алгоритмического мышления$\ldots$\ 
уметь действовать в~соответствии с~алгоритмом и~строить простейшие 
алгоритмы, исследовать, распознавать и~изображать геометрические фигуры, 
работать с~таблицами, схемами, графиками и~диаграммами, цепочками, 
совокупностями>>.
  
  Общий взгляд на проблемы реализации в~школьной программе 
\textit{computational thinking}~--- основ информатики, алгоритмики и~программирования~---
 изложен в~статье А.\,Л.~Семенова~[20]. Под его 
руководством за последние 30~лет для начальной школы были подготовлены 
программы  
и~учеб\-но-ме\-то\-ди\-че\-ские комплексы для курсов <<Информатика>> и~<<Информатика 
и~математика>>. Очередная версия учебника 
<<Информатика. 1--4~классы>> издана в~2019~г.~[21]. В~начальной школе 
России, как правило, один учитель ведет б$\acute{\mbox{о}}$льшую часть 
предметов, он может для изучения математики, информатики и~технологии использовать 
общий ресурс учебного времени, для освоения 
компьютера использовать уроки русского языка и~искусства и~т.\,п. Возможны различные 
варианты работы, в~том числе <<бескомпьютерный>>. При интеграции 
программирования в~визуальных средах типа обсуждаемых ниже \mbox{ПервоЛого} 
и~ПиктоМира, модуль информатики занимает, в~среднем, около~2~ч в~неделю,
 что дает вариант целостного освоения языковой, числовой и~цифровой грамотности в~начальной школе.
  
  Осознание программирования уже не как <<второй>> грамотности, а как 
важнейшего элемента\linebreak комплексной грамотности XXI~в.\ приводит к~идее 
микромира, работа с~которым не предполагает %\linebreak
 <<текс\-то\-вой>> грамотности. 
А.\,Л.~Семенов в~\mbox{1990-е~гг.}\ предложил создать версию Лого, не 
исполь\-зу\-ющую\linebreak словесной и~числовой грамотности, как они обычно 
понимаются, а помогающую ребенку осваивать\linebreak различные грамотности 
одновременно и~параллельно. Эта идея была реализована коллективом\linebreak 
С.\,Ф.~Сопрунова в~проекте ПервоЛого (LogoFirst, IconLogo), высоко 
оцененном С.~Папертом~[22]. В ПервоЛого все команды задаются 
пиктограммами (icons). Например, поворот Черепашки вправо на~50$^\circ$, 
который в~Лого представляется как: \textbf{направо~50}, в~ПервоЛого 
задается поворотом штурвала, одна из ручек которого помечена красным. 
Число в~центре штурвала возникает автоматически при повороте или может 
задаваться учеником, и~тогда штурвал повернется в~соответствии с~числом 
(см.\ рисунок).



 
  Когда ребенок использует эту команду для управ\-ле\-ния Черепашкой, он 
одновременно осваивает плоскую геометрию и~начинает сопоставлять числа 
с~геометрическими величинами. Использование ПервоЛого в~школах России 
показало правильность исходной идеи параллельного освоения\linebreak\vspace*{-12pt}

{ \begin{center}  %fig1
 \vspace*{12pt}
   \mbox{%
 \epsfxsize=79mm 
 \epsfbox{bet-1.eps}
 }



\vspace*{6pt}
 
{\small{Команда <<направо 50>> в~ПервоЛого
}}
\end{center}}

\noindent
 различных 
видов грамотности: словесной, числовой, алгоритмической, 
про\-стран\-ст\-вен\-но-гео\-мет\-ри\-че\-ской.
  
  В последнее десятилетие идея визуальной, бестекстовой среды освоения 
алгоритмического мышления была реализована группой учебной 
информатики А.\,Г.~Кушниренко\,--\,А.\,Г.~Леонова под научным 
руководством академика В.\,Б.~Бетелина в~системе ПиктоМир как развитие 
в~направлении младших возрастов идеи  КуМира~[23].
  
  Еще одним шагом стало создание <<овеществленных>> сред 
программирования, где и~Черепашка может перемещаться по полу, 
и~программа может составляться из <<кубиков>> тоже на полу (или на\linebreak столе), 
рисоваться на бумаге и~т.\,д. Той же \mbox{груп\-пой} А.\,Г.~Куш\-ни\-рен\-ко\,--\,А.\,Г.~Лео\-но\-ва
 был разработан и~прошел апробацию в~детских садах 
учеб\-но-ме\-то\-ди\-че\-ский комплекс <<Алгоритмика для дошколят>>, в~основе 
которого лежит управление реальным ро\-бо\-том-ма\-шин\-кой Ползуном. 
Программу ребенок составляет на столе из кубиков с~нарисованными на 
гранях пиктограммами команд и~конструкций языка. Собранная из таких 
кубиков программа фотографируется, фотография распознается компьютером и~переводится 
в~звуковые команды, которые слышит и~выполняет 
Ползун~[24].
  
  О разнообразии подобных исполнителей можно судить, например, по 
примерам реального использования программируемых роботов в~российских 
детских садах на портале {\sf маам.ру}~[25]. Один из самых популярных 
детских роботов Bee-Bot~[26] программируется на корпусе ро\-бо\-та-<<пчел\-ки>>.
 Роботы Ozobot~[27] следуют нарисованной линии, при этом на 
самой линии могут цветом кодироваться действия.
  
  В работе~\cite{24-bet}, подготовленной с~участием ряда авторов настоящей 
статьи, подчеркивается еще одно важное свойство роботоподобных 
овеществленных исполнителей~--- их использование позволяет построить 
четкую, наглядную, доступную шести\-лет\-кам-се\-ми\-лет\-кам систему 
научных понятий (больших идей, см.\ далее) программирования.

\section{Задачи цифровой грамотности}

   Основная цель общего образования сегодня~--- формирование 
\textit{навыков XXI~в.}\ и~общая ориентация в~мире, позволяющая 
осознавать потребность в~конкретных знаниях (включая умения, навыки) 
и~получать эти знания (при необходимости формируя у себя умения и~навыки). 
Б$\acute{\mbox{о}}$льшая часть <<\textit{навыков XXI~в.}>>  оказывается 
более древней и~присущей образованному человеку системой, чем остальная 
часть результатов образования века~XX. К~таким навыкам относится само 
умение учиться, понимать другого человека, ставить цели и~анализировать 
неудачи и~т.\,п. Ориентация в~мире меняется быстрее, и~особенно быст\-ро~--- 
сейчас. К~ней относятся так называемые \textit{большие идеи} разного 
высокого уровня общности, например общее понятие энергии, 
наследственности, возможности математического описания физических 
процессов и~т.\,п. Концепция \textit{большой идеи} (Big Idea) возникла 
в~ес\-тест\-вен\-но-на\-уч\-ном образовании как оппозиция 
к~<<гор$\acute{\mbox{е}}$ фактов>>. \textit{Большая идея}~--- это часть 
представления человека о мире, без которой представление в~целом, 
поведение в~мире становятся другими. Все более необходимыми для 
ориентации становятся большие идеи \textit{цифровой грамотности}. 
Способность использовать ориентацию, вместе с~формирующейся 
в~информатике способностью решать совершенно новые задачи, составляют 
основу 
 пре-адап\-тив\-ности (А.\,Г.~Асмолов). 
  
  И \textit{навыки XXI~в.}, и~ориентация в~мире должны формироваться 
  в~конкретной интересной и~важной для ученика деятельности~--- созидании, 
коммуникации и~т. п. Конкретные среды, освоение конкретных умений 
и~навыков, приобретение необходимых знаний важны в~первую очередь не 
сами по себе, а как элементы пути формирования более общих результатов. 
<<Образование~--- это то, что остается, когда мы уже забыли все, чему нас 
учили>>, как, возможно, сказал Джордж Галифакс (XVII~в.)\ и,~судя по 
цитатам, многократно повторяли крупнейшие физики XX~в.\ (например, 
Макс фон Лауэ).
  
  Особенно этот подход актуален сегодня, когда учится, общается, действует и~живет 
  не сам по себе человек (взрослый или ребенок), а~\textit{человек 
расширенный}, способный в~мире что-то делать, обращаясь, кроме ресурсов 
собственного организма, к~циф\-ро\-вым ресурсам (источникам, инструментам, 
средам и~сервисам) и~через них присваивающий культуру человечества~[28].
  
 \textbf{На  взгляд авторов, основой для} {\bfseries\textit{computational 
thinking}}, \textbf{как и~для цифровой компетенции (даже если различать 
эти понятия), должны быть осознание себя (как и~других) как} 
{\bfseries\textit{человека расширенного}} \textbf{и эффективная 
деятельность в~качестве такового}. И~в~процессе образования  следует 
адресоваться к~такому \textit{расширенному человеку}, к~его 
\textit{расширенному сознанию}.
  
  Современный контекст информатики~--- программирования в~широком, 
<<ершовском>> смысле~--- позволяет:
  \begin{enumerate}[(1)]
  \item формировать многие из ключевых навыков XXI~века, в~том числе 
относящихся к~коллективной работе;
\columnbreak

\item интегрировать для \textit{расширенной личности} циф\-ро\-вую грамотность, 
прежде всего~--- ее \textit{большие идеи}, в~общую грамотность и~систему 
ориентации в~мире; 
\item делать это на конкретном материале разнообразных, интересных, 
посильно трудных задач с~высоким уровнем индивидуальной новизны для 
ученика.
\end{enumerate}
%
  При этом конкретные среды, в~которых идет достижение указанных 
целей~(1) и~(2), могут быть различными. Условие~(3) может достигаться 
по-раз\-но\-му, в~комбинации различных факторов:
  \begin{itemize}
\item интеллектуальный вызов~--- понятная новая задача с~неочевидным, 
неожиданным решением;
\item эмоциональная сторона~--- эстетическая привлекательность картин, 
разворачивающихся на экране, вовлеченность в~игровые ситуации, 
отождествление с~персонажами;
\item связь со взрослой жизнью~--- сходство твоего программирования 
с~профессиональным, профессиональная ориентация, польза в~будущем, а 
иногда и~оплачиваемая работа сейчас.
  \end{itemize}
  
\section{Роль среды. Язык программирования}

  Представленная перспектива стояла перед  авторами и~тогда, когда под 
руководством Андрея Петровича Ершова  проектировались уже упомянутые 
в~связи с~Постановлением ЦК и~СМ процессы. Одним из принципиальных 
решений было максимальное упрощение синтаксиса языка 
программирования, приближение его по экономности средств к~обычному 
математическому языку, б$\acute{\mbox{о}}$льшая, чем в~математическом 
языке, логичность. Реализацией языка занималась группа В.\,Б.~Бетелина 
и~А.\,Г.~Кушниренко.  Рабочая гипотеза тогда состояла в~том, что именно такой 
минимальный, базовый, логично построенный Школьный алгоритмический 
язык и~должен  стать важнейшим компонентом цифровой грамотности.
  
  В последующие десятилетия ядро языка оставалось стабильным, 
развивалась система программирования на его основе, включающая 
структурный редактор, средства пошагового исполнения и~отладки. Сегодня 
эта система обеспечивает учащегося максимальным комфортом в~разработке и~отладке
 программ. Жизнь показала плодотворность такого подхода. 
Обладая сформированной алгоритмической грамотностью, можно осваивать 
новые конструкции и~языки других систем программирования. Но при этом, 
как и~для других видов грамотности, \textbf{решается задача охвата 
грамотностью\linebreak всех детей}, что и~подразумевается в~лозунге Ер\-шова. На 
завершающем этапе общего образования~--- в~старшей школе, которая 
строится как профильная,~--- дальнейшее формирование цифровой\linebreak 
компетенции идет различными путями. Для учащихся c уже сформированной 
цифровой грамот\-ностью, освоивших большие идеи информатики, этого 
достаточно для дальнейшего образования и~ориентации в~мире. При 
подготовке будущих математиков основой  служит Школьный 
алгоритмический язык и~его реализация в~КуМире, где могут строиться 
алгоритмы работы с~математическими объектами, при необходимости 
с~добавлением новых структур данных и~исполнителей. Одновременно 
формируется культура работы c цифровыми инструментами в~математике, 
в~том числе с~системами компьютерной алгебры (пример~--- Wolfram 
Mathematica~[29]). В~ИТ-про\-фи\-лях идет освоение какого-нибудь 
<<производственного>> языка, например одного из языков С, C Sharp, С++, 
Python и~т.\,п.
  
  Остается вопрос: можно ли при формировании начальной цифровой 
грамотности \textbf{после вводного дошкольного нетекстового этапа освоения 
программирования} использовать не минимальный алгоритмический язык,  
а~ка\-кой-то иной конкретный язык программирования? Это возможно, но 
при этом следует учитывать следующие факторы, в~порядке важности:
  \begin{enumerate}[(1)]
\item простота освоения~--- грамотность для всех;
\item наличие исполнителей, сразу используемых для мотивирующего 
решения задач, выполнения проектов;
\item освоение конструкций популярных сегодня языков 
профессионального программирования.
  \end{enumerate}
  
  Первые два фактора реализованы в~проекте\linebreak Лого~--- <<языка без порога 
  и~потолка>>. Возможны и~другие среды, специально созданные для школы, 
  в~которых можно делать что-то интересное для школьника, ценное вне 
программирования. И~осваивая какой-то язык, используемый вне школы, 
можно сформировать базовые представления цифровой грамотности, 
получить опыт работы с~ними, возможно, жертвуя факторами~(1) и~(2).
  
  Ситуация с~исполнителями заслуживает отдельного рассмотрения. 
Безусловно, Робот, как и~Черепашка, не относятся к~\textit{большим идеям}, 
ценным вне контекста первоначального обучения программированию. Однако 
внутри этого контекста они занимают особое положение. То, что Робота 
изобрели, видимо, независимо в~противоположных точках глобуса, служит 
этому подтверждением. 

\section{Большие идеи цифровой грамотности в~ее системе 
целей}
  
  Анализ и~попытки применения самого понятия \textit{большой идеи} 
показывают возможность и~необходимость классификации больших идей по 
широте их применимости. Некоторые из больших идей информатики значимы 
и~вне контекста цифровой грамотности, другие универсальны в~цифровой 
сфере, третьи, все еще оставаясь большими идеями, приложимы в~более 
специальных контекстах. 
  
  Видимо, наиболее актуальной проблемой в~парадигме больших идей 
образования  становится \textbf{выделение больших идей для сферы 
искусственного интеллекта и~подбор учебных ситуаций, задач, 
проектов, где эти идеи будут осваиваться}.
   
\vspace*{-6pt}

\section{Выводы}
\vspace*{-2pt}

  Опыт раннего использования программирования для формирования 
цифровой грамотности в~нашей стране и~за рубежом показывает критическую 
важ\-ность применения визуальных и~материальных (в~том чис\-ле 
бестекстовых) сред для развития такой гра\-мот\-ности как важного элемента 
общего развития современного ребенка.

\vspace*{-6pt}

{\small\frenchspacing
 {%\baselineskip=10.8pt
 %\addcontentsline{toc}{section}{References}
 \begin{thebibliography}{99}

\vspace*{-2pt}

  \bibitem{1-bet}
  \Au{Шварцбурд С.\,И.} Из опыта работы с~учащимися 9 класса, овладевающими 
специальностью лаборантов-про\-грам\-ми\-стов~// Математика в~школе, 1960. №\,5. С.~9--19.
  \bibitem{2-bet}
  \Au{Семенов А.\,Л.} Симор Паперт и~мы. Конструкционизм~--- образовательная 
философия XXI~века~// Вопросы образования, 2017. №\,1. С.~269--294. 
  \bibitem{3-bet}
  \Au{Звенигородский Г.\,А.} Первые уроки программирования~/ Под ред. А.\,П.~Ершова.~--- 
М.: Наука, 1985. 208~с. %(Библиотечка <<Квант>>.)
  \bibitem{4-bet}
\Au{Ershov A.\,P.} Programming, the second literacy~// Computer and Education: 
IFIP TC-3 3rd World Conference on Computer in Education Proceedings.~--- 
Lausanne, Switzerland: North-Holland Pub. Co., 1981. P.~1--17.
  \bibitem{5-bet}
  \Au{Ершов А\, П.} Программирование~--- вторая грамотность: Русская версия доклада~// 
3-й Всемирный конгресс по обучению математике.~--- Лозанна,  Швейцария, 1981. {\sf 
http://ershov.iis.nsk.su/ru/\linebreak second\_literacy/article}. 
  \bibitem{6-bet}
  \Au{Пейперт С.} Переворот в~сознании: дети, компьютеры и~плодотворные идеи~/ Пер. 
  с~англ.~--- М.: Педагогика, 1989. 224~с. (\Au{Papert~S.} Mindstorms: Children, computers and 
powerful ideas.~--- New York, NY, USA: Basic Books Inc. Publs., 1980. 252~p.)
  \bibitem{7-bet}
  \Au{Сопрунов С.\,Ф., Ушакова~А.\,С., Яковлева~Е.\,И.} ПервоЛого~4.0: Справочное 
пособие.~--- М.: Институт новых технологий, 2012. 144~с.
\pagebreak

  \bibitem{8-bet}
  ЛогоМиры 3.0. Интегрированная творческая среда. {\sf http://www.int-edu.ru/content/logomiry-30-integrirovannaya-tvorcheskaya-sreda}.
  \bibitem{9-bet}
    Scratch. Создавай истории, игры и~мультфильмы. Делись с~другими по всему миру. {\sf 
https://scratch.mit.edu}.
  \bibitem{10-bet}
  \Au{Pattis R.\,E.} Karel The Robot: A~gentle introduction to the art of programming.~---  New York, NY, USA: 
John Wiley \& Sons, 1981. 176~p. 
  \bibitem{11-bet}
  \Au{Кушниренко А.\,Г., Лебедев~Г.\,В.} Программирование для математиков.~--- 
М.: Наука, 1988. 384~с.
  \bibitem{12-bet}
  О мерах по обеспечению компьютерной грамот\-ности учащихся средних учебных 
заведений и~широкого внедрения электронно-вычислительной техники в~учебный процесс: 
Постановление ЦК КПСС и~Совета Министров СССР от 28~марта 1985~г.\ №\,271~// 
Вопросы образования, 2005. №\,3. С.~341--346.
{\sf 
https://vo.\linebreak hse.ru/data/2015/04/20/1095612939/22post0.pdf}.
  \bibitem{13-bet}
  \Au{Ершов А.\,П., Кушниренко~А.\,Г., Лебедев~Г.\,В., Семенов~А.\,Л., Шень~А.\,Х.} 
Основы информатики и~вы\-чис\-ли\-тель\-ной техники: Пробный учебник для средних учебных 
заведений~/ Под ред. А.\,П.~Ершова.~--- М.: Просвещение, 1988. 207~с. 
  \bibitem{14-bet}
  \Au{Кушниренко А.\,Г., Лебедев~Г.\,В., Сворень~Р.\,А.} Основы информатики 
  и~вычислительной техники.~--- М.: Просвещение, 1990. 224~с. 

  
  \bibitem{16-bet} %15
  \Au{Звонкин А.\,К., Ландо~С.\,К., Семенов~А.\,Л., Вялый~Н.\,М.} Информатика. 
Алгоритмика. 6--7~классы.~--- М.: Просвещение,  
2006--2008. 
\bibitem{15-bet} %16
  \Au{Кушниренко А.\,Г., Леонов~А.\,Г., Зайдельман~Я.\,Н., Тарасова~В.\,В.} Информатика. 
7--9~классы.~--- М.: Дрофа, 2017. 

  \bibitem{17-bet}
  Проект <<КуМир>>. {\sf https://www.niisi.ru/kumir}.
  \bibitem{18-bet}
  \Au{Richtel M.} Reading, writing, arithmetic, and lately, coding~// New York Times, May~10, 2014. 
{\sf https://www.\linebreak nytimes.com/2014/05/11/us/reading-writing-arithmetic-and-lately-coding.html}.
  \bibitem{19-bet}
  Об утверждении и~введении в~действие федерального государственного образовательного 
стандарта начального общего образования: Приказ министерства образования и~науки РФ 
от~6~октября 2009~г. №\,373. 20~с. {\sf http://base.garant.ru/197127}.
  \bibitem{20-bet}
  \Au{Семенов А.\,Л.} Концептуальные проблемы информатики, алгоритмики 
  и~программирования в~школе~// Вестник кибернетики, 2016. №\,2(22). С.~12--16.
  \bibitem{21-bet}
  \Au{Рудченко Т.\,А., Семенов~А.\,Л.} Информатика. 1--4~классы.~--- М.: Просвещение, Институт новых технологий, 2011--2019.
  \bibitem{22-bet}
  \Au{Papert S.} The connected family: Bringing the digital generation gap.~--- Taylor Trade Publs., 
1996. 211~p.
  \bibitem{23-bet}
  Стартовая страница проекта <<ПиктоМир>>. {\sf 
https://www.niisi.ru/piktomir}.
  \bibitem{24-bet}
  \Au{Бетелин В.\,Б., Кушниренко~А.\,Г., Леонов~А.\,Г.} Основные понятия 
программирования в~изложении для дошкольников~// Информатика и~её применения, 2020. 
Т.~14. Вып.~3. С.~56--62.
  \bibitem{25-bet}
  Портал маам.ру. {\sf https://www.maam.ru/obrazovanie/\linebreak robototehnika}.
  \bibitem{26-bet}
  Возможности мини-робота Bee-Bot: Мастер-класс. {\sf  
  https://www.maam.ru/detskijsad/master-klas-vozmozhnosti-mini-robota-bee-bot.html}.
  \bibitem{27-bet}
  Robots to teach coding and creativity. {\sf https://\linebreak ozobot.com/stem-education}.
  \bibitem{28-bet}
  \Au{Семенов А.\,Л.} Цели общего образования в~цифровом мире~// Информатизация 
образования и~методика электронного обучения: Мат-лы III Междунар. конф.: в~2 ч.~--- 
Красноярск: СФУ, 2019. Ч.~2. С.~383--388.
  \bibitem{29-bet}
  Wolfram Mathematica. {\sf https://www.wolfram.com/\linebreak mathematica}.
\end{thebibliography}

 }
 }

\end{multicols}

\vspace*{-12pt}

\hfill{\small\textit{Поступила в~редакцию 21.11.2019 (последняя правка 
11.09.2020)}}


\vspace*{8pt}

%\pagebreak

%\newpage

%\vspace*{-28pt}

\hrule

\vspace*{2pt}

\hrule

%\vspace*{-2pt}

\def\tit{ABOUT DIGITAL LITERACY AND~ENVIRONMENTS FOR~ITS~DEVELOPMENT}


\def\titkol{About digital literacy and environments for its development}


\def\aut{V.\,B.~Betelin$^1$, A.\,G.~Kushnirenko$^1$, A.\,L.~Semenov$^{2,3,4}$, 
and~S.\,F.~Soprunov$^5$}

\def\autkol{V.\,B.~Betelin, A.\,G.~Kushnirenko, A.\,L.~Semenov, 
and~S.\,F.~Soprunov}

\titel{\tit}{\aut}{\autkol}{\titkol}

\vspace*{-11pt}


\noindent
$^1$Federal State Research Center Scientific Research Institute of System Development,  Russian Academy of 
Sciences,\linebreak
$\hphantom{^1}$36-1~Nakhimovsky Prosp., Moscow 117218, Russian Federation

\noindent
$^2$M.\,V.~Lomonosov Moscow State University, 1~Leninskie Gory, GSP-1, Moscow 119991, Russian 
Federation

\noindent
$^3$Alex Berg Institute of Cybernetics and Educational Computing of
Federal Research Center ``Computer Science\linebreak
$\hphantom{^1}$and Control'' of the Russian Academy of Sciences, 
44-2~Vavilov Str., Moscow 119333, Russian Federation

\noindent
$^4$Moscow Institute of Physics and Technology (National Research University), 9~Institutskiy Per., 
Dolgoprudny,\linebreak
$\hphantom{^1}$Moscow 141701, Russian Federation

\noindent
$^5$Center for Pedagogical Excellence, 15-1~Bol'shaya Spasskaya Str., Moscow 129090, Russian 
Federation


\def\leftfootline{\small{\textbf{\thepage}
\hfill INFORMATIKA I EE PRIMENENIYA~--- INFORMATICS AND
APPLICATIONS\ \ \ 2020\ \ \ volume~14\ \ \ issue\ 4}
}%
 \def\rightfootline{\small{INFORMATIKA I EE PRIMENENIYA~---
INFORMATICS AND APPLICATIONS\ \ \ 2020\ \ \ volume~14\ \ \ issue\ 4
\hfill \textbf{\thepage}}}

\vspace*{3pt} 
  
  


\Abste{ Digital literacy is becoming a key characteristic of a XXI century person. It can and should be 
formed from an early age. Digital environments can be environments for achieving subject, metasubject, 
and personal\linebreak\vspace*{-12pt}}

\Abstend{educational outcomes. The work analyzes domestic and international experience in this 
direction, beginning in 1960s, offers a general view of the ``big ideas'' mastered in digital environments 
and provides a system of principles that ensure the effectiveness for educational purposes of digital 
environments and systems developed under the guidance with the participation of the authors, they are 
effectively used in the Russian Federation in the formation of digital literacy in young children, the basis for 
computational thinking~--- systemic thinking of the XXI~century. The article gives an overview of the 
results obtained in this direction over the last years. Special attention is paid to the problem of visualization 
and representation in the real world of algorithmic processes and the programs that set them.}

\KWE{informatics; computer science; digital literacy; computational thinking; robot; programming 
language; Logo; PervoLogo; LogoFirst; PictoMir; learning environment; visualization}


\DOI{10.14357/19922264200414} 

%\vspace*{-20pt}

\Ack
\noindent
The work was supported by the State Assignment at the FSRC Scientific Research 
Institute of System Development, RAS of the 2020, topic  
0065-2019-0010 (V.\,B.~Betelin and A.\,G.~Kushnirenko), Russian Science Foundation 
(grant No.\,17-11-01377~--- A.\,L.~Semenov,  parts~3 
and~5), and Russian Foundation for Basic Research (grant  
No.\,19-29-14199~--- S.\,F.~Soprunov).

%\vspace*{6pt}

  \begin{multicols}{2}

\renewcommand{\bibname}{\protect\rmfamily References}
%\renewcommand{\bibname}{\large\protect\rm References}

{\small\frenchspacing
 {%\baselineskip=10.8pt
 \addcontentsline{toc}{section}{References}
 \begin{thebibliography}{99}

\bibitem{1-bet-1}
\Aue{Shvartsburd, S.\,I.} 1960. Iz opyta raboty s~uchashchimisya 9~klassa, ovladevayushchimi 
spetsial'nost'yu laborantov-programmistov [From the experience of working with students in grade~9, 
mastering the specialty of laboratory assistants-programmers]. \textit{Matematika v~shkole} 
[Mathematics in School] 5:9--19.
\bibitem{2-bet-1}
\Aue{Semenov, A.\,L.} 2017. Simor Papert i~my. Konstruk\-tsio\-nizm~--- obrazovatel'naya filosofiya 
XXI~veka [Seymour Papert and us. Constructionism as the educational philosophy of the 21st century]. 
\textit{Voprosy obrazovaniya} [Education Studies] 1:269--294. 
\bibitem{3-bet-1}
\Aue{Zvenigorodskiy, G.\,A.} 1985. \textit{Pervye uroki programmirovaniya} 
[First programming lessons]. Moscow: Nauka. 208~p.
\bibitem{4-bet-1}
\Aue{Ershov, A.\,P.} 1981. Programming, the second literacy. 
\textit{Computer and Education: IFIP TC-3 3rd World Conference on Computer 
in Education Proceedings}.~--- Lausanne, Switzerland: North-Holland Pub. Co. 1--17.
\bibitem{5-bet-1}
\Aue{Ershov, A.\,P.} 1981. Programmirovanie~--- vtoraya gramotnost'. Russkaya versiya doklada
[Programming, the second 
literacy. Russian version of lecture].   
\textit{3-y Vsemirnyy kongress po obucheniyu matematike} [3rd World Conference on Computer Education].
Lausanne, Switzerland. Available\linebreak at: {\sf 
http://ershov.iis.nsk.su/ru/second\_literacy/article} (accessed October~26, 
2020). 
\bibitem{6-bet-1}
\Aue{Papert, S.} 1980. \textit{Mindstorms: Children, computers and powerful ideas.} New York, 
NY: Basic Books Inc. Publs. 252~p.
\bibitem{7-bet-1}
\Aue{Soprunov, S.\,F., A.\,S.~Ushakova, and E.\,I.~Yakovleva.} 2012. \textit{PervoLogo~4.0. 
Spravochnoe posobie} [PervoLogo~4.0. Reference manual]. Moscow: Institut novykh tekhnologiy. 
144~p.
\bibitem{8-bet-1}
LogoMiry 3.0. Integrirovannaya tvorcheskaya sreda [LogoMiry~3.0. Integrated Creative Environment]. Available at: {\sf  
 http://www.int-edu.ru/content/logomiry-30-integrirovannaya-tvorcheskaya-sreda} (accessed 
October~26, 2020).
\bibitem{9-bet-1}
Scratsh: Sozdavay istorii, igry i~mul'tfil'my. Delis' s~drugimi po vsemu miru [Scratch: Create stories, 
games, and animations. Share with other around the word]. Available at: {\sf https://scratsh.mit.edu/} 
(accessed October~26, 2020).
\bibitem{10-bet-1}
\Aue{Pattis, R.\,E.} 1981. \textit{Karel the robot: A~gentle introduction to the art of programming}. 
New York, NY: John Wiley \& Sons. 176~p. 
\bibitem{11-bet-1}
\Aue{Kushnirenko, A.\,G., and G.\,V.~Lebedev.} 1988. \textit{Programmirovanie dlya 
matematikov} [Programming for mathematicians]. Moscow: Nauka. 384~p.
\bibitem{12-bet-1}
TsK KPSS i~Sovet Ministrov SSSR. March~28, 1985. O~merakh po obespecheniyu komp'yuternoy 
gramotnosti uchashchikhsya srednikh uchebnykh zavedeniy i~shirokogo vnedreniya 
elektronno-vychislitel'noy tekh\-ni\-ki v~uchebnyy protsess: Postanovlenie No.\,271 [On measures to 
ensure computer 
literacy of secondary school students and the widespread introduction of electronic computing technology 
in the educational process: Decree No.\,271]. Available at:  {\sf 
https://vo.hse.ru/\linebreak data/2015/04/20/1095612939/22post0.pdf} (accessed October~26, 2020).
\bibitem{13-bet-1}
\Aue{Ershov, A.\,P., A.\,G.~Kushnirenko, G.\,V.~Lebedev, A.\,L.~Semenov, and A.\,Kh.~Shen'}. 
1988. \textit{Osnovy informatiki i~vychislitel'noy tekhniki: Probnyy uchebnik dlya srednikh uchebnykh 
zavedeniy} [Fundamentals of computer science and computer technology: A~trial textbook for secondary 
schools]. Moscow: Prosveshchenie. 207~p. 
\bibitem{14-bet-1}
\Aue{Kushnirenko, A.\,G., G.\,V.~Lebedev, and R.\,A.~Svoren'.} 1990. \textit{Osnovy informatiki 
i~vychislitel'noy tekhniki} [Fundamentals of computer science and computer engineering]. Moscow: Prosveshchenie. 224~p. 


\bibitem{16-bet-1} %15
\Aue{Zvonkin, A.\,K., S.\,K.~Lando, A.\,L.~Semenov, and N.\,M.~Vyalyy.} 2006--2008. 
\textit{Informatika. Algoritmika. 6--7~klassy} [Informatics. Algorithmics. Grades 6--7]. Moscow: 
Prosveshchenie. 

\bibitem{15-bet-1} %16
\Aue{Kushnirenko, A.\,G., A.\,G.~Leonov, Ya.\,N.~Zaydel'man, and V.\,V.~Tarasova.} 2017. 
\textit{Informatika. 7--9 klassy} [Informatics.  
7--9~grades]. Moscow: Drofa. 

\bibitem{17-bet-1}
Proekt ``KuMir'' [Project ``KuMir'']. Available at: {\sf https://www.niisi.ru/kumir/} (accessed October~26, 2020).
\bibitem{18-bet-1}
\Aue{Richtel, M.} May~10, 2014. Reading, writing, arithmetic, and lately, coding. \textit{New York Times}. 
Available at: {\sf https://www.nytimes.com/2014/05/11/us/reading-writing-arithmetic-and-lately-coding.html} 
(accessed October~26, 2020).
\bibitem{19-bet-1}
Minobrnauki Rossii. October~6, 2009. Ob utverzhdenii i~vvedenii v~deystvie federal'nogo 
gosudarstvennogo obrazovatel'nogo standarta nachal'nogo obshchego obrazovaniya: Prikaz No.\,373 
[On the approval and implementation of the federal state educational standard of\linebreak primary general 
education: Order No.\,373]. 20~p. Available at: {\sf http://base.garant.ru/197127/} (accessed 
October~26, 2020).
\bibitem{20-bet-1}
\Aue{Semyonov, A.\,L.} 2016. Kontseptual'nye problemy informatiki, algoritmiki i~programmirovaniya 
v~shkole [Conceptual problems of teaching computer science, algorithm studies, and programming at school]. 
\textit{Vestnik kibernetiki} [Proceedings in Cybernetics] 2(22):12--16.
\bibitem{21-bet-1}
\Aue{Rudchenko, T.\,A., and A.\,L.~Semenov.} 2011--2019. \textit{Informatika. 1--4~klassy}
 [Informatics. Grades 1--4]. Moscow: Prosveshchenie, Institut novykh tekhnologiy. 
\bibitem{22-bet-1}
\Aue{Papert, S.} 1996. \textit{The connected family: Bringing the digital generation gap.} Taylor 
Trade Publs. 211~p.

\columnbreak

\bibitem{23-bet-1}
PiktoMir: Startovaya stranitsa proekta [The start page of the 
PictoMir project]. Available at: {\sf 
https://www.niisi.\linebreak ru/piktomir/} (accessed October~26, 2020).
\bibitem{24-bet-1}
\Aue{Betelin, V.\,B., A.\,G.~Kushnirenko, and A.\,G.~Leonov.} 2020. Osnovnye ponyatiya 
programmirovaniya v~iz\-lo\-zhe\-nii dlya doshkol'nikov [Basic programming concepts for preschoolers]. 
\textit{Informatika i~ee Pri\-me\-ne\-nieya~--- Inform. Appl.} 14(3):56--62.
\bibitem{25-bet-1}
Portal maam.ru. Available at: {\sf  https://www.maam.ru/\linebreak obrazovanie/robototehnika} (assessed 
October~26, 2020).
\bibitem{26-bet-1}
Vozmozhnosti mini-robota Bee-bot. Master-klass [Features of the Bee-bot mini-robot. Master class]. 
Available at: {\sf https://www.maam.ru/detskijsad/master-klas-vozmozhnosti-mini-robota-bee-bot.html} 
(accessed\linebreak October~26, 2020).
\bibitem{27-bet-1}
Robots to teach coding and creativity. Available at: {\sf https://ozobot.com/stem-education} (accessed 
October~26, 2020).
\bibitem{28-bet-1}
\Aue{Semenov, A.\,L.} 2019. Tseli obshchego obrazovaniya v~tsif\-ro\-vom mire [The goals of general 
education in the digital world]. \textit{Mat-ly III Mezhdunar. konf. ``Informatizatsiya obrazovaniya 
i~metodika elektronnogo obucheniya''} [3rd  Conference (International) ``Informatization of 
Education and the Methods of Electronic Learning'' Proceedings]. Krasnoyarsk: SFU. 2:383--388.
\bibitem{29-bet-1}
Wolfram Mathematica. Available at: {\sf https://www.\linebreak wolfram.com/mathematica/} (accessed October~26, 
2020).
\end{thebibliography}

 }
 }

\end{multicols}

\vspace*{-3pt}

\hfill{\small\textit{ Received November~21, 2019 (last revision September~11, 
2020)}}


%\pagebreak

%\vspace*{-24pt}


\Contr

\noindent
\textbf{Betelin Vladimir B.} (b.\ 1946)~--- %Doctor of Science in physics and mathematics, 
professor, 
Academician of RAS, scientific director, Federal State Research Center 
Scientific Research Institute of System 
Development, Russian Academy of Sciences, 36-1~Nakhimovsky Prosp., Moscow 117218, Russian 
Federation; \mbox{betelin@niisi.msk.ru}

\vspace*{3pt}

\noindent
\textbf{Kushnirenko Anatoliy G.} (b.\ 1944)~--- Candidate of Science (PhD) in physics and mathematics, 
Head of Department of Educational Informatics, Federal State Research Center Scientific Research Institute of System 
Development, Russian Academy of Sciences,'' 36-1~Nakhimovsky Prosp., Moscow 117218, Russian Federation; 
\mbox{agk\_@mail.ru}

\vspace*{3pt}

\noindent
\textbf{Semenov Alexei L.} (b.\ 1950)~--- %Doctor of Science in physics and mathematics, 
professor, 
Academician of RAS, Academician of the Russian Academy of Education, Head of Department, 
M.\,V.~Lomonosov Moscow State University, 1~Leninskie Gory, GSP-1, Moscow 119991, Russian 
Federation; director, Alex Berg Institute of Cybernetics and Educational Computing, Federal Research Center 
``Computer Science and Control'' of the Russian Academy of Sciences, 44-2~Vavilov Str., Moscow 
119333, Russian Federation; principal scientist, Moscow Institute of Physics and Technology (National 
Research University), 9~Institutskiy Per., Dolgoprudny, Moscow 141701, Russian Federation; 
\mbox{alsemno@ya.ru}

\vspace*{3pt}

\noindent
\textbf{Soprunov Sergey F.} (b.\ 1949)~--- Candidate of Science (PhD) in physics and mathematics, 
methodologist, Center for Pedagogical Excellence, 15-1~Bol'shaya Spasskaya Str., Moscow 129090, 
Russian Federation; \mbox{soprunov@mail.ru}

\label{end\stat}

\renewcommand{\bibname}{\protect\rm Литература} 
  