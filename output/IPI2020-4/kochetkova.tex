%есть рисунки

\def\stat{kochetkova}

\def\tit{СИСТЕМА МАССОВОГО ОБСЛУЖИВАНИЯ С~ОРБИТАМИ ДЛЯ~АНАЛИЗА
 СОВМЕСТНОГО ОБСЛУЖИВАНИЯ ТРАФИКА С~МАЛЫМИ ЗАДЕРЖКАМИ URLLC 
 И~ШИРОКОПОЛОСНОГО ДОСТУПА eMBB В~БЕСПРОВОДНЫХ СЕТЯХ ПЯТОГО ПОКОЛЕНИЯ$^*$}

\def\titkol{Система массового обслуживания с~орбитами для анализа 
совместного обслуживания трафика}
% с~малыми задержками URLLC и~широкополосного 
%доступа 
%\mbox{eMBB} в~беспроводных сетях пятого поколения}

\def\aut{П.\,А.~Харин$^1$, Е.\,Д.~Макеева$^2$, И.\,А.~Кочеткова$^3$,
 Д.\,В.~Ефросинин$^4$, С.\,Я.~Шоргин$^5$}

\def\autkol{П.\,А.~Харин, Е.\,Д.~Макеева, И.\,А.~Кочеткова и~др.}
%, Д.\,В.~Ефросинин$^4%, С.\,Я.~Шоргин$^5$}

\titel{\tit}{\aut}{\autkol}{\titkol}

\index{Харин П.\,А.}
\index{Макеева Е.\,Д.}
\index{Кочеткова И.\,А.}
\index{Ефросинин Д.\,В.}
\index{Шоргин С.\,Я.}
\index{Kharin P.\,A.}
\index{Makeeva E.\,D.}
\index{Kochetkova I.\,A.}
\index{Efrosinin D.\,V.}
\index{Shorgin S.\,Ya.}

{\renewcommand{\thefootnote}{\fnsymbol{footnote}} \footnotetext[1]
{Публикация подготовлена при поддержке Программы 
РУДН <<5-100>>. Исследование выполнено при финансовой поддержке РФФИ 
(проекты 18-00-01555(18-00-01685) и~20-37-70079).}}


\renewcommand{\thefootnote}{\arabic{footnote}}
\footnotetext[1]{Российский университет дружбы народов, {pxarin@mail.ru}}
\footnotetext[2]{Российский университет дружбы народов, {len16730637@yandex.ru}}
\footnotetext[3]{Российский университет дружбы народов; Институт 
проблем информатики Федерального исследовательского центра <<Информатика и~управление>> 
Российской академии наук, \mbox{gudkova-ia@rudn.ru}}
\footnotetext[4]{Университет имени Иоганна Кеплера, Австрия; 
Российский университет дружбы народов, \mbox{dmitry.efrosinin@jku.at}}
\footnotetext[5]{Институт проблем информатики Федерального исследовательского центра 
<<Информатика и~управление>> Российской академии наук, \mbox{sshorgin@ipiran.ru}}


\vspace*{-10pt}


\Abst{Для современных беспроводных сетей пятого поколения (5G) характерны три 
сценария их использования~--- крупномасштабные системы межмашинной связи 
(massive machine-type communication, mMTC), сверхнадежная передача данных с~малой задержкой (ultrareliable 
low-latency communication, URLLC) и~усовершенствованная подвижная широкополосная 
связь (enhanced mobile broadband, \mbox{eMBB}). Требования к~качеству предоставления услуг и~их 
параметры в~рамках данных сценариев существенно разнятся: так, для URLLC характерна сверхнизкая, 
до 1~мс, задержка, а для \mbox{eMBB}~--- сверхвысокая скорость передачи данных. Возникает актуальная задача 
организации совместного предоставления таких услуг. В~статье построена схема совместного обслуживания 
трафика URLLC и~\mbox{eMBB}, исходя из того что данные URLLC имеют малый объем и~могут занимать менее одного
 ресурсного блока (physical resource block, PRB). Анализ схемы проведен при помощи разработанной 
 системы массового обслуживания (СМО) с~двумя орбитами, предназначенными для временного хранения 
 прерываемых и~ожидающих начала обслуживания менее приоритетных запросов \mbox{eMBB}. Получен матричный алгоритм 
 для расчета распределения вероятностей и~формулы для вероятностных характеристик системы.}

\KW{5G; сверхнадежная передача данных с~малой задержкой (URLLC); усовершенствованная подвижная 
широкополосная связь (\mbox{eMBB}); система массового обслуживания с~орбитами; прерывание обслуживания}

\DOI{10.14357/19922264200403} 
  
%\vspace*{9pt}


\vskip 10pt plus 9pt minus 6pt

\thispagestyle{headings}

\begin{multicols}{2}

\label{st\stat}


\section{Введение}

С развитием технологий беспроводной передачи данных увеличивается объем генерируемой и~передаваемой
 информации. Согласно прогнозам Cisco Systems, рост числа мобильных устройств к~2021~г.\ составит 
 до 11,6~млрд при 8~млрд в~2016~г. 
 
 Для сетей 5G выделяют три сценария их 
 использования в~соответствии со следующими характеристиками: сверхвысокая скорость передачи данных, 
 спектральная эффективность, плотность\linebreak подключения устройств, сверхнизкая задержка и~сверхвысокая 
 надежность передачи данных. Сценарии~--- крупномасштабные системы межмашинной связи (mMTC), 
 сверхнадежная передача данных с~малой задержкой (URLLC) 
 и~усовершенствованная подвижная широкополосная связь 
 (\mbox{eMBB})~\cite{Popovski2018}.
 
  Для mMTC специфична высокая плотность подключения устройств, что необходимо 
  для приложений умного города, умного дома, индустриального интернета вещей. 
  
  Сценарий URLLC используется для 
  передачи небольшого объема данных, но с~высокой надежностью и~очень низкими задержками для обеспечения 
  мониторинга в~реальном времени, например сети беспилотных автомобилей или беспилотных летательных 
  аппаратов~\cite{Cheng2020, Tun2020}. 
  
  Сценарий \mbox{eMBB} используется для предоставления пользователю высоких скоростей
   передачи, которые необходимы, например, для потоковой передачи видео в~реальном времени с~высокой 
   чет\-костью, виртуальной реальности,  дополненной реальности, телемедицины и~дистанционного обучения.


Требования к~качеству обслуживания для каж\-до\-го сценария существенно различаются, поэтому возникает 
задача организации совместного обслуживания соответствующего трафика~\cite{DosSantos2020, Kim2020}. 
В~данной работе построена модель совместного обслуживания трафика URLLC и~\mbox{eMBB} в~условиях, когда 
высший приоритет имеет трафик URLLC~\cite{Anand2020, Li2020}. Ввиду приоритетного обслуживания URLLC, 
допустимо прерывание \mbox{eMBB}, как, например, в~работах~\cite{Gudkova2012, Makeeva2019}. 

Авторы продолжают 
исследование~\cite{Makeeva2019}, добавив более гибкий механизм в~отношении трафика \mbox{eMBB}~--- 
допустимо возобновление его обслуживания после прерывания.


\section{Системная модель обслуживания трафика URLLC~и~\mbox{eMBB}}

Рассмотрим выделенную соту беспроводной сети, в~которой реализованы сценарии URLLC и~\mbox{eMBB}. 
Предположим, что запросы на предостав\-ле\-ние услуг \mbox{eMBB} и~URLLC~--- передачу соответствующего 
трафика~--- поступают согласно пуассоновскому закону с~интенсивностями~$\lambda_{m}$ и~$\lambda_{u}$ соответственно. Трафик URLLC имеет высокие требования к~задержке и~надежности 
передачи данных, поэтому является приоритетным по отношению к~\mbox{eMBB}.

Организованы два буфера для устройств \mbox{eMBB}. Первый буфер~$Q_1$ предназначен для устройств 
\mbox{eMBB} с~задержкой передачи данных, т.\,е.\ для устройств \mbox{eMBB}, которые не смогли поступить на 
обслуживание в~связи с~отсутствием свободных ресурсов. Второй буфер~$Q_2$ предназначен для прерванных 
\mbox{eMBB}-устрой\-ств новыми поступившими URLLC-устрой\-ст\-ва\-ми в~систему. Буферы~$Q_1$ и~$Q_2$ системы 
ограничены~$C_1$ и~$C_2$ соответственно. Пример такой соты показан на рис.~1.



Фрагмент канала такой системы показан на  рис.~2. Данный канал состоит 
из $N$ физических ресурсных блоков (PRB), которые, в~свою очередь, делятся на~$b$ ресурсных единиц. 
Длительность одного PRB равна 1~слоту ($\sim0{,}5$~мс). Для удовлетворения требований устройств 
URLLC к~задержке передачи данных каждый слот делится на минислоты.

Так как трафик \mbox{eMBB} предназначен для передачи данных большого объема, а~трафик URLLC~--- для 
передачи небольшого объема данных, то планирование передачи данных устройств \mbox{eMBB}\linebreak
 проводится относительно 
PRB, а~устройств URLLC~--- относительно ресурсных единиц
(resource unit, RU). Совместное планирование обслуживания проводится 
на разных частотах и~в~разное время для каждого типа трафика согласно\linebreak ортогональному множественному доступу 
(orthogonal multiple access, OMA). Для рассматриваемой модели сделаем следующее предположение по нагрузке:
 каждое устройство \mbox{eMBB}
 требует $1$~PRB, т.\,е.\ $b$ ресурсных единиц, а~каждое устройство URLLC~--- $1$~ресурсную единицу.
Для того чтобы принять устройство \mbox{eMBB}, необходимо наличие хотя бы одного свободного PRB, 
при этом буферы должны быть пусты. Таким образом, если данные условия выполнены, планировщик назначает 
один из свободных PRB, $b$ ресурсных единиц, новому устройству \mbox{eMBB}. Иначе, при наличии хотя бы одного 
свободного места в~первом буфере, планировщик записывает это устройство в~буфер~$Q_1$, в~противном случае 
это устройство будет заблокировано.

{ \begin{center}  %fig1
 \vspace*{12pt}
     \mbox{%
 \epsfxsize=78.325mm 
 \epsfbox{koc-1.eps}
 }

\vspace*{6pt}

\noindent
{{\figurename~1}\ \ \small{
Схема соты с~устройствами URLLC и~\mbox{eMBB}
}}
\end{center}}

\vspace*{6pt}



{ \begin{center}  %fig2
 \vspace*{3pt}
    \mbox{%
 \epsfxsize=79mm 
 \epsfbox{koc-2.eps}
 }

\end{center}

\noindent
{{\figurename~2}\ \ \small{
Схема занятия радиоресурсов устройствами \mbox{URLLC}~(\textit{1}) и~\mbox{eMBB}~(\textit{2})
}}}

%\vspace*{6pt}

\setcounter{figure}{2}
\begin{figure*}[b] %fig3
\vspace*{-3pt}
 \begin{center}  
 \mbox{%
 \epsfxsize=136.498mm 
 \epsfbox{koc-3.eps}
 }
\end{center}
\vspace*{-9pt}
\Caption{Схема СМО с~двумя орбитами}
\label{fig_QS}
\end{figure*}


В случае же устройства URLLC для принятия к~обслуживанию достаточно наличия хотя бы одной ресурсной 
единицы в~системе, иначе возможны два исхода событий. Если в~системе обслуживается хотя бы одно устройство
 \mbox{eMBB}, то планировщик приостанавливает передачу данных случайно выбранного устройства \mbox{eMBB}, таким образом
  освободив один PRB, и~назначает одну ресурсную единицу освободившегося PRB новому устройству URLLC. 
  А~прерванный сеанс записывается во второй буфер для прерванных устройств \mbox{eMBB} для последующего 
  восстановления соединения. Если все ресурсные единицы заняты под устройства URLLC, то новое устройство 
  URLLC блокируется.
После окончания обслуживания каждого из устройств освобождаются ресурсные единицы и~больше ничего не 
происходит, т.\,е.\ устройства в~буферах продолжают свое ожидание обслуживания.

 Устройства из первого буфера~$Q_1$ и~второго буфера~$Q_2$ пытаются перезапустить или возобновить\linebreak 
 обслуживание через экспоненциально распределенное время с~параметрами 
 $ \gamma_ {1}\hm> 0 $ и~$ \gamma_ {2}\hm> 0 $ соответственно. Поскольку прерванные 
 сеансы \mbox{eMBB} должны быть возобновлены как можно скорее,  буфер~$Q_2$ считается приоритетным. 
 Таким образом, устройство \mbox{eMBB} с~задержкой из буфера~$Q_1$ может установить соединение только в~том случае, 
 когда буфер~$Q_2$ полностью свободен и~в~системе есть хотя бы один свободный PRB. А~для восстановления сеанса 
 прерванного устройства \mbox{eMBB} достаточно наличия хотя бы одного свободного PRB в~системе, несмотря на состояние
  первого буфера. В~случае когда устройства с~буферов не могут быть приняты к~обслуживанию, они продолжают
   ожидание.

\section{Система массового обслуживания с~орбитами}

Рассматриваемую системную модель можно представить в~виде СМО с~двумя 
орбитами с~исходными параметрами, описанными выше. Схема такой СМО показана на рис.~\ref{fig_QS}.

Функционирование данной СМО опишем четырехмерным марковским случайным процессом (СП): 
$$
\vec{X}(t)_{t\geq 0}=\left(N_{{m}}(t), N_{{u}}(t),Q_{1}(t), Q_{2}(t)\right)_{t \geq 0},
$$ 
где $N_{{m}} (t)$ и~$N_{{u}} (t)$~--- число сессий соответственно типа \mbox{eMBB} и~URLLC в~зоне 
обслуживания в~момент времени~$t$;
$Q_{1} (t)$~--- число новых сессий \mbox{eMBB} на орбите~1; $Q_{2} (t)$~--- число прерванных сессий 
\mbox{eMBB} на орбите~2  в~момент времени~$t$.

Пространство состояний СП имеет следующий вид:
\begin{multline}
\hspace*{-8pt}\mathcal{N}=   \left\{ \left( n_{{m}},n_{{u}},q_{1},q_{2} \right):n_{{m}}\ge 0,n_{{u}}\ge 0, 
    0\le q_{1}\le C_{1},\right.\\
   \left. 0\le q_{2}\le C_{2}, bn_{{m}}+n_{{u}}\le C,n_{{m}}+q_{2}\le N\right\}\,,
      \label{eq_N}
\end{multline}
где  $n_{{m}}$ и~$n_{{u}}$--- соответственно число \mbox{eMBB}- и~\mbox{URLLC}-сес\-сий на 
обслуживании; 
$q_1$ и~$q_2$~--- число \mbox{eMBB}-сес\-сий соответственно на орбите~$Q_{1}$ и~$Q_{2}$. Обозначим через 
$ \pi \left( n_{m},n_{u},q_1,q_2 \right)$  вероятность нахождения сис\-те\-мы в~со\-сто\-янии 
$\left( n_{m},n_{u},q_1,q_2 \right)$.



\section{Алгоритм расчета распределения вероятностей}

Получим рекуррентный алгоритм для расчета финального распределения вероятностей состояний  
рассматриваемой системы. Матрица интенсивностей переходов представима 
в~блоч\-но-трех\-диа\-го\-наль\-ном виде, представленном ниже в~виде утверждения~1.

\smallskip

\noindent
\textbf{Утверждение 1.}\ \textit{
%\label{utv_1}
Если на множестве}~(\ref{eq_N}) \textit{введен лексикографический порядок}
\begin{equation*} 
%\label{lexicographic}
    \begin{array}{l}
(n'_{{m}},n'_{{u}},q'_{1},q'_{2})>(n_{{m}},n_{{u}},q_{1},q_{2})<=>q'_{1}>q_{1}\\ 
\mbox{\textit{или}}\  
        (q'_{1}=q_{1},(q'_{2}>q_{2} \mbox{\textit{ или}}\
        ( q'_{2}=q_{2},(n'_{{m}}>n_{{m}}\\
\mbox{\textit{или}}~(n'_{{m}}=n_{{m}},n'_{{u}}>n_{{u}}))))),
    \end{array}
\end{equation*}
 %\begin{enumerate}
    %\item
   \textit{то матрица интенсивностей переходов СП представима в~следующем блочном трехдиагональном виде}:
    \begin{equation*}
    \mathbf\Lambda=
    %\left(
    \begin{array}{|cccccccc|}
    \mathbf\Lambda_{00}&\mathbf\Lambda_{11}&0&0&0&\cdots&\cdots&0
    \\
    \mathbf\Lambda_{21}&\mathbf\Lambda_{01}&\mathbf\Lambda_{12}&0&0&\cdots&\cdots&0
    \\
    0&\mathbf\Lambda_{22}&\mathbf\Lambda_{02}&\mathbf\Lambda_{13}&0&\cdots&\cdots&0
    \\
    \cdots&\cdots &\cdots &\cdots &\cdots&\cdots&\cdots&\cdots
    \\
    0&0&0&0&0&0&\mathbf\Lambda_{2C_1}&\mathbf\Lambda_{0C_1}
    \end{array}\,;
    %\right)
    \end{equation*}
\textit{размерности блоков матрицы $\Lambda$ представимы в~виде}:
    \begin{equation*}
    \begin{array}{l}
       \mathbf\Lambda_{0k}:\,\mathrm{dim}\,(N(k))\times \mathrm{dim}\,(N(k)),\ k=0,\ldots ,C_1;
       \\
       \mathbf\Lambda _{1k}:\,\mathrm{dim}\,(N(k-1))\times \mathrm{dim}\,(N(k)),\ k=1,\ldots ,C_1;
       \\
       \mathbf\Lambda_{2k}:\,\mathrm{dim}\,(N(k))\times \mathrm{dim}\,(N(k-1)),\ k=1,\ldots ,C_1,
    \end{array}
    \end{equation*}
    \textit{где} 
    $$\mathrm{dim}\,(N(k))=\sum\limits_{q_2=0}^{C_2}
    {\sum\limits_{n_{m}=0}^{N-q_2}{\sum\limits_{n_{u}=0}^{C-bn_{m}}1}};
    $$

    \noindent
    \textit{ненулевые элементы блоков $~\mathbf{\Lambda}_{1k}$, 
    $\mathbf{\Lambda}_{2k}$ и~$\mathbf{\Lambda} _{0k}$ мат\-ри\-цы~$\mathbf{\Lambda} $ выглядят 
    следующим образом}:
\begin{multline*}
   % \label{eq_Lambda1k}
    \mathbf\Lambda_{1k}\left(
     (n_{m},n_{u},q_1,q_2), (n'_{m},n'_{u},q'_1,q'_2)
    \right)={}\\
    {}=\left\{
 \lambda_{m},
 q'_1=q_1+1,\ q_1\neq 0~\mbox{или}\ q_2\neq 0;
 \right.
 \end{multline*}
    
  %  \vspace*{-12pt}
 
 \noindent
 \begin{multline*}
% \label{eq_Lambda2k}
 \mathbf\Lambda_{2k}\left(
  (n_{m},n_{u},q_1,q_2),
  (n'_{m},n'_{u},q'_1,q'_2)
 \right)={}\\
 {}=\left\{
 \gamma_1,
 n_{m}'=n_{m}+1,~q'_1=q_1-1, \right.\\
 bn_{m}+n_{u}<C-b,~q_2=0,~q_1\neq0;
 \end{multline*}
    
    \vspace*{-12pt}

    \noindent
    \begin{multline*}
   % \label{eq_Lambda0k}
     \mathbf\Lambda_{0k}\left(
     %\begin{array}{l}
     (n_{m},n_{u}, q_1,q_2), \left(n'_{m},n'_{u}, q'_1,q'_2\right)
     %\end{array}
     \right)={}\\
     {}=\!
     \left\{
     \begin{array}{ll}
     \lambda_{u}, & n'_{m}=n_m,~n'_{u}=n_{u}+1,~q'_1=q_1,\\
     &\hspace*{44.5mm}q'_2=q_2; \\
&    \mbox{или}\\
     &
     n'_{m}=n_{m}-1,~n'_{u}=n_{u}+1,\\
     &\hspace*{38mm}q'_2=q_2+1;
     \\[2.5pt]
     n_{u}\mu_{u},&
     n'_{m}=n_{m},~n'_{u}=n_{u}-1,q'_1=q_1,\\
     &\hspace*{44.5mm}q'_2=q_2;
     \\[2.5pt]
     \lambda_{m},&
     n'_{m}=n_{m}+1,~n'_{u}=n_{u},~q'_1=q_1=0,\\
     &
     \hspace*{38mm}q'_2=q_2=0;
     \\[2.5pt]
     n_{m}\mu_{m},& n'_{m}=n_{m}-1,n'_{u}=n_{u},~q'_1=q_1,\\
     &\hspace*{44.5mm}q'_2=q_2;
     \\[2.5pt]
     \gamma_2, & n'_{m}=n_{m}+1,~n'_{u}=n_{u},~q'_1=q_1,\\
     &
     \hspace*{38.5mm}q'_2=q_2-1;
     \\[2.5pt]
     *, &
     n'_{m}=n_{m},~n'_{u}=n_{u},~q'_1=q_1,~q'_2=q_2,
     \end{array}
    \right.
    \end{multline*}
    \textit{где $*$ равно сумме всех ненулевых элементов матрицы~$\mathbf\Lambda$ по строке 
    со знаком минус}.


\smallskip

После упорядочения пространства состояний согласно утверждению выше, 
получим алгоритм нахождения финальных вероятностей в~форме утверждения~2.

\smallskip

\noindent
\textbf{Утверждение 2.}
\label{utv_2}
\textit{Компоненты $\boldsymbol{\pi} _{k}$, $k\hm=\overline {0,{C_1} - 1}$, вектора 
финальных вероятностей $\boldsymbol{\pi}$ можно вы\-чис\-лить, используя подход 
<<блок\,--\,впе\-ред\,--\,исклю\-че\-ние\,--\,обрат\-ное замещение>>, согласно формуле}
\begin{equation*}
{{\boldsymbol{\pi }}_k} = {{\boldsymbol{\pi }}_{{C_1}}}\prod\limits_{j = 1}^{N - k}
{{{\mathbf{M}}_{{C_1} - j}}} ,\enskip k = \overline {0,{C_1} - 1},
%\label{eq:2}
\end{equation*}
\textit{где компонента $\boldsymbol{\pi} _{C_1}$ является единственным решением системы уравнений}
\begin{align*}
%\left\{ \begin{array}{l}
\sum\limits_{k = 0}^{{C_1}} \left( \boldsymbol{\pi }_{C_1}\prod\limits_{j =1}^{C_1 - k} 
\mathbf{M}_{{C_1} - j}  \right)\mathbf{e} &= \mathbf{1} ;\\
\boldsymbol{\pi }_{C_1}\left(\boldsymbol{\Lambda }_{0{C_1}} + \boldsymbol{\Lambda
}_{2{C_1}}{{\mathbf{M}}_{{C_1} - 1}}\right) &= \mathbf{0},
%\end{array} \right.
%\label{eq:3}
\end{align*}
\textit{где $\mathbf{e}$~--- вектор-стол\-бец из единиц размерностью $\mathrm{dim}\,\mathcal{N}(k)$,
 а матрицы~$\mathbf{M}_{C_1-j}$ вычисляются следующим образом}:
\begin{equation*} 
%\label{GrindEQ__2_64_}
\begin{array}{l} 
{\boldsymbol{M}_{0}  = -\boldsymbol{\Lambda} _{11} \cdot \boldsymbol{\Lambda} _{00} {}^{-1} }; \\[6pt] 
\boldsymbol{M}_{k} =-\boldsymbol{\Lambda} _{1 \ k+1} \left(\boldsymbol{\Lambda} _{0k} +\boldsymbol{M}_{k-1}
 \cdot \boldsymbol{\Lambda} _{2k} \right)^{-1},\\[6pt]
\hspace*{44mm}k=\overline{1,C_1-1}.  
\end{array}
\end{equation*}

\begin{figure*}[b] %fig4
\vspace*{-6pt}
 \begin{center}  
 \mbox{%
 \epsfxsize=163mm 
 \epsfbox{koc-4.eps}
 }
\end{center}
\vspace*{-6pt}
    \Caption{Первый сценарий: (\textit{а})~среднее число активных сессий \mbox{eMBB};
    (\textit{б})~вероятность прерывания сессии \mbox{eMBB};
    \textit{1}~--- $\lambda_u\hm=40$; \textit{2}~--- 80; \textit{3}~--- $\lambda_u\hm=120$}
    \label{fig_Ne}
\end{figure*}

\section{Алгоритм расчета вероятностных характеристик модели}

Получив финальное распределение вероятностей по утверждению~2, можно рассчитать характеристики 
рассматриваемой модели.
\begin{enumerate}[1.]
    \item Вероятность блокировки новой сессии URLLC:
    \begin{equation*}
    B_{u}=\sum\limits_{q_1=0}^{C_1}{\sum\limits_{q_2=0}^{C_2}\pi\left(0,C,q_1,q_2\right)}.
    \end{equation*}
    \item Вероятность задержки передачи новой сессии \mbox{eMBB}:
    \begin{equation*}
    D=\sum\limits_{\left(n_{m},n_{u},q_1,q_2\right)\in {\cal{N}}_{D}}\!\!\!\! \pi\left(n_{m},n_{u},q_1,q_2\right),
    \end{equation*}
    где 
    $\mathcal N_{D}=\left(n_{m},n_{u},q_1,q_2\right)\hm 
    \in \mathcal N: \left( b\left( n_{m}+1\right)+\right.$\linebreak
    $+\;n_{u} \hm\le C$, $q_1\hm<C_1$, $q_2\hm<C_2$,
    $\left(\left(q_1\neq 0\right)\right.$ {или} 
    $\left.\left.\left(q_2\hm\neq 0 \right)\right)\right)$, {или} 
    $\left(b\left(n_{m}\hm+1\right)n_{u}\hm>C,q_1\hm<C_1\right)$~--- 
    множество состояний, где новое устройство \mbox{eMBB} будет отправлено на орбиту $Q_ {1}$.
    \item Вероятность прерывания сессии \mbox{eMBB}:
    \begin{multline*}
   \hspace*{-8mm} I=\!\!\!\sum\limits_{n_{m} =1}^{N}\sum\limits_{q_{1} =0}^{C_{1} }
   \sum\limits_{q_{2} =0}^{C_{2} }\!
    \left(\!\fr{\pi\left(n_{m} ,C-bn_{m} ,q_{1} ,q_{2} \right)}{n_m}\right. \lambda_u  /
     \left(\lambda _{u} +\right.\hspace*{-0.4pt} \\
\left. \hspace*{-4.69067mm}{}+ \left.{} n_{m} \mu _{m} +n_{u} \mu _{u}
   1(A)+\lambda _{m} +\gamma _{1} 1(B)+\gamma _{2} 1(C)\right)
   \vphantom{\fr{\pi\left(n_{m} ,C-bn_{m} ,q_{1} ,q_{2} \right)}{n_m}}
     \!\!\right)\!,\hspace*{-6.28122pt}
    \end{multline*}
    где $A=n_{u}>0$, $B\hm=b\left( n_{m}\hm+1 \right)\hm+n_{u}\hm\le C$, $q_1\hm>0$, $q_2\hm=0$;     
    $C\hm=b\left( n_{m}\hm+1 \right)\hm+n_{u} \hm\le C$, $q_2\hm>0$.
    \item Вероятность блокировки новой сессии \mbox{eMBB}:
    \begin{equation*}
    B_{m}=
    \sum\limits_{n_{m}=0}^{N}
    {\sum\limits_{n_{u}=0}^{C}
    {\sum\limits_{q_2=0}^{C_2}
    {\pi\left(n_{m},n_{u},q_1,q_2 \right).}}}
    \end{equation*}
    \item Среднее число обслуживаемых устройств URLLC в~системе (активных сессий URLLC):
    \begin{equation*}
        \overline{n_{u}}=
        \sum\limits_{n_{u}=1}^{C}n_{u}\!
        {\sum\limits_{n_{m}=0}^
        {\left\lfloor ({C-n_{u}})/{b} \right\rfloor}
\!        {\sum\limits_{q_1=0}^{C_1}
        {\sum\limits_{q_2=0}^{C_2}
        {\pi \left(n_{m},n_{u},q_1,q_2 \right)}}}}.
    \end{equation*}
    \item Среднее число обслуживаемых устройств \mbox{eMBB} в~системе (активных сессий \mbox{eMBB}):
    \begin{equation*}
    \overline{n_{m}}=
        \sum\limits_{n_{m}=1}^{N}n_{m}
        {\sum\limits_{n_{u}=0}^{C-bn_{m}}
        {\sum\limits_{q_1=0}^{C_1}
        {\sum\limits_{q_2=0}^{C_2}
        {\pi \left(n_{m},n_{u},q_1,q_2 \right)}}}}.
    \end{equation*}
    \item Среднее число устройств \mbox{eMBB} с~задержкой передачи данных, т.\,е.\ 
    среднее число сессий \mbox{eMBB} на орбите~$Q_1$:
    \begin{equation*}
        \overline{q_1}=
        \sum\limits_{q_1=1}^{C_1}
        q_1
        {\sum\limits_{n_{m}=0}^{N}
        {\sum\limits_{n_{u}=0}^{C-bn_{m}}
        {\sum\limits_{q_2=0}^{C_2}
        {\pi\left(n_{m},n_{u},q_1,q_2 \right)}}}}.
    \end{equation*}
    \item Среднее число прерванных устройств \mbox{eMBB}, т.\,е.\ среднее число сессий \mbox{eMBB} на орбите~$Q_2$:
    \begin{equation*}
         \overline{q_2}=
        \sum\limits_{q_2=1}^{C_2}
        q_2
        {\sum\limits_{n_{m}=0}^{N}
        {\sum\limits_{n_{u}=0}^{C-bn_{m}}
        {\sum\limits_{q_1=0}^{C_1}
        {\pi\left(n_{m},n_{u},q_1,q_2 \right)}}}}.
    \end{equation*}
\end{enumerate}

\vspace*{-9pt}


\section{Пример численного анализа}

\vspace*{-2pt}


\begin{figure*} %fig5
\vspace*{1pt}
 \begin{center}  
 \mbox{%
 \epsfxsize=163mm 
 \epsfbox{koc-5.eps}
 }
\end{center}
\vspace*{-6pt}
    \Caption{Второй сценарий: (\textit{а})~вероятность блокировки новой сессии URLLC;
    (\textit{б})~среднее число сессий \mbox{eMBB} на орбите~$Q_1$;  \textit{1}~--- 
    $\lambda_m\hm=40$; \textit{2}~--- 80; \textit{3}~--- $\lambda_m\hm=120$}
    \label{fig_Bu}
\end{figure*}


Для проведения численного анализа рас\-смот\-рим следующие исходные данные: $N \hm= 6$ 
ресурсных блоков по $b \hm= 12$ ресурсных единиц; $C_1 \hm= C_2 \hm= 6$ мест на 
орбитах~$Q_1$ и~$Q_2$. Анализ проведем для двух сценариев: в~первом сценарии рас\-смат\-ри\-ва\-ет\-ся зависимость 
от изменения  ин\-тен\-сив\-ности поступления  сессий \mbox{eMBB}~$\lambda_{m}$ с~шагом~$40$
(рис.~4),\linebreak\vspace*{-12pt}

\pagebreak

\noindent
 во втором~--- от 
изменения интенсивности поступления сессий URLLC~$\lambda_{u}$ с~тем же шагом
(рис.~5). Среднее время 
обслуживания сессий фиксированное: $\mu_{{m}}^{- 1} \hm= 1^{- 1}$, $\mu_{{u }}^{- 1} 
\hm= 3^{- 1}$, интенсивности повторных попыток установления соединения сессиями \mbox{eMBB} с~орбит~$Q_1$ и~$Q_2$ 
равны $\gamma_1 \hm= \gamma_2 \hm= 100$ .

Как можно видеть на рис.~\ref{fig_Bu},\,\textit{а}, интенсивность поступления в~систему сессий 
\mbox{eMBB} не оказывает влияния на вероятность блокировки новых сессий URLLC. Этот результат объясняется 
абсолютным приоритетом сессий URLLC по отношению к~сессиям \mbox{eMBB}. Что касается сессий \mbox{eMBB} в~системе 
на рис.~\ref{fig_Ne},\,\textit{а}, наибольшее влияние на их число оказывает интенсивность поступления сессий URLLC. 
Следовательно, чем ниже скорость поступления устройств URLLC, тем больше устройств \mbox{eMBB} может быть 
обслужено.

Отметим, что на рис.~5,\,\textit{б} видно, что орбита~$Q_1$ почти все время занята, т.\,е.\ вероятность 
блокировки новой сессии \mbox{eMBB} близка к~100\%. Заметим, графики вероятности прерывания сессий \mbox{eMBB} 
(см.\ рис.~4,\,\textit{б}) сначала возрастают, затем резко убывают. Это можно объяснить снижением среднего числа
 сессий \mbox{eMBB} в~системе (рис.~4,\,\textit{а}), что вполне логично: чем меньше число обслуживаемых сессий, 
 тем меньше вероятность их прерывания.
Также отметим, что рассмотренные характеристики орбит~$Q_1$ и~$Q_2$ зависят от двух рассмотренных 
интенсивностей поступления сессий.

\section{Заключение}

В данной  работе представлена модель совместного обслуживания трафика \mbox{eMBB} и~URLLC в~виде СМО с~двумя 
орбитами. Разработан алгоритм расчета финальных вероятностей в~матричной форме. Проведен численный анализ 
характеристик такой системы.


\vspace*{-6pt}


{\small\frenchspacing
 {%\baselineskip=10.8pt
 %\addcontentsline{toc}{section}{References}
 \begin{thebibliography}{9}
 
 \vspace*{-2pt}

\bibitem{Popovski2018}
\Au{Popovski P., Trillingsgaard K.\,F., Simeone~O., Durisi~G.} 5G wireless network slicing for \mbox{eMBB}, 
URLLC, and mMTC: A~communication-theoretic view~// IEEE Access, 2018. Vol.~6. 
P.~55765--55779.

\bibitem{Cheng2020}
\Au{Cheng S., Wang L., Hwang~C., Chen~J., Cheng~L.}  On-device cognitive spectrum allocation for 
coexisting URLLC and eMBB users in 5G systems~// IEEE T. Cognitive Communications 
 Networking, 2020. 13~p. doi: 10.1109/ \mbox{TCCN}.2020.3007890.
(This article has been accepted for publication in a future issue of this journal, but 
has not been fully edited.) %у этой статьи нету волюма и~номер страниц

\bibitem{Tun2020}
\Au{Tun Y.\,K., Kim~D.\,H., Alsenwi~M., Tran~N.\,H., Han~Z., Hong~C.\,S.}  
Energy efficient communication and computation resource slicing for \mbox{eMBB} and URLLC 
coexistence in 5G and beyond~// IEEE Access, 2020. Vol.~8. P.~136024--136035.

\bibitem{DosSantos2020}
\Au{Dos Santos E.\,J., Jr., Souza~R.\,D., Rebelatto~J.\,L., Alves~H.} Network slicing 
for URLLC and \mbox{eMBB} with max-matching diversity channel allocation~// IEEE Commun. 
Lett., 2020. Vol.~24. Iss.~3. P.~658--661.

\bibitem{Kim2020}
\Au{Kim Y., Park S.} Calculation method of spectrum requirement for IMT-2020 \mbox{eMBB} and 
URLLC with puncturing based on $M/G/1$ priority queuing model~// IEEE Access, 2020. Vol.~8. P.~25027--25040.

\bibitem{Anand2020}
\Au{Anand A., De Veciana G., Shakkottai~S.} Joint scheduling of URLLC and \mbox{eMBB} traffic in 
5G wireless networks~// IEEE ACM~T. Network.,  2020. Vol.~28. Iss.~2. P.~477--490.

\bibitem{Li2020}
\Au{Li J., Zhang X.} Deep reinforcement learning-based joint scheduling of \mbox{eMBB} and URLLC 
in 5G networks~// IEEE Wirel. Commun. Lett.,  2020. Vol.~9. Iss.~9. P.~1543--1546.

\bibitem{Gudkova2012}
\Au{Gudkova I.\,A., Samouylov~K.\,E.} Modelling a radio admission control scheme for video 
telephony service in wireless networks~//  Internet of things, smart spaces, and next generation networking~/  
Eds. S.~Andreev, S.~Balandin,\linebreak\vspace*{-10pt}

\pagebreak

\noindent  Y.~Koucheryavy.~--- Lecture notes in computer sacience ser.~--- 
Springer, 2012. Vol.~7469. P.~208--215.

\bibitem{Makeeva2019}
\Au{Makeeva E., Polyakov~N., Kharin~P., Gudkova~I.} Probability model for performance analysis 
of joint URLLC and \mbox{eMBB} transmission in 5G networks~//
{Internet of things, smart spaces, and next generation networks and systems}~/ Eds. O.~Galinina, 
S.~Andreev, S.~Balandin,  Y.~Koucheryavy.~--- Lecture notes in computer science ser.~--- Springer,  
2019. Vol.~11660. P.~635--648.
{\looseness=1

}
\end{thebibliography}

 }
 }

\end{multicols}

\vspace*{-3pt}

\hfill{\small\textit{Поступила в~редакцию 15.10.20}}

\vspace*{8pt}

%\pagebreak

%\newpage

%\vspace*{-28pt}

\hrule

\vspace*{2pt}

\hrule

\vspace*{-2pt}

\def\tit{RETRIAL QUEUING MODEL FOR~ANALYZING JOINT URLLC AND~eMBB TRANSMISSION IN~5G~NETWORKS}


\def\titkol{Retrial queuing model for analyzing joint URLLC and eMBB transmission in 5G networks}


\def\aut{P.\,A.~Kharin$^1$, E.\,D.~Makeeva$^1$, I.\,A.~Kochetkova$^{1,2}$, 
D.\,V.~Efrosinin$^{1,3}$, and~S.\,Ya.~Shorgin$^2$}

\def\autkol{P.\,A.~Kharin, E.\,D.~Makeeva, I.\,A.~Kochetkova, et al.}
%D.\,V.~Efrosinin$^{1,2}$, and~S.\,Ya.~Shorgin$^3$

\titel{\tit}{\aut}{\autkol}{\titkol}

\vspace*{-11pt}


\noindent
$^1$Peoples' Friendship University of Russia (RUDN University), 6~Miklukho-Maklaya Str., 
Moscow 117198, Russian\linebreak
$\hphantom{^1}$Federation

\noindent
$^2$Institute of Informatics Problems, Federal Research Center ``Computer Science and 
Control'' of the Russian\linebreak
$\hphantom{^1}$Academy of Sciences; 44-2~Vavilov Str., Moscow 119133, Russian Federation

\noindent
$^3$Johannes Kepler Universitaet Linz, 69 Altenbergerstrasse, Linz 4040, Austria



\def\leftfootline{\small{\textbf{\thepage}
\hfill INFORMATIKA I EE PRIMENENIYA~--- INFORMATICS AND
APPLICATIONS\ \ \ 2020\ \ \ volume~14\ \ \ issue\ 4}
}%
 \def\rightfootline{\small{INFORMATIKA I EE PRIMENENIYA~---
INFORMATICS AND APPLICATIONS\ \ \ 2020\ \ \ volume~14\ \ \ issue\ 4
\hfill \textbf{\thepage}}}

     \vspace*{3pt}


\Abste{Fifth-generation (5G) networks are characterized by three use cases~--- massive machine-type 
communication (mMTC), ultra-reliable low-latency communication (URLLC), and enhanced 
mobile broadband (\mbox{eMBB}). Quality of service requirements as well as service parameters within 
these use cases vary significantly, e.\,g.,
 URLLC is characterized by an ultralow latency of up to 1~ms, while \mbox{eMBB} has an ultrahigh data 
 transfer rate. The task is to organize the joint provision of such services. The paper proposes 
 a~scheme for joint URLLC and \mbox{eMBB} traffic transmission. It is based on the fact that URLLC data 
 has a small volume and can occupy less than one physical resource block. The authors analyze 
 the scheme using the developed retrial queuing system with two orbits~--- one for temporary 
 storage of interrupted \mbox{eMBB} users and the other for \mbox{eMBB} users waiting for service to start. 
 The authors propose the matrix geometric algorithm for calculating the probability distribution
  as well as formulas for probabilistic characteristics.}

\KWE{5G; ultra-reliable low latency communication (URLLC); enhanced mobile broadband (\mbox{eMBB}); 
retrial queuing system; interruption}

\DOI{10.14357/19922264200403} 

%\vspace*{-20pt}

\Ack
\noindent
The publication has been prepared with the support of the ``RUDN University Program 5-100.''
 The reported study was funded by RFBR, projects Nos.\,18-00-01555(18-00-01685) and 20-37-70079.



%\vspace*{6pt}

  \begin{multicols}{2}

\renewcommand{\bibname}{\protect\rmfamily References}
%\renewcommand{\bibname}{\large\protect\rm References}

{\small\frenchspacing
 {%\baselineskip=10.8pt
 \addcontentsline{toc}{section}{References}
 \begin{thebibliography}{9}

\bibitem{1-koc}
\Aue{Popovski, P., K.\,F.~Trillingsgaard, O.~Simeone, and G.~Durisi}. 2018. 5G wireless
 network slicing for \mbox{eMBB}, URLLC, and mMTC: A~communication-theoretic view. \textit{IEEE Access} 
 6:55765--55779.
\bibitem{2-koc}
\Aue{Cheng, S., L. Wang, C.~Hwang, J.~Chen, and L.~Cheng.} 2020. On-device cognitive spectrum 
allocation for coexisting URLLC and \mbox{eMBB} users in 5G systems. \textit{IEEE T. Cognitive 
Communications  Networking}.  13~p. doi: 10.1109/ \mbox{TCCN}.2020.3007890.
(This article has been accepted for publication in a future issue of this journal, but 
has not been fully edited.)
\bibitem{3-koc}
\Aue{Tun, Y.\,K., D.\,H. Kim, M.~Alsenwi, N.\,H.~Tran, Z.~Han, and C.\,S.~Hong}. 2020. Energy 
efficient communication and computation resource slicing for \mbox{eMBB} and URLLC coexistence in 5G and beyond. 
\textit{IEEE Access} 8:136024--136035.
\bibitem{4-koc}
\Aue{Dos Santos, E.\,J., R.\,D.~Souza, Jr., J.\,L.~Rebelatto, and H.~Alves.} 2020.  Network 
slicing for URLLC and \mbox{eMBB} with max-matching diversity channel allocation. \textit{IEEE Commun. Lett.} 24(3):658--661.
\bibitem{5-koc}
\Aue{Kim, Y., and S. Park.} 2020. Calculation method of spectrum requirement for IMT-2020 
\mbox{eMBB} and URLLC with puncturing based on $M/G/1$ priority queuing model. \textit{IEEE Access} 8:25027--25040.
\bibitem{6-koc}
\Aue{Anand, A., G. De Veciana, and S.~Shakkottai}. 2020. Joint scheduling of URLLC and \mbox{eMBB} traffic 
in 5G wireless networks. \textit{IEEE ACM~T. Network.} 28(2):477--490.
\bibitem{7-koc}
\Aue{Li, J., and X. Zhang.} 2020. Deep reinforcement learning-based joint scheduling of \mbox{eMBB} 
and URLLC in 5G networks. \textit{IEEE Wirel. Commun. Lett.} 9(9):1543--1546.
\bibitem{8-koc}
\Aue{Gudkova, I.\,A., and K.\,E.~Samouylov.} 2012. Modelling a~radio admission control scheme 
for video telephony service in wireless networks. \textit{Internet of things, smart spaces, and 
next generation networking}. Eds. S.~Andreev, S.~Balandin, and Y.~Koucheryavy. Lecture notes in 
computer science ser. Springer. 7469:208--215.
\bibitem{9-koc}
\Aue{Makeeva, E., N. Polyakov, P.~Kharin, and I.~Gudkova.} 2019. Probability model for performance 
analysis of joint URLLC and \mbox{eMBB} transmission in 5G networks. \textit{Internet of things, smart spaces, 
and next generation networks and systems}. Eds. O.~Galinina, S.~Andreev, S.~Balandin, and Y.~Koucheryavy. 
Lecture notes in computer science ser. Springer. 11660:635--648.
{\looseness=1

}
\end{thebibliography}

 }
 }

\end{multicols}

\vspace*{-3pt}

\hfill{\small\textit{Received October 15, 2020}}

%\pagebreak

%\vspace*{-24pt}

\Contr

\noindent
\textbf{Kharin Petr A.} (b.\ 1993)~--- PhD student, Peoples' Friendship University of Russia 
(RUDN University), 
6~Miklukho-Maklaya Str., Moscow 117198, Russian Federation; \mbox{pxarin@mail.ru}

\vspace*{3pt}

\noindent
\textbf{Makeeva Elena D.} (b.\ 1996)~--- PhD student, Peoples' Friendship University of Russia 
(RUDN University), 
6~Miklukho-Maklaya Str., Moscow 117198, Russian Federation; \mbox{len16730637@yandex.ru}

\vspace*{3pt}

\noindent
\textbf{Kochetkova Irina A.} (b.\ 1985)~--- Candidate of Science (PhD) in physics and mathematics, 
associate professor, Peoples' Friendship University of Russia (RUDN University), 
6~Miklukho-Maklaya Str., Moscow 117198, Russian Federation; senior scientist, Institute of Informatics 
Problems, Federal Research Center ``Computer Science and Control'' of the Russian Academy of Sciences, 
44-2~Vavilov Str., Moscow 119133, Russian Federation; \mbox{gudkova-ia@rudn.ru}

\vspace*{3pt}

\noindent
\textbf{Efrosinin Dmitry V.} (b.\ 1977)~--- Doctor of Science in physics and mathematics, associate 
professor, Johannes Kepler Universitaet Linz, 69~Altenbergerstrasse, Linz 4040, Austria; associate 
professor, Peoples' Friendship University of Russia (RUDN University), 6~Miklukho-Maklaya Str., 
Moscow 117198, Russian Federation; \mbox{dmitry.efrosinin@jku.at}

\vspace*{3pt}

\noindent
\textbf{Shorgin Sergey Ya.} (b.\ 1952)~--- Doctor of Science in physics and mathematics, professor, 
principal scientist, Institute of Informatics Problems, Federal Research Center 
``Computer Science and Control'' of the Russian Academy of Sciences, 44-2~Vavilov Str., 
Moscow 119133, Russian Federation; \mbox{sshorgin@ipiran.ru}
\label{end\stat}

\renewcommand{\bibname}{\protect\rm Литература} 