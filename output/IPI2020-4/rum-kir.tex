   \def\stat{rum-kir}
   
   \def\tit{МЕТОД ВИЗУАЛЬНОГО ПРЕДСТАВЛЕНИЯ КОНФЛИКТОВ 
В~ГИБРИДНЫХ ИНТЕЛЛЕКТУАЛЬНЫХ МНОГОАГЕНТНЫХ 
СИСТЕМАХ}
   
   \def\titkol{Метод визуального представления 
конфликтов в~гибридных интеллектуальных 
многоагентных системах}
   
   \def\aut{С.\,Б.~Румовская$^1$, И.\,А.~Кириков$^2$}
   
   \def\autkol{С.\,Б.~Румовская, И.\,А.~Кириков}
   
   \titel{\tit}{\aut}{\autkol}{\titkol}
   
   \index{Румовская С.\,Б.}
  \index{Кириков И.\,А.}
   \index{Rumovskaya S.\,B.}
  \index{Kirikov I.\,A.}
   
   
   %{\renewcommand{\thefootnote}{\fnsymbol{footnote}} \footnotetext[1]
   %{Работа выполнена при частичной поддержке РФФИ (проект 19-07-00187-A).}}
   
   
   \renewcommand{\thefootnote}{\arabic{footnote}}
   \footnotetext[1]{Калининградский филиал Федерального исследовательского центра 
<<Информатика и~управление>> Российской академии наук, \mbox{sophiyabr@gmail.com 
}}
   \footnotetext[2]{Калининградский филиал Федерального исследовательского центра 
<<Информатика и~управление>> Российской академии наук, 
\mbox{baltbipiran@mail.ru}}
   
   %\vspace*{-12pt}

   \Abst{Малые коллективы экспертов, включающие специалистов различных направлений, 
эффективно решают сложные проблемы благодаря их анализу с~различных точек зрения 
и~получению более качественного интегрированного решения. Конфликт в~малых коллективах 
экспертов может как завести в~тупик процесс принятия решения, так и~породить позитивные 
изменения: развитие группы, диагностику отношений, сплачивание группы. Конфликт 
порождает дискуссии, позволяющие получить более продуманные и~согласованные 
решения. Подобные коллективы эффективно решают проблемы, и~моделирование их работы, 
в~частности возможной конфликтной ситуации и~процесса управления ею, позволяет 
вырабатывать метод решения, релевантный сложной задаче. Визуализация конфликтной 
ситуации делает возникшие противоречия контрастными, видимыми. В~работе коллектив 
агентов-экспертов представляется в~виде неориентированного взвешенного графа 
и~рассматриваются методы визуализации (укладки) графов. Для визуализации  
проб\-лем\-но- и~про\-цес\-сно-ори\-ен\-ти\-ро\-ван\-ных конфликтов в~рамках гибридных 
интеллектуальных многоагентных систем (ГиИМАС) предложен метод, разработанный на базе 
пружинной модели укладки графов.}
    
  \KW{коллектив экспертов; конфликт агентов; визуализация конфликта}

   \DOI{10.14357/19922264200411} 
     
   \vspace*{2pt}
   
   
   \vskip 10pt plus 9pt minus 6pt
   
   \thispagestyle{headings}
   
   \begin{multicols}{2}
   
   \label{st\stat}

\section{Введение}

  Малые коллективы экспертов, включающие специалистов различных 
направлений, эффективно решают проблемы благодаря их всестороннему 
анализу, получая интегрированное компромиссное качественное решение~[1]. 
Как следствие, исследование методов коллективного решения проблем и~их 
моделирование~--- важное направление научных исследований в~области 
системного анализа, имеющее большое практическое значение для медицины, 
транспорта и~логистики, экономики и~т.\,д. %\linebreak 
В~[2] разработана и~описана 
модель самоорганизации в~коллективе агентов (экспертов) \mbox{гибридными}
интеллектуальными многоагентными системами, алгоритм 
функционирования которых динамически перестраивается, вырабатывая 
релевантный проблеме метод решения. 
  
  С другой стороны, при решении проблем малыми коллективами снижается 
скорость принятия решений из-за процессов распределения задач 
и~интеграции частных решений, а также возникновения конфликтов, которые 
могут носить как деструктивный, так и~конструктивный характер. 

Конструктивные конфликты позволяют получить более продуманные 
и~согласованные решения~[3], обеспечивают уникальность и~автономность 
каждого из взаимодействующих субъектов, а также развитие отношений 
между ними, предоставляют информацию о~возможностях 
противодействующих субъектов, высвобождают накапливающееся внут\-рен\-нее 
напряжение, сохраняя связи, актуализируют разные позиции и~мнения по 
поводу возникающих проблем, способствуя поиску оптимальных способов их 
решения, усиливают групповую идентичность и~сплоченность. В~связи с~этим 
для повышения релевантности и~качества принимаемых решений 
гибридного и~синергетического искусственного интеллекта, моделирующего работу 
малого коллектива экспертов, в~[4] предложена модель \mbox{ГиИМАС} 
с~проблемно- и~про\-цес\-сно-ори\-ен\-ти\-ро\-ван\-ны\-ми конфликтами, 
а~в~[5] описан метод идентификации конфликтов между агентами в~рамках 
предложенной в~[4] модели. 

Управление конфликтами в~\mbox{ГиИМАС} 
позволит по аналогии с~реальными малыми коллективами экспертов подавлять 
деструктивные проявления конфликта и~стимулировать конструктивные~[3].
   
  В~[3] рассмотрены методы моделирования и~визуального представления 
конфликта в~коллективе экспертов при решении проблем, по результатам 
которого было решено использовать опыт моделирования конфликтов  
ло\-ги\-ко-струк\-тур\-ным методом~[6, 7] с~помощью графов. 

Цель настоящей 
работы~--- разработка метода визуализации конфликтов на базе предложенной 
модели~[4] и~метода их идентификации~[5], который сделает возникшие 
противоречия между агентами \mbox{ГиИМАС} контрастными, видимыми 
для пользователя системы. При визуализации конфликтов в~\mbox{ГиИМАС} 
будем рассматривать ее как неориентированный взвешенный полный граф без 
петель, множество вершин которого взаимно однозначно соответствует 
множеству агентов \mbox{ГиИМАС}, реб\-ра представляют отношения между 
ними, а~вес ре\-бер~--- напряженность возникающего конфликта~[5]. 
{\looseness=-1

}

\vspace*{-6pt}
  
\section{Методы и~модели визуализации графов}

\vspace*{-2pt}

  К наиболее известным методам рисования (визуализации, укладки) графов 
относятся:
  \begin{itemize}
\item основанные на физических аналогиях (П.~Эйдэс~[8], Т.~Камада 
и~С.~Каваи~[9], T.~Фрухтерман и~Э.~Рейнгольд~[10])~--- обладают 
наибольшим потенциалом;\\[-14pt] 
\item поуровневый подход (К.~Сигуяма с~соавторами~[11]) и~восходящее или нисходящее 
представление~[12] для ориентированных графов;\\[-14pt] 
\item рисование деревьев (Э.~Рейнгольд и~Дж.~Тилфорд~[13]);\\[-14pt]
\item планарные укладки графов без пересечения ребер (В.\,Т.~Татт~[14]);\\[-14pt] 
\item ортогональные изображения графов для снижения числа пересечений ребер  
(Ди~Батиста с~соавторами~\cite{15-rum})~--- ребра изображаются прямыми, 
параллельными осям координат;\\[-14pt]
\item произвольное представление графов~\cite{12-rum};\\[-14pt] 
\item прямолинейное (ребра представляются отрезками), сеточное 
и~полигональное (для отображения ребер используются ломаные), главная 
цель которых~--- снижение числа пересечений ребер~\cite{12-rum}.
\end{itemize}

  Для укладки неориентированных взвешенных полных графов без петель 
хорошо 
 зарекомендовали себя~\cite{12-rum}  алгоритмы, основанные на физических  
аналогиях~\cite{8-rum, 9-rum, 10-rum}, в~которых строится специальная 
модель~--- вершины и~ребра графа соответствуют\linebreak <<реальным>> физическим 
взаимодействующим объектам, вводится функция энергии. Лучшая укладка\linebreak
\vspace*{-12pt} 

\columnbreak

\noindent
графа соответствует минимуму энергии сис\-те\-мы. Выделяют силовой 
(force-directed) алгоритм рисования графов и~пружинный (spring), который 
эквивалентен методу многомерного шкалирования (MDS, multidimensional 
scaling)~\cite{16-rum}. 
  
  Силовая и~пружинная модели при определенном наборе параметров дают 
совпадение минимизируемых функций энергии и,~соответственно, \mbox{похожие} 
укладки (совпадают с~точ\-ностью до поворотов и~масштабирования). Поэтому 
иногда их не различают, называя общим термином <<методы, основанные на 
физических аналогиях>> (force-directed techniques). Тем не менее силовая 
модель более <<гибкая>> за счет большего числа настраиваемых параметров: 
возможность регулировать веса объектов позволяет учитывать 
дополнительные атрибуты вершин и~ребер. Также силовую модель проще 
интегрировать и~модифицировать для учета пересечения ребер, размера 
доступного пространства для рисования и~т.\,д. Однако пружинная модель 
проще и~обладает большей производительностью. 
  
  Для дальнейшей работы выбираем пружинную модель~\cite{17-rum}, так как  
для достижения поставленной цели нет необходимости в~учете множества 
дополнительных параметров, связанных со свойствами вершин и~ребер, 
помимо весов последних. Реб\-ра графа заменяют пружинами, при растяжении и~сжатии которых 
возникают силы упругости, действующие по закону Гука и~стремящиеся 
вернуть пружине ее первоначальную длину. Энергия системы 
пружин прямо пропорциональна расстоянию между вершинами $\vert p_i\hm- 
p_j\vert$:
  $$
  E:\ \sum\limits_{(i,j)\in n} \left\vert p_i -p_j\right\vert^2\,,
  $$
где $p_i=(x_i,y_i)\hm\in R^2$ и~$p_j\hm= (x_j,y_j)\hm\in R^2$~--- образы $i$-й 
и~$j$-й вершин в~$R^2$, их позиции (координаты), $i,j\hm\in [1,n]$ (ребра 
графа отображаются на прямые, соединяющие соответствующие вершины).
  
  Для неориентированных взвешенных графов, число вершин и~ребер 
которых не превышает~200 (малые коллективы, моделируемые 
\mbox{ГиИМАС}, содержат не более 20~экспертов, преимущественно 
менее~10), наиболее часто применяется пружинная модель Т.~Камада 
и~С.~Каваи~\cite{9-rum}, на базе которой описан предлагаемый метод 
визуализации конфликтов (МВК).
     
     \vspace*{-8pt}
     
\section{Метод визуализации конфликтов между агентами} 

\vspace*{-2pt}
  
  В~[4] было введено понятие процесса управления конфликтами: 
  \begin{equation*}
  \mathrm{cnfm}=\left\langle \mathbf{CNF}, \mathrm{cnfcl}, \mathrm{cmkb}, 
  \mathrm{act_{cnfm}}, 
\mathrm{ACT_{agcr}}\right\rangle\,.
  %\label{e1-rum}
  \end{equation*}

\pagebreak

\noindent
Здесь $\mathbf{CNF}$~--- матрица, описывающая конфликт между каждой парой 
агентов кортежем, представленным выражением:
\begin{equation*}
  \mathrm{cnf}_{ij\,\mathrm{cnft}}=\left\langle \mathrm{id}_i, \mathrm{id}_j, 
  \mathrm{cnfin}, \mathrm{cnft}, \mathrm{ACT}_{\mathrm{agcr}\,i}, 
\mathrm{ACT}_{\mathrm{agcr}\,j}\right\rangle\,,
  %\label{e2-rum}
  \end{equation*}
где $\mathrm{id}_i$ и~$\mathrm{id}_j$~--- идентификаторы аген\-тов-субъ\-ек\-тов конфликта, 
$\mathrm{cnfin}\hm\in [0,1]$~--- напряженность конфликта, $\mathrm{cnft}$~--- символьная 
переменная <<тип конфликта>>, определенная на множестве 
$\mathrm{CNFT}$\;=\;\{<<проб\-лем\-но-ори\-ен\-ти\-ро\-ван\-ный>>, 
<<про\-цес\-сно-ори\-ен\-ти\-ро\-ван\-ный>>\}, $\mathrm{ACT}_{\mathrm{agcr}\,i}, \mathrm{ACT}_{\mathrm{agcr}\,j}\hm\subseteq 
\mathrm{ACT}_{\mathrm{agcr}}$~--- множества до\-пус\-ти\-мых действий агентов~$\mathrm{ag}_i$ и~$\mathrm{ag}_j$ по 
разрешению противоречий;\linebreak
  $\mathrm{cnfcl}$~--- 
классификатор конфликтов агентов, идентифицирующий их характер 
и~оценивающий напряженность, т.\,е.\ фор\-ми\-ру\-ющий для каж\-дой пары 
агентов значение элемента мат\-ри\-цы~$\mathbf{CNF}$~\cite{5-rum}; $\mathrm{cmkb}$~--- 
база знаний об эффективности методов управ\-ле\-ния конфликтами 
в~зависимости от характеристик проб\-ле\-мы и~конфликтов между агентами; 
$\mathrm{act_{cnfm}}$~--- функция  
аген\-та-фа\-си\-ли\-та\-то\-ра <<управ\-ле\-ние конфликтом>>; $\mathrm{ACT_{agcr}}$~--- 
множество до\-пус\-ти\-мых действий агентов по разрешению противоречий.
  
  
  
  Матрица $\mathbf{CNF}$ вместе с~пороговым минимальным значением 
напряженности визуализируемого конфликта~$\eta$ (задается пользователем, 
по умолчанию $\eta\hm=0$)~--- входные данные МВК.
  
  Первый шаг МВК~--- вычисление промежуточных матриц $\mathbf{CP}$, 
$\mathbf{CPR}$ и~$\mathbf{D}$. 

Элементы $\mathrm{cp}_{ij}\hm= \Pi {p}_3(\mathrm{cnf}_{ij\,\mathrm{prob}})$ 
матрицы~$\mathbf{CP}$ описывают величину напряженности проб\-лем\-но-ори\-ен\-ти\-ро\-ван\-но\-го 
конфликта между агентами. 

Элементы $\mathrm{cpr}_{ij}\hm= \Pi 
{p}_3(\mathrm{cnf}_{ij\,\mathrm{prob}})$ матрицы~$\mathbf{CPR}$ описывают величину 
напряженности про\-цес\-сно-ори\-ен\-ти\-ро\-ван\-но\-го конфликта между 
агентами.

 Элементы матрицы~$\mathbf{D}$ расстояний графа конфликтов 
рассчитываются в~соответствии с~выражением:
  $$
  d_{ij}\!=\! \begin{cases}
  0\,, & \!\mbox{если } i=j\,;\\
  0{,}00001\,, & \!\mbox{если } \mathrm{cpr}_{ij}=\mathrm{cp}_{ij}=0\,;\hspace*{-0.12793pt}\\
  \left(0{,}5 \left( \mathrm{cp}^2_{ij}+cpr^2_{ij}\right)\right)^{0{,}5} & \!\mbox{в\ 
противном\ случае.}
  \end{cases}
  $$
  
Второй шаг МВК~--- запуск алгоритма Ка\-ма\-да--Ка\-ваи~\cite{9-rum}:
\begin{enumerate}[(1)] 
\item вершины графа, представляющие агентов \mbox{ГиИМАС}, 
помещаются в~случайные координаты~$p_i$;
\item выбирается вершина~$m$, на которую действует максимальная сила;
\item остальные вершины фиксируются, энергия системы минимизируется 
двумерным методом  
Нью\-то\-на--Раф\-со\-на, вычисляется смещение для вершины~$m$;
\item шаги 2 и~3 повторяются до достижения одного из признаков останова: 
либо заданного числа итераций, либо порога силы, действующей на 
вершины, ниже которого алгоритм не запускается.
   \end{enumerate}
   
   Алгоритм Ка\-ма\-да--Ка\-ваи на выходе даст оптимальную укладку графа, 
описывающего \mbox{ГиИМАС} (соответствует состоянию с~минимальной 
суммарной энергией системы). Энергия всей системы рассчитывается как
   $$
   E=\sum\limits_{i=1}^{n-1} \sum\limits_{j=i+1}^{n} 0{,}5 k_{ij}\left( \left\vert 
p_i-p_j\right\vert -l_{ij}\right)^2\,.
   $$
Здесь $n$~--- число вершин; $k_{ij}\hm= (d_{ij})^{-2}$~--- сила пружины между 
вершинами;  $p_i$ и~$p_j$~--- положение на плоскости вершин~$i$ и~$j$ 
соответственно; $l_{ij}\hm= L_0 d_{ij} (\max\nolimits_{i<j}  
d_{ij})^{-1}$~--- идеальная длина пружины, где~$L_0$~--- длина стороны 
квадратной области дисплея.
   
   Третий шаг МВК~--- получение результирующей визуализации (см.\ 
рисунок) конфликтующих агентов. Корректируем укладку графа:
   \begin{itemize}
\item с~учетом матриц $\mathbf{CP}$ и~$\mathbf{CPR}$: если превалирует  
проб\-лем\-но-ори\-ен\-ти\-ро\-ван\-ный конфликт между агентами 
($\mathrm{cp}_{ij}\hm\geq \mathrm{cpr}_{ij}$), то ребро между двумя вершинами (агентами) 
прорисовывается сплошной линией, а~если  
про\-цес\-сно-ори\-ен\-ти\-ро\-ван\-ный ($\mathrm{cp}_{ij}\hm< \mathrm{cpr}_{ij}$)~--- штриховой. 
Толщина и~цвет (от светло-серого до черного на рисунке) линии указывают на величину 
среднего квадратического напряженностей конфликтов между агентами;
\end{itemize}

{ \begin{center}  %fig1
 \vspace*{3pt}
    \mbox{%
    \epsfxsize=78.651mm 
\epsfbox{rum-1.eps}
 }

\end{center}

\noindent
{\small
Визуализация конфликта (на примере коллектива агентов, решающего задачу диагностики 
рака поджелудочной железы): Х~--- хирург; 
ОНЛ~--- онколог по нехирургическому лечению; 
ЛПР-Т~--- лицо, принимающее решение (терапевт); 
сУЗИ~--- специалист по ультразвуковому исследованию; 
вЛаД~--- врач лабораторной диагностики; 
сЛД~--- специалист по лучевой диагностике
}}

%\vspace*{6pt}

\begin{itemize}
\item к вершинам добавляем подписи~--- идентификаторы агентов;\\[-14pt]
\item если $\eta>0$, то для агентов с~напряженностью конфликта 
ниже~$\eta$ ребра на результирующем графе не будут отображены.\\[-14pt]
\end{itemize}

  При наведении указателя мыши на ребро отоб\-ра\-жа\-ют\-ся напряженности 
конфликтов между соответствующей парой агентов. При каждой фиксации 
изменений напряженности конфликтов в~мат\-ри\-це $\mathbf{CNF}$ будет 
запускаться МВК. Для каждой сессии работы системы весь визуальный ряд 
конфликта между агентами сохраняется, обеспечивая возможность более 
детального изучения пользователем. Таким образом, пользователь может 
отслеживать развитие конфликта с~самого начала работы системы и~до 
момента получения решения. Визуализация дает быстро воспринимаемое 
знание о том, между какими агентами возникают конфликты при решении 
проблемы, какого они типа, как меняется их напряженность в~процессе 
работы системы.  
С~по\-мощью данных знаний можно предотвратить и/или быстрее разрешить 
развитие конфликта в~естественных малых коллективах экспертов, решающих 
подобную проблему.

\vspace*{-11pt}

\section{Заключение}

\vspace*{-4pt}

  В работе представлены результаты обзора методов укладки графов, на 
основе анализа которого для разработки метода визуализации конфликтов 
в~коллективе агентов выбран метод рисования графа, основанный на 
физических аналогиях, а~именно: эффективная пружинная модель укладки 
графов Т.~Ка\-ма\-да\,--\,С.~Ка\-ваи, зарекомендовавшая себя для 
неориентированных взвешенных графов малой и~средней размерности. 
Результирующая визуализация конфликта агентов предоставляет пользо\-вателю 
быстрое понимание того, между какими\linebreak членами коллектива и~на каких этапах 
возникает конфликт, какого он типа и~напряженности, а~также на каком этапе 
он минимизируется/нивелируется. Разработанный подход к визуализации 
повышает прозрачность работы \mbox{ГиИМАС} для пользователя, не 
зависит от численности коллектива агентов и~легко реализуем.
  
\vspace*{-11pt}

{\small\frenchspacing
    {\baselineskip=10.4pt
    %\addcontentsline{toc}{section}{References}
    \begin{thebibliography}{99}

\vspace*{-2pt}

\bibitem{1-rum}
\Au{Колесников А.\,В.} Гетерогенные естественные и~искусственные системы~// 
Интегрированные модели и~мягкие вычисления в~искусственном интеллекте.~--- 
М.: Физматлит, 2013. Т.~1. С.~86--103.
\bibitem{2-rum}
   \Au{Колесников А.\,В., Кириков~И.\,А., Листопад~С.\,В.} Гиб\-рид\-ные интеллектуальные 
системы с~самоорганиза- цией: координация, согласованность, спор.~--- М.: ИПИ РАН, 2014. 
189~с.
\bibitem{3-rum}
\Au{Румовская С.\,Б., Кириков~И.\,А.} Методы моделирования и~визуального 
представления конфликта в~малом коллективе экспертов, решающих проблемы (обзор)~// 
Информатика и~её применения, 2019. Т.~13. Вып.~3. С.~122--130. doi: 
10.14357/19922264190317.
\bibitem{4-rum}
\Au{Листопад С.\,В., Кириков~И.\,А.} Моделирование конфликтов агентов в~гибридных 
интеллектуальных многоагентных системах~// Системы и~средства информатики, 2019. 
Т.~29. №\,3. С.~139--148. doi: 10.14357/08696527190312.
\bibitem{5-rum}
\Au{Листопад С.\,В., Кириков~И.\,А.} Метод идентификации конфликтов агентов 
в~гибридных интеллектуальных многоагентных системах~// Сис\-те\-мы и~средства 
информатики, 2020. Т.~30. №\,1. С.~56--65. doi: 10.14357/08696527200105.

\bibitem{7-rum} %6
\Au{Готин С.\,В., Калоша~Л.\,П.} Ло\-ги\-ко-струк\-тур\-ный подход и~его применение для 
анализа и~планирования деятельности.~--- М.: Вариант, 2007. 118~с.

\bibitem{6-rum} %7
\Au{Новиков Д.\,А.} Иерархические модели военных действий~// Управление большими 
системами, 2012. №\,37. С.~25--62.
\bibitem{8-rum}
\Au{Eades P.} A heuristic for graph drawing~// Congressus Numerantium, 1984. Vol.~42. P.~149--160.
\bibitem{9-rum}
\Au{Kamada Т., Kawai~S.} An algorithm for drawing general undirected graphs~// Inform. 
Process. Lett., 1989. Vol.~31. Iss.~1. 
 P.~7--15.  doi: 10.1016/0020-0190(89)90102-6.
\bibitem{10-rum}
\Au{Fruchterman T., Reingold~E.} Graph drawing by force-directed placement~// Software 
Pract. Exper., 1991. Vol.~21. Iss.~11. P.~1129--1164. doi: 10.1002/spe.\linebreak 4380211102.
\bibitem{11-rum}
\Au{Sugiyama К., Tagawa~S., Toda~M.} Methods for visual understanding of hierarchical system 
structures~// IEEE~T. Syst. Man Cyb., 1981. Vol.~11. Iss.~2. P.~109--125. doi: 
10.1109/TSMC.1981.4308636.
\bibitem{12-rum}
\Au{Tamassia R., Battista G.\,D., Ioannis~G., Eades~P.} Graph drawing: Algorithms for the 
visualization of graphs.~--- Englewood Cliffs, NJ, USA: Prentice Hall, 1999. 397~p. 
\bibitem{13-rum}
\Au{Reingold E., Tilford~J.} Tidier drawing of trees~// IEEE~T. Software Eng., 1981. 
Vol.~SE-7. Iss.~2. P.~223--228. doi: 10.1109/TSE.1981.234519.
\bibitem{14-rum}
\Au{Tutte W.\,T.} How to draw a graph~// P.~Lond. Math. Soc., 1963. Vol.~S3-13. Iss.~1. 
P.~743--767. doi: 10.1112/plms/s3-13.1.743.
\bibitem{15-rum}
\Au{Battista G.\,D., Liotta G., Vargiu~F.} Spirality of orthogonal representations and optimal 
drawings of series-parallel graphs and 3-planar graphs~// Algorithms and data 
structures~/ Eds. F.\,K.\,H.\,A.~Dehne, J.-R.~Sack, N.~Santoro, S.~Whikesides.~---  
Lecture notes in computer science ser. ~--- Springer, 1993. Vol.~709.  
P.~151--162. doi: 10.1007/3-540-57155-8\_244.
\bibitem{16-rum}
\Au{Kruskal J.\,В., Seery~J.\,B.} Designing network diagrams~// 1st General Conference 
(International) on Social Graphics Proceedings, 1980. P.~22--50. 
\bibitem{17-rum}
\Au{Пупырев С.\,Н.} Модели, алгоритмы и~программный комплекс визуализации сложных 
сетей:  
Дис.\ \ldots\ канд. физ.-мат. наук.~--- Екатеринбург, 2010. 136~с.
   \end{thebibliography}
   
    }
    }
   
   \end{multicols}
   
\vspace*{-6pt}
%\vspace*{-12pt}
   
\hfill{\small\textit{Поступила в~редакцию 30.09.20}}
   
%   \vspace*{8pt}
   
%\pagebreak
   
\newpage
   
\vspace*{-28pt}
   
%   \hrule
   
%   \vspace*{2pt}
   
%   \hrule
   
   %\vspace*{-2pt}
   
   \def\tit{CONFLICT VISUAL REPRESENTATION METHOD\\ 
    IN~HYBRID INTELLIGENT MULTIAGENT SYSTEMS}
   
   
   \def\titkol{Conflict visual representation method in hybrid intelligent 
multiagent systems}
  
   
   \def\aut{S.\,B.~Rumovskaya and I.\,A.~Kirikov}
   
   \def\autkol{S.\,B.~Rumovskaya and I.\,A.~Kirikov}
   
   \titel{\tit}{\aut}{\autkol}{\titkol}
   
   \vspace*{-11pt}
   
   
   \noindent
   Kaliningrad Branch of the Federal Research Center ``Computer Science and Control'' of the Russian 
Academy of Sciences, 5~Gostinaya Str., Kaliningrad 236000, Russian Federation
   
   
   \def\leftfootline{\small{\textbf{\thepage}
   \hfill INFORMATIKA I EE PRIMENENIYA~--- INFORMATICS AND
   APPLICATIONS\ \ \ 2020\ \ \ volume~14\ \ \ issue\ 4}
   }%
    \def\rightfootline{\small{INFORMATIKA I EE PRIMENENIYA~---
   INFORMATICS AND APPLICATIONS\ \ \ 2020\ \ \ volume~14\ \ \ issue\ 4
   \hfill \textbf{\thepage}}}
   
   \vspace*{9pt} 
   
   \Abste{Small collectives of experts, including specialists from different fields, effectively solve complex 
problems by analyzing them from different points of view and obtaining a better-integrated solution. 
A~conflict in small collectives of experts can both lead to a deadlock in the decision-making process and 
generate positive changes: development of the group, diagnostics of relations, and
consolidation of the group. 
A~conflict breeds debate, the depth of which allows for more thoughtful and coordinated solutions. Such 
collectives of experts solve problems effectively. Thus, modeling of their work and possible conflict situation 
with managing it allows developing a~decision-support method that is relevant to solving a complex problem. 
Visualization of a~conflict situation makes appeared contradictions contrast and observable. In the research, 
the authors represent a~collective of agents-experts in the form of an undirected weighted graph. 
The methods of 
graph visualization are considered. To visualize problem- and process-oriented conflicts within 
hybrid intelligent multiagent systems, the authors propose a~method based on the spring model of graph 
drawing.}
   
   
   \KWE{collective of experts; conflict; visualization of conflict}
   
  
   \DOI{10.14357/19922264200411} 
   
   %\vspace*{-20pt}
   
   %\Ack
   %\noindent
   
   
   \vspace*{6pt}
   
     \begin{multicols}{2}
   
   \renewcommand{\bibname}{\protect\rmfamily References}
   %\renewcommand{\bibname}{\large\protect\rm References}
   
   {\small\frenchspacing
    {%\baselineskip=10.8pt
    \addcontentsline{toc}{section}{References}
    \begin{thebibliography}{99}
   
   \bibitem{1-rum-1}
   \Aue{Kolesnikov, A.\,V.} 2013. Geterogennye estestvennye i~iskusstvennye sistemy [Natural and 
artificial heterogeneous systems]. \textit{Integrirovannye modeli i~myagkie vychisleniya 
v~iskusstvennom intellekte} [Integrated models and soft computing in artificial intelligence]. Moscow: 
Fizmatlit. 1:86--103.
   \bibitem{2-rum-1}
   \Aue{Kolesnikov, A.\,V., I.\,A.~Kirikov, and S.\,V.~Listopad.} 2014. \textit{Gibridnye intellektual'nye 
sistemy s~samoorganizatsiey: koordinatsiya, soglasovannost', spor} [Hybrid artificial systems with 
self-organization: Coordination, conformance, row]. Мoscow: IPI RAN. 189~p.
   \bibitem{3-rum-1}
   \Aue{Rumovskaya, S.\,B., and I.\,A.~Kirikov.} 2019. Metody mo\-de\-li\-ro\-va\-niya i~vizual'nogo 
predstavleniya konflikta v~ma\-lom kollektive ekspertov, reshayushchikh problemy (obzor) [Methods of 
modeling and visual representation of a~conflict in small collective of experts solving problems (review)]. 
\textit{Informatika i~ee Primeneniya~--- Inform. Appl.} 13(3):122--130. doi: 10.14357/19922264190317.
   \bibitem{4-rum-1}
   \Aue{Listopad, S.\,V., and I.\,A.~Kirikov.} 2019. Modelirovanie konfliktov agentov v~gibridnykh 
intellektual'nykh mnogoagentnykh sistemakh [Modeling of agent conflicts in hybrid intelligent multiagent 
systems]. \textit{Sistemy i~Sredstva Informatiki~--- Systems and Means of Informatics} 29(3):139--148. 
doi: 10.14357/08696527190312.
   \bibitem{5-rum-1}
   \Aue{Listopad, S.\,V., and I.\,A.~Kirikov.} 2020. Metod identifikatsii konfliktov agentov v~gibridnykh 
intellektual'nykh mnogoagentnykh sistemakh [Agent conflict identification method in hybrid intelligent 
multiagent systems]. \textit{Sistemy}

\columnbreak

\textit{i~Sredstva Informatiki~--- Systems and Means of Informatics} 30(1):56--
65. doi: 10.14357/08696527200105.
   
   \bibitem{7-rum-1} %6
   \Aue{Gotin, S.\,V., and L.\,P.~Kalosha.} 2007. \textit{Logiko-strukturnyy podkhod i~ego primenenie 
dlya analiza i~planirovaniya de\-ya\-tel'\-nosti} [Logical-structural approach and its application for the analysis 
and planning of activities]. Moscow: Variant. 118~p.

\bibitem{6-rum-1} %7
   \Aue{Novikov, D.\,A.} 2012. Ierarkhicheskie modeli voennykh deystviy [Hierarchical models of combat]. 
\textit{Upravlenie bol'shimi sistemami} [Control of Large Systems] 37:25--62.
   \bibitem{8-rum-1}
   \Aue{Eades, P.} 1984. A~heuristic for graph drawing. \textit{Congressus Numerantium} 42:149--160.
   \bibitem{9-rum-1}
   \Aue{Kamada, Т., and S.~Kawai.} 1989. An algorithm for drawing general undirected graphs. 
\textit{Inform. Process. Lett.} 31(1):7--15. doi: 10.1016/0020-0190(89)90102-6.
   \bibitem{10-rum-1}
   \Aue{Fruchterman, T., and E.~Reingold.} 1991. Graph drawing by force-directed placement. 
\textit{Software Pract. Exper.} 
 21(11):1129--1164. doi: 10.1002/spe.4380211102.
   \bibitem{11-rum-1}
   \Aue{Sugiyama, К., S.~Tagawa, and M.~Toda.} 1981. Methods for visual understanding of hierarchical 
system structures. \textit{IEEE~T. Syst. Man Cyb.} 11(2):109--125. doi: 
10.1109/TSMC.1981.4308636.
   \bibitem{12-rum-1}
   \Aue{Tamassia, R., G.\,D.~Battista, G.~Ioannis, and P.~Eades.} 1999. \textit{Graph drawing: 
Algorithms for the visualization of graphs}. Englewood Cliffs, NJ: Prentice Hall. 397~p.
   \bibitem{13-rum-1}
   \Aue{Reingold, E., and J.~Tilford.} 1981. Tidier drawing of trees. \textit{IEEE~T. Software 
Eng.} SE-7(2):223--228. doi: 10.1109/TSE.1981.234519.
   \bibitem{14-rum-1}
   \Aue{Tutte, W.\,T.} 1963. How to draw a graph. \textit{P.~Lond. Math. Soc.} S3-13(1):743--767. doi: 10.1112/plms/s3-13.1.743.

\pagebreak

   \bibitem{15-rum-1}
   \Aue{Di Battista, G., G.~Liotta, and F.~Vargiu.} 1993. Spirality of orthogonal representations and optimal 
drawings of series-parallel graphs 
 and 3-planar graphs. \textit{Algorithms and data structures}. 
 Eds. F.\,K.\,H.\,A.~Dehne, J.-R.~Sack, N.~Santoro, and S.~Whikesides.
 Lecture notes in 
computer science ser. Springer. 709:151--162.
   doi: 10.1007/3-540-57155-8\_244.
 {\looseness=1
 
 }  
   \bibitem{16-rum-1}
   \Aue{Kruskal, J.\,В., and J.\,B.~Seery.} 1980. Designing network diagrams. \textit{1st General 
Conference (International) on Social Graphics Proceedings}. 22--50. 
\vspace*{-3pt}

   \bibitem{17-rum-1}
   \Aue{Pupyrev, S.\,N.} 2010. Modeli, algoritmy i~programmnyy kompleks vizualizatsii slozhnykh setey 
[Models, algorithms and software for visualization of complex networks].  Ekaterinburg. PhD Diss.
136~p.
   \end{thebibliography}
   
    }
    }
   
   \end{multicols}
   
   \vspace*{-3pt}
   
   \hfill{\small\textit{Received September 30, 2020}}
   
   %\pagebreak
   
   %\vspace*{-24pt}
   
   
   \Contr
   
   \noindent
   \textbf{Rumovskaya Sophiya B.} (b.\ 1985)~--- Candidate of Science (PhD) in technology, scientist, 
Kaliningrad Branch of the Federal Research Center ``Computer Science and Control'' of the Russian 
Academy of Sciences, 5~Gostinaya Str., Kaliningrad 236000, Russian Federation; 
\mbox{sophiyabr@gmail.com}
   
   \vspace*{3pt}
   
   \noindent
   \textbf{Kirikov Igor A.} (b.\ 1955)~--- Candidate of Science (PhD) in technology, director, Kaliningrad 
Branch of the Federal Research Center ``Computer Science and Control'' of the Russian Academy of 
Sciences, 5~Gostinaya Str., Kaliningrad 236000, Russian Federation; \mbox{baltbipiran@mail.ru}
   \label{end\stat}
   
   \renewcommand{\bibname}{\protect\rm Литература} 
         