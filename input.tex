\newcommand {\col}{\mathop{\mathrm{col}}}
\newcommand {\ebd}{\triangleq}
\newcommand {\ff}{{\mathcal F}}
\newcommand {\ppp}{{\mathcal P}}
\newcommand {\diag}{\mathop{\mathrm{diag}\,}}
\newcommand{\me}[2]{\mathsf{E}_{ #1 }\left\{ \mathop{#2} \right\} }
\newcommand{\pp}[1]{\mathsf{P}\left\{ #1 \right\}}




\def\stat{borisov}

\def\tit{РЫНОК С МАРКОВСКОЙ СКАЧКООБРАЗНОЙ ВОЛАТИЛЬНОСТЬЮ III: АЛГОРИТМ МОНИТОРИНГА ЦЕНЫ 
РИСКА ПО~ДИСКРЕТНЫМ НАБЛЮДЕНИЯМ ЦЕН АКТИВОВ$^*$}

\def\titkol{Рынок с~марковской скачкообразной волатильностью III: алгоритм мониторинга цены 
риска} % по дискретным наблюдениям цен активов}

\def\aut{А.\,В.~Борисов$^1$}

\def\autkol{А.\,В.~Борисов}

\titel{\tit}{\aut}{\autkol}{\titkol}

\index{Борисов А.\,В.}
\index{Borisov A.\,V.}


{\renewcommand{\thefootnote}{\fnsymbol{footnote}} \footnotetext[1]
{Работа выполнялась с~использованием инфраструктуры Центра коллективного пользования 
<<Высокопроизводительные вы\-чис\-ле\-ния и~большие данные>> (ЦКП <<Информатика>>) ФИЦ ИУ РАН 
(г.~Москва).}}


\renewcommand{\thefootnote}{\arabic{footnote}}
\footnotetext[1]{Федеральный исследовательский центр <<Информатика и~управление>> Российской академии наук;
\mbox{aborisov@frccsc.ru}}

\vspace*{-12pt}


\Abst{Tретья часть цикла посвящена задаче оценивания в~реальном масштабе рыночной цены 
риска в~финансовой системе, состоящей из банковского депозита, базовых рисковых 
бумаг и~их деривативов.
Базовые активы описываются моделью со стохастической волатильностью, 
представленной скрытым марковским скачкообразным процессом (МСП). Исследуемый 
рынок предполагается безарбитражным, и~рыночная цена риска для него является 
функцией текущего значения МСП. Таким образом, мониторинг цены риска сводится 
к~задаче фильтрации состояния МСП. Статистическая информация поступает в~неслучайные дискретные моменты времени и~содержит прямые наблюдения цен базовых 
активов и~косвенные зашумленные наблюдения деривативов. Статья содержит решение 
задачи оптимальной фильтрации состояния МСП, а также соответствующий численный 
алгоритм. Представлены результаты вычислительного примера, 
демонстрирующего качество мониторинга в~зависимости от состава и~вида доступных 
наблюдений.}

\KW{рыночная цена риска; безарбитражный рынок; марковский 
скачкообразный процесс; задача оптимальной фильтрации; численный алгоритм}

\DOI{10.14357/19922264230402}{OFYELT}
  
\vspace*{-6pt}


\vskip 10pt plus 9pt minus 6pt

\thispagestyle{headings}

\begin{multicols}{2}

\label{st\stat}


\section{Введение}

Заметка продолжает цикл работ~\cite{B_23_1_IA, B_23_2_IA}, посвященных 
исследованию модели безарбитражного финансового рынка со стохастической 
во\-ла\-тиль\-ностью, описываемой внешним МСП. 
Отсутствие арбитража означает неединственность мартингальной меры, и~выбор
преобладающей меры из множества~--- актуальная задача, обеспечивающая возможность 
оптимизации портфельных инвестиций и~по\-стро\-ения хеджирующих стратегий.
Описание этой меры возможно в~терминах рыночной цены рис\-ка (MPR, market price of risk). Цель 
предлагаемой работы заключается в~решении задачи оценивания MPR в~реальном 
масштабе времени по наблюдениям, по\-сту\-па\-ющим в~дискретные неслучайные моменты 
времени, с~привлечением математического аппарата оптимальной фильтрации.

Статья организована следующим образом. В~разд.~2 представлена исследуемая 
модель рынка: эволюция цен базовых активов описывается решением некоторой 
стохастической дифференциальной системы (СДС), в~то время как цены деривативов 
определяются  решениями системы уравнений в~частных производных.  В разделе 
предложена модель доступных наблюдений и~представлена аргументация в~ее пользу. 
Задачу мониторинга MPR удается свести к~задаче оптимальной фильтрации состояний 
МСП, определяющего стохастическую волатильность. В~разд.~3 пред\-став\-ле\-ны 
утверждение о~виде оценки оптимальной фильт\-ра\-ции и~алгоритм ее чис\-лен\-ной 
реализации. Раздел~4 содержит чис\-лен\-ный пример, ил\-люст\-ри\-ру\-ющий за\-ви\-си\-мость 
качества оценивания со\-сто\-яний МСП от со\-ста\-ва и~вида об\-ра\-ба\-ты\-ва\-емых наблюдений. 
Заключительные замечания по результатам, пред\-став\-лен\-ным в~данной час\-ти цик\-ла, 
даны в~разд.~5.

\section{Модель рынка и~постановка задачи мониторинга рыночной цены риска}

На вероятностном базисе с~фильтрацией $(\Omega, \ff, \ppp, \{\ff_t\}_{t \in 
[0,T]})$ рассматривается модель финансового рынка, состоящего из
\begin{itemize}
\item
банковского вклада с~детерминированной ставкой $r_t$;
\item
$N$ базовых финансовых инструментов $S_t \hm= \col\,(S_t^1,\ldots,S_t^N)$;
\item
$M$ деривативов $F_t = \col\,(F_t^1,\ldots,F_t^M)$ c~единым сроком погашения $T$, 
опре\-де\-ля\-емых платежным требованием $H(S_T)\hm = \col\,(H^1(S_T),\ldots,H^M(S_T))$.
\end{itemize}

Цена $S_t $  представляет собой решение СДС
  \begin{multline}
  dS_t = \diag S_t a(t,Z_t)\,dt + \diag S_t \sigma(t,Z_t)\,dw_t,\\ 
  t \in (0,T], \enskip S_0 \sim \pi_0(s),
  \label{eq:S_sys_P}
  \end{multline}
  где $w_t \ebd \col\,(w_t^1,\ldots,w_t^K)$~--- $K$-мерный стандартный винеровский 
процесс ($K \geqslant N$), а~случайные функции мгновенной процентной ставки~$a$ и~внутренней волатильности~$\sigma$  
имеют вид:
  $$
  a(t,Z_t) = \sum\limits_{\ell=1}^L Z_t^{\ell} a^{\ell}(t);\enskip \sigma(t,Z_t) = 
\sum\limits_{\ell=1}^L Z_t^{\ell}  \sigma^{\ell}(t)
$$
с набором известных детерминированных функций 
$\{a^{\ell}(t)\}_{\ell=\overline{1,L}}$  и~$\{\sigma^{\ell}(t)\}_{\ell=\overline{1,L}}$. Распределение начальной цены~$S_0$ 
таково, что все ее компоненты положительны с~ве\-ро\-ят\-ностью~1.

Стохастическая волатильность $\sigma(t,Z_t)$ определяется внешним скрытым МСП 
$Z_t \ebd$\linebreak $\ebd\,\col(Z_t^1, \ldots, Z_t^L) \in \{e_1,\ldots, e_L\}$  с~мат\-ри\-цей 
\mbox{интенсивностей} переходов~$\Lambda(\cdot)$ и~начальным распределением~$\pi_0^Z$. 
Марковский скачкообразный процесс~$Z_t$ представляет собой решение СДС с~мартингалом $M_t$ в~правой части:
\begin{equation}
  dZ_t = \Lambda^{\top}(t)Z_t\,dt + dM_t,\quad t \in (0,T], \quad Z_0 \sim 
\pi_0^Z.
  \label{eq:Z_sys}
  \end{equation}

В~\cite{B_23_1_IA} показано, что в~условиях безарбит\-раж\-ности рынка  справедливая 
цена $F^m$ $m$-го дериватива ($m=\overline{1,M}$) определяется формулой:
$$
F^m (t, S_t, Z_t) = \sum\limits_{\ell=1}^L Z_t^{\ell} F^{m\ell} (t, S_t),
$$
где детерминированные функции $\{F^{m\ell} (t, s)\}_{\ell = \overline{1,L}}$ 
являются решением задачи Коши  системы дифференциальных уравнений в~частных 
производных
\begin{multline}
F_t^{m\ell} = rF^{m\ell} - \displaystyle\sum\limits_{j=1}^L \Lambda^{\ell j}F^{mj} - \sum\limits_{n=1}^N 
F_{s^n}^{m\ell}s^n(r-a_n^{\ell} ) -{}\\
{}- 
\fr{1}{2}  \sum\limits_{i,j=1}^N s^is^j F_{s^i,s^j}^{m\ell}B_{ij}^{\ell},\enskip
\ell=\overline{1,M}, \\
 m=\overline{1,M}, \enskip
t \in [0,T), \enskip
F^{m\ell}(T,s) = H^m(s);
\label{eq:F_m_eq}
\end{multline}

\noindent
$$
B^{\ell} = \left\|B_{ij}^{\ell}\right\|_{i,j=\overline{1,N}} \ebd 
\sigma^{\ell}\left( \sigma^{\ell} \right)^{\top}.
$$

Согласно~\cite{B_23_1_IA}, MPR $\theta_t$ в~данной модели определяется формулой
$$
\theta_t = \sum\limits_{\ell=1}^L Z_t^{\ell} \left(\sigma^{\ell}(t)\right)^+ \left( 
a^{\ell}(t) - r_t \mathbf{1} \right)
$$
(здесь $\mathbf{1}$~--- вектор-стол\-бец подходящей раз\-мер\-ности, составленный из 
единиц) и~однозначно определяется значением МСП~$Z_t$.

Цены $S_t$ и~$F_t$, а также фак\-тор-про\-цесс~$Z_t$ недоступны прямому наблюдению, 
однако в~дискретные моменты $t_i \ebd i h,\; i \in \mathbb{N}$ ($T$ делится на~$h$ 
нацело и~$\mathcal{I} \ebd T/h$), имеется следующая статистическая 
информация:
безошибочные наблюдения текущей стоимости базовых активов
$
\mathsf{S}_i \hm= S_{t_i}$ и~косвенные наблюдения текущей стоимости деривативов $\mathsf{G}_i \hm\in 
\mathbb{R}^{M'}$ (в общем случае $M \hm\neq M'$).

Введем следующие обозначения:
\begin{itemize}
\item
$\mathcal{O}_i \ebd \sigma \{\mathsf{S}_j , \mathsf{G}_j: \; 0 \hm\leqslant j 
\leqslant i\}$~--- семейство \mbox{$\sigma$-ал}\-гебр, порожденных всеми наблюдениями, 
полученными до момента  $t_i$ включительно;\\[-14pt]
\item
$\mathcal{G}_i \ebd \sigma \{\mathsf{G}_j, \; 0 \leqslant j \hm\leqslant i\}$~--- 
семейство $\sigma$-алгебр, порожденных наблюдениями $\{\mathsf{G}_j\}$;\\[-14pt]
\item
$\mathcal{H}_i \ebd \sigma \{\mathsf{S}_j , \mathsf{Z}_{t_j}: \; 0\hm \leqslant j 
\leqslant i\}$~--- семейство \mbox{$\sigma$-ал}\-гебр, порожденных ценами базового актива и~МСП на дискретной временн$\acute{\mbox{о}}$й сетке до момента $t_i$ включительно.
\end{itemize}

Относительно наблюдаемой по\-сле\-до\-ва\-тель\-ности~$\{\mathsf{G}_i\}$ предполагается, 
что она обладает марковским свойством при известной паре~$(S,Z)$, т.\,е.\ для   
любого $B \hm\in \mathcal{B}(\mathbb{R}^{M'})$
и $i \hm= \overline{1,\mathcal{I}}$
$\ppp$-п.~н.\ выполняется цепочка равенств:

\vspace*{-6pt}

\noindent
\begin{multline}
\me{}{\mathbf{I}_B(\mathsf{G}_i)|\mathcal{H}_i \vee \mathcal{G}_{i-1}} =
\me{}{\mathbf{I}_B(\mathsf{G}_i)|\mathsf{S}_i, Z_{t_i}, \mathsf{G}_{i-1}} ={}\\
{}=
\sum\limits_{m=1}^M Z_{t_i}^m \int\limits_{B} \Phi_i^m(dq | \mathsf{S}_i,  \mathsf{G}_{i-1}),
\label{eq:vars}
\end{multline}

\vspace*{-6pt}

\noindent
где $\{\Phi_i^m(\cdot | \mathsf{s},  \mathsf{g})\}_{i=\overline{1,\mathcal{I}},\
m=\overline{1,M}}$~--- набор известных мер, определяющих соответствующее 
условное распределение. Также предполагается, что для каждого 
$i=\overline{1,\mathcal{I}}$ выбрана такая мера~$\mu_i$, что $\Phi_i^m \ll 
\mu_i$ для всех $m=\overline{1,M}$ и~$\phi_i^m(\cdot | \mathsf{s},  \mathsf{g}) 
\ebd {d \Phi_i^m}(\cdot | \mathsf{s},  \mathsf{g})/{d \mu_i}$~--- 
со\-от\-вет\-ст\-ву\-ющие производные Ра\-до\-на--Ни\-ко\-ди\-ма. Для начального наблюдения 
$\mathsf{G}_0$ также заданы такие мера~$\mu_0$ и~плот\-ности $\phi_0^m(\cdot | s)$, что для любого $B \in \mathcal{B}(\mathbb{R}^{M'})$ 
$\ppp$-п.~н. 
выполняется равенство
$$
\me{}{\mathbf{I}_B(\mathsf{G}_0)|\mathsf{S}_0, Z_{0}}\hm =
\sum\limits_{m=1}^M Z_{0}^m \int\limits_{B} \phi_0^m(q | \mathsf{S}_0)\mu_0(dq).
$$

\textit{Задача мониторинга MPR} сводится к~вы\-чис\-лению 
$$
\widehat{Z}_i \ebd  \me{}{Z_{t_i}|\mathcal{O}_i},\ i\hm=\overline{1,\mathcal{I}}.
$$

Предложенная модель наблюдений нуждается в~объяснении. 

Во-первых, на практике 
цены базовых активов и~деривативов недоступны наблюдению в~непрерывном времени, 
иначе при выполнении дополнительного условия иден\-ти\-фи\-ци\-ру\-емости~\cite{BS_20} 
значение скрытого МСП мож\-но было бы восстановить безошибочно по $\{S_t\}$, 
сделав тем самым рынок,\linebreak\vspace*{-12pt}

\pagebreak

\noindent
 со\-сто\-ящий только из депозита и~базовых активов, пол\-ным~\cite{Sh_98, KS_16}.

Во-вторых, в~предложенной модели рынка справедливые цены деривативов не могут 
быть известны безошибочно даже в~дискретные моменты времени, иначе структура 
цены $F^m(t,S_t,Z_t)$ позволила бы восстановить значение~$Z$ в~момент~$t_i$.

В-третьих, модель рынка известна его участникам, а значит, они обладают знанием 
набора возможных справедливых цен деривативов $\{F^{m\ell}(t,s)\}$~--- решений~(\ref{eq:F_m_eq}). 
Единственное, чего они не знают точно,~---  правильный выбор 
одного из этих вариантов, определяемый МСП~$Z_t$. Трейдеры в~соответствии со 
своими соображениями выставляют заявки на покупку ({bid}) и~продажу ({ask}) 
деривативов по цене, не совпадающей со справедливой, внося в~заявки 
непредумышленные или намеренные ошибки. Именно последовательности заявок bid, 
ask и~реализованных сделок ({trade}) и~служат доступной статистической 
информацией.

Cтохастические модели тройки <<bid--ask--trade>> достаточно разнообразны 
(см., например,~\cite{BSW_04, CKS_12, CM_21, LVSS_13, KR_17}). Здесь 
предполагается доступность только цен сделок. Предложенная модель наблюдений цен 
деривативов~(\ref{eq:vars}) является достаточно общей и~описывает большое число 
схем, имеющих практическое значение. Ниже на примере случая одного 
базового инструмента и~одного дериватива  рас\-смат\-ри\-ва\-ют\-ся сле\-ду\-ющие схемы 
(возможности модели этими схемами не исчерпываются):
\begin{enumerate}[(1)]
\item
наблюдения $\mathsf{G}_i$ представляют собой истинную справедливую цену, 
искаженную независимым мультипликативным белым шумом;\\[-13pt]
\item
наблюдения $\mathsf{G}_i$ цены дериватива~--- марковская цепь с~множеством 
значений $\{F^{\ell}(t_i, \mathsf{S}_i)\}_{\ell}$ и~мат\-ри\-цей переходных 
вероятностей $\Gamma(i, \mathsf{S}_i,Z_{t_i})$, зависящей от текущего времени, 
цены базового актива и~текущего со\-сто\-яния МСП.
\end{enumerate}

\vspace*{-7pt}


\section{Аналитическое решение задачи фильтрации и~численный алгоритм}

\vspace*{-3pt}


Рассматриваемая модель рынка гарантирует корректное определение вспомогательного 
процесса
$V_t \hm= \col(V_t^1, \ldots,V_t^N)$, $V_t^n \ebd \ln {S_t^n}/{S_0^n}$, 
$n\hm=\overline{1,N}$, который, согласно правилу Ито, представляет собой 
единственное сильное решение СДС

\vspace*{-6pt}

\noindent
  \begin{multline*}
  dV_t =
  \sum\limits_{\ell=1}^L
  Z_t^{\ell}
  \left( a^{\ell}(t) -\fr{1}{2}\,b^{\ell}(t)
  \right)dt+ \sum\limits_{\ell=1}^L
  Z_t^{\ell} \sigma^{\ell}(t)
  \,dw_t,\\
   t \in (0,T], \enskip V_0 \equiv \mathbf{1},
%  \label{eq:V_sys_P}
  \end{multline*}
  
  \noindent
  где 
  $$
  b^{\ell}(t) \ebd \col \left( B_{11}^{\ell}(t), \ldots, 
B_{NN}^{\ell}(t)\right).
$$
 Если $\mathsf{V}_0 \ebd S_0$ и~$\mathsf{V}_i \ebd 
V_{t_i}-V_{t_{i-1}}$ для любых $i \hm=\overline{1,\mathcal{I}}$, то легко 
проверить, что наблюдения $\{\mathsf{V}_i \}$ взаимно однозначны наблюдениям 
$\{\mathsf{S}_i \}$ и~порожденные ими наборы $\sigma$-ал\-гебр совпадают. 
Наблюдения $\mathsf{V}_i$ имеют плот\-ность распределения при $i \geqslant 1$, 
и~совместное распределение пары $(Z_{t_i}, \mathsf{V}_i)$ относительно $Z_{t_{i-1}}$ определяется плотностями 
$\{\xi_i^{jk}(v)\}_{i=\overline{1,\mathcal{I}},\ j,k =\overline{1,L}}$:
  $$
  \pp{Z_{t_i} = e_k, \mathsf{V}_i \in A | Z_{t_{i-1}} = e_j} =\int\limits_A 
\xi_i^{jk}(v)\,dv.
  $$
  
  \smallskip
  
  \noindent
\textbf{Теорема 1.}\ 
\textit{ Оценка оптимальной фильтрации $\widehat{Z}_i$ определяется следующей 
рекуррентной процедурой}:
  \begin{align}
  \widehat{Z}_i^{\ell} &= \fr{\widetilde{Z}_i^{\ell}}{\sum\nolimits_{j=1}^L 
\widetilde{Z}_i^{j}}, \enskip
\ell = \overline{1,L}, \enskip i > 1;
  \label{eq:filt_1}
 \\
     \widetilde{Z}_i^{\ell} &= \sum\limits_{k=1}^L
  \phi_i^{\ell}(\mathsf{G}_i | \mathsf{S}_i ,\mathsf{G}_{i-1} )
  \xi_i^{k\ell}\left(\mathsf{V}_{i} \right) \widehat{Z}_{i-1}^{k}, \quad
\ell = \overline{1,L}, \notag\\
&\hspace*{10mm} \mathsf{V}_{i} = \col \left(\ln 
\fr{\mathsf{S}_i^1}{\mathsf{S}_{i-1}^1},
\ldots, \ln \fr{\mathsf{S}_i^N}{\mathsf{S}_{i-1}^N}
 \right),
  \label{eq:filt_2}
  \end{align}
  \textit{с начальным условием}
    \begin{equation}
  \widehat{Z}_0^{\ell} = \fr{\pi_0^{Z\ell} \phi_0^{\ell}(\mathsf{G}_0 | 
\mathsf{S}_0)}{\sum\nolimits_{j=1}^L \pi_0^{Z j} \phi_0^{j}(\mathsf{G}_0 | \mathsf{S}_0) 
}, \enskip
\ell = \overline{1,L}.
  \label{eq:filt_3}
  \end{equation}
Доказательство теоремы~1 приведено в~приложении.

Основная слож\-ность численной реализации рекурсии~(\ref{eq:filt_1}), 
(\ref{eq:filt_2}) связана с~проблемой вычисления плотностей $\xi_i^{jk}(v)$. 
Дело в~том, что они представляют собой сдвиг-мас\-штаб\-ные смеси гауссиан, причем 
смешивающее распределение определяется поведением МСП~$Z$ на отрезках $[t_{i-1}, t_i]$ 
при фиксированных начальном и~конечном значениях. Для частного случая 
матрицы интенсивностей переходов~$\Lambda$, коэффициентов $\{a^{\ell}\}$ и~$\{b^{\ell}\}$, не зависящих от времени~$t$, в~\cite{B_20_3_IA} предложена 
составная схема <<средних>> прямоугольников, которую предлагается применить 
и~в~данной задаче. Введем сле\-ду\-ющие обозначения:
\begin{itemize}
\item
$\mathcal{N}(v,m,K)$~--- плотность распределения гауссовского случайного вектора 
со средним $m$ и~невырожденной ковариационной матрицей~$K$;
\item
$U_i \ebd ({t_{i-1}+t_{i}})/{2}$~--- середина отрезка $[t_{i-1}, t_{i}]$;
\item[--]
$u_{i,m} \ebd t_{i-1} + h^{1+\alpha}(m-{1}/{2})$~--- середины отрезков более 
мелкого разбиения с~шагом $h^{1+\alpha}$, $m\hm=\overline{1,[h^{-\alpha}]}$, $0 \hm< 
\alpha \hm\leqslant 1$;
\item
$Q^{k\ell} (v,u)  \ebd \exp\left[ \left(\Lambda_{kk}(u) \hm- \Lambda_{\ell 
\ell}(u)\right)\right] \times$\linebreak
$\times \mathcal{N}\left(
v, u \left(a^k(u)\hm + ({1}/{2})b^k(u)\right)\hm + (h\hm-u) \left(a^{\ell}(u) +\right.\right.\hspace*{-0.93018pt}$\linebreak 
$\left.\left.\!+({1}/{2})b^{\ell}(u)\right)\!,
u B^k(u) \hm+ (h-u)B^{\ell}(u)\right)
$
--- вспомогательная функция.
\end{itemize}

\vspace*{-1pt}

\noindent
Для вычисления $\xi_i^{jk}(\cdot)$ предлагается воспользоваться аппроксимацией

\vspace*{-5pt}

\noindent
\begin{multline*}
\xi_i^{jk}(\mathsf{V}_i) \approx
\delta_{k\ell}e^{\Lambda_{\ell \ell}(U_i) h}\times{}\\[-1.5pt]
{}\times 
\mathcal{N}\left(
\mathsf{V}_i, h \left(a^{\ell}(U_i) + 
\fr{1}{2}\,b^{\ell}(U_i)\right),
hB^{\ell}(U_i)
\right) +{} \\[-1.5pt]
{}+ (1-\delta_{k\ell}) \Lambda_{k \ell}(U_i) h^{1+\alpha}
\sum\limits_{m=1}^{[h^{-\alpha}]} Q^{k\ell} (\mathsf{V}_i,u_{i,m}).
\end{multline*}

\vspace*{-14pt}

\section{Численный пример}

\vspace*{-2pt}

Предлагаемый пример иллюстрирует влияние наблюдений цен дериватива
в дополнение к~име\-ющим\-ся наблюдениям цены базового актива
на качество оценивания МСП в~зависимости от вида этих наблюдений.

На отрезке времени $[0,1]$ ($1$ год, $250$ торговых\linebreak дней по $8$ часов каждый) 
моделируется поведение одного базового актива, стохастическая волатильность 
которого представляет собой МСП\linebreak с~четырьмя возможными состояниями <<рост\,--\,со\-сто\-яние, предшествующее панике,\,--\,па\-ни\-ка\,--\,ре\-цес\-сия>>.
Ставка по депозиту $r \hm= 2 \%$ годовых, на рынке присутствует один дериватив~--- 
{call}-оп\-ци\-он с~ценой исполнения $1$. Для моделирования (\ref{eq:S_sys_P}), (\ref{eq:Z_sys}) были выбраны следующие параметры:
$
S_0 \hm\equiv 1$; $a \hm= (0{,}04; 0{,}035;0{,}015;0{,}02)$; $\sigma \hm= (0{,}1;0{,}12;0{,}25;0{,}15)$;
$$
\Lambda =
\left[
\begin{array}{cccc}
-12{,}5 & 12{,}5 & 0 & 0 \\
100 & -500 & 400 & 0 \\
0 & 0 & -125 & 125 \\
40 & 0 & 10 & -50 \\
\end{array}
\right]; \enskip
\pi_0^Z =
\left[
\begin{array}{c}
0{,}7273  \\
0{,}0182  \\
0{,}0727  \\
0{,}1818  \\
\end{array}
\right].
$$
Интегрирование СДС выполняется методом Эй\-ле\-ра--Ма\-ру\-ямы, адаптированным к~скачкам~\cite{PB_10}, с~основным шагом $\tau\hm = 10^{-6}$,
интегрирование сис\-те\-мы~(\ref{eq:F_m_eq})~--- методом, предложенным в~\cite{B_23_2_IA},
 c~шагом по времени $H \hm= 0{,}001$ (шаг соответствует 2~ч торгового времени) 
 и~шагом по координате $\delta \hm= 0{,}002$, что соответствует $0{,}2$~п.\,п.\ начальной цены базового актива.

Все наблюдения проводятся с~шагом $h \hm= 0{,}0001$, что соответствует 12~мин 
торгового времени. Сравнивается качество оценок МСП~$\{Z_{t_i}\}$, построенных 
по  двум комплексам доступных наблюдений:\\[-13pt]
\begin{enumerate}[(1)]
\item
имеются прямые наблюдения цен базового актива $\{\mathsf{S}_i\}$, а также цена 
дериватива с~мультипликативным шумом:
$$
\mathsf{G}_i^{LN} = F_{t_i} \varepsilon_i,
$$
где $\{\varepsilon_i\}$~--- последовательность независимых одинаково 
распределенных случайных вели-\linebreak\vspace*{-12pt}

\columnbreak

\noindent
чин, имеющих логнормальное распределение 
с~параметрами $a^{\mathrm{LN}}(h)\hm = 0{,}04h$ и~$\sigma^{\mathrm{LN}}(h) \hm= 0{,}28h$;\\[-14pt]
\item
имеются прямые наблюдения $\{\mathsf{S}_i\}$ и~косвенные наблюдения цен 
дериватива $\{\mathsf{G}_i^{MC}\}$. Последние представляют собой цепь со 
значениями $\{F^{\ell}(t_i, \mathsf{S}_i)\}$~--- вариантами возможных цен 
опциона, обладающую марковским свойством при известной траектории $\{Z_t\}$. 
Условные мат\-ри\-цы переходов этой цепи с~компонентами

\vspace*{-7pt}

\noindent
\begin{multline*}
{\sf P}\left\{\mathsf{G}_i^{\mathrm{MC}} = F^{k}(t_i, \mathsf{S}_i) | \mathsf{G}_{i-1}^{\mathrm{MC}} = 
F^{j}(t_{i-1},\mathsf{S}_{i-1}),\right.\\ 
\left.Z_{t_i}=e_m\right\} =\Gamma_{jkm},
\end{multline*}

\vspace*{-6pt}

\noindent
где $\Gamma_m \hm= 
\|\Gamma_{jkm}\|_{j,k}$ ($m\hm=\overline{1,L}$) являются мат\-рич\-ны\-ми экспонентами 
$$
\Gamma_m(h) =\exp(h\Upsilon_m),
$$ 

\vspace*{-2pt}

\noindent
построенными по <<матрицам интенсивностей 
переходов>>

\vspace*{-7pt}

\noindent
\begin{align*}
\Upsilon_1&=\left[
\begin{array}{cccc}
-30 & 10 & 10 & 10 \\
100 & -300 & 100 & 100 \\
100 & 100 & -300 & 100 \\
100 & 100 & 100 & -300 \\
\end{array}
\right];
\\
\Upsilon_2&=\left[
\begin{array}{cccc}
-3000 & 1000 & 1000 & 1000 \\
1 & -3 & 1 & 1 \\
1000 & 1000 & -3000 & 1000 \\
1000 & 1000 & 1000 & -3000 \\
\end{array}
\right];
\\
\Upsilon_3&=\left[
\begin{array}{cccc}
-300 & 100 & 100 & 100 \\
100 & -300 & 100 & 100 \\
10 & 10 & -30 & 10 \\
100 & 100 & 100 & -300 \\
\end{array}
\right];
\\
\Upsilon_4&=\left[
\begin{array}{cccc}
-300 & 100 & 100 & 100 \\
100 & -300 & 100 & 100 \\
100 & 100 & -300 & 100 \\
10 & 10 & 10 & -30 \\
\end{array}
\right].
\end{align*}
\end{enumerate}

\vspace*{-8pt}

\noindent
Следует отметить, что параметры распределений наблюдений цен деривативов зависят 
от времен$\acute{\mbox{о}}$го шага~$h$ между поступающими наблюдениями.

\begin{figure*} %fig1
  \vspace*{1pt}
\begin{center}
   \mbox{%
\epsfxsize=135.581mm 
\epsfbox{bor-1.eps}
}
\end{center}
\vspace*{-9pt}
\Caption{Ошибки наблюдений цен деривативов: \textit{1}~--- $\Delta^{\mathrm{LN}}_t$; \textit{2}~--- $\Delta^{\mathrm{MC}}_t$
} 
\label{pic:pic1}
%\end{figure*}
%\begin{figure*} %fig2
   \vspace*{6pt}
\begin{center}
   \mbox{%
\epsfxsize=135.178mm 
\epsfbox{bor-2.eps}
}
\end{center}
\vspace*{-9pt}
\Caption{Оценки компонентов МСП по разным комплексам наблюдений: \textit{1}~--- $X_t^l$; \textit{2}~--- $\hat{X}_t^S$;
\textit{3}~--- $\hat{X}_t^{\mathrm{LN}}$;
\textit{4}~--- $\hat{X}_t^{\mathrm{MC}}$
} 
\label{pic:pic2}
\end{figure*}

На рис.~\ref{pic:pic1} представлены ошибки наблюдений $\Delta_i^{\mathrm{LN}}\hm = 
\mathsf{G}_i^{\mathrm{LN}} \hm- F_{t_i}$ и~$\Delta_i^{\mathrm{MC}}\hm = \mathsf{G}_i^{\mathrm{MC}}\hm - F_{t_i}$.
Результаты оценивания МСП по различной статистической информации на интервале 
$[0;0{,}2]$ приведены на рис.~\ref{pic:pic2}. На нем для всех компонент МСП 
изобра\-жены:\\[-14pt]
\begin{itemize}
\item
истинное значение компоненты $X_t^{\ell}$;\\[-14pt]
\item
оценка фильтрации $\widehat{X}_t^{S}$, вычисленная только по наблюдениям 
$\mathsf{S}_i$;\\[-14pt]
\item
оценка фильтрации $\widehat{X}_t^{\mathrm{LN}}$, вычисленная по наблюдениям 
$\mathsf{S}_i$ и~$\mathsf{G}_i^{\mathrm{LN}}$;\\[-14pt]
\item
оценка фильтрации $\widehat{X}_t^{\mathrm{MC}}$, вычисленная по наблюдениям 
$\mathsf{S}_i$ и~$\mathsf{G}_i^{\mathrm{MC}}$.
\end{itemize}

\pagebreak
 
  

По результатам расчетов можно сделать сле\-ду\-ющие заключения.
\begin{enumerate}
\item
Параметры в~примере выбраны таким образом, что рынок пребывает во 2-м состоянии, 
предшествующем панике, достаточно короткое время: в~среднем 2~ч. За это 
время можно получить только 20 наблюдений цен базового актива. В~то же время 
значения мгновенной процентной ставки и~волатильности во 2-м состоянии близки к~соответствующим значениям состояния $1$ (рост): $0,\!035$ против $0,\!04$ 
и~$0{,}12$ против $0{,}1$ соответственно. Поэтому вполне ожидаемо, что качество мониторинга состояния~2, выполненного только по наблюдениям цен базового 
актива, низкое (см.\ рис.~\ref{pic:pic2},\,\textit{б}). Очевидно, что любая 
дополнительная статистическая информация актуальна для улучшения качества 
оценивания 2-го состояния.
\item
Траектории возможных значений цен опциона $\{F^{\ell}\}$ визуально малоотличимы 
друг от друга, а~также от наблюдений $\mathsf{G}_i^{\mathrm{LN}}$ и~$\mathsf{G}_i^{\mathrm{MC}}$, 
поэтому на  рис.~\ref{pic:pic1} представлены не сами наблюдения, а их ошибки.  
Видна принципиальная разница характера траекторий этих ошибок: $\Delta_i^{\mathrm{LN}}$ 
характерна для белого шума, в~то время как  в~$\Delta_i^{\mathrm{MC}}$ очевидна 
временн$\acute{\mbox{а}}$я зависимость.
\item
Рисунок~\ref{pic:pic2} позволяет сделать вывод о том, что включение в~процедуру 
фильтрации наблюдений обоих деривативов позволяет повысить качество оценивания.
Выборочные средние квадраты значений ошибок наблюдений (см.\ рис.~\ref{pic:pic1}), вычисленные по данным траекториям, для обоих видов наблюдений 
практически совпадают. Тем не менее б$\acute{\mbox{о}}$льший вклад в~повышение качества 
оценивания вносят наблюдения с~логнормальным шумом. Этот факт можно объяснить 
тем, что различия в~$\{F^{\ell}\}$ хотя и~малые, но есть. Из-за 
мультипликативного характера зашумления дисперсии ошибок для разных вариантов~$F^{\ell}$ 
различны, что для случая непрерывного времени означает выполнение 
условий точной идентифицируемости МСП~\cite{BS_20}. В рассматриваемом 
допредельном случае наличие двух наблюдений $(\mathsf{S}_i, \mathsf{G}_i^{\mathrm{LN}})$ 
обеспечивает весьма высокое качество оценивания всех компонент МСП~$Z$.
\end{enumerate}

\vspace*{-9pt}

\section{Промежуточные выводы}

\vspace*{-3pt}

Мониторинг MPR по наблюдениям, полученным в~дискретные моменты времени, 
представляет собой задачу оптимальной фильтрации скрытого МСП. Ее теоретическое 
решение совместно с~чис\-лен\-ным алгоритмом и~иллюстративным примером приведено в~данной час\-ти цик\-ла. 
Однако использованная в~данной статье модель наблюдений~--- 
это идеализация и~упрощение: реальные статистические данные представляют собой 
потоки \mbox{значений} цен заявок и~сделок, фик\-си\-ру\-емых в~случайные моменты времени,~--- 
так на\-зы\-ва\-емые мультивариантные точечные процессы (МТП). Обычно в~сис\-те\-мах электронных 
торгов они оказываются вы\-со\-ко\-час\-тот\-ны\-ми.
Решение задач оптимальной фильт\-ра\-ции со\-сто\-яний МСП по наблюдениям МТП  
нетривиально как в~тео\-ре\-ти\-че\-ском плане, так и~с~точ\-ки зрения практической 
реализации. Один из вариантов создания эффективных субоптимальных алгоритмов 
фильтрации основан на использовании диффузионной аппроксимации по\-лу\-ча\-емых МТП. 
Использованию этого подхода и~чис\-лен\-но\-му анализу полученных оценок и~будет 
посвящена заключительная часть цикла.


{\small
  \section*{\raggedleft Приложение}


\noindent
Д\,о\,к\,а\,з\,а\,т\,е\,л\,ь\,с\,т\,в\,о\ теоремы~1.\
Используем метод математической индукции. Пусть на $(i-1)$-м шаге известна 
оптимальная оценка  
$$
\widehat{Z}_{i-1} \hm= \me{}{Z_{t_{i-1}}|\mathcal{O}_{i-1}}.
$$
Обозначим 
$$
\mathcal{E}(v) \ebd \col(\exp(v^1),\ldots,\exp(v^L)).
$$
Свойства условных математических ожиданий гарантируют, что следующая цепочка 
равенств выполняется\linebreak $\ppp$-п.~н. для любых множеств $A \hm\in 
\mathcal{B}(\mathbb{R}^N)$ и~$B \hm\in \mathcal{B}(\mathbb{R}^{M'})$:
\begin{multline*}
\me{}{Z_{i}^{\ell}\mathbf{I}_A(\mathsf{V}_{i})\mathbf{I}_B(\mathsf{G}_{i})|\mathcal{O}_{i-1}} ={}\\
{}=
\me{}{\me{}{Z_{i}^{\ell}\mathbf{I}_A(\mathsf{V}_{i})\mathbf{I}_B(\mathsf{G}_{i})
|\mathcal{O}_{i-1} \vee \mathcal{H}_{i}}
|\mathcal{O}_{i-1}}={}\\
{}=
\me{}{Z_{i}^{\ell}\mathbf{I}_A(\mathsf{V}_{i}) 
\me{}{\mathbf{I}_B(\mathsf{G}_{i})|\mathcal{O}_{i-1} \vee \mathcal{H}_{i}}
|\mathcal{O}_{i-1}}={}\\
{}=
\me{}{Z_{i}^{\ell}\mathbf{I}_A(\mathsf{V}_{i})
\int\limits_B \phi_i^{\ell}(q|\mathsf{S}_{i},\mathsf{G}_{i-
1})\mu_i(dq)\Bigl|\mathcal{O}_{i-1}} ={}
\\
{}=
{\sf E}\left\{ {\sf E}\left\{
\vphantom{\int\limits_B}
Z_{i}^{\ell}\mathbf{I}_A(\mathsf{V}_{i})\times{}\right.\right.\\
\left.\left.{}\times \int\limits_B \phi_i^{\ell}(q|\mathsf{S}_{i},\mathsf{G}_{i-1})\mu_i(dq)\Bigl|\mathcal{O}_{i-1} \vee Z_{t_{i-1}}\right\}\Bigl| \mathcal{O}_{i-1}\right\} = {}\\
{}=
{\sf E}\left\{ {\sf E}\left\{
\vphantom{\int\limits_B}
Z_{i}^{\ell}\mathbf{I}_A(\mathsf{V}_{i})\times{}\right.\right.\\
{}\times\int\limits_B \phi_i^{\ell}(q|\diag(\mathsf{S}_{i-1})\mathcal{E}(\mathsf{V}_{i}),\mathsf{G}_{i-1})\mu_i(dq)\Bigl|\mathcal{O}_{i-1} 
\vee\\
\hspace*{-20mm}\left.\left.\vee Z_{t_{i-1}}
\vphantom{\int\limits_B}
\right\}\Bigl| \mathcal{O}_{i-1}\right\} ={}\hspace*{20mm}
\end{multline*}

\noindent
\begin{multline*}
{}=
{\sf E}\left\{\int\limits_A\int\limits_B 
 \phi_i^{\ell}(q|\diag(\mathsf{S}_{i-1})\mathcal{E}(v),\mathsf{G}_{i-
1})\mu_i(dq)\times{}\right.\\
\left.{}\times \sum\limits_{k=1}^L Z_{t_{i-1}}^{k} \xi_i^{k\ell}(v) dv
\Bigl| \mathcal{O}_{i-1}
\vphantom{\int\limits_B}
\right\} = {}\\
{}=
\int\limits_A\int\limits_B
\left[
\sum_{k=1}^L \widehat{Z}_{i-1}^k
 \phi_i^{\ell}(q|\diag(\mathsf{S}_{i-1})\mathcal{E}(v),\mathsf{G}_{i-1})
  \xi_i^{k\ell}(v)
\right]\times{}\\
{}\times \mu_i(dq)\, dv.
\end{multline*}
Выражение в~квадратных скобках
является производной Ра\-до\-на--Ни\-ко\-ди\-ма условного распределения тройки $(Z_{t_i}, 
\mathsf{S}_{i}, \mathsf{G}_{i})$ относительно $\mathcal{O}_{i-1}$ по мере $\mu_i 
\times \lambda \times \nu$, где $\lambda$~--- мера Лебега на $\mathbb{R}^{M'}$; 
$\nu$~--- считающая мера, сосредоточенная на множестве $\{e_1,\ldots,e_L\}$. 
Тогда по формуле Байеса

\noindent
\begin{multline*}
\widehat{Z}_i^{\ell} ={}\\
{}=
\fr{\sum\nolimits_{k=1}^L \widehat{Z}_{i-1}^k
 \phi_i^{\ell}(\mathsf{G}_{i}|\diag(\mathsf{S}_{i-1})\mathcal{E}(\mathsf{V}_{i}),\mathsf{G}_{i-1})
  \xi_i^{k\ell}(\mathsf{V}_{i})}{\sum\nolimits_{m,j=1}^L \widehat{Z}_{i-1}^m
 \phi_i^{j}(\mathsf{G}_{i}|\diag(\mathsf{S}_{i-1})\mathcal{E}(\mathsf{V}_{i}),\mathsf{G}_{i-1})
  \xi_i^{mj}(\mathsf{V}_{i})} ={}\hspace*{-1.97446pt}\\
  {}=
  \fr{\sum\nolimits_{k=1}^L \widehat{Z}_{i-1}^k
 \phi_i^{\ell}(\mathsf{G}_{i}|\mathsf{S}_{i},\mathsf{G}_{i-1})
  \xi_i^{k\ell}(\mathsf{V}_{i})}{\sum\nolimits_{m,j=1}^L \widehat{Z}_{i-1}^m
 \phi_i^{j}(\mathsf{G}_{i}|\mathsf{S}_{i},\mathsf{G}_{i-1})
  \xi_i^{mj}(\mathsf{V}_{i})},
\end{multline*}
что доказывает истинность формул~(\ref{eq:filt_1}) и~(\ref{eq:filt_2}). 
Истинность формулы~(\ref{eq:filt_3}) вычисления оценки при $t\hm=0$ также следует 
из формулы Байеса.
Теорема~1 доказана.

}

{\small\frenchspacing
 {\baselineskip=10.6pt
 %\addcontentsline{toc}{section}{References}
 \begin{thebibliography}{99}
 \bibitem{B_23_1_IA}
\Au{Борисов A.} Рынок с~марковской скачкообразной волатильностью~I: мониторинг 
цены риска как задача оптимальной фильтрации~// Информатика и~её 
применения, 2023. Т.~17. Вып.~2. С.~27--33. doi: 10.14357/19922264230204. EDN: GAXCHQ.

   \bibitem{B_23_2_IA}
\Au{Борисов A.} Рынок с~марковской скачкообразной волатильностью II: алгоритм 
вычисления справедливой цены деривативов~// Информатика и~её 
применения, 2023. Т.~17. Вып.~3. С.~18--24. doi: 10.14357/19922264230303. EDN: DNXJGB.

\bibitem{BS_20} %3
\Au{Borisov A., Sokolov I.} Optimal filtering of Markov jump processes given 
observations with state-dependent \mbox{noises}: Exact solution and stable numerical 
schemes~// Mathematics, 2020. Vol.~8. No.\,4. Art.~506. 22~p. doi: 10.3390/ math8040506.

 \bibitem{Sh_98}
 \Au{Shiryayev A.} Essentials of stochastic finance: Facts, models, 
theory.~--- Hackensack, NJ, USA:~World Scientific, 1999. 834~p.

 \bibitem{KS_16}
\Au{Karatzas I., Shreve S.} Methods of mathematical finance.~--- Berlin: 
Springer, 2016. 416~p.

\bibitem{BSW_04} %6
\Au{Bollen N., Smith~T., Whaley~R.}
Modeling the bid/ask spread: Measuring the inventory-holding premium~//
J.~Financ. Econ., 2004.
Vol.~72. No.\,1. P.~97--141. doi: 10.1016/S0304-405X(03)00169-7.

\bibitem{LVSS_13} %7
 \Au{Lim T., Vath~V., Sahut~J., Scotti~S.}  Bid-ask spread modelling, a~perturbation approach~//
 Progress in probability~/ Eds. R.~Dalang, M.~Dozzi, F.~Russo.~---  Basel: Birkh$\ddot{\mbox{a}}$auser, 
2013. Vol.~67. P.~411--434. doi: 10.1007/978-3-0348-0545-2\_21.

\bibitem{CKS_12} %8
\Au{Cont R., Ku\-ka\-nov A., Stoi\-kov~S.} The price impact of order book events~//  
J.~Financ. Economet., 2014. Vol.~12. No.\,1. P.~47--88. doi: 10.2139/ssrn.1712822.


  \bibitem{KR_17} %9
 \Au{K$\ddot{\mbox{u}}$hn C., Riedel~M.} Price setting of market makers: A~filtering problem 
with endogenous filtration~// Math. Financ., 2017. Vol.~27. No.~1. 
P.~251--275. doi: 10.1111/mafi.12079.

\bibitem{CM_21} %10
{\it Cont R., M$\ddot{\mbox{u}}$ller M.} A~stochastic partial differential equation model for 
limit order book dynamics~//
SIAM J.~Financ. Math., 2021. Vol.~12. No.\,2.  P.~744--787. doi: 10.1137/19M1254489.


    \bibitem{B_20_3_IA}
\Au{Борисов A.} Численные схемы фильтрации марковских скачкообразных процессов 
по дискретизованным наблюдениям~III: случай мультипликативных шумов~// 
Информатика и~её применения, 2020. Т.~14. Вып.~2. С.~10--18. doi: 10.14357/19922264200202. EDN: OVICVK.

   \bibitem{PB_10}
\Au{Platen E., Bruti-Liberati N.} Numerical solution of stochastic differential 
equations with jumps in finance.~--- Berlin: Springer, 2010. 884~p.
\end{thebibliography}

 }
 }

\end{multicols}

\vspace*{-6pt}

\hfill{\small\textit{Поступила в~редакцию 27.01.23}}

\vspace*{8pt}

%\pagebreak

%\newpage

%\vspace*{-28pt}

\hrule

\vspace*{2pt}

\hrule



\def\tit{MARKET WITH MARKOV JUMP VOLATILITY III: PRICE OF~RISK MONITORING ALGORITHM
GIVEN DISCRETE-TIME OBSERVATIONS OF ASSET PRICES}


\def\titkol{Market with Markov jump volatility III: Price of~risk monitoring algorithm
given discrete-time observations of asset prices}


\def\aut{A.\,V.~Borisov}

\def\autkol{A.\,V.~Borisov}

\titel{\tit}{\aut}{\autkol}{\titkol}

\vspace*{-10pt}


\noindent
Federal Research Center ``Computer Science and Control'' of the Russian Academy 
of Sciences, 44-2~Vavilov Str., Moscow 119333, Russian Federation


\def\leftfootline{\small{\textbf{\thepage}
\hfill INFORMATIKA I EE PRIMENENIYA~--- INFORMATICS AND
APPLICATIONS\ \ \ 2023\ \ \ volume~17\ \ \ issue\ 4}
}%
 \def\rightfootline{\small{INFORMATIKA I EE PRIMENENIYA~---
INFORMATICS AND APPLICATIONS\ \ \ 2023\ \ \ volume~17\ \ \ issue\ 4
\hfill \textbf{\thepage}}}

\vspace*{3pt}




\Abste{The third part of the series is devoted to the online estimation of the market price of risk. 
The market model includes a deposit, underlying, and derivative assets. The model of the underlying asset prices 
contains stochastic volatility represented by a~arkov jump process (MJP). There is no arbitrage in the considered market; 
so,\linebreak\vspace*{-12pt}}

\Abstend{the market price of risk is a function of the MJP's current value. So, the MJP monitoring problem transforms into the 
MJP state filtering one. The statistical data are available at discrete moments and contain the direct observations of 
the underlying assets and indirect observations of the derivative ones. The paper presents the solution to the optimal 
filtering problem and the corresponding algorithm of its numerical realization. The paper also contains a numerical example
 demonstrating the performance of the MJP state estimates in dependence on the type and structure of the available observations.}


\KWE{Markov jump process; optimal filtering; stochastic volatility; market price of risk; prevailing martingale measure}




\DOI{10.14357/19922264230402}{OFYELT}

\vspace*{-12pt}

\Ack
\noindent
The research was carried out using the infrastructure of the Shared Research Facilities ``High Performance Computing and Big Data'' 
(CKP ``Informatics'') of FRC CSC RAS (Moscow).

%\vspace*{6pt}

  \begin{multicols}{2}

\renewcommand{\bibname}{\protect\rmfamily References}
%\renewcommand{\bibname}{\large\protect\rm References}

{\small\frenchspacing
 {%\baselineskip=10.8pt
 \addcontentsline{toc}{section}{References}
 \begin{thebibliography}{99} 

%1
\bibitem{B_23_1_IA-1} 
\Aue{Borisov, A.} 2023. Rynok s~markovskoy skachkoobraznoy volatil'nost'yu I: monitoring tseny riska kak zadacha optimal'noy fil'tratsii 
[Market with Markov jump volatility I: Price of risk monitoring as an optimal filtering problem]. \textit{Informatika i ee Primeneniya~--- 
Inform. Appl.} 17(2):~27--33. doi: 10.14357/19922264230204. EDN: GAXCHQ.

%2
\bibitem{B_23_2-1}
\Aue{Borisov, A.} 2023. Rynok s~markovskoy skachkoobraznoy volatil'nost'yu II: algoritm vychisleniya spravedlivoy tseny derivativov 
[Market with Markov jump volatility II: Algorithm of derivative fair price calculation].
\textit{Informatika i~ee primeneniya~--- Inform. Appl.} 17(3):18--24. doi: 10.14357/19922264230303. EDN: DNXJGB.

%3
\bibitem{BS_20-1}
\Aue{Borisov, A., and I.~Sokolov.}
 2020. Optimal filtering of Markov jump processes given observations with state-dependent noises: Exact solution and stable numerical schemes. 
 \textit{Mathematics} 8(4):506. 22~p. doi: 10.3390/math8040506.

%4
\bibitem{Sh_98-1}
\Aue{Shiryayev, A.} 1999. \textit{Essentials of stochastic finance: Facts, models, theory}. Hackensack, NJ: World Scientific. 834~p.

%5
\bibitem{KS_16-1}
\Aue{Karatzas, I., and S.~Shreve.} 2016. \textit{Methods of mathematical finance}. Berlin: Springer. 416~p.

%6
\bibitem{BSW_04-1}
\Aue{Bollen, N., T.~Smith, and R.~Whaley.} 2004.
Modeling the bid/ask spread: Measuring the inventory-holding premium.
\textit{J.~Financ. Econ.} 72(1):97--141. doi: 10.1016/S0304-405X(03)00169-7.

%7
\bibitem{LVSS_13-1}
\Aue{Lim, T., V.~Vath, J.~Sahut, and S.~Scotti.} 2013. Bid-ask spread modelling, a~perturbation approach. 
\textit{Progress in probability}. 
Eds. R.~Dalang, M.~Dozzi, and F.~Russo. Basel: Birkh$\ddot{\mbox{a}}$user. 67:411--434. doi: 10.1007/978-3-0348-0545-2\_21.

%8
\bibitem{CKS_12-1}
\Aue{Cont, R., A.~Kukanov, and S.~Stoikov.} 2014. The price impact of order book events. \textit{J.~Financ. Economet.} 12(1):47--88. 
doi: 10.2139/ssrn.1712822.

 
%9
\bibitem{KR_17-1}
\Aue{K$\ddot{\mbox{u}}$hn, C., and M.~Riedel.} 2017. Price setting of market makers: A~filtering problem with endogenous filtration. 
\textit{Math. Financ.} 27(1):251--275. 
doi: 10.1111/mafi.12079.

%10
\bibitem{CM_21-1}
\Aue{Cont, R., and M.~M$\ddot{\mbox{u}}$ller.} 2021. A~stochastic partial differential equation model for limit order book dynamics. 
\textit{SIAM J. Financ. Math.} 12(2):744--787. doi: 10.1137/ 19M1254489.
 
%11
\bibitem{B_20_1_ARC-1}
\Aue{Borisov, A.} 2020. Chislennye skhemy fil'tratsii markovskikh skachkoobraznykh protsessov po diskretizovannym nablyudeniyam III:
 sluchay mul'tiplikativnykh shumov [Numerical schemes of Markov jump process filtering given discretized observations III: Multiplicative noises case]
\textit{Informatika i~ee primeneniya~--- Inform. Appl.} 14(2)10--18. doi: 10.14357/19922264200202. EDN: OVICVK.

%12
\bibitem{KP_92-1}
\Aue{Platen, E., and N.~Bruti-Liberati.} 2010. 
\textit{Numerical solution of stochastic differential equations with jumps in finance}. Berlin: Springer. 884~p.

\end{thebibliography}

 }
 }

\end{multicols}

\vspace*{-6pt}

\hfill{\small\textit{Received January 27, 2023}} 

\vspace*{-12pt}

\Contrl

\vspace*{-4pt}

\noindent
\textbf{Borisov Andrey V.} (b.\ 1965)~--- 
Doctor of Science in physics and mathematics, principal scientist, Federal Research Center ``Computer Science and Control'' 
of the Russian Academy of Sciences, 44-2~Vavilov Str., Moscow 119333, Russian Federation; \mbox{aborisov@frccsc.ru}






\label{end\stat}

\renewcommand{\bibname}{\protect\rm Литература} 